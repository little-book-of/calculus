% Options for packages loaded elsewhere
\PassOptionsToPackage{unicode}{hyperref}
\PassOptionsToPackage{hyphens}{url}
\PassOptionsToPackage{dvipsnames,svgnames,x11names}{xcolor}
%
\documentclass[
  letterpaper,
  DIV=11,
  numbers=noendperiod]{scrartcl}

\usepackage{amsmath,amssymb}
\usepackage{iftex}
\ifPDFTeX
  \usepackage[T1]{fontenc}
  \usepackage[utf8]{inputenc}
  \usepackage{textcomp} % provide euro and other symbols
\else % if luatex or xetex
  \usepackage{unicode-math}
  \defaultfontfeatures{Scale=MatchLowercase}
  \defaultfontfeatures[\rmfamily]{Ligatures=TeX,Scale=1}
\fi
\usepackage{lmodern}
\ifPDFTeX\else  
    % xetex/luatex font selection
\fi
% Use upquote if available, for straight quotes in verbatim environments
\IfFileExists{upquote.sty}{\usepackage{upquote}}{}
\IfFileExists{microtype.sty}{% use microtype if available
  \usepackage[]{microtype}
  \UseMicrotypeSet[protrusion]{basicmath} % disable protrusion for tt fonts
}{}
\makeatletter
\@ifundefined{KOMAClassName}{% if non-KOMA class
  \IfFileExists{parskip.sty}{%
    \usepackage{parskip}
  }{% else
    \setlength{\parindent}{0pt}
    \setlength{\parskip}{6pt plus 2pt minus 1pt}}
}{% if KOMA class
  \KOMAoptions{parskip=half}}
\makeatother
\usepackage{xcolor}
\setlength{\emergencystretch}{3em} % prevent overfull lines
\setcounter{secnumdepth}{-\maxdimen} % remove section numbering
% Make \paragraph and \subparagraph free-standing
\makeatletter
\ifx\paragraph\undefined\else
  \let\oldparagraph\paragraph
  \renewcommand{\paragraph}{
    \@ifstar
      \xxxParagraphStar
      \xxxParagraphNoStar
  }
  \newcommand{\xxxParagraphStar}[1]{\oldparagraph*{#1}\mbox{}}
  \newcommand{\xxxParagraphNoStar}[1]{\oldparagraph{#1}\mbox{}}
\fi
\ifx\subparagraph\undefined\else
  \let\oldsubparagraph\subparagraph
  \renewcommand{\subparagraph}{
    \@ifstar
      \xxxSubParagraphStar
      \xxxSubParagraphNoStar
  }
  \newcommand{\xxxSubParagraphStar}[1]{\oldsubparagraph*{#1}\mbox{}}
  \newcommand{\xxxSubParagraphNoStar}[1]{\oldsubparagraph{#1}\mbox{}}
\fi
\makeatother


\providecommand{\tightlist}{%
  \setlength{\itemsep}{0pt}\setlength{\parskip}{0pt}}\usepackage{longtable,booktabs,array}
\usepackage{calc} % for calculating minipage widths
% Correct order of tables after \paragraph or \subparagraph
\usepackage{etoolbox}
\makeatletter
\patchcmd\longtable{\par}{\if@noskipsec\mbox{}\fi\par}{}{}
\makeatother
% Allow footnotes in longtable head/foot
\IfFileExists{footnotehyper.sty}{\usepackage{footnotehyper}}{\usepackage{footnote}}
\makesavenoteenv{longtable}
\usepackage{graphicx}
\makeatletter
\newsavebox\pandoc@box
\newcommand*\pandocbounded[1]{% scales image to fit in text height/width
  \sbox\pandoc@box{#1}%
  \Gscale@div\@tempa{\textheight}{\dimexpr\ht\pandoc@box+\dp\pandoc@box\relax}%
  \Gscale@div\@tempb{\linewidth}{\wd\pandoc@box}%
  \ifdim\@tempb\p@<\@tempa\p@\let\@tempa\@tempb\fi% select the smaller of both
  \ifdim\@tempa\p@<\p@\scalebox{\@tempa}{\usebox\pandoc@box}%
  \else\usebox{\pandoc@box}%
  \fi%
}
% Set default figure placement to htbp
\def\fps@figure{htbp}
\makeatother

\KOMAoption{captions}{tableheading}
\makeatletter
\@ifpackageloaded{caption}{}{\usepackage{caption}}
\AtBeginDocument{%
\ifdefined\contentsname
  \renewcommand*\contentsname{Table of contents}
\else
  \newcommand\contentsname{Table of contents}
\fi
\ifdefined\listfigurename
  \renewcommand*\listfigurename{List of Figures}
\else
  \newcommand\listfigurename{List of Figures}
\fi
\ifdefined\listtablename
  \renewcommand*\listtablename{List of Tables}
\else
  \newcommand\listtablename{List of Tables}
\fi
\ifdefined\figurename
  \renewcommand*\figurename{Figure}
\else
  \newcommand\figurename{Figure}
\fi
\ifdefined\tablename
  \renewcommand*\tablename{Table}
\else
  \newcommand\tablename{Table}
\fi
}
\@ifpackageloaded{float}{}{\usepackage{float}}
\floatstyle{ruled}
\@ifundefined{c@chapter}{\newfloat{codelisting}{h}{lop}}{\newfloat{codelisting}{h}{lop}[chapter]}
\floatname{codelisting}{Listing}
\newcommand*\listoflistings{\listof{codelisting}{List of Listings}}
\makeatother
\makeatletter
\makeatother
\makeatletter
\@ifpackageloaded{caption}{}{\usepackage{caption}}
\@ifpackageloaded{subcaption}{}{\usepackage{subcaption}}
\makeatother

\usepackage{bookmark}

\IfFileExists{xurl.sty}{\usepackage{xurl}}{} % add URL line breaks if available
\urlstyle{same} % disable monospaced font for URLs
\hypersetup{
  colorlinks=true,
  linkcolor={blue},
  filecolor={Maroon},
  citecolor={Blue},
  urlcolor={Blue},
  pdfcreator={LaTeX via pandoc}}


\author{}
\date{}

\begin{document}


\#الكتاب الصغير في حساب التفاضل والتكامل

مقدمة موجزة وسهلة للمبتدئين للأفكار الأساسية لحساب التفاضل والتكامل.

\subsection{التنسيقات}\label{ux627ux644ux62aux646ux633ux64aux642ux627ux62a}

\begin{itemize}
\tightlist
\item
  \href{../artifacts/ar/book.pdf}{Download PDF} -- نسخة جاهزة للطباعة
\item
  \href{../artifacts/ar/book.epub}{Download EPUB} -- صديق للقارئ
  الإلكتروني
\item
  \href{../artifacts/ar/book.tex}{View LaTeX} - مصدر اللاتكس
\end{itemize}

\section{الجزء 1. الحدود
والمشتقات}\label{ux627ux644ux62cux632ux621-1.-ux627ux644ux62dux62fux648ux62f-ux648ux627ux644ux645ux634ux62aux642ux627ux62a}

\section{الفصل 1. الوظائف
والحدود}\label{ux627ux644ux641ux635ux644-1.-ux627ux644ux648ux638ux627ux626ux641-ux648ux627ux644ux62dux62fux648ux62f}

\subsection{1.1 الوظائف}\label{ux627ux644ux648ux638ux627ux626ux641}

الدالة هي إحدى الأشياء الأساسية في الرياضيات. الدالة في جوهرها هي قاعدة
تأخذ مدخلاً وتنتج مخرجًا واحدًا بالضبط. تتيح لنا الوظائف وصف العلاقات،
ونمذجة ظواهر العالم الحقيقي، وبناء آلية حساب التفاضل والتكامل بأكملها.

\subsubsection{التعريف}\label{ux627ux644ux62aux639ux631ux64aux641}

رسميًا، تتم كتابة دالة \(f\) من مجموعة \(X\) (تسمى المجال) إلى مجموعة
\(Y\) (تسمى المجال الكودي)

\[
f : X \to Y.
\]

لكل عنصر \(x \in X\)، يوجد عنصر فريد \(f(x) \in Y\). القيمة \(f(x)\)
تسمى صورة \(x\) ضمن \(f\).

إذا كان \(y = f(x)\)، فإن \(y\) هو الإخراج المطابق للإدخال \(x\). مجموعة
جميع المخرجات التي تظهر بالفعل تسمى النطاق (مجموعة فرعية من المجال
الكودي).

\subsubsection{أمثلة}\label{ux623ux645ux62bux644ux629}

\begin{enumerate}
\def\labelenumi{\arabic{enumi}.}
\item
  تقوم الدالة \(f(x) = x^2\) بتعيين كل رقم حقيقي \(x\) إلى مربعه.

  \begin{itemize}
  \tightlist
  \item
    المجال: جميع الأرقام الحقيقية \(\mathbb{R}\).
  \item
    المجال الكودي: جميع الأرقام الحقيقية \(\mathbb{R}\).
  \item
    المدى: جميع الأعداد الحقيقية غير السالبة \([0, \infty)\).
  \end{itemize}
\item
  تقوم الدالة \(g(x) = \dfrac{1}{x}\) بتعيين مقلوبه لكل رقم حقيقي غير
  صفري.

  \begin{itemize}
  \tightlist
  \item
    النطاق: \(\mathbb{R} \setminus \{0\}\).
  \item
    النطاق: \(\mathbb{R} \setminus \{0\}\).
  \end{itemize}
\item
  مثال من العالم الحقيقي: افترض أن \(T(t)\) هي درجة الحرارة الخارجية
  (بالدرجة المئوية) في الوقت \(t\) (بالساعات). هذه دالة من ``الوقت من
  اليوم'' إلى ``درجة الحرارة''.
\end{enumerate}

\subsubsection{طرق تمثيل
الوظائف}\label{ux637ux631ux642-ux62aux645ux62bux64aux644-ux627ux644ux648ux638ux627ux626ux641}

يمكن تمثيل الوظائف بعدة طرق مفيدة:

\begin{itemize}
\tightlist
\item
  الصيغ: على سبيل المثال، \(f(x) = \sin x + x^2\).- الرسوم البيانية: رسم
  جميع النقاط \((x, f(x))\) في المستوى الإحداثي.
\item
  الجداول: اقتران المدخلات والمخرجات لمجموعات منفصلة من البيانات.
\item
  الأوصاف اللفظية: ``خصص لكل طالب درجاته''.
\end{itemize}

يسلط كل تمثيل الضوء على جوانب مختلفة من نفس الوظيفة.

\subsubsection{المصطلحات}\label{ux627ux644ux645ux635ux637ux644ux62dux627ux62a}

\begin{itemize}
\tightlist
\item
  المتغير المستقل: الإدخال (يكتب عادة \(x\)).
\item
  المتغير التابع: الناتج (يكتب عادة \(y\)، حيث \(y = f(x)\)).
\item
  تدوين الوظيفة: \(f(x)\) تتم قراءته ``\(f\) من \(x\).''
\end{itemize}

\subsubsection{أهمية الدوال في حساب التفاضل
والتكامل}\label{ux623ux647ux645ux64aux629-ux627ux644ux62fux648ux627ux644-ux641ux64a-ux62dux633ux627ux628-ux627ux644ux62aux641ux627ux636ux644-ux648ux627ux644ux62aux643ux627ux645ux644}

حساب التفاضل والتكامل هو دراسة كيفية تغير الوظائف. تقيس المشتقات معدلات
التغير اللحظية، بينما تقيس التكاملات التأثيرات المتراكمة. لإتقان هذه
الأفكار، نحتاج أولاً إلى فهم قوي لماهية الوظائف وكيف تتصرف.

\subsubsection{تمارين}\label{ux62aux645ux627ux631ux64aux646}

\begin{enumerate}
\def\labelenumi{\arabic{enumi}.}
\item
  بالنسبة للوظيفة \(f(x) = 3x - 2\):

  \begin{itemize}
  \tightlist
  \item
    ابحث عن المجال والمجال الكودي والنطاق.
  \end{itemize}
\item
  الوظيفة \(h(x) = \sqrt{x-1}\) محددة لأي مدخلات؟ ما هو نطاقها؟
\item
  أعط مثالاً واقعيًا لوظيفة من حياتك اليومية. اذكر بوضوح المجال والمجال
  الكودي.
\item
  ارسم الرسم البياني لـ \(f(x) = |x|\). ما هو النطاق؟
\item
  افترض أن \(g(x) = \dfrac{1}{x^2+1}\). اشرح لماذا يكون نطاقه هو الفاصل
  الزمني \((0, 1]\).
\end{enumerate}

\subsection{1.2 الرسوم البيانية
والتحويلات}\label{ux627ux644ux631ux633ux648ux645-ux627ux644ux628ux64aux627ux646ux64aux629-ux648ux627ux644ux62aux62dux648ux64aux644ux627ux62a}

يمكن فهم الدالة ليس فقط من خلال الصيغ، ولكن أيضًا من خلال الرسم البياني
الخاص بها. الرسم البياني للدالة \(f\) هو مجموعة من جميع الأزواج المرتبة
\((x, f(x))\)، حيث ينتمي \(x\) إلى مجال \(f\). إن رسم هذه الأزواج في
المستوى الإحداثي يعطي صورة لكيفية تصرف الوظيفة.

\subsubsection{الرسوم البيانية
الأساسية}\label{ux627ux644ux631ux633ux648ux645-ux627ux644ux628ux64aux627ux646ux64aux629-ux627ux644ux623ux633ux627ux633ux64aux629}

بعض الرسوم البيانية أساسية جدًا بحيث يجب حفظها:

\begin{itemize}
\tightlist
\item
  \(f(x) = x\): خط مستقيم يمر بنقطة الأصل.
\item
  \(f(x) = x^2\): قطع مكافئ يفتح لأعلى.
\item
  \(f(x) = |x|\): رسم بياني على شكل حرف ``V''.
\item
  \(f(x) = \frac{1}{x}\): قطع زائد ذو فرعين.- \(f(x) = \sin x\): منحنى
  دوري يشبه الموجة.
\end{itemize}

هذه بمثابة اللبنات الأساسية لوظائف أكثر تعقيدًا.

\subsubsection{التحولات}\label{ux627ux644ux62aux62dux648ux644ux627ux62a}

يمكن إزاحة الرسوم البيانية أو تمديدها أو عكسها باستخدام قواعد بسيطة:

\begin{enumerate}
\def\labelenumi{\arabic{enumi}.}
\item
  التحولات الرأسية: تؤدي إضافة ثابت إلى تحريك الرسم البياني لأعلى أو
  لأسفل.

  \[
  y = f(x) + c \quad \text{is } f(x) \text{ shifted upward by } c.
  \]
\item
  التحولات الأفقية: تؤدي الإضافة داخل الوسيطة إلى تحريك الرسم البياني
  إلى اليسار أو اليمين.

  \[
  y = f(x - c) \quad \text{is } f(x) \text{ shifted right by } c.
  \]
\item
  القياس الرأسي: الضرب بثابت يمتد أو يضغط الرسم البياني عموديًا.

  \[
  y = a f(x), \quad a > 1 \text{ stretches; } 0 < a < 1 \text{ compresses.}
  \]
\item
  القياس الأفقي: يؤدي الضرب داخل الوسيطة إلى تمديد الرسم البياني أو ضغطه
  أفقيًا.

  \[
  y = f(bx), \quad b > 1 \text{ compresses toward the } y\text{-axis}.
  \]
\item
  التأملات:

  \begin{itemize}
  \tightlist
  \item
    \(y = -f(x)\): الانعكاس عبر محور \(x\).
  \item
    \(y = f(-x)\): الانعكاس عبر محور \(y\).
  \end{itemize}
\end{enumerate}

\subsubsection{الجمع بين
التحولات}\label{ux627ux644ux62cux645ux639-ux628ux64aux646-ux627ux644ux62aux62dux648ux644ux627ux62a}

غالبًا ما تأتي الرسوم البيانية المعقدة من الجمع بين عدة تحويلات بالتسلسل.
على سبيل المثال:

\[
y = 2(x-1)^2 + 3
\]

يتم الحصول عليها عن طريق أخذ القطع المكافئ \(y = x^2\)، والتحول إلى
اليمين بمقدار 1، والتمدد عموديًا بمقدار 2، والتحول لأعلى بمقدار 3.

\subsubsection{تمارين}\label{ux62aux645ux627ux631ux64aux646-1}

\begin{enumerate}
\def\labelenumi{\arabic{enumi}.}
\tightlist
\item
  ارسم الرسم البياني لـ \(y = (x+2)^2 - 1\). حدد تسلسل التحويلات من
  \(y = x^2\).
\item
  ماذا يحدث للرسم البياني لـ \(y = f(x)\) إذا استبدلنا \(x\) بـ \(-x\)؟
  جربه باستخدام \(f(x) = \sqrt{x}\).
\item
  قم بوصف التحويلات التي تحول \(y = \sin x\) إلى
  \(y = 3\sin(x - \pi/4)\).
\item
  ارسم الرسم البياني لـ \(y = |x-1| + 2\). اذكر قمة الرأس وميل كل فرع.
\item
  بالنسبة إلى \(y = \frac{1}{x-2}\)، اشرح كيف تم تحويل الرسم البياني لـ
  \(y = \frac{1}{x}\).
\end{enumerate}

\subsection{1.3 فكرة بديهية للحدودفي العديد من المواقف، تكون قيمة الدالة
عند نقطة ما أقل أهمية من القيم التي تأخذها بالقرب من تلك النقطة. مفهوم
الحد يجسد هذه
الفكرة.}\label{ux641ux643ux631ux629-ux628ux62fux64aux647ux64aux629-ux644ux644ux62dux62fux648ux62fux641ux64a-ux627ux644ux639ux62fux64aux62f-ux645ux646-ux627ux644ux645ux648ux627ux642ux641-ux62aux643ux648ux646-ux642ux64aux645ux629-ux627ux644ux62fux627ux644ux629-ux639ux646ux62f-ux646ux642ux637ux629-ux645ux627-ux623ux642ux644-ux623ux647ux645ux64aux629-ux645ux646-ux627ux644ux642ux64aux645-ux627ux644ux62aux64a-ux62aux623ux62eux630ux647ux627-ux628ux627ux644ux642ux631ux628-ux645ux646-ux62aux644ux643-ux627ux644ux646ux642ux637ux629.-ux645ux641ux647ux648ux645-ux627ux644ux62dux62f-ux64aux62cux633ux62f-ux647ux630ux647-ux627ux644ux641ux643ux631ux629.}

\subsubsection{الاقتراب من
القيمة}\label{ux627ux644ux627ux642ux62aux631ux627ux628-ux645ux646-ux627ux644ux642ux64aux645ux629}

تخيل المشي نحو الحائط. حتى قبل أن تلمسه، فإنك تقترب أكثر فأكثر. بنفس
الطريقة، عندما يقترب \(x\) من الرقم \(a\)، قد تقترب قيم \(f(x)\) من رقم
ما \(L\). ثم نقول:

\[
\lim_{x \to a} f(x) = L.
\]

يعبر هذا عن فكرة أنه يمكن جعل \(f(x)\) قريبًا بقدر ما نريد من \(L\)، وذلك
ببساطة عن طريق تقريب \(x\) بدرجة كافية من \(a\).

\subsubsection{أمثلة}\label{ux623ux645ux62bux644ux629-1}

\begin{enumerate}
\def\labelenumi{\arabic{enumi}.}
\item
  بالنسبة إلى \(f(x) = 2x + 3\): مثل \(x \to 1\)، \(f(x) \to 5\).
\item
  بالنسبة إلى \(f(x) = \dfrac{\sin x}{x}\): مثل \(x \to 0\)، تقترب
  الدالة من 1، على الرغم من عدم تعريف \(f(0)\).
\item
  بالنسبة إلى \(f(x) = \dfrac{1}{x}\): مثل \(x \to 0^+\) (يقترب من
  اليمين)، \(f(x) \to +\infty\). مثل \(x \to 0^-\) (يقترب من اليسار)،
  \(f(x) \to -\infty\). نظرًا لاختلاف السلوكيات اليمنى واليسرى، فإن الحد
  عند 0 غير موجود.
\end{enumerate}

\subsubsection{أهمية
الحدود}\label{ux623ux647ux645ux64aux629-ux627ux644ux62dux62fux648ux62f}

\begin{itemize}
\tightlist
\item
  تسمح لنا بتحديد الوظائف في نقاط لم يتم تعريفها فيها في الأصل.
\item
  يلتقطون السلوك بالقرب من الانقطاعات والتفردات.
\item
  أنها تشكل الأساس للمشتقات (معدلات التغيير اللحظية) والتكاملات
  (المساحات كحدود للمبالغ).
\end{itemize}

\subsubsection{حدود من جانب
واحد}\label{ux62dux62fux648ux62f-ux645ux646-ux62cux627ux646ux628-ux648ux627ux62dux62f}

في بعض الأحيان يجب دراسة السلوك من اليسار واليمين بشكل منفصل:

\[
\lim_{x \to a^-} f(x), \quad \lim_{x \to a^+} f(x).
\]

إذا اتفق كلاهما، فإن الحد من الجانبين موجود.

\subsubsection{تمارين}\label{ux62aux645ux627ux631ux64aux646-2}

\begin{enumerate}
\def\labelenumi{\arabic{enumi}.}
\tightlist
\item
  حساب \(\lim_{x \to 2} (3x^2 - x)\).
\item
  ما هو \(\lim_{x \to 0} \frac{\sin x}{x}\)؟ استخدم الحدس من الرسم
  البياني \(\sin x\).
\item
  قم بتقييم \(\lim_{x \to 0} |x|/x\). هل الحد ذو الوجهين موجود؟
\item
  ابحث عن \(\lim_{x \to \infty} \frac{1}{x}\). تفسير هذه النتيجة
  بالكلمات.5. بالنسبة إلى \(f(x) = \frac{x^2-1}{x-1}\)، ما هو
  \(\lim_{x \to 1} f(x)\)؟ قارن بقيمة \(f(1)\).
\end{enumerate}

\subsection{1.4 التعريف الرسمي
للحدود}\label{ux627ux644ux62aux639ux631ux64aux641-ux627ux644ux631ux633ux645ux64a-ux644ux644ux62dux62fux648ux62f}

يمكن جعل الفكرة البديهية للحدود دقيقة باستخدام تعريف إبسيلون-دلتا.
يمنحنا هذا طريقة صارمة للقول بأن \(f(x)\) يقترب من القيمة \(L\) بينما
يقترب \(x\) من \(a\).

\subsubsection{التعريف}\label{ux627ux644ux62aux639ux631ux64aux641-1}

نحن نكتب

\[
\lim_{x \to a} f(x) = L
\]

إذا تحقق الشرط التالي:

لكل \(\varepsilon > 0\) (مهما كان صغيرًا)، يوجد \(\delta > 0\) بحيث كلما

\[
0 < |x - a| < \delta,
\]

ويترتب على ذلك

\[
|f(x) - L| < \varepsilon.
\]

بالكلمات: يمكننا أن نجعل \(f(x)\) أقرب ما يكون إلى \(L\)، بشرط أن يكون
\(x\) قريبًا بدرجة كافية من \(a\) (ولكن لا يساوي \(a\)).

\subsubsection{مثال 1: الدالة
الخطية}\label{ux645ux62bux627ux644-1-ux627ux644ux62fux627ux644ux629-ux627ux644ux62eux637ux64aux629}

بالنسبة إلى \(f(x) = 2x + 1\)، أظهر أن \(\lim_{x \to 3} f(x) = 7\).

\begin{itemize}
\tightlist
\item
  نريد \(|f(x) - 7| < \varepsilon\).
\item
  لكن \(f(x) - 7 = 2x + 1 - 7 = 2(x - 3)\).
\item
  إذن \(|f(x) - 7| = 2|x - 3|\).
\item
  إذا اخترنا \(\delta = \varepsilon / 2\)، فكلما \(|x - 3| < \delta\)،
  لدينا \(|f(x) - 7| < \varepsilon\). وهذا يثبت الحد.
\end{itemize}

\subsubsection{مثال 2: دالة
متبادلة}\label{ux645ux62bux627ux644-2-ux62fux627ux644ux629-ux645ux62aux628ux627ux62fux644ux629}

بالنسبة إلى \(f(x) = \frac{1}{x}\)، فكر في
\(\lim_{x \to 2} f(x) = \tfrac{1}{2}\).

\begin{itemize}
\tightlist
\item
  نريد \(\left|\frac{1}{x} - \frac{1}{2}\right| < \varepsilon\).
\item
  تتطلب هذه المتراجحة معالجة جبرية، ولكن يمكن تلبيتها باختيار \(\delta\)
  اعتمادًا على \(\varepsilon\). العملية أكثر تعقيدا، ولكن المبدأ هو نفسه.
\end{itemize}

\subsubsection{لماذا هذا
مهم}\label{ux644ux645ux627ux630ux627-ux647ux630ux627-ux645ux647ux645}

\begin{itemize}
\tightlist
\item
  يضمن تعريف إبسيلون-دلتا أن الحدود ليست غامضة أو مبنية على الحدس فقط.
\item
  وهو أساس الاستمرارية والمشتقات والتكاملات.
\item
  على الرغم من أن المبتدئين قد يجدونها مجردة، إلا أن العمل بأمثلة بسيطة
  يبني الألفة.
\end{itemize}

\subsubsection{تمارين}\label{ux62aux645ux627ux631ux64aux646-3}

\begin{enumerate}
\def\labelenumi{\arabic{enumi}.}
\tightlist
\item
  باستخدام تعريف epsilon-delta، أثبت أن \(\lim_{x \to 4} (x+1) = 5\).2.
  أظهر أن \(\lim_{x \to 0} 5x = 0\) باستخدام التعريف الرسمي.
\item
  اشرح سبب عدم وجود \(\lim_{x \to 0} \frac{1}{x}\).
\item
  بالنسبة إلى \(f(x) = x^2\)، أظهر أن \(\lim_{x \to 2} f(x) = 4\).
\item
  بكلماتك الخاصة، اشرح دور \(\varepsilon\) و\(\delta\) في تعريف الحد.
\end{enumerate}

\subsection{1.5
الاستمرارية}\label{ux627ux644ux627ux633ux62aux645ux631ux627ux631ux64aux629}

تكون الدالة متصلة إذا أمكن رسم رسمها البياني دون رفع قلم الرصاص عن
الورقة. وبشكل أكثر دقة، تضمن الاستمرارية أن تؤدي التغييرات الصغيرة في
المدخلات إلى تغييرات صغيرة في المخرجات.

\subsubsection{التعريف}\label{ux627ux644ux62aux639ux631ux64aux641-2}

تكون الدالة \(f\) متصلة عند النقطة \(a\) إذا تم استيفاء ثلاثة شروط:

\begin{enumerate}
\def\labelenumi{\arabic{enumi}.}
\tightlist
\item
  تم تعريف \(f(a)\).
\item
  \(\lim_{x \to a} f(x)\) موجود.
\item
  \(\lim_{x \to a} f(x) = f(a)\).
\end{enumerate}

إذا كانت الدالة متصلة عند كل نقطة في الفترة، نقول إنها متصلة في تلك
الفترة.

\subsubsection{أمثلة}\label{ux623ux645ux62bux644ux629-2}

\begin{enumerate}
\def\labelenumi{\arabic{enumi}.}
\item
  دوال كثيرة الحدود: دوال مثل \(f(x) = x^2 + 3x - 5\) مستمرة في كل مكان
  على \(\mathbb{R}\).
\item
  الدوال المنطقية: \(f(x) = \frac{1}{x-1}\) مستمر في كل مكان باستثناء
  \(x = 1\)، حيث يكون غير محدد.
\item
  وظائف القطع:

  \[
  f(x) =
  \begin{cases}
  x^2 & x < 1, \\
  2 & x = 1, \\
  x+1 & x > 1,
  \end{cases}
  \]

  تحتوي هذه الوظيفة على ``قفزة'' عند \(x = 1\)، لذا فهي ليست مستمرة
  هناك.
\end{enumerate}

\subsubsection{أنواع
الانقطاعات}\label{ux623ux646ux648ux627ux639-ux627ux644ux627ux646ux642ux637ux627ux639ux627ux62a}

\begin{enumerate}
\def\labelenumi{\arabic{enumi}.}
\tightlist
\item
  انقطاع قابل للإزالة: ``ثغرة'' في الرسم البياني. مثال:
  \(f(x) = \frac{x^2-1}{x-1}\) في \(x=1\).
\item
  توقف القفز: الحدود اليسرى واليمنى مختلفة.
\item
  الانقطاع اللانهائي: تنتقل الدالة إلى \(\pm\infty\) بالقرب من نقطة ما،
  كما هو الحال مع \(f(x) = 1/x\) بالقرب من \(x = 0\).
\end{enumerate}

\subsubsection{نظرية القيمة
المتوسطة}\label{ux646ux638ux631ux64aux629-ux627ux644ux642ux64aux645ux629-ux627ux644ux645ux62aux648ux633ux637ux629}

إذا كانت الدالة متصلة على الفاصل الزمني \([a, b]\)، إذن لأي رقم \(N\)
بين \(f(a)\) و\(f(b)\)، يوجد بعض \(c \in [a, b]\) مثل \(f(c) = N\).هذه
الخاصية مهمة في إثبات وجود جذور وحلول المعادلات.

\subsubsection{تمارين}\label{ux62aux645ux627ux631ux64aux646-4}

\begin{enumerate}
\def\labelenumi{\arabic{enumi}.}
\tightlist
\item
  قرر ما إذا كانت الدالة \(f(x) = |x|\) مستمرة عند \(x = 0\).
\item
  حدد نقاط الانقطاع لـ \(f(x) = \frac{x+2}{x^2-1}\).
\item
  اشرح لماذا تكون كل دالة كثيرة الحدود متصلة في كل مكان.
\item
  أعط مثالاً على دالة ذات قفزة متقطعة. رسم الرسم البياني الخاص به.
\item
  استخدم نظرية القيمة المتوسطة لتوضيح أن المعادلة \(x^3 + x - 1 = 0\)
  لها حل بين 0 و1.
\end{enumerate}

\section{الفصل 2.
المشتقات}\label{ux627ux644ux641ux635ux644-2.-ux627ux644ux645ux634ux62aux642ux627ux62a}

\subsection{2.1 المشتق كمعدل
للتغير}\label{ux627ux644ux645ux634ux62aux642-ux643ux645ux639ux62fux644-ux644ux644ux62aux63aux64aux631}

المشتق هو أحد الأفكار المركزية في حساب التفاضل والتكامل. فهو يقيس كيفية
تغير الوظيفة مع تغير مدخلاتها - وبعبارة أخرى، معدل تغير المخرجات فيما
يتعلق بالمدخلات.

\subsubsection{متوسط معدل
التغيير}\label{ux645ux62aux648ux633ux637-ux645ux639ux62fux644-ux627ux644ux62aux63aux64aux64aux631}

بالنسبة للدالة \(f(x)\)، متوسط معدل التغيير بين نقطتين \(x = a\)
و\(x = b\) هو

\[
\frac{f(b) - f(a)}{b - a}.
\]

هذا هو ميل الخط القاطع عبر النقطتين \((a, f(a))\) و\((b, f(b))\).

\subsubsection{معدل التغير
اللحظي}\label{ux645ux639ux62fux644-ux627ux644ux62aux63aux64aux631-ux627ux644ux644ux62dux638ux64a}

لقياس مدى سرعة تغير \(f(x)\) عند نقطة واحدة، تركنا الفاصل الزمني يتقلص:

\[
f'(a) = \lim_{h \to 0} \frac{f(a+h) - f(a)}{h}.
\]

هذا الحد، إن وجد، يسمى مشتق \(f\) عند \(a\). هندسيًا، هو ميل خط المماس
للرسم البياني \(f\) عند النقطة \((a, f(a))\).

\subsubsection{التدوين}\label{ux627ux644ux62aux62fux648ux64aux646}

\begin{itemize}
\tightlist
\item
  \(f'(x)\): علامة أولية.
\item
  \(\dfrac{dy}{dx}\): تدوين لايبنتز، يُستخدم عند \(y = f(x)\).
\item
  \(Df(x)\): تدوين عامل التشغيل.
\end{itemize}

كل هذه الرموز تشير إلى نفس المفهوم.

\subsubsection{أمثلة}\label{ux623ux645ux62bux644ux629-3}

\begin{enumerate}
\def\labelenumi{\arabic{enumi}.}
\item
  بالنسبة إلى \(f(x) = x^2\):

  \[
  f'(x) = \lim_{h \to 0} \frac{(x+h)^2 - x^2}{h} = \lim_{h \to 0} \frac{2xh + h^2}{h} = 2x.
  \]

  ميل القطع المكافئ عند \(x\) هو \(2x\).
\item
  بالنسبة إلى \(f(x) = \sin x\):

  \[
  f'(x) = \cos x.
  \]3. بالنسبة إلى \(f(x) = c\) (ثابت):

  \[
  f'(x) = 0.
  \]

  دالة ثابتة لا تتغير أبدًا.
\end{enumerate}

\subsubsection{التفسير}\label{ux627ux644ux62aux641ux633ux64aux631}

\begin{itemize}
\tightlist
\item
  في الفيزياء: إذا كان \(s(t)\) هو الموضع، فإن \(s'(t)\) هو السرعة.
\item
  في الاقتصاد: إذا كان \(C(x)\) يمثل التكلفة، فإن \(C'(x)\) يمثل التكلفة
  الحدية.
\item
  في علم الأحياء: إذا كان \(P(t)\) هو عدد السكان، فإن \(P'(t)\) هو معدل
  النمو.
\end{itemize}

المشتق يجعل ``التغيير'' دقيقًا في العديد من السياقات.

\subsubsection{تمارين}\label{ux62aux645ux627ux631ux64aux646-5}

\begin{enumerate}
\def\labelenumi{\arabic{enumi}.}
\tightlist
\item
  قم بحساب \(f'(x)\) لـ \(f(x) = 3x^2 - 2x + 1\).
\item
  أوجد ميل خط المماس لـ \(f(x) = x^3\) عند \(x = 2\).
\item
  إذا كان \(s(t) = t^2 + 2t\) يمثل المسافة بالأمتار، فما هي السرعة عند
  \(t = 5\)؟
\item
  استخدم تعريف الحد لحساب مشتق \(f(x) = \frac{1}{x}\).
\item
  ارسم الرسم البياني لـ \(y = x^2\) وارسم خط المماس عند \(x = 1\).
\end{enumerate}

\subsection{2.2 قواعد
التمايز}\label{ux642ux648ux627ux639ux62f-ux627ux644ux62aux645ux627ux64aux632}

بمجرد تعريف المشتقة، نحتاج إلى طرق فعالة لحسابها. قواعد التفاضل هي
اختصارات تمنعنا من تطبيق تعريف الحد بشكل متكرر.

\subsubsection{القاعدة
الثابتة}\label{ux627ux644ux642ux627ux639ux62fux629-ux627ux644ux62bux627ux628ux62aux629}

إذا كان \(f(x) = c\) حيث \(c\) ثابتًا، إذن

\[
f'(x) = 0.
\]

\subsubsection{قاعدة
القوة}\label{ux642ux627ux639ux62fux629-ux627ux644ux642ux648ux629}

بالنسبة إلى \(f(x) = x^n\) حيث \(n\) رقم حقيقي،

\[
\frac{d}{dx} \big( x^n \big) = n x^{n-1}.
\]

أمثلة:

\begin{itemize}
\tightlist
\item
  \(\frac{d}{dx}(x^2) = 2x\).
\item
  \(\frac{d}{dx}(x^5) = 5x^4\).
\item
  \(\frac{d}{dx}(\sqrt{x}) = \frac{1}{2\sqrt{x}}\).
\end{itemize}

\subsubsection{قاعدة التعدد
الثابت}\label{ux642ux627ux639ux62fux629-ux627ux644ux62aux639ux62fux62f-ux627ux644ux62bux627ux628ux62a}

إذا كان \(f(x) = c \cdot g(x)\)، إذن

\[
f'(x) = c \cdot g'(x).
\]

\subsubsection{قواعد المجموع
والفرق}\label{ux642ux648ux627ux639ux62f-ux627ux644ux645ux62cux645ux648ux639-ux648ux627ux644ux641ux631ux642}

\begin{itemize}
\tightlist
\item
  \((f + g)' = f' + g'\).
\item
  \((f - g)' = f' - g'\).
\end{itemize}

\subsubsection{قاعدة
المنتج}\label{ux642ux627ux639ux62fux629-ux627ux644ux645ux646ux62aux62c}

بالنسبة إلى \(f(x)\) و\(g(x)\):

\[
(fg)' = f'g + fg'.
\]

مثال: إذا كان \(f(x) = x^2\)، \(g(x) = \sin x\):

\[
(fg)' = (2x)(\sin x) + (x^2)(\cos x).
\]

\subsubsection{قاعدة
القسمة}\label{ux642ux627ux639ux62fux629-ux627ux644ux642ux633ux645ux629}

بالنسبة إلى \(f(x)\) و\(g(x)\):

\[
\left(\frac{f}{g}\right)' = \frac{f'g - fg'}{g^2}, \quad g(x) \neq 0.
\]

مثال: إذا كان \(f(x) = x^2\)، \(g(x) = x+1\):

\[\left(\frac{x^2}{x+1}\right)' = \frac{(2x)(x+1) - (x^2)(1)}{(x+1)^2}.
\]

\subsubsection{Derivatives of Common
Functions}\label{derivatives-of-common-functions}

\begin{itemize}
\tightlist
\item
  \(\frac{d}{dx}(\sin x) = \cos x\).
\item
  \(\frac{d}{dx}(\cos x) = -\sin x\).
\item
  \(\frac{d}{dx}(e^x) = e^x\).
\item
  \(\frac{d}{dx}(\ln x) = \frac{1}{x}, \quad x > 0\).
\end{itemize}

\subsubsection{Exercises}\label{exercises}

\begin{enumerate}
\def\labelenumi{\arabic{enumi}.}
\tightlist
\item
  Differentiate \(f(x) = 7x^3 - 4x + 9\).
\item
  Use the product rule to find the derivative of \(f(x) = x^2 e^x\).
\item
  Apply the quotient rule to \(f(x) = \frac{\sin x}{x}\).
\item
  Compute \(\frac{d}{dx}(\ln(x^2))\) using the chain of rules.
\item
  Show that the derivative of \(f(x) = \frac{1}{x}\) is
  \(-\frac{1}{x^2}\).
\end{enumerate}

\subsection{2.3 The Chain Rule}\label{the-chain-rule}

Often, functions are built by combining simpler functions together. To
differentiate such composite functions, we use the chain rule.

\subsubsection{The Rule}\label{the-rule}

If \(y = f(g(x))\), then

\[
\frac{dy}{dx} = f'(g(x)) \cdot g'(x).
\]

In words: differentiate the outer function, keep the inside unchanged,
then multiply by the derivative of the inside.

\subsubsection{Examples}\label{examples}

\begin{enumerate}
\def\labelenumi{\arabic{enumi}.}
\item
  Square of a linear function

  \[
  ص = (3س+2)^2
  \]

  Outer function: \(f(u) = u^2\), inner function: \(g(x) = 3x+2\).

  \[
  y' = 2(3x+2) \cdot 3 = 6(3x+2).
  \]
\item
  Exponential with quadratic inside

  \[
  ص = ه^{س^2}
  \]

  Outer function: \(f(u) = e^u\), inner function: \(g(x) = x^2\).

  \[
  y' = e^{x^2} \cdot 2x = 2x e^{x^2}.
  \]
\item
  Logarithm with root inside

  \[
  ص = \ln(\sqrt{x})
  \]

  Outer: \(f(u) = \ln u\), inner: \(g(x) = \sqrt{x}\).

  \[
  y' = \frac{1}{\sqrt{x}} \cdot \frac{1}{2\sqrt{x}} = \frac{1}{2x}.
  \]
\end{enumerate}

\subsubsection{Generalized Chain Rule}\label{generalized-chain-rule}

For multiple nested functions \(y = f(g(h(x)))\):

\[
\frac{dy}{dx} = f'(g(h(x))) \cdot g'(h(x)) \cdot h'(x).
\]

يمتد هذا بشكل طبيعي إلى التراكيب الأعمق.

\subsubsection{لماذا تعتبر قاعدة السلسلة مهمة؟- يتعامل مع جميع نماذج
العالم الحقيقي تقريبًا حيث تعتمد كمية ما على أخرى بشكل غير
مباشر.}\label{ux644ux645ux627ux630ux627-ux62aux639ux62aux628ux631-ux642ux627ux639ux62fux629-ux627ux644ux633ux644ux633ux644ux629-ux645ux647ux645ux629--ux64aux62aux639ux627ux645ux644-ux645ux639-ux62cux645ux64aux639-ux646ux645ux627ux630ux62c-ux627ux644ux639ux627ux644ux645-ux627ux644ux62dux642ux64aux642ux64a-ux62aux642ux631ux64aux628ux627-ux62dux64aux62b-ux62aux639ux62aux645ux62f-ux643ux645ux64aux629-ux645ux627-ux639ux644ux649-ux623ux62eux631ux649-ux628ux634ux643ux644-ux63aux64aux631-ux645ux628ux627ux634ux631.}

\begin{itemize}
\tightlist
\item
  يربط حساب التفاضل والتكامل مع الفيزياء (على سبيل المثال، السرعة تعتمد
  على الوقت من خلال الموقع).
\item
  لا غنى عنه في التمايز الضمني والمواضيع المتقدمة.
\end{itemize}

\subsubsection{تمارين}\label{ux62aux645ux627ux631ux64aux646-6}

\begin{enumerate}
\def\labelenumi{\arabic{enumi}.}
\tightlist
\item
  التمييز بين \(y = (5x^2 + 1)^3\).
\item
  ابحث عن \(\frac{d}{dx}(\sin(3x))\).
\item
  حساب \(\frac{d}{dx}(\ln(1+x^2))\).
\item
  التمييز بين \(y = \cos^2(x)\).
\item
  قم بتطبيق قاعدة السلسلة المعممة على \(y = e^{\sin(x^2)}\).
\end{enumerate}

\subsection{2.4 التمايز
الضمني}\label{ux627ux644ux62aux645ux627ux64aux632-ux627ux644ux636ux645ux646ux64a}

لا يتم تقديم جميع الوظائف في النموذج \(y = f(x)\). في بعض الأحيان، يرتبط
\(x\) و\(y\) بمعادلة، ويكون حل \(y\) بشكل صريح أمرًا صعبًا أو مستحيلًا. في
مثل هذه الحالات، نستخدم التمايز الضمني.

\subsubsection{الفكرة}\label{ux627ux644ux641ux643ux631ux629}

إذا كانت المعادلة تتضمن كلا من \(x\) و\(y\)، فيمكننا التفريق بين الطرفين
فيما يتعلق بـ \(x\)، مع التعامل مع \(y\) كدالة لـ \(x\). في كل مرة نفرق
فيها مصطلحًا يتضمن \(y\)، نضرب في \(\frac{dy}{dx}\).

\subsubsection{مثال 1:
دائرة}\label{ux645ux62bux627ux644-1-ux62fux627ux626ux631ux629}

المعادلة:

\[
x^2 + y^2 = 25
\]

التفريق فيما يتعلق بـ \(x\):

\[
2x + 2y \frac{dy}{dx} = 0.
\]

حل لـ \(\frac{dy}{dx}\):

\[
\frac{dy}{dx} = -\frac{x}{y}.
\]

وهذا يعطي ميل المماس للدائرة عند أي نقطة.

\subsubsection{مثال 2: منتج
المتغيرات}\label{ux645ux62bux627ux644-2-ux645ux646ux62aux62c-ux627ux644ux645ux62aux63aux64aux631ux627ux62a}

المعادلة:

\[
xy = 1
\]

التفريق:

\[
x \frac{dy}{dx} + y = 0.
\]

لذا،

\[
\frac{dy}{dx} = -\frac{y}{x}.
\]

\subsubsection{مثال 3: العلاقة
المثلثية}\label{ux645ux62bux627ux644-3-ux627ux644ux639ux644ux627ux642ux629-ux627ux644ux645ux62bux644ux62bux64aux629}

المعادلة:

\[
\sin(xy) = x
\]

التفريق:

\[
\cos(xy) \cdot \Big(y + x\frac{dy}{dx}\Big) = 1.
\]

حل لـ \(\frac{dy}{dx}\):

\[
\frac{dy}{dx} = \frac{1 - y\cos(xy)}{x\cos(xy)}.
\]

\subsubsection{لماذا يعتبر التمايز الضمني
مفيدًا}\label{ux644ux645ux627ux630ux627-ux64aux639ux62aux628ux631-ux627ux644ux62aux645ux627ux64aux632-ux627ux644ux636ux645ux646ux64a-ux645ux641ux64aux62fux627}

\begin{itemize}
\tightlist
\item
  يتم تعريف العديد من المنحنيات المهمة (الدوائر، القطع الناقص، القطع
  الزائدة) بشكل طبيعي ضمنيًا.
\item
  يتيح لنا تفريق المعادلات دون حل \(y\) أولاً.- إنها خطوة أساسية في
  موضوعات أكثر تقدمًا مثل المعدلات ذات الصلة والمعادلات التفاضلية.
\end{itemize}

\subsubsection{تمارين}\label{ux62aux645ux627ux631ux64aux646-7}

\begin{enumerate}
\def\labelenumi{\arabic{enumi}.}
\tightlist
\item
  بالنسبة للمنحنى \(x^2 + xy + y^2 = 7\)، ابحث عن \(\frac{dy}{dx}\).
\item
  ميّز \(\cos(x) + \cos(y) = 1\) ضمنيًا.
\item
  أوجد ميل خط المماس لـ \(x^3 + y^3 = 9\) عند النقطة \((1, 2)\).
\item
  بالنظر إلى \(x^2 + y^2 = 10\)، قم بحساب \(\frac{dy}{dx}\) عندما يكون
  \((x, y) = (1, 3)\).
\item
  اشتق \(e^{xy} = x + y\) للعثور على \(\frac{dy}{dx}\).
\end{enumerate}

\subsection{2.5 المشتقات ذات الرتبة
الأعلى}\label{ux627ux644ux645ux634ux62aux642ux627ux62a-ux630ux627ux62a-ux627ux644ux631ux62aux628ux629-ux627ux644ux623ux639ux644ux649}

لقد درسنا حتى الآن المشتقة الأولى، التي تقيس معدل تغير الدالة. ولكن يمكن
أيضًا التمييز بين المشتقات نفسها، مما يؤدي إلى ظهور مشتقات ذات ترتيب
أعلى.

\subsubsection{التعريف}\label{ux627ux644ux62aux639ux631ux64aux641-3}

\begin{itemize}
\item
  المشتق الثاني لـ \(f\) هو مشتق المشتق:

  \[
  f''(x) = \frac{d}{dx}\left(f'(x)\right).
  \]
\item
  بشكل عام، يتم كتابة المشتق \(n\) على النحو التالي

  \[
  f^{(n)}(x) = \frac{d^n}{dx^n} f(x).
  \]
\end{itemize}

\subsubsection{أمثلة}\label{ux623ux645ux62bux644ux629-4}

\begin{enumerate}
\def\labelenumi{\arabic{enumi}.}
\item
  \(f(x) = x^3\)

  \begin{itemize}
  \tightlist
  \item
    المشتق الأول: \(f'(x) = 3x^2\).
  \item
    المشتق الثاني: \(f''(x) = 6x\).
  \item
    المشتق الثالث: \(f^{(3)}(x) = 6\).
  \item
    المشتق الرابع: \(f^{(4)}(x) = 0\).
  \end{itemize}
\item
  \(f(x) = \sin x\)

  \begin{itemize}
  \tightlist
  \item
    \(f'(x) = \cos x\).
  \item
    \(f''(x) = -\sin x\).
  \item
    \(f^{(3)}(x) = -\cos x\).
  \item
    \(f^{(4)}(x) = \sin x\). تتكرر المشتقات في دورة طولها 4.
  \end{itemize}
\item
  \(f(x) = e^x\)

  \begin{itemize}
  \tightlist
  \item
    كل مشتق هو \(e^x\).
  \end{itemize}
\end{enumerate}

\subsubsection{التطبيقات}\label{ux627ux644ux62aux637ux628ux64aux642ux627ux62a}

\begin{itemize}
\item
  التقعر: تشير علامة \(f''(x)\) إلى ما إذا كان الرسم البياني لـ \(f\)
  مقعرًا لأعلى (\(f'' > 0\)) أو مقعرًا لأسفل (\(f'' < 0\)).
\item
  نقاط الانعطاف: النقاط التي يتغير فيها \(f''(x) = 0\) والتقعر.
\item
  الحركة: في الفيزياء، إذا كان \(s(t)\) هو الموضع:

  \begin{itemize}
  \tightlist
  \item
    \(s'(t)\) = السرعة،
  \item
    \(s''(t)\) = التسارع،
  \item
    \(s^{(3)}(t)\) = الرعشة (معدل تغير التسارع).
  \end{itemize}
\item
  التقريبات: تظهر المشتقات ذات الترتيب الأعلى في متسلسلة تايلور، وتستخدم
  لتقريب الدوال.\#\#\# تمارين
\end{itemize}

\begin{enumerate}
\def\labelenumi{\arabic{enumi}.}
\tightlist
\item
  احسب المشتقات الأربعة الأولى لـ \(f(x) = \cos x\).
\item
  ابحث عن \(f''(x)\) لـ \(f(x) = x^4 - 2x^2 + 3\).
\item
  بالنسبة إلى \(f(x) = e^{2x}\)، أظهر أن \(f^{(n)}(x) = 2^n e^{2x}\).
\item
  حدد الفترات التي يكون فيها \(f(x) = x^3 - 3x\) مقعرًا لأعلى ومقعرًا
  لأسفل.
\item
  إذا كان \(s(t) = t^3 - 6t^2 + 9t\)، فأوجد السرعة المتجهة والتسارع عند
  \(t = 2\).
\end{enumerate}

\section{الفصل 3. تطبيقات
المشتقات}\label{ux627ux644ux641ux635ux644-3.-ux62aux637ux628ux64aux642ux627ux62a-ux627ux644ux645ux634ux62aux642ux627ux62a}

\subsection{3.1 الظلال
والأعراف}\label{ux627ux644ux638ux644ux627ux644-ux648ux627ux644ux623ux639ux631ux627ux641}

أحد التطبيقات الأولى للمشتقات هو إيجاد معادلات الخطوط المماسية والعادية
للمنحنى. تلتقط هذه الخطوط الشكل الهندسي المحلي للدالة عند نقطة معينة.

\subsubsection{خط الظل}\label{ux62eux637-ux627ux644ux638ux644}

خط المماس لمنحنى \(y = f(x)\) عند نقطة \((a, f(a))\) هو الخط الذي
``يلامس'' الرسم البياني هناك وله نفس ميل المنحنى.

يتم إعطاء ميل خط المماس بواسطة المشتق:

\[
m_{\text{tangent}} = f'(a).
\]

وبالتالي، فإن معادلة خط المماس عند \((a, f(a))\) هي

\[
y - f(a) = f'(a)(x - a).
\]

\subsubsection{خط عادي}\label{ux62eux637-ux639ux627ux62fux64a}

الخط العادي يكون عموديا على خط المماس عند نفس النقطة. ميله هو المقلوب
السلبي لمنحدر الظل:

\[
m_{\text{normal}} = -\frac{1}{f'(a)}.
\]

إذن معادلة الخط الطبيعي هي

\[
y - f(a) = -\frac{1}{f'(a)} (x - a), \quad f'(a) \neq 0.
\]

\subsubsection{أمثلة}\label{ux623ux645ux62bux644ux629-5}

\begin{enumerate}
\def\labelenumi{\arabic{enumi}.}
\item
  \(f(x) = x^2\) في \(x = 1\).

  \begin{itemize}
  \tightlist
  \item
    \(f(1) = 1\)، \(f'(x) = 2x\)، إذن \(f'(1) = 2\).
  \item
    الظل: \(y - 1 = 2(x - 1)\)، أو \(y = 2x - 1\).
  \item
    عادي: الميل = \(-\tfrac{1}{2}\)، إذن المعادلة هي
    \(y - 1 = -\tfrac{1}{2}(x - 1)\).
  \end{itemize}
\item
  \(f(x) = \sin x\) في \(x = \tfrac{\pi}{4}\).

  \begin{itemize}
  \tightlist
  \item
    \(f(\tfrac{\pi}{4}) = \tfrac{\sqrt{2}}{2}\)،
    \(f'(\tfrac{\pi}{4}) = \cos(\tfrac{\pi}{4}) = \tfrac{\sqrt{2}}{2}\).
  \item
    الظل:
    \(y - \tfrac{\sqrt{2}}{2} = \tfrac{\sqrt{2}}{2}(x - \tfrac{\pi}{4})\).
  \end{itemize}
\end{enumerate}

\subsubsection{سبب أهمية الظلال والمعايير- المماسات تقريب المنحنى محليا
(تقريب
خطي).}\label{ux633ux628ux628-ux623ux647ux645ux64aux629-ux627ux644ux638ux644ux627ux644-ux648ux627ux644ux645ux639ux627ux64aux64aux631--ux627ux644ux645ux645ux627ux633ux627ux62a-ux62aux642ux631ux64aux628-ux627ux644ux645ux646ux62dux646ux649-ux645ux62dux644ux64aux627-ux62aux642ux631ux64aux628-ux62eux637ux64a.}

\begin{itemize}
\tightlist
\item
  تعتبر القيم الطبيعية مفيدة في الهندسة والبصريات (الانعكاس/الانكسار)
  والميكانيكا (اتجاهات القوة).
\item
  كلاهما يلعب دورًا في دراسات التحسين والانحناء.
\end{itemize}

\subsubsection{تمارين}\label{ux62aux645ux627ux631ux64aux646-8}

\begin{enumerate}
\def\labelenumi{\arabic{enumi}.}
\tightlist
\item
  ابحث عن خطوط المماس والخطوط العادية لـ \(y = x^3\) في \(x = 2\).
\item
  حدد خطوط المماس والخطوط العادية لـ \(y = e^x\) عند \(x = 0\).
\item
  بالنسبة إلى \(y = \ln x\)، احسب خط المماس عند \(x = 1\).
\item
  يتم إعطاء الدائرة بواسطة \(x^2 + y^2 = 9\). استخدم الاشتقاق الضمني
  لإيجاد ميل المماس عند \((0,3)\).
\item
  ارسم الرسم البياني لـ \(y = \sqrt{x}\) وارسم خطوط المماس والعادية عند
  \(x = 4\).
\end{enumerate}

\subsection{3.2 الأسعار ذات
الصلة}\label{ux627ux644ux623ux633ux639ux627ux631-ux630ux627ux62a-ux627ux644ux635ux644ux629}

في العديد من مسائل العالم الحقيقي، تتغير كميتان أو أكثر بالنسبة للزمن،
وتكون معدلات تغيرهما مرتبطة ببعضها البعض. تستخدم مشكلات الأسعار ذات
الصلة المشتقات لوصف هذه العلاقات.

\subsubsection{النهج
العام}\label{ux627ux644ux646ux647ux62c-ux627ux644ux639ux627ux645}

\begin{enumerate}
\def\labelenumi{\arabic{enumi}.}
\tightlist
\item
  التعرف على المتغيرات التي تعتمد على الزمن \(t\).
\item
  اكتب معادلة تتعلق بالمتغيرات.
\item
  قم بالتفريق بين الطرفين فيما يتعلق بـ \(t\)، بتطبيق قاعدة السلسلة.
\item
  استبدل القيم المعروفة في اللحظة المحددة.
\item
  أوجد المعدل المجهول.
\end{enumerate}

\subsubsection{مثال 1: توسيع
الدائرة}\label{ux645ux62bux627ux644-1-ux62aux648ux633ux64aux639-ux627ux644ux62fux627ux626ux631ux629}

دائرة نصف قطرها \(r\)، والذي يزداد بمعدل
\(\frac{dr}{dt} = 2 \,\text{cm/s}\). أوجد المعدل الذي تزيد به المساحة
\(A = \pi r^2\) عندما \(r = 5\).

التفريق:

\[
\frac{dA}{dt} = 2\pi r \frac{dr}{dt}.
\]

البديل:

\[
\frac{dA}{dt} = 2\pi (5)(2) = 20\pi \,\text{cm}^2/\text{s}.
\]

\subsubsection{مثال 2: سلم
منزلق}\label{ux645ux62bux627ux644-2-ux633ux644ux645-ux645ux646ux632ux644ux642}

سلم طوله ١٠ أقدام يميل على الحائط. ينزلق الجزء السفلي بعيدًا عند
\(\frac{dx}{dt} = 1 \,\text{ft/s}\). ما مدى سرعة انزلاق الجزء العلوي
للأسفل عندما يكون الجزء السفلي على بعد 6 أقدام من الحائط؟

المعادلة: \(x^2 + y^2 = 100\)، حيث \(y\) هو الارتفاع.

التفريق:

\[
2x \frac{dx}{dt} + 2y \frac{dy}{dt} = 0.
\]في \(x = 6\)، \(y = 8\). البديل:

\[
2(6)(1) + 2(8)\frac{dy}{dt} = 0 \quad \Rightarrow \quad \frac{dy}{dt} = -\tfrac{6}{8} = -\tfrac{3}{4}.
\]

ومن ثم ينزلق الجزء العلوي للأسفل عند \(0.75 \,\text{ft/s}\).

\subsubsection{مثال 3: الماء في
المخروط}\label{ux645ux62bux627ux644-3-ux627ux644ux645ux627ux621-ux641ux64a-ux627ux644ux645ux62eux631ux648ux637}

يُسكب الماء في مخروط ارتفاعه 12 سم ونصف قطره 6 سم. عندما يصل عمق الماء
إلى 4 سم، يرتفع مستوى الماء عند \(2 \,\text{cm/s}\). بأي معدل يزداد
الحجم؟

المعادلة: \(V = \tfrac{1}{3}\pi r^2 h\). باستخدام التشابه،
\(r = \tfrac{h}{2}\). الاستبدال:

\[
V = \tfrac{1}{12}\pi h^3.
\]

التفريق:

\[
\frac{dV}{dt} = \tfrac{1}{4}\pi h^2 \frac{dh}{dt}.
\]

في \(h = 4\)، \(\frac{dh}{dt} = 2\):

\[
\frac{dV}{dt} = \tfrac{1}{4}\pi (16)(2) = 8\pi \,\text{cm}^3/\text{s}.
\]

\subsubsection{لماذا تعتبر الأسعار ذات الصلة
مهمة؟}\label{ux644ux645ux627ux630ux627-ux62aux639ux62aux628ux631-ux627ux644ux623ux633ux639ux627ux631-ux630ux627ux62a-ux627ux644ux635ux644ux629-ux645ux647ux645ux629}

\begin{itemize}
\tightlist
\item
  يصفون الحركة والتغير في الفيزياء والهندسة والأحياء.
\item
  يربطون الهندسة بحساب التفاضل والتكامل من خلال عمليات تعتمد على الوقت.
\item
  يقومون بتدريبنا على نمذجة الأنظمة الديناميكية رياضياً.
\end{itemize}

\subsubsection{تمارين}\label{ux62aux645ux627ux631ux64aux646-9}

\begin{enumerate}
\def\labelenumi{\arabic{enumi}.}
\tightlist
\item
  تم نفخ البالون بحيث يزيد نصف قطره عند \(0.5 \,\text{cm/s}\). أوجد مدى
  سرعة زيادة حجمه عندما يكون نصف قطره ١٠ سم.
\item
  تسير سيارة شمالًا بسرعة 40 km/h، وأخرى شرقًا بسرعة 30 km/h. ما مدى سرعة
  زيادة المسافة بينهما بعد ساعتين؟
\item
  يسطع ضوء موجه على مسافة 20 m من الحائط على رجل طوله 2 m يمشي بسرعة 1.5
  m/s. ما مدى سرعة تغير طول ظله على الحائط عندما يكون على بعد 5 m من
  الضوء؟
\item
  يزداد طول ضلع المكعب بمعدل 2 سم/ث. ما مدى سرعة زيادة مساحة السطح عندما
  يكون طول الضلع 3 سم؟
\item
  يتم صب الرمل على كومة مكونة مخروطًا نصف قطره يساوي الارتفاع دائمًا. إذا
  زاد الارتفاع بمقدار 5 سم/ث، فما معدل زيادة الحجم عندما يكون الارتفاع
  10 سم؟
\end{enumerate}

\subsection{3.3 مشاكل التحسينتستخدم مشكلات التحسين المشتقات للعثور على
القيم القصوى أو الدنيا للدالة، غالبًا تحت قيود معينة. تمثل هذه المشكلات
المواقف التي نريد فيها تعظيم الكفاءة أو الربح أو المساحة، أو تقليل
التكلفة أو المسافة أو
الوقت.}\label{ux645ux634ux627ux643ux644-ux627ux644ux62aux62dux633ux64aux646ux62aux633ux62aux62eux62fux645-ux645ux634ux643ux644ux627ux62a-ux627ux644ux62aux62dux633ux64aux646-ux627ux644ux645ux634ux62aux642ux627ux62a-ux644ux644ux639ux62bux648ux631-ux639ux644ux649-ux627ux644ux642ux64aux645-ux627ux644ux642ux635ux648ux649-ux623ux648-ux627ux644ux62fux646ux64aux627-ux644ux644ux62fux627ux644ux629-ux63aux627ux644ux628ux627-ux62aux62dux62a-ux642ux64aux648ux62f-ux645ux639ux64aux646ux629.-ux62aux645ux62bux644-ux647ux630ux647-ux627ux644ux645ux634ux643ux644ux627ux62a-ux627ux644ux645ux648ux627ux642ux641-ux627ux644ux62aux64a-ux646ux631ux64aux62f-ux641ux64aux647ux627-ux62aux639ux638ux64aux645-ux627ux644ux643ux641ux627ux621ux629-ux623ux648-ux627ux644ux631ux628ux62d-ux623ux648-ux627ux644ux645ux633ux627ux62dux629-ux623ux648-ux62aux642ux644ux64aux644-ux627ux644ux62aux643ux644ux641ux629-ux623ux648-ux627ux644ux645ux633ux627ux641ux629-ux623ux648-ux627ux644ux648ux642ux62a.}

\subsubsection{خطوات
عامة}\label{ux62eux637ux648ux627ux62a-ux639ux627ux645ux629}

\begin{enumerate}
\def\labelenumi{\arabic{enumi}.}
\tightlist
\item
  افهم المشكلة: حدد الكمية المطلوب تحسينها.
\item
  نموذج مع دالة: اكتب الدالة الهدف بدلالة متغير واحد.
\item
  تطبيق القيود: استخدم شروطًا معينة لتقليل المتغيرات.
\item
  التفريق: حساب مشتقة الدالة الهدف.
\item
  ابحث عن النقاط الحرجة: قم بحل \(f'(x) = 0\) أو حيث يكون \(f'(x)\) غير
  محدد.
\item
  اختبار الحد الأقصى/الحد الأدنى: استخدم اختبار المشتق الثاني أو تحقق من
  نقاط النهاية.
\item
  فسّر النتيجة: اذكر الإجابة في السياق الأصلي.
\end{enumerate}

\subsubsection{مثال 1: أقصى مساحة
للمستطيل}\label{ux645ux62bux627ux644-1-ux623ux642ux635ux649-ux645ux633ux627ux62dux629-ux644ux644ux645ux633ux62aux637ux64aux644}

مستطيل محيطه 40. ما الأبعاد التي تزيد مساحته إلى أقصى حد؟

\begin{itemize}
\tightlist
\item
  الطول \(x\)، العرض \(y\). القيد:
  \(2x + 2y = 40 \Rightarrow y = 20 - x\).
\item
  المنطقة: \(A = xy = x(20 - x) = 20x - x^2\).
\item
  المشتق: \(A'(x) = 20 - 2x\). تعيين يساوي 0: \(x = 10\).
\item
  ثم \(y = 10\).
\item
  المساحة القصوى: \(100\). المستطيل عبارة عن مربع.
\end{itemize}

\subsubsection{مثال 2: تقليل
المسافة}\label{ux645ux62bux627ux644-2-ux62aux642ux644ux64aux644-ux627ux644ux645ux633ux627ux641ux629}

ابحث عن النقطة على القطع المكافئ \(y = x^2\) الأقرب إلى \((0,3)\).

\begin{itemize}
\tightlist
\item
  مربع المسافة: \(D(x) = (x-0)^2 + (x^2 - 3)^2\).
\item
  توسيع:
  \(D(x) = x^2 + (x^2 - 3)^2 = x^2 + x^4 - 6x^2 + 9 = x^4 - 5x^2 + 9\).
\item
  المشتق: \(D'(x) = 4x^3 - 10x\). حل: \(x(4x^2 - 10) = 0\).
\item
  الحلول: \(x = 0\), \(x = \pm \sqrt{2.5}\).
\item
  الفحص يعطي الحد الأدنى للمسافة عند \(x = \pm \sqrt{2.5}\).
\end{itemize}

\subsubsection{مثال 3: صندوق ذو حجم
أقصى}\label{ux645ux62bux627ux644-3-ux635ux646ux62fux648ux642-ux630ux648-ux62dux62cux645-ux623ux642ux635ux649}

يُصنع صندوق بدون سطح من قطعة مربعة من الورق المقوى طول كل جانب منها 20
سم، وذلك بقطع مربعات متساوية من الزوايا وطي الجوانب. ابحث عن حجم القطع
الذي يزيد الحجم.- دع حجم القطع = \(x\). ثم الأبعاد:
\((20 - 2x) \times (20 - 2x) \times x\). - المجلد:
\(V(x) = x(20 - 2x)^2\). - المشتق: \(V'(x) = (20 - 2x)(20 - 6x)\). -
النقاط الحرجة: \(x = 10\) (يعطي حجمًا صفرًا) أو
\(x = \tfrac{20}{6} \approx 3.33\). - في \(x \approx 3.33\)، تم رفع
مستوى الصوت إلى الحد الأقصى.

\subsubsection{سبب أهمية
التحسين}\label{ux633ux628ux628-ux623ux647ux645ux64aux629-ux627ux644ux62aux62dux633ux64aux646}

\begin{itemize}
\tightlist
\item
  يستخدمه المهندسون لتصميم الهياكل الفعالة.
\item
  تستخدمه الشركات لتعظيم الربح أو تقليل التكاليف.
\item
  يستخدمه العلماء لنمذجة النظم الطبيعية التي تسعى إلى التوازن.
\end{itemize}

\subsubsection{تمارين}\label{ux62aux645ux627ux631ux64aux646-10}

\begin{enumerate}
\def\labelenumi{\arabic{enumi}.}
\tightlist
\item
  لدى مزارع سياج بطول 100 متر لإحاطة حقل مستطيل على طول النهر (لذلك
  تحتاج 3 جوانب فقط إلى سياج). البحث عن أبعاد تعظيم المساحة.
\item
  ابحث عن رقمين موجبين مجموعهما 20 وحاصل ضربهما أكبر ما يمكن.
\item
  تصنع الاسطوانة من مادة مقاس 100 سم\(^2\). البحث عن أبعاد الحد الأقصى
  للحجم.
\item
  تم قطع سلك طوله 10 أمتار إلى قطعتين، إحداهما منحنية على شكل مربع،
  والأخرى على شكل دائرة. كيف ينبغي قطعها لتعظيم المساحة الإجمالية
  المغلقة؟
\item
  سيتم بناء صندوق مغلق ذو قاعدة مربعة وحجم 32 م\(^3\). البحث عن أبعاد
  التقليل من مساحة السطح.
\end{enumerate}

\subsection{3.4 التقعر ونقاط
الانعطاف}\label{ux627ux644ux62aux642ux639ux631-ux648ux646ux642ux627ux637-ux627ux644ux627ux646ux639ux637ux627ux641}

لا تخبرنا المشتقات عن المنحدرات فحسب، بل تخبرنا أيضًا عن شكل الرسم
البياني. المشتق الثاني مفيد بشكل خاص في فهم التقعر وتحديد نقاط الانقلاب.

\subsubsection{التقعر}\label{ux627ux644ux62aux642ux639ux631}

\begin{itemize}
\item
  تكون الدالة \(f(x)\) مقعرة لأعلى على الفاصل الزمني إذا كان
  \(f''(x) > 0\). ينحني الرسم البياني للأعلى، مثل الكأس.
\item
  تكون الدالة \(f(x)\) مقعرة لأسفل على فترة إذا كان \(f''(x) < 0\).
  ينحني الرسم البياني للأسفل، مثل العبوس.
\end{itemize}

يصف التقعر كيفية تغير ميل الدالة: إذا زادت المنحدرات، يكون الرسم البياني
مقعرًا للأعلى؛ إذا كانت المنحدرات تتناقص، يكون الرسم البياني مقعرًا
للأسفل.

\subsubsection{نقاط انعطافنقطة الانعطاف هي نقطة على الرسم البياني حيث
يتغير
التقعر.}\label{ux646ux642ux627ux637-ux627ux646ux639ux637ux627ux641ux646ux642ux637ux629-ux627ux644ux627ux646ux639ux637ux627ux641-ux647ux64a-ux646ux642ux637ux629-ux639ux644ux649-ux627ux644ux631ux633ux645-ux627ux644ux628ux64aux627ux646ux64a-ux62dux64aux62b-ux64aux62aux63aux64aux631-ux627ux644ux62aux642ux639ux631.}

\begin{itemize}
\tightlist
\item
  إذا كان \(f''(x) = 0\) أو \(f''(x)\) غير محدد، فإن النقطة تكون مرشحة
  لنقطة انعطاف.
\item
  للتأكيد يجب أن تتغير إشارة التقعر على جانبي النقطة.
\end{itemize}

\subsubsection{أمثلة}\label{ux623ux645ux62bux644ux629-6}

\begin{enumerate}
\def\labelenumi{\arabic{enumi}.}
\item
  \(f(x) = x^3\)

  \begin{itemize}
  \tightlist
  \item
    \(f''(x) = 6x\).
  \item
    في \(x = 0\)، \(f''(0) = 0\).
  \item
    بالنسبة إلى \(x < 0\)، \(f''(x) < 0\) → مقعر للأسفل.
  \item
    بالنسبة إلى \(x > 0\)، \(f''(x) > 0\) → مقعر للأعلى.
  \item
    وبالتالي، \((0,0)\) هي نقطة انعطاف.
  \end{itemize}
\item
  \(f(x) = x^4\)

  \begin{itemize}
  \tightlist
  \item
    \(f''(x) = 12x^2\).
  \item
    عند \(x = 0\)، \(f''(0) = 0\)، لكن التقعر لا يغير الإشارة (دائمًا ≥
    0).
  \item
    عدم وجود نقطة انعطاف.
  \end{itemize}
\end{enumerate}

\subsubsection{رسم التقعر
والمنحنى}\label{ux631ux633ux645-ux627ux644ux62aux642ux639ux631-ux648ux627ux644ux645ux646ux62dux646ux649}

\begin{itemize}
\tightlist
\item
  إذا كان \(f'(x) = 0\) و\(f''(x) > 0\)، فإن \(f\) لديه حد أدنى محلي.
\item
  إذا كان \(f'(x) = 0\) و\(f''(x) < 0\)، فإن \(f\) له حد أقصى محلي.
\item
  ويعرف هذا باختبار المشتقة الثانية.
\end{itemize}

\subsubsection{لماذا هذا
مهم}\label{ux644ux645ux627ux630ux627-ux647ux630ux627-ux645ux647ux645-1}

تساعدنا نقاط التقعر والانعطاف على فهم ``شكل'' الرسوم البيانية: حيث تنحني
أو تتسطح أو تدور. تعتبر هذه الأفكار أساسية في رسم المنحنى، والفيزياء
(التسارع)، والاقتصاد (تناقص العائدات).

\subsubsection{تمارين}\label{ux62aux645ux627ux631ux64aux646-11}

\begin{enumerate}
\def\labelenumi{\arabic{enumi}.}
\tightlist
\item
  تحديد فترات التقعر لـ \(f(x) = x^3 - 3x\). أوجد نقاط انعطافها.
\item
  بالنسبة إلى \(f(x) = \ln(x)\)، حدد التقعر ونقاط الانقلاب المحتملة.
\item
  قم بتطبيق اختبار المشتقة الثانية على \(f(x) = x^2 e^{-x}\) لتصنيف
  النقاط الحرجة.
\item
  رسم تخطيطي \(f(x) = \sin x\)، مع تحديد فترات التقعر ونقاط الانعطاف.
\item
  اشرح لماذا لا يحتوي \(f(x) = e^x\) على نقاط انعطاف.
\end{enumerate}

\subsection{3.5 رسم
المنحنى}\label{ux631ux633ux645-ux627ux644ux645ux646ux62dux646ux649}

رسم المنحنى هو عملية رسم الرسم البياني للدالة باستخدام المعلومات من
مشتقاتها. بدلاً من رسم العديد من النقاط، نقوم بتحليل السمات الرئيسية:
التقاطعات، والخطوط المقاربة، والفواصل الزمنية المتزايدة/المتناقصة،
والتقعر.

\subsubsection{خطوات رسم المنحنى1. المجال: تحديد مكان تعريف
الوظيفة.}\label{ux62eux637ux648ux627ux62a-ux631ux633ux645-ux627ux644ux645ux646ux62dux646ux6491.-ux627ux644ux645ux62cux627ux644-ux62aux62dux62fux64aux62f-ux645ux643ux627ux646-ux62aux639ux631ux64aux641-ux627ux644ux648ux638ux64aux641ux629.}

\begin{enumerate}
\def\labelenumi{\arabic{enumi}.}
\setcounter{enumi}{1}
\item
  الاعتراضات: ابحث عن مكان تقاطع الرسم البياني مع المحاور.
\item
  الخطوط المقاربة:

  \begin{itemize}
  \tightlist
  \item
    الخطوط المقاربة الرأسية تحدث عندما تكون الدالة غير محددة وتميل إلى
    اللانهاية.
  \item
    الخطوط المقاربة الأفقية أو المائلة تصف السلوك النهائي كـ
    \(x \to \pm\infty\).
  \end{itemize}
\item
  المشتق الأول \(f'(x)\):

  \begin{itemize}
  \tightlist
  \item
    الوظيفة → الإيجابية آخذة في الازدياد.
  \item
    وظيفة → سلبية آخذة في التناقص.
  \item
    أصفار \(f'(x)\) → النقاط الحرجة (الحد الأقصى/الحد الأدنى المحتمل).
  \end{itemize}
\item
  المشتق الثاني \(f''(x)\):

  \begin{itemize}
  \tightlist
  \item
    موجب → مقعر للأعلى.
  \item
    سلبي → مقعر للأسفل.
  \item
    أصفار أو غير محددة → نقاط انعطاف محتملة.
  \end{itemize}
\item
  دمج المعلومات: استخدم جميع النتائج لرسم رسم بياني واضح ودقيق.
\end{enumerate}

\subsubsection{\texorpdfstring{المثال 1:
\(f(x) = x^3 - 3x\)}{المثال 1: f(x) = x\^{}3 - 3x}}\label{ux627ux644ux645ux62bux627ux644-1-fx-x3---3x}

\begin{itemize}
\item
  المجال: جميع الأعداد الحقيقية.
\item
  الاعتراضات: عند \((0,0)\).
\item
  المشتق: \(f'(x) = 3x^2 - 3 = 3(x-1)(x+1)\).

  \begin{itemize}
  \tightlist
  \item
    زيادة: \((-\infty, -1) \cup (1, \infty)\).
  \item
    متناقص: \((-1, 1)\).
  \end{itemize}
\item
  المشتق الثاني: \(f''(x) = 6x\).

  \begin{itemize}
  \tightlist
  \item
    مقعر للأسفل لـ \(x < 0\)، مقعر للأعلى لـ \(x > 0\).
  \item
    نقطة الانعطاف عند \((0,0)\).
  \end{itemize}
\item
  الشكل: منحنى على شكل حرف S، الحد الأقصى المحلي عند \((-1, 2)\)، والحد
  الأدنى المحلي عند \((1, -2)\).
\end{itemize}

\subsubsection{\texorpdfstring{المثال 2:
\(f(x) = \frac{1}{x}\)}{المثال 2: f(x) = \textbackslash frac\{1\}\{x\}}}\label{ux627ux644ux645ux62bux627ux644-2-fx-frac1x}

\begin{itemize}
\item
  النطاق: \(x \neq 0\).
\item
  الخط المقارب العمودي: \(x = 0\).
\item
  الخط المقارب الأفقي: \(y = 0\).
\item
  المشتق: \(f'(x) = -\frac{1}{x^2}\) (سلبي دائمًا). الوظيفة تتناقص دائمًا.
\item
  المشتق الثاني: \(f''(x) = \frac{2}{x^3}\).

  \begin{itemize}
  \tightlist
  \item
    مقعر لـ \(x > 0\).
  \item
    مقعر للأسفل لـ \(x < 0\).
  \end{itemize}
\item
  الرسم البياني: القطع الزائد بفرعين.
\end{itemize}

\subsubsection{لماذا يعد رسم المنحنى
مفيدًا}\label{ux644ux645ux627ux630ux627-ux64aux639ux62f-ux631ux633ux645-ux627ux644ux645ux646ux62dux646ux649-ux645ux641ux64aux62fux627}

\begin{itemize}
\tightlist
\item
  يوفر نظرة ثاقبة للسلوك العام للوظائف دون حساب شامل.
\item
  ضروري في امتحانات حساب التفاضل والتكامل والمسائل التطبيقية.
\item
  تحليل الجسور الجبرية والفهم الهندسي.
\end{itemize}

\subsubsection{تمارين}\label{ux62aux645ux627ux631ux64aux646-12}

\begin{enumerate}
\def\labelenumi{\arabic{enumi}.}
\tightlist
\item
  ارسم منحنى \(f(x) = x^4 - 2x^2\). تحديد النقاط القصوى والصغرى ونقاط
  الانعطاف.2. تحليل ورسم \(f(x) = \ln(x)\). إظهار الاعتراضات والخطوط
  المقاربة والتقعر.
\item
  بالنسبة إلى \(f(x) = e^{-x}\)، قم بوصف النمو/الاضمحلال، والخطوط
  المقاربة، والتقعر.
\item
  ارسم الرسم البياني لـ \(f(x) = \tan x\) على الفاصل الزمني
  \((- \pi, \pi)\). وضع علامة على الخطوط المقاربة.
\item
  استخدم اختبارات المشتقة الأولى والثانية لتصنيف النقاط الحرجة لـ
  \(f(x) = x^3 - 6x^2 + 9x\).
\end{enumerate}

\#الجزء الثاني. التكاملات

\section{الفصل الرابع. المشتقات العكسية والتكاملات
المحددة}\label{ux627ux644ux641ux635ux644-ux627ux644ux631ux627ux628ux639.-ux627ux644ux645ux634ux62aux642ux627ux62a-ux627ux644ux639ux643ux633ux64aux629-ux648ux627ux644ux62aux643ux627ux645ux644ux627ux62a-ux627ux644ux645ux62dux62fux62fux629}

\subsection{4.1 التكاملات غير
المحددة}\label{ux627ux644ux62aux643ux627ux645ux644ux627ux62a-ux63aux64aux631-ux627ux644ux645ux62dux62fux62fux629}

التكامل غير المحدد هو عملية عكسية للتمايز. إذا تغيرت قياسات المشتقة، فإن
التكامل يستعيد الوظيفة الأصلية من معدل تغيره.

\subsubsection{التعريف}\label{ux627ux644ux62aux639ux631ux64aux641-4}

إذا كان \(F'(x) = f(x)\)، إذن

\[
\int f(x)\,dx = F(x) + C,
\]

حيث \(C\) هو ثابت التكامل.

يمثل كل تكامل غير محدد مجموعة من الدوال التي تختلف فقط بثابت، حيث أن
التمايز يلغي الثوابت.

\subsubsection{القواعد
الأساسية}\label{ux627ux644ux642ux648ux627ux639ux62f-ux627ux644ux623ux633ux627ux633ux64aux629}

\begin{enumerate}
\def\labelenumi{\arabic{enumi}.}
\tightlist
\item
  القاعدة الثابتة
\end{enumerate}

\[
\int c\,dx = cx + C.
\]

\begin{enumerate}
\def\labelenumi{\arabic{enumi}.}
\setcounter{enumi}{1}
\tightlist
\item
  قاعدة القوة
\end{enumerate}

\[
\int x^n\,dx = \frac{x^{n+1}}{n+1} + C, \quad n \neq -1.
\]

\begin{enumerate}
\def\labelenumi{\arabic{enumi}.}
\setcounter{enumi}{2}
\tightlist
\item
  حكم المجموع
\end{enumerate}

\[
\int \big(f(x) + g(x)\big)\,dx = \int f(x)\,dx + \int g(x)\,dx.
\]

\begin{enumerate}
\def\labelenumi{\arabic{enumi}.}
\setcounter{enumi}{3}
\tightlist
\item
  قاعدة متعددة ثابتة
\end{enumerate}

\[
\int c f(x)\,dx = c \int f(x)\,dx.
\]

\subsubsection{التكاملات
المشتركة}\label{ux627ux644ux62aux643ux627ux645ux644ux627ux62a-ux627ux644ux645ux634ux62aux631ux643ux629}

\begin{itemize}
\tightlist
\item
  \(\int e^x dx = e^x + C\)
\item
  \(\int \sin x dx = -\cos x + C\)
\item
  \(\int \cos x dx = \sin x + C\)
\item
  \(\int \frac{1}{x} dx = \ln|x| + C\)
\end{itemize}

\subsubsection{أمثلة}\label{ux623ux645ux62bux644ux629-7}

\begin{enumerate}
\def\labelenumi{\arabic{enumi}.}
\item
  \(\int (3x^2 - 4)\,dx = x^3 - 4x + C\).
\item
  \(\int \cos(2x)\,dx = \tfrac{1}{2}\sin(2x) + C\).
\item
  \(\int \frac{1}{x}\,dx = \ln|x| + C\).
\end{enumerate}

\subsubsection{التفسير}\label{ux627ux644ux62aux641ux633ux64aux631-1}

\begin{itemize}
\tightlist
\item
  التكاملات غير المحددة هي مشتقات عكسية.
\item
  إنها أساس التكاملات المحددة التي تقيس الكميات المتراكمة مثل المساحة
  والمسافة والكتلة.
\item
  في السياقات التطبيقية، يسمح لنا التكامل بالانتقال من المعدلات إلى
  الإجماليات.
\end{itemize}

\subsubsection{تمارين}\label{ux62aux645ux627ux631ux64aux646-13}

\begin{enumerate}
\def\labelenumi{\arabic{enumi}.}
\tightlist
\item
  ابحث عن \(\int (5x^4 + 2x)\,dx\).2. حساب \(\int (e^x + 3)\,dx\).
\item
  أوجد الحل العام لـ \(f'(x) = 6x\) باستخدام التكامل.
\item
  قم بتقييم \(\int \frac{2}{x}\,dx\).
\item
  إذا كانت السرعة \(v(t) = 4t\)، فابحث عن دالة الموضع \(s(t)\).
\end{enumerate}

\subsection{4.2 التكامل المحدد
كمساحة}\label{ux627ux644ux62aux643ux627ux645ux644-ux627ux644ux645ux62dux62fux62f-ux643ux645ux633ux627ux62dux629}

بينما تمثل التكاملات غير المحددة عائلات من المشتقات العكسية، فإن التكامل
المحدد يعطي قيمة عددية: المساحة المتراكمة تحت المنحنى بين نقطتين.

\subsubsection{التعريف}\label{ux627ux644ux62aux639ux631ux64aux641-5}

بالنسبة للدالة \(f(x)\) المعرفة في \([a, b]\)، التكامل المحدد هو

\[
\int_a^b f(x)\,dx = \lim_{n \to \infty} \sum_{i=1}^n f(x_i^-) \,\Delta x,
\]

حيث يتم تقسيم الفاصل الزمني \([a, b]\) إلى \(n\) فترات فرعية بعرض
\(\Delta x\)، و\(x_i^-\) هي نقطة عينة في كل فترة فرعية.

هذه هي نهاية مجاميع ريمان.

\subsubsection{التفسير
الهندسي}\label{ux627ux644ux62aux641ux633ux64aux631-ux627ux644ux647ux646ux62fux633ux64a}

\begin{itemize}
\tightlist
\item
  إذا كان \(f(x) \geq 0\) على \([a, b]\)، فإن \(\int_a^b f(x)\,dx\)
  يساوي المساحة الموجودة أسفل المنحنى \(y = f(x)\) من \(x=a\) إلى
  \(x=b\).
\item
  إذا انخفض \(f(x)\) أسفل المحور \(x\)، فإن التكامل يحسب المنطقة
  الموقعة: المناطق الموجودة أسفل المحور تعتبر سلبية.
\end{itemize}

\subsubsection{خواص التكامل
المحدد}\label{ux62eux648ux627ux635-ux627ux644ux62aux643ux627ux645ux644-ux627ux644ux645ux62dux62fux62f}

\begin{enumerate}
\def\labelenumi{\arabic{enumi}.}
\tightlist
\item
  الإضافة على فترات
\end{enumerate}

\[
\int_a^c f(x)\,dx = \int_a^b f(x)\,dx + \int_b^c f(x)\,dx.
\]

\begin{enumerate}
\def\labelenumi{\arabic{enumi}.}
\setcounter{enumi}{1}
\tightlist
\item
  عكس الحدود
\end{enumerate}

\[
\int_a^b f(x)\,dx = -\int_b^a f(x)\,dx.
\]

\begin{enumerate}
\def\labelenumi{\arabic{enumi}.}
\setcounter{enumi}{2}
\tightlist
\item
  الفاصل الزمني للعرض صفر
\end{enumerate}

\[
\int_a^a f(x)\,dx = 0.
\]

\begin{enumerate}
\def\labelenumi{\arabic{enumi}.}
\setcounter{enumi}{3}
\tightlist
\item
  الخطية
\end{enumerate}

\[
\int_a^b \big( cf(x) + g(x)\big)\,dx = c\int_a^b f(x)\,dx + \int_a^b g(x)\,dx.
\]

\subsubsection{أمثلة}\label{ux623ux645ux62bux644ux629-8}

\begin{enumerate}
\def\labelenumi{\arabic{enumi}.}
\item
  \(\int_0^2 x\,dx = \left[\tfrac{1}{2}x^2\right]_0^2 = 2.\) هذه هي
  مساحة المثلث القائم أسفل الخط \(y=x\).
\item
  \(\int_{-1}^1 x^3\,dx = 0.\) تحتوي الدالة الفردية \(x^3\) على مساحات
  متماثلة يمكن إلغاؤها.
\item
  \(\int_0^\pi \sin x\,dx = 2.\) وهذا يساوي المساحة الواقعة تحت أحد قوسي
  منحنى الجيب.
\end{enumerate}

\subsubsection{لماذا هذا
مهم}\label{ux644ux645ux627ux630ux627-ux647ux630ux627-ux645ux647ux645-2}

\begin{itemize}
\tightlist
\item
  التكاملات المحددة تقيس الكميات المتراكمة: المسافة، الكتلة، الطاقة،
  الاحتمال.- يربطون بين الحسابات الجبرية والحدس الهندسي.
\item
  الخطوة التالية هي النظرية الأساسية لحساب التفاضل والتكامل، والتي تربط
  التكاملات المحددة بالمشتقات العكسية.
\end{itemize}

\subsubsection{تمارين}\label{ux62aux645ux627ux631ux64aux646-14}

\begin{enumerate}
\def\labelenumi{\arabic{enumi}.}
\tightlist
\item
  حساب \(\int_0^3 (2x+1)\,dx\).
\item
  ابحث عن المنطقة الواقعة بين \(y = x^2\) ومحور \(x\) من \(x = 0\) إلى
  \(x = 2\).
\item
  قم بتقييم \(\int_{-2}^2 (x^2 - 1)\,dx\).
\item
  أظهر أن \(\int_{-a}^a f(x)\,dx = 0\) إذا كان \(f(x)\) فرديًا.
\item
  قم بتقريب \(\int_0^1 e^x\,dx\) باستخدام مجموع ريمان مع \(n=4\) الفترات
  الفرعية ونقاط النهاية اليمنى.
\end{enumerate}

\subsection{4.3 النظرية الأساسية في حساب التفاضل
والتكامل}\label{ux627ux644ux646ux638ux631ux64aux629-ux627ux644ux623ux633ux627ux633ux64aux629-ux641ux64a-ux62dux633ux627ux628-ux627ux644ux62aux641ux627ux636ux644-ux648ux627ux644ux62aux643ux627ux645ux644}

النظرية الأساسية لحساب التفاضل والتكامل (FTC) توحد الفكرتين الرئيسيتين
لحساب التفاضل والتكامل: التفاضل والتكامل. ويبين أن إيجاد المساحات وإيجاد
معدلات التغير وجهان لعملة واحدة.

\subsubsection{الجزء الأول: اشتقاق
التكامل}\label{ux627ux644ux62cux632ux621-ux627ux644ux623ux648ux644-ux627ux634ux62aux642ux627ux642-ux627ux644ux62aux643ux627ux645ux644}

إذا كان \(f\) مستمرًا على \([a, b]\)، فحدد

\[
F(x) = \int_a^x f(t)\,dt.
\]

ثم \(F\) قابل للتمييز، و

\[
F'(x) = f(x).
\]

بمعنى: مشتق دالة المساحة المتراكمة هو الدالة الأصلية نفسها.

\subsubsection{الجزء الثاني: تقييم التكاملات
المحددة}\label{ux627ux644ux62cux632ux621-ux627ux644ux62bux627ux646ux64a-ux62aux642ux64aux64aux645-ux627ux644ux62aux643ux627ux645ux644ux627ux62a-ux627ux644ux645ux62dux62fux62fux629}

إذا كان \(f\) مستمرًا على \([a, b]\) وكان \(F\) أي مشتق عكسي لـ \(f\)،
إذن

\[
\int_a^b f(x)\,dx = F(b) - F(a).
\]

يخبرنا هذا أنه يمكننا تقييم التكاملات المحددة ببساطة عن طريق إيجاد
المشتقة العكسية، بدلًا من حساب حدود مجاميع ريمان.

\subsubsection{أمثلة}\label{ux623ux645ux62bux644ux629-9}

\begin{enumerate}
\def\labelenumi{\arabic{enumi}.}
\item
  \(\int_0^2 x^2\,dx\).

  \begin{itemize}
  \tightlist
  \item
    المشتق العكسي: \(F(x) = \tfrac{1}{3}x^3\).
  \item
    تطبيق FTC: \(F(2) - F(0) = \tfrac{8}{3} - 0 = \tfrac{8}{3}.\)
  \end{itemize}
\item
  إذا كان \(F(x) = \int_1^x \cos t \, dt\)، فإن \(F'(x) = \cos x\).
\item
  \(\int_1^4 \frac{1}{x}\,dx\).

  \begin{itemize}
  \tightlist
  \item
    المشتق العكسي: \(\ln|x|\).
  \item
    تطبيق FTC: \(\ln 4 - \ln 1 = \ln 4.\)
  \end{itemize}
\end{enumerate}

\subsubsection{سبب أهمية لجنة التجارة الفيدرالية
(FTC).}\label{ux633ux628ux628-ux623ux647ux645ux64aux629-ux644ux62cux646ux629-ux627ux644ux62aux62cux627ux631ux629-ux627ux644ux641ux64aux62fux631ux627ux644ux64aux629-ftc.}

\begin{itemize}
\tightlist
\item
  يحول التكامل من عملية محدودة إلى حساب عملي.- يؤكد أن التفاضل والتكامل
  عمليتان عكسيتان.
\item
  إنها النظرية المركزية التي تجعل حساب التفاضل والتكامل مفيدا في
  الرياضيات والعلوم والهندسة.
\end{itemize}

\subsubsection{تمارين}\label{ux62aux645ux627ux631ux64aux646-15}

\begin{enumerate}
\def\labelenumi{\arabic{enumi}.}
\tightlist
\item
  قم بتقييم \(\int_0^3 (2x+1)\,dx\) باستخدام FTC.
\item
  إذا كان \(F(x) = \int_0^x e^t\,dt\)، فابحث عن \(F'(x)\).
\item
  حساب \(\int_0^\pi \sin x \, dx\).
\item
  أظهر أنه إذا كان \(f'(x) = g(x)\)، فإن
  \(\int_a^b g(x)\,dx = f(b) - f(a)\).
\item
  استخدم FTC لتوضيح سبب تساوي المنطقة الموجودة تحت \(y = \cos x\) من
  \(0\) إلى \(\pi/2\) 1.
\end{enumerate}

\subsection{4.4 خصائص
التكاملات}\label{ux62eux635ux627ux626ux635-ux627ux644ux62aux643ux627ux645ux644ux627ux62a}

يتمتع التكامل المحدد بالعديد من الخصائص المهمة التي تجعله مرنًا وقويًا في
التطبيقات. تتبع هذه الخصائص التعريف كحد للمجاميع ومن النظرية الأساسية
لحساب التفاضل والتكامل.

\subsubsection{الخطية}\label{ux627ux644ux62eux637ux64aux629}

بالنسبة للوظائف \(f(x)\) و\(g(x)\) والثوابت \(c, d\):

\[
\int_a^b \big(c f(x) + d g(x)\big)\,dx = c \int_a^b f(x)\,dx + d \int_a^b g(x)\,dx.
\]

وهذا يسمح لنا بتقسيم التكاملات المعقدة إلى أجزاء أبسط.

\subsubsection{الجمع على
فترات}\label{ux627ux644ux62cux645ux639-ux639ux644ux649-ux641ux62aux631ux627ux62a}

إذا كان \(a < c < b\)، إذن

\[
\int_a^b f(x)\,dx = \int_a^c f(x)\,dx + \int_c^b f(x)\,dx.
\]

يمكننا حساب التكاملات قطعة قطعة.

\subsubsection{عكس
الحدود}\label{ux639ux643ux633-ux627ux644ux62dux62fux648ux62f}

\[
\int_a^b f(x)\,dx = -\int_b^a f(x)\,dx.
\]

تبديل الحدود يغير إشارة التكامل.

\subsubsection{خاصية
المقارنة}\label{ux62eux627ux635ux64aux629-ux627ux644ux645ux642ux627ux631ux646ux629}

إذا كان \(f(x) \leq g(x)\) لجميع \(x\) في \([a, b]\)، فعندئذٍ

\[
\int_a^b f(x)\,dx \leq \int_a^b g(x)\,dx.
\]

وهذا يتيح لنا مقارنة المناطق دون حساب مباشر.

\subsubsection{عدم المساواة في القيمة
المطلقة}\label{ux639ux62fux645-ux627ux644ux645ux633ux627ux648ux627ux629-ux641ux64a-ux627ux644ux642ux64aux645ux629-ux627ux644ux645ux637ux644ux642ux629}

\[
\left| \int_a^b f(x)\,dx \right| \leq \int_a^b |f(x)|\,dx.
\]

هذه الخاصية ضرورية في اختبارات التحليل والتقارب.

\subsubsection{التماثل}\label{ux627ux644ux62aux645ux627ux62bux644}

\begin{itemize}
\item
  إذا كان \(f(x)\) زوجيًا (متماثل حول محور \(y\)):

  \[
  \int_{-a}^a f(x)\,dx = 2\int_0^a f(x)\,dx.
  \]
\item
  إذا كان \(f(x)\) فرديًا (متماثل بالنسبة إلى الأصل):

  \[
  \int_{-a}^a f(x)\,dx = 0.
  \]\#\#\# أمثلة
\end{itemize}

\begin{enumerate}
\def\labelenumi{\arabic{enumi}.}
\item
  \(\int_0^2 (3x^2 + 4)\,dx = \int_0^2 3x^2\,dx + \int_0^2 4\,dx = 8 + 8 = 16.\)
\item
  بما أن \(f(x) = x^3\) أمر غريب، \(\int_{-1}^1 x^3\,dx = 0.\)
\item
  بما أن \(f(x) = x^2\) زوجي، فإن
  \(\int_{-2}^2 x^2\,dx = 2\int_0^2 x^2\,dx = 2\cdot \tfrac{8}{3} = \tfrac{16}{3}.\)
\end{enumerate}

\subsubsection{سبب أهمية هذه
الخصائص}\label{ux633ux628ux628-ux623ux647ux645ux64aux629-ux647ux630ux647-ux627ux644ux62eux635ux627ux626ux635}

\begin{itemize}
\tightlist
\item
  يبسطون الحسابات.
\item
  تكشف عن السمات الهندسية والتناظرية للوظائف.
\item
  أنها توفر الأدوات النظرية لتحليل أكثر تقدما.
\end{itemize}

\subsubsection{تمارين}\label{ux62aux645ux627ux631ux64aux646-16}

\begin{enumerate}
\def\labelenumi{\arabic{enumi}.}
\tightlist
\item
  استخدم التماثل لتقييم \(\int_{-5}^5 (x^4 - x^3)\,dx\).
\item
  أظهر أن
  \(\int_1^4 (2x+3)\,dx = \int_1^2 (2x+3)\,dx + \int_2^4 (2x+3)\,dx\).
\item
  قم بتقييم \(\int_0^\pi \sin(x)\,dx\) ومقارنته بـ
  \(\int_{-\pi}^\pi \sin(x)\,dx\).
\item
  أثبت أنه إذا كان \(f(x) \geq 0\) على \([a, b]\)، فإن
  \(\int_a^b f(x)\,dx \geq 0\).
\item
  قم بحساب \(\int_{-3}^3 (x^2 + 1)\,dx\) باستخدام الخصائص
  الزوجية/الفردية.
\end{enumerate}

\section{الفصل الخامس. تقنيات
التكامل}\label{ux627ux644ux641ux635ux644-ux627ux644ux62eux627ux645ux633.-ux62aux642ux646ux64aux627ux62a-ux627ux644ux62aux643ux627ux645ux644}

\subsection{5.1
الاستبدال}\label{ux627ux644ux627ux633ux62aux628ux62fux627ux644}

إحدى أكثر تقنيات التكامل فائدة هي طريقة الاستبدال، وتسمى أيضًا
-u-substitution-. إنها العملية العكسية لقاعدة السلسلة للمشتقات.

\subsubsection{الفكرة}\label{ux627ux644ux641ux643ux631ux629-1}

إذا كان التكامل يحتوي على دالة مركبة، فيمكننا تبسيطه عن طريق تغيير
المتغيرات.

رسميًا، إذا كانت \(u = g(x)\) دالة قابلة للتفاضل، إذن

\[
\int f(g(x)) g'(x)\,dx = \int f(u)\,du.
\]

هذا الاستبدال يجعل تقييم التكامل أسهل.

\subsubsection{خطوات
الاستبدال}\label{ux62eux637ux648ux627ux62a-ux627ux644ux627ux633ux62aux628ux62fux627ux644}

\begin{enumerate}
\def\labelenumi{\arabic{enumi}.}
\tightlist
\item
  حدد دالة داخلية \(u = g(x)\) والتي تظهر مشتقتها أيضًا في التكامل.
\item
  حساب \(du = g'(x)\,dx\).
\item
  أعد كتابة التكامل بدلالة \(u\).
\item
  التكامل فيما يتعلق بـ \(u\).
\item
  استبدل \(u = g(x)\) بالخلف.
\end{enumerate}

\subsubsection{أمثلة}\label{ux623ux645ux62bux644ux629-10}

\begin{enumerate}
\def\labelenumi{\arabic{enumi}.}
\item
  استبدال بسيط

  \[
  \int 2x \cos(x^2)\,dx
  \]

  دع \(u = x^2\)، إذن \(du = 2x\,dx\). ثم يصبح التكامل
  \(\int \cos u \,du = \sin u + C = \sin(x^2) + C\).
\item
  حالة لوغاريتمية

  \[\int \frac{2x}{x^2+1}\,dx
  \]

  Let \(u = x^2 + 1\), so \(du = 2x\,dx\). Then integral becomes
  \(\int \frac{1}{u}\,du = \ln|u| + C = \ln(x^2+1) + C\).
\item
  Trigonometric substitution

  \[
  \int \sin(3x)\,dx
  \]

  Let \(u = 3x\), so \(du = 3\,dx\), hence \(dx = \frac{du}{3}\).
  Integral becomes
  \(\tfrac{1}{3}\int \sin u\,du = -\tfrac{1}{3}\cos u + C = -\tfrac{1}{3}\cos(3x) + C\).
\end{enumerate}

\subsubsection{Definite Integrals with
Substitution}\label{definite-integrals-with-substitution}

When evaluating definite integrals, we must also change the limits:

\[
\int_a^b f(g(x)) g'(x)\,dx = \int_{g(a)}^{g(b)} f(u)\,du.
\]

Example:

\[
\int_0^1 2x e^{x^2}\,dx.
\]

Let \(u = x^2\), \(du = 2x\,dx\). Limits: when \(x=0, u=0\); when
\(x=1, u=1\). So the integral becomes

\[
\int_0^1 e^u\,du = e - 1.
\]

\subsubsection{Exercises}\label{exercises-1}

\begin{enumerate}
\def\labelenumi{\arabic{enumi}.}
\tightlist
\item
  Evaluate \(\int (x^2+1)^5 (2x)\,dx\).
\item
  Compute \(\int \frac{\cos x}{\sin x}\,dx\).
\item
  Evaluate \(\int_0^\pi \sin(2x)\,dx\) using substitution.
\item
  Find \(\int e^{3x}\,dx\).
\item
  Compute \(\int \frac{1}{\sqrt{1+x^2}}\,dx\) by letting \(u = 1+x^2\).
\end{enumerate}

\subsection{5.2 Integration by Parts}\label{integration-by-parts}

Integration by parts is a technique that comes from the product rule for
derivatives. It helps evaluate integrals involving products of functions
that are not easily handled by substitution alone.

\subsubsection{The Formula}\label{the-formula}

From the product rule:

\[
\frac{d}{dx[u(x)v(x)] = u'(x)v(x) + u(x)v'(x).
\]

Integrating both sides gives the integration by parts formula:

\[
\int u\,dv = uv - \int v\,du.
\]

Here:

\begin{itemize}
\tightlist
\item
  \(u\) = a function chosen to be differentiated,
\item
  \(dv\) = the remaining part of the integrand to be integrated.
\end{itemize}

\subsubsection{\texorpdfstring{Choosing \(u\) and
\(dv\)}{Choosing u and dv}}\label{choosing-u-and-dv}

A common guideline is LIATE (Logarithmic, Inverse trig, Algebraic,
Trigonometric, Exponential).

\begin{itemize}
\tightlist
\item
  Choose \(u\) from the earliest category present.
\item
  Choose \(dv\) as the rest.
\end{itemize}

\subsubsection{Examples}\label{examples-1}

\begin{enumerate}
\def\labelenumi{\arabic{enumi}.}
\tightlist
\item
  Polynomial × Exponential
\end{enumerate}

\[
\int x e^x\,dx
\]دع \(u = x\)، \(dv = e^x dx\). ثم \(du = dx\)، \(v = e^x\).

\[
\int x e^x\,dx = x e^x - \int e^x dx = x e^x - e^x + C.
\]

\begin{enumerate}
\def\labelenumi{\arabic{enumi}.}
\setcounter{enumi}{1}
\tightlist
\item
  كثير الحدود × علم حساب المثلثات
\end{enumerate}

\[
\int x \cos x\,dx
\]

دع \(u = x\)، \(dv = \cos x dx\). ثم \(du = dx\)، \(v = \sin x\).

\[
\int x \cos x\,dx = x \sin x - \int \sin x dx = x \sin x + \cos x + C.
\]

\begin{enumerate}
\def\labelenumi{\arabic{enumi}.}
\setcounter{enumi}{2}
\tightlist
\item
  اللوغاريتم
\end{enumerate}

\[
\int \ln x\,dx
\]

دع \(u = \ln x\)، \(dv = dx\). ثم \(du = \frac{1}{x}dx\)، \(v = x\).

\[
\int \ln x\,dx = x \ln x - \int 1 dx = x \ln x - x + C.
\]

\subsubsection{مثال على التكامل
المحدد}\label{ux645ux62bux627ux644-ux639ux644ux649-ux627ux644ux62aux643ux627ux645ux644-ux627ux644ux645ux62dux62fux62f}

\[
\int_0^1 x e^x\,dx
\]

باستخدام النتيجة السابقة: \(\int x e^x dx = (x-1)e^x\). تقييم:

\[
\big[(x-1)e^x\big]_0^1 = (0)e^1 - (-1)e^0 = 0 + 1 = 1.
\]

\subsubsection{لماذا هذا
مهم}\label{ux644ux645ux627ux630ux627-ux647ux630ux627-ux645ux647ux645-3}

يعد التكامل بالأجزاء أمرًا بالغ الأهمية عندما يفشل الاستبدال، خاصة مع
اللوغاريتمات والدوال المثلثية العكسية والمنتجات التي تتضمن كثيرات الحدود
ذات الدوال الأسية أو المثلثية.

\subsubsection{تمارين}\label{ux62aux645ux627ux631ux64aux646-17}

\begin{enumerate}
\def\labelenumi{\arabic{enumi}.}
\tightlist
\item
  قم بتقييم \(\int x \sin x\,dx\).
\item
  ابحث عن \(\int e^x \cos x\,dx\).
\item
  حساب \(\int_1^2 \ln x\,dx\).
\item
  قم بتقييم \(\int x^2 e^x\,dx\).
\item
  استخدم التكامل بالأجزاء لإظهار
  \(\int \arctan x\,dx = x\arctan x - \tfrac{1}{2}\ln(1+x^2) + C\).
\end{enumerate}

\subsection{5.3 التكاملات والبدائل
المثلثية}\label{ux627ux644ux62aux643ux627ux645ux644ux627ux62a-ux648ux627ux644ux628ux62fux627ux626ux644-ux627ux644ux645ux62bux644ux62bux64aux629}

تشتمل العديد من التكاملات على دوال مثلثية. ويمكن في كثير من الأحيان
تبسيط هذه الأمور باستخدام الهويات أو عن طريق إجراء بدائل خاصة.

\subsubsection{التكاملات
المثلثية}\label{ux627ux644ux62aux643ux627ux645ux644ux627ux62a-ux627ux644ux645ux62bux644ux62bux64aux629}

\begin{enumerate}
\def\labelenumi{\arabic{enumi}.}
\tightlist
\item
  قوى الجيب وجيب التمام
\end{enumerate}

\begin{itemize}
\tightlist
\item
  إذا كان أس الجيب فرديًا: احفظ واحدًا \(\sin x\)، وحول الباقي بـ
  \(\sin^2x = 1 - \cos^2x\)، واستبدل \(u = \cos x\).
\item
  إذا كانت قوة جيب التمام فردية: احفظ واحدًا \(\cos x\)، وقم بتحويل
  الباقي بـ \(\cos^2x = 1 - \sin^2x\)، واستبدل \(u = \sin x\).
\item
  إذا كان كلاهما متساويًا: استخدم متطابقات نصف الزاوية.
\end{itemize}

مثال:

\[
\int \sin^3x \cos x \, dx
\]

دع \(u = \sin x\)، \(du = \cos x\,dx\):

\[
\int u^3\,du = \tfrac{u^4}{4} + C = \tfrac{\sin^4x}{4} + C.\]

\begin{enumerate}
\def\labelenumi{\arabic{enumi}.}
\setcounter{enumi}{1}
\tightlist
\item
  Products of sine and cosine with different angles Use product-to-sum
  formulas:
\end{enumerate}

\[
\sin A \cos B = \tfrac{1}{2}[\sin(A+B) + \sin(A-B)].
\]

Example:

\[
\int \sin(2x)\cos(3x)\,dx = \tfrac{1}{2}\int [\sin(5x) - \sin(x)]\,dx.
\]

\begin{enumerate}
\def\labelenumi{\arabic{enumi}.}
\setcounter{enumi}{2}
\tightlist
\item
  Powers of secant and tangent
\end{enumerate}

\begin{itemize}
\tightlist
\item
  If the power of secant is even: save \(\sec^2x\), convert the rest
  with \(\sec^2x = 1 + \tan^2x\), and substitute \(u = \tan x\).
\item
  If the power of tangent is odd: save \(\sec^2x\), convert the rest
  with \(\tan^2x = \sec^2x - 1\), and substitute \(u = \tan x\).
\end{itemize}

Example:

\[
\int \tan^3x \sec^2x \, dx
\]

Let \(u = \tan x\), \(du = \sec^2x\,dx\):

\[
\int u^3\,du = \tfrac{u^4}{4} + C = \tfrac{\tan^4x}{4} + C.
\]

\subsubsection{Trigonometric
Substitutions}\label{trigonometric-substitutions}

For integrals involving \(\sqrt{a^2 - x^2}\), \(\sqrt{a^2 + x^2}\), or
\(\sqrt{x^2 - a^2}\), use special substitutions:

\begin{enumerate}
\def\labelenumi{\arabic{enumi}.}
\tightlist
\item
  \(x = a \sin \theta\), for \(\sqrt{a^2 - x^2}\).
\item
  \(x = a \tan \theta\), for \(\sqrt{a^2 + x^2}\).
\item
  \(x = a \sec \theta\), for \(\sqrt{x^2 - a^2}\).
\end{enumerate}

Example:

\[
\int \sqrt{a^2 - x^2}\,dx
\]

Let \(x = a\sin\theta\), so \(dx = a\cos\theta\,d\theta\):

\[
\int \sqrt{a^2 - a^2\sin^2\theta}(a\cos\theta\,d\theta) = \int a^2 \cos^2\theta \, d\theta.
\]

تبسيط باستخدام متطابقات نصف الزاوية.

\subsubsection{سبب أهمية هذه
التقنيات}\label{ux633ux628ux628-ux623ux647ux645ux64aux629-ux647ux630ux647-ux627ux644ux62aux642ux646ux64aux627ux62a}

\begin{itemize}
\tightlist
\item
  يحولون الأشكال الجبرية الصعبة إلى أشكال مثلثية يمكن التحكم فيها.
\item
  إنها مفيدة بشكل خاص في المسائل المتعلقة بالمساحات والأحجام وأطوال
  القوس.
\item
  أنها تضع الأساس لأساليب التكامل المتقدمة.
\end{itemize}

\subsubsection{تمارين}\label{ux62aux645ux627ux631ux64aux646-18}

\begin{enumerate}
\def\labelenumi{\arabic{enumi}.}
\tightlist
\item
  قم بتقييم \(\int \sin^4x \cos^2x \, dx\).
\item
  حساب \(\int \sin(5x)\cos(2x)\,dx\).
\item
  قم بتقييم \(\int \tan^2x \sec^2x \, dx\).
\item
  ابحث عن \(\int \sqrt{9 - x^2}\,dx\) باستخدام الاستبدال.
\item
  أظهر أن
  \(\int \frac{dx}{\sqrt{x^2 + a^2}} = \ln|x + \sqrt{x^2 + a^2}| + C\)
  باستخدام \(x = a\tan\theta\).
\end{enumerate}

\subsection{5.4 الكسور الجزئيةعند دمج الدوال الكسرية (نسب كثيرات
الحدود)، إحدى الطرق القوية هي تحليل الكسور الجزئية. تعبر هذه التقنية عن
الكسر المعقد كمجموع الكسور الأبسط التي يسهل
دمجها.}\label{ux627ux644ux643ux633ux648ux631-ux627ux644ux62cux632ux626ux64aux629ux639ux646ux62f-ux62fux645ux62c-ux627ux644ux62fux648ux627ux644-ux627ux644ux643ux633ux631ux64aux629-ux646ux633ux628-ux643ux62bux64aux631ux627ux62a-ux627ux644ux62dux62fux648ux62f-ux625ux62dux62fux649-ux627ux644ux637ux631ux642-ux627ux644ux642ux648ux64aux629-ux647ux64a-ux62aux62dux644ux64aux644-ux627ux644ux643ux633ux648ux631-ux627ux644ux62cux632ux626ux64aux629.-ux62aux639ux628ux631-ux647ux630ux647-ux627ux644ux62aux642ux646ux64aux629-ux639ux646-ux627ux644ux643ux633ux631-ux627ux644ux645ux639ux642ux62f-ux643ux645ux62cux645ux648ux639-ux627ux644ux643ux633ux648ux631-ux627ux644ux623ux628ux633ux637-ux627ux644ux62aux64a-ux64aux633ux647ux644-ux62fux645ux62cux647ux627.}

\subsubsection{الفكرة}\label{ux627ux644ux641ux643ux631ux629-2}

إذا كانت \(R(x) = \frac{P(x)}{Q(x)}\) دالة كسرية، حيث تكون درجة \(P(x)\)
أقل من درجة \(Q(x)\)، فيمكننا تحليل \(R(x)\) إلى كسور أبسط.

تتوافق هذه القطع الأبسط مع عوامل المقام \(Q(x)\).

\subsubsection{النماذج
المشتركة}\label{ux627ux644ux646ux645ux627ux630ux62c-ux627ux644ux645ux634ux62aux631ux643ux629}

\begin{enumerate}
\def\labelenumi{\arabic{enumi}.}
\tightlist
\item
  العوامل الخطية المميزة إذا
\end{enumerate}

\[
\frac{1}{(x-a)(x-b)},
\]

ثم تتحلل كما

\[
\frac{A}{x-a} + \frac{B}{x-b}.
\]

\begin{enumerate}
\def\labelenumi{\arabic{enumi}.}
\setcounter{enumi}{1}
\tightlist
\item
  العوامل الخطية المتكررة إذا كان المقام يحتوي على \((x-a)^n\)، فالحدود
  هي
\end{enumerate}

\[
\frac{A_1}{x-a} + \frac{A_2}{(x-a)^2} + \dots + \frac{A_n}{(x-a)^n}.
\]

\begin{enumerate}
\def\labelenumi{\arabic{enumi}.}
\setcounter{enumi}{2}
\tightlist
\item
  العوامل التربيعية غير القابلة للاختزال إذا كان المقام يحتوي على
  \((x^2+bx+c)\)، فإن البسط خطي:
\end{enumerate}

\[
\frac{Ax+B}{x^2+bx+c}.
\]

\subsubsection{مثال 1: العوامل الخطية
المميزة}\label{ux645ux62bux627ux644-1-ux627ux644ux639ux648ux627ux645ux644-ux627ux644ux62eux637ux64aux629-ux627ux644ux645ux645ux64aux632ux629}

\[
\int \frac{1}{x^2 - 1}\,dx
\]

مقام العامل: \((x-1)(x+1)\). تتحلل:

\[
\frac{1}{x^2-1} = \frac{1}{2}\left(\frac{1}{x-1} - \frac{1}{x+1}\right).
\]

دمج:

\[
\int \frac{1}{x^2 - 1}\,dx = \tfrac{1}{2}\ln\left|\frac{x-1}{x+1}\right| + C.
\]

\subsubsection{مثال 2: العامل الخطي
المتكرر}\label{ux645ux62bux627ux644-2-ux627ux644ux639ux627ux645ux644-ux627ux644ux62eux637ux64a-ux627ux644ux645ux62aux643ux631ux631}

\[
\int \frac{1}{(x-1)^2}\,dx
\]

هذا بسيط بالفعل:

\[
\int (x-1)^{-2}\,dx = -\frac{1}{x-1} + C.
\]

\subsubsection{مثال 3: العامل التربيعي غير القابل
للاختزال}\label{ux645ux62bux627ux644-3-ux627ux644ux639ux627ux645ux644-ux627ux644ux62aux631ux628ux64aux639ux64a-ux63aux64aux631-ux627ux644ux642ux627ux628ux644-ux644ux644ux627ux62eux62aux632ux627ux644}

\[
\int \frac{x}{x^2+1}\,dx
\]

استبدل \(u = x^2+1\)، أو تعرف على أن البسط مشتق من المقام.

\[
\int \frac{x}{x^2+1}\,dx = \tfrac{1}{2}\ln(x^2+1) + C.
\]

\subsubsection{خطوات تحليل الكسور
الجزئية}\label{ux62eux637ux648ux627ux62a-ux62aux62dux644ux64aux644-ux627ux644ux643ux633ux648ux631-ux627ux644ux62cux632ux626ux64aux629}

\begin{enumerate}
\def\labelenumi{\arabic{enumi}.}
\tightlist
\item
  عامل المقام.
\item
  اكتب صيغة الكسر الجزئي العام.
\item
  اضرب بالمقام لمسح الكسور.
\item
  حل الثوابت المجهولة.
\item
  دمج كل مصطلح.\#\#\# لماذا هذا مهم
\end{enumerate}

\begin{itemize}
\tightlist
\item
  تحويل الوظائف العقلانية المعقدة إلى أشكال لوغاريتمية أو ظلية بسيطة.
\item
  مفيدة بشكل خاص في المعادلات التفاضلية وتحويلات لابلاس.
\item
  أساسيات في حساب التفاضل والتكامل المتقدم والهندسة.
\end{itemize}

\subsubsection{تمارين}\label{ux62aux645ux627ux631ux64aux646-19}

\begin{enumerate}
\def\labelenumi{\arabic{enumi}.}
\tightlist
\item
  قم بتحليل ودمج \(\int \frac{3x+5}{x^2-1}\,dx\).
\item
  قم بتقييم \(\int \frac{1}{x^2(x+1)}\,dx\).
\item
  حساب \(\int \frac{2x+1}{x^2+2x+2}\,dx\).
\item
  ابحث عن \(\int \frac{1}{x^3 - x}\,dx\).
\item
  أظهر أن \(\int \frac{dx}{x^2+1} = \arctan x + C\) باستخدام الكسور
  الجزئية أو التعويض.
\end{enumerate}

\subsection{5.5 التكاملات غير
الصحيحة}\label{ux627ux644ux62aux643ux627ux645ux644ux627ux62a-ux63aux64aux631-ux627ux644ux635ux62dux64aux62dux629}

لا يمكن تقييم بعض التكاملات بشكل مباشر لأن الفاصل الزمني لا نهائي أو أن
التكامل يصبح غير محدود. وتسمى هذه التكاملات غير لائقة. يتم تعريفها
باستخدام الحدود.

\subsubsection{التعريف}\label{ux627ux644ux62aux639ux631ux64aux641-6}

\begin{enumerate}
\def\labelenumi{\arabic{enumi}.}
\tightlist
\item
  الفاصل الزمني اللانهائي
\end{enumerate}

\[
\int_a^\infty f(x)\,dx = \lim_{b \to \infty} \int_a^b f(x)\,dx.
\]

\[
\int_{-\infty}^a f(x)\,dx = \lim_{b \to -\infty} \int_b^a f(x)\,dx.
\]

\begin{enumerate}
\def\labelenumi{\arabic{enumi}.}
\setcounter{enumi}{1}
\tightlist
\item
  التكامل غير المحدود إذا كان \(f(x)\) يحتوي على خط تقارب عمودي عند
  \(c\)، إذن
\end{enumerate}

\[
\int_a^c f(x)\,dx = \lim_{t \to c^-} \int_a^t f(x)\,dx,
\]

\[
\int_c^b f(x)\,dx = \lim_{t \to c^+} \int_t^b f(x)\,dx.
\]

\subsubsection{التقارب
والاختلاف}\label{ux627ux644ux62aux642ux627ux631ux628-ux648ux627ux644ux627ux62eux62aux644ux627ux641}

\begin{itemize}
\tightlist
\item
  إذا كانت النهاية موجودة ومنتهية فإن التكامل غير الصحيح يتقارب.
\item
  إذا كانت النهاية غير موجودة أو لا نهائية، فإن التكامل غير الصحيح
  يتباعد.
\end{itemize}

\subsubsection{أمثلة}\label{ux623ux645ux62bux644ux629-11}

\begin{enumerate}
\def\labelenumi{\arabic{enumi}.}
\tightlist
\item
  الاضمحلال الأسي
\end{enumerate}

\[
\int_1^\infty \frac{1}{x^2}\,dx = \lim_{b \to \infty} \Big[-\tfrac{1}{x}\Big]_1^b = 1.
\]

هذا يتقارب.

\begin{enumerate}
\def\labelenumi{\arabic{enumi}.}
\setcounter{enumi}{1}
\tightlist
\item
  الوظيفة التوافقية
\end{enumerate}

\[
\int_1^\infty \frac{1}{x}\,dx = \lim_{b \to \infty} \ln b.
\]

وهذا يتباعد إلى ما لا نهاية.

\begin{enumerate}
\def\labelenumi{\arabic{enumi}.}
\setcounter{enumi}{2}
\tightlist
\item
  الخط المقارب عند 0
\end{enumerate}

\[
\int_0^1 \frac{1}{\sqrt{x}}\,dx = \lim_{t \to 0^+} \int_t^1 x^{-1/2}\,dx.
\]

\[
= \lim_{t \to 0^+} [2\sqrt{x}]_t^1 = 2.
\]

هذا يتقارب.

\begin{enumerate}
\def\labelenumi{\arabic{enumi}.}
\setcounter{enumi}{3}
\tightlist
\item
  الخط المقارب عند 0 (متباعد)
\end{enumerate}

\[\int_0^1 \frac{1}{x}\,dx = \lim_{t \to 0^+} \ln(1) - \ln(t).
\]

This diverges since \(\ln(t) \to -\infty\).

\subsubsection{Comparison Test for Improper
Integrals}\label{comparison-test-for-improper-integrals}

\begin{itemize}
\tightlist
\item
  If \(0 \leq f(x) \leq g(x)\) for large \(x\), and \(\int g(x)\,dx\)
  converges, then \(\int f(x)\,dx\) also converges.
\item
  If \(\int f(x)\,dx\) diverges and \(f(x) \geq g(x) \geq 0\), then
  \(\int g(x)\,dx\) also diverges.
\end{itemize}

\subsubsection{Why Improper Integrals
Matter}\label{why-improper-integrals-matter}

\begin{itemize}
\tightlist
\item
  They extend integration to infinite domains and unbounded functions.
\item
  They are essential in probability (continuous distributions), physics
  (gravitational/electric fields), and Fourier analysis.
\end{itemize}

\subsubsection{Exercises}\label{exercises-2}

\begin{enumerate}
\def\labelenumi{\arabic{enumi}.}
\tightlist
\item
  Determine whether \(\int_1^\infty \frac{1}{x^p}\,dx\) converges for
  various values of \(p\).
\item
  Evaluate \(\int_0^\infty e^{-x}\,dx\).
\item
  Test convergence of \(\int_0^1 \frac{1}{x^p}\,dx\) depending on \(p\).
\item
  Compute \(\int_{-\infty}^\infty \frac{1}{1+x^2}\,dx\).
\item
  Use the comparison test to show that
  \(\int_1^\infty \frac{1}{x^2+1}\,dx\) converges.
\end{enumerate}

\section{Chapter 6. Applications of
Integration}\label{chapter-6.-applications-of-integration}

\subsection{6.1 Areas and Volumes}\label{areas-and-volumes}

One of the most important applications of integration is finding areas
under curves and volumes of solids.

\subsubsection{Area Between Curves}\label{area-between-curves}

If \(f(x) \geq g(x)\) on \([a, b]\), then the area between the curves
\(y=f(x)\) and \(y=g(x)\) is

\[
A = \int_a^b \big(f(x) - g(x)\big)\,dx.
\]

Example: Find the area between \(y=x^2\) and \(y=x\) on \([0,1]\).

\[
A = \int_0^1 (x - x^2)\,dx = \left[\tfrac{1}{2}x^2 - \tfrac{1}{3}x^3\right]_0^1 = \tfrac{1}{6}.
\]

\subsubsection{Volumes by Slicing}\label{volumes-by-slicing}

If a solid has cross-sectional area \(A(x)\) at position \(x\), then the
volume is

\[
V = \int_a^b A(x)\,dx.
\]

\subsubsection{مجلدات
الثورة}\label{ux645ux62cux644ux62fux627ux62a-ux627ux644ux62bux648ux631ux629}

عندما تدور منطقة ما حول محور، يمكن إيجاد حجم المادة الصلبة الناتجة عن
طريق التكامل.

\begin{enumerate}
\def\labelenumi{\arabic{enumi}.}
\tightlist
\item
  طريقة القرصإذا كانت المنطقة الموجودة ضمن \(y=f(x)\)، \(x\in[a,b]\)،
  تدور حول محور \(x\):
\end{enumerate}

\[
V = \pi \int_a^b [f(x)]^2\,dx.
\]

\begin{enumerate}
\def\labelenumi{\arabic{enumi}.}
\setcounter{enumi}{1}
\tightlist
\item
  طريقة الغسالة إذا كانت المنطقة الواقعة بين \(y=f(x)\) و\(y=g(x)\) تدور
  حول محور \(x\):
\end{enumerate}

\[
V = \pi \int_a^b \Big([f(x)]^2 - [g(x)]^2\Big)\,dx.
\]

\begin{enumerate}
\def\labelenumi{\arabic{enumi}.}
\setcounter{enumi}{2}
\tightlist
\item
  طريقة شل إذا كانت المنطقة الموجودة ضمن \(y=f(x)\) تدور حول محور \(y\):
\end{enumerate}

\[
V = 2\pi \int_a^b x f(x)\,dx.
\]

\subsubsection{أمثلة}\label{ux623ux645ux62bux644ux629-12}

\begin{enumerate}
\def\labelenumi{\arabic{enumi}.}
\tightlist
\item
  طريقة القرص تدور \(y=\sqrt{x}\)، \(0 \leq x \leq 4\)، حول محور \(x\):
\end{enumerate}

\[
V = \pi \int_0^4 (\sqrt{x})^2\,dx = \pi \int_0^4 x\,dx = \pi \left[\tfrac{1}{2}x^2\right]_0^4 = 8\pi.
\]

\begin{enumerate}
\def\labelenumi{\arabic{enumi}.}
\setcounter{enumi}{1}
\tightlist
\item
  طريقة الغسالة المنطقة الدائرية بين \(y=\sqrt{x}\) و \(y=1\)،
  \(0 \leq x \leq 1\)، حول \(x\)-المحور:
\end{enumerate}

\[
V = \pi \int_0^1 \big((\sqrt{x})^2 - (1)^2\big)\,dx = \pi \int_0^1 (x-1)\,dx = -\tfrac{\pi}{2}.
\]

(خذ القيمة المطلقة للحجم: \(V = \tfrac{\pi}{2}\)).

\begin{enumerate}
\def\labelenumi{\arabic{enumi}.}
\setcounter{enumi}{2}
\tightlist
\item
  طريقة شل المنطقة الدائرية ضمن \(y=x\)، \(0 \leq x \leq 1\)، حول محور
  \(y\):
\end{enumerate}

\[
V = 2\pi \int_0^1 x(x)\,dx = 2\pi \int_0^1 x^2\,dx = 2\pi \cdot \tfrac{1}{3} = \tfrac{2\pi}{3}.
\]

\subsubsection{لماذا هذا
مهم}\label{ux644ux645ux627ux630ux627-ux647ux630ux627-ux645ux647ux645-4}

\begin{itemize}
\tightlist
\item
  يوفر طرقًا دقيقة لحساب المساحات والأحجام في الهندسة.
\item
  أساسي في الفيزياء والهندسة والاحتمالات.
\item
  يقدم التفكير الهندسي مع التكامل.
\end{itemize}

\subsubsection{تمارين}\label{ux62aux645ux627ux631ux64aux646-20}

\begin{enumerate}
\def\labelenumi{\arabic{enumi}.}
\tightlist
\item
  ابحث عن المنطقة الواقعة بين \(y=\cos x\) و\(y=\sin x\) على
  \([0, \pi/2]\).
\item
  احسب حجم المادة الصلبة المتكونة عن طريق الدوران \(y=x^2\)،
  \(0 \leq x \leq 1\)، حول المحور \(x\).
\item
  أوجد حجم المادة الصلبة المتكونة من خلال دوران المنطقة بين \(y=x\) و
  \(y=\sqrt{x}\) على \([0,1]\) حول المحور \(y\).
\item
  استخدم طريقة الحلقة لحساب حجم المادة الصلبة المتكونة بتدوير
  \(y=\sqrt{1-x^2}\) (نصف دائرة) حول محور \(x\).
\item
  ابحث عن المنطقة المحصورة بين \(y=x^2+1\) و\(y=3x\).
\end{enumerate}

\subsection{6.2 طول القوس ومساحة السطحيمكن أيضًا استخدام التكامل لقياس
طول المنحنيات ومساحة سطح المواد الصلبة الناتجة عن المنحنيات
الدوارة.}\label{ux637ux648ux644-ux627ux644ux642ux648ux633-ux648ux645ux633ux627ux62dux629-ux627ux644ux633ux637ux62dux64aux645ux643ux646-ux623ux64aux636ux627-ux627ux633ux62aux62eux62fux627ux645-ux627ux644ux62aux643ux627ux645ux644-ux644ux642ux64aux627ux633-ux637ux648ux644-ux627ux644ux645ux646ux62dux646ux64aux627ux62a-ux648ux645ux633ux627ux62dux629-ux633ux637ux62d-ux627ux644ux645ux648ux627ux62f-ux627ux644ux635ux644ux628ux629-ux627ux644ux646ux627ux62aux62cux629-ux639ux646-ux627ux644ux645ux646ux62dux646ux64aux627ux62a-ux627ux644ux62fux648ux627ux631ux629.}

\subsubsection{طول
القوس}\label{ux637ux648ux644-ux627ux644ux642ux648ux633}

للحصول على منحنى سلس \(y=f(x)\) على الفاصل الزمني \([a,b]\)، يكون طول
المنحنى

\[
L = \int_a^b \sqrt{1 + \big(f'(x)\big)^2}\,dx.
\]

يأتي هذا من تقريب المنحنى بأجزاء خطية وأخذ النهاية.

مثال: أوجد طول \(y=\tfrac{1}{2}x^{3/2}\) من \(x=0\) إلى \(x=4\).

\begin{itemize}
\tightlist
\item
  المشتق: \(f'(x) = \tfrac{3}{4}\sqrt{x}\).
\item
  الصيغة:
\end{itemize}

\[
L = \int_0^4 \sqrt{1 + \Big(\tfrac{3}{4}\sqrt{x}\Big)^2}\,dx
= \int_0^4 \sqrt{1 + \tfrac{9}{16}x}\,dx.
\]

يمكن تقييم هذا التكامل باستخدام الاستبدال.

\subsubsection{المساحة السطحية
للثورة}\label{ux627ux644ux645ux633ux627ux62dux629-ux627ux644ux633ux637ux62dux64aux629-ux644ux644ux62bux648ux631ux629}

إذا كان المنحنى \(y=f(x)\)، \(a \leq x \leq b\)، يدور حول محور \(x\)،
فإن مساحة سطح المادة الصلبة الناتجة هي

\[
S = 2\pi \int_a^b f(x)\sqrt{1 + \big(f'(x)\big)^2}\,dx.
\]

إذا كان يدور حول محور \(y\):

\[
S = 2\pi \int_a^b x \sqrt{1 + \big(f'(x)\big)^2}\,dx.
\]

\subsubsection{أمثلة}\label{ux623ux645ux62bux644ux629-13}

\begin{enumerate}
\def\labelenumi{\arabic{enumi}.}
\tightlist
\item
  طول قوس الخط بالنسبة إلى \(y=x\)، \(0 \leq x \leq 3\):
\end{enumerate}

\[
L = \int_0^3 \sqrt{1+(1)^2}\,dx = \int_0^3 \sqrt{2}\,dx = 3\sqrt{2}.
\]

\begin{enumerate}
\def\labelenumi{\arabic{enumi}.}
\setcounter{enumi}{1}
\tightlist
\item
  مساحة سطح الكرة خذ \(y = \sqrt{r^2 - x^2}\)، \(-r \leq x \leq r\)، وقم
  بالدوران حول محور \(x\).
\end{enumerate}

\[
S = 2\pi \int_{-r}^r \sqrt{r^2 - x^2}\sqrt{1+\left(\frac{-x}{\sqrt{r^2-x^2}}\right)^2}\,dx.
\]

التبسيط يعطي \(S = 4\pi r^2\)، الصيغة المألوفة لمساحة سطح الكرة.

\subsubsection{لماذا هذا
مهم}\label{ux644ux645ux627ux630ux627-ux647ux630ux627-ux645ux647ux645-5}

\begin{itemize}
\tightlist
\item
  طول القوس يمتد فكرة المسافة إلى المسارات المنحنية.
\item
  مساحة سطح الثورة لها تطبيقات في الفيزياء والهندسة والتصميم.
\item
  يوفر جسرا بين حساب التفاضل والتكامل والهندسة.
\end{itemize}

\subsubsection{تمارين}\label{ux62aux645ux627ux631ux64aux646-21}

\begin{enumerate}
\def\labelenumi{\arabic{enumi}.}
\tightlist
\item
  أوجد طول قوس \(y=\sqrt{x}\) من \(x=0\) إلى \(x=4\).2. احسب مساحة سطح
  المادة الصلبة التي تم الحصول عليها عن طريق تدوير \(y=x^2\)،
  \(0 \leq x \leq 1\)، حول محور \(x\).
\item
  أوجد طول قوس \(y=\ln(\cosh x)\) من \(x=0\) إلى \(x=1\).
\item
  أظهر أن دوران \(y=\sqrt{r^2 - x^2}\) من \(0\) إلى \(r\) حول محور \(x\)
  يعطي نصف مساحة سطح الكرة.
\item
  اشتق صيغة مساحة سطح المخروط عن طريق تدوير الخط.
\end{enumerate}

\subsection{6.3 العمل
والمتوسطات}\label{ux627ux644ux639ux645ux644-ux648ux627ux644ux645ux62aux648ux633ux637ux627ux62a}

التكامل لا يقتصر على الهندسة. كما أنه يساعد في حساب الشغل الذي تبذله
القوة ومتوسط \hspace{0pt}\hspace{0pt}قيمة الدالة خلال فترة زمنية.

\subsubsection{العمل}\label{ux627ux644ux639ux645ux644}

إذا قامت قوة متغيرة \(F(x)\) بتحريك جسم على طول خط مستقيم من \(x=a\) إلى
\(x=b\)، فإن الشغل الإجمالي هو

\[
W = \int_a^b F(x)\,dx.
\]

تعمل هذه الصيغة على تعميم الحالة البسيطة \(W = F \cdot d\) للقوة
الثابتة.

المثال 1: قوة الزنبرك (قانون هوك) لزنبرك ممتد من الطول \(a\) إلى \(b\)،
بقوة \(F(x) = kx\):

\[
W = \int_a^b kx\,dx = \tfrac{1}{2}k(b^2-a^2).
\]

مثال 2: ضخ المياه إذا تم ضخ الماء من الخزان، فإن الشغل المطلوب يساوي

\[
W = \int_a^b \text{(weight density)} \times \text{(cross-sectional area)} \times \text{(distance lifted)} \, dx.
\]

\subsubsection{متوسط قيمة
الدالة}\label{ux645ux62aux648ux633ux637-ux642ux64aux645ux629-ux627ux644ux62fux627ux644ux629}

القيمة المتوسطة للدالة المستمرة \(f(x)\) على \([a,b]\) هي

\[
f_{\text{avg}} = \frac{1}{b-a} \int_a^b f(x)\,dx.
\]

هذا هو التناظرية المستمرة لحساب متوسط قائمة الأرقام.

مثال 1: بالنسبة إلى \(f(x)=x^2\) على \([0,2]\):

\[
f_{\text{avg}} = \tfrac{1}{2-0}\int_0^2 x^2 dx = \tfrac{1}{2}\cdot \tfrac{8}{3} = \tfrac{4}{3}.
\]

مثال 2: إذا كانت سرعة الجسيم هي \(v(t)\)، فإن متوسط السرعة على \([a,b]\)
هو

\[
v_{\text{avg}} = \frac{1}{b-a}\int_a^b v(t)\,dt.
\]

\subsubsection{لماذا هذا
مهم}\label{ux644ux645ux627ux630ux627-ux647ux630ux627-ux645ux647ux645-6}

\begin{itemize}
\tightlist
\item
  تظهر تكاملات العمل في حسابات الفيزياء والهندسة والطاقة.- القيمة
  المتوسطة تعطي رقم تمثيلي واحد لكميات متفاوتة.
\item
  كلاهما يربط حساب التفاضل والتكامل بمشاكل العالم الحقيقي المتعلقة
  بالحركة والقوة والكفاءة.
\end{itemize}

\subsubsection{تمارين}\label{ux62aux645ux627ux631ux64aux646-22}

\begin{enumerate}
\def\labelenumi{\arabic{enumi}.}
\tightlist
\item
  احسب الشغل المطلوب لمد زنبرك من 2 م إلى 5 م إذا كان \(k=10\).
\item
  تم رفع جسم كتلته 100 كجم عموديًا مسافة 5 أمتار في مجال الجاذبية
  (\(g=9.8 \,\text{m/s}^2\)). التعبير عن العمل باعتباره جزءا لا يتجزأ
  وتقييمه.
\item
  ابحث عن متوسط \hspace{0pt}\hspace{0pt}قيمة \(f(x)=\sin x\) على
  \([0,\pi]\).
\item
  احسب متوسط \hspace{0pt}\hspace{0pt}درجة الحرارة إذا كان
  \(T(t)=20+5\cos(\tfrac{\pi t}{12})\) على مدار 24 ساعة في اليوم.
\item
  خزان عمقه 10 m مملوء بالماء . احسب الشغل المطلوب لضخ كل الماء إلى
  الأعلى، بمعلومية وزن الماء \(9800 \,\text{N/m}^3\).
\end{enumerate}

\#\#6.4 الكثافات الاحتمالية والتوزيعات المستمرة

يلعب التكامل أيضًا دورًا مركزيًا في نظرية الاحتمالات، خاصة بالنسبة
للمتغيرات العشوائية المستمرة. بدلاً من النتائج المنفصلة،
\hspace{0pt}\hspace{0pt}نقوم بوصف الاحتمالات باستخدام وظائف تسمى وظائف
كثافة الاحتمال (pdfs).

\subsubsection{دوال الكثافة
الاحتمالية}\label{ux62fux648ux627ux644-ux627ux644ux643ux62bux627ux641ux629-ux627ux644ux627ux62dux62aux645ux627ux644ux64aux629}

يجب أن تستوفي دالة الكثافة الاحتمالية \(f(x)\) شرطين:

\begin{enumerate}
\def\labelenumi{\arabic{enumi}.}
\item
  \(f(x) \geq 0\) لجميع \(x\).
\item
  المساحة الإجمالية تحت المنحنى هي 1:

  \[
  \int_{-\infty}^\infty f(x)\,dx = 1.
  \]
\end{enumerate}

إذا كان \(X\) متغيرًا عشوائيًا مستمرًا بتنسيق pdf \(f(x)\)، فإن احتمال أن
يقع \(X\) بين \(a\) و\(b\) هو

\[
P(a \leq X \leq b) = \int_a^b f(x)\,dx.
\]

\subsubsection{دالة التوزيع
التراكمي}\label{ux62fux627ux644ux629-ux627ux644ux62aux648ux632ux64aux639-ux627ux644ux62aux631ux627ux643ux645ux64a}

يتم تعريف دالة التوزيع التراكمي (cdf) على أنها

\[
F(x) = \int_{-\infty}^x f(t)\,dt.
\]

إنه يعطي احتمال أن يكون المتغير العشوائي أقل من أو يساوي \(x\).

\subsubsection{القيمة المتوقعة
(المتوسط)}\label{ux627ux644ux642ux64aux645ux629-ux627ux644ux645ux62aux648ux642ux639ux629-ux627ux644ux645ux62aux648ux633ux637}

القيمة المتوقعة للمتغير العشوائي المستمر هي المتوسط المرجح:

\[
E[X] = \int_{-\infty}^\infty x f(x)\,dx.
\]

\subsubsection{أمثلة}\label{ux623ux645ux62bux644ux629-14}

\begin{enumerate}
\def\labelenumi{\arabic{enumi}.}
\tightlist
\item
  التوزيع الموحدبالنسبة إلى \(f(x) = \tfrac{1}{b-a}\) على \([a,b]\):
\end{enumerate}

\begin{itemize}
\item
  احتمال الفاصل الزمني \([c,d]\):

  \[
  P(c \leq X \leq d) = \frac{d-c}{b-a}.
  \]
\item
  القيمة المتوقعة: \(E[X] = \tfrac{a+b}{2}\).
\end{itemize}

\begin{enumerate}
\def\labelenumi{\arabic{enumi}.}
\setcounter{enumi}{1}
\tightlist
\item
  التوزيع الأسي بالنسبة إلى \(f(x) = \lambda e^{-\lambda x}\)،
  \(x \geq 0\):
\end{enumerate}

\begin{itemize}
\tightlist
\item
  \(\int_0^\infty \lambda e^{-\lambda x}\,dx = 1\).
\item
  يعني: \(E[X] = \tfrac{1}{\lambda}\).
\end{itemize}

\begin{enumerate}
\def\labelenumi{\arabic{enumi}.}
\setcounter{enumi}{2}
\tightlist
\item
  التوزيع الطبيعي منحنى الجرس:
\end{enumerate}

\[
f(x) = \frac{1}{\sqrt{2\pi\sigma^2}} e^{-\frac{(x-\mu)^2}{2\sigma^2}}.
\]

إنه يتكامل مع 1، ولكنه يتطلب تقنيات متقدمة.

\subsubsection{لماذا هذا
مهم}\label{ux644ux645ux627ux630ux627-ux647ux630ux627-ux645ux647ux645-7}

\begin{itemize}
\tightlist
\item
  تصف كثافات الاحتمال عدم اليقين في العلوم والهندسة والإحصاء.
\item
  التكاملات تربط المناطق الواقعة تحت المنحنيات بالاحتمالات.
\item
  التوزيعات المستمرة تعمم فكرة عد النتائج لقياس الاحتمالات على فترات.
\end{itemize}

\subsubsection{تمارين}\label{ux62aux645ux627ux631ux64aux646-23}

\begin{enumerate}
\def\labelenumi{\arabic{enumi}.}
\tightlist
\item
  أظهر أن الكثافة الموحدة \(f(x) = \tfrac{1}{b-a}\) على \([a,b]\) تتكامل
  مع 1.
\item
  بالنسبة للتوزيع الأسي باستخدام \(\lambda = 2\)، قم بحساب
  \(P(0 \leq X \leq 1)\).
\item
  ابحث عن القيمة المتوقعة لـ \(X\) إذا كان \(f(x) = 3x^2\) على
  \([0,1]\).
\item
  تحقق من أن التوزيع الطبيعي بمتوسط \hspace{0pt}\hspace{0pt}0 وتباين 1
  له احتمال إجمالي 1 (لا حاجة لإثبات كامل، ولكن اشرح سبب صحة ذلك).
\item
  قم بحساب cdf للتوزيع الموحد على \([0,1]\).
\end{enumerate}

\#الجزء الثالث. حساب التفاضل والتكامل متعدد المتغيرات

\section{الفصل السابع. وظائف المتجهات
والمنحنيات}\label{ux627ux644ux641ux635ux644-ux627ux644ux633ux627ux628ux639.-ux648ux638ux627ux626ux641-ux627ux644ux645ux62aux62cux647ux627ux62a-ux648ux627ux644ux645ux646ux62dux646ux64aux627ux62a}

\subsection{7.1 وظائف المتجهات ومنحنيات
الفضاء}\label{ux648ux638ux627ux626ux641-ux627ux644ux645ux62aux62cux647ux627ux62a-ux648ux645ux646ux62dux646ux64aux627ux62a-ux627ux644ux641ux636ux627ux621}

في حساب التفاضل والتكامل متعدد المتغيرات، يمكن للوظائف إخراج المتجهات
بدلاً من الأرقام. وتسمى هذه الدوال ذات القيمة المتجهة، وهي ضرورية لوصف
المنحنيات في الفضاء.

\subsubsection{التعريف}\label{ux627ux644ux62aux639ux631ux64aux641-7}

وظيفة المتجهات هي وظيفة النموذج

\[
\mathbf{r}(t) = \langle x(t), y(t), z(t) \rangle,
\]

حيث \(x(t), y(t), z(t)\) هي وظائف ذات قيمة حقيقية.

\begin{itemize}
\tightlist
\item
  غالبًا ما يُطلق على الإدخال \(t\) اسم المعلمة.- الإخراج عبارة عن متجه في
  مساحة ثنائية أو ثلاثية الأبعاد.
\item
  الرسم البياني للدالة المتجهة ثلاثي الأبعاد هو منحنى فضائي.
\end{itemize}

\subsubsection{أمثلة}\label{ux623ux645ux62bux644ux629-15}

\begin{enumerate}
\def\labelenumi{\arabic{enumi}.}
\tightlist
\item
  الخط
\end{enumerate}

\[
\mathbf{r}(t) = \langle 1+2t, \; 3-t, \; 4+5t \rangle.
\]

يصف هذا خطًا مستقيمًا عبر النقطة \((1,3,4)\) مع متجه الاتجاه
\(\langle 2,-1,5 \rangle\).

\begin{enumerate}
\def\labelenumi{\arabic{enumi}.}
\setcounter{enumi}{1}
\tightlist
\item
  دائرة في الطائرة
\end{enumerate}

\[
\mathbf{r}(t) = \langle \cos t, \; \sin t, \; 0 \rangle, \quad 0 \leq t < 2\pi.
\]

\begin{enumerate}
\def\labelenumi{\arabic{enumi}.}
\setcounter{enumi}{2}
\tightlist
\item
  اللولب
\end{enumerate}

\[
\mathbf{r}(t) = \langle \cos t, \; \sin t, \; t \rangle.
\]

هذا شكل حلزوني يرتفع حول محور \(z\).

\subsubsection{الحدود
والاستمرارية}\label{ux627ux644ux62dux62fux648ux62f-ux648ux627ux644ux627ux633ux62aux645ux631ux627ux631ux64aux629}

تكون الدالة المتجهة مستمرة عند \(t=a\) إذا كان كل مكون
\(x(t), y(t), z(t)\) مستمرًا عند \(t=a\).

\[
\lim_{t \to a} \mathbf{r}(t) = \langle \lim_{t \to a} x(t), \; \lim_{t \to a} y(t), \; \lim_{t \to a} z(t) \rangle.
\]

\subsubsection{هندسة منحنيات
الفضاء}\label{ux647ux646ux62fux633ux629-ux645ux646ux62dux646ux64aux627ux62a-ux627ux644ux641ux636ux627ux621}

\begin{itemize}
\tightlist
\item
  كل منحنى له اتجاه مماس يعطى بواسطة المشتق.
\item
  يمكن لمنحنيات الفضاء أن تصمم مسارات الحركة، ومسارات الجسيمات، والأشكال
  الهندسية.
\end{itemize}

\subsubsection{لماذا هذا
مهم}\label{ux644ux645ux627ux630ux627-ux647ux630ux627-ux645ux647ux645-8}

الدوال المتجهة هي الأساس لحساب التفاضل والتكامل متعدد المتغيرات، مما
يسمح لنا بتوسيع أفكار المشتقات والتكاملات إلى أبعاد أعلى. كما أنها تظهر
بشكل طبيعي في الفيزياء (الحركة ثلاثية الأبعاد، والكهرومغناطيسية،
وديناميكيات الموائع).

\subsubsection{تمارين}\label{ux62aux645ux627ux631ux64aux646-24}

\begin{enumerate}
\def\labelenumi{\arabic{enumi}.}
\tightlist
\item
  اكتب دالة متجهة لخط يمر عبر \((0,1,2)\) موازيًا للمتجه
  \(\langle 3,-2,1 \rangle\).
\item
  قم بوصف المنحنى المعطى بواسطة
  \(\mathbf{r}(t) = \langle 2\cos t, \; 2\sin t, \; 3 \rangle\).
\item
  حدد ما إذا كان
  \(\mathbf{r}(t) = \langle e^t, \; \ln t, \; t^2 \rangle\) مستمرًا عند
  \(t=1\).
\item
  ارسم الحلزون
  \(\mathbf{r}(t) = \langle \cos t, \; \sin t, \; 2t \rangle\).
\item
  أوجد النقطة على المنحنى
  \(\mathbf{r}(t) = \langle t, \; t^2, \; t^3 \rangle\) عند \(t=2\).
\end{enumerate}

\subsection{7.2 مشتقات وتكاملات الدوال المتجهةيمكن تمييز الوظائف المتجهة
وتكاملها تمامًا مثل الوظائف العادية - فنحن ببساطة نطبق العملية على كل
مكون. وهذا يسمح لنا بدراسة الحركة والسرعة والتسارع والتراكم في أبعاد
أعلى.}\label{ux645ux634ux62aux642ux627ux62a-ux648ux62aux643ux627ux645ux644ux627ux62a-ux627ux644ux62fux648ux627ux644-ux627ux644ux645ux62aux62cux647ux629ux64aux645ux643ux646-ux62aux645ux64aux64aux632-ux627ux644ux648ux638ux627ux626ux641-ux627ux644ux645ux62aux62cux647ux629-ux648ux62aux643ux627ux645ux644ux647ux627-ux62aux645ux627ux645ux627-ux645ux62bux644-ux627ux644ux648ux638ux627ux626ux641-ux627ux644ux639ux627ux62fux64aux629---ux641ux646ux62dux646-ux628ux628ux633ux627ux637ux629-ux646ux637ux628ux642-ux627ux644ux639ux645ux644ux64aux629-ux639ux644ux649-ux643ux644-ux645ux643ux648ux646.-ux648ux647ux630ux627-ux64aux633ux645ux62d-ux644ux646ux627-ux628ux62fux631ux627ux633ux629-ux627ux644ux62dux631ux643ux629-ux648ux627ux644ux633ux631ux639ux629-ux648ux627ux644ux62aux633ux627ux631ux639-ux648ux627ux644ux62aux631ux627ux643ux645-ux641ux64a-ux623ux628ux639ux627ux62f-ux623ux639ux644ux649.}

\subsubsection{مشتق من دالة
المتجهات}\label{ux645ux634ux62aux642-ux645ux646-ux62fux627ux644ux629-ux627ux644ux645ux62aux62cux647ux627ux62a}

إذا

\[
\mathbf{r}(t) = \langle x(t), y(t), z(t) \rangle,
\]

ثم

\[
\mathbf{r}'(t) = \langle x'(t), y'(t), z'(t) \rangle.
\]

يشير هذا المتجه المشتق في اتجاه الظل للمنحنى عند المعلمة \(t\).

\begin{itemize}
\tightlist
\item
  السرعة: إذا كان \(\mathbf{r}(t)\) يعطي موضع الجسيم في الزمن \(t\)، فإن
  \(\mathbf{v}(t) = \mathbf{r}'(t)\) هو متجه سرعته.
\item
  السرعة: الحجم \(|\mathbf{v}(t)|\) هو سرعة الجسيم.
\item
  التسارع: \(\mathbf{a}(t) = \mathbf{v}'(t) = \mathbf{r}''(t)\).
\end{itemize}

\subsubsection{أمثلة}\label{ux623ux645ux62bux644ux629-16}

\begin{enumerate}
\def\labelenumi{\arabic{enumi}.}
\tightlist
\item
  اللولب
\end{enumerate}

\[
\mathbf{r}(t) = \langle \cos t, \sin t, t \rangle.
\]

\begin{itemize}
\tightlist
\item
  السرعة: \(\mathbf{v}(t) = \langle -\sin t, \cos t, 1 \rangle\).
\item
  السرعة:
  \(|\mathbf{v}(t)| = \sqrt{(-\sin t)^2 + (\cos t)^2 + 1^2} = \sqrt{2}\).
\item
  التسارع: \(\mathbf{a}(t) = \langle -\cos t, -\sin t, 0 \rangle\).
\end{itemize}

\begin{enumerate}
\def\labelenumi{\arabic{enumi}.}
\setcounter{enumi}{1}
\tightlist
\item
  حركة المقذوفات
\end{enumerate}

\[
\mathbf{r}(t) = \langle v_0 \cos\theta \cdot t, \; v_0 \sin\theta \cdot t - \tfrac{1}{2}gt^2 \rangle.
\]

وهذا يمثل المسار المكافئ للقذيفة تحت الجاذبية.

\subsubsection{تكامل دالة
المتجهات}\label{ux62aux643ux627ux645ux644-ux62fux627ux644ux629-ux627ux644ux645ux62aux62cux647ux627ux62a}

إذا

\[
\mathbf{r}(t) = \langle x(t), y(t), z(t) \rangle,
\]

ثم

\[
\int \mathbf{r}(t)\,dt = \left\langle \int x(t)\,dt, \; \int y(t)\,dt, \; \int z(t)\,dt \right\rangle + \mathbf{C},
\]

حيث \(\mathbf{C}\) هو متجه ثابت.

\subsubsection{مثال}\label{ux645ux62bux627ux644}

\[
\mathbf{r}(t) = \langle t, t^2, t^3 \rangle.
\]

\begin{itemize}
\tightlist
\item
  المشتق: \(\mathbf{r}'(t) = \langle 1, 2t, 3t^2 \rangle\).
\item
  لا يتجزأ:
\end{itemize}

\[
\int \mathbf{r}(t)\,dt = \langle \tfrac{1}{2}t^2, \tfrac{1}{3}t^3, \tfrac{1}{4}t^4 \rangle + \mathbf{C}.
\]

\subsubsection{لماذا هذا مهم- مشتقات الدوال المتجهة تصف الحركة والقوى في
الفضاء.}\label{ux644ux645ux627ux630ux627-ux647ux630ux627-ux645ux647ux645--ux645ux634ux62aux642ux627ux62a-ux627ux644ux62fux648ux627ux644-ux627ux644ux645ux62aux62cux647ux629-ux62aux635ux641-ux627ux644ux62dux631ux643ux629-ux648ux627ux644ux642ux648ux649-ux641ux64a-ux627ux644ux641ux636ux627ux621.}

\begin{itemize}
\tightlist
\item
  التكاملات تعطي الإزاحة والشغل والكميات المتراكمة.
\item
  تربط هذه الأدوات حساب التفاضل والتكامل مباشرة بالفيزياء والهندسة.
\end{itemize}

\subsubsection{تمارين}\label{ux62aux645ux627ux631ux64aux646-25}

\begin{enumerate}
\def\labelenumi{\arabic{enumi}.}
\tightlist
\item
  بالنسبة إلى \(\mathbf{r}(t) = \langle t, \cos t, \sin t \rangle\)،
  أوجد السرعة المتجهة والسرعة والتسارع.
\item
  قم بحساب \(\mathbf{r}'(t)\) لـ
  \(\mathbf{r}(t) = \langle e^t, \ln t, t^2 \rangle\).
\item
  دمج \(\mathbf{r}(t) = \langle 1, t, t^2 \rangle\).
\item
  سرعة الجسيم هي \(\mathbf{v}(t) = \langle t, 2, 0 \rangle\). ابحث عن
  متجه موضعه إذا كان \(\mathbf{r}(0) = \langle 1, 0, 0 \rangle\).
\item
  أظهر أن سرعة \(\mathbf{r}(t) = \langle \cos t, \sin t, 0 \rangle\)
  ثابتة.
\end{enumerate}

\subsection{7.3 طول القوس
والانحناء}\label{ux637ux648ux644-ux627ux644ux642ux648ux633-ux648ux627ux644ux627ux646ux62dux646ux627ux621}

يوفر حساب التفاضل والتكامل المتجه أدوات ليس فقط لقياس المسار الذي يرسمه
المنحنى ولكن أيضًا مدى حدة انحناءه. يتم التعبير عنها من خلال طول القوس
والانحناء.

\subsubsection{طول قوس منحنى
الفضاء}\label{ux637ux648ux644-ux642ux648ux633-ux645ux646ux62dux646ux649-ux627ux644ux641ux636ux627ux621}

إذا تم إعطاء منحنى بواسطة

\[
\mathbf{r}(t) = \langle x(t), y(t), z(t) \rangle, \quad a \leq t \leq b,
\]

ثم طول القوس هو

\[
L = \int_a^b |\mathbf{r}'(t)|\,dt,
\]

أين

\[
|\mathbf{r}'(t)| = \sqrt{(x'(t))^2 + (y'(t))^2 + (z'(t))^2}.
\]

مثال: بالنسبة للحلزون
\(\mathbf{r}(t) = \langle \cos t, \sin t, t \rangle, \, 0 \leq t \leq 2\pi\):

\begin{itemize}
\tightlist
\item
  السرعة: \(\mathbf{r}'(t) = \langle -\sin t, \cos t, 1 \rangle\).
\item
  السرعة:
  \(|\mathbf{r}'(t)| = \sqrt{(-\sin t)^2 + (\cos t)^2 + 1^2} = \sqrt{2}\).
\item
  طول القوس:
\end{itemize}

\[
L = \int_0^{2\pi} \sqrt{2}\,dt = 2\pi\sqrt{2}.
\]

\subsubsection{انحناء}\label{ux627ux646ux62dux646ux627ux621}

يقيس الانحناء مدى سرعة تغيير المنحنى لاتجاهه.

للحصول على منحنى سلس \(\mathbf{r}(t)\):

\[
\kappa(t) = \frac{|\mathbf{r}'(t) \times \mathbf{r}''(t)|}{|\mathbf{r}'(t)|^3}.
\]

\begin{itemize}
\tightlist
\item
  \(\kappa = 0\): خط مستقيم.
\item
  \(\kappa\) أكبر: ينحني المنحنى بشكل أكثر حدة.
\end{itemize}

مثال: للحصول على دائرة نصف قطرها \(r\):\[
\mathbf{r}(t) = \langle r\cos t, r\sin t \rangle.
\]

ثم \(\kappa = \tfrac{1}{r}\). إذن الانحناء ثابت ويتناسب عكسيًا مع نصف
القطر.

\subsubsection{وحدة الظل والمتجهات
العادية}\label{ux648ux62dux62fux629-ux627ux644ux638ux644-ux648ux627ux644ux645ux62aux62cux647ux627ux62a-ux627ux644ux639ux627ux62fux64aux629}

\begin{itemize}
\tightlist
\item
  ناقل الظل:
\end{itemize}

\[
\mathbf{T}(t) = \frac{\mathbf{r}'(t)}{|\mathbf{r}'(t)|}.
\]

\begin{itemize}
\tightlist
\item
  المتجه العادي: يشير إلى مركز الانحناء، ويعرف بـ
\end{itemize}

\[
\mathbf{N}(t) = \frac{\mathbf{T}'(t)}{|\mathbf{T}'(t)|}.
\]

تصف هذه المتجهات هندسة الحركة: اتجاه السفر واتجاه الدوران.

\subsubsection{لماذا هذا
مهم}\label{ux644ux645ux627ux630ux627-ux647ux630ux627-ux645ux647ux645-9}

\begin{itemize}
\tightlist
\item
  طول القوس يعمم مفهوم المسافة على المنحنيات في الفضاء.
\item
  يصف الانحناء الانحناء، وهو أمر بالغ الأهمية في الفيزياء (تسارع
  الجاذبية)، والهندسة (الطرق، والأفعوانيات)، ورسومات الكمبيوتر.
\end{itemize}

\subsubsection{تمارين}\label{ux62aux645ux627ux631ux64aux646-26}

\begin{enumerate}
\def\labelenumi{\arabic{enumi}.}
\tightlist
\item
  أوجد طول قوس \(\mathbf{r}(t) = \langle t, t^2, 0 \rangle\) من \(t=0\)
  إلى \(t=1\).
\item
  احسب انحناء الدائرة
  \(\mathbf{r}(t) = \langle \cos t, \sin t \rangle\).
\item
  بالنسبة إلى \(\mathbf{r}(t) = \langle t, \cos t, \sin t \rangle\)،
  احسب \(|\mathbf{r}'(t)|\).
\item
  أظهر أن الخط المستقيم له انحناء \(\kappa = 0\).
\item
  ابحث عن المتجه المماس لـ
  \(\mathbf{r}(t) = \langle e^t, e^{-t}, t \rangle\) في \(t=0\).
\end{enumerate}

\subsection{7.4 الحركة في
الفضاء}\label{ux627ux644ux62dux631ux643ux629-ux641ux64a-ux627ux644ux641ux636ux627ux621}

تعتبر الدوال المتجهة فعالة بشكل خاص في وصف الحركة في بعدين أو ثلاثة
أبعاد. يتم التعبير عن الموضع والسرعة والتسارع بشكل طبيعي باستخدام مشتقات
وتكاملات الدوال ذات القيمة المتجهة.

\subsubsection{الموضع والسرعة
والتسارع}\label{ux627ux644ux645ux648ux636ux639-ux648ux627ux644ux633ux631ux639ux629-ux648ux627ux644ux62aux633ux627ux631ux639}

\begin{itemize}
\tightlist
\item
  ناقل الموقف:
\end{itemize}

\[
\mathbf{r}(t) = \langle x(t), y(t), z(t) \rangle
\]

\begin{itemize}
\tightlist
\item
  ناقل السرعة (مشتق من الموضع):
\end{itemize}

\[
\mathbf{v}(t) = \mathbf{r}'(t) = \langle x'(t), y'(t), z'(t) \rangle
\]

\begin{itemize}
\tightlist
\item
  السرعة (حجم السرعة):
\end{itemize}

\[
|\mathbf{v}(t)| = \sqrt{(x'(t))^2 + (y'(t))^2 + (z'(t))^2}
\]

\begin{itemize}
\tightlist
\item
  متجه التسارع (مشتق السرعة):
\end{itemize}

\[\mathbf{a}(t) = \mathbf{v}'(t) = \mathbf{r}''(t).
\]

\subsubsection{Tangential and Normal
Components}\label{tangential-and-normal-components}

Acceleration can be decomposed into two components:

\[
\mathbf{a}(t) = a_T \mathbf{T}(t) + a_N \mathbf{N}(t),
\]

where:

\begin{itemize}
\tightlist
\item
  \(\mathbf{T}(t)\) = unit tangent vector,
\item
  \(\mathbf{N}(t)\) = principal normal vector,
\item
  \(a_T = \frac{d}{dt}|\mathbf{v}(t)|\) = tangential acceleration
  (change in speed),
\item
  \(a_N = \kappa |\mathbf{v}(t)|^2\) = normal acceleration (change in
  direction).
\end{itemize}

\subsubsection{Projectile Motion in 3D}\label{projectile-motion-in-3d}

With gravity acting in the \(-z\) direction:

\[
\mathbf{r}(t) = \langle v_0 \cos\theta \cos\phi \cdot t,\; v_0 \cos\theta \sin\phi \cdot t,\; v_0 \sin\theta \cdot t - \tfrac{1}{2}gt^2 \rangle,
\]

where \(v_0\) is initial speed, \(\theta\) launch angle, and \(\phi\)
azimuthal direction.

\subsubsection{Example: Helical Motion}\label{example-helical-motion}

\[
\mathbf{r}(t) = \langle \cos t, \sin t, t \rangle
\]

\begin{itemize}
\tightlist
\item
  السرعة: \(\mathbf{v}(t) = \langle -\sin t, \cos t, 1 \rangle\).
\item
  السرعة: \(|\mathbf{v}(t)| = \sqrt{2}\).
\item
  التسارع: \(\mathbf{a}(t) = \langle -\cos t, -\sin t, 0 \rangle\).
\item
  الحركة منتظمة في سرعتها، وتتصاعد إلى الأعلى.
\end{itemize}

\subsubsection{لماذا هذا
مهم}\label{ux644ux645ux627ux630ux627-ux647ux630ux627-ux645ux647ux645-10}

\begin{itemize}
\tightlist
\item
  يوفر لغة رياضية للحركة في العالم الحقيقي.
\item
  أساسي في الفيزياء (القوى، المسارات، الحركة الدائرية).
\item
  مؤسسة الميكانيكا المتقدمة والنماذج الهندسية.
\end{itemize}

\subsubsection{تمارين}\label{ux62aux645ux627ux631ux64aux646-27}

\begin{enumerate}
\def\labelenumi{\arabic{enumi}.}
\tightlist
\item
  يتحرك الجسيم على طول \(\mathbf{r}(t) = \langle t, t^2, t^3 \rangle\).
  أوجد السرعة والتسارع عند \(t=1\).
\item
  أظهر أن السرعة ثابتة للحلزون
  \(\mathbf{r}(t) = \langle \cos t, \sin t, t \rangle\).
\item
  تم إطلاق مقذوف يحمل \(v_0 = 20 \,\text{m/s}\) بزاوية \(45^\circ\).
  اكتب متجه موضعه مفترضًا الحركة في مستوى رأسي.
\item
  بالنسبة إلى \(\mathbf{r}(t) = \langle e^t, e^{-t}, t \rangle\)، ابحث
  عن \(\mathbf{v}(t)\) و\(\mathbf{a}(t)\).5. قم بتحليل متجه التسارع إلى
  مكونات عرضية وطبيعية للحركة على طول دائرة نصف قطرها \(r\).
\end{enumerate}

\section{الفصل 8. وظائف العديد من
المتغيرات}\label{ux627ux644ux641ux635ux644-8.-ux648ux638ux627ux626ux641-ux627ux644ux639ux62fux64aux62f-ux645ux646-ux627ux644ux645ux62aux63aux64aux631ux627ux62a}

\subsection{8.1 الحدود والاستمرارية في عدة
متغيرات}\label{ux627ux644ux62dux62fux648ux62f-ux648ux627ux644ux627ux633ux62aux645ux631ux627ux631ux64aux629-ux641ux64a-ux639ux62fux629-ux645ux62aux63aux64aux631ux627ux62a}

في حساب التفاضل والتكامل متعدد المتغيرات، قد تعتمد الوظائف على متغيرين
أو أكثر، مثل \(f(x,y)\) أو \(f(x,y,z)\). يمتد مفهوما الحدود والاستمرارية
بشكل طبيعي من حساب التفاضل والتكامل ذو المتغير الواحد، لكنها أكثر دقة
لأنه يجب علينا أن نأخذ في الاعتبار جميع مسارات النهج الممكنة.

\subsubsection{الحدود في
متغيرين}\label{ux627ux644ux62dux62fux648ux62f-ux641ux64a-ux645ux62aux63aux64aux631ux64aux646}

بالنسبة للدالة \(f(x,y)\)، نقول

\[
\lim_{(x,y) \to (a,b)} f(x,y) = L
\]

إذا اقترب \(f(x,y)\) عشوائيًا من \(L\) بينما يقترب \((x,y)\) من \((a,b)\)
على طول أي مسار.

إذا كانت المسارات المختلفة تعطي قيمًا نهائية مختلفة، فهذا يعني أن الحد
غير موجود.

مثال 1 (الحد موجود):

\[
f(x,y) = x^2 + y^2, \quad \lim_{(x,y) \to (0,0)} f(x,y) = 0.
\]

المثال 2 (الحد غير موجود):

\[
f(x,y) = \frac{xy}{x^2+y^2}, \quad (x,y) \to (0,0).
\]

\begin{itemize}
\tightlist
\item
  على طول \(y=0\)، الدالة هي 0.
\item
  على طول \(y=x\)، الوظيفة هي \(\tfrac{1}{2}\). نتائج مختلفة → الحد غير
  موجود.
\end{itemize}

\#\#\#الاستمرارية

تكون الدالة \(f(x,y)\) متصلة عند \((a,b)\) إذا

\[
\lim_{(x,y)\to(a,b)} f(x,y) = f(a,b).
\]

كثيرات الحدود والوظائف العقلانية (حيث المقام ≠ 0) مستمرة في كل مكان في
مجالاتها.

\subsubsection{تمديد لثلاثة متغيرات أو
أكثر}\label{ux62aux645ux62fux64aux62f-ux644ux62bux644ux627ux62bux629-ux645ux62aux63aux64aux631ux627ux62a-ux623ux648-ux623ux643ux62bux631}

بالنسبة إلى \(f(x,y,z)\)، يتم تعريف الحدود والاستمرارية بنفس الطريقة،
ولكن يجب الاقتراب من النقطة \((a,b,c)\) من اتجاهات عديدة لا نهائية في
الفضاء.

\subsubsection{لماذا هذا
مهم}\label{ux644ux645ux627ux630ux627-ux647ux630ux627-ux645ux647ux645-11}

\begin{itemize}
\tightlist
\item
  تضمن الاستمرارية عدم وجود قفزات أو ثقوب أو خطوط مقاربة في الوظائف
  متعددة المتغيرات.
\item
  النهايات أساسية لتعريف المشتقات الجزئية والتكاملات المتعددة.
\item
  هذه المفاهيم هي اللبنات الأساسية لحساب التفاضل والتكامل متعدد
  المتغيرات.
\end{itemize}

\subsubsection{\texorpdfstring{تمارين1. حدد ما إذا كان
\(\lim_{(x,y)\to(0,0)} (x^2+y^2)\) موجودًا أم
لا.}{تمارين1. حدد ما إذا كان \textbackslash lim\_\{(x,y)\textbackslash to(0,0)\} (x\^{}2+y\^{}2) موجودًا أم لا.}}\label{ux62aux645ux627ux631ux64aux6461.-ux62dux62fux62f-ux645ux627-ux625ux630ux627-ux643ux627ux646-lim_xyto00-x2y2-ux645ux648ux62cux648ux62fux627-ux623ux645-ux644ux627.}

\begin{enumerate}
\def\labelenumi{\arabic{enumi}.}
\setcounter{enumi}{1}
\tightlist
\item
  أظهر أن \(\lim_{(x,y)\to(0,0)} \frac{x^2y}{x^2+y^2} = 0\) على طول جميع
  المسارات المستقيمة \(y=mx\).
\item
  هل الحد الأقصى لـ \(f(x,y) = \frac{x^2-y^2}{x^2+y^2}\) هو
  \((x,y)\to(0,0)\)؟
\item
  اشرح لماذا تكون كثيرات الحدود في متغيرين متصلة في كل مكان.
\item
  أعط مثالاً لدالة مكونة من متغيرين غير متصلة عند نقطة ما، واشرح السبب.
\end{enumerate}

\subsection{8.2 المشتقات
الجزئية}\label{ux627ux644ux645ux634ux62aux642ux627ux62a-ux627ux644ux62cux632ux626ux64aux629}

في الدوال ذات المتغيرات المتعددة، غالبًا ما نرغب في قياس كيفية تغير
الدالة عندما يتغير متغير واحد فقط بينما تبقى المتغيرات الأخرى ثابتة.
وهذا يؤدي إلى فكرة المشتقات الجزئية.

\subsubsection{التعريف}\label{ux627ux644ux62aux639ux631ux64aux641-8}

بالنسبة للدالة \(f(x,y)\)، المشتق الجزئي بالنسبة إلى \(x\) عند النقطة
\((a,b)\) هو

\[
\frac{\partial f}{\partial x}(a,b) = \lim_{h \to 0} \frac{f(a+h, b) - f(a,b)}{h}.
\]

وبالمثل، فإن المشتق الجزئي فيما يتعلق بـ \(y\) هو

\[
\frac{\partial f}{\partial y}(a,b) = \lim_{h \to 0} \frac{f(a, b+h) - f(a,b)}{h}.
\]

نحن نتعامل مع جميع المتغيرات الأخرى كثوابت عند التفاضل.

\subsubsection{التدوين}\label{ux627ux644ux62aux62fux648ux64aux646-1}

\begin{itemize}
\tightlist
\item
  \(\frac{\partial f}{\partial x}\)، \(f_x\)، \(\partial_x f\).
\item
  \(\frac{\partial f}{\partial y}\)، \(f_y\)، \(\partial_y f\).
\end{itemize}

بالنسبة للمتغيرات الثلاثة \(f(x,y,z)\)، لدينا أيضًا \(f_x, f_y, f_z\).

\subsubsection{أمثلة}\label{ux623ux645ux62bux644ux629-17}

\begin{enumerate}
\def\labelenumi{\arabic{enumi}.}
\tightlist
\item
  \(f(x,y) = x^2y + y^3\)
\end{enumerate}

\begin{itemize}
\tightlist
\item
  \(f_x = 2xy\).
\item
  \(f_y = x^2 + 3y^2\).
\end{itemize}

\begin{enumerate}
\def\labelenumi{\arabic{enumi}.}
\setcounter{enumi}{1}
\tightlist
\item
  \(f(x,y) = e^{xy}\)
\end{enumerate}

\begin{itemize}
\tightlist
\item
  \(f_x = y e^{xy}\).
\item
  \(f_y = x e^{xy}\).
\end{itemize}

\begin{enumerate}
\def\labelenumi{\arabic{enumi}.}
\setcounter{enumi}{2}
\tightlist
\item
  \(f(x,y,z) = x^2 + yz\)
\end{enumerate}

\begin{itemize}
\tightlist
\item
  \(f_x = 2x\).
\item
  \(f_y = z\).
\item
  \(f_z = y\).
\end{itemize}

\subsubsection{المشتقات الجزئية ذات الترتيب
الأعلى}\label{ux627ux644ux645ux634ux62aux642ux627ux62a-ux627ux644ux62cux632ux626ux64aux629-ux630ux627ux62a-ux627ux644ux62aux631ux62aux64aux628-ux627ux644ux623ux639ux644ux649}

يمكننا أن نأخذ المشتقات الجزئية مرارا وتكرارا:

\begin{itemize}
\tightlist
\item
  \(f_{xx} = \frac{\partial}{\partial x}\Big(f_x\Big)\).
\item
  \(f_{yy}, f_{xy}, f_{yx}\)، وما إلى ذلك.
\end{itemize}

نظرية كليروت: إذا كان \(f\) يحتوي على مشتقات جزئية ثانية متصلة، إذن

\[
f_{xy} = f_{yx}.
\]

\subsubsection{\texorpdfstring{المعنى الهندسي- \(f_x\): ميل السطح في
اتجاه
\(x\).}{المعنى الهندسي- f\_x: ميل السطح في اتجاه x.}}\label{ux627ux644ux645ux639ux646ux649-ux627ux644ux647ux646ux62fux633ux64a--f_x-ux645ux64aux644-ux627ux644ux633ux637ux62d-ux641ux64a-ux627ux62aux62cux627ux647-x.}

\begin{itemize}
\tightlist
\item
  \(f_y\): ميل السطح في اتجاه \(y\).
\item
  يصفان معًا كيفية ميل السطح.
\end{itemize}

\subsubsection{لماذا هذا
مهم}\label{ux644ux645ux627ux630ux627-ux647ux630ux627-ux645ux647ux645-12}

\begin{itemize}
\tightlist
\item
  المشتقات الجزئية هي أساس التدرجات ومستويات الظل والتحسين في متغيرات
  متعددة.
\item
  تُستخدم على نطاق واسع في الفيزياء والهندسة والاقتصاد لنمذجة الأنظمة ذات
  المدخلات المتعددة.
\end{itemize}

\subsubsection{تمارين}\label{ux62aux645ux627ux631ux64aux646-28}

\begin{enumerate}
\def\labelenumi{\arabic{enumi}.}
\tightlist
\item
  ابحث عن \(f_x\) و\(f_y\) لـ \(f(x,y) = x^3y^2\).
\item
  قم بحساب \(f_x, f_y, f_z\) لـ \(f(x,y,z) = xyz + x^2\).
\item
  تحقق من نظرية كليروت لـ \(f(x,y) = x^2y + y^3\).
\item
  فسر هندسيًا ما يعنيه \(f_x\) و\(f_y\) لـ \(f(x,y) = \sqrt{x^2+y^2}\).
\item
  ابحث عن جميع المشتقات الجزئية من الدرجة الثانية لـ
  \(f(x,y) = e^{x^2+y^2}\).
\end{enumerate}

\subsection{8.3 مشتقات التدرج
والاتجاه}\label{ux645ux634ux62aux642ux627ux62a-ux627ux644ux62aux62fux631ux62c-ux648ux627ux644ux627ux62aux62cux627ux647}

تقيس المشتقات الجزئية التغير على طول محاور الإحداثيات، لكن في بعض
الأحيان نريد معرفة معدل تغير الدالة في أي اتجاه. وهذا يؤدي إلى مفاهيم
التدرج ومشتقات الاتجاه.

\subsubsection{ناقل
التدرج}\label{ux646ux627ux642ux644-ux627ux644ux62aux62fux631ux62c}

بالنسبة للدالة \(f(x,y)\)، فإن التدرج هو المتجه

\[
\nabla f(x,y) = \left\langle \frac{\partial f}{\partial x}, \frac{\partial f}{\partial y} \right\rangle.
\]

لثلاثة متغيرات \(f(x,y,z)\):

\[
\nabla f(x,y,z) = \left\langle f_x, f_y, f_z \right\rangle.
\]

يشير التدرج في اتجاه الزيادة القصوى للدالة، وحجمه يعطي المنحدر الأكثر
انحدارًا.

\subsubsection{المشتقات
الاتجاهية}\label{ux627ux644ux645ux634ux62aux642ux627ux62a-ux627ux644ux627ux62aux62cux627ux647ux64aux629}

معدل تغير \(f(x,y)\) عند نقطة في اتجاه متجه الوحدة
\(\mathbf{u} = \langle u_1, u_2 \rangle\) هو

\[
D_{\mathbf{u}} f(x,y) = \nabla f(x,y) \cdot \mathbf{u}.
\]

هذا هو المنتج النقطي للتدرج مع متجه الاتجاه.

\subsubsection{أمثلة}\label{ux623ux645ux62bux644ux629-18}

\begin{enumerate}
\def\labelenumi{\arabic{enumi}.}
\tightlist
\item
  \(f(x,y) = x^2 + y^2\)
\end{enumerate}

\begin{itemize}
\tightlist
\item
  التدرج: \(\nabla f = \langle 2x, 2y \rangle\).
\item
  عند (1,2): \(\nabla f = \langle 2,4 \rangle\).- مشتق الاتجاه على طول
  \(\mathbf{u} = \langle \tfrac{3}{5}, \tfrac{4}{5} \rangle\):
\end{itemize}

\[
D_{\mathbf{u}} f(1,2) = \langle 2,4 \rangle \cdot \langle \tfrac{3}{5}, \tfrac{4}{5} \rangle = \tfrac{26}{5}.
\]

\begin{enumerate}
\def\labelenumi{\arabic{enumi}.}
\setcounter{enumi}{1}
\tightlist
\item
  \(f(x,y,z) = x y z\)
\end{enumerate}

\begin{itemize}
\tightlist
\item
  التدرج: \(\nabla f = \langle yz, xz, xy \rangle\).
\item
  عند (1,1,1): \(\nabla f = \langle 1,1,1 \rangle\).
\item
  أقصى اتجاه للزيادة هو على طول \(\langle 1,1,1 \rangle\).
\end{itemize}

\subsubsection{التفسير
الهندسي}\label{ux627ux644ux62aux641ux633ux64aux631-ux627ux644ux647ux646ux62fux633ux64a-1}

\begin{itemize}
\tightlist
\item
  يكون متجه التدرج عموديًا (عاديًا) على منحنيات المستوى أو الأسطح المستوية
  لـ \(f\).
\item
  المشتقات الاتجاهية تعميم الميل في اتجاهات تعسفية.
\end{itemize}

\subsubsection{لماذا هذا
مهم}\label{ux644ux645ux627ux630ux627-ux647ux630ux627-ux645ux647ux645-13}

\begin{itemize}
\tightlist
\item
  في عملية التحسين، يخبرنا التدرج بالاتجاه الذي يجب التحرك فيه للوصول
  إلى أقصى درجات الصعود أو الهبوط.
\item
  في الفيزياء، تصف التدرجات مجالات مثل التدفق الحراري والإمكانات
  الكهربائية.
\item
  المشتقات الاتجاهية توحد معدلات التغيير ذات المتغير الواحد والمتعددة
  المتغيرات.
\end{itemize}

\subsubsection{تمارين}\label{ux62aux645ux627ux631ux64aux646-29}

\begin{enumerate}
\def\labelenumi{\arabic{enumi}.}
\tightlist
\item
  قم بحساب \(\nabla f(x,y)\) لـ \(f(x,y) = e^{xy}\).
\item
  ابحث عن التدرج اللوني لـ \(f(x,y,z) = x^2+y^2+z^2\) وقم بتقييمه عند
  (1,1,1).
\item
  احسب المشتق الاتجاهي لـ \(f(x,y) = x^2-y\) عند (2,1) في اتجاه
  \(\mathbf{u} = \langle 0,1 \rangle\).
\item
  أظهر أن التدرج اللوني لـ \(f(x,y) = x^2+y^2\) عمودي على الدائرة
  \(x^2+y^2=1\).
\item
  ابحث عن اتجاه متجه الوحدة الذي يزيد من مشتق الاتجاه لـ \(f(x,y) = xy\)
  عند (1,2).
\end{enumerate}

\subsection{8.4 المستويات المماسية والتقريبات
الخطية}\label{ux627ux644ux645ux633ux62aux648ux64aux627ux62a-ux627ux644ux645ux645ux627ux633ux64aux629-ux648ux627ux644ux62aux642ux631ux64aux628ux627ux62a-ux627ux644ux62eux637ux64aux629}

في حساب التفاضل والتكامل أحادي المتغير، يقترب خط المماس من منحنى بالقرب
من نقطة ما. في حساب التفاضل والتكامل متعدد المتغيرات، المفهوم المماثل هو
مستوى الظل، والذي يوفر تقريبًا خطيًا لسطح بالقرب من نقطة ما.

\subsubsection{المستوى المماس
للسطح}\label{ux627ux644ux645ux633ux62aux648ux649-ux627ux644ux645ux645ux627ux633-ux644ux644ux633ux637ux62d}

لنفترض أن \(z = f(x,y)\) قابل للتمييز في \((a,b)\). يتم إعطاء مستوى الظل
عند \((a,b,f(a,b))\) بواسطة

\[
z = f(a,b) + f_x(a,b)(x-a) + f_y(a,b)(y-b).
\]تلامس هذه الطائرة السطح عند هذه النقطة وتقترب منه بالقرب منها.

\subsubsection{مثال 1: القطع
المكافئ}\label{ux645ux62bux627ux644-1-ux627ux644ux642ux637ux639-ux627ux644ux645ux643ux627ux641ux626}

بالنسبة إلى \(f(x,y) = x^2 + y^2\) على \((1,2)\):

\begin{itemize}
\tightlist
\item
  \(f(1,2) = 1^2+2^2=5\).
\item
  \(f_x = 2x\)، إذن \(f_x(1,2) = 2\).
\item
  \(f_y = 2y\)، إذن \(f_y(1,2) = 4\).
\end{itemize}

معادلة المستوى المماس:

\[
z = 5 + 2(x-1) + 4(y-2).
\]

\subsubsection{التقريب
الخطي}\label{ux627ux644ux62aux642ux631ux64aux628-ux627ux644ux62eux637ux64a}

يمكن استخدام مستوى الظل لتقريب \(f(x,y)\) بالقرب من \((a,b)\):

\[
f(x,y) \approx f(a,b) + f_x(a,b)(x-a) + f_y(a,b)(y-b).
\]

هذه هي الخطية لـ \(f\) في \((a,b)\).

\subsubsection{مثال 2: التقريب
الخطي}\label{ux645ux62bux627ux644-2-ux627ux644ux62aux642ux631ux64aux628-ux627ux644ux62eux637ux64a}

\(f(x,y) = \sqrt{x+y}\) تقريبي بالقرب من \((4,5)\).

\begin{itemize}
\tightlist
\item
  \(f(4,5) = \sqrt{9} = 3\).
\item
  \(f_x = \frac{1}{2\sqrt{x+y}}, \quad f_y = \frac{1}{2\sqrt{x+y}}\).
\item
  عند (4,5): \(f_x = f_y = \tfrac{1}{6}\).
\end{itemize}

لذا،

\[
f(x,y) \approx 3 + \tfrac{1}{6}(x-4) + \tfrac{1}{6}(y-5).
\]

\subsubsection{لماذا هذا
مهم}\label{ux644ux645ux627ux630ux627-ux647ux630ux627-ux645ux647ux645-14}

\begin{itemize}
\tightlist
\item
  تعطي المستويات المماسية أفضل تقريب خطي للسطح.
\item
  الخطي يبسط الوظائف المعقدة للحساب.
\item
  تستخدم على نطاق واسع في الطرق العددية، والفيزياء، والاقتصاد.
\end{itemize}

\subsubsection{تمارين}\label{ux62aux645ux627ux631ux64aux646-30}

\begin{enumerate}
\def\labelenumi{\arabic{enumi}.}
\tightlist
\item
  ابحث عن مستوى المماس لـ \(z = x^2y + y^2\) عند \((1,1)\).
\item
  \(f(x,y) = e^{x+y}\) تقريبي بالقرب من \((0,0)\).
\item
  اشتق معادلة مستوى الظل لـ \(z = \ln(x^2+y^2)\) عند \((1,1)\).
\item
  استخدم التقريب الخطي لتقدير \(\sqrt{10.1}\) باستخدام
  \(f(x,y) = \sqrt{x+y}\) بالقرب من (4,6).
\item
  اشرح سبب تحسن تقريب مستوى الظل مع اقتراب \((x,y)\) من \((a,b)\).
\end{enumerate}

\subsection{8.5 التحسين في العديد من
المتغيرات}\label{ux627ux644ux62aux62dux633ux64aux646-ux641ux64a-ux627ux644ux639ux62fux64aux62f-ux645ux646-ux627ux644ux645ux62aux63aux64aux631ux627ux62a}

يؤدي التحسين في حساب التفاضل والتكامل متعدد المتغيرات إلى توسيع أفكار
الحد الأقصى والحد الأدنى من وظائف ذات متغير واحد إلى وظائف ذات متغيرين
أو أكثر.

\subsubsection{النقاط
الحرجة}\label{ux627ux644ux646ux642ux627ux637-ux627ux644ux62dux631ux62cux629}

بالنسبة إلى \(f(x,y)\)، تحدث نقطة حرجة حيث

\[
f_x(x,y) = 0 \quad \text{and} \quad f_y(x,y) = 0,
\]

أو في حالة عدم وجود المشتقات الجزئية.

\subsubsection{اختبار المشتقة الثانيةلتصنيف النقاط الحرجة، حساب المشتقات
الجزئية
الثانية:}\label{ux627ux62eux62aux628ux627ux631-ux627ux644ux645ux634ux62aux642ux629-ux627ux644ux62bux627ux646ux64aux629ux644ux62aux635ux646ux64aux641-ux627ux644ux646ux642ux627ux637-ux627ux644ux62dux631ux62cux629-ux62dux633ux627ux628-ux627ux644ux645ux634ux62aux642ux627ux62a-ux627ux644ux62cux632ux626ux64aux629-ux627ux644ux62bux627ux646ux64aux629}

\[
D = f_{xx}(a,b) f_{yy}(a,b) - \big(f_{xy}(a,b)\big)^2.
\]

\begin{itemize}
\tightlist
\item
  إذا كان \(D > 0\) و\(f_{xx}(a,b) > 0\): الحد الأدنى المحلي.
\item
  إذا كان \(D > 0\) و\(f_{xx}(a,b) < 0\): الحد الأقصى المحلي.
\item
  إذا كان \(D < 0\): نقطة السرج.
\item
  إذا كان \(D = 0\): الاختبار غير حاسم.
\end{itemize}

\subsubsection{مثال 1: القطع
المكافئ}\label{ux645ux62bux627ux644-1-ux627ux644ux642ux637ux639-ux627ux644ux645ux643ux627ux641ux626-1}

\(f(x,y) = x^2 + y^2\).

\begin{itemize}
\tightlist
\item
  \(f_x = 2x, f_y = 2y\). النقطة الحرجة عند (0,0).
\item
  \(f_{xx} = 2, f_{yy} = 2, f_{xy} = 0\).
\item
  \(D = (2)(2) - 0 = 4 > 0\)، و\(f_{xx} > 0\).
\item
  إذًا (0,0) هو الحد الأدنى المحلي.
\end{itemize}

\subsubsection{مثال 2: نقطة
السرج}\label{ux645ux62bux627ux644-2-ux646ux642ux637ux629-ux627ux644ux633ux631ux62c}

\(f(x,y) = x^2 - y^2\).

\begin{itemize}
\tightlist
\item
  \(f_x = 2x, f_y = -2y\). النقطة الحرجة عند (0,0).
\item
  \(f_{xx} = 2, f_{yy} = -2, f_{xy} = 0\).
\item
  \(D = (2)(-2) - 0 = -4 < 0\).
\item
  إذًا (0,0) هي نقطة سرج.
\end{itemize}

\subsubsection{التحسين المقيد ومضاعفات
لاغرانج}\label{ux627ux644ux62aux62dux633ux64aux646-ux627ux644ux645ux642ux64aux62f-ux648ux645ux636ux627ux639ux641ux627ux62a-ux644ux627ux63aux631ux627ux646ux62c}

في بعض الأحيان، نريد تحسين \(f(x,y)\) مع مراعاة القيد \(g(x,y) = c\).

طريقة مضاعفات لاغرانج: حل

\[
\nabla f(x,y) = \lambda \nabla g(x,y).
\]

مثال: تكبير \(f(x,y) = xy\) مع مراعاة \(x^2+y^2=1\).

\begin{itemize}
\tightlist
\item
  التدرجات:
  \(\nabla f = \langle y,x \rangle, \quad \nabla g = \langle 2x,2y \rangle\).
\item
  المعادلات: \(y = 2\lambda x, \, x = 2\lambda y\).
\item
  تؤدي الحلول إلى الحد الأقصى عند
  \((\pm \tfrac{1}{\sqrt{2}}, \pm \tfrac{1}{\sqrt{2}})\).
\end{itemize}

\subsubsection{لماذا هذا
مهم}\label{ux644ux645ux627ux630ux627-ux647ux630ux627-ux645ux647ux645-15}

\begin{itemize}
\tightlist
\item
  يعد التحسين أمرًا ضروريًا في الاقتصاد والهندسة والتعلم الآلي والفيزياء.
\item
  تسمح مضاعفات لاغرانج بالتحسين مع القيود، وهي أداة رئيسية في الرياضيات
  التطبيقية.
\end{itemize}

\subsubsection{تمارين}\label{ux62aux645ux627ux631ux64aux646-31}

\begin{enumerate}
\def\labelenumi{\arabic{enumi}.}
\tightlist
\item
  ابحث عن النقاط الحرجة في \(f(x,y) = x^2+xy+y^2\) وصنفها.
\item
  صنف النقطة (0,0) لـ \(f(x,y) = x^3-y^3\).
\item
  استخدم اختبار المشتقة الثانية لـ \(f(x,y) = x^4+y^4-4xy\).
\item
  قم بتكبير \(f(x,y) = x+y\) الخاضع لـ \(x^2+y^2=1\).
\item
  قم بتصغير \(f(x,y) = x^2+2y^2\) الخاضع لـ \(x+y=1\).
\end{enumerate}

\section{الفصل 9. التكاملات
المتعددة}\label{ux627ux644ux641ux635ux644-9.-ux627ux644ux62aux643ux627ux645ux644ux627ux62a-ux627ux644ux645ux62aux639ux62fux62fux629}

\subsection{9.1 التكاملات المزدوجةفي حساب التفاضل والتكامل ذو المتغير
الواحد، يعطي التكامل المحدد المساحة تحت المنحنى. في متغيرين، يحسب
التكامل المزدوج الحجم تحت السطح (أو بشكل أعم، تراكم القيم على
المنطقة).}\label{ux627ux644ux62aux643ux627ux645ux644ux627ux62a-ux627ux644ux645ux632ux62fux648ux62cux629ux641ux64a-ux62dux633ux627ux628-ux627ux644ux62aux641ux627ux636ux644-ux648ux627ux644ux62aux643ux627ux645ux644-ux630ux648-ux627ux644ux645ux62aux63aux64aux631-ux627ux644ux648ux627ux62dux62f-ux64aux639ux637ux64a-ux627ux644ux62aux643ux627ux645ux644-ux627ux644ux645ux62dux62fux62f-ux627ux644ux645ux633ux627ux62dux629-ux62aux62dux62a-ux627ux644ux645ux646ux62dux646ux649.-ux641ux64a-ux645ux62aux63aux64aux631ux64aux646-ux64aux62dux633ux628-ux627ux644ux62aux643ux627ux645ux644-ux627ux644ux645ux632ux62fux648ux62c-ux627ux644ux62dux62cux645-ux62aux62dux62a-ux627ux644ux633ux637ux62d-ux623ux648-ux628ux634ux643ux644-ux623ux639ux645-ux62aux631ux627ux643ux645-ux627ux644ux642ux64aux645-ux639ux644ux649-ux627ux644ux645ux646ux637ux642ux629.}

\subsubsection{التعريف}\label{ux627ux644ux62aux639ux631ux64aux641-9}

إذا كان \(f(x,y)\) مستمرًا في منطقة \(R\)، فإن التكامل المزدوج هو

\[
\iint_R f(x,y)\, dA = \lim_{m,n \to \infty} \sum_{i=1}^m \sum_{j=1}^n f(x_{ij}^-, y_{ij}^-) \Delta A,
\]

حيث يتم تقسيم \(R\) إلى مستطيلات صغيرة بمساحة \(\Delta A\).

\subsubsection{التكاملات
المتكررة}\label{ux627ux644ux62aux643ux627ux645ux644ux627ux62a-ux627ux644ux645ux62aux643ux631ux631ux629}

بواسطة نظرية فوبيني، يمكننا حساب التكامل المزدوج باعتباره تكاملًا متكررًا:

\[
\iint_R f(x,y)\, dA = \int_a^b \int_c^d f(x,y)\, dy\, dx,
\]

إذا كان \(R\) مستطيلًا \([a,b] \times [c,d]\).

غالبًا ما يمكن تبديل ترتيب التكامل:

\[
\int_a^b \int_c^d f(x,y)\,dy\,dx = \int_c^d \int_a^b f(x,y)\,dx\,dy.
\]

\subsubsection{أمثلة}\label{ux623ux645ux62bux644ux629-19}

\begin{enumerate}
\def\labelenumi{\arabic{enumi}.}
\tightlist
\item
  منطقة المستطيل
\end{enumerate}

\[
\iint_R (x+y)\, dA, \quad R=[0,1]\times[0,2].
\]

\[
= \int_0^1 \int_0^2 (x+y)\,dy\,dx = \int_0^1 \Big[xy+\tfrac{1}{2}y^2\Big]_0^2 dx
= \int_0^1 (2x+2)dx = 3.
\]

\begin{enumerate}
\def\labelenumi{\arabic{enumi}.}
\setcounter{enumi}{1}
\tightlist
\item
  المنطقة المثلثة
\end{enumerate}

\[
R = \{(x,y): 0 \leq x \leq 1, 0 \leq y \leq x\}.
\]

\[
\iint_R (x+y)\, dA = \int_0^1 \int_0^x (x+y)\,dy\,dx.
\]

التقييم يعطي \(\tfrac{2}{3}\).

\subsubsection{التطبيقات}\label{ux627ux644ux62aux637ux628ux64aux642ux627ux62a-1}

\begin{itemize}
\tightlist
\item
  الحجم تحت السطح:
\end{itemize}

\[
V = \iint_R f(x,y)\, dA.
\]

\begin{itemize}
\tightlist
\item
  متوسط قيمة الدالة على المنطقة:
\end{itemize}

\[
f_{\text{avg}} = \frac{1}{A(R)} \iint_R f(x,y)\, dA.
\]

\subsubsection{لماذا هذا
مهم}\label{ux644ux645ux627ux630ux627-ux647ux630ux627-ux645ux647ux645-16}

تعمل التكاملات المزدوجة على توسيع التكامل إلى بعدين. فهي ضرورية في
الفيزياء (الكتلة، التوزيعات الاحتمالية)، والاقتصاد (القيم المتوقعة)،
والهندسة (النقط الوسطى، والتدفق).

\subsubsection{تمارين}\label{ux62aux645ux627ux631ux64aux646-32}

\begin{enumerate}
\def\labelenumi{\arabic{enumi}.}
\tightlist
\item
  قم بتقييم \(\iint_R (x^2+y^2)\, dA\) حيث \(R=[0,1]\times[0,1]\).
\item
  قم بحساب \(\iint_R xy\, dA\) حيث
  \(R=\{(x,y):0\leq x\leq2,0\leq y\leq x\}\).3. أوجد متوسط
  \hspace{0pt}\hspace{0pt}قيمة \(f(x,y) = x+y\) على مربع الوحدة
  \([0,1]\times[0,1]\).
\item
  قم بتفسير \(\iint_R f(x,y)\, dA\) من حيث الاحتمالية إذا كان \(f(x,y)\)
  دالة كثافة الاحتمالية.
\item
  أظهر أن تبديل ترتيب التكامل يعطي نفس النتيجة لـ
  \(\iint_{[0,1]\times[0,2]} (x+y)\,dA\).
\end{enumerate}

\subsection{9.2 التكاملات
الثلاثية}\label{ux627ux644ux62aux643ux627ux645ux644ux627ux62a-ux627ux644ux62bux644ux627ux62bux64aux629}

تعمل التكاملات الثلاثية على توسيع فكرة التكامل لتشمل ثلاثة متغيرات، مما
يسمح لنا بحساب الأحجام والكتل والكميات الأخرى في مناطق ثلاثية الأبعاد.

\subsubsection{التعريف}\label{ux627ux644ux62aux639ux631ux64aux641-10}

إذا كان \(f(x,y,z)\) متصلاً على منطقة صلبة \(E\)، فإن التكامل الثلاثي هو

\[
\iiint_E f(x,y,z)\, dV = \lim_{m,n,p \to \infty} \sum f(x_{ijk}^-, y_{ijk}^-, z_{ijk}^-) \Delta V,
\]

حيث يتم تقسيم المنطقة إلى مربعات من الحجم \(\Delta V\).

\subsubsection{التكاملات
المتكررة}\label{ux627ux644ux62aux643ux627ux645ux644ux627ux62a-ux627ux644ux645ux62aux643ux631ux631ux629-1}

بواسطة نظرية فوبيني، يمكن حساب التكامل الثلاثي كتكامل متكرر:

\[
\iiint_E f(x,y,z)\, dV = \int_a^b \int_c^d \int_e^f f(x,y,z)\, dz\, dy\, dx,
\]

لصندوق مستطيل \(E = [a,b]\times[c,d]\times[e,f]\).

يمكن اختيار ترتيب التكامل للراحة.

\subsubsection{أمثلة}\label{ux623ux645ux62bux644ux629-20}

\begin{enumerate}
\def\labelenumi{\arabic{enumi}.}
\tightlist
\item
  صندوق مستطيل
\end{enumerate}

\[
\iiint_E xyz\, dV, \quad E=[0,1]\times[0,2]\times[0,3].
\]

\[
= \int_0^1 \int_0^2 \int_0^3 xyz\,dz\,dy\,dx.
\]

التكامل أولاً عبر \(z\):

\[
\int_0^3 xyz\,dz = xy \left[\tfrac{1}{2}z^2\right]_0^3 = \tfrac{9}{2}xy.
\]

التكامل الآن عبر \(y\):

\[
\int_0^2 \tfrac{9}{2}xy\,dy = \tfrac{9}{2}x \cdot \left[\tfrac{1}{2}y^2\right]_0^2 = 9x.
\]

أخيرًا التكامل عبر \(x\):

\[
\int_0^1 9x\,dx = \tfrac{9}{2}.
\]

\begin{enumerate}
\def\labelenumi{\arabic{enumi}.}
\setcounter{enumi}{1}
\tightlist
\item
  المنطقة التي تحدها الطائرات دع
  \(E = \{(x,y,z) \mid 0 \leq x \leq 1, 0 \leq y \leq x, 0 \leq z \leq y\}\).
\end{enumerate}

\[
\iiint_E 1\,dV = \int_0^1 \int_0^x \int_0^y 1\,dz\,dy\,dx.
\]

تقييم:

\[
= \int_0^1 \int_0^x y\,dy\,dx = \int_0^1 \tfrac{1}{2}x^2\,dx = \tfrac{1}{6}.
\]إذن حجم هذه المنطقة المثلثة هو \(\tfrac{1}{6}\).

\subsubsection{التطبيقات}\label{ux627ux644ux62aux637ux628ux64aux642ux627ux62a-2}

\begin{itemize}
\item
  المجلد: \(V = \iiint_E 1 \, dV\).
\item
  الكتلة: إذا كانت الكثافة \(\rho(x,y,z)\)، إذن

  \[
  M = \iiint_E \rho(x,y,z)\, dV.
  \]
\item
  متوسط القيمة:

  \[
  f_{\text{avg}} = \frac{1}{V(E)} \iiint_E f(x,y,z)\,dV.
  \]
\end{itemize}

\subsubsection{لماذا هذا
مهم}\label{ux644ux645ux627ux630ux627-ux647ux630ux627-ux645ux647ux645-17}

تعمل التكاملات الثلاثية على تعميم حسابات المساحة والحجم على المواد
الصلبة العشوائية. يتم استخدامها في الفيزياء (التوزيعات الجماعية، ومركز
الكتلة، ومجالات الجاذبية)، والهندسة، والاحتمالات.

\subsubsection{تمارين}\label{ux62aux645ux627ux631ux64aux646-33}

\begin{enumerate}
\def\labelenumi{\arabic{enumi}.}
\tightlist
\item
  قم بحساب \(\iiint_E (x+y+z)\,dV\) فوق المكعب
  \(E=[0,1]\times[0,1]\times[0,1]\).
\item
  أوجد حجم الشكل الرباعي المحدود بـ \(x=0, y=0, z=0, x+y+z=1\).
\item
  قم بتقييم \(\iiint_E x^2 \, dV\) حيث
  \(E=[0,2]\times[0,1]\times[0,1]\).
\item
  أظهر أن \(\iiint_E 1\,dV\) يساوي الحجم الهندسي لـ \(E\).
\item
  إذا كانت الكثافة \(\rho(x,y,z)=x+y+z\)، فاحسب كتلة مكعب الوحدة.
\end{enumerate}

\subsection{9.3 تطبيقات: الحجم، الكتلة،
الاحتمال}\label{ux62aux637ux628ux64aux642ux627ux62a-ux627ux644ux62dux62cux645-ux627ux644ux643ux62aux644ux629-ux627ux644ux627ux62dux62aux645ux627ux644}

التكاملات الثلاثية قوية لأنها تسمح لنا بحساب الكميات في ثلاثة أبعاد من
خلال تجميع القيم على منطقة صلبة.

\subsubsection{الحجم}\label{ux627ux644ux62dux62cux645}

أبسط تطبيق هو العثور على حجم المنطقة \(E\):

\[
V = \iiint_E 1 \, dV.
\]

مثال: أوجد حجم المجسم المحصور بين المستويات الإحداثية والمستوى
\(x+y+z=1\).

\[
V = \iiint_E 1 \, dV = \int_0^1 \int_0^{1-x} \int_0^{1-x-y} 1 \, dz\, dy\, dx.
\]

التقييم يعطي \(V = \tfrac{1}{6}\).

\subsubsection{الكتلة
والكثافة}\label{ux627ux644ux643ux62aux644ux629-ux648ux627ux644ux643ux62bux627ux641ux629}

إذا كانت المادة الصلبة لها دالة الكثافة \(\rho(x,y,z)\)، فإن كتلتها تكون

\[
M = \iiint_E \rho(x,y,z)\, dV.
\]

يتم إعطاء مركز الكتلة بواسطة

\[
\bar{x} = \frac{1}{M}\iiint_E x\rho(x,y,z)\,dV, \quad
\bar{y} = \frac{1}{M}\iiint_E y\rho(x,y,z)\,dV, \quad
\bar{z} = \frac{1}{M}\iiint_E z\rho(x,y,z)\,dV.
\]

مثال:بالنسبة لمكعب الوحدة ذو الكثافة الثابتة \(\rho=1\)، يكون مركز
الكتلة عند \((0.5,0.5,0.5)\).

\subsubsection{الاحتمالية}\label{ux627ux644ux627ux62dux62aux645ux627ux644ux64aux629}

إذا كانت \(f(x,y,z)\) عبارة عن دالة كثافة احتمالية ثلاثية الأبعاد، فإن
احتمال وجود المتغير العشوائي في منطقة \(E\) هو

\[
P(E) = \iiint_E f(x,y,z)\, dV,
\]

حيث \(f(x,y,z) \geq 0\) و

\[
\iiint_{\mathbb{R}^3} f(x,y,z)\,dV = 1.
\]

مثال: إذا كان \(f(x,y,z) = \tfrac{3}{4}z^2\) لـ \(0 \leq z \leq 1\)،
بشكل موحد في \(x,y\)، إذن

\[
P(0 \leq z \leq 0.5) = \int_0^{0.5} \tfrac{3}{4}z^2 \, dz = \tfrac{1}{32}.
\]

\subsubsection{لماذا هذا
مهم}\label{ux644ux645ux627ux630ux627-ux647ux630ux627-ux645ux647ux645-18}

\begin{itemize}
\tightlist
\item
  الأحجام تعمم الهندسة على المواد الصلبة غير المنتظمة.
\item
  تكاملات الكتلة والكثافة تربط حساب التفاضل والتكامل بالفيزياء والهندسة.
\item
  تُستخدم دوال الكثافة الاحتمالية في الأبعاد العليا على نطاق واسع في
  الإحصاء وعلوم البيانات.
\end{itemize}

\subsubsection{تمارين}\label{ux62aux645ux627ux631ux64aux646-34}

\begin{enumerate}
\def\labelenumi{\arabic{enumi}.}
\tightlist
\item
  أوجد حجم المادة الصلبة المحصورة بـ \(x^2+y^2+z^2 \leq 1\) (كرة
  الوحدة).
\item
  احسب كتلة المخروط الذي تتناسب كثافته مع \(z\).
\item
  أوجد مركز كتلة شكل رباعي منتظم يحده \(x=0, y=0, z=0, x+y+z=1\).
\item
  إذا كان \(f(x,y,z) = \frac{1}{8}\) على المكعب
  \([0,2]\times[0,2]\times[0,2]\)، فتحقق من أنها دالة كثافة احتمالية.
\item
  استخدم التكامل الثلاثي لحساب احتمال أن يكون لنقطة مختارة عشوائيًا في
  كرة الوحدة \(z > 0\).
\end{enumerate}

\subsection{9.4 تغيير المتغيرات: الإحداثيات القطبية والأسطوانية
والكروية}\label{ux62aux63aux64aux64aux631-ux627ux644ux645ux62aux63aux64aux631ux627ux62a-ux627ux644ux625ux62dux62fux627ux62bux64aux627ux62a-ux627ux644ux642ux637ux628ux64aux629-ux648ux627ux644ux623ux633ux637ux648ux627ux646ux64aux629-ux648ux627ux644ux643ux631ux648ux64aux629}

تصبح العديد من التكاملات أسهل عند التعبير عنها في أنظمة إحداثية تتوافق
مع تناظر المنطقة. بدلاً من الإحداثيات الديكارتية \((x,y,z)\)، يمكننا
استخدام الإحداثيات القطبية أو الأسطوانية أو الكروية.

\subsubsection{الإحداثيات القطبية (ثنائية
الأبعاد)}\label{ux627ux644ux625ux62dux62fux627ux62bux64aux627ux62a-ux627ux644ux642ux637ux628ux64aux629-ux62bux646ux627ux626ux64aux629-ux627ux644ux623ux628ux639ux627ux62f}

بالنسبة للدوال ذات المتغيرين، يمكننا التبديل إلى الإحداثيات القطبية:

\[
x = r\cos\theta, \quad y = r\sin\theta, \quad r \geq 0, \; 0 \leq \theta < 2\pi.
\]

يتحول عنصر المنطقة كـ

\[
dA = r\,dr\,d\theta.
\]

مثال:أوجد مساحة دائرة الوحدة.

\[
A = \iint_{x^2+y^2\leq 1} 1\,dA = \int_0^{2\pi}\int_0^1 r\,dr\,d\theta = \pi.
\]

\subsubsection{الإحداثيات الأسطوانية (ثلاثية
الأبعاد)}\label{ux627ux644ux625ux62dux62fux627ux62bux64aux627ux62a-ux627ux644ux623ux633ux637ux648ux627ux646ux64aux629-ux62bux644ux627ux62bux64aux629-ux627ux644ux623ux628ux639ux627ux62f}

في الوضع ثلاثي الأبعاد، تعمل الإحداثيات الأسطوانية على توسيع الإحداثيات
القطبية باستخدام \(z\):

\[
x = r\cos\theta, \quad y = r\sin\theta, \quad z = z.
\]

عنصر الحجم هو

\[
dV = r\,dr\,d\theta\,dz.
\]

مثال: حجم الأسطوانة نصف القطر \(R\) والارتفاع \(h\):

\[
V = \int_0^h \int_0^{2\pi} \int_0^R r\,dr\,d\theta\,dz = \pi R^2 h.
\]

\subsubsection{الإحداثيات الكروية (ثلاثية
الأبعاد)}\label{ux627ux644ux625ux62dux62fux627ux62bux64aux627ux62a-ux627ux644ux643ux631ux648ux64aux629-ux62bux644ux627ux62bux64aux629-ux627ux644ux623ux628ux639ux627ux62f}

للتناظر الكروي، استخدم:

\[
x = \rho \sin\phi \cos\theta, \quad y = \rho \sin\phi \sin\theta, \quad z = \rho \cos\phi,
\]

أين

\begin{itemize}
\tightlist
\item
  \(\rho \geq 0\) هي المسافة من نقطة الأصل،
\item
  \(0 \leq \phi \leq \pi\) هي الزاوية من المحور الموجب \(z\)،
\item
  \(0 \leq \theta < 2\pi\) هي الزاوية في المستوى \(xy\).
\end{itemize}

عنصر الحجم هو

\[
dV = \rho^2 \sin\phi \, d\rho\, d\phi\, d\theta.
\]

مثال: حجم وحدة الكرة:

\[
V = \int_0^{2\pi} \int_0^\pi \int_0^1 \rho^2 \sin\phi \, d\rho\, d\phi\, d\theta.
\]

التقييم:

\[
V = \left(\int_0^1 \rho^2 d\rho\right)\left(\int_0^\pi \sin\phi d\phi\right)\left(\int_0^{2\pi} d\theta\right) = \tfrac{1}{3}(2)(2\pi) = \tfrac{4\pi}{3}.
\]

\subsubsection{لماذا هذا
مهم}\label{ux644ux645ux627ux630ux627-ux647ux630ux627-ux645ux647ux645-19}

\begin{itemize}
\tightlist
\item
  الإحداثيات القطبية تبسط المناطق الدائرية.
\item
  الإحداثيات الأسطوانية تتعامل مع الأسطوانات والتماثل الدوراني.
\item
  الإحداثيات الكروية تبسط المجالات والأقماع والمسائل الشعاعية.
\item
  هذه التغييرات في المتغيرات تجعل التكاملات المستحيلة قابلة للإدارة.
\end{itemize}

\subsubsection{تمارين}\label{ux62aux645ux627ux631ux64aux646-35}

\begin{enumerate}
\def\labelenumi{\arabic{enumi}.}
\tightlist
\item
  قم بحساب \(\iint_{x^2+y^2\leq 4} (x^2+y^2)\,dA\) باستخدام الإحداثيات
  القطبية.
\item
  أوجد حجم المخروط الذي ارتفاعه \(h\) ونصف قطره \(R\) باستخدام
  الإحداثيات الأسطوانية.
\item
  استخدم الإحداثيات الكروية لتقييم حجم كرة نصف قطرها \(R\).
\item
  بيّن أن العامل اليعقوبي للإحداثيات القطبية هو \(r\).5. أوجد كتلة كرة
  صلبة نصف قطرها \(R\) وكثافتها متناسبة مع المسافة من نقطة الأصل
  باستخدام الإحداثيات الكروية.
\end{enumerate}

\section{الفصل 10. حساب التفاضل والتكامل
المتجه}\label{ux627ux644ux641ux635ux644-10.-ux62dux633ux627ux628-ux627ux644ux62aux641ux627ux636ux644-ux648ux627ux644ux62aux643ux627ux645ux644-ux627ux644ux645ux62aux62cux647}

\subsection{10.1 الحقول
المتجهة}\label{ux627ux644ux62dux642ux648ux644-ux627ux644ux645ux62aux62cux647ux629}

يقوم الحقل المتجه بتعيين متجه لكل نقطة في الفضاء، مثلما تقوم الدالة
العددية بتعيين رقم. تُستخدم حقول المتجهات لنمذجة التدفقات والقوى والكميات
الاتجاهية الأخرى.

\subsubsection{التعريف}\label{ux627ux644ux62aux639ux631ux64aux641-11}

في البعدين، يكون الحقل المتجه دالة

\[
\mathbf{F}(x,y) = \langle P(x,y), Q(x,y) \rangle,
\]

حيث \(P\) و \(Q\) دالتان عدديتان.

في ثلاثة أبعاد،

\[
\mathbf{F}(x,y,z) = \langle P(x,y,z), Q(x,y,z), R(x,y,z) \rangle.
\]

\subsubsection{أمثلة}\label{ux623ux645ux62bux644ux629-21}

\begin{enumerate}
\def\labelenumi{\arabic{enumi}.}
\tightlist
\item
  المجال الشعاعي
\end{enumerate}

\[
\mathbf{F}(x,y) = \langle x, y \rangle.
\]

تشير المتجهات إلى الخارج من نقطة الأصل.

\begin{enumerate}
\def\labelenumi{\arabic{enumi}.}
\setcounter{enumi}{1}
\tightlist
\item
  مجال التناوب
\end{enumerate}

\[
\mathbf{F}(x,y) = \langle -y, x \rangle.
\]

ناقلات تدور حول الأصل.

\begin{enumerate}
\def\labelenumi{\arabic{enumi}.}
\setcounter{enumi}{2}
\tightlist
\item
  مجال الجاذبية
\end{enumerate}

\[
\mathbf{F}(x,y,z) = -\frac{GM}{r^3}\langle x,y,z \rangle, \quad r=\sqrt{x^2+y^2+z^2}.
\]

\subsubsection{تصور حقول
المتجهات}\label{ux62aux635ux648ux631-ux62dux642ux648ux644-ux627ux644ux645ux62aux62cux647ux627ux62a}

\begin{itemize}
\tightlist
\item
  ارسم أسهمًا صغيرة عند نقاط العينة للإشارة إلى الاتجاه والحجم.
\item
  أسهم أكثر كثافة حيث تكون الأحجام أكبر.
\item
  مفيد في تفسير خطوط التدفق والمسارات والقوى.
\end{itemize}

\subsubsection{خطوط
التدفق}\label{ux62eux637ux648ux637-ux627ux644ux62aux62fux641ux642}

خط التدفق (أو المنحنى المتكامل) لحقل المتجه هو منحنى \(\mathbf{r}(t)\)
الذي يتطابق متجه الظل عند كل نقطة مع الحقل:

\[
\mathbf{r}'(t) = \mathbf{F}(\mathbf{r}(t)).
\]

تصف خطوط التدفق مسارات الجسيمات في مجال السرعة.

\subsubsection{لماذا هذا
مهم}\label{ux644ux645ux627ux630ux627-ux647ux630ux627-ux645ux647ux645-20}

\begin{itemize}
\tightlist
\item
  الحقول المتجهة أساسية في الفيزياء (تدفق الموائع، الكهرومغناطيسية،
  الجاذبية).
\item
  أنها تشكل أساس التكاملات الخطية، والتكاملات السطحية، والنظريات الكبرى
  لحساب التفاضل والتكامل المتجه (جرين، ستوكس، التباعد).
\item
  توفير طريقة هندسية لتمثيل الكميات الاتجاهية.
\end{itemize}

\subsubsection{\texorpdfstring{تمارين1. ارسم حقل المتجه
\(\mathbf{F}(x,y) = \langle y, -x \rangle\).}{تمارين1. ارسم حقل المتجه \textbackslash mathbf\{F\}(x,y) = \textbackslash langle y, -x \textbackslash rangle.}}\label{ux62aux645ux627ux631ux64aux6461.-ux627ux631ux633ux645-ux62dux642ux644-ux627ux644ux645ux62aux62cux647-mathbffxy-langle-y--x-rangle.}

\begin{enumerate}
\def\labelenumi{\arabic{enumi}.}
\setcounter{enumi}{1}
\tightlist
\item
  حدد ما إذا كانت متجهات \(\mathbf{F}(x,y) = \langle x,y \rangle\) تشير
  إلى نقطة الأصل أم بعيدًا عنها.
\item
  بالنسبة إلى \(\mathbf{F}(x,y,z) = \langle y, z, x \rangle\)، قم بحساب
  \(\mathbf{F}(1,2,3)\).
\item
  وصف خطوط التدفق لـ \(\mathbf{F}(x,y) = \langle -y, x \rangle\).
\item
  اشرح لماذا تعد مجالات الجاذبية والكهرباء أمثلة على حقول المتجهات
  الشعاعية.
\end{enumerate}

\subsection{10.2 تكاملات
الخط}\label{ux62aux643ux627ux645ux644ux627ux62a-ux627ux644ux62eux637}

يوسع التكامل الخطي فكرة التكامل إلى الوظائف التي يتم تقييمها على طول
المنحنى. بدلًا من التكامل على فترة أو منطقة، فإننا نتكامل على مسار في
الفضاء.

\subsubsection{التعريف: تكامل الخط
العددي}\label{ux627ux644ux62aux639ux631ux64aux641-ux62aux643ux627ux645ux644-ux627ux644ux62eux637-ux627ux644ux639ux62fux62fux64a}

إذا كانت \(f(x,y)\) عبارة عن دالة عددية و\(C\) عبارة عن منحنى تم تحديد
معلماته بواسطة
\(\mathbf{r}(t) = \langle x(t), y(t) \rangle, \; a \leq t \leq b\)، فإن
التكامل الخطي هو

\[
\int_C f(x,y)\, ds = \int_a^b f(x(t),y(t)) \, |\mathbf{r}'(t)|\, dt,
\]

حيث \(ds\) هو طول القوس.

يقيس هذا تراكم \(f\) على طول المنحنى.

\subsubsection{التعريف: تكامل الخط
المتجه}\label{ux627ux644ux62aux639ux631ux64aux641-ux62aux643ux627ux645ux644-ux627ux644ux62eux637-ux627ux644ux645ux62aux62cux647}

بالنسبة للحقل المتجه
\(\mathbf{F}(x,y) = \langle P(x,y), Q(x,y) \rangle\)، فإن تكامل الخط على
طول \(C\) هو

\[
\int_C \mathbf{F} \cdot d\mathbf{r} = \int_a^b \mathbf{F}(\mathbf{r}(t)) \cdot \mathbf{r}'(t)\, dt.
\]

هذا يقيس العمل الذي يبذله المجال على طول المنحنى.

\subsubsection{أمثلة}\label{ux623ux645ux62bux644ux629-22}

\begin{enumerate}
\def\labelenumi{\arabic{enumi}.}
\tightlist
\item
  تكامل الخط العددي
\end{enumerate}

\[
f(x,y) = x+y, \quad C: \mathbf{r}(t) = \langle t, t^2 \rangle, \; 0 \leq t \leq 1.
\]

ثم

\[
\int_C f(x,y)\, ds = \int_0^1 (t+t^2)\sqrt{(1)^2+(2t)^2}\, dt.
\]

\begin{enumerate}
\def\labelenumi{\arabic{enumi}.}
\setcounter{enumi}{1}
\tightlist
\item
  الشغل الذي تقوم به القوة
\end{enumerate}

\[
\mathbf{F}(x,y) = \langle y, x \rangle, \quad C: \mathbf{r}(t) = \langle t, t^2 \rangle, \; 0 \leq t \leq 1.
\]

\[
\int_C \mathbf{F} \cdot d\mathbf{r} = \int_0^1 \langle t^2, t \rangle \cdot \langle 1, 2t \rangle\, dt = \int_0^1 (t^2 + 2t^2)\, dt = \int_0^1 3t^2\, dt = 1.\]

\subsubsection{Physical Interpretation}\label{physical-interpretation}

\begin{itemize}
\tightlist
\item
  Scalar line integral: accumulation of density along a wire.
\item
  Vector line integral: work done by a force moving an object along a
  path.
\end{itemize}

\subsubsection{Why This Matters}\label{why-this-matters}

\begin{itemize}
\tightlist
\item
  Line integrals connect vector fields with physical quantities like
  work and circulation.
\item
  They are building blocks for Green's Theorem and Stokes' Theorem.
\item
  Appear in physics (electric potential, fluid flow, mechanics).
\end{itemize}

\subsubsection{Exercises}\label{exercises-3}

\begin{enumerate}
\def\labelenumi{\arabic{enumi}.}
\tightlist
\item
  Compute \(\int_C (x^2+y^2)\, ds\) where \(C\) is the line segment from
  (0,0) to (1,1).
\item
  Evaluate \(\int_C \mathbf{F}\cdot d\mathbf{r}\) for
  \(\mathbf{F}(x,y) = \langle -y, x \rangle\) along the unit circle
  \(x^2+y^2=1\).
\item
  Interpret the meaning of \(\int_C 1\,ds\).
\item
  For \(\mathbf{F}(x,y,z) = \langle z,0,x \rangle\), compute the line
  integral along
  \(\mathbf{r}(t) = \langle t,t,1 \rangle, 0 \leq t \leq 1\).
\item
  Explain the difference between scalar and vector line integrals.
\end{enumerate}

\subsection{10.3 Surface Integrals}\label{surface-integrals}

A surface integral generalizes line integrals to two-dimensional
surfaces in three-dimensional space. They allow us to compute flux
through surfaces and accumulation of scalar fields over curved surfaces.

\subsubsection{Scalar Surface Integral}\label{scalar-surface-integral}

If a surface \(S\) is parameterized by

\[
\mathbf{r}(u,v) = \langle x(u,v), y(u,v), z(u,v) \rangle,
\]

then the surface integral of a scalar function \(f(x,y,z)\) is

\[
\iint_S f(x,y,z)\, dS = \iint_D f(\mathbf{r}(u,v)) \, |\mathbf{r}_u \times \mathbf{r}_v| \, دو\,دف,
\]

where \(\mathbf{r}_u\) and \(\mathbf{r}_v\) are partial derivatives of
\(\mathbf{r}(u,v)\), and \(D\) is the parameter domain.

\subsubsection{Vector Surface Integral
(Flux)}\label{vector-surface-integral-flux}

For a vector field \(\mathbf{F}(x,y,z)\), the flux through a surface
\(S\) is

\[
\iint_S \mathbf{F}\cdot d\mathbf{S} = \iint_S \mathbf{F}\cdot \mathbf{n}\, dS,
\]حيث \(\mathbf{n}\) هو متجه الوحدة العادي. باستخدام المعلمة،

\[
\iint_S \mathbf{F}\cdot d\mathbf{S} = \iint_D \mathbf{F}(\mathbf{r}(u,v)) \cdot (\mathbf{r}_u \times \mathbf{r}_v)\,du\,dv.
\]

\subsubsection{أمثلة}\label{ux623ux645ux62bux644ux629-23}

\begin{enumerate}
\def\labelenumi{\arabic{enumi}.}
\tightlist
\item
  تكامل السطح العددي السطح: المستوى \(z=1\) فوق قرص الوحدة
  \(x^2+y^2 \leq 1\).
\end{enumerate}

\[
\iint_S 1\, dS = \text{area of the disk} = \pi.
\]

\begin{enumerate}
\def\labelenumi{\arabic{enumi}.}
\setcounter{enumi}{1}
\tightlist
\item
  التدفق عبر المجال دع \(\mathbf{F}(x,y,z) = \langle x,y,z \rangle\) و
  \(S\) = مجال نصف القطر \(R\). المتجه العادي هو
  \(\mathbf{n} = \frac{1}{R}\langle x,y,z \rangle\).
\end{enumerate}

\[
\mathbf{F}\cdot \mathbf{n} = \frac{x^2+y^2+z^2}{R} = R.
\]

هكذا

\[
\iint_S \mathbf{F}\cdot d\mathbf{S} = \iint_S R\, dS = R \cdot 4\pi R^2 = 4\pi R^3.
\]

\subsubsection{لماذا هذا
مهم}\label{ux644ux645ux627ux630ux627-ux647ux630ux627-ux645ux647ux645-21}

\begin{itemize}
\tightlist
\item
  التكاملات السطحية العددية تقيس المساحة وتوزيعات السطح.
\item
  تكاملات السطح المتجه تقيس التدفق: مقدار المجال الذي يمر عبر السطح.
\item
  التطبيقات: الكهرومغناطيسية، وتدفق السوائل، ونقل الحرارة، وأكثر من ذلك.
\end{itemize}

\subsubsection{تمارين}\label{ux62aux645ux627ux631ux64aux646-36}

\begin{enumerate}
\def\labelenumi{\arabic{enumi}.}
\tightlist
\item
  احسب \(\iint_S 1\, dS\) لسطح مكعب طول ضلعه 2.
\item
  أوجد تدفق \(\mathbf{F}(x,y,z) = \langle x,y,z \rangle\) عبر كرة
  الوحدة.
\item
  قم بتقييم \(\iint_S z\, dS\) للشكل المكافئ
  \(z = x^2+y^2, \, z \leq 1\).
\item
  بالنسبة إلى \(\mathbf{F}(x,y,z) = \langle y,0,0 \rangle\)، حساب التدفق
  عبر المستوى \(x=1\)، \(0 \leq y,z \leq 1\).
\item
  اشرح ماديًا ما الذي يعنيه إذا كان تدفق المجال المتجه عبر سطح مغلق يساوي
  صفرًا.
\end{enumerate}

\subsection{10.4 نظرية
جرين}\label{ux646ux638ux631ux64aux629-ux62cux631ux64aux646}

نظرية جرين هي نتيجة أساسية في حساب التفاضل والتكامل المتجه الذي يربط
التكامل الخطي حول منحنى مغلق بالتكامل المزدوج فوق المنطقة التي يحيط بها.
إنها نسخة ثنائية الأبعاد من نظرية ستوكس.

\subsubsection{\texorpdfstring{بيان نظرية جريناجعل \(C\) منحنى إيجابيًا
وبسيطًا ومغلقًا في المستوى، ودع \(R\) هو المنطقة التي يحيط بها. إذا كان
\(\mathbf{F}(x,y) = \langle P(x,y), Q(x,y) \rangle\) يحتوي على مشتقات
جزئية متصلة في منطقة مفتوحة تحتوي على \(R\)،
إذن}{بيان نظرية جريناجعل C منحنى إيجابيًا وبسيطًا ومغلقًا في المستوى، ودع R هو المنطقة التي يحيط بها. إذا كان \textbackslash mathbf\{F\}(x,y) = \textbackslash langle P(x,y), Q(x,y) \textbackslash rangle يحتوي على مشتقات جزئية متصلة في منطقة مفتوحة تحتوي على R، إذن}}\label{ux628ux64aux627ux646-ux646ux638ux631ux64aux629-ux62cux631ux64aux646ux627ux62cux639ux644-c-ux645ux646ux62dux646ux649-ux625ux64aux62cux627ux628ux64aux627-ux648ux628ux633ux64aux637ux627-ux648ux645ux63aux644ux642ux627-ux641ux64a-ux627ux644ux645ux633ux62aux648ux649-ux648ux62fux639-r-ux647ux648-ux627ux644ux645ux646ux637ux642ux629-ux627ux644ux62aux64a-ux64aux62dux64aux637-ux628ux647ux627.-ux625ux630ux627-ux643ux627ux646-mathbffxy-langle-pxy-qxy-rangle-ux64aux62dux62aux648ux64a-ux639ux644ux649-ux645ux634ux62aux642ux627ux62a-ux62cux632ux626ux64aux629-ux645ux62aux635ux644ux629-ux641ux64a-ux645ux646ux637ux642ux629-ux645ux641ux62aux648ux62dux629-ux62aux62dux62aux648ux64a-ux639ux644ux649-r-ux625ux630ux646}

\[
\oint_C \mathbf{F} \cdot d\mathbf{r} = \oint_C P\,dx + Q\,dy = \iint_R \left( \frac{\partial Q}{\partial x} - \frac{\partial P}{\partial y} \right)\, dA.
\]

\subsubsection{التفسير}\label{ux627ux644ux62aux641ux633ux64aux631-2}

\begin{itemize}
\tightlist
\item
  يقيس الخط المتكامل حول \(C\) دوران حقل المتجه على طول الحدود.
\item
  التكامل المزدوج على \(R\) يقيس الالتفاف الكلي (الدوران) للحقل داخل
  المنطقة.
\end{itemize}

\subsubsection{مثال 1: صيغة
المساحة}\label{ux645ux62bux627ux644-1-ux635ux64aux63aux629-ux627ux644ux645ux633ux627ux62dux629}

إذا كان \(\mathbf{F} = \langle -y/2, x/2 \rangle\)، إذن

\[
\frac{\partial Q}{\partial x} - \frac{\partial P}{\partial y} = 1.
\]

وبالتالي، فإن نظرية جرين تعطي

\[
\text{Area}(R) = \iint_R 1\,dA = \oint_C \left(-\tfrac{y}{2}\,dx + \tfrac{x}{2}\,dy\right).
\]

يوفر هذا طريقة لحساب المساحة باستخدام تكامل خطي.

\subsubsection{مثال 2:
التداول}\label{ux645ux62bux627ux644-2-ux627ux644ux62aux62fux627ux648ux644}

اجعل \(\mathbf{F}(x,y) = \langle -y, x \rangle\) و\(C\) دائرة الوحدة.

\begin{itemize}
\tightlist
\item
  \(P=-y, Q=x\).
\item
  \(Q_x - P_y = 1 - (-1) = 2\).
\item
  تكامل مزدوج على قرص الوحدة:
\end{itemize}

\[
\iint_R 2\,dA = 2\pi (1^2) = 2\pi.
\]

وبالتالي فإن الدوران حول الدائرة هو \(2\pi\).

\subsubsection{لماذا هذا
مهم}\label{ux644ux645ux627ux630ux627-ux647ux630ux627-ux645ux647ux645-22}

\begin{itemize}
\tightlist
\item
  تحويل التكاملات الخطية الصعبة إلى تكاملات مزدوجة، أو العكس.
\item
  يوفر جسرا بين الخصائص المحلية (الضفيرة) والخصائص العالمية (التداول).
\item
  يستخدم على نطاق واسع في الفيزياء لتدفق السوائل، والكهرومغناطيسية،
  ومجالات المتجهات المستوية.
\end{itemize}

\subsubsection{تمارين}\label{ux62aux645ux627ux631ux64aux646-37}

\begin{enumerate}
\def\labelenumi{\arabic{enumi}.}
\tightlist
\item
  استخدم نظرية جرين لحساب المساحة داخل القطع الناقص
  \(\frac{x^2}{a^2} + \frac{y^2}{b^2} = 1\).
\item
  تحقق من نظرية جرين لـ \(\mathbf{F}(x,y) = \langle -y, x \rangle\) على
  طول المربع ذو الرؤوس (0,0)، (1,0)، (1,1)، (0,1).3. احسب دوران
  \(\mathbf{F}(x,y) = \langle -y, x \rangle\) حول دائرة الوحدة.
\item
  أظهر أنه إذا كان \(\nabla \times \mathbf{F} = 0\)، فإن التكامل الخطي
  لـ \(\mathbf{F}\) حول أي منحنى مغلق هو صفر.
\item
  استخدم نظرية جرين لإظهار ذلك
\end{enumerate}

\[
\oint_C x\,dy = -\oint_C y\,dx
\]

لأي منحنى مغلق \(C\).

\subsection{10.5 نظرية
ستوكس}\label{ux646ux638ux631ux64aux629-ux633ux62aux648ux643ux633}

تعمل نظرية ستوكس على تعميم نظرية جرين على ثلاثة أبعاد. إنه يربط تكامل
السطح من حليقة حقل المتجه على سطح ما بخط متكامل من حقل المتجه حول حدود
ذلك السطح.

\subsubsection{بيان نظرية
ستوكس}\label{ux628ux64aux627ux646-ux646ux638ux631ux64aux629-ux633ux62aux648ux643ux633}

دع \(S\) يكون سطحًا أملسًا موجهًا مع منحنى حدودي \(C\) (موجه بشكل إيجابي).
إذا كان \(\mathbf{F}(x,y,z)\) حقلًا متجهًا بمشتقات جزئية مستمرة، إذن

\[
\iint_S (\nabla \times \mathbf{F}) \cdot d\mathbf{S} = \oint_C \mathbf{F} \cdot d\mathbf{r}.
\]

\begin{itemize}
\tightlist
\item
  الجانب الأيسر: تدفق تجعيد \(\mathbf{F}\) عبر السطح.
\item
  الجانب الأيمن: تداول \(\mathbf{F}\) على طول المنحنى الحدودي.
\end{itemize}

\subsubsection{التفسير}\label{ux627ux644ux62aux641ux633ux64aux631-3}

\begin{itemize}
\tightlist
\item
  التكامل الخطي حول الحد يساوي إجمالي ``الدوران'' داخل السطح.
\item
  تمديد نظرية جرين (حالة خاصة عندما يكون السطح في المستوى).
\end{itemize}

\subsubsection{مثال 1: نظرية جرين كحالة
خاصة}\label{ux645ux62bux627ux644-1-ux646ux638ux631ux64aux629-ux62cux631ux64aux646-ux643ux62dux627ux644ux629-ux62eux627ux635ux629}

إذا كانت \(S\) منطقة مسطحة في المستوى \(xy\)، فإن نظرية ستوكس تُختزل إلى
نظرية جرين.

\subsubsection{مثال 2: الدورة الدموية في نصف الكرة
الأرضية}\label{ux645ux62bux627ux644-2-ux627ux644ux62fux648ux631ux629-ux627ux644ux62fux645ux648ux64aux629-ux641ux64a-ux646ux635ux641-ux627ux644ux643ux631ux629-ux627ux644ux623ux631ux636ux64aux629}

دع \(\mathbf{F}(x,y,z) = \langle -y, x, 0 \rangle\) و\(S\) يكونان نصف
الكرة العلوي من نصف القطر 1.

\begin{itemize}
\tightlist
\item
  الحد \(C\): دائرة الوحدة في المستوى \(xy\).
\item
  حسب نظرية ستوكس:
\end{itemize}

\[
\oint_C \mathbf{F}\cdot d\mathbf{r} = \iint_S (\nabla \times \mathbf{F})\cdot d\mathbf{S}.
\]

\begin{itemize}
\tightlist
\item
  الضفيرة: \(\nabla \times \mathbf{F} = \langle 0,0,2 \rangle\).
\item
  من الطبيعي أن يشير نصف الكرة الأرضية إلى الخارج:
  \(\mathbf{n} = \langle 0,0,1 \rangle\).
\item
  إذن التكامل = 2.- مساحة نصف الكرة الأرضية = \(2\pi (1^2)\).
\end{itemize}

\[
\iint_S 2\, dS = 2 \cdot 2\pi = 4\pi.
\]

وبالتالي، فإن الدوران حول خط الاستواء هو \(4\pi\).

\subsubsection{لماذا هذا
مهم}\label{ux644ux645ux627ux630ux627-ux647ux630ux627-ux645ux647ux645-23}

\begin{itemize}
\tightlist
\item
  يوفر اتصالاً عميقًا بين التكاملات السطحية والتكاملات الخطية.
\item
  يبسط العمليات الحسابية من خلال السماح باختيار الأسطح المناسبة.
\item
  يستخدم على نطاق واسع في الكهرومغناطيسية (قانون فاراداي) وديناميكيات
  الموائع.
\end{itemize}

\subsubsection{تمارين}\label{ux62aux645ux627ux631ux64aux646-38}

\begin{enumerate}
\def\labelenumi{\arabic{enumi}.}
\tightlist
\item
  تحقق من نظرية ستوكس لـ
  \(\mathbf{F}(x,y,z) = \langle y, -x, 0 \rangle\) على قرص الوحدة في
  المستوى \(xy\).
\item
  قم بحساب \(\oint_C \mathbf{F}\cdot d\mathbf{r}\) حيث
  \(\mathbf{F}(x,y,z) = \langle z, 0, x \rangle\)، و\(C\) هي حدود المثلث
  ذو الرؤوس (0,0,0)، (1,0,0)، (0,1,0).
\item
  وضح أنه إذا كان \(\nabla \times \mathbf{F} = 0\)، فإن الدوران حول أي
  منحنى مغلق يكون صفرًا.
\item
  طبّق نظرية ستوكس لحساب دوران
  \(\mathbf{F}(x,y,z) = \langle -y, x, z \rangle\) حول حدود مربع الوحدة
  في المستوى \(z=0\).
\item
  اشرح كيف تعمل نظرية ستوكس على تعميم نظرية جرين.
\end{enumerate}

\#\#10.6 نظرية التباعد

تربط نظرية التباعد (وتسمى أيضًا نظرية غاوس) تدفق المجال المتجه عبر سطح
مغلق بالتكامل الثلاثي لتباعد المجال داخل السطح.

\subsubsection{بيان نظرية
التباعد}\label{ux628ux64aux627ux646-ux646ux638ux631ux64aux629-ux627ux644ux62aux628ux627ux639ux62f}

اجعل \(E\) منطقة صلبة في \(\mathbb{R}^3\) مع سطح حدودي \(S\) (موجه نحو
الخارج). إذا كان \(\mathbf{F}(x,y,z)\) حقلًا متجهًا بمشتقات جزئية مستمرة
في \(E\)، إذن

\[
\iint_S \mathbf{F} \cdot d\mathbf{S} = \iiint_E (\nabla \cdot \mathbf{F}) \, dV.
\]

\begin{itemize}
\tightlist
\item
  الجانب الأيسر: تدفق \(\mathbf{F}\) عبر السطح المغلق \(S\).
\item
  الجانب الأيمن: التكامل الثلاثي للتباعد داخل المنطقة.
\end{itemize}

\subsubsection{الاختلاف}\label{ux627ux644ux627ux62eux62aux644ux627ux641}

تباعد حقل المتجه \(\mathbf{F}(x,y,z) = \langle P, Q, R \rangle\) هو

\[\nabla \cdot \mathbf{F} = \frac{\partial P}{\partial x} + \frac{\partial Q}{\partial y} + \frac{\partial R}{\partial z}.
\]

It measures the ``net outflow'' per unit volume at each point.

\subsubsection{Example 1: Flux of a Radial
Field}\label{example-1-flux-of-a-radial-field}

Let \(\mathbf{F}(x,y,z) = \langle x, y, z \rangle\), and let \(E\) be
the unit ball \(x^2+y^2+z^2 \leq 1\).

\begin{itemize}
\tightlist
\item
  Divergence: \(\nabla \cdot \mathbf{F} = 1+1+1 = 3\).
\item
  Volume of unit ball: \(\tfrac{4}{3}\pi\). So
\end{itemize}

\[
\iiint_E (\nabla \cdot \mathbf{F})\, dV = 3 \cdot \tfrac{4}{3}\pi = 4\pi.
\]

وبالتالي، فإن التدفق عبر كرة الوحدة هو \(4\pi\).

\subsubsection{مثال 2: حقل
ثابت}\label{ux645ux62bux627ux644-2-ux62dux642ux644-ux62bux627ux628ux62a}

دع \(\mathbf{F}(x,y,z) = \langle 1, 0, 0 \rangle\).

\begin{itemize}
\tightlist
\item
  التباعد: \(\nabla \cdot \mathbf{F} = 0\).
\item
  لذا فإن التدفق عبر أي سطح مغلق يكون صفرًا، وهو ما يتوافق مع الحدس (لا
  يوجد تدفق خارجي صافي).
\end{itemize}

\subsubsection{لماذا هذا
مهم}\label{ux644ux645ux627ux630ux627-ux647ux630ux627-ux645ux647ux645-24}

\begin{itemize}
\item
  تحويل التكاملات السطحية إلى تكاملات حجمية أبسط.
\item
  يستخدم في الفيزياء: قانون غاوس في الكهرومغناطيسية، وتدفق الموائع،
  وانتقال الحرارة.
\item
  استكمال الإطار الموحد:

  \begin{itemize}
  \tightlist
  \item
    نظرية جرين (الالتفاف ثنائي الأبعاد ↔ الدوران)
  \item
    نظرية ستوكس (التجعيد ثلاثي الأبعاد ↔ الدوران على الأسطح)
  \item
    نظرية التباعد (التباعد ثلاثي الأبعاد ↔ التدفق على الأسطح المغلقة)
  \end{itemize}
\end{itemize}

\subsubsection{تمارين}\label{ux62aux645ux627ux631ux64aux646-39}

\begin{enumerate}
\def\labelenumi{\arabic{enumi}.}
\tightlist
\item
  استخدم نظرية التباعد لحساب تدفق
  \(\mathbf{F}(x,y,z) = \langle x,y,z \rangle\) عبر سطح كرة نصف قطرها
  \(R\).
\item
  تحقق من نظرية التباعد لـ
  \(\mathbf{F}(x,y,z) = \langle y, z, x \rangle\) على مكعب الوحدة
  \([0,1]^3\).
\item
  أظهر أنه إذا كان \(\nabla \cdot \mathbf{F} = 0\)، فإن التدفق الإجمالي
  عبر أي سطح مغلق يساوي صفرًا.
\item
  احسب تدفق \(\mathbf{F}(x,y,z) = \langle x^2, y^2, z^2 \rangle\) عبر
  كرة الوحدة.
\item
  اشرح كيف تعمل نظرية التباعد على تعميم النظرية الأساسية لحساب التفاضل
  والتكامل ذات البعد الواحد.
\end{enumerate}

\#الجزء الرابع. العمليات اللانهائية

\section{الفصل 11. المتتابعات والتقارب\#\# 11.1 تعريفات
وأمثلة}\label{ux627ux644ux641ux635ux644-11.-ux627ux644ux645ux62aux62aux627ux628ux639ux627ux62a-ux648ux627ux644ux62aux642ux627ux631ux628-11.1-ux62aux639ux631ux64aux641ux627ux62a-ux648ux623ux645ux62bux644ux629}

التسلسل عبارة عن قائمة مرتبة من الأرقام، وعادة ما يتم كتابتها كـ

\[
a_1, a_2, a_3, \dots
\]

أو بشكل أعم \((a_n)_{n=1}^\infty\). يُسمى كل \(a_n\) بالحد n من التسلسل.

\subsubsection{تحديد
التسلسل}\label{ux62aux62dux62fux64aux62f-ux627ux644ux62aux633ux644ux633ux644}

يمكن تعريف التسلسل بطريقتين:

\begin{enumerate}
\def\labelenumi{\arabic{enumi}.}
\item
  الصيغة الصريحة - تعطي قاعدة مباشرة للحد n.

  \begin{itemize}
  \item
    مثال: \(a_n = \frac{1}{n}\) يحدد التسلسل

    \[
    1, \tfrac{1}{2}, \tfrac{1}{3}, \tfrac{1}{4}, \dots
    \]
  \end{itemize}
\item
  التعريف العودي -- يحدد المصطلحات باستخدام المصطلحات السابقة.

  \begin{itemize}
  \item
    مثال: تسلسل فيبوناتشي:

    \[
    a_1 = 1, \quad a_2 = 1, \quad a_{n} = a_{n-1} + a_{n-2} \quad (n \geq 3).
    \]
  \end{itemize}
\end{enumerate}

\subsubsection{أمثلة على
التسلسلات}\label{ux623ux645ux62bux644ux629-ux639ux644ux649-ux627ux644ux62aux633ux644ux633ux644ux627ux62a}

\begin{enumerate}
\def\labelenumi{\arabic{enumi}.}
\item
  التسلسل الحسابي:

  \[
  a_n = a_1 + (n-1)d.
  \]

  مثال: \(a_n = 2n+1\) → تسلسل الأرقام الفردية.
\item
  التسلسل الهندسي:

  \[
  a_n = a_1 r^{n-1}.
  \]

  مثال: \(a_n = 2^n\) → قوى الرقم 2.
\item
  التسلسل التوافقي:

  \[
  a_n = \frac{1}{n}.
  \]
\item
  التسلسل المتناوب:

  \[
  a_n = (-1)^n.
  \]
\end{enumerate}

\subsubsection{المتتابعات في حساب التفاضل
والتكامل}\label{ux627ux644ux645ux62aux62aux627ux628ux639ux627ux62a-ux641ux64a-ux62dux633ux627ux628-ux627ux644ux62aux641ux627ux636ux644-ux648ux627ux644ux62aux643ux627ux645ux644}

التسلسلات هي الأساس للعمليات اللانهائية:

\begin{itemize}
\tightlist
\item
  حدود المتتابعات → تعريف التقارب.
\item
  سلسلة → مبالغ لا حصر لها مبنية على تسلسلات.
\item
  وظائف تقريبية عن طريق التسلسلات والسلاسل.
\end{itemize}

\subsubsection{لماذا هذا
مهم}\label{ux644ux645ux627ux630ux627-ux647ux630ux627-ux645ux647ux645-25}

\begin{itemize}
\tightlist
\item
  توفر التسلسلات اللبنات الأساسية للسلاسل والتقريبات اللانهائية.
\item
  إنها تسمح لنا بتعريف ``الاقتراب من اللانهاية'' والتقارب بدقة.
\item
  يمكن التعبير عن العديد من الدوال المهمة (الأسية، المثلثية) من خلال
  المتتابعات والمتسلسلات.
\end{itemize}

\subsubsection{تمارين}\label{ux62aux645ux627ux631ux64aux646-40}

\begin{enumerate}
\def\labelenumi{\arabic{enumi}.}
\tightlist
\item
  اكتب الحدود الخمسة الأولى من التسلسل \(a_n = \frac{n}{n+1}\).
\item
  حدد ما إذا كان \(a_n = (-1)^n n\) محددًا أم لا.
\item
  أعط تعريفًا متكررًا للتسلسل \(2,4,8,16,\dots\).
\item
  أوجد الحد العاشر من المتتابعة الحسابية باستخدام \(a_1=3\) و\(d=5\).5.
  اكتب صيغة صريحة للتسلسل المحدد بواسطة \(a_1=1\)، \(a_{n+1}=2a_n\).
\end{enumerate}

\subsection{11.2 التسلسلات الرتيبة
والمحدودة}\label{ux627ux644ux62aux633ux644ux633ux644ux627ux62a-ux627ux644ux631ux62aux64aux628ux629-ux648ux627ux644ux645ux62dux62fux648ux62fux629}

لفهم ما إذا كان التسلسل يتقارب، نحتاج إلى دراسة سلوكه: هل يزيد أم ينقص
أم يظل ضمن الحدود أم ينمو بلا حدود؟ هناك مفهومان مهمان هما الرتابة
والحدود.

\subsubsection{تسلسلات
رتيبة}\label{ux62aux633ux644ux633ux644ux627ux62a-ux631ux62aux64aux628ux629}

يسمى التسلسل \((a_n)\) رتيبًا إذا كان يتزايد دائمًا أو يتناقص دائمًا.

\begin{itemize}
\item
  زيادة الرتابة:

  \[
  a_{n+1} \geq a_n \quad \text{for all } n.
  \]
\item
  التناقص الرتيب:

  \[
  a_{n+1} \leq a_n \quad \text{for all } n.
  \]
\end{itemize}

أمثلة:

\begin{enumerate}
\def\labelenumi{\arabic{enumi}.}
\tightlist
\item
  \(a_n = n\) يزداد رتيبة.
\item
  \(a_n = \frac{1}{n}\) يتناقص رتيبًا.
\end{enumerate}

\subsubsection{تسلسلات
محدودة}\label{ux62aux633ux644ux633ux644ux627ux62a-ux645ux62dux62fux648ux62fux629}

يتم تحديد التسلسل أعلاه إذا كان هناك رقم \(M\) بحيث يكون \(a_n \leq M\)
لجميع \(n\). يتم تحديده أدناه إذا كان هناك \(m\) بحيث يكون
\(a_n \geq m\) لجميع \(n\).

إذا تحقق كلا الشرطين، فإن التسلسل محدود.

أمثلة:

\begin{enumerate}
\def\labelenumi{\arabic{enumi}.}
\tightlist
\item
  \(a_n = \frac{1}{n}\) محصور بين 0 و1.
\item
  \(a_n = (-1)^n\) محصور بين -1 و1.
\item
  \(a_n = n\) غير محدود.
\end{enumerate}

\subsubsection{نظرية التقارب
الرتيب}\label{ux646ux638ux631ux64aux629-ux627ux644ux62aux642ux627ux631ux628-ux627ux644ux631ux62aux64aux628}

نتيجة أساسية في التحليل:

\begin{itemize}
\tightlist
\item
  كل تسلسل رتيب متزايد يحده أعلاه يتقارب.
\item
  كل تسلسل رتيب تنازلي يحده أدناه يتقارب.
\end{itemize}

تضمن هذه النظرية التقارب دون إيجاد النهاية بشكل صريح.

\subsubsection{مثال}\label{ux645ux62bux627ux644-1}

\begin{enumerate}
\def\labelenumi{\arabic{enumi}.}
\item
  التسلسل: \(a_n = 1 - \frac{1}{n}\).

  \begin{itemize}
  \tightlist
  \item
    متزايد: منذ \(a_{n+1} - a_n = \frac{1}{n} - \frac{1}{n+1} > 0\).
  \item
    يحدها من الأعلى 1.
  \item
    ولذلك فهو متقارب.
  \item
    الحد: \(\lim_{n\to\infty} a_n = 1\).
  \end{itemize}
\end{enumerate}

\subsubsection{لماذا هذا
مهم}\label{ux644ux645ux627ux630ux627-ux647ux630ux627-ux645ux647ux645-26}

\begin{itemize}
\tightlist
\item
  الرتابة والحدود تعطي اختبارات سريعة للتقارب.
\item
  أنها ضرورية في البراهين وفي إقامة الحدود على نحو صارم.
\item
  تمتد هذه الأفكار بشكل طبيعي إلى الوظائف والسلاسل.\#\#\# تمارين
\end{itemize}

\begin{enumerate}
\def\labelenumi{\arabic{enumi}.}
\tightlist
\item
  حدد ما إذا كان \(a_n = \frac{n}{n+1}\) رتيبًا ومحدودًا.
\item
  أظهر أن \(a_n = \sqrt{n}\) رتابة متزايدة ولكنها غير محدودة.
\item
  أثبت أن \(a_n = 2 - \frac{1}{n}\) يتقارب، وأوجد حده.
\item
  أعط مثالاً على تسلسل محدد غير رتيب.
\item
  قم بتطبيق نظرية التقارب الرتيب على
  \(a_n = \ln\!\big(1+\frac{1}{n}\big)\).
\end{enumerate}

\subsection{11.3 حدود
التسلسلات}\label{ux62dux62fux648ux62f-ux627ux644ux62aux633ux644ux633ux644ux627ux62a}

السؤال المركزي حول التسلسل هو ما إذا كانت حدوده تقترب من قيمة واحدة مع
نمو \(n\). وهذا يؤدي إلى مفهوم الحد من التسلسل.

\subsubsection{التعريف}\label{ux627ux644ux62aux639ux631ux64aux641-12}

التسلسل \((a_n)\) له حد \(L\) إذا كان لكل \(\varepsilon > 0\) عدد صحيح
\(N\) بحيث يكون ذلك

\[
|a_n - L| < \varepsilon \quad \text{whenever } n > N.
\]

ثم نكتب

\[
\lim_{n\to\infty} a_n = L.
\]

إذا لم يكن هناك مثل \(L\)، فسيختلف التسلسل.

\subsubsection{الحدس}\label{ux627ux644ux62dux62fux633}

\begin{itemize}
\tightlist
\item
  تقترب شروط التسلسل بشكل تعسفي من \(L\) عندما يصبح \(n\) كبيرًا.
\item
  بخلاف بعض الفهرس \(N\)، تظل جميع المصطلحات ضمن نطاق صغير حول \(L\).
\end{itemize}

\subsubsection{أمثلة}\label{ux623ux645ux62bux644ux629-24}

\begin{enumerate}
\def\labelenumi{\arabic{enumi}.}
\item
  \(a_n = \frac{1}{n}\). مع نمو \(n\)، تتقلص المصطلحات نحو 0.

  \[
  \lim_{n\to\infty} \frac{1}{n} = 0.
  \]
\item
  \(a_n = (-1)^n\). تتناوب المصطلحات بين -1 و1، لذلك لا يوجد حد واحد.
  التسلسل يختلف.
\item
  \(a_n = \frac{n}{n+1}\). بما أن \(n \to \infty\)، فإن البسط والمقام
  متساويان تقريبًا، لذا

  \[
  \lim_{n\to\infty} \frac{n}{n+1} = 1.
  \]
\end{enumerate}

\subsubsection{خصائص
الحدود}\label{ux62eux635ux627ux626ux635-ux627ux644ux62dux62fux648ux62f}

إذا كان \(\lim a_n = A\) و\(\lim b_n = B\):

\begin{itemize}
\item
  \(\lim (a_n+b_n) = A+B\).
\item
  \(\lim (a_n b_n) = AB\).
\item
  \(\lim (c a_n) = cA\) للثابت \(c\).
\item
  إذا كان \(b_n \neq 0\) و\(B \neq 0\)، فحينئذٍ

  \[
  \lim \frac{a_n}{b_n} = \frac{A}{B}.
  \]
\end{itemize}

\subsubsection{النظرية: مبدأ
الضغط}\label{ux627ux644ux646ux638ux631ux64aux629-ux645ux628ux62fux623-ux627ux644ux636ux63aux637}

إذا كان \(a_n \leq b_n \leq c_n\) لجميع \(n\) الكبيرة، و

\[
\lim_{n\to\infty} a_n = \lim_{n\to\infty} c_n = L,
\]

ثم

\[\lim_{n\to\infty} b_n = L.
\]

Example:

\[
a_n = -\tfrac{1}{n}، \quad b_n = \tfrac{\sin n}{n}، \quad c_n = \tfrac{1}{n}.
\]

Since \(-\tfrac{1}{n} \leq \tfrac{\sin n}{n} \leq \tfrac{1}{n}\) and
both bounding sequences go to 0,

\[
\lim_{n\to\infty} \frac{\sin n}{n} = 0.
\]

\subsubsection{Why This Matters}\label{why-this-matters-1}

\begin{itemize}
\tightlist
\item
  Limits make rigorous the idea of sequences ``approaching'' a value.
\item
  Convergence of sequences underpins infinite series and continuity.
\item
  These concepts are essential in defining real numbers via limits.
\end{itemize}

\subsubsection{Exercises}\label{exercises-4}

\begin{enumerate}
\def\labelenumi{\arabic{enumi}.}
\tightlist
\item
  Find \(\lim_{n\to\infty} \frac{2n+1}{3n+4}\).
\item
  Determine if \(a_n = \sqrt{n+1} - \sqrt{n}\) converges.
\item
  Does \(a_n = \cos n\) converge? Why or why not?
\item
  Use the Squeeze Principle to show
  \(\lim_{n\to\infty} \frac{\sin n}{n} = 0\).
\item
  Prove that if \(\lim a_n = L\), then \(\lim |a_n| = |L|\).
\end{enumerate}

\section{Chapter 12. Infinite series}\label{chapter-12.-infinite-series}

\subsection{12.1 Series and Convergence}\label{series-and-convergence}

A series is the sum of the terms of a sequence. Instead of just listing
numbers, we add them together and study whether the infinite sum
approaches a finite value.

\subsubsection{Definition}\label{definition}

Given a sequence \((a_n)\), the corresponding series is

\[
\sum_{n=1}^\infty a_n = a_1 + a_2 + a_3 + \dots
\]

We define the nth partial sum as

\[
S_n = \sum_{k=1}^n a_k.
\]

If the sequence \((S_n)\) converges to a finite limit \(S\), then the
series converges and

\[
\sum_{n=1}^\infty a_n = S.
\]

If \((S_n)\) diverges, then the series diverges.

\subsubsection{Examples}\label{examples-2}

\begin{enumerate}
\def\labelenumi{\arabic{enumi}.}
\tightlist
\item
  Geometric series
\end{enumerate}

\[
\sum_{n=0}^\infty ar^n = \frac{a}{1-r}, \quad |r| < 1.
\]

Example:

\[
1 + \tfrac{1}{2} + \tfrac{1}{4} + \tfrac{1}{8} + \dots = 2.
\]

\begin{enumerate}
\def\labelenumi{\arabic{enumi}.}
\setcounter{enumi}{1}
\tightlist
\item
  Harmonic series
\end{enumerate}

\[
\sum_{n=1}^\infty \frac{1}{n}.
\]

This series diverges, even though the terms go to 0.

\begin{enumerate}
\def\labelenumi{\arabic{enumi}.}
\setcounter{enumi}{2}
\tightlist
\item
  p-series
\end{enumerate}

\[
\sum_{n=1}^\infty \frac{1}{n^p}.
\]

\begin{itemize}
\tightlist
\item
  يتقارب إذا كان \(p > 1\).
\item
  يتباعد إذا كان \(p \leq 1\).\#\#\# شرط ضروري للتقارب
\end{itemize}

إذا تقارب \(\sum a_n\)، فمن الضروري

\[
\lim_{n\to\infty} a_n = 0.
\]

إذا كان \(\lim a_n \neq 0\)، فإن المتسلسلة تتباعد. لكن العكس غير صحيح:
\(\lim a_n = 0\) لا يضمن التقارب (على سبيل المثال، المتسلسلة التوافقية).

\subsubsection{لماذا هذا
مهم}\label{ux644ux645ux627ux630ux627-ux647ux630ux627-ux645ux647ux645-27}

\begin{itemize}
\tightlist
\item
  توسع السلسلة الإضافة المحدودة للعمليات اللانهائية.
\item
  تستخدم المتسلسلات المتقاربة لتقريب الدوال وحساب المناطق ونمذجة
  العمليات الفيزيائية.
\item
  دراسة المتسلسلة تؤدي إلى اختبارات تقارب قوية.
\end{itemize}

\subsubsection{تمارين}\label{ux62aux645ux627ux631ux64aux646-41}

\begin{enumerate}
\def\labelenumi{\arabic{enumi}.}
\tightlist
\item
  حدد ما إذا كان \(\sum_{n=1}^\infty \frac{2}{3^n}\) متقاربًا، ثم ابحث عن
  مجموعه.
\item
  أظهر أن \(\sum_{n=1}^\infty \frac{1}{n^2}\) يتقارب.
\item
  هل يتقارب \(\sum_{n=1}^\infty \frac{1}{\sqrt{n}}\)؟
\item
  اكتب أول أربعة مجاميع جزئية من السلسلة
  \(\sum_{n=1}^\infty \frac{1}{2^n}\).
\item
  اشرح لماذا يعد \(\lim a_n = 0\) ضروريًا ولكنه غير كافٍ للتقارب.
\end{enumerate}

\subsection{12.2 اختبارات
التقارب}\label{ux627ux62eux62aux628ux627ux631ux627ux62a-ux627ux644ux62aux642ux627ux631ux628}

نظرًا لأن العديد من المتسلسلات لا يمكن جمعها بشكل مباشر، فقد طور علماء
الرياضيات اختبارات لتحديد ما إذا كانت المتسلسلة متقاربة أم متباعدة. هذه
الاختبارات هي أدوات لتحليل المبالغ اللانهائية.

\subsubsection{1. اختبار الفصل التاسع
للتباعد}\label{ux627ux62eux62aux628ux627ux631-ux627ux644ux641ux635ux644-ux627ux644ux62aux627ux633ux639-ux644ux644ux62aux628ux627ux639ux62f}

إذا

\[
\lim_{n\to\infty} a_n \neq 0 \quad \text{or does not exist},
\]

ثم

\[
\sum a_n
\]

يتباعد.

إذا كان \(\lim a_n = 0\)، فإن الاختبار غير حاسم.

\subsubsection{2. اختبار
المقارنة}\label{ux627ux62eux62aux628ux627ux631-ux627ux644ux645ux642ux627ux631ux646ux629}

لنفترض أن \(0 \leq a_n \leq b_n\) لجميع \(n\).

\begin{itemize}
\tightlist
\item
  إذا تقارب \(\sum b_n\)، فإن \(\sum a_n\) يتقارب أيضًا.
\item
  إذا تباعد \(\sum a_n\)، فإن \(\sum b_n\) يتباعد أيضًا.
\end{itemize}

\subsubsection{3. اختبار مقارنة
الحدود}\label{ux627ux62eux62aux628ux627ux631-ux645ux642ux627ux631ux646ux629-ux627ux644ux62dux62fux648ux62f}

إذا كان \(a_n, b_n > 0\) و

\[
\lim_{n\to\infty} \frac{a_n}{b_n} = c,
\]

حيث \(0 < c < \infty\)، ثم \(\sum a_n\) و\(\sum b_n\) إما يتقاربان أو
يتباعدان.

\subsubsection{4. اختبار
النسبة}\label{ux627ux62eux62aux628ux627ux631-ux627ux644ux646ux633ux628ux629}

بالنسبة إلى \(\sum a_n\)، قم بالحساب

\[
L = \lim_{n\to\infty} \left| \frac{a_{n+1}}{a_n} \right|.
\]- إذا كان \(L < 1\)، فإن المتسلسلة تتقارب تمامًا. - إذا كان \(L > 1\)
أو \(L = \infty\)، فإن المتسلسلة تتباعد. - إذا كان \(L = 1\)، فإن
الاختبار غير حاسم.

\subsubsection{5. اختبار
الجذر}\label{ux627ux62eux62aux628ux627ux631-ux627ux644ux62cux630ux631}

بالنسبة إلى \(\sum a_n\)، قم بالحساب

\[
L = \lim_{n\to\infty} \sqrt[n]{|a_n|}.
\]

\begin{itemize}
\tightlist
\item
  إذا كان \(L < 1\)، فإن المتسلسلة تتقارب تمامًا.
\item
  إذا كان \(L > 1\) فإن المتسلسلة تتباعد.
\item
  إذا كان \(L = 1\)، فإن الاختبار غير حاسم.
\end{itemize}

\subsubsection{6. اختبار المتسلسلات المتناوبة (اختبار
لايبنتز)}\label{ux627ux62eux62aux628ux627ux631-ux627ux644ux645ux62aux633ux644ux633ux644ux627ux62a-ux627ux644ux645ux62aux646ux627ux648ux628ux629-ux627ux62eux62aux628ux627ux631-ux644ux627ux64aux628ux646ux62aux632}

لسلسلة من النموذج

\[
\sum (-1)^n b_n \quad \text{or} \quad \sum (-1)^{n+1} b_n,
\]

إذا

\begin{enumerate}
\def\labelenumi{\arabic{enumi}.}
\tightlist
\item
  \(b_{n+1} \leq b_n\) (متناقص)، و
\item
  \(\lim_{n\to\infty} b_n = 0\),
\end{enumerate}

ثم تتقارب السلسلة.

\subsubsection{أمثلة}\label{ux623ux645ux62bux644ux629-25}

\begin{enumerate}
\def\labelenumi{\arabic{enumi}.}
\tightlist
\item
  \(\sum \frac{1}{n^2}\): اختبار المقارنة → يتقارب.
\item
  \(\sum \frac{1}{n}\): المتسلسلة التوافقية → المتباعدة.
\item
  \(\sum \frac{(-1)^n}{n}\): اختبار المتسلسلات المتناوبة ← التقارب.
\item
  \(\sum \frac{n!}{n^n}\): اختبار النسبة → يتقارب.
\item
  \(\sum \frac{2^n}{n}\): اختبار الجذر → يتباعد.
\end{enumerate}

\subsubsection{لماذا هذا
مهم}\label{ux644ux645ux627ux630ux627-ux647ux630ux627-ux645ux647ux645-28}

\begin{itemize}
\tightlist
\item
  تتيح لنا اختبارات التقارب تصنيف المتسلسلات دون الحاجة إلى مجاميع
  صريحة.
\item
  أنها توفر طرق منهجية للتعامل مع العمليات اللانهائية في حساب التفاضل
  والتكامل.
\item
  إنها مهمة لمواضيع لاحقة مثل متسلسلة القوى ومتسلسلة فورييه.
\end{itemize}

\subsubsection{تمارين}\label{ux62aux645ux627ux631ux64aux646-42}

\begin{enumerate}
\def\labelenumi{\arabic{enumi}.}
\tightlist
\item
  اختبار تقارب \(\sum \frac{1}{n^3}\).
\item
  استخدم اختبار النسبة لـ \(\sum \frac{3^n}{n!}\).
\item
  قم بتطبيق اختبار الجذر على \(\sum \left(\frac{1}{2}\right)^n\).
\item
  تحديد تقارب \(\sum \frac{(-1)^n}{\sqrt{n}}\).
\item
  استخدم اختبار مقارنة الحدود مع \(\frac{1}{n^2}\) لاختبار
  \(\sum \frac{1}{n^2+1}\).
\end{enumerate}

\subsection{12.3 التقارب المطلق مقابل التقارب
الشرطي}\label{ux627ux644ux62aux642ux627ux631ux628-ux627ux644ux645ux637ux644ux642-ux645ux642ux627ux628ux644-ux627ux644ux62aux642ux627ux631ux628-ux627ux644ux634ux631ux637ux64a}

لا تتصرف كل السلاسل بنفس الطريقة عندما تتناوب العلامات. ولمعالجة ذلك
نميز بين التقارب المطلق والتقارب المشروط.

\subsubsection{التقارب
المطلق}\label{ux627ux644ux62aux642ux627ux631ux628-ux627ux644ux645ux637ux644ux642}

المتسلسلة \(\sum a_n\) متقاربة تمامًا إذا

\[
\sum |a_n|
\]

يتقارب.النظرية: إذا كانت المتسلسلة متقاربة تقاربا مطلقا،
\hspace{0pt}\hspace{0pt}فإنها تتقارب أيضا.

مثال:

\[
\sum \frac{(-1)^n}{n^2}.
\]

هنا يتقارب \(\sum \left|\frac{(-1)^n}{n^2}\right| = \sum \frac{1}{n^2}\)
(سلسلة p، \(p=2\)). وبالتالي فإن المتسلسلة متقاربة تمامًا.

\subsubsection{التقارب
الشرطي}\label{ux627ux644ux62aux642ux627ux631ux628-ux627ux644ux634ux631ux637ux64a}

تكون المتسلسلة \(\sum a_n\) متقاربة شرطيًا إذا تقاربت، ولكن ليس بشكل
مطلق.

مثال:

\[
\sum \frac{(-1)^n}{n}.
\]

\begin{itemize}
\tightlist
\item
  اختبار المتسلسلات المتناوبة → المتقاربة.
\item
  لكن \(\sum \left|\frac{(-1)^n}{n}\right| = \sum \frac{1}{n}\) يتباعد
  (السلسلة التوافقية). وبالتالي فإن المتسلسلة متقاربة شرطيًا.
\end{itemize}

\subsubsection{نظرية إعادة
الترتيب}\label{ux646ux638ux631ux64aux629-ux625ux639ux627ux62fux629-ux627ux644ux62aux631ux62aux64aux628}

بالنسبة للمتسلسلات المتقاربة بشكل مشروط، فإن إعادة ترتيب الحدود يمكن أن
تغير المجموع - بل وتجعله يتباعد أو يتقارب إلى قيمة مختلفة.

تظهر هذه النتيجة المدهشة الطبيعة الدقيقة للتقارب الشرطي.

\subsubsection{لماذا هذا
مهم}\label{ux644ux645ux627ux630ux627-ux647ux630ux627-ux645ux647ux645-29}

\begin{itemize}
\tightlist
\item
  التقارب المطلق أقوى ويضمن الاستقرار.
\item
  التقارب الشرطي يسلط الضوء على أهمية النظام في المبالغ اللانهائية.
\item
  العديد من المتسلسلات المتناوبة التي نواجهها عمليًا تكون متقاربة بشكل
  مشروط فقط.
\end{itemize}

\subsubsection{تمارين}\label{ux62aux645ux627ux631ux64aux646-43}

\begin{enumerate}
\def\labelenumi{\arabic{enumi}.}
\tightlist
\item
  أظهر أن \(\sum \frac{(-1)^n}{n^3}\) يتقارب تمامًا.
\item
  أظهر أن \(\sum \frac{(-1)^n}{n}\) متقارب بشكل مشروط.
\item
  اختبار \(\sum \frac{(-1)^n}{\sqrt{n}}\) للتقارب المطلق والشرطي.
\item
  اشرح لماذا التقارب المطلق يعني التقارب، ولكن العكس ليس صحيحا.
\item
  ابحث ولخص نظرية إعادة ترتيب ريمان بكلماتك الخاصة.
\end{enumerate}

\section{الفصل 13. سلسلة القوى
والتوسعات}\label{ux627ux644ux641ux635ux644-13.-ux633ux644ux633ux644ux629-ux627ux644ux642ux648ux649-ux648ux627ux644ux62aux648ux633ux639ux627ux62a}

\#\#13.1 سلسلة الطاقة

متسلسلة القوى هي متسلسلة لا نهائية حيث يتضمن كل حد قوة المتغير. تعد
متسلسلة القوى عنصرًا أساسيًا في حساب التفاضل والتكامل لأنها تتيح لنا تمثيل
الدوال كمتعددات حدود لا نهائية.

\subsubsection{النموذج
العام}\label{ux627ux644ux646ux645ux648ux630ux62c-ux627ux644ux639ux627ux645}

سلسلة الطاقة المتمركزة في \(a\) لها الشكل

\[\sum_{n=0}^\infty c_n (x-a)^n,
\]

where \(c_n\) are constants called the coefficients.

\begin{itemize}
\item
  If \(a=0\), the series is centered at the origin:

  \[
  \sum_{n=0}^\infty c_n x^n.
  \]
\end{itemize}

\subsubsection{Examples}\label{examples-3}

\begin{enumerate}
\def\labelenumi{\arabic{enumi}.}
\tightlist
\item
  Geometric series
\end{enumerate}

\[
\sum_{n=0}^\infty x^n = \frac{1}{1-x}, \quad |x|<1.
\]

\begin{enumerate}
\def\labelenumi{\arabic{enumi}.}
\setcounter{enumi}{1}
\tightlist
\item
  Exponential function
\end{enumerate}

\[
e^x = \sum_{n=0}^\infty \frac{x^n}{n!}.
\]

\begin{enumerate}
\def\labelenumi{\arabic{enumi}.}
\setcounter{enumi}{2}
\tightlist
\item
  Sine and cosine
\end{enumerate}

\[
\sin x = \sum_{n=0}^\infty (-1)^n \frac{x^{2n+1}}{(2n+1)!}, \quad  
\cos x = \sum_{n=0}^\infty (-1)^n \frac{x^{2n}}{(2n)!}.
\]

\subsubsection{Interval of Convergence}\label{interval-of-convergence}

For each power series, there exists a radius of convergence \(R\) such
that:

\begin{itemize}
\tightlist
\item
  The series converges if \(|x-a| < R\).
\item
  The series diverges if \(|x-a| > R\).
\item
  At \(|x-a| = R\), convergence must be checked separately.
\end{itemize}

\subsubsection{Why This Matters}\label{why-this-matters-2}

\begin{itemize}
\tightlist
\item
  Power series allow us to approximate functions by polynomials.
\item
  They connect calculus with analysis and differential equations.
\item
  Many special functions in mathematics and physics are defined by their
  power series.
\end{itemize}

\subsubsection{Exercises}\label{exercises-5}

\begin{enumerate}
\def\labelenumi{\arabic{enumi}.}
\tightlist
\item
  Write the power series for \(\sum_{n=0}^\infty \frac{(x-2)^n}{n!}\).
\item
  Find the first four terms of the power series for \(e^x\).
\item
  Express \(\frac{1}{1+x}\) as a power series centered at 0.
\item
  Determine whether the series \(\sum_{n=0}^\infty n! x^n\) converges at
  \(x=0.1\).
\item
  Explain why power series are sometimes called ``infinite
  polynomials.''
\end{enumerate}

\subsection{13.2 Radius of Convergence}\label{radius-of-convergence}

Every power series converges for some values of \(x\) and diverges for
others. The boundary between these two behaviors is described by the
radius of convergence.

\subsubsection{Definition}\label{definition-1}

For a power series

\[
\sum_{n=0}^\infty c_n (x-a)^n,
\]

يوجد رقم \(R \geq 0\) (ربما لا نهائي) مثل:

\begin{itemize}
\tightlist
\item
  تتقارب المتسلسلة بشكل مطلق إذا كان \(|x-a| < R\).
\item
  تتباعد السلسلة إذا كان \(|x-a| > R\).- عند \(|x-a| = R\)، يجب التحقق
  من التقارب بشكل منفصل.
\end{itemize}

هذا الرقم \(R\) يسمى نصف قطر التقارب.

\subsubsection{إيجاد نصف قطر
التقارب}\label{ux625ux64aux62cux627ux62f-ux646ux635ux641-ux642ux637ux631-ux627ux644ux62aux642ux627ux631ux628}

طريقتان شائعتان:

\begin{enumerate}
\def\labelenumi{\arabic{enumi}.}
\tightlist
\item
  اختبار النسبة
\end{enumerate}

\[
R = \lim_{n\to\infty} \left| \frac{c_n}{c_{n+1}} \right|.
\]

\begin{enumerate}
\def\labelenumi{\arabic{enumi}.}
\setcounter{enumi}{1}
\tightlist
\item
  اختبار الجذر
\end{enumerate}

\[
R = \frac{1}{\limsup_{n\to\infty} \sqrt[n]{|c_n|}}.
\]

\subsubsection{أمثلة}\label{ux623ux645ux62bux644ux629-26}

\begin{enumerate}
\def\labelenumi{\arabic{enumi}.}
\tightlist
\item
  السلسلة:
\end{enumerate}

\[
\sum_{n=0}^\infty \frac{x^n}{n!}.
\]

باستخدام اختبار النسبة:

\[
\lim_{n\to\infty} \frac{1/(n!)}{1/((n+1)!)} = \infty.
\]

إذن \(R = \infty\) (يتقارب مع كل \(x\) الحقيقي).

\begin{enumerate}
\def\labelenumi{\arabic{enumi}.}
\setcounter{enumi}{1}
\tightlist
\item
  السلسلة:
\end{enumerate}

\[
\sum_{n=0}^\infty x^n.
\]

هنا \(c_n = 1\).

\[
R = 1.
\]

يتقارب مع \(|x| < 1\).

\begin{enumerate}
\def\labelenumi{\arabic{enumi}.}
\setcounter{enumi}{2}
\tightlist
\item
  السلسلة:
\end{enumerate}

\[
\sum_{n=1}^\infty \frac{x^n}{n}.
\]

اختبار النسبة:

\[
\lim_{n\to\infty} \left|\frac{(x^{n+1}/(n+1))}{(x^n/n)}\right| = |x|.
\]

إذن \(R = 1\). يتقارب مع \(|x| < 1\)، ويتباعد مع \(|x| > 1\). في
\(x=\pm 1\)، اختبر بشكل منفصل.

\subsubsection{فترة
التقارب}\label{ux641ux62aux631ux629-ux627ux644ux62aux642ux627ux631ux628}

مجموعة قيم \(x\) التي تتقارب فيها المتسلسلة تسمى فترة التقارب.

\begin{itemize}
\tightlist
\item
  يتم التركيز دائمًا على \(a\).
\item
  يمتد \(R\) وحدة في كلا الاتجاهين.
\item
  يجب التحقق من نقاط النهاية \(x=a\pm R\) بشكل فردي.
\end{itemize}

\subsubsection{لماذا هذا
مهم}\label{ux644ux645ux627ux630ux627-ux647ux630ux627-ux645ux647ux645-30}

\begin{itemize}
\tightlist
\item
  نصف قطر التقارب يخبرنا أين تتصرف متسلسلة القوى مثل الدوال.
\item
  ضروري لاستخدام توسعات سلسلة تايلور في الممارسة العملية.
\item
  تحديد مجال صلاحية حلول المتسلسلة في الفيزياء والهندسة.
\end{itemize}

\subsubsection{تمارين}\label{ux62aux645ux627ux631ux64aux646-44}

\begin{enumerate}
\def\labelenumi{\arabic{enumi}.}
\tightlist
\item
  أوجد نصف قطر التقارب لـ \(\sum_{n=0}^\infty \frac{(x-3)^n}{n!}\).
\item
  احسب نصف قطر تقارب \(\sum_{n=1}^\infty \frac{x^n}{n^2}\).
\item
  استخدم اختبار النسبة للعثور على \(R\) لـ \(\sum_{n=0}^\infty n!x^n\).
\item
  تحديد فترة التقارب لـ \(\sum_{n=1}^\infty \frac{(x+1)^n}{n}\).
\item
  اشرح سبب تقارب المتسلسلة الأسية عند جميع \(x\)، بينما تتقارب المتسلسلة
  الهندسية عند \(|x|<1\) فقط.\#\# 13.3 سلسلة تايلور وماكلورين
\end{enumerate}

تصبح متسلسلة الطاقة قوية بشكل خاص عندما يتم استخدامها لتمثيل وظائف
مألوفة. ويتم ذلك من خلال متسلسلة تايلور، والحالة الخاصة المتمركزة عند 0
تسمى متسلسلة ماكلورين.

\subsubsection{سلسلة
تايلور}\label{ux633ux644ux633ux644ux629-ux62aux627ux64aux644ux648ux631}

إذا كانت الدالة \(f(x)\) قابلة للاشتقاق بشكل لا نهائي عند \(x=a\)، فإن
متسلسلة تايلور الخاصة بها حول \(a\) هي

\[
f(x) = \sum_{n=0}^\infty \frac{f^{(n)}(a)}{n!}(x-a)^n.
\]

هنا يشير \(f^{(n)}(a)\) إلى المشتق \(n\) من \(f\) عند \(a\).

\subsubsection{سلسلة
ماكلورين}\label{ux633ux644ux633ux644ux629-ux645ux627ux643ux644ux648ux631ux64aux646}

سلسلة تايلور تتمحور حول \(a=0\):

\[
f(x) = \sum_{n=0}^\infty \frac{f^{(n)}(0)}{n!} x^n.
\]

\subsubsection{أمثلة}\label{ux623ux645ux62bux644ux629-27}

\begin{enumerate}
\def\labelenumi{\arabic{enumi}.}
\tightlist
\item
  الدالة الأسية
\end{enumerate}

\[
e^x = 1 + x + \frac{x^2}{2!} + \frac{x^3}{3!} + \cdots
\]

\begin{enumerate}
\def\labelenumi{\arabic{enumi}.}
\setcounter{enumi}{1}
\tightlist
\item
  الجيب وجيب التمام
\end{enumerate}

\[
\sin x = x - \frac{x^3}{3!} + \frac{x^5}{5!} - \cdots
\]

\[
\cos x = 1 - \frac{x^2}{2!} + \frac{x^4}{4!} - \cdots
\]

\begin{enumerate}
\def\labelenumi{\arabic{enumi}.}
\setcounter{enumi}{2}
\tightlist
\item
  اللوغاريتم الطبيعي (لـ \(|x|<1\))
\end{enumerate}

\[
\ln(1+x) = x - \frac{x^2}{2} + \frac{x^3}{3} - \frac{x^4}{4} + \cdots
\]

\subsubsection{تقريب تايلور متعدد
الحدود}\label{ux62aux642ux631ux64aux628-ux62aux627ux64aux644ux648ux631-ux645ux62aux639ux62fux62f-ux627ux644ux62dux62fux648ux62f}

المجموع المحدود لحدود \(n\) الأولى هو كثيرة حدود تايلور من الدرجة \(n\):

\[
P_n(x) = \sum_{k=0}^n \frac{f^{(k)}(a)}{k!}(x-a)^k.
\]

يقترب كثير الحدود هذا من \(f(x)\) بالقرب من \(x=a\).

\subsubsection{الباقي (مصطلح
الخطأ)}\label{ux627ux644ux628ux627ux642ux64a-ux645ux635ux637ux644ux62d-ux627ux644ux62eux637ux623}

الفرق بين الدالة ومتعددة حدود تايلور هو الباقي:

\[
R_n(x) = f(x) - P_n(x).
\]

أحد النماذج (نموذج لاغرانج) هو

\[
R_n(x) = \frac{f^{(n+1)}(c)}{(n+1)!}(x-a)^{n+1},
\]

لبعض \(c\) بين \(a\) و\(x\).

\subsubsection{لماذا هذا
مهم}\label{ux644ux645ux627ux630ux627-ux647ux630ux627-ux645ux647ux645-31}

\begin{itemize}
\tightlist
\item
  توفر متسلسلة تايلور تقريبيات متعددة الحدود للدوال المعقدة.
\item
  إنها ضرورية في التحليل العددي والفيزياء والهندسة.
\item
  تعطي توسعات متسلسلة ماكلورين صيغًا بسيطة للدوال الأسية والمثلثية
  واللوغاريتمية.
\end{itemize}

\subsubsection{تمارين}\label{ux62aux645ux627ux631ux64aux646-45}

\begin{enumerate}
\def\labelenumi{\arabic{enumi}.}
\tightlist
\item
  ابحث عن سلسلة Maclaurin لـ \(f(x)=\cosh x = \tfrac{e^x+e^{-x}}{2}\).2.
  اكتب سلسلة تايلور لـ \(f(x)=e^x\) المتمركزة في \(a=2\).
\item
  احسب كثيرة حدود تايلور من الدرجة 3 لـ \(f(x)=\ln(1+x)\) في \(a=0\).
\item
  استخدم سلسلة Maclaurin لـ \(\sin x\) لتقريب \(\sin(0.1)\).
\item
  اشرح لماذا توفر متسلسلة تايلور في كثير من الأحيان تقديرات محلية جيدة
  ولكنها قد تتباين بالنسبة إلى \(|x|\) الكبيرة.
\end{enumerate}

\subsection{13.4 تطبيقات سلسلة
تايلور}\label{ux62aux637ux628ux64aux642ux627ux62a-ux633ux644ux633ux644ux629-ux62aux627ux64aux644ux648ux631}

متسلسلة تايلور ليست أدوات نظرية فحسب، بل تُستخدم لتقريب الدوال وحل
المعادلات وتحليل الأنظمة الفيزيائية. وتشمل تطبيقاتها الرياضيات والعلوم
والهندسة.

\subsubsection{تقريب
الوظيفة}\label{ux62aux642ux631ux64aux628-ux627ux644ux648ux638ux64aux641ux629}

يمكن تقريب الوظائف المعقدة عن طريق كثيرات الحدود بالقرب من نقطة ما.

مثال: \(e^x\) تقريبي بالقرب من \(x=0\) باستخدام متعددة حدود Maclaurin من
الدرجة 3:

\[
P_3(x) = 1 + x + \tfrac{x^2}{2} + \tfrac{x^3}{6}.
\]

بالنسبة إلى \(x\) الصغيرة، يوفر هذا تقديرات دقيقة لـ \(e^x\).

\subsubsection{الطرق
العددية}\label{ux627ux644ux637ux631ux642-ux627ux644ux639ux62fux62fux64aux629}

توفر سلسلة تايلور الأساس للخوارزميات العددية:

\begin{itemize}
\tightlist
\item
  تقريب الجذور التربيعية واللوغاريتمات والقيم المثلثية.
\item
  تقدير الخطأ خلال المدة المتبقية.
\item
  تستخدم في الطرق التكرارية مثل طريقة نيوتن (حيث تأتي الخطية المحلية من
  متسلسلة تايلور).
\end{itemize}

\subsubsection{حل المعادلات
التفاضلية}\label{ux62dux644-ux627ux644ux645ux639ux627ux62fux644ux627ux62a-ux627ux644ux62aux641ux627ux636ux644ux64aux629}

العديد من المعادلات التفاضلية لها حلول معبر عنها بمتسلسلة تايلور (أو
القوة).

مثال: حل \(y'' + y = 0\) مع \(y(0)=0, y'(0)=1\) هو \(\sin x\)، والذي
ينشأ بشكل طبيعي من سلسلة Maclaurin الخاصة بها.

\subsubsection{الفيزياء
والهندسة}\label{ux627ux644ux641ux64aux632ux64aux627ux621-ux648ux627ux644ux647ux646ux62fux633ux629}

\begin{itemize}
\item
  تقريب الزاوية الصغيرة:

  \[
  \sin x \approx x, \quad \cos x \approx 1 - \tfrac{x^2}{2}, \quad |x| \ll 1.
  \]

  تستخدم في حركة البندول، والبصريات، وميكانيكا الموجات.
\item
  النسبية وميكانيكا الكم: توسعات تايلور تبسط التعبيرات غير الخطية
  للاستخدام العملي.- تقريب وظائف الطاقة: في الميكانيكا، يتم توسيع وظائف
  الطاقة المحتملة بالقرب من نقاط التوازن.
\end{itemize}

\subsubsection{الاحتمالات
والإحصائيات}\label{ux627ux644ux627ux62dux62aux645ux627ux644ux627ux62a-ux648ux627ux644ux625ux62dux635ux627ux626ux64aux627ux62a}

\begin{itemize}
\tightlist
\item
  دوال توليد العزوم والدوال المميزة تستخدم متسلسلة القوى.
\item
  تقريب التوزيعات الاحتمالية (على سبيل المثال، التقريب الطبيعي إلى
  الحدين) يستخدم توسعات تايلور.
\end{itemize}

\subsubsection{لماذا هذا
مهم}\label{ux644ux645ux627ux630ux627-ux647ux630ux627-ux645ux647ux645-32}

\begin{itemize}
\tightlist
\item
  توفر متسلسلة تايلور جسرًا بين الصيغ الدقيقة والحسابات العملية.
\item
  إنها تسمح لنا بتقليص المشكلات المعقدة إلى تقديرات تقريبية متعددة
  الحدود يمكن التحكم فيها.
\item
  التطبيقات تجعلها من أهم الأدوات في الرياضيات التطبيقية.
\end{itemize}

\subsubsection{تمارين}\label{ux62aux645ux627ux631ux64aux646-46}

\begin{enumerate}
\def\labelenumi{\arabic{enumi}.}
\tightlist
\item
  استخدم سلسلة Maclaurin لـ \(e^x\) لتقريب \(e^{0.1}\) بما يصل إلى أربع
  منازل عشرية.
\item
  قم بتطبيق تقريب الزاوية الصغيرة لتقدير \(\sin(5^\circ)\).
\item
  حل المعادلة التفاضلية \(y'' = -y\) باستخدام طريقة متسلسلة القوى.
\item
  قم بتوسيع \(\ln(1+x)\) حتى الدرجة الرابعة واستخدمه لتقريب
  \(\ln(1.1)\).
\item
  اشرح لماذا تعتبر التقريبات متعددة الحدود مفيدة بشكل خاص لأجهزة
  الكمبيوتر والآلات الحاسبة.
\end{enumerate}

\#الملاحق

\subsection{الملحق أ. أساسيات ما قبل حساب التفاضل
والتكامل}\label{ux627ux644ux645ux644ux62dux642-ux623.-ux623ux633ux627ux633ux64aux627ux62a-ux645ux627-ux642ux628ux644-ux62dux633ux627ux628-ux627ux644ux62aux641ux627ux636ux644-ux648ux627ux644ux62aux643ux627ux645ux644}

\subsubsection{أ.1 تنشيطية
للجبر}\label{ux623.1-ux62aux646ux634ux64aux637ux64aux629-ux644ux644ux62cux628ux631}

قبل الغوص في حساب التفاضل والتكامل، من المفيد مراجعة بعض مهارات الجبر
التي ستظهر مرارًا وتكرارًا. هذه هي ``الأدوات'' التي ستحتاجها لمعالجة
التعبيرات وحل المعادلات وتبسيط النتائج.

\paragraph{الأسس
والقوى}\label{ux627ux644ux623ux633ux633-ux648ux627ux644ux642ux648ux649}

\begin{itemize}
\item
  القواعد الأساسية:

  \[
  a^m \cdot a^n = a^{m+n}, \quad \frac{a^m}{a^n} = a^{m-n}, \quad (a^m)^n = a^{mn}.
  \]
\item
  الأسس السلبية:

  \[
  a^{-n} = \frac{1}{a^n}, \quad a \neq 0.
  \]
\item
  الأسس الكسرية:

  \[
  a^{1/n} = \sqrt[n]{a}, \quad a^{m/n} = \sqrt[n]{a^m}.
  \]
\end{itemize}

\paragraph{التخصيم}\label{ux627ux644ux62aux62eux635ux64aux645}

التخصيم يبسط التعبيرات ويساعد في حل المعادلات.

\begin{enumerate}
\def\labelenumi{\arabic{enumi}.}
\item
  العامل المشترك:

  \[
  6x^2+9x = 3x(2x+3).
  \]
\item
  اختلاف المربعات:

  \[أ^2-ب^2 = (أ-ب)(أ+ب).
  \]
\item
  Quadratic trinomials:

  \[
  س^2+5س+6 = (س+2)(س+3).
  \]
\end{enumerate}

\paragraph{Polynomials}\label{polynomials}

\begin{itemize}
\tightlist
\item
  Standard form: \(P(x) = a_nx^n + a_{n-1}x^{n-1} + \cdots + a_0\).
\item
  Degree: the largest power of \(x\).
\item
  Long division and synthetic division are useful for simplifying
  rational functions.
\end{itemize}

\paragraph{Rational Expressions}\label{rational-expressions}

Simplify by factoring numerator and denominator:

\[
\frac{x^2-1}{x^2-2x+1} = \frac{(x-1)(x+1)}{(x-1)^2} = \frac{x+1}{x-1}, \quad x \neq 1.
\]

\paragraph{Logarithms}\label{logarithms}

\begin{itemize}
\item
  Definition: \(\log_a b = c\) means \(a^c = b\).
\item
  Common bases: natural log (\(\ln x = \log_e x\)) and base 10
  (\(\log x\)).
\item
  Rules:

  \[
  \log(ab) = \log a + \log b, \quad \log\left(\frac{a}{b}\right) = \log a - \log b, \quad \log(a^n) = n\log a.
  \]
\end{itemize}

\paragraph{Equations}\label{equations}

\begin{itemize}
\item
  Linear: solve \(ax+b=0\) → \(x=-b/a\).
\item
  Quadratic: \(ax^2+bx+c=0\) has solutions

  \[
  x=\frac{-b\pm \sqrt{b^2-4ac}}{2a}.
  \]
\item
  Exponential: \(e^x = k\) → \(x = \ln k\).
\end{itemize}

\subsubsection{A.2 Trigonometry Basics}\label{a.2-trigonometry-basics}

Trigonometry provides the language of angles and periodic phenomena.
Since calculus often deals with oscillations, motion, and waves, a solid
grasp of trigonometric functions and their properties is essential.

\paragraph{The Unit Circle}\label{the-unit-circle}

\begin{itemize}
\item
  Defined as the circle of radius 1 centered at the origin in the
  coordinate plane.
\item
  For an angle \(\theta\) measured from the positive \(x\)-axis:

  \[
  (\كوس \ثيتا، \سين \ثيتا)
  \]

  يعطي إحداثيات النقطة على الدائرة.
\end{itemize}

القيم الخاصة:

\begin{longtable}[]{@{}
  >{\raggedright\arraybackslash}p{(\linewidth - 6\tabcolsep) * \real{0.3333}}
  >{\raggedright\arraybackslash}p{(\linewidth - 6\tabcolsep) * \real{0.1667}}
  >{\raggedright\arraybackslash}p{(\linewidth - 6\tabcolsep) * \real{0.1667}}
  >{\raggedright\arraybackslash}p{(\linewidth - 6\tabcolsep) * \real{0.3333}}@{}}
\toprule\noalign{}
\begin{minipage}[b]{\linewidth}\raggedright
\(\theta\)
\end{minipage} & \begin{minipage}[b]{\linewidth}\raggedright
\(\sin \theta\)
\end{minipage} & \begin{minipage}[b]{\linewidth}\raggedright
\(\cos \theta\)
\end{minipage} & \begin{minipage}[b]{\linewidth}\raggedright
\(\tan \theta = \frac{\sin \theta}{\cos \theta}\)
\end{minipage} \\
\midrule\noalign{}
\endhead
\bottomrule\noalign{}
\endlastfoot
\(0\) & 0 & 1 & 0 \\
\(\pi/6\) & 1/2 & \(\sqrt{3}/2\) & \(1/\sqrt{3}\) \\
\(\pi/3\) & \(\sqrt{3}/2\) & 1/2 & \(\sqrt{3}\) \\
\(\pi/2\) & 1 & 0 & غير محدد \\
\end{longtable}

\paragraph{الهويات
الأساسية}\label{ux627ux644ux647ux648ux64aux627ux62a-ux627ux644ux623ux633ux627ux633ux64aux629}

\begin{enumerate}
\def\labelenumi{\arabic{enumi}.}
\tightlist
\item
  هوية فيثاغورس
\end{enumerate}

\[
\sin^2\theta + \cos^2\theta = 1.
\]

\begin{enumerate}
\def\labelenumi{\arabic{enumi}.}
\setcounter{enumi}{1}
\tightlist
\item
  الهويات الحاصلة
\end{enumerate}

\[
\tan\theta = \frac{\sin\theta}{\cos\theta}, \quad \cot\theta = \frac{\cos\theta}{\sin\theta}.
\]

\begin{enumerate}
\def\labelenumi{\arabic{enumi}.}
\setcounter{enumi}{2}
\tightlist
\item
  الهويات المتبادلة
\end{enumerate}

\[
\sec\theta = \frac{1}{\cos\theta}, \quad \csc\theta = \frac{1}{\sin\theta}.
\]

\paragraph{صيغ إضافة
الزوايا}\label{ux635ux64aux63a-ux625ux636ux627ux641ux629-ux627ux644ux632ux648ux627ux64aux627}

\[
\sin(\alpha+\beta) = \sin\alpha\cos\beta + \cos\alpha\sin\beta,
\]

\[
\cos(\alpha+\beta) = \cos\alpha\cos\beta - \sin\alpha\sin\beta.
\]

حالات خاصة:

\begin{itemize}
\item
  زاوية مزدوجة:

  \[
  \sin(2\theta) = 2\sin\theta\cos\theta, \quad
  \cos(2\theta) = \cos^2\theta - \sin^2\theta.
  \]
\end{itemize}

\paragraph{الرسوم
البيانية}\label{ux627ux644ux631ux633ux648ux645-ux627ux644ux628ux64aux627ux646ux64aux629}

\begin{itemize}
\tightlist
\item
  \(\sin x\): موجة تبدأ عند 0، سعة 1، الفترة \(2\pi\).
\item
  \(\cos x\): موجة تبدأ عند 1، سعة 1، الفترة \(2\pi\).
\item
  \(\tan x\): يتكرر كل \(\pi\)، غير محدد عند المضاعفات الفردية لـ
  \(\pi/2\).
\end{itemize}

\subsubsection{أ.3 الهندسة
الإحداثية}\label{ux623.3-ux627ux644ux647ux646ux62fux633ux629-ux627ux644ux625ux62dux62fux627ux62bux64aux629}

تربط الهندسة الإحداثية بين الجبر والهندسة من خلال وصف الكائنات الهندسية
(الخطوط والدوائر والمنحنيات) باستخدام المعادلات. يعتمد حساب التفاضل
والتكامل بشكل كبير على هذا الإطار لرسم الوظائف وإيجاد المنحدرات وتحليل
المنحنيات.

\paragraph{الطائرة
الديكارتية}\label{ux627ux644ux637ux627ux626ux631ux629-ux627ux644ux62fux64aux643ux627ux631ux62aux64aux629}

\begin{itemize}
\item
  يتم تمثيل النقطة بالإحداثيات \((x,y)\).
\item
  المسافة بين النقطتين \((x_1,y_1)\) و \((x_2,y_2)\):

  \[
  d = \sqrt{(x_2-x_1)^2 + (y_2-y_1)^2}.
  \]
\item
  منتصف القطعة المستقيمة:

  \[
  M = \left(\frac{x_1+x_2}{2}, \frac{y_1+y_2}{2}\right).
  \]
\end{itemize}

\paragraph{خطوط}\label{ux62eux637ux648ux637}

\begin{enumerate}
\def\labelenumi{\arabic{enumi}.}
\item
  صيغة المنحدر

  \[
  m = \frac{y_2-y_1}{x_2-x_1}.
  \]
\item
  معادلة الخط

  \begin{itemize}
  \item
    شكل نقطة الميل:

    \[ص-y_1 = م(س-x_1).
    \]
  \item
    Slope-intercept form:

    \[
    ص = مكس+ب.
    \]
  \end{itemize}
\item
  Parallel and perpendicular lines

  \begin{itemize}
  \tightlist
  \item
    Parallel lines: same slope.
  \item
    Perpendicular lines: slopes satisfy \(m_1m_2 = -1\).
  \end{itemize}
\end{enumerate}

\paragraph{Circles}\label{circles}

Equation of a circle with center \((h,k)\) and radius \(r\):

\[
(س-ح)^2+(ص-ك)^2 = ص^2.
\]

Special case: unit circle centered at origin:

\[
س^2+ص^2=1.
\]

\paragraph{Conic Sections}\label{conic-sections}

\begin{enumerate}
\def\labelenumi{\arabic{enumi}.}
\item
  Parabola:

  \begin{itemize}
  \item
    Standard form (opening up/down):

    \[
    ص = الفأس^2+بكس+ج.
    \]
  \end{itemize}
\item
  Ellipse (centered at origin):

  \[
  \frac{x^2}{a^2}+\frac{y^2}{b^2}=1.
  \]
\item
  Hyperbola (centered at origin):

  \[
  \frac{x^2}{a^2}-\frac{y^2}{b^2}=1.
  \]
\end{enumerate}

\subsection{Appendix B. Key Formulas and
Tables}\label{appendix-b.-key-formulas-and-tables}

\subsubsection{B.1 Derivative Table}\label{b.1-derivative-table}

Derivatives measure rates of change and slopes of functions. Having a
quick-reference table helps learners avoid re-deriving formulas each
time.

\paragraph{Basic Rules}\label{basic-rules}

\begin{enumerate}
\def\labelenumi{\arabic{enumi}.}
\tightlist
\item
  Constant rule
\end{enumerate}

\[
\frac{d}{dx[c] = 0
\]

\begin{enumerate}
\def\labelenumi{\arabic{enumi}.}
\setcounter{enumi}{1}
\tightlist
\item
  Power rule
\end{enumerate}

\[
\frac{d}{dx[x^n] = nx^{n-1}, \quad (n \in \mathbb{R})
\]

\begin{enumerate}
\def\labelenumi{\arabic{enumi}.}
\setcounter{enumi}{2}
\tightlist
\item
  Constant multiple rule
\end{enumerate}

\[
\frac{d}{dx}[c f(x)] = c f'(x)
\]

\begin{enumerate}
\def\labelenumi{\arabic{enumi}.}
\setcounter{enumi}{3}
\tightlist
\item
  Sum and difference rule
\end{enumerate}

\[
\frac{d}{dx}[f(x)\pm g(x)] = f'(x)\pm g'(x)
\]

\paragraph{Trigonometric Functions}\label{trigonometric-functions}

\[
\frac{d}{dx}[\sin x] = \cos x
\]

\[
\frac{d}{dx}[\cos x] = -\sin x
\]

\[
\frac{d}{dx}[\tan x] = \sec^2 x, \quad x \neq \tfrac{\pi}{2}+k\pi
\]

\[
\frac{d}{dx}[\cot x] = -\csc^2 x
\]

\[
\frac{d}{dx}[\sec x] = \sec x \tan x
\]

\[
\frac{d}{dx}[\csc x] = -\csc x \cot x
\]

\paragraph{Exponential and Logarithmic
Functions}\label{exponential-and-logarithmic-functions}

\[
\frac{d}{dx[e^x] = e^x
\]

\[
\frac{d}{dx[a^x] = a^x \ln a, \quad a>0, a\neq 1
\]

\[
\frac{d}{dx}[\ln x] = \frac{1}{x}, \quad x>0
\]

\[
\frac{d}{dx}[\log_a x] = \frac{1}{x\ln a}, \quad a>0, a\neq 1
\]

\paragraph{Inverse Trigonometric
Functions}\label{inverse-trigonometric-functions}

\[\frac{d}{dx}[\arcsin x] = \frac{1}{\sqrt{1-x^2}}, \quad |x|<1
\]

\[
\frac{d}{dx}[\arccos x] = -\frac{1}{\sqrt{1-x^2}}, \quad |x|<1
\]

\[
\frac{d}{dx}[\arctan x] = \frac{1}{1+x^2}, \quad x \in \mathbb{R}
\]

\paragraph{Product, Quotient, and Chain
Rules}\label{product-quotient-and-chain-rules}

\begin{enumerate}
\def\labelenumi{\arabic{enumi}.}
\tightlist
\item
  Product Rule
\end{enumerate}

\[
\frac{d}{dx[f(x)g(x)] = f'(x)g(x)+f(x)g'(x)
\]

\begin{enumerate}
\def\labelenumi{\arabic{enumi}.}
\setcounter{enumi}{1}
\tightlist
\item
  Quotient Rule
\end{enumerate}

\[
\frac{d}{dx}\left[\frac{f(x)}{g(x)}\right] = \frac{f'(x)g(x)-f(x)g'(x)}{[g(x)]^2}, \quad g(x)\neq 0
\]

\begin{enumerate}
\def\labelenumi{\arabic{enumi}.}
\setcounter{enumi}{2}
\tightlist
\item
  Chain Rule
\end{enumerate}

\[
\frac{d}{dx}[f(g(x))] = f'(g(x))\cdot g'(x)
\]

\subsubsection{B.3 Common Series
Expansions}\label{b.3-common-series-expansions}

Power series let us express functions as infinite polynomials. These
expansions are essential for approximations, solving differential
equations, and building intuition about functions in calculus.

\paragraph{Geometric Series}\label{geometric-series}

\[
\frac{1}{1-x} = \sum_{n=0}^\infty x^n, \quad |x| < 1
\]

\paragraph{Exponential Function}\label{exponential-function}

\[
e^x = \sum_{n=0}^\infty \frac{x^n}{n!}
= 1 + x + \frac{x^2}{2!} + \frac{x^3}{3!} + \cdots
\]

Valid for all \(x\).

\paragraph{Trigonometric Functions}\label{trigonometric-functions-1}

\[
\sin x = \sum_{n=0}^\infty (-1)^n \frac{x^{2n+1}}{(2n+1)!}
= x - \frac{x^3}{3!} + \frac{x^5}{5!} - \cdots
\]

\[
\cos x = \sum_{n=0}^\infty (-1)^n \frac{x^{2n}}{(2n)!}
= 1 - \frac{x^2}{2!} + \frac{x^4}{4!} - \cdots
\]

\[
\tan^{-1} x = \sum_{n=0}^\infty (-1)^n \frac{x^{2n+1}}{2n+1}, \quad |x|\leq 1
\]

\paragraph{Logarithm}\label{logarithm}

\[
\ln(1+x) = \sum_{n=1}^\infty (-1)^{n+1} \frac{x^n}{n}, \quad -1 < x \leq 1
\]

\paragraph{Binomial Expansion
(Generalized)}\label{binomial-expansion-generalized}

\[
(1+x)^r = \sum_{n=0}^\infty \binom{r}{n} x^n, \quad |x|<1
\]

where

\[
\binom{r}{n} = \frac{r(r-1)(r-2)\cdots(r-n+1)}{n!}.
\]

\subsection{الملحق ج. الرسومات
التوضيحية}\label{ux627ux644ux645ux644ux62dux642-ux62c.-ux627ux644ux631ux633ux648ux645ux627ux62a-ux627ux644ux62aux648ux636ux64aux62dux64aux629}

\subsubsection{\texorpdfstring{ج.1 قوانين الحدود وتعريف
\(\varepsilon\)--\(\delta\)حساب التفاضل والتكامل يعتمد على المعنى الدقيق
للحد. في حين أن الحدس (``القيم تقترب أكثر فأكثر'') مفيد، فإن التعريف
الرسمي يضمن الدقة ويتجنب
المفارقات.}{ج.1 قوانين الحدود وتعريف \textbackslash varepsilon--\textbackslash deltaحساب التفاضل والتكامل يعتمد على المعنى الدقيق للحد. في حين أن الحدس (``القيم تقترب أكثر فأكثر'') مفيد، فإن التعريف الرسمي يضمن الدقة ويتجنب المفارقات.}}\label{ux62c.1-ux642ux648ux627ux646ux64aux646-ux627ux644ux62dux62fux648ux62f-ux648ux62aux639ux631ux64aux641-varepsilondeltaux62dux633ux627ux628-ux627ux644ux62aux641ux627ux636ux644-ux648ux627ux644ux62aux643ux627ux645ux644-ux64aux639ux62aux645ux62f-ux639ux644ux649-ux627ux644ux645ux639ux646ux649-ux627ux644ux62fux642ux64aux642-ux644ux644ux62dux62f.-ux641ux64a-ux62dux64aux646-ux623ux646-ux627ux644ux62dux62fux633-ux627ux644ux642ux64aux645-ux62aux642ux62aux631ux628-ux623ux643ux62bux631-ux641ux623ux643ux62bux631-ux645ux641ux64aux62f-ux641ux625ux646-ux627ux644ux62aux639ux631ux64aux641-ux627ux644ux631ux633ux645ux64a-ux64aux636ux645ux646-ux627ux644ux62fux642ux629-ux648ux64aux62aux62cux646ux628-ux627ux644ux645ux641ux627ux631ux642ux627ux62a.}

\paragraph{فكرة
بديهية}\label{ux641ux643ux631ux629-ux628ux62fux64aux647ux64aux629}

نحن نكتب

\[
\lim_{x \to a} f(x) = L
\]

هذا يعني أنه عندما يقترب \(x\) بشكل عشوائي من \(a\)، فإن قيم \(f(x)\)
تقترب بشكل عشوائي من \(L\).

\paragraph{\texorpdfstring{التعريف الرسمي (\(\varepsilon\)--\(\delta\))
التعريف}{التعريف الرسمي (\textbackslash varepsilon--\textbackslash delta) التعريف}}\label{ux627ux644ux62aux639ux631ux64aux641-ux627ux644ux631ux633ux645ux64a-varepsilondelta-ux627ux644ux62aux639ux631ux64aux641}

نحن نقول ذلك

\[
\lim_{x \to a} f(x) = L
\]

إذا كان لكل \(\varepsilon > 0\)، يوجد \(\delta > 0\) بحيث كلما

\[
0 < |x-a| < \delta,
\]

لدينا

\[
|f(x) - L| < \varepsilon.
\]

\begin{itemize}
\tightlist
\item
  \(\varepsilon\): إلى أي مدى نريد أن يكون \(f(x)\) قريبًا من \(L\).
\item
  \(\delta\): ما مدى قرب \(x\) من \(a\) لتحقيق ذلك.
\end{itemize}

\paragraph{مثال}\label{ux645ux62bux627ux644-2}

أظهر ذلك

\[
\lim_{x \to 2} (3x+1) = 7.
\]

\begin{itemize}
\tightlist
\item
  دع \(\varepsilon > 0\).
\item
  نريد \(|(3x+1)-7| < \varepsilon\).
\item
  تبسيط: \(|3x-6| = 3|x-2| < \varepsilon\).
\item
  ينطبق هذا إذا اخترنا \(\delta = \varepsilon/3\).
\end{itemize}

وبالتالي، حسب التعريف، الحد هو 7.

\paragraph{قوانين
الحدود}\label{ux642ux648ux627ux646ux64aux646-ux627ux644ux62dux62fux648ux62f}

إذا كان \(\lim_{x \to a} f(x) = L\) و\(\lim_{x \to a} g(x) = M\)،
فحينئذٍ:

\begin{enumerate}
\def\labelenumi{\arabic{enumi}.}
\tightlist
\item
  المجموع/الفرق
\end{enumerate}

\[
\lim_{x \to a} [f(x) \pm g(x)] = L \pm M
\]

\begin{enumerate}
\def\labelenumi{\arabic{enumi}.}
\setcounter{enumi}{1}
\tightlist
\item
  متعددة ثابتة
\end{enumerate}

\[
\lim_{x \to a} [c f(x)] = cL
\]

\begin{enumerate}
\def\labelenumi{\arabic{enumi}.}
\setcounter{enumi}{2}
\tightlist
\item
  المنتج
\end{enumerate}

\[
\lim_{x \to a} [f(x)g(x)] = LM
\]

\begin{enumerate}
\def\labelenumi{\arabic{enumi}.}
\setcounter{enumi}{3}
\tightlist
\item
  القسمة (إذا كان \(M \neq 0\))
\end{enumerate}

\[
\lim_{x \to a} \frac{f(x)}{g(x)} = \frac{L}{M}
\]

\begin{enumerate}
\def\labelenumi{\arabic{enumi}.}
\setcounter{enumi}{4}
\tightlist
\item
  القوى والجذور
\end{enumerate}

\[
\lim_{x \to a} [f(x)]^n = L^n, \quad \lim_{x \to a} \sqrt[n]{f(x)} = \sqrt[n]{L} \ (\text{if defined}).
\]

\subsubsection{ج.2 رسم البرهان: النظرية الأساسية في حساب التفاضل
والتكامل}\label{ux62c.2-ux631ux633ux645-ux627ux644ux628ux631ux647ux627ux646-ux627ux644ux646ux638ux631ux64aux629-ux627ux644ux623ux633ux627ux633ux64aux629-ux641ux64a-ux62dux633ux627ux628-ux627ux644ux62aux641ux627ux636ux644-ux648ux627ux644ux62aux643ux627ux645ux644}

تربط النظرية الأساسية لحساب التفاضل والتكامل (FTC) بين العمليتين
المركزيتين لحساب التفاضل والتكامل: التفاضل والتكامل. ويظهر أنها في
الواقع عمليات عكسية.

\paragraph{بيان
النظرية}\label{ux628ux64aux627ux646-ux627ux644ux646ux638ux631ux64aux629}

الجزء الأول (تمايز التكامل): إذا كان \(f\) مستمرًا على \([a,b]\) وقمنا
بتعريفه

\[F(x) = \int_a^x f(t)\,dt,
\]

then \(F\) is differentiable on \((a,b)\) and

\[
F '(س) = و(خ).
\]

Part II (Evaluation of a Definite Integral): If \(F\) is any
antiderivative of \(f\) on \([a,b]\), then

\[
\int_a^b f(x)\,dx = F(b)-F(a).
\]

\paragraph{Proof Sketch of Part I}\label{proof-sketch-of-part-i}

\begin{enumerate}
\def\labelenumi{\arabic{enumi}.}
\item
  Start with the definition of the derivative:

  \[
  F'(x) = \lim_{h\to 0} \frac{F(x+h)-F(x)}{h}.
  \]
\item
  Substituting \(F(x) = \int_a^x f(t)\,dt\):

  \[
  F(x+h)-F(x) = \int_a^{x+h} f(t)\,dt - \int_a^x f(t)\,dt.
  \]
\item
  By the additivity of integrals:

  \[
  F(x+h)-F(x) = \int_x^{x+h} f(t)\,dt.
  \]
\item
  Therefore:

  \[
  \frac{F(x+h)-F(x)}{h} = \frac{1}{h}\int_x^{x+h} f(t)\,dt.
  \]
\item
  By the Mean Value Theorem for integrals, there exists
  \(c \in [x,x+h]\) such that

  \[
  \frac{1}{h}\int_x^{x+h} f(t)\,dt = f(c).
  \]
\item
  As \(h \to 0\), \(c \to x\), and since \(f\) is continuous:

  \[
  \lim_{h\to 0} f(c) = f(x).
  \]
\end{enumerate}

Thus, \(F'(x) = f(x)\).

\paragraph{Proof Sketch of Part II}\label{proof-sketch-of-part-ii}

\begin{enumerate}
\def\labelenumi{\arabic{enumi}.}
\item
  Let \(F\) be an antiderivative of \(f\), so \(F'(x) = f(x)\).
\item
  By Part I, the function

  \[
  G(x) = \int_a^x f(t)\,dt
  \]

  is also an antiderivative of \(f\).
\item
  Since \(F\) and \(G\) differ only by a constant,

  \[
  F(x) = G(x) + C.
  \]
\item
  Evaluating at the endpoints:

  \[
  \int_a^b f(x)\,dx = G(b)-G(a) = (F(b)+C)-(F(a)+C) = F(b)-F(a).
  \]
\end{enumerate}

\subsubsection{C.3 Proof Sketch: Convergence of the Geometric
Series}\label{c.3-proof-sketch-convergence-of-the-geometric-series}

The geometric series is one of the simplest and most important infinite
series. It serves as a model for understanding convergence and is the
foundation for many later results in calculus.

\paragraph{The Series}\label{the-series}

\[
\sum_{n=0}^\infty ar^n = a + ar + ar^2 + ar^3 + \cdots
\]

where \(a\) is the first term and \(r\) is the common ratio.

\paragraph{Partial Sum Formula}\label{partial-sum-formula}

The \(n\)-th partial sum is

\[S_n = a + ar + ar^2 + \cdots + ar^n.
\]

Multiply both sides by \(r\):

\[
rS_n = ar + ar^2 + \cdots + ar^{n+1}.
\]

Subtract the two equations:

\[
S_n - rS_n = أ - ar^{n+1}.
\]

\[
S_n(1-r) = أ(1-r^{n+1}).
\]

So

\[
S_n = \frac{a(1-r^{n+1})}{1-r}, \quad r \neq 1.
\]

\paragraph{Convergence}\label{convergence}

Take the limit as \(n \to \infty\):

\begin{itemize}
\item
  If \(|r| < 1\), then \(r^{n+1} \to 0\).

  \[
  \lim_{n\to\infty} S_n = \frac{a}{1-r}.
  \]
\item
  If \(|r| \geq 1\), then \(r^{n+1}\) does not go to 0. The series
  diverges.
\end{itemize}

\paragraph{Result}\label{result}

\[
\sum_{n=0}^\infty ar^n =
\بداية{الحالات}
\dfrac{a}{1-r}, & |r|<1, \\[6pt]
\text{يتباعد}، & |r|\geq 1.
\النهاية{الحالات}
\]

\subsection{Appendix D. Applications and
Connections}\label{appendix-d.-applications-and-connections}

\subsubsection{D.1 Physics Connections: Velocity, Acceleration, and
Work}\label{d.1-physics-connections-velocity-acceleration-and-work}

Calculus was originally developed to solve problems in physics -
especially motion and change. Here are some of the most important
connections.

\paragraph{Position, Velocity, and
Acceleration}\label{position-velocity-and-acceleration}

\begin{itemize}
\item
  Position function: \(s(t)\) gives the location of an object at time
  \(t\).
\item
  Velocity: the derivative of position.

  \[
  v(t) = s'(t) = \frac{ds}{dt}
  \]
\item
  Acceleration: the derivative of velocity (or second derivative of
  position).

  \[
  أ(t) = v'(t) = s''(t) = \frac{d^2s}{dt^2}
  \]
\end{itemize}

Example: If \(s(t) = 4t^2\) meters, then:

\[
v(t) = 8t، \quad a(t) = 8.
\]

So the object moves faster linearly with time, under constant
acceleration.

\paragraph{Work and Force}\label{work-and-force}

In physics, work is the product of force and distance. If force varies
with position, calculus gives:

\[
W = \int_a^b F(x)\, dx
\]

where \(F(x)\) is the force at position \(x\), and the object moves from
\(x=a\) to \(x=b\).

Example: A spring with Hooke's law force \(F(x) = kx\) requires work

\[
W = \int_0^d kx\, dx = \frac{1}{2}kd^2
\]

لتمديد الربيع مسافة \(d\).

\paragraph{\texorpdfstring{الطاقة والمساحات تحت المنحنيات- الطاقة
الحركية:
\(E_k = \tfrac{1}{2}mv^2\).}{الطاقة والمساحات تحت المنحنيات- الطاقة الحركية: E\_k = \textbackslash tfrac\{1\}\{2\}mv\^{}2.}}\label{ux627ux644ux637ux627ux642ux629-ux648ux627ux644ux645ux633ux627ux62dux627ux62a-ux62aux62dux62a-ux627ux644ux645ux646ux62dux646ux64aux627ux62a--ux627ux644ux637ux627ux642ux629-ux627ux644ux62dux631ux643ux64aux629-e_k-tfrac12mv2.}

\begin{itemize}
\tightlist
\item
  تتضمن الطاقة الكامنة في كثير من الأحيان تكاملات (على سبيل المثال، طاقة
  الجاذبية الناتجة عن قوة الجاذبية).
\item
  بشكل عام، تكامل دالة القوة يعطي الطاقة المخزنة أو الشغل المنجز.
\end{itemize}

\paragraph{الممارسة
السريعة}\label{ux627ux644ux645ux645ux627ux631ux633ux629-ux627ux644ux633ux631ux64aux639ux629}

\begin{enumerate}
\def\labelenumi{\arabic{enumi}.}
\tightlist
\item
  إذا كان \(s(t) = t^3 - 3t\)، فابحث عن \(v(t)\) و\(a(t)\).
\item
  احسب الشغل الذي تبذله قوة ثابتة مقدارها 10 N لتحريك جسم مسافة 5 m.
\item
  الربيع له ثابت \(k=200\). ما مقدار الشغل اللازم لتمديده بمقدار 0.1 م؟
\item
  أثبت أن التسارع هو المشتقة الثانية للموضع.
\item
  اشرح كيفية ارتباط التكامل \(\int v(t)\, dt\) بالإزاحة.
\end{enumerate}

\subsubsection{د.2 اتصالات الاحتمالية
والإحصائيات}\label{ux62f.2-ux627ux62aux635ux627ux644ux627ux62a-ux627ux644ux627ux62dux62aux645ux627ux644ux64aux629-ux648ux627ux644ux625ux62dux635ux627ux626ux64aux627ux62a}

يرتبط حساب التفاضل والتكامل ارتباطًا وثيقًا بالاحتمالات والإحصائيات، خاصة
عند التعامل مع المتغيرات العشوائية المستمرة. تصبح التكاملات ضرورية
لتحديد الاحتمالات والمتوسطات والتوقعات.

\paragraph{دوال الكثافة الاحتمالية
(PDFs)}\label{ux62fux648ux627ux644-ux627ux644ux643ux62bux627ux641ux629-ux627ux644ux627ux62dux62aux645ux627ux644ux64aux629-pdfs}

بالنسبة للمتغير العشوائي المستمر \(X\)، يتم وصف الاحتمالات بواسطة دالة
كثافة الاحتمال \(f(x)\):

\begin{enumerate}
\def\labelenumi{\arabic{enumi}.}
\item
  \(f(x) \geq 0\) لجميع \(x\).
\item
  الاحتمال الإجمالي يساوي 1:

  \[
  \int_{-\infty}^{\infty} f(x)\, dx = 1.
  \]
\end{enumerate}

احتمال أن يقع \(X\) في الفاصل الزمني \([a,b]\) هو

\[
P(a \leq X \leq b) = \int_a^b f(x)\, dx.
\]

\paragraph{القيمة المتوقعة
(المتوسط)}\label{ux627ux644ux642ux64aux645ux629-ux627ux644ux645ux62aux648ux642ux639ux629-ux627ux644ux645ux62aux648ux633ux637-1}

القيمة المتوقعة (متوسط النتيجة) هي

\[
E[X] = \int_{-\infty}^{\infty} x f(x)\, dx.
\]

هذه هي النسخة الحسابية للمتوسط المرجح.

\#\#\#\#التباين

انتشار مقاييس التباين:

\[
\text{Var}(X) = E[(X-\mu)^2] = \int_{-\infty}^{\infty} (x-\mu)^2 f(x)\, dx,
\]

حيث \(\mu = E[X]\).

\paragraph{التوزيعات
الشائعة}\label{ux627ux644ux62aux648ux632ux64aux639ux627ux62a-ux627ux644ux634ux627ux626ux639ux629}

\begin{enumerate}
\def\labelenumi{\arabic{enumi}.}
\item
  التوزيع الموحد على \([a,b]\):

  \[
  f(x) = \frac{1}{b-a}, \quad a \leq x \leq b.
  \]

  يعني: \(\frac{a+b}{2}\).
\item
  التوزيع الأسي مع المعلمة \(\lambda > 0\):

  \[
  f(x) = \lambda e^{-\lambda x}, \quad x \geq 0.\]

  Mean: \(1/\lambda\).
\item
  Normal (Gaussian) distribution:

  \[
  f(x) = \frac{1}{\sqrt{2\pi\sigma^2}} e^{-(x-\mu)^2/(2\sigma^2)}.
  \]

  Integrals of this distribution connect to the error function.
\end{enumerate}

\paragraph{Why This Matters}\label{why-this-matters-3}

\begin{itemize}
\tightlist
\item
  Integrals turn probabilities into areas under curves.
\item
  Expectation and variance link calculus to averages and variability.
\item
  Most real-world data models (finance, physics, biology, AI) use these
  continuous probability distributions.
\end{itemize}

\paragraph{Quick Practice}\label{quick-practice}

\begin{enumerate}
\def\labelenumi{\arabic{enumi}.}
\tightlist
\item
  For \(f(x) = \tfrac{1}{2}\) on \([0,2]\), compute
  \(P(0.5 \leq X \leq 1.5)\).
\item
  For exponential distribution with \(\lambda = 2\), compute \(E[X]\).
\item
  Show that the total area under the standard normal curve equals 1.
\item
  Find the mean of a uniform distribution on \([3,7]\).
\item
  Explain why probabilities are computed with integrals, not sums, for
  continuous variables.
\end{enumerate}

\subsubsection{D.3 Computer Science Connections: Taylor Approximations
in
Algorithms}\label{d.3-computer-science-connections-taylor-approximations-in-algorithms}

Calculus is not only for physics - it also underpins many tools and
techniques in computer science. One of the clearest bridges is through
Taylor series, which provide efficient ways to approximate functions in
numerical computing and algorithms.

\paragraph{Function Approximation for
Computing}\label{function-approximation-for-computing}

Computers cannot directly store or calculate most functions exactly
(like \(e^x\), \(\sin x\), or \(\ln x\)). Instead, they use polynomial
approximations derived from Taylor expansions.

Example: To approximate \(e^x\), truncate the Maclaurin series:

\[
e^x \approx 1 + x + \frac{x^2}{2!} + \frac{x^3}{3!}.
\]

بالنسبة إلى \(x\) الصغيرة، فإن كثير الحدود هذا يعطي نتائج دقيقة بعدد
قليل من الحدود فقط.

\paragraph{الكفاءة في
الخوارزميات}\label{ux627ux644ux643ux641ux627ux621ux629-ux641ux64a-ux627ux644ux62eux648ux627ux631ux632ux645ux64aux627ux62a}

\begin{itemize}
\tightlist
\item
  الدوال المثلثية: غالبًا ما تستخدم خوارزميات الآلات الحاسبة ووحدات
  المعالجة المركزية (CPU) توسعات متسلسلة (أو أشكال مختلفة مثل كثيرات
  حدود تشيبيشيف).- الأسي/اللوغاريتم: توسعات تايلور هي أساس التقريب
  السريع في المكتبات الرقمية.
\item
  إيجاد الجذر: تعتمد طريقة نيوتن على التقريب الخطي، وهو تطبيق مباشر
  لمتسلسلة تايلور (المشتقة الأولى).
\end{itemize}

\paragraph{التحليل
العددي}\label{ux627ux644ux62aux62dux644ux64aux644-ux627ux644ux639ux62fux62fux64a}

تعتبر توسعات تايلور أساسية في تحليل الأخطاء:

\begin{itemize}
\item
  تقريب حد الخطأ باستخدام صيغة الباقي:

  \[
  R_n(x) = \frac{f^{(n+1)}(c)}{(n+1)!}(x-a)^{n+1}.
  \]
\item
  يخبرنا هذا بعدد المصطلحات المطلوبة للحصول على دقة معينة.
\end{itemize}

\paragraph{اتصالات التعلم
الآلي}\label{ux627ux62aux635ux627ux644ux627ux62a-ux627ux644ux62aux639ux644ux645-ux627ux644ux622ux644ux64a}

\begin{itemize}
\tightlist
\item
  يستخدم التحسين القائم على التدرج (مثل نزول التدرج) المشتقات لتحديث
  المعلمات بكفاءة.
\item
  غالبًا ما يتم تقريب وظائف التنشيط (مثل \(\tanh x\) أو
  \(\sigma(x)=1/(1+e^{-x})\)) بواسطة متعددات الحدود أو وظائف متعددة
  التعريف للسرعة.
\item
  يمكن لتقريب السلسلة تسريع التدريب والاستدلال في البيئات المقيدة.
\end{itemize}

\paragraph{لماذا هذا
مهم}\label{ux644ux645ux627ux630ux627-ux647ux630ux627-ux645ux647ux645-33}

\begin{itemize}
\tightlist
\item
  تربط تقديرات تايلور بين الرياضيات المستمرة والحوسبة المنفصلة.
\item
  توضح كيفية استخدام مفاهيم حساب التفاضل والتكامل في الخوارزميات
  والأساليب العددية والتعلم الآلي.
\item
  فهم التقريبات يساعد على تجنب المخاطر عند الاعتماد على أجهزة الكمبيوتر
  لإجراء العمليات الحسابية.
\end{itemize}

\paragraph{الممارسة
السريعة}\label{ux627ux644ux645ux645ux627ux631ux633ux629-ux627ux644ux633ux631ux64aux639ux629-1}

\begin{enumerate}
\def\labelenumi{\arabic{enumi}.}
\tightlist
\item
  تقريب \(\sin(0.1)\) باستخدام الحدود الثلاثة الأولى من سلسلة ماكلورين.
\item
  استخدم الحد المتبقي لتقدير الخطأ في تقريب \(e^1\) مع كثير الحدود من
  الدرجة 3.
\item
  اشرح كيف تستخدم طريقة نيوتن نظرية تايلور.
\item
  لماذا قد تفضل أجهزة الكمبيوتر التقريبية متعددة الحدود على الصيغ
  الدقيقة للوظائف؟
\item
  في التعلم الآلي، لماذا يعتبر المشتق (التدرج) بالغ الأهمية للتحسين؟
\end{enumerate}




\end{document}
