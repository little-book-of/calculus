% Options for packages loaded elsewhere
\PassOptionsToPackage{unicode}{hyperref}
\PassOptionsToPackage{hyphens}{url}
\PassOptionsToPackage{dvipsnames,svgnames,x11names}{xcolor}
%
\documentclass[
  letterpaper,
  DIV=11]{scrartcl}

\usepackage{amsmath,amssymb}
\usepackage{iftex}
\ifPDFTeX
  \usepackage[T1]{fontenc}
  \usepackage[utf8]{inputenc}
  \usepackage{textcomp} % provide euro and other symbols
\else % if luatex or xetex
  \usepackage{unicode-math}
  \defaultfontfeatures{Scale=MatchLowercase}
  \defaultfontfeatures[\rmfamily]{Ligatures=TeX,Scale=1}
\fi
\usepackage{lmodern}
\ifPDFTeX\else  
    % xetex/luatex font selection
\fi
% Use upquote if available, for straight quotes in verbatim environments
\IfFileExists{upquote.sty}{\usepackage{upquote}}{}
\IfFileExists{microtype.sty}{% use microtype if available
  \usepackage[]{microtype}
  \UseMicrotypeSet[protrusion]{basicmath} % disable protrusion for tt fonts
}{}
\makeatletter
\@ifundefined{KOMAClassName}{% if non-KOMA class
  \IfFileExists{parskip.sty}{%
    \usepackage{parskip}
  }{% else
    \setlength{\parindent}{0pt}
    \setlength{\parskip}{6pt plus 2pt minus 1pt}}
}{% if KOMA class
  \KOMAoptions{parskip=half}}
\makeatother
\usepackage{xcolor}
\setlength{\emergencystretch}{3em} % prevent overfull lines
\setcounter{secnumdepth}{-\maxdimen} % remove section numbering
% Make \paragraph and \subparagraph free-standing
\makeatletter
\ifx\paragraph\undefined\else
  \let\oldparagraph\paragraph
  \renewcommand{\paragraph}{
    \@ifstar
      \xxxParagraphStar
      \xxxParagraphNoStar
  }
  \newcommand{\xxxParagraphStar}[1]{\oldparagraph*{#1}\mbox{}}
  \newcommand{\xxxParagraphNoStar}[1]{\oldparagraph{#1}\mbox{}}
\fi
\ifx\subparagraph\undefined\else
  \let\oldsubparagraph\subparagraph
  \renewcommand{\subparagraph}{
    \@ifstar
      \xxxSubParagraphStar
      \xxxSubParagraphNoStar
  }
  \newcommand{\xxxSubParagraphStar}[1]{\oldsubparagraph*{#1}\mbox{}}
  \newcommand{\xxxSubParagraphNoStar}[1]{\oldsubparagraph{#1}\mbox{}}
\fi
\makeatother


\providecommand{\tightlist}{%
  \setlength{\itemsep}{0pt}\setlength{\parskip}{0pt}}\usepackage{longtable,booktabs,array}
\usepackage{calc} % for calculating minipage widths
% Correct order of tables after \paragraph or \subparagraph
\usepackage{etoolbox}
\makeatletter
\patchcmd\longtable{\par}{\if@noskipsec\mbox{}\fi\par}{}{}
\makeatother
% Allow footnotes in longtable head/foot
\IfFileExists{footnotehyper.sty}{\usepackage{footnotehyper}}{\usepackage{footnote}}
\makesavenoteenv{longtable}
\usepackage{graphicx}
\makeatletter
\newsavebox\pandoc@box
\newcommand*\pandocbounded[1]{% scales image to fit in text height/width
  \sbox\pandoc@box{#1}%
  \Gscale@div\@tempa{\textheight}{\dimexpr\ht\pandoc@box+\dp\pandoc@box\relax}%
  \Gscale@div\@tempb{\linewidth}{\wd\pandoc@box}%
  \ifdim\@tempb\p@<\@tempa\p@\let\@tempa\@tempb\fi% select the smaller of both
  \ifdim\@tempa\p@<\p@\scalebox{\@tempa}{\usebox\pandoc@box}%
  \else\usebox{\pandoc@box}%
  \fi%
}
% Set default figure placement to htbp
\def\fps@figure{htbp}
\makeatother

\KOMAoption{captions}{tableheading}
\makeatletter
\@ifpackageloaded{caption}{}{\usepackage{caption}}
\AtBeginDocument{%
\ifdefined\contentsname
  \renewcommand*\contentsname{Inhaltsverzeichnis}
\else
  \newcommand\contentsname{Inhaltsverzeichnis}
\fi
\ifdefined\listfigurename
  \renewcommand*\listfigurename{Abbildungsverzeichnis}
\else
  \newcommand\listfigurename{Abbildungsverzeichnis}
\fi
\ifdefined\listtablename
  \renewcommand*\listtablename{Tabellenverzeichnis}
\else
  \newcommand\listtablename{Tabellenverzeichnis}
\fi
\ifdefined\figurename
  \renewcommand*\figurename{Abbildung}
\else
  \newcommand\figurename{Abbildung}
\fi
\ifdefined\tablename
  \renewcommand*\tablename{Tabelle}
\else
  \newcommand\tablename{Tabelle}
\fi
}
\@ifpackageloaded{float}{}{\usepackage{float}}
\floatstyle{ruled}
\@ifundefined{c@chapter}{\newfloat{codelisting}{h}{lop}}{\newfloat{codelisting}{h}{lop}[chapter]}
\floatname{codelisting}{Listing}
\newcommand*\listoflistings{\listof{codelisting}{Listingverzeichnis}}
\makeatother
\makeatletter
\makeatother
\makeatletter
\@ifpackageloaded{caption}{}{\usepackage{caption}}
\@ifpackageloaded{subcaption}{}{\usepackage{subcaption}}
\makeatother

\ifLuaTeX
\usepackage[bidi=basic]{babel}
\else
\usepackage[bidi=default]{babel}
\fi
\babelprovide[main,import]{ngerman}
% get rid of language-specific shorthands (see #6817):
\let\LanguageShortHands\languageshorthands
\def\languageshorthands#1{}
\ifLuaTeX
  \usepackage[german]{selnolig} % disable illegal ligatures
\fi
\usepackage{bookmark}

\IfFileExists{xurl.sty}{\usepackage{xurl}}{} % add URL line breaks if available
\urlstyle{same} % disable monospaced font for URLs
\hypersetup{
  pdftitle={Das kleine Buch der Analysis},
  pdflang={de},
  colorlinks=true,
  linkcolor={blue},
  filecolor={Maroon},
  citecolor={Blue},
  urlcolor={Blue},
  pdfcreator={LaTeX via pandoc}}


\title{Das kleine Buch der Analysis}
\author{}
\date{}

\begin{document}
\maketitle


\section{Das kleine Buch der
Analysis}\label{das-kleine-buch-der-analysis}

Eine prägnante, anfängerfreundliche Einführung in die Kernideen der
Analysis.

\subsection{Formate}\label{formate}

\begin{itemize}
\tightlist
\item
  \href{../artifacts/de/book.pdf}{Download PDF} -- druckfertige Version
\item
  \href{../artifacts/de/book.epub}{Download EPUB} -- E-Reader-freundlich
\item
  \href{../artifacts/de/book.tex}{View LaTeX} -- Latexquelle
\end{itemize}

\section{Teil 1. Limits und Derivate}\label{teil-1.-limits-und-derivate}

\section{Kapitel 1. Funktionen und
Grenzen}\label{kapitel-1.-funktionen-und-grenzen}

\subsection{1.1 Funktionen}\label{funktionen}

Eine Funktion ist eines der grundlegendsten Objekte in der Mathematik.
Im Kern ist eine Funktion eine Regel, die eine Eingabe entgegennimmt und
genau eine Ausgabe erzeugt. Mit Funktionen können wir Beziehungen
beschreiben, reale Phänomene modellieren und die gesamte Maschinerie der
Analysis aufbauen.

\subsubsection{Definition}\label{definition}

Formal wird eine Funktion \(f\) von einer Menge \(X\) (Domäne genannt)
in eine Menge \(Y\) (Kodomäne genannt) geschrieben

\[
f : X \to Y.
\]

Für jedes Element \(x \in X\) gibt es ein eindeutiges Element
\(f(x) \in Y\). Der Wert \(f(x)\) heißt das Bild von \(x\) unter \(f\).

Wenn \(y = f(x)\), dann ist \(y\) der Ausgang, der dem Eingang \(x\)
entspricht. Die Menge aller tatsächlich erscheinenden Ausgaben wird als
Bereich (eine Teilmenge der Codomäne) bezeichnet.

\subsubsection{Beispiele}\label{beispiele}

\begin{enumerate}
\def\labelenumi{\arabic{enumi}.}
\item
  Die Funktion \(f(x) = x^2\) bildet jede reelle Zahl \(x\) ihrem
  Quadrat zu.

  \begin{itemize}
  \tightlist
  \item
    Domain: alle reellen Zahlen \(\mathbb{R}\).
  \item
    Codomain: alle reellen Zahlen \(\mathbb{R}\).
  \item
    Bereich: alle nichtnegativen reellen Zahlen \([0, \infty)\).
  \end{itemize}
\item
  Die Funktion \(g(x) = \dfrac{1}{x}\) weist jeder reellen Zahl ungleich
  Null ihren Kehrwert zu.

  \begin{itemize}
  \tightlist
  \item
    Domain: \(\mathbb{R} \setminus \{0\}\).
  \item
    Bereich: \(\mathbb{R} \setminus \{0\}\).
  \end{itemize}
\item
  Ein Beispiel aus der Praxis: Sei \(T(t)\) die Außentemperatur (in °C)
  zum Zeitpunkt \(t\) (in Stunden). Dies ist eine Funktion von
  „Tageszeit`` bis „Temperatur``.
\end{enumerate}

\subsubsection{Möglichkeiten zur Darstellung von
Funktionen}\label{muxf6glichkeiten-zur-darstellung-von-funktionen}

Funktionen können auf verschiedene nützliche Arten dargestellt werden:

\begin{itemize}
\tightlist
\item
  Formeln: z. B. \(f(x) = \sin x + x^2\).
\item
  Diagramme: Darstellung aller Punkte \((x, f(x))\) in der
  Koordinatenebene.
\item
  Tabellen: Paarung von Ein- und Ausgängen für diskrete Datensätze.
\item
  Mündliche Beschreibungen: „Weisen Sie jedem Schüler seine Note zu.``
\end{itemize}

Jede Darstellung hebt verschiedene Aspekte derselben Funktion hervor.

\subsubsection{Terminologie}\label{terminologie}

\begin{itemize}
\tightlist
\item
  Unabhängige Variable: die Eingabe (normalerweise geschrieben \(x\)).
\item
  Abhängige Variable: die Ausgabe (normalerweise geschrieben \(y\),
  wobei \(y = f(x)\)).
\item
  Funktionsschreibweise: \(f(x)\) lautet „\(f\) von \(x\).``
\end{itemize}

\subsubsection{Warum Funktionen in der Analysis wichtig
sind}\label{warum-funktionen-in-der-analysis-wichtig-sind}

In der Analysis wird untersucht, wie sich Funktionen ändern. Derivate
messen momentane Änderungsraten, während Integrale akkumulierte Effekte
messen. Um diese Ideen zu meistern, benötigen wir zunächst ein solides
Verständnis davon, was Funktionen sind und wie sie sich verhalten.

\subsubsection{Übungen}\label{uxfcbungen}

\begin{enumerate}
\def\labelenumi{\arabic{enumi}.}
\item
  Für die Funktion \(f(x) = 3x - 2\):- Finden Sie die Domäne, die
  Co-Domäne und den Bereich.
\item
  Für welche Eingänge ist die Funktion \(h(x) = \sqrt{x-1}\) definiert?
  Welche Reichweite hat es?
\item
  Geben Sie ein reales Beispiel für eine Funktion aus Ihrem täglichen
  Leben. Geben Sie die Domäne und Co-Domäne klar an.
\item
  Skizzieren Sie den Graphen von \(f(x) = |x|\). Wie groß ist die
  Reichweite?
\item
  Angenommen \(g(x) = \dfrac{1}{x^2+1}\). Erklären Sie, warum sein
  Bereich das Intervall \((0, 1]\) ist.
\end{enumerate}

\subsection{1.2 Diagramme und
Transformationen}\label{diagramme-und-transformationen}

Eine Funktion kann nicht nur durch Formeln, sondern auch durch ihren
Graphen verstanden werden. Der Graph einer Funktion \(f\) ist die Menge
aller geordneten Paare \((x, f(x))\), wobei \(x\) zum Definitionsbereich
von \(f\) gehört. Durch die Darstellung dieser Paare in der
Koordinatenebene erhält man ein Bild davon, wie sich die Funktion
verhält.

\subsubsection{Grundlegende Diagramme}\label{grundlegende-diagramme}

Einige Diagramme sind so grundlegend, dass sie auswendig gelernt werden
sollten:

\begin{itemize}
\tightlist
\item
  \(f(x) = x\): eine gerade Linie durch den Ursprung.
\item
  \(f(x) = x^2\): eine Parabel, die sich nach oben öffnet.
\item
  \(f(x) = |x|\): ein „V``-förmiges Diagramm.
\item
  \(f(x) = \frac{1}{x}\): eine Hyperbel mit zwei Zweigen.
\item
  \(f(x) = \sin x\): eine wellenförmige periodische Kurve.
\end{itemize}

Diese dienen als Bausteine \hspace{0pt}\hspace{0pt}für komplexere
Funktionen.

\subsubsection{Transformationen}\label{transformationen}

Mithilfe einfacher Regeln können Diagramme verschoben, gestreckt oder
gespiegelt werden:

\begin{enumerate}
\def\labelenumi{\arabic{enumi}.}
\item
  Vertikale Verschiebungen: Das Hinzufügen einer Konstante verschiebt
  den Graphen nach oben oder unten.

  \[
  y = f(x) + c \quad \text{is } f(x) \text{ shifted upward by } c.
  \]
\item
  Horizontale Verschiebungen: Durch Hinzufügen innerhalb des Arguments
  wird der Graph nach links oder rechts verschoben.

  \[
  y = f(x - c) \quad \text{is } f(x) \text{ shifted right by } c.
  \]
\item
  Vertikale Skalierung: Durch Multiplikation mit einer Konstanten wird
  der Graph vertikal gestreckt oder gestaucht.

  \[
  y = a f(x), \quad a > 1 \text{ stretches; } 0 < a < 1 \text{ compresses.}
  \]
\item
  Horizontale Skalierung: Durch Multiplikation innerhalb des Arguments
  wird der Graph horizontal gestreckt oder gestaucht.

  \[
  y = f(bx), \quad b > 1 \text{ compresses toward the } y\text{-axis}.
  \]
\item
  Überlegungen:

  \begin{itemize}
  \tightlist
  \item
    \(y = -f(x)\): Spiegelung entlang der \(x\)-Achse.
  \item
    \(y = f(-x)\): Spiegelung entlang der \(y\)-Achse.
  \end{itemize}
\end{enumerate}

\subsubsection{Transformationen
kombinieren}\label{transformationen-kombinieren}

Komplexe Diagramme entstehen oft durch die Kombination mehrerer
Transformationen nacheinander. Zum Beispiel:

\[
y = 2(x-1)^2 + 3
\]

erhält man, indem man die Parabel \(y = x^2\) nimmt, um 1 nach rechts
verschiebt, die Vertikale um 2 streckt und um 3 nach oben verschiebt.

\subsubsection{Übungen}\label{uxfcbungen-1}

\begin{enumerate}
\def\labelenumi{\arabic{enumi}.}
\tightlist
\item
  Skizzieren Sie den Graphen von \(y = (x+2)^2 - 1\). Identifizieren Sie
  die Reihenfolge der Transformationen aus \(y = x^2\).
\item
  Was passiert mit dem Graphen von \(y = f(x)\), wenn wir \(x\) durch
  \(-x\) ersetzen? Versuchen Sie es mit \(f(x) = \sqrt{x}\).
\item
  Beschreiben Sie die Transformationen, die aus \(y = \sin x\)
  \(y = 3\sin(x - \pi/4)\) machen.4. Zeichnen Sie den Graphen von
  \(y = |x-1| + 2\). Geben Sie den Scheitelpunkt und die Steigung jedes
  Zweigs an.
\item
  Erklären Sie für \(y = \frac{1}{x-2}\), wie sich der Graph von
  \(y = \frac{1}{x}\) verändert hat.
\end{enumerate}

\subsection{1.3 Intuitive Vorstellung von
Grenzen}\label{intuitive-vorstellung-von-grenzen}

In vielen Situationen ist der Wert einer Funktion an einem Punkt weniger
wichtig als die Werte, die sie in der Nähe dieses Punktes annimmt. Das
Konzept einer Grenze greift diese Idee auf.

\subsubsection{Annäherung an einen
Wert}\label{annuxe4herung-an-einen-wert}

Stellen Sie sich vor, Sie gehen auf eine Wand zu. Noch bevor man es
berührt, kommt man immer näher. Auf die gleiche Weise können sich die
Werte von \(f(x)\) einer Zahl \(L\) nähern, wenn sich \(x\) einer Zahl
\(a\) nähert. Wir sagen dann:

\[
\lim_{x \to a} f(x) = L.
\]

Dies drückt die Idee aus, dass \(f(x)\) so nah an \(L\) herangebracht
werden kann, wie wir wollen, indem einfach \(x\) nah genug an \(a\)
herangeführt wird.

\subsubsection{Beispiele}\label{beispiele-1}

\begin{enumerate}
\def\labelenumi{\arabic{enumi}.}
\item
  Für \(f(x) = 2x + 3\): Als \(x \to 1\), \(f(x) \to 5\).
\item
  Für \(f(x) = \dfrac{\sin x}{x}\): Als \(x \to 0\) geht die Funktion
  gegen 1, obwohl \(f(0)\) nicht definiert ist.
\item
  Für \(f(x) = \dfrac{1}{x}\): Als \(x \to 0^+\) (von rechts kommend),
  \(f(x) \to +\infty\). Als \(x \to 0^-\) (von links kommend),
  \(f(x) \to -\infty\). Da sich das linke und rechte Verhalten
  unterscheiden, existiert die Grenze bei 0 nicht.
\end{enumerate}

\subsubsection{Bedeutung von Grenzen}\label{bedeutung-von-grenzen}

\begin{itemize}
\tightlist
\item
  Sie ermöglichen es uns, Funktionen an Stellen zu definieren, an denen
  sie ursprünglich nicht definiert waren.
\item
  Sie erfassen das Verhalten in der Nähe von Diskontinuitäten und
  Singularitäten.
\item
  Sie bilden die Grundlage für Ableitungen (momentane Änderungsraten)
  und Integrale (Flächen als Grenzen von Summen).
\end{itemize}

\subsubsection{Einseitige Grenzen}\label{einseitige-grenzen}

Manchmal muss das Verhalten von links und rechts getrennt untersucht
werden:

\[
\lim_{x \to a^-} f(x), \quad \lim_{x \to a^+} f(x).
\]

Wenn beide übereinstimmen, liegt die zweiseitige Grenze vor.

\subsubsection{Übungen}\label{uxfcbungen-2}

\begin{enumerate}
\def\labelenumi{\arabic{enumi}.}
\tightlist
\item
  Berechnen Sie \(\lim_{x \to 2} (3x^2 - x)\).
\item
  Was ist \(\lim_{x \to 0} \frac{\sin x}{x}\)? Nutzen Sie die Intuition
  aus der Grafik von \(\sin x\).
\item
  Bewerten Sie \(\lim_{x \to 0} |x|/x\). Gibt es die zweiseitige Grenze?
\item
  Finden Sie \(\lim_{x \to \infty} \frac{1}{x}\). Interpretieren Sie
  dieses Ergebnis in Worten.
\item
  Was sind für 29 ¤¤ ¤¤ 30 ¤¤? Vergleichen Sie mit dem Wert von
  \(f(1)\).
\end{enumerate}

\subsection{1.4 Formale Definition von
Grenzwerten}\label{formale-definition-von-grenzwerten}

Die intuitive Vorstellung einer Grenze kann mithilfe der
Epsilon-Delta-Definition präzisiert werden. Dies gibt uns eine strenge
Möglichkeit zu sagen, dass sich \(f(x)\) einem Wert von \(L\) annähert,
während \(x\) sich einem Wert von \(a\) annähert.

\subsubsection{Die Definition}\label{die-definition}

Wir schreiben

\[
\lim_{x \to a} f(x) = L
\]

wenn die folgende Bedingung gilt:

Für jedes \(\varepsilon > 0\) (egal wie klein) gibt es ein
\(\delta > 0\), also wann immer

\[
0 < |x - a| < \delta,
\]

Daraus folgt

\[
|f(x) - L| < \varepsilon.
\]In Worten: Wir können \(f(x)\) so nah an \(L\) heranbringen, wie wir
möchten, vorausgesetzt, dass \(x\) nahe genug an \(a\) liegt (aber nicht
gleich \(a\)).

\subsubsection{Beispiel 1: Lineare
Funktion}\label{beispiel-1-lineare-funktion}

Zeigen Sie für \(f(x) = 2x + 1\), dass \(\lim_{x \to 3} f(x) = 7\).

\begin{itemize}
\tightlist
\item
  Wir wollen \(|f(x) - 7| < \varepsilon\).
\item
  Aber \(f(x) - 7 = 2x + 1 - 7 = 2(x - 3)\).
\item
  Also \(|f(x) - 7| = 2|x - 3|\).
\item
  Wenn wir \(\delta = \varepsilon / 2\) wählen, dann haben wir jedes
  Mal, wenn \(|x - 3| < \delta\), \(|f(x) - 7| < \varepsilon\). Dies
  beweist die Grenze.
\end{itemize}

\subsubsection{Beispiel 2: Reziproke
Funktion}\label{beispiel-2-reziproke-funktion}

Für \(f(x) = \frac{1}{x}\) sollten Sie
\(\lim_{x \to 2} f(x) = \tfrac{1}{2}\) in Betracht ziehen.

\begin{itemize}
\tightlist
\item
  Wir wollen \(\left|\frac{1}{x} - \frac{1}{2}\right| < \varepsilon\).
\item
  Diese Ungleichung erfordert algebraische Manipulation, kann aber durch
  die Wahl von \(\delta\) in Abhängigkeit von \(\varepsilon\) erfüllt
  werden. Der Prozess ist komplizierter, aber das Prinzip ist dasselbe.
\end{itemize}

\subsubsection{Warum das wichtig ist}\label{warum-das-wichtig-ist}

\begin{itemize}
\tightlist
\item
  Die Epsilon-Delta-Definition garantiert, dass Grenzwerte nicht vage
  sind oder nur auf Intuition basieren.
\item
  Es ist die Grundlage für Kontinuität, Ableitungen und Integrale.
\item
  Auch wenn Anfänger es vielleicht abstrakt finden, führt die Arbeit mit
  einfachen Beispielen zu einer Vertrautheit.
\end{itemize}

\subsubsection{Übungen}\label{uxfcbungen-3}

\begin{enumerate}
\def\labelenumi{\arabic{enumi}.}
\tightlist
\item
  Beweisen Sie mithilfe der Epsilon-Delta-Definition, dass
  \(\lim_{x \to 4} (x+1) = 5\).
\item
  Zeigen Sie, dass \(\lim_{x \to 0} 5x = 0\) unter Verwendung der
  formalen Definition gilt.
\item
  Erklären Sie, warum es \(\lim_{x \to 0} \frac{1}{x}\) nicht gibt.
\item
  Zeigen Sie für \(f(x) = x^2\), dass \(\lim_{x \to 2} f(x) = 4\) ist.
\item
  Erklären Sie mit Ihren eigenen Worten die Rolle von \(\varepsilon\)
  und \(\delta\) bei der Definition eines Limits.
\end{enumerate}

\subsection{1.5 Kontinuität}\label{kontinuituxe4t}

Eine Funktion ist stetig, wenn ihr Graph gezeichnet werden kann, ohne
den Bleistift vom Papier zu nehmen. Genauer gesagt stellt Kontinuität
sicher, dass kleine Änderungen in der Eingabe kleine Änderungen in der
Ausgabe hervorrufen.

\subsubsection{Definition}\label{definition-1}

Eine Funktion \(f\) ist an einem Punkt \(a\) stetig, wenn drei
Bedingungen erfüllt sind:

\begin{enumerate}
\def\labelenumi{\arabic{enumi}.}
\tightlist
\item
  \(f(a)\) ist definiert.
\item
  \(\lim_{x \to a} f(x)\) existiert.
\item
  \(\lim_{x \to a} f(x) = f(a)\).
\end{enumerate}

Wenn eine Funktion an jedem Punkt in einem Intervall stetig ist, sagen
wir, dass sie in diesem Intervall stetig ist.

\subsubsection{Beispiele}\label{beispiele-2}

\begin{enumerate}
\def\labelenumi{\arabic{enumi}.}
\item
  Polynomfunktionen: Funktionen wie \(f(x) = x^2 + 3x - 5\) sind überall
  auf \(\mathbb{R}\) stetig.
\item
  Rationale Funktionen: \(f(x) = \frac{1}{x-1}\) ist überall stetig,
  außer bei \(x = 1\), wo es undefiniert ist.
\item
  Stückweise Funktionen:

  \[
  f(x) =
  \begin{cases}
  x^2 & x < 1, \\
  2 & x = 1, \\
  x+1 & x > 1,
  \end{cases}
  \]

  Diese Funktion hat einen „Sprung`` bei \(x = 1\), ist also dort nicht
  kontinuierlich.
\end{enumerate}

\subsubsection{Arten von
Diskontinuitäten}\label{arten-von-diskontinuituxe4ten}

\begin{enumerate}
\def\labelenumi{\arabic{enumi}.}
\tightlist
\item
  Entfernbare Diskontinuität: Ein „Loch`` im Diagramm. Beispiel:
  \(f(x) = \frac{x^2-1}{x-1}\) zu \(x=1\).2. Sprungdiskontinuität: Die
  linken und rechten Grenzen sind unterschiedlich.
\item
  Unendliche Diskontinuität: Die Funktion geht zu \(\pm\infty\) in der
  Nähe eines Punktes, wie bei \(f(x) = 1/x\) in der Nähe von \(x = 0\).
\end{enumerate}

\subsubsection{Der Zwischenwertsatz}\label{der-zwischenwertsatz}

Wenn eine Funktion in einem Intervall \([a, b]\) stetig ist, dann gibt
es für jede Zahl \(N\) zwischen \(f(a)\) und \(f(b)\) ein
\(c \in [a, b]\) mit \(f(c) = N\).

Diese Eigenschaft ist entscheidend für den Nachweis der Existenz von
Wurzeln und Lösungen von Gleichungen.

\subsubsection{Übungen}\label{uxfcbungen-4}

\begin{enumerate}
\def\labelenumi{\arabic{enumi}.}
\tightlist
\item
  Entscheiden Sie, ob die Funktion \(f(x) = |x|\) bei \(x = 0\) stetig
  ist.
\item
  Identifizieren Sie die Diskontinuitätspunkte für
  \(f(x) = \frac{x+2}{x^2-1}\).
\item
  Erklären Sie, warum jede Polynomfunktion überall stetig ist.
\item
  Geben Sie ein Beispiel für eine Funktion mit einer Sprungunstetigkeit.
  Skizzieren Sie dessen Diagramm.
\item
  Verwenden Sie den Zwischenwertsatz, um zu zeigen, dass die Gleichung
  \(x^3 + x - 1 = 0\) eine Lösung zwischen 0 und 1 hat.
\end{enumerate}

\section{Kapitel 2. Derivate}\label{kapitel-2.-derivate}

\subsection{2.1 Die Ableitung als
Änderungsrate}\label{die-ableitung-als-uxe4nderungsrate}

Die Ableitung ist eine der zentralen Ideen der Analysis. Es misst, wie
sich eine Funktion ändert, wenn sich ihre Eingabe ändert -- mit anderen
Worten, die Änderungsrate der Ausgabe im Verhältnis zur Eingabe.

\subsubsection{Durchschnittliche
Änderungsrate}\label{durchschnittliche-uxe4nderungsrate}

Für eine Funktion \(f(x)\) beträgt die durchschnittliche Änderungsrate
zwischen zwei Punkten \(x = a\) und \(x = b\)

\[
\frac{f(b) - f(a)}{b - a}.
\]

Dies ist die Steigung der Sekantenlinie durch die Punkte \((a, f(a))\)
und \((b, f(b))\).

\subsubsection{Momentane
Änderungsrate}\label{momentane-uxe4nderungsrate}

Um zu messen, wie schnell sich \(f(x)\) an einem einzelnen Punkt ändert,
lassen wir das Intervall schrumpfen:

\[
f'(a) = \lim_{h \to 0} \frac{f(a+h) - f(a)}{h}.
\]

Diese Grenze, sofern vorhanden, wird als Ableitung von \(f\) zu \(a\)
bezeichnet. Geometrisch gesehen ist es die Steigung der Tangente an den
Graphen von \(f\) am Punkt \((a, f(a))\).

\subsubsection{Notation}\label{notation}

\begin{itemize}
\tightlist
\item
  \(f'(x)\): Primzahlschreibweise.
\item
  \(\dfrac{dy}{dx}\): Leibniz-Notation, verwendet bei \(y = f(x)\).
\item
  \(Df(x)\): Operatornotation.
\end{itemize}

Alle diese Symbole beziehen sich auf dasselbe Konzept.

\subsubsection{Beispiele}\label{beispiele-3}

\begin{enumerate}
\def\labelenumi{\arabic{enumi}.}
\item
  Für \(f(x) = x^2\):

  \[
  f'(x) = \lim_{h \to 0} \frac{(x+h)^2 - x^2}{h} = \lim_{h \to 0} \frac{2xh + h^2}{h} = 2x.
  \]

  Die Steigung der Parabel beträgt bei \(x\) \(2x\).
\item
  Für \(f(x) = \sin x\):

  \[
  f'(x) = \cos x.
  \]
\item
  Für \(f(x) = c\) (eine Konstante):

  \[
  f'(x) = 0.
  \]

  Eine konstante Funktion ändert sich nie.
\end{enumerate}

\subsubsection{Interpretation}\label{interpretation}

\begin{itemize}
\tightlist
\item
  In der Physik: Wenn \(s(t)\) die Position ist, dann ist \(s'(t)\) die
  Geschwindigkeit.
\item
  In der Wirtschaftswissenschaft: Wenn \(C(x)\) Kosten sind, dann sind
  \(C'(x)\) Grenzkosten.
\item
  In der Biologie: Wenn \(P(t)\) die Bevölkerung ist, dann ist \(P'(t)\)
  die Wachstumsrate.
\end{itemize}

Die Ableitung präzisiert „Veränderung`` in vielen Zusammenhängen.

\subsubsection{Übungen}\label{uxfcbungen-5}

\begin{enumerate}
\def\labelenumi{\arabic{enumi}.}
\tightlist
\item
  Berechnen Sie \(f'(x)\) für \(f(x) = 3x^2 - 2x + 1\).2. Finden Sie die
  Steigung der Tangente an \(f(x) = x^3\) bei \(x = 2\).
\item
  Wenn \(s(t) = t^2 + 2t\) die Entfernung in Metern darstellt, wie groß
  ist dann die Geschwindigkeit bei \(t = 5\)?
\item
  Verwenden Sie die Grenzwertdefinition, um die Ableitung von
  \(f(x) = \frac{1}{x}\) zu berechnen.
\item
  Skizzieren Sie den Graphen von \(y = x^2\) und zeichnen Sie die
  Tangente bei \(x = 1\).
\end{enumerate}

\subsection{2.2 Differenzierungsregeln}\label{differenzierungsregeln}

Sobald die Ableitung definiert ist, benötigen wir effiziente Methoden,
um sie zu berechnen. Die Differenzierungsregeln sind Abkürzungen, die
uns die wiederholte Anwendung der Grenzwertdefinition ersparen.

\subsubsection{Die konstante Regel}\label{die-konstante-regel}

Wenn \(f(x) = c\) wobei \(c\) eine Konstante ist, dann

\[
f'(x) = 0.
\]

\subsubsection{Die Machtregel}\label{die-machtregel}

Für \(f(x) = x^n\), wobei \(n\) eine reelle Zahl ist,

\[
\frac{d}{dx} \big( x^n \big) = n x^{n-1}.
\]

Beispiele:

\begin{itemize}
\tightlist
\item
  \(\frac{d}{dx}(x^2) = 2x\).
\item
  \(\frac{d}{dx}(x^5) = 5x^4\).
\item
  \(\frac{d}{dx}(\sqrt{x}) = \frac{1}{2\sqrt{x}}\).
\end{itemize}

\subsubsection{Die konstante
Vielfachesregel}\label{die-konstante-vielfachesregel}

Wenn \(f(x) = c \cdot g(x)\), dann

\[
f'(x) = c \cdot g'(x).
\]

\subsubsection{Die Summen- und
Differenzregeln}\label{die-summen--und-differenzregeln}

\begin{itemize}
\tightlist
\item
  \((f + g)' = f' + g'\).
\item
  \((f - g)' = f' - g'\).
\end{itemize}

\subsubsection{Die Produktregel}\label{die-produktregel}

Für \(f(x)\) und \(g(x)\):

\[
(fg)' = f'g + fg'.
\]

Beispiel: Wenn \(f(x) = x^2\), \(g(x) = \sin x\):

\[
(fg)' = (2x)(\sin x) + (x^2)(\cos x).
\]

\subsubsection{Die Quotientenregel}\label{die-quotientenregel}

Für \(f(x)\) und \(g(x)\):

\[
\left(\frac{f}{g}\right)' = \frac{f'g - fg'}{g^2}, \quad g(x) \neq 0.
\]

Beispiel: Wenn \(f(x) = x^2\), \(g(x) = x+1\):

\[
\left(\frac{x^2}{x+1}\right)' = \frac{(2x)(x+1) - (x^2)(1)}{(x+1)^2}.
\]

\subsubsection{Ableitungen gemeinsamer
Funktionen}\label{ableitungen-gemeinsamer-funktionen}

\begin{itemize}
\tightlist
\item
  \(\frac{d}{dx}(\sin x) = \cos x\).
\item
  \(\frac{d}{dx}(\cos x) = -\sin x\).
\item
  \(\frac{d}{dx}(e^x) = e^x\).
\item
  \(\frac{d}{dx}(\ln x) = \frac{1}{x}, \quad x > 0\).
\end{itemize}

\subsubsection{Übungen}\label{uxfcbungen-6}

\begin{enumerate}
\def\labelenumi{\arabic{enumi}.}
\tightlist
\item
  Differenzieren Sie \(f(x) = 7x^3 - 4x + 9\).
\item
  Verwenden Sie die Produktregel, um die Ableitung von
  \(f(x) = x^2 e^x\) zu finden.
\item
  Wenden Sie die Quotientenregel auf \(f(x) = \frac{\sin x}{x}\) an.
\item
  Berechnen Sie \(\frac{d}{dx}(\ln(x^2))\) anhand der Regelkette.
\item
  Zeigen Sie, dass die Ableitung von \(f(x) = \frac{1}{x}\)
  \(-\frac{1}{x^2}\) ist.
\end{enumerate}

\subsection{2.3 Die Kettenregel}\label{die-kettenregel}

Funktionen werden häufig durch die Kombination einfacherer Funktionen
erstellt. Um solche zusammengesetzten Funktionen zu unterscheiden,
verwenden wir die Kettenregel.

\subsubsection{Die Regel}\label{die-regel}

Wenn \(y = f(g(x))\), dann

\[
\frac{dy}{dx} = f'(g(x)) \cdot g'(x).
\]

In Worten: Differenzieren Sie die äußere Funktion, lassen Sie die innere
unverändert und multiplizieren Sie sie dann mit der Ableitung der
inneren Funktion.

\subsubsection{Beispiele}\label{beispiele-4}

\begin{enumerate}
\def\labelenumi{\arabic{enumi}.}
\item
  Quadrat einer linearen Funktion

  \[
  y = (3x+2)^2
  \]

  Äußere Funktion: \(f(u) = u^2\), innere Funktion: \(g(x) = 3x+2\).

  \[
  y' = 2(3x+2) \cdot 3 = 6(3x+2).
  \]
\item
  Exponential mit quadratischem Inneren

  \[
  y = e^{x^2}
  \]

  Äußere Funktion: \(f(u) = e^u\), innere Funktion: \(g(x) = x^2\).

  \[y' = e^{x^2} \cdot 2x = 2x e^{x^2}.
  \]
\item
  Logarithm with root inside

  \[
  y = \ln(\sqrt{x})
  \]

  Outer: \(f(u) = \ln u\), inner: \(g(x) = \sqrt{x}\).

  \[
  y' = \frac{1}{\sqrt{x}} \cdot \frac{1}{2\sqrt{x}} = \frac{1}{2x}.
  \]
\end{enumerate}

\subsubsection{Generalized Chain Rule}\label{generalized-chain-rule}

For multiple nested functions \(y = f(g(h(x)))\):

\[
\frac{dy}{dx} = f'(g(h(x))) \cdot g'(h(x)) \cdot h'(x).
\]

This extends naturally to deeper compositions.

\subsubsection{Why the Chain Rule
Matters}\label{why-the-chain-rule-matters}

\begin{itemize}
\tightlist
\item
  It handles nearly all real-world models where one quantity depends on
  another indirectly.
\item
  It connects calculus with physics (e.g., velocity depending on time
  through position).
\item
  It is essential in implicit differentiation and advanced topics.
\end{itemize}

\subsubsection{Exercises}\label{exercises}

\begin{enumerate}
\def\labelenumi{\arabic{enumi}.}
\tightlist
\item
  Differentiate \(y = (5x^2 + 1)^3\).
\item
  Find \(\frac{d}{dx}(\sin(3x))\).
\item
  Compute \(\frac{d}{dx}(\ln(1+x^2))\).
\item
  Differentiate \(y = \cos^2(x)\).
\item
  Apply the generalized chain rule to \(y = e^{\sin(x^2)}\).
\end{enumerate}

\subsection{2.4 Implicit
Differentiation}\label{implicit-differentiation}

Not all functions are given in the form \(y = f(x)\). Sometimes \(x\)
and \(y\) are related by an equation, and solving explicitly for \(y\)
is difficult or impossible. In such cases, we use implicit
differentiation.

\subsubsection{The Idea}\label{the-idea}

If an equation involves both \(x\) and \(y\), we can differentiate both
sides with respect to \(x\), treating \(y\) as a function of \(x\). Each
time we differentiate a term involving \(y\), we multiply by
\(\frac{dy}{dx}\).

\subsubsection{Example 1: A Circle}\label{example-1-a-circle}

Equation:

\[
x^2 + y^2 = 25
\]

Differentiate with respect to \(x\):

\[
2x + 2y \frac{dy}{dx} = 0.
\]

Solve for \(\frac{dy}{dx}\):

\[
\frac{dy}{dx} = -\frac{x}{y}.
\]

This gives the slope of the tangent to the circle at any point.

\subsubsection{Example 2: A Product of
Variables}\label{example-2-a-product-of-variables}

Equation:

\[
xy = 1
\]

Differentiate:

\[
x \frac{dy}{dx} + y = 0.
\]

So,

\[
\frac{dy}{dx} = -\frac{y}{x}.
\]

\subsubsection{Example 3: Trigonometric
Relation}\label{example-3-trigonometric-relation}

Equation:

\[
\sin(xy) = x
\]

Differentiate:

\[
\cos(xy) \cdot \Big(y + x\frac{dy}{dx}\Big) = 1.
\]

Solve for \(\frac{dy}{dx}\):

\[
\frac{dy}{dx} = \frac{1 - y\cos(xy)}{x\cos(xy)}.
\]

\subsubsection{Warum implizite Differenzierung nützlich
ist}\label{warum-implizite-differenzierung-nuxfctzlich-ist}

\begin{itemize}
\tightlist
\item
  Viele wichtige Kurven (Kreise, Ellipsen, Hyperbeln) sind natürlich
  implizit definiert.
\item
  Es ermöglicht uns, Gleichungen zu differenzieren, ohne zuerst nach
  \(y\) aufzulösen. -- Dies ist ein wichtiger Schritt in
  fortgeschritteneren Themen wie verwandten Raten und
  Differentialgleichungen.
\end{itemize}

\subsubsection{Übungen}\label{uxfcbungen-7}

\begin{enumerate}
\def\labelenumi{\arabic{enumi}.}
\tightlist
\item
  Suchen Sie für die Kurve \(x^2 + xy + y^2 = 7\) nach
  \(\frac{dy}{dx}\).
\item
  Differenzieren Sie \(\cos(x) + \cos(y) = 1\) implizit.
\item
  Ermitteln Sie die Steigung der Tangente an \(x^3 + y^3 = 9\) am Punkt
  \((1, 2)\).4. Berechnen Sie bei gegebenem \(x^2 + y^2 = 10\)
  \(\frac{dy}{dx}\), wenn \((x, y) = (1, 3)\).
\item
  Differenzieren Sie \(e^{xy} = x + y\), um \(\frac{dy}{dx}\) zu finden.
\end{enumerate}

\subsection{2.5 Derivate höherer
Ordnung}\label{derivate-huxf6herer-ordnung}

Bisher haben wir die erste Ableitung untersucht, die die Änderungsrate
einer Funktion misst. Aber auch Derivate selbst können differenziert
werden, wodurch Derivate höherer Ordnung entstehen.

\subsubsection{Definition}\label{definition-2}

\begin{itemize}
\item
  Die zweite Ableitung von \(f\) ist die Ableitung der Ableitung:

  \[
  f''(x) = \frac{d}{dx}\left(f'(x)\right).
  \]
\item
  Allgemeiner wird die \(n\)-te Ableitung geschrieben als

  \[
  f^{(n)}(x) = \frac{d^n}{dx^n} f(x).
  \]
\end{itemize}

\subsubsection{Beispiele}\label{beispiele-5}

\begin{enumerate}
\def\labelenumi{\arabic{enumi}.}
\item
  \(f(x) = x^3\)

  \begin{itemize}
  \tightlist
  \item
    Erste Ableitung: \(f'(x) = 3x^2\).
  \item
    Zweite Ableitung: \(f''(x) = 6x\).
  \item
    Dritte Ableitung: \(f^{(3)}(x) = 6\).
  \item
    Vierte Ableitung: \(f^{(4)}(x) = 0\).
  \end{itemize}
\item
  \(f(x) = \sin x\)

  \begin{itemize}
  \tightlist
  \item
    \(f'(x) = \cos x\).
  \item
    \(f''(x) = -\sin x\).
  \item
    \(f^{(3)}(x) = -\cos x\).
  \item
    \(f^{(4)}(x) = \sin x\). Die Ableitungen wiederholen sich in einem
    Zyklus der Länge 4.
  \end{itemize}
\item
  \(f(x) = e^x\)

  \begin{itemize}
  \tightlist
  \item
    Jedes Derivat kostet \(e^x\).
  \end{itemize}
\end{enumerate}

\subsubsection{Anwendungen}\label{anwendungen}

\begin{itemize}
\item
  Konkavität: Das Vorzeichen von \(f''(x)\) gibt an, ob der Graph von
  \(f\) konkav nach oben (\(f'' > 0\)) oder konkav nach unten
  (\(f'' < 0\)) ist.
\item
  Wendepunkte: Punkte, an denen sich die Konkavität ändert.
\item
  Bewegung: In der Physik, wenn \(s(t)\) die Position ist:

  \begin{itemize}
  \tightlist
  \item
    \(s'(t)\) = Geschwindigkeit,
  \item
    \(s''(t)\) = Beschleunigung,
  \item
    \(s^{(3)}(t)\) = Ruck (Änderungsrate der Beschleunigung).
  \end{itemize}
\item
  Näherungen: Ableitungen höherer Ordnung erscheinen in Taylor-Reihen
  und werden zur Approximation von Funktionen verwendet.
\end{itemize}

\subsubsection{Übungen}\label{uxfcbungen-8}

\begin{enumerate}
\def\labelenumi{\arabic{enumi}.}
\tightlist
\item
  Berechnen Sie die ersten vier Ableitungen von \(f(x) = \cos x\).
\item
  Finden Sie \(f''(x)\) für \(f(x) = x^4 - 2x^2 + 3\).
\item
  Zeigen Sie für \(f(x) = e^{2x}\), dass \(f^{(n)}(x) = 2^n e^{2x}\).
\item
  Bestimmen Sie die Intervalle, in denen \(f(x) = x^3 - 3x\) nach oben
  und nach unten konkav ist.
\item
  Wenn \(s(t) = t^3 - 6t^2 + 9t\), ermitteln Sie die Geschwindigkeit und
  Beschleunigung bei \(t = 2\).
\end{enumerate}

\section{Kapitel 3. Anwendungen von
Derivaten}\label{kapitel-3.-anwendungen-von-derivaten}

\subsection{3.1 Tangenten und Normalen}\label{tangenten-und-normalen}

Eine der ersten Anwendungen von Ableitungen besteht darin, die
Gleichungen von Tangenten und Normalen an eine Kurve zu finden. Diese
Linien erfassen die lokale Geometrie einer Funktion an einem bestimmten
Punkt.

\subsubsection{Tangente}\label{tangente}

Die Tangente an eine Kurve \(y = f(x)\) an einem Punkt \((a, f(a))\) ist
die Linie, die den Graphen dort gerade „berührt`` und die gleiche
Steigung wie die Kurve hat.

Die Steigung der Tangente ergibt sich aus der Ableitung:

\[
m_{\text{tangent}} = f'(a).
\]

Somit lautet die Gleichung der Tangente bei \((a, f(a))\)

\[
y - f(a) = f'(a)(x - a).
\]

\subsubsection{Normale Linie}\label{normale-linie}

Die Normale steht im selben Punkt senkrecht zur Tangente. Seine Steigung
ist der negative Kehrwert der Tangentensteigung:

\[m_{\text{normal}} = -\frac{1}{f'(a)}.
\]

So the equation of the normal line is

\[
y - f(a) = -\frac{1}{f'(a)} (x - a), \quad f'(a) \neq 0.
\]

\subsubsection{Examples}\label{examples}

\begin{enumerate}
\def\labelenumi{\arabic{enumi}.}
\item
  \(f(x) = x^2\) at \(x = 1\).

  \begin{itemize}
  \tightlist
  \item
    \(f(1) = 1\), \(f'(x) = 2x\), so \(f'(1) = 2\).
  \item
    Tangent: \(y - 1 = 2(x - 1)\), or \(y = 2x - 1\).
  \item
    Normal: slope = \(-\tfrac{1}{2}\), so equation is
    \(y - 1 = -\tfrac{1}{2}(x - 1)\).
  \end{itemize}
\item
  \(f(x) = \sin x\) at \(x = \tfrac{\pi}{4}\).

  \begin{itemize}
  \tightlist
  \item
    \(f(\tfrac{\pi}{4}) = \tfrac{\sqrt{2}}{2}\),
    \(f'(\tfrac{\pi}{4}) = \cos(\tfrac{\pi}{4}) = \tfrac{\sqrt{2}}{2}\).
  \item
    Tangent:
    \(y - \tfrac{\sqrt{2}}{2} = \tfrac{\sqrt{2}}{2}(x - \tfrac{\pi}{4})\).
  \end{itemize}
\end{enumerate}

\subsubsection{Why Tangents and Normals
Matter}\label{why-tangents-and-normals-matter}

\begin{itemize}
\tightlist
\item
  Tangents approximate the curve locally (linear approximation).
\item
  Normals are useful in geometry, optics (reflection/refraction), and
  mechanics (force directions).
\item
  Both play a role in optimization and curvature studies.
\end{itemize}

\subsubsection{Exercises}\label{exercises-1}

\begin{enumerate}
\def\labelenumi{\arabic{enumi}.}
\tightlist
\item
  Find the tangent and normal lines to \(y = x^3\) at \(x = 2\).
\item
  Determine the tangent and normal lines to \(y = e^x\) at \(x = 0\).
\item
  For \(y = \ln x\), compute the tangent line at \(x = 1\).
\item
  A circle is given by \(x^2 + y^2 = 9\). Use implicit differentiation
  to find the slope of the tangent at \((0,3)\).
\item
  Sketch the graph of \(y = \sqrt{x}\) and draw the tangent and normal
  lines at \(x = 4\).
\end{enumerate}

\subsection{3.2 Related Rates}\label{related-rates}

In many real-world problems, two or more quantities change with respect
to time, and their rates of change are connected. Related rates problems
use derivatives to describe these relationships.

\subsubsection{General Approach}\label{general-approach}

\begin{enumerate}
\def\labelenumi{\arabic{enumi}.}
\tightlist
\item
  Identify the variables that depend on time \(t\).
\item
  Write an equation relating the variables.
\item
  Differentiate both sides with respect to \(t\), applying the chain
  rule.
\item
  Substitute the known values at the given instant.
\item
  Solve for the unknown rate.
\end{enumerate}

\subsubsection{Example 1: Expanding
Circle}\label{example-1-expanding-circle}

A circle has radius \(r\), which increases at the rate of
\(\frac{dr}{dt} = 2 \,\text{cm/s}\). Find the rate at which the area
\(A = \pi r^2\) increases when \(r = 5\).

Differentiate:

\[
\frac{dA}{dt} = 2\pi r \frac{dr}{dt}.
\]

Substitute:

\[
\frac{dA}{dt} = 2\pi (5)(2) = 20\pi \,\text{cm}^2/\text{s}.
\]

\subsubsection{Example 2: Sliding
Ladder}\label{example-2-sliding-ladder}

A 10 ft ladder leans against a wall. The bottom slides away at
\(\frac{dx}{dt} = 1 \,\text{ft/s}\). How fast is the top sliding down
when the bottom is 6 ft from the wall?

Equation: \(x^2 + y^2 = 100\), where \(y\) is the height.

Differentiate:

\[
2x \frac{dx}{dt} + 2y \frac{dy}{dt} = 0.
\]

At \(x = 6\), \(y = 8\). Substitute:

\[
2(6)(1) + 2(8)\frac{dy}{dt} = 0 \quad \Rightarrow \quad \frac{dy}{dt} = -\tfrac{6}{8} = -\tfrac{3}{4}.
\]Die Oberseite gleitet also bei \(0.75 \,\text{ft/s}\) nach unten.

\subsubsection{Beispiel 3: Wasser in einem
Kegel}\label{beispiel-3-wasser-in-einem-kegel}

Wasser wird in einen Kegel mit einer Höhe von 12 cm und einem Radius von
6 cm gegossen. Bei einer Wassertiefe von 4 cm steigt der Wasserstand um
\(2 \,\text{cm/s}\). Mit welcher Geschwindigkeit nimmt die Lautstärke
zu?

Gleichung: \(V = \tfrac{1}{3}\pi r^2 h\). Mit Ähnlichkeit,
\(r = \tfrac{h}{2}\). Ersetzen:

\[
V = \tfrac{1}{12}\pi h^3.
\]

Unterscheiden:

\[
\frac{dV}{dt} = \tfrac{1}{4}\pi h^2 \frac{dh}{dt}.
\]

Bei \(h = 4\), \(\frac{dh}{dt} = 2\):

\[
\frac{dV}{dt} = \tfrac{1}{4}\pi (16)(2) = 8\pi \,\text{cm}^3/\text{s}.
\]

\subsubsection{Warum verwandte Tarife wichtig
sind}\label{warum-verwandte-tarife-wichtig-sind}

\begin{itemize}
\tightlist
\item
  Sie beschreiben Bewegung und Veränderung in der Physik, Technik und
  Biologie.
\item
  Sie verbinden Geometrie mit Analysis durch zeitabhängige Prozesse.
\item
  Sie schulen uns darin, dynamische Systeme mathematisch zu modellieren.
\end{itemize}

\subsubsection{Übungen}\label{uxfcbungen-9}

\begin{enumerate}
\def\labelenumi{\arabic{enumi}.}
\tightlist
\item
  Ein Ballon wird so aufgeblasen, dass sein Radius um
  \(0.5 \,\text{cm/s}\) zunimmt. Finden Sie heraus, wie schnell sein
  Volumen zunimmt, wenn der Radius 10 cm beträgt.
\item
  Ein Auto fährt mit 40 km/h nach Norden und ein anderes mit 30 km/h
  nach Osten. Wie schnell vergrößert sich der Abstand zwischen ihnen 2
  Stunden später?
\item
  Ein 20 m von einer Wand entfernter Scheinwerfer beleuchtet einen 2 m
  großen Mann, der sich mit 1,5 m/s entfernt. Wie schnell ändert sich
  die Länge seines Schattens an der Wand, wenn er 5 m vom Licht entfernt
  ist?
\item
  Die Seitenlänge eines Würfels wächst mit 2 cm/s. Wie schnell
  vergrößert sich die Oberfläche bei einer Seitenlänge von 3 cm?
\item
  Sand wird auf einen Haufen gegossen, der einen Kegel bildet, dessen
  Radius immer gleich der Höhe ist. Wenn die Höhe mit 5 cm/s zunimmt,
  wie schnell nimmt das Volumen zu, wenn die Höhe 10 cm beträgt?
\end{enumerate}

\subsection{3.3 Optimierungsprobleme}\label{optimierungsprobleme}

Bei Optimierungsproblemen werden Ableitungen verwendet, um die Maximal-
oder Minimalwerte einer Funktion zu ermitteln, häufig unter bestimmten
Einschränkungen. Diese Probleme modellieren Situationen, in denen wir
Effizienz, Gewinn oder Fläche maximieren oder Kosten, Entfernung oder
Zeit minimieren möchten.

\subsubsection{Allgemeine Schritte}\label{allgemeine-schritte}

\begin{enumerate}
\def\labelenumi{\arabic{enumi}.}
\tightlist
\item
  Verstehen Sie das Problem: Identifizieren Sie die zu optimierende
  Menge.
\item
  Modell mit einer Funktion: Schreiben Sie die Zielfunktion anhand einer
  Variablen.
\item
  Wenden Sie Einschränkungen an: Verwenden Sie gegebene Bedingungen, um
  Variablen zu reduzieren.
\item
  Differenzieren: Berechnen Sie die Ableitung der Zielfunktion.
\item
  Finden Sie kritische Punkte: Lösen Sie \(f'(x) = 0\) oder wo \(f'(x)\)
  undefiniert ist.
\item
  Test auf Maxima/Minima: Verwenden Sie den Test der zweiten Ableitung
  oder überprüfen Sie die Endpunkte.
\item
  Interpretieren Sie das Ergebnis: Geben Sie die Antwort im
  ursprünglichen Kontext an.
\end{enumerate}

\subsubsection{Beispiel 1: Maximale Fläche eines
Rechtecks}\label{beispiel-1-maximale-fluxe4che-eines-rechtecks}

Ein Rechteck hat einen Umfang von 40. Welche Abmessungen maximieren
seine Fläche?

\begin{itemize}
\tightlist
\item
  Länge \(x\), Breite \(y\). Einschränkung:
  \(2x + 2y = 40 \Rightarrow y = 20 - x\).
\item
  Fläche: \(A = xy = x(20 - x) = 20x - x^2\).- Ableitung:
  \(A'(x) = 20 - 2x\). Gleich 0 setzen: \(x = 10\).
\item
  Dann \(y = 10\).
\item
  Maximale Fläche: \(100\). Das Rechteck ist ein Quadrat.
\end{itemize}

\subsubsection{Beispiel 2: Entfernung
minimieren}\label{beispiel-2-entfernung-minimieren}

Finden Sie den Punkt auf der Parabel \(y = x^2\), der \((0,3)\) am
nächsten liegt.

\begin{itemize}
\tightlist
\item
  Entfernung im Quadrat: \(D(x) = (x-0)^2 + (x^2 - 3)^2\).
\item
  Erweitern:
  \(D(x) = x^2 + (x^2 - 3)^2 = x^2 + x^4 - 6x^2 + 9 = x^4 - 5x^2 + 9\).
\item
  Ableitung: \(D'(x) = 4x^3 - 10x\). Lösen: \(x(4x^2 - 10) = 0\).
\item
  Lösungen: \(x = 0\), \(x = \pm \sqrt{2.5}\).
\item
  Bei der Prüfung liegt der Mindestabstand bei \(x = \pm \sqrt{2.5}\).
\end{itemize}

\subsubsection{Beispiel 3: Box mit maximalem
Volumen}\label{beispiel-3-box-mit-maximalem-volumen}

Eine Schachtel ohne Deckel wird aus einem quadratischen Stück Pappe mit
einer Seitenlänge von 20 cm hergestellt, indem man an den Ecken gleich
große Quadrate ausschneidet und die Seiten hochfaltet. Finden Sie die
Schnittgröße, die das Volumen maximiert.

\begin{itemize}
\tightlist
\item
  Schnittgröße = \(x\). Dann Maße:
  \((20 - 2x) \times (20 - 2x) \times x\).
\item
  Volumen: \(V(x) = x(20 - 2x)^2\).
\item
  Derivat: \(V'(x) = (20 - 2x)(20 - 6x)\).
\item
  Kritische Punkte: \(x = 10\) (ergibt Nullvolumen) oder
  \(x = \tfrac{20}{6} \approx 3.33\).
\item
  Bei \(x \approx 3.33\) ist die Lautstärke maximiert.
\end{itemize}

\subsubsection{Warum Optimierung wichtig
ist}\label{warum-optimierung-wichtig-ist}

\begin{itemize}
\tightlist
\item
  Ingenieure nutzen es, um effiziente Strukturen zu entwerfen.
\item
  Unternehmen nutzen es, um den Gewinn zu maximieren oder die Kosten zu
  minimieren.
\item
  Wissenschaftler nutzen es, um natürliche Systeme zu modellieren, die
  ein Gleichgewicht anstreben.
\end{itemize}

\subsubsection{Übungen}\label{uxfcbungen-10}

\begin{enumerate}
\def\labelenumi{\arabic{enumi}.}
\tightlist
\item
  Ein Bauer verfügt über einen 100 m langen Zaun, um ein rechteckiges
  Feld entlang eines Flusses einzuzäunen (es müssen also nur drei Seiten
  eingezäunt werden). Finden Sie Abmessungen, die den Bereich
  maximieren.
\item
  Finden Sie zwei positive Zahlen, deren Summe 20 ist und deren Produkt
  möglichst groß ist.
\item
  Ein Zylinder soll aus 100 cm²20¤¤ Material gefertigt werden. Finden
  Sie Abmessungen mit maximalem Volumen.
\item
  Ein 10 m langer Draht wird in zwei Stücke geschnitten, eines zu einem
  Quadrat, das andere zu einem Kreis gebogen. Wie sollte es geschnitten
  werden, um die umschlossene Gesamtfläche zu maximieren?
\item
  Es soll eine geschlossene Box mit quadratischer Grundfläche und einem
  Volumen von 32 m\(^3\) gebaut werden. Finden Sie Abmessungen, die die
  Oberfläche minimieren.
\end{enumerate}

\subsection{3.4 Konkavität und
Wendepunkte}\label{konkavituxe4t-und-wendepunkte}

Ableitungen geben uns nicht nur Aufschluss über Steigungen, sondern auch
über die Form eines Graphen. Die zweite Ableitung ist besonders
nützlich, um die Konkavität zu verstehen und Wendepunkte zu
identifizieren.

\subsubsection{Konkavität}\label{konkavituxe4t}

\begin{itemize}
\item
  Eine Funktion \(f(x)\) ist auf einem Intervall konkav, wenn
  \(f''(x) > 0\). Der Graph biegt sich nach oben, wie eine Tasse.
\item
  Eine Funktion \(f(x)\) ist in einem Intervall konkav, wenn
  \(f''(x) < 0\). Die Grafik neigt sich nach unten, wie ein
  Stirnrunzeln.
\end{itemize}

Konkavität beschreibt, wie sich die Steigung einer Funktion ändert: Wenn
die Steigungen zunehmen, ist der Graph nach oben konkav; Wenn die
Steigungen abnehmen, ist der Graph nach unten konkav.

\subsubsection{Wendepunkte}\label{wendepunkte}

Ein Wendepunkt ist ein Punkt im Diagramm, an dem sich die Konkavität
ändert.- Wenn \(f''(x) = 0\) oder \(f''(x)\) nicht definiert ist, ist
der Punkt ein Kandidat für einen Wendepunkt. - Zur Bestätigung muss die
Konkavität auf beiden Seiten des Punktes ihr Vorzeichen ändern.

\subsubsection{Beispiele}\label{beispiele-6}

\begin{enumerate}
\def\labelenumi{\arabic{enumi}.}
\item
  \(f(x) = x^3\)

  \begin{itemize}
  \tightlist
  \item
    \(f''(x) = 6x\).
  \item
    Bei \(x = 0\), \(f''(0) = 0\).
  \item
    Für \(x < 0\), \(f''(x) < 0\) → konkav nach unten.
  \item
    Für \(x > 0\), \(f''(x) > 0\) → konkav nach oben.
  \item
    Somit ist \((0,0)\) ein Wendepunkt.
  \end{itemize}
\item
  \(f(x) = x^4\)

  \begin{itemize}
  \tightlist
  \item
    \(f''(x) = 12x^2\).
  \item
    Bei \(x = 0\), \(f''(0) = 0\), aber die Konkavität ändert das
    Vorzeichen nicht (immer ≥ 0).
  \item
    Kein Wendepunkt.
  \end{itemize}
\end{enumerate}

\subsubsection{Konkavität und
Kurvenskizze}\label{konkavituxe4t-und-kurvenskizze}

\begin{itemize}
\tightlist
\item
  Wenn \(f'(x) = 0\) und \(f''(x) > 0\), dann hat \(f\) ein lokales
  Minimum.
\item
  Wenn \(f'(x) = 0\) und \(f''(x) < 0\), dann hat \(f\) ein lokales
  Maximum.
\item
  Dies wird als Test der zweiten Ableitung bezeichnet.
\end{itemize}

\subsubsection{Warum das wichtig ist}\label{warum-das-wichtig-ist-1}

Konkavität und Wendepunkte helfen uns, die „Form`` von Graphen zu
verstehen: wo sie sich biegen, abflachen oder drehen. Diese Ideen
spielen eine zentrale Rolle beim Zeichnen von Kurven, in der Physik
(Beschleunigung) und in der Wirtschaft (abnehmende Erträge).

\subsubsection{Übungen}\label{uxfcbungen-11}

\begin{enumerate}
\def\labelenumi{\arabic{enumi}.}
\tightlist
\item
  Bestimmen Sie die Konkavitätsintervalle für \(f(x) = x^3 - 3x\).
  Finden Sie seine Wendepunkte.
\item
  Identifizieren Sie für \(f(x) = \ln(x)\) die Konkavität und mögliche
  Wendepunkte.
\item
  Wenden Sie den Test der zweiten Ableitung auf \(f(x) = x^2 e^{-x}\)
  an, um kritische Punkte zu klassifizieren.
\item
  Skizzieren Sie \(f(x) = \sin x\) und markieren Sie die
  Konkavitätsintervalle und Wendepunkte.
\item
  Erklären Sie, warum \(f(x) = e^x\) keine Wendepunkte hat.
\end{enumerate}

\subsection{3.5 Kurvenskizze}\label{kurvenskizze}

Beim Kurvenskizzieren wird der Graph einer Funktion unter Verwendung von
Informationen aus ihren Ableitungen gezeichnet. Anstatt viele Punkte
darzustellen, analysieren wir Schlüsselmerkmale: Achsenabschnitte,
Asymptoten, zunehmende/abfallende Intervalle und Konkavität.

\subsubsection{Schritte zum Skizzieren von
Kurven}\label{schritte-zum-skizzieren-von-kurven}

\begin{enumerate}
\def\labelenumi{\arabic{enumi}.}
\item
  Domäne: Identifizieren Sie, wo die Funktion definiert ist.
\item
  Achsenabschnitte: Finden Sie heraus, wo der Graph die Achsen
  schneidet.
\item
  Asymptoten:

  \begin{itemize}
  \tightlist
  \item
    Vertikale Asymptoten treten auf, wenn die Funktion undefiniert ist
    und gegen Unendlich tendiert.
  \item
    Horizontale oder schräge Asymptoten beschreiben das Endverhalten als
    \(x \to \pm\infty\).
  \end{itemize}
\item
  Erste Ableitung \(f'(x)\):

  \begin{itemize}
  \tightlist
  \item
    Positiv → Funktion nimmt zu.
  \item
    Negativ → Funktion nimmt ab.
  \item
    Nullstellen von \(f'(x)\) → kritische Punkte (mögliche
    Maxima/Minima).
  \end{itemize}
\item
  Zweite Ableitung \(f''(x)\):

  \begin{itemize}
  \tightlist
  \item
    Positiv → konkav nach oben.
  \item
    Negativ → konkav nach unten.
  \item
    Nullen oder undefiniert → mögliche Wendepunkte.
  \end{itemize}
\item
  Informationen kombinieren: Verwenden Sie alle Ergebnisse, um ein
  klares und genaues Diagramm zu entwerfen.
\end{enumerate}

\subsubsection{\texorpdfstring{Beispiel 1:
\(f(x) = x^3 - 3x\)}{Beispiel 1: f(x) = x\^{}3 - 3x}}\label{beispiel-1-fx-x3---3x}

\begin{itemize}
\item
  Bereich: alle reellen Zahlen.
\item
  Intercepts: bei \((0,0)\).
\item
  Derivat: \(f'(x) = 3x^2 - 3 = 3(x-1)(x+1)\).

  \begin{itemize}
  \item
    Steigend: \((-\infty, -1) \cup (1, \infty)\).
  \item
    Abnehmend: \((-1, 1)\).- Zweite Ableitung: \(f''(x) = 6x\).
  \item
    Konkav nach unten für \(x < 0\), konkav nach oben für \(x > 0\).
  \item
    Wendepunkt bei \((0,0)\).
  \end{itemize}
\item
  Form: eine S-Kurve mit lokalem Maximum bei \((-1, 2)\), lokalem
  Minimum bei \((1, -2)\).
\end{itemize}

\subsubsection{\texorpdfstring{Beispiel 2:
\(f(x) = \frac{1}{x}\)}{Beispiel 2: f(x) = \textbackslash frac\{1\}\{x\}}}\label{beispiel-2-fx-frac1x}

\begin{itemize}
\item
  Domain: \(x \neq 0\).
\item
  Vertikale Asymptote: \(x = 0\).
\item
  Horizontale Asymptote: \(y = 0\).
\item
  Ableitung: \(f'(x) = -\frac{1}{x^2}\) (immer negativ). Die Funktion
  nimmt immer ab.
\item
  Zweite Ableitung: \(f''(x) = \frac{2}{x^3}\).

  \begin{itemize}
  \tightlist
  \item
    Konkav für \(x > 0\).
  \item
    Konkav nach unten für \(x < 0\).
  \end{itemize}
\item
  Grafik: Hyperbel mit zwei Ästen.
\end{itemize}

\subsubsection{Warum das Skizzieren von Kurven nützlich
ist}\label{warum-das-skizzieren-von-kurven-nuxfctzlich-ist}

\begin{itemize}
\tightlist
\item
  Bietet Einblick in das Gesamtverhalten von Funktionen ohne umfassende
  Berechnungen.
\item
  Unverzichtbar bei Prüfungen zur Analysis und bei angewandten
  Problemen.
\item
  Brücke zwischen algebraischer Analyse und geometrischem Verständnis.
\end{itemize}

\subsubsection{Übungen}\label{uxfcbungen-12}

\begin{enumerate}
\def\labelenumi{\arabic{enumi}.}
\tightlist
\item
  Skizzieren Sie die Kurve von \(f(x) = x^4 - 2x^2\). Identifizieren Sie
  Maxima, Minima und Wendepunkte.
\item
  Analysieren und skizzieren \(f(x) = \ln(x)\). Zeigen Sie
  Achsenabschnitte, Asymptoten und Konkavität an.
\item
  Beschreiben Sie für \(f(x) = e^{-x}\) Wachstum/Verfall, Asymptoten und
  Konkavität.
\item
  Skizzieren Sie den Graphen von \(f(x) = \tan x\) auf dem Intervall
  \((- \pi, \pi)\). Markieren Sie Asymptoten.
\item
  Verwenden Sie den Test der ersten und zweiten Ableitung, um kritische
  Punkte von \(f(x) = x^3 - 6x^2 + 9x\) zu klassifizieren.
\end{enumerate}

\section{Teil II. Integrale}\label{teil-ii.-integrale}

\section{Kapitel 4. Stammfunktionen und bestimmte
Integrale}\label{kapitel-4.-stammfunktionen-und-bestimmte-integrale}

\subsection{4.1 Unbestimmte Integrale}\label{unbestimmte-integrale}

Ein unbestimmtes Integral ist der umgekehrte Prozess der
Differentiation. Wenn eine Ableitung die Änderung misst, stellt ein
Integral die ursprüngliche Funktion anhand ihrer Änderungsrate wieder
her.

\subsubsection{Definition}\label{definition-3}

Wenn \(F'(x) = f(x)\), dann

\[
\int f(x)\,dx = F(x) + C,
\]

wobei \(C\) die Integrationskonstante ist.

Jedes unbestimmte Integral stellt eine Familie von Funktionen dar, die
sich nur durch eine Konstante unterscheiden, da bei der Differentiation
Konstanten eliminiert werden.

\subsubsection{Grundregeln}\label{grundregeln}

\begin{enumerate}
\def\labelenumi{\arabic{enumi}.}
\tightlist
\item
  Konstante Regel
\end{enumerate}

\[
\int c\,dx = cx + C.
\]

\begin{enumerate}
\def\labelenumi{\arabic{enumi}.}
\setcounter{enumi}{1}
\tightlist
\item
  Machtregel
\end{enumerate}

\[
\int x^n\,dx = \frac{x^{n+1}}{n+1} + C, \quad n \neq -1.
\]

\begin{enumerate}
\def\labelenumi{\arabic{enumi}.}
\setcounter{enumi}{2}
\tightlist
\item
  Summenregel
\end{enumerate}

\[
\int \big(f(x) + g(x)\big)\,dx = \int f(x)\,dx + \int g(x)\,dx.
\]

\begin{enumerate}
\def\labelenumi{\arabic{enumi}.}
\setcounter{enumi}{3}
\tightlist
\item
  Konstante Mehrfachregel
\end{enumerate}

\[
\int c f(x)\,dx = c \int f(x)\,dx.
\]

\subsubsection{Gemeinsame Integrale}\label{gemeinsame-integrale}

\begin{itemize}
\tightlist
\item
  \(\int e^x dx = e^x + C\)
\item
  \(\int \sin x dx = -\cos x + C\)
\item
  \(\int \cos x dx = \sin x + C\)
\item
  \(\int \frac{1}{x} dx = \ln|x| + C\)
\end{itemize}

\subsubsection{Beispiele}\label{beispiele-7}

\begin{enumerate}
\def\labelenumi{\arabic{enumi}.}
\item
  \(\int (3x^2 - 4)\,dx = x^3 - 4x + C\).
\item
  \(\int \cos(2x)\,dx = \tfrac{1}{2}\sin(2x) + C\).
\item
  \(\int \frac{1}{x}\,dx = \ln|x| + C\).
\end{enumerate}

\subsubsection{Interpretation}\label{interpretation-1}

\begin{itemize}
\tightlist
\item
  Unbestimmte Integrale sind Stammfunktionen.
\item
  Sie sind die Grundlage für bestimmte Integrale, die akkumulierte
  Größen wie Fläche, Entfernung und Masse messen.- In angewandten
  Kontexten ermöglicht uns die Integration, von Raten zurück zu
  Gesamtwerten zu gelangen.
\end{itemize}

\subsubsection{Übungen}\label{uxfcbungen-13}

\begin{enumerate}
\def\labelenumi{\arabic{enumi}.}
\tightlist
\item
  Suchen Sie nach \(\int (5x^4 + 2x)\,dx\).
\item
  Berechnen Sie \(\int (e^x + 3)\,dx\).
\item
  Finden Sie die allgemeine Lösung von \(f'(x) = 6x\) durch Integration.
\item
  Bewerten Sie \(\int \frac{2}{x}\,dx\).
\item
  Wenn die Geschwindigkeit \(v(t) = 4t\) beträgt, suchen Sie die
  Positionsfunktion \(s(t)\).
\end{enumerate}

\subsection{4.2 Das bestimmte Integral als
Fläche}\label{das-bestimmte-integral-als-fluxe4che}

Während unbestimmte Integrale Familien von Stammfunktionen darstellen,
gibt das bestimmte Integral einen numerischen Wert an: die akkumulierte
Fläche unter einer Kurve zwischen zwei Punkten.

\subsubsection{Definition}\label{definition-4}

Für eine Funktion \(f(x)\) definiert auf \([a, b]\) ist das bestimmte
Integral

\[
\int_a^b f(x)\,dx = \lim_{n \to \infty} \sum_{i=1}^n f(x_i^-) \,\Delta x,
\]

wobei das Intervall \([a, b]\) in \(n\) Teilintervalle der Breite
\(\Delta x\) unterteilt ist und \(x_i^-\) ein Abtastpunkt in jedem
Teilintervall ist.

Dies ist der Grenzwert der Riemann-Summen.

\subsubsection{Geometrische
Interpretation}\label{geometrische-interpretation}

\begin{itemize}
\tightlist
\item
  Wenn \(f(x) \geq 0\) auf \([a, b]\), dann entspricht
  \(\int_a^b f(x)\,dx\) der Fläche unter der Kurve \(y = f(x)\) von
  \(x=a\) bis \(x=b\).
\item
  Wenn \(f(x)\) unter die \(x\)-Achse fällt, berechnet das Integral die
  vorzeichenbehaftete Fläche: Bereiche unterhalb der Achse zählen als
  negativ.
\end{itemize}

\subsubsection{Eigenschaften des bestimmten
Integrals}\label{eigenschaften-des-bestimmten-integrals}

\begin{enumerate}
\def\labelenumi{\arabic{enumi}.}
\tightlist
\item
  Additivität über Intervalle
\end{enumerate}

\[
\int_a^c f(x)\,dx = \int_a^b f(x)\,dx + \int_b^c f(x)\,dx.
\]

\begin{enumerate}
\def\labelenumi{\arabic{enumi}.}
\setcounter{enumi}{1}
\tightlist
\item
  Reversiergrenzen
\end{enumerate}

\[
\int_a^b f(x)\,dx = -\int_b^a f(x)\,dx.
\]

\begin{enumerate}
\def\labelenumi{\arabic{enumi}.}
\setcounter{enumi}{2}
\tightlist
\item
  Intervall mit Nullbreite
\end{enumerate}

\[
\int_a^a f(x)\,dx = 0.
\]

\begin{enumerate}
\def\labelenumi{\arabic{enumi}.}
\setcounter{enumi}{3}
\tightlist
\item
  Linearität
\end{enumerate}

\[
\int_a^b \big( cf(x) + g(x)\big)\,dx = c\int_a^b f(x)\,dx + \int_a^b g(x)\,dx.
\]

\subsubsection{Beispiele}\label{beispiele-8}

\begin{enumerate}
\def\labelenumi{\arabic{enumi}.}
\item
  \(\int_0^2 x\,dx = \left[\tfrac{1}{2}x^2\right]_0^2 = 2.\) Dies ist
  die Fläche eines rechtwinkligen Dreiecks unter der Linie \(y=x\).
\item
  \(\int_{-1}^1 x^3\,dx = 0.\) Die ungerade Funktion \(x^3\) hat
  symmetrische Bereiche, die sich aufheben.
\item
  \(\int_0^\pi \sin x\,dx = 2.\) Dies entspricht der Fläche unter einem
  Bogen der Sinuskurve.
\end{enumerate}

\subsubsection{Warum das wichtig ist}\label{warum-das-wichtig-ist-2}

\begin{itemize}
\tightlist
\item
  Bestimmte Integrale messen akkumulierte Größen: Entfernung, Masse,
  Energie, Wahrscheinlichkeit.
\item
  Sie verbinden algebraisches Rechnen mit geometrischer Intuition.
\item
  Der nächste Schritt ist der Fundamentalsatz der Analysis, der
  bestimmte Integrale mit Stammfunktionen verbindet.
\end{itemize}

\subsubsection{Übungen}\label{uxfcbungen-14}

\begin{enumerate}
\def\labelenumi{\arabic{enumi}.}
\tightlist
\item
  Berechnen Sie \(\int_0^3 (2x+1)\,dx\).
\item
  Finden Sie den Bereich zwischen \(y = x^2\) und der \(x\)-Achse von
  \(x = 0\) bis \(x = 2\).
\item
  Bewerten Sie \(\int_{-2}^2 (x^2 - 1)\,dx\).
\item
  Zeigen Sie, dass \(\int_{-a}^a f(x)\,dx = 0\), wenn \(f(x)\) ungerade
  ist.
\item
  Ungefähr \(\int_0^1 e^x\,dx\) unter Verwendung einer Riemann-Summe mit
  \(n=4\)-Teilintervallen und rechten Endpunkten.
\end{enumerate}

\subsection{4.3 Der Fundamentalsatz der AnalysisDer Fundamentalsatz der
Analysis (FTC) vereint die beiden Hauptideen der Analysis:
Differenzierung und Integration. Es zeigt, dass die Ermittlung von
Flächen und die Ermittlung von Änderungsraten zwei Seiten derselben
Medaille
sind.}\label{der-fundamentalsatz-der-analysisder-fundamentalsatz-der-analysis-ftc-vereint-die-beiden-hauptideen-der-analysis-differenzierung-und-integration.-es-zeigt-dass-die-ermittlung-von-fluxe4chen-und-die-ermittlung-von-uxe4nderungsraten-zwei-seiten-derselben-medaille-sind.}

\subsubsection{Teil 1: Differentiation eines
Integrals}\label{teil-1-differentiation-eines-integrals}

Wenn \(f\) kontinuierlich auf \([a, b]\) ist, definieren Sie

\[
F(x) = \int_a^x f(t)\,dt.
\]

Dann ist \(F\) differenzierbar, und

\[
F'(x) = f(x).
\]

In Worten: Die Ableitung der akkumulierten Flächenfunktion ist die
ursprüngliche Funktion selbst.

\subsubsection{Teil 2: Auswertung bestimmter
Integrale}\label{teil-2-auswertung-bestimmter-integrale}

Wenn \(f\) stetig auf \([a, b]\) ist und \(F\) eine Stammfunktion von
\(f\) ist, dann

\[
\int_a^b f(x)\,dx = F(b) - F(a).
\]

Dies sagt uns, dass wir bestimmte Integrale einfach durch die Bestimmung
einer Stammfunktion auswerten können, anstatt Grenzen von Riemann-Summen
zu berechnen.

\subsubsection{Beispiele}\label{beispiele-9}

\begin{enumerate}
\def\labelenumi{\arabic{enumi}.}
\item
  \(\int_0^2 x^2\,dx\).

  \begin{itemize}
  \tightlist
  \item
    Stammfunktion: \(F(x) = \tfrac{1}{3}x^3\).
  \item
    FTC anwenden: \(F(2) - F(0) = \tfrac{8}{3} - 0 = \tfrac{8}{3}.\)
  \end{itemize}
\item
  Wenn \(F(x) = \int_1^x \cos t \, dt\), dann \(F'(x) = \cos x\).
\item
  \(\int_1^4 \frac{1}{x}\,dx\).

  \begin{itemize}
  \tightlist
  \item
    Stammfunktion: \(\ln|x|\).
  \item
    FTC anwenden: \(\ln 4 - \ln 1 = \ln 4.\)
  \end{itemize}
\end{enumerate}

\subsubsection{Warum die FTC wichtig
ist}\label{warum-die-ftc-wichtig-ist}

\begin{itemize}
\tightlist
\item
  Es verwandelt die Integration von einem Grenzwertprozess in eine
  praktische Berechnung.
\item
  Es bestätigt, dass Differenzierung und Integration inverse Operationen
  sind.
\item
  Es ist der zentrale Satz, der die Analysis in Mathematik,
  Naturwissenschaften und Ingenieurwesen nützlich macht.
\end{itemize}

\subsubsection{Übungen}\label{uxfcbungen-15}

\begin{enumerate}
\def\labelenumi{\arabic{enumi}.}
\tightlist
\item
  Bewerten Sie \(\int_0^3 (2x+1)\,dx\) mithilfe der FTC.
\item
  Wenn \(F(x) = \int_0^x e^t\,dt\), finden Sie \(F'(x)\).
\item
  Berechnen Sie \(\int_0^\pi \sin x \, dx\).
\item
  Zeigen Sie, dass wenn \(f'(x) = g(x)\), dann
  \(\int_a^b g(x)\,dx = f(b) - f(a)\).
\item
  Erklären Sie anhand der FTC, warum der Bereich unter \(y = \cos x\)
  von \(0\) bis \(\pi/2\) gleich 1 ist.
\end{enumerate}

\subsection{4.4 Eigenschaften von
Integralen}\label{eigenschaften-von-integralen}

Das bestimmte Integral hat mehrere wichtige Eigenschaften, die es in
Anwendungen flexibel und leistungsstark machen. Diese Eigenschaften
ergeben sich aus der Definition als Grenzwert von Summen und aus dem
Fundamentalsatz der Analysis.

\subsubsection{Linearität}\label{linearituxe4t}

Für Funktionen \(f(x)\) und \(g(x)\) und Konstanten \(c, d\):

\[
\int_a^b \big(c f(x) + d g(x)\big)\,dx = c \int_a^b f(x)\,dx + d \int_a^b g(x)\,dx.
\]

Dadurch können wir komplizierte Integrale in einfachere Teile zerlegen.

\subsubsection{Additivität über
Intervalle}\label{additivituxe4t-uxfcber-intervalle}

Wenn \(a < c < b\), dann

\[
\int_a^b f(x)\,dx = \int_a^c f(x)\,dx + \int_c^b f(x)\,dx.
\]

Wir können Integrale Stück für Stück berechnen.

\subsubsection{Umkehrung der Limits}\label{umkehrung-der-limits}

\[
\int_a^b f(x)\,dx = -\int_b^a f(x)\,dx.
\]

Durch Vertauschen der Grenzen ändert sich das Vorzeichen des Integrals.

\subsubsection{Vergleichseigenschaft}\label{vergleichseigenschaft}

Wenn \(f(x) \leq g(x)\) für alle \(x\) in \([a, b]\), dann

\[
\int_a^b f(x)\,dx \leq \int_a^b g(x)\,dx.
\]Dadurch können wir Flächen ohne direkte Berechnung vergleichen.

\subsubsection{Absolute Wertungleichung}\label{absolute-wertungleichung}

\[
\left| \int_a^b f(x)\,dx \right| \leq \int_a^b |f(x)|\,dx.
\]

Diese Eigenschaft ist bei Analysen und Konvergenztests von wesentlicher
Bedeutung.

\subsubsection{Symmetrie}\label{symmetrie}

\begin{itemize}
\item
  Wenn \(f(x)\) gerade ist (symmetrisch zur \(y\)-Achse):

  \[
  \int_{-a}^a f(x)\,dx = 2\int_0^a f(x)\,dx.
  \]
\item
  Wenn \(f(x)\) ungerade ist (symmetrisch zum Ursprung):

  \[
  \int_{-a}^a f(x)\,dx = 0.
  \]
\end{itemize}

\subsubsection{Beispiele}\label{beispiele-10}

\begin{enumerate}
\def\labelenumi{\arabic{enumi}.}
\item
  \(\int_0^2 (3x^2 + 4)\,dx = \int_0^2 3x^2\,dx + \int_0^2 4\,dx = 8 + 8 = 16.\)
\item
  Da \(f(x) = x^3\) ungerade ist, \(\int_{-1}^1 x^3\,dx = 0.\)
\item
  Da \(f(x) = x^2\) gerade ist,
  \(\int_{-2}^2 x^2\,dx = 2\int_0^2 x^2\,dx = 2\cdot \tfrac{8}{3} = \tfrac{16}{3}.\)
\end{enumerate}

\subsubsection{Warum diese Eigenschaften wichtig
sind}\label{warum-diese-eigenschaften-wichtig-sind}

\begin{itemize}
\tightlist
\item
  Sie vereinfachen Berechnungen.
\item
  Sie offenbaren geometrische und symmetrische Merkmale von Funktionen.
\item
  Sie bieten theoretische Werkzeuge für eine fortgeschrittenere Analyse.
\end{itemize}

\subsubsection{Übungen}\label{uxfcbungen-16}

\begin{enumerate}
\def\labelenumi{\arabic{enumi}.}
\tightlist
\item
  Verwenden Sie die Symmetrie, um \(\int_{-5}^5 (x^4 - x^3)\,dx\)
  auszuwerten.
\item
  Zeigen Sie, dass
  \(\int_1^4 (2x+3)\,dx = \int_1^2 (2x+3)\,dx + \int_2^4 (2x+3)\,dx\).
\item
  \(\int_0^\pi \sin(x)\,dx\) auswerten und mit
  \(\int_{-\pi}^\pi \sin(x)\,dx\) vergleichen.
\item
  Beweisen Sie, dass wenn \(f(x) \geq 0\) auf \([a, b]\), dann
  \(\int_a^b f(x)\,dx \geq 0\).
\item
  Berechnen Sie \(\int_{-3}^3 (x^2 + 1)\,dx\) unter Verwendung
  gerader/ungerade Eigenschaften.
\end{enumerate}

\section{Kapitel 5. Techniken der
Integration}\label{kapitel-5.-techniken-der-integration}

\subsection{5.1 Substitution}\label{substitution}

Eine der nützlichsten Integrationstechniken ist die
Substitutionsmethode, auch -u-Substitution- genannt. Es handelt sich um
den umgekehrten Prozess der Kettenregel für Derivate.

\subsubsection{Die Idee}\label{die-idee}

Wenn ein Integral eine zusammengesetzte Funktion enthält, können wir sie
durch Ändern von Variablen vereinfachen.

Wenn \(u = g(x)\) formal eine differenzierbare Funktion ist, dann

\[
\int f(g(x)) g'(x)\,dx = \int f(u)\,du.
\]

Diese Substitution erleichtert die Auswertung des Integrals.

\subsubsection{Schritte zur
Substitution}\label{schritte-zur-substitution}

\begin{enumerate}
\def\labelenumi{\arabic{enumi}.}
\tightlist
\item
  Identifizieren Sie eine innere Funktion \(u = g(x)\), deren Ableitung
  auch im Integranden vorkommt.
\item
  Berechnen Sie \(du = g'(x)\,dx\).
\item
  Schreiben Sie das Integral in Form von \(u\) um.
\item
  Integrieren Sie in Bezug auf \(u\).
\item
  Ersatz zurück \(u = g(x)\).
\end{enumerate}

\subsubsection{Beispiele}\label{beispiele-11}

\begin{enumerate}
\def\labelenumi{\arabic{enumi}.}
\item
  Einfache Substitution

  \[
  \int 2x \cos(x^2)\,dx
  \]

  Sei \(u = x^2\), also \(du = 2x\,dx\). Dann wird das Integral zu
  \(\int \cos u \,du = \sin u + C = \sin(x^2) + C\).
\item
  Logarithmischer Fall

  \[
  \int \frac{2x}{x^2+1}\,dx
  \]

  Sei \(u = x^2 + 1\), also \(du = 2x\,dx\). Dann wird das Integral zu
  \(\int \frac{1}{u}\,du = \ln|u| + C = \ln(x^2+1) + C\).
\item
  Trigonometrische Substitution

  \[
  \int \sin(3x)\,dx
  \]

  Sei \(u = 3x\), also \(du = 3\,dx\), also
  \(dx = \frac{du}{3}\).Integral wird zu
  \(\tfrac{1}{3}\int \sin u\,du = -\tfrac{1}{3}\cos u + C = -\tfrac{1}{3}\cos(3x) + C\).
\end{enumerate}

\subsubsection{Bestimmte Integrale mit
Substitution}\label{bestimmte-integrale-mit-substitution}

Bei der Auswertung bestimmter Integrale müssen wir auch die Grenzen
ändern:

\[
\int_a^b f(g(x)) g'(x)\,dx = \int_{g(a)}^{g(b)} f(u)\,du.
\]

Beispiel:

\[
\int_0^1 2x e^{x^2}\,dx.
\]

Sei \(u = x^2\), \(du = 2x\,dx\). Limits: wenn \(x=0, u=0\); wenn
\(x=1, u=1\). Das Integral wird also

\[
\int_0^1 e^u\,du = e - 1.
\]

\subsubsection{Übungen}\label{uxfcbungen-17}

\begin{enumerate}
\def\labelenumi{\arabic{enumi}.}
\tightlist
\item
  Bewerten Sie \(\int (x^2+1)^5 (2x)\,dx\).
\item
  Berechnen Sie \(\int \frac{\cos x}{\sin x}\,dx\).
\item
  Bewerten Sie \(\int_0^\pi \sin(2x)\,dx\) durch Substitution.
\item
  Finden Sie \(\int e^{3x}\,dx\).
\item
  Berechnen Sie \(\int \frac{1}{\sqrt{1+x^2}}\,dx\), indem Sie
  \(u = 1+x^2\) verwenden.
\end{enumerate}

\subsection{5.2 Integration nach Teilen}\label{integration-nach-teilen}

Die partielle Integration ist eine Technik, die aus der Produktregel für
Derivate stammt. Es hilft bei der Bewertung von Integralen mit Produkten
von Funktionen, die durch Substitution allein nicht einfach zu handhaben
sind.

\subsubsection{Die Formel}\label{die-formel}

Aus der Produktregel:

\[
\frac{d}{dx}[u(x)v(x)] = u'(x)v(x) + u(x)v'(x).
\]

Die Integration beider Seiten ergibt die Formel für die partielle
Integration:

\[
\int u\,dv = uv - \int v\,du.
\]

Hier:

\begin{itemize}
\tightlist
\item
  \(u\) = eine zur Differenzierung ausgewählte Funktion,
\item
  \(dv\) = der verbleibende Teil des zu integrierenden Integranden.
\end{itemize}

\subsubsection{\texorpdfstring{Wählen Sie \(u\) und
\(dv\)}{Wählen Sie u und dv}}\label{wuxe4hlen-sie-u-und-dv}

Eine gängige Richtlinie ist LIATE (Logarithmisch, Invers
trigonometrische, Algebraisch, Trigonometrisch, Exponentiell).

\begin{itemize}
\tightlist
\item
  Wählen Sie \(u\) aus der frühesten verfügbaren Kategorie.
\item
  Wählen Sie als Rest \(dv\).
\end{itemize}

\subsubsection{Beispiele}\label{beispiele-12}

\begin{enumerate}
\def\labelenumi{\arabic{enumi}.}
\tightlist
\item
  Polynom × Exponential
\end{enumerate}

\[
\int x e^x\,dx
\]

Sei \(u = x\), \(dv = e^x dx\). Dann \(du = dx\), \(v = e^x\).

\[
\int x e^x\,dx = x e^x - \int e^x dx = x e^x - e^x + C.
\]

\begin{enumerate}
\def\labelenumi{\arabic{enumi}.}
\setcounter{enumi}{1}
\tightlist
\item
  Polynom × Trig
\end{enumerate}

\[
\int x \cos x\,dx
\]

Lassen Sie \(u = x\), \(dv = \cos x dx\). Dann \(du = dx\),
\(v = \sin x\).

\[
\int x \cos x\,dx = x \sin x - \int \sin x dx = x \sin x + \cos x + C.
\]

\begin{enumerate}
\def\labelenumi{\arabic{enumi}.}
\setcounter{enumi}{2}
\tightlist
\item
  Logarithmus
\end{enumerate}

\[
\int \ln x\,dx
\]

Lass \(u = \ln x\), \(dv = dx\). Dann \(du = \frac{1}{x}dx\), \(v = x\).

\[
\int \ln x\,dx = x \ln x - \int 1 dx = x \ln x - x + C.
\]

\subsubsection{Definitives
Integralbeispiel}\label{definitives-integralbeispiel}

\[
\int_0^1 x e^x\,dx
\]

Unter Verwendung des früheren Ergebnisses: \(\int x e^x dx = (x-1)e^x\).
Bewerten:

\[
\big[(x-1)e^x\big]_0^1 = (0)e^1 - (-1)e^0 = 0 + 1 = 1.
\]

\subsubsection{Warum das wichtig ist}\label{warum-das-wichtig-ist-3}

Die partielle Integration ist von entscheidender Bedeutung, wenn die
Substitution fehlschlägt, insbesondere bei Logarithmen, inversen
trigonometrischen Funktionen und Produkten, die Polynome mit
Exponentialfunktionen oder trigonometrischen Funktionen umfassen.

\subsubsection{Übungen}\label{uxfcbungen-18}

\begin{enumerate}
\def\labelenumi{\arabic{enumi}.}
\tightlist
\item
  Bewerten Sie \(\int x \sin x\,dx\).
\item
  Finden Sie \(\int e^x \cos x\,dx\).
\item
  Berechnen Sie \(\int_1^2 \ln x\,dx\).
\item
  Bewerten Sie \(\int x^2 e^x\,dx\).5. Verwenden Sie die partielle
  Integration, um
  \(\int \arctan x\,dx = x\arctan x - \tfrac{1}{2}\ln(1+x^2) + C\)
  anzuzeigen.
\end{enumerate}

\subsection{5.3 Trigonometrische Integrale und
Substitutionen}\label{trigonometrische-integrale-und-substitutionen}

Viele Integrale beinhalten trigonometrische Funktionen. Diese können
häufig durch Identitäten oder spezielle Ersetzungen vereinfacht werden.

\subsubsection{Trigonometrische
Integrale}\label{trigonometrische-integrale}

\begin{enumerate}
\def\labelenumi{\arabic{enumi}.}
\tightlist
\item
  Potenzen von Sinus und Cosinus
\end{enumerate}

\begin{itemize}
\tightlist
\item
  Wenn die Potenz des Sinus ungerade ist: Speichern Sie ein \(\sin x\),
  wandeln Sie den Rest durch \(\sin^2x = 1 - \cos^2x\) um und ersetzen
  Sie es durch \(u = \cos x\).
\item
  Wenn die Potenz des Kosinus ungerade ist: Speichern Sie ein
  \(\cos x\), wandeln Sie den Rest durch \(\cos^2x = 1 - \sin^2x\) um
  und ersetzen Sie \(u = \sin x\).
\item
  Wenn beide gerade sind: Halbwinkelidentitäten verwenden.
\end{itemize}

Beispiel:

\[
\int \sin^3x \cos x \, dx
\]

Sei \(u = \sin x\), \(du = \cos x\,dx\):

\[
\int u^3\,du = \tfrac{u^4}{4} + C = \tfrac{\sin^4x}{4} + C.
\]

\begin{enumerate}
\def\labelenumi{\arabic{enumi}.}
\setcounter{enumi}{1}
\tightlist
\item
  Produkte von Sinus und Cosinus mit unterschiedlichen Winkeln Verwenden
  Sie Produkt-zu-Summe-Formeln:
\end{enumerate}

\[
\sin A \cos B = \tfrac{1}{2}[\sin(A+B) + \sin(A-B)].
\]

Beispiel:

\[
\int \sin(2x)\cos(3x)\,dx = \tfrac{1}{2}\int [\sin(5x) - \sin(x)]\,dx.
\]

\begin{enumerate}
\def\labelenumi{\arabic{enumi}.}
\setcounter{enumi}{2}
\tightlist
\item
  Sekanten- und Tangenskräfte
\end{enumerate}

\begin{itemize}
\tightlist
\item
  Wenn die Sekantenpotenz gerade ist: Speichern Sie \(\sec^2x\),
  konvertieren Sie den Rest durch \(\sec^2x = 1 + \tan^2x\) und ersetzen
  Sie \(u = \tan x\).
\item
  Wenn die Potenz der Tangente ungerade ist: Speichern Sie \(\sec^2x\),
  konvertieren Sie den Rest durch \(\tan^2x = \sec^2x - 1\) und ersetzen
  Sie \(u = \tan x\).
\end{itemize}

Beispiel:

\[
\int \tan^3x \sec^2x \, dx
\]

Sei \(u = \tan x\), \(du = \sec^2x\,dx\):

\[
\int u^3\,du = \tfrac{u^4}{4} + C = \tfrac{\tan^4x}{4} + C.
\]

\subsubsection{Trigonometrische
Substitutionen}\label{trigonometrische-substitutionen}

Für Integrale mit \(\sqrt{a^2 - x^2}\), \(\sqrt{a^2 + x^2}\) oder
\(\sqrt{x^2 - a^2}\) verwenden Sie spezielle Substitutionen:

\begin{enumerate}
\def\labelenumi{\arabic{enumi}.}
\tightlist
\item
  \(x = a \sin \theta\), für \(\sqrt{a^2 - x^2}\).
\item
  \(x = a \tan \theta\), für \(\sqrt{a^2 + x^2}\).
\item
  \(x = a \sec \theta\), für \(\sqrt{x^2 - a^2}\).
\end{enumerate}

Beispiel:

\[
\int \sqrt{a^2 - x^2}\,dx
\]

Sei \(x = a\sin\theta\), also \(dx = a\cos\theta\,d\theta\):

\[
\int \sqrt{a^2 - a^2\sin^2\theta}(a\cos\theta\,d\theta) = \int a^2 \cos^2\theta \, d\theta.
\]

Vereinfachen Sie die Verwendung von Halbwinkelidentitäten.

\subsubsection{Warum diese Techniken wichtig
sind}\label{warum-diese-techniken-wichtig-sind}

\begin{itemize}
\tightlist
\item
  Sie wandeln schwierige algebraische Formen in handhabbare
  trigonometrische Formen um.
\item
  Sie sind besonders nützlich bei Problemen mit Flächen, Volumina und
  Bogenlängen.
\item
  Sie legen den Grundstein für fortschrittliche Integrationsmethoden.
\end{itemize}

\subsubsection{Übungen}\label{uxfcbungen-19}

\begin{enumerate}
\def\labelenumi{\arabic{enumi}.}
\tightlist
\item
  Bewerten Sie \(\int \sin^4x \cos^2x \, dx\).
\item
  Berechnen Sie \(\int \sin(5x)\cos(2x)\,dx\).
\item
  Bewerten Sie \(\int \tan^2x \sec^2x \, dx\).
\item
  Finden Sie \(\int \sqrt{9 - x^2}\,dx\) mithilfe der Substitution.
\item
  Zeigen Sie, dass
  \(\int \frac{dx}{\sqrt{x^2 + a^2}} = \ln|x + \sqrt{x^2 + a^2}| + C\)
  mit \(x = a\tan\theta\) gilt.
\end{enumerate}

\subsection{5.4 TeilbrücheBei der Integration rationaler Funktionen
(Verhältnisse von Polynomen) ist die Partialbruchzerlegung eine
leistungsstarke Methode. Diese Technik drückt einen komplizierten Bruch
als Summe einfacherer Brüche aus, die leichter zu integrieren
sind.}\label{teilbruxfcchebei-der-integration-rationaler-funktionen-verhuxe4ltnisse-von-polynomen-ist-die-partialbruchzerlegung-eine-leistungsstarke-methode.-diese-technik-druxfcckt-einen-komplizierten-bruch-als-summe-einfacherer-bruxfcche-aus-die-leichter-zu-integrieren-sind.}

\subsubsection{Die Idee}\label{die-idee-1}

Wenn \(R(x) = \frac{P(x)}{Q(x)}\) eine rationale Funktion ist, bei der
der Grad von \(P(x)\) kleiner ist als der Grad von \(Q(x)\), können wir
\(R(x)\) in einfachere Brüche zerlegen.

Diese einfacheren Stücke entsprechen den Faktoren des Nenners \(Q(x)\).

\subsubsection{Gemeinsame Formen}\label{gemeinsame-formen}

\begin{enumerate}
\def\labelenumi{\arabic{enumi}.}
\tightlist
\item
  Ausgeprägte lineare Faktoren Wenn
\end{enumerate}

\[
\frac{1}{(x-a)(x-b)},
\]

dann zerlegen als

\[
\frac{A}{x-a} + \frac{B}{x-b}.
\]

\begin{enumerate}
\def\labelenumi{\arabic{enumi}.}
\setcounter{enumi}{1}
\tightlist
\item
  Wiederholte lineare Faktoren Wenn der Nenner \((x-a)^n\) hat, dann
  sind die Terme
\end{enumerate}

\[
\frac{A_1}{x-a} + \frac{A_2}{(x-a)^2} + \dots + \frac{A_n}{(x-a)^n}.
\]

\begin{enumerate}
\def\labelenumi{\arabic{enumi}.}
\setcounter{enumi}{2}
\tightlist
\item
  Irreduzible quadratische Faktoren Wenn der Nenner \((x^2+bx+c)\) hat,
  dann ist der Zähler linear:
\end{enumerate}

\[
\frac{Ax+B}{x^2+bx+c}.
\]

\subsubsection{Beispiel 1: Eindeutige lineare
Faktoren}\label{beispiel-1-eindeutige-lineare-faktoren}

\[
\int \frac{1}{x^2 - 1}\,dx
\]

Faktor Nenner: \((x-1)(x+1)\). Zerlegen:

\[
\frac{1}{x^2-1} = \frac{1}{2}\left(\frac{1}{x-1} - \frac{1}{x+1}\right).
\]

Integrieren:

\[
\int \frac{1}{x^2 - 1}\,dx = \tfrac{1}{2}\ln\left|\frac{x-1}{x+1}\right| + C.
\]

\subsubsection{Beispiel 2: Wiederholter linearer
Faktor}\label{beispiel-2-wiederholter-linearer-faktor}

\[
\int \frac{1}{(x-1)^2}\,dx
\]

Das ist schon einfach:

\[
\int (x-1)^{-2}\,dx = -\frac{1}{x-1} + C.
\]

\subsubsection{Beispiel 3: Irreduzibler quadratischer
Faktor}\label{beispiel-3-irreduzibler-quadratischer-faktor}

\[
\int \frac{x}{x^2+1}\,dx
\]

Ersetzen Sie \(u = x^2+1\) oder erkennen Sie, dass der Zähler eine
Ableitung des Nenners ist.

\[
\int \frac{x}{x^2+1}\,dx = \tfrac{1}{2}\ln(x^2+1) + C.
\]

\subsubsection{Schritte bei der partiellen
Fraktionszerlegung}\label{schritte-bei-der-partiellen-fraktionszerlegung}

\begin{enumerate}
\def\labelenumi{\arabic{enumi}.}
\tightlist
\item
  Faktorisieren Sie den Nenner.
\item
  Schreiben Sie die allgemeine Partialbruchform.
\item
  Multiplizieren Sie mit dem Nenner, um Brüche aufzulösen.
\item
  Lösen Sie nach unbekannten Konstanten auf.
\item
  Integrieren Sie jeden Begriff.
\end{enumerate}

\subsubsection{Warum das wichtig ist}\label{warum-das-wichtig-ist-4}

\begin{itemize}
\tightlist
\item
  Konvertiert komplexe rationale Funktionen in einfache logarithmische
  oder Arcustangens-Formen.
\item
  Besonders nützlich bei Differentialgleichungen und
  Laplace-Transformationen.
\item
  Grundkenntnisse in fortgeschrittener Analysis und
  Ingenieurwissenschaften.
\end{itemize}

\subsubsection{Übungen}\label{uxfcbungen-20}

\begin{enumerate}
\def\labelenumi{\arabic{enumi}.}
\tightlist
\item
  Zerlegen und integrieren Sie \(\int \frac{3x+5}{x^2-1}\,dx\).
\item
  Bewerten Sie \(\int \frac{1}{x^2(x+1)}\,dx\).
\item
  Berechnen Sie \(\int \frac{2x+1}{x^2+2x+2}\,dx\).
\item
  Finden Sie \(\int \frac{1}{x^3 - x}\,dx\).
\item
  Zeigen Sie, dass \(\int \frac{dx}{x^2+1} = \arctan x + C\) durch
  Teilbrüche oder Substitution gilt.
\end{enumerate}

\subsection{5.5 Uneigentliche Integrale}\label{uneigentliche-integrale}

Einige Integrale können nicht direkt ausgewertet werden, da das
Intervall unendlich ist oder der Integrand unbegrenzt wird. Man nennt
sie uneigentliche Integrale. Sie werden über Grenzwerte definiert.

\subsubsection{Definition}\label{definition-5}

\begin{enumerate}
\def\labelenumi{\arabic{enumi}.}
\tightlist
\item
  Unendliches Intervall
\end{enumerate}

\[\int_a^\infty f(x)\,dx = \lim_{b \to \infty} \int_a^b f(x)\,dx.
\]

\[
\int_{-\infty}^a f(x)\,dx = \lim_{b \to -\infty} \int_b^a f(x)\,dx.
\]

\begin{enumerate}
\def\labelenumi{\arabic{enumi}.}
\setcounter{enumi}{1}
\tightlist
\item
  Unbounded integrand If \(f(x)\) has a vertical asymptote at \(c\),
  then
\end{enumerate}

\[
\int_a^c f(x)\,dx = \lim_{t \to c^-} \int_a^t f(x)\,dx,
\]

\[
\int_c^b f(x)\,dx = \lim_{t \to c^+} \int_t^b f(x)\,dx.
\]

\subsubsection{Convergence and
Divergence}\label{convergence-and-divergence}

\begin{itemize}
\tightlist
\item
  If the limit exists and is finite, the improper integral converges.
\item
  If the limit does not exist or is infinite, the improper integral
  diverges.
\end{itemize}

\subsubsection{Examples}\label{examples-1}

\begin{enumerate}
\def\labelenumi{\arabic{enumi}.}
\tightlist
\item
  Exponential decay
\end{enumerate}

\[
\int_1^\infty \frac{1}{x^2}\,dx = \lim_{b \to \infty} \Big[-\tfrac{1}{x}\Big]_1^b = 1.
\]

This converges.

\begin{enumerate}
\def\labelenumi{\arabic{enumi}.}
\setcounter{enumi}{1}
\tightlist
\item
  Harmonic function
\end{enumerate}

\[
\int_1^\infty \frac{1}{x}\,dx = \lim_{b \to \infty} \ln b.
\]

This diverges to infinity.

\begin{enumerate}
\def\labelenumi{\arabic{enumi}.}
\setcounter{enumi}{2}
\tightlist
\item
  Asymptote at 0
\end{enumerate}

\[
\int_0^1 \frac{1}{\sqrt{x}}\,dx = \lim_{t \to 0^+} \int_t^1 x^{-1/2}\,dx.
\]

\[
= \lim_{t \to 0^+} [2\sqrt{x}]_t^1 = 2.
\]

This converges.

\begin{enumerate}
\def\labelenumi{\arabic{enumi}.}
\setcounter{enumi}{3}
\tightlist
\item
  Asymptote at 0 (divergent)
\end{enumerate}

\[
\int_0^1 \frac{1}{x}\,dx = \lim_{t \to 0^+} \ln(1) - \ln(t).
\]

This diverges since \(\ln(t) \to -\infty\).

\subsubsection{Comparison Test for Improper
Integrals}\label{comparison-test-for-improper-integrals}

\begin{itemize}
\tightlist
\item
  If \(0 \leq f(x) \leq g(x)\) for large \(x\), and \(\int g(x)\,dx\)
  converges, then \(\int f(x)\,dx\) also converges.
\item
  If \(\int f(x)\,dx\) diverges and \(f(x) \geq g(x) \geq 0\), then
  \(\int g(x)\,dx\) also diverges.
\end{itemize}

\subsubsection{Why Improper Integrals
Matter}\label{why-improper-integrals-matter}

\begin{itemize}
\tightlist
\item
  They extend integration to infinite domains and unbounded functions.
\item
  They are essential in probability (continuous distributions), physics
  (gravitational/electric fields), and Fourier analysis.
\end{itemize}

\subsubsection{Exercises}\label{exercises-2}

\begin{enumerate}
\def\labelenumi{\arabic{enumi}.}
\tightlist
\item
  Determine whether \(\int_1^\infty \frac{1}{x^p}\,dx\) converges for
  various values of \(p\).
\item
  Evaluate \(\int_0^\infty e^{-x}\,dx\).
\item
  Test convergence of \(\int_0^1 \frac{1}{x^p}\,dx\) depending on \(p\).
\item
  Compute \(\int_{-\infty}^\infty \frac{1}{1+x^2}\,dx\).
\item
  Use the comparison test to show that
  \(\int_1^\infty \frac{1}{x^2+1}\,dx\) converges.
\end{enumerate}

\section{Chapter 6. Applications of
Integration}\label{chapter-6.-applications-of-integration}

\subsection{6.1 Areas and Volumes}\label{areas-and-volumes}

One of the most important applications of integration is finding areas
under curves and volumes of solids.

\subsubsection{Area Between Curves}\label{area-between-curves}

If \(f(x) \geq g(x)\) on \([a, b]\), then the area between the curves
\(y=f(x)\) and \(y=g(x)\) is

\[
A = \int_a^b \big(f(x) - g(x)\big)\,dx.
\]

Example: Find the area between \(y=x^2\) and \(y=x\) on \([0,1]\).

\[
A = \int_0^1 (x - x^2)\,dx = \left[\tfrac{1}{2}x^2 - \tfrac{1}{3}x^3\right]_0^1 = \tfrac{1}{6}.
\]

\subsubsection{Volumes by Slicing}\label{volumes-by-slicing}

If a solid has cross-sectional area \(A(x)\) at position \(x\), then the
volume is

\[
V = \int_a^b A(x)\,dx.
\]\#\#\# Bände der Revolution

Wenn eine Region um eine Achse gedreht wird, kann das Volumen des
resultierenden Festkörpers durch Integration ermittelt werden.

\begin{enumerate}
\def\labelenumi{\arabic{enumi}.}
\tightlist
\item
  Festplattenmethode Wenn der Bereich unter \(y=f(x)\), \(x\in[a,b]\),
  um die \(x\)-Achse gedreht wird:
\end{enumerate}

\[
V = \pi \int_a^b [f(x)]^2\,dx.
\]

\begin{enumerate}
\def\labelenumi{\arabic{enumi}.}
\setcounter{enumi}{1}
\tightlist
\item
  Waschmaschinenmethode Wenn der Bereich zwischen \(y=f(x)\) und
  \(y=g(x)\) um die \(x\)-Achse gedreht wird:
\end{enumerate}

\[
V = \pi \int_a^b \Big([f(x)]^2 - [g(x)]^2\Big)\,dx.
\]

\begin{enumerate}
\def\labelenumi{\arabic{enumi}.}
\setcounter{enumi}{2}
\tightlist
\item
  Shell-Methode Wenn Region unter \(y=f(x)\) um die \(y\)-Achse gedreht
  wird:
\end{enumerate}

\[
V = 2\pi \int_a^b x f(x)\,dx.
\]

\subsubsection{Beispiele}\label{beispiele-13}

\begin{enumerate}
\def\labelenumi{\arabic{enumi}.}
\tightlist
\item
  Festplattenmethode \(y=\sqrt{x}\), \(0 \leq x \leq 4\), um die
  \(x\)-Achse drehen:
\end{enumerate}

\[
V = \pi \int_0^4 (\sqrt{x})^2\,dx = \pi \int_0^4 x\,dx = \pi \left[\tfrac{1}{2}x^2\right]_0^4 = 8\pi.
\]

\begin{enumerate}
\def\labelenumi{\arabic{enumi}.}
\setcounter{enumi}{1}
\tightlist
\item
  Waschmaschinenmethode Bereich zwischen \(y=\sqrt{x}\) und \(y=1\),
  \(0 \leq x \leq 1\), um \(x\)-Achse drehen:
\end{enumerate}

\[
V = \pi \int_0^1 \big((\sqrt{x})^2 - (1)^2\big)\,dx = \pi \int_0^1 (x-1)\,dx = -\tfrac{\pi}{2}.
\]

(Nimm den absoluten Wert für das Volumen: \(V = \tfrac{\pi}{2}\)).

\begin{enumerate}
\def\labelenumi{\arabic{enumi}.}
\setcounter{enumi}{2}
\tightlist
\item
  Shell-Methode Bereich unter \(y=x\), \(0 \leq x \leq 1\) um die
  \(y\)-Achse drehen:
\end{enumerate}

\[
V = 2\pi \int_0^1 x(x)\,dx = 2\pi \int_0^1 x^2\,dx = 2\pi \cdot \tfrac{1}{3} = \tfrac{2\pi}{3}.
\]

\subsubsection{Warum das wichtig ist}\label{warum-das-wichtig-ist-5}

\begin{itemize}
\tightlist
\item
  Bietet genaue Möglichkeiten zur Berechnung von Flächen und Volumina in
  der Geometrie.
\item
  Unverzichtbar in Physik, Ingenieurwesen und
  Wahrscheinlichkeitstheorie.
\item
  Führt geometrisches Denken mit Integration ein.
\end{itemize}

\subsubsection{Übungen}\label{uxfcbungen-21}

\begin{enumerate}
\def\labelenumi{\arabic{enumi}.}
\tightlist
\item
  Finden Sie den Bereich zwischen \(y=\cos x\) und \(y=\sin x\) auf
  \([0, \pi/2]\).
\item
  Berechnen Sie das Volumen des Festkörpers, der durch die Drehung von
  \(y=x^2\), \(0 \leq x \leq 1\) um die \(x\)-Achse entsteht.
\item
  Ermitteln Sie das Volumen des Festkörpers, der durch Drehen des
  Bereichs zwischen \(y=x\) und \(y=\sqrt{x}\) auf \([0,1]\) um die
  \(y\)-Achse entsteht.
\item
  Verwenden Sie die Scheibenmethode, um das Volumen des Festkörpers zu
  berechnen, der durch die Drehung von \(y=\sqrt{1-x^2}\) (einem
  Halbkreis) um die \(x\)-Achse entsteht.
\item
  Finden Sie den Bereich zwischen \(y=x^2+1\) und \(y=3x\).
\end{enumerate}

\subsection{6.2 Bogenlänge und
Oberfläche}\label{bogenluxe4nge-und-oberfluxe4che}

Die Integration kann auch verwendet werden, um die Länge von Kurven und
die Oberfläche von Festkörpern zu messen, die durch rotierende Kurven
erzeugt werden.

\subsubsection{Bogenlänge}\label{bogenluxe4nge}

Für eine glatte Kurve \(y=f(x)\) im Intervall \([a,b]\) beträgt die
Länge der Kurve

\[
L = \int_a^b \sqrt{1 + \big(f'(x)\big)^2}\,dx.
\]

Dies ergibt sich aus der Annäherung der Kurve mit Liniensegmenten und
der Bestimmung des Grenzwerts.

Beispiel: Finden Sie die Länge von \(y=\tfrac{1}{2}x^{3/2}\) von \(x=0\)
bis \(x=4\).

\begin{itemize}
\tightlist
\item
  Derivat: \(f'(x) = \tfrac{3}{4}\sqrt{x}\).
\item
  Formel:
\end{itemize}

\[
L = \int_0^4 \sqrt{1 + \Big(\tfrac{3}{4}\sqrt{x}\Big)^2}\,dx
= \int_0^4 \sqrt{1 + \tfrac{9}{16}x}\,dx.
\]

Dieses Integral kann durch Substitution ausgewertet werden.\#\#\#
Oberfläche der Revolution

Wenn eine Kurve \(y=f(x)\), \(a \leq x \leq b\), um die \(x\)-Achse
gedreht wird, beträgt die Oberfläche des resultierenden Körpers

\[
S = 2\pi \int_a^b f(x)\sqrt{1 + \big(f'(x)\big)^2}\,dx.
\]

Bei Drehung um die \(y\)-Achse:

\[
S = 2\pi \int_a^b x \sqrt{1 + \big(f'(x)\big)^2}\,dx.
\]

\subsubsection{Beispiele}\label{beispiele-14}

\begin{enumerate}
\def\labelenumi{\arabic{enumi}.}
\tightlist
\item
  Bogenlänge einer Linie Für \(y=x\), \(0 \leq x \leq 3\):
\end{enumerate}

\[
L = \int_0^3 \sqrt{1+(1)^2}\,dx = \int_0^3 \sqrt{2}\,dx = 3\sqrt{2}.
\]

\begin{enumerate}
\def\labelenumi{\arabic{enumi}.}
\setcounter{enumi}{1}
\tightlist
\item
  Oberfläche einer Kugel Nehmen Sie \(y = \sqrt{r^2 - x^2}\),
  \(-r \leq x \leq r\) und drehen Sie es um die \(x\)-Achse.
\end{enumerate}

\[
S = 2\pi \int_{-r}^r \sqrt{r^2 - x^2}\sqrt{1+\left(\frac{-x}{\sqrt{r^2-x^2}}\right)^2}\,dx.
\]

Vereinfacht ergibt sich \(S = 4\pi r^2\), die bekannte Formel für die
Oberfläche einer Kugel.

\subsubsection{Warum das wichtig ist}\label{warum-das-wichtig-ist-6}

\begin{itemize}
\tightlist
\item
  Die Bogenlänge erweitert die Idee der Entfernung auf gekrümmte Pfade.
\item
  Der Oberflächenbereich der Revolution findet Anwendung in der Physik,
  im Ingenieurwesen und im Design.
\item
  Bietet eine Brücke zwischen Analysis und Geometrie.
\end{itemize}

\subsubsection{Übungen}\label{uxfcbungen-22}

\begin{enumerate}
\def\labelenumi{\arabic{enumi}.}
\tightlist
\item
  Ermitteln Sie die Bogenlänge von \(y=\sqrt{x}\) von \(x=0\) bis
  \(x=4\).
\item
  Berechnen Sie die Oberfläche des Festkörpers, die Sie durch Drehen von
  \(y=x^2\), \(0 \leq x \leq 1\) um die \(x\)-Achse erhalten.
\item
  Ermitteln Sie die Bogenlänge von \(y=\ln(\cosh x)\) von \(x=0\) bis
  \(x=1\).
\item
  Zeigen Sie, dass die Drehung von \(y=\sqrt{r^2 - x^2}\) von \(0\) nach
  \(r\) um die \(x\)-Achse die halbe Oberfläche einer Kugel ergibt.
\item
  Leiten Sie die Formel für die Oberfläche eines Kegels her, indem Sie
  eine Linie drehen.
\end{enumerate}

\subsection{6.3 Arbeit und
Durchschnittswerte}\label{arbeit-und-durchschnittswerte}

Integration ist nicht auf die Geometrie beschränkt. Es hilft auch bei
der Berechnung der von einer Kraft geleisteten Arbeit und des
Durchschnittswerts einer Funktion über ein Intervall.

\subsubsection{Arbeit}\label{arbeit}

Wenn eine variable Kraft \(F(x)\) einen Körper entlang einer geraden
Linie von \(x=a\) nach \(x=b\) bewegt, dann beträgt die Gesamtarbeit

\[
W = \int_a^b F(x)\,dx.
\]

Diese Formel verallgemeinert den einfachen Fall \(W = F \cdot d\) für
konstante Kraft.

Beispiel 1: Federkraft (Hookes Gesetz) Für eine Feder, die von der Länge
\(a\) auf \(b\) gedehnt wird, mit der Kraft \(F(x) = kx\):

\[
W = \int_a^b kx\,dx = \tfrac{1}{2}k(b^2-a^2).
\]

Beispiel 2: Wasser pumpen Wenn Wasser aus einem Tank gepumpt wird, ist
der Arbeitsaufwand gleich

\[
W = \int_a^b \text{(weight density)} \times \text{(cross-sectional area)} \times \text{(distance lifted)} \, dx.
\]

\subsubsection{Durchschnittswert einer
Funktion}\label{durchschnittswert-einer-funktion}

Der Durchschnittswert einer stetigen Funktion beträgt \(f(x)\) auf
\([a,b]\)

\[
f_{\text{avg}} = \frac{1}{b-a} \int_a^b f(x)\,dx.
\]

Dies ist das kontinuierliche Analogon zur Mittelung einer Liste von
Zahlen.

Beispiel 1: Für \(f(x)=x^2\) auf \([0,2]\):

\[
f_{\text{avg}} = \tfrac{1}{2-0}\int_0^2 x^2 dx = \tfrac{1}{2}\cdot \tfrac{8}{3} = \tfrac{4}{3}.
\]

Beispiel 2:Wenn die Geschwindigkeit eines Teilchens \(v(t)\) beträgt,
dann beträgt die Durchschnittsgeschwindigkeit über \([a,b]\)

\[
v_{\text{avg}} = \frac{1}{b-a}\int_a^b v(t)\,dt.
\]

\subsubsection{Warum das wichtig ist}\label{warum-das-wichtig-ist-7}

\begin{itemize}
\tightlist
\item
  Arbeitsintegrale kommen in physikalischen, technischen und
  Energieberechnungen vor.
\item
  Der Durchschnittswert gibt eine einzige repräsentative Zahl für
  unterschiedliche Mengen an.
\item
  Beide verbinden die Infinitesimalrechnung mit realen Problemen der
  Bewegung, Kraft und Effizienz.
\end{itemize}

\subsubsection{Übungen}\label{uxfcbungen-23}

\begin{enumerate}
\def\labelenumi{\arabic{enumi}.}
\tightlist
\item
  Berechnen Sie die Arbeit, die erforderlich ist, um eine Feder von 2 m
  auf 5 m zu dehnen, wenn \(k=10\).
\item
  Ein 100 kg schwerer Gegenstand wird in einem Gravitationsfeld 5 m
  vertikal angehoben (\(g=9.8 \,\text{m/s}^2\)). Drücken Sie die Arbeit
  als Integral aus und bewerten Sie sie.
\item
  Ermitteln Sie den Durchschnittswert von \(f(x)=\sin x\) auf
  \([0,\pi]\).
\item
  Berechnen Sie die Durchschnittstemperatur von
  \(T(t)=20+5\cos(\tfrac{\pi t}{12})\) über einen 24-Stunden-Tag.
\item
  Ein 10 m tiefer Tank ist mit Wasser gefüllt. Berechnen Sie die Arbeit,
  die erforderlich ist, um das gesamte Wasser nach oben zu pumpen,
  vorausgesetzt, das Wasser wiegt \(9800 \,\text{N/m}^3\).
\end{enumerate}

\subsection{6.4 Wahrscheinlichkeitsdichten und kontinuierliche
Verteilungen}\label{wahrscheinlichkeitsdichten-und-kontinuierliche-verteilungen}

Auch in der Wahrscheinlichkeitstheorie spielt die Integration eine
zentrale Rolle, insbesondere für kontinuierliche Zufallsvariablen.
Anstelle diskreter Ergebnisse beschreiben wir Wahrscheinlichkeiten mit
Funktionen, die als Wahrscheinlichkeitsdichtefunktionen (PDFs)
bezeichnet werden.

\subsubsection{Wahrscheinlichkeitsdichtefunktionen}\label{wahrscheinlichkeitsdichtefunktionen}

Eine Wahrscheinlichkeitsdichtefunktion \(f(x)\) muss zwei Bedingungen
erfüllen:

\begin{enumerate}
\def\labelenumi{\arabic{enumi}.}
\item
  \(f(x) \geq 0\) für alle \(x\).
\item
  Die Gesamtfläche unter der Kurve beträgt 1:

  \[
  \int_{-\infty}^\infty f(x)\,dx = 1.
  \]
\end{enumerate}

Wenn \(X\) eine stetige Zufallsvariable mit pdf \(f(x)\) ist, dann
beträgt die Wahrscheinlichkeit, dass \(X\) zwischen \(a\) und \(b\)
liegt

\[
P(a \leq X \leq b) = \int_a^b f(x)\,dx.
\]

\subsubsection{Kumulative
Verteilungsfunktion}\label{kumulative-verteilungsfunktion}

Die kumulative Verteilungsfunktion (cdf) ist definiert als

\[
F(x) = \int_{-\infty}^x f(t)\,dt.
\]

Sie gibt die Wahrscheinlichkeit an, dass die Zufallsvariable kleiner
oder gleich \(x\) ist.

\subsubsection{Erwarteter Wert
(Mittelwert)}\label{erwarteter-wert-mittelwert}

Der erwartete Wert einer kontinuierlichen Zufallsvariablen ist der
gewichtete Durchschnitt:

\[
E[X] = \int_{-\infty}^\infty x f(x)\,dx.
\]

\subsubsection{Beispiele}\label{beispiele-15}

\begin{enumerate}
\def\labelenumi{\arabic{enumi}.}
\tightlist
\item
  Gleichmäßige Verteilung Für \(f(x) = \tfrac{1}{b-a}\) auf \([a,b]\):
\end{enumerate}

\begin{itemize}
\item
  Wahrscheinlichkeit des Intervalls \([c,d]\):

  \[
  P(c \leq X \leq d) = \frac{d-c}{b-a}.
  \]
\item
  Erwarteter Wert: \(E[X] = \tfrac{a+b}{2}\).
\end{itemize}

\begin{enumerate}
\def\labelenumi{\arabic{enumi}.}
\setcounter{enumi}{1}
\tightlist
\item
  Exponentielle Verteilung Für \(f(x) = \lambda e^{-\lambda x}\),
  \(x \geq 0\):
\end{enumerate}

\begin{itemize}
\tightlist
\item
  \(\int_0^\infty \lambda e^{-\lambda x}\,dx = 1\).
\item
  Durchschnitt: \(E[X] = \tfrac{1}{\lambda}\).
\end{itemize}

\begin{enumerate}
\def\labelenumi{\arabic{enumi}.}
\setcounter{enumi}{2}
\tightlist
\item
  Normalverteilung Die Glockenkurve:
\end{enumerate}

\[
f(x) = \frac{1}{\sqrt{2\pi\sigma^2}} e^{-\frac{(x-\mu)^2}{2\sigma^2}}.
\]

Es lässt sich zu 1 integrieren, erfordert jedoch fortgeschrittene
Techniken.

\subsubsection{Warum das wichtig ist- Wahrscheinlichkeitsdichten
beschreiben die Unsicherheit in Naturwissenschaften, Technik und
Statistik.}\label{warum-das-wichtig-ist--wahrscheinlichkeitsdichten-beschreiben-die-unsicherheit-in-naturwissenschaften-technik-und-statistik.}

\begin{itemize}
\tightlist
\item
  Integrale verbinden Flächen unter Kurven mit Wahrscheinlichkeiten.
\item
  Kontinuierliche Verteilungen verallgemeinern die Idee, Ergebnisse zu
  zählen, um Wahrscheinlichkeiten über Intervalle zu messen.
\end{itemize}

\subsubsection{Übungen}\label{uxfcbungen-24}

\begin{enumerate}
\def\labelenumi{\arabic{enumi}.}
\tightlist
\item
  Zeigen Sie, dass die gleichmäßige Dichte \(f(x) = \tfrac{1}{b-a}\) auf
  \([a,b]\) zu 1 integriert.
\item
  Berechnen Sie für die Exponentialverteilung mit \(\lambda = 2\)
  \(P(0 \leq X \leq 1)\).
\item
  Ermitteln Sie den erwarteten Wert von \(X\), wenn \(f(x) = 3x^2\) auf
  \([0,1]\).
\item
  Überprüfen Sie, ob die Normalverteilung mit Mittelwert 0 und Varianz 1
  die Gesamtwahrscheinlichkeit 1 hat (kein vollständiger Beweis
  erforderlich, aber erklären Sie, warum sie gilt).
\item
  Berechnen Sie den CDF der Gleichverteilung auf \([0,1]\).
\end{enumerate}

\section{Teil III.
Multivariablenrechnung}\label{teil-iii.-multivariablenrechnung}

\section{Kapitel 7. Vektorfunktionen und
Kurven}\label{kapitel-7.-vektorfunktionen-und-kurven}

\subsection{7.1 Vektorfunktionen und
Raumkurven}\label{vektorfunktionen-und-raumkurven}

In der Multivariablenrechnung können Funktionen Vektoren anstelle von
Zahlen ausgeben. Diese werden vektorwertige Funktionen genannt und sind
für die Beschreibung von Kurven im Raum unerlässlich.

\subsubsection{Definition}\label{definition-6}

Eine Vektorfunktion ist eine Funktion der Form

\[
\mathbf{r}(t) = \langle x(t), y(t), z(t) \rangle,
\]

wobei \(x(t), y(t), z(t)\) reellwertige Funktionen sind.

\begin{itemize}
\tightlist
\item
  Die Eingabe \(t\) wird oft als Parameter bezeichnet.
\item
  Die Ausgabe ist ein Vektor im 2D- oder 3D-Raum.
\item
  Der Graph einer Vektorfunktion in 3D ist eine Raumkurve.
\end{itemize}

\subsubsection{Beispiele}\label{beispiele-16}

\begin{enumerate}
\def\labelenumi{\arabic{enumi}.}
\tightlist
\item
  Linie
\end{enumerate}

\[
\mathbf{r}(t) = \langle 1+2t, \; 3-t, \; 4+5t \rangle.
\]

Dieser beschreibt eine Gerade durch den Punkt \((1,3,4)\) mit
Richtungsvektor \(\langle 2,-1,5 \rangle\).

\begin{enumerate}
\def\labelenumi{\arabic{enumi}.}
\setcounter{enumi}{1}
\tightlist
\item
  Kreisen Sie in der Ebene
\end{enumerate}

\[
\mathbf{r}(t) = \langle \cos t, \; \sin t, \; 0 \rangle, \quad 0 \leq t < 2\pi.
\]

\begin{enumerate}
\def\labelenumi{\arabic{enumi}.}
\setcounter{enumi}{2}
\tightlist
\item
  Helix
\end{enumerate}

\[
\mathbf{r}(t) = \langle \cos t, \; \sin t, \; t \rangle.
\]

Dabei handelt es sich um eine Spirale, die um die \(z\)-Achse ansteigt.

\subsubsection{Grenzen und
Kontinuität}\label{grenzen-und-kontinuituxe4t}

Eine Vektorfunktion ist stetig bei \(t=a\), wenn jede Komponente
\(x(t), y(t), z(t)\) stetig bei \(t=a\) ist.

\[
\lim_{t \to a} \mathbf{r}(t) = \langle \lim_{t \to a} x(t), \; \lim_{t \to a} y(t), \; \lim_{t \to a} z(t) \rangle.
\]

\subsubsection{Geometrie der Raumkurven}\label{geometrie-der-raumkurven}

\begin{itemize}
\tightlist
\item
  Jede Kurve hat eine durch die Ableitung gegebene Tangentenrichtung.
\item
  Raumkurven können Bewegungspfade, Partikelbahnen und geometrische
  Formen modellieren.
\end{itemize}

\subsubsection{Warum das wichtig ist}\label{warum-das-wichtig-ist-8}

Vektorfunktionen sind die Grundlage für die Multivariablenrechnung und
ermöglichen es uns, die Ideen von Ableitungen und Integralen auf höhere
Dimensionen auszudehnen. Sie kommen auch natürlicherweise in der Physik
vor (Bewegung in 3D, Elektromagnetismus, Strömungsdynamik).

\subsubsection{Übungen}\label{uxfcbungen-25}

\begin{enumerate}
\def\labelenumi{\arabic{enumi}.}
\tightlist
\item
  Schreiben Sie eine Vektorfunktion für eine Gerade durch \((0,1,2)\)
  parallel zum Vektor \(\langle 3,-2,1 \rangle\).2. Beschreiben Sie die
  durch \(\mathbf{r}(t) = \langle 2\cos t, \; 2\sin t, \; 3 \rangle\)
  gegebene Kurve.
\item
  Bestimmen Sie, ob
  \(\mathbf{r}(t) = \langle e^t, \; \ln t, \; t^2 \rangle\) bei \(t=1\)
  stetig ist.
\item
  Skizzieren Sie die Helix
  \(\mathbf{r}(t) = \langle \cos t, \; \sin t, \; 2t \rangle\).
\item
  Finden Sie den Punkt auf der Kurve
  \(\mathbf{r}(t) = \langle t, \; t^2, \; t^3 \rangle\) wenn \(t=2\).
\end{enumerate}

\subsection{7.2 Ableitungen und Integrale von
Vektorfunktionen}\label{ableitungen-und-integrale-von-vektorfunktionen}

Vektorfunktionen können wie gewöhnliche Funktionen differenziert und
integriert werden -- wir wenden die Operation einfach auf jede
Komponente an. Dadurch können wir Bewegung, Geschwindigkeit,
Beschleunigung und Akkumulation in höheren Dimensionen untersuchen.

\subsubsection{Ableitung einer
Vektorfunktion}\label{ableitung-einer-vektorfunktion}

Wenn

\[
\mathbf{r}(t) = \langle x(t), y(t), z(t) \rangle,
\]

dann

\[
\mathbf{r}'(t) = \langle x'(t), y'(t), z'(t) \rangle.
\]

Dieser Ableitungsvektor zeigt in Tangentenrichtung zur Kurve beim
Parameter \(t\).

\begin{itemize}
\tightlist
\item
  Geschwindigkeit: Wenn \(\mathbf{r}(t)\) die Position eines Teilchens
  zum Zeitpunkt \(t\) angibt, dann ist
  \(\mathbf{v}(t) = \mathbf{r}'(t)\) sein Geschwindigkeitsvektor.
\item
  Geschwindigkeit: Die Größe \(|\mathbf{v}(t)|\) ist die Geschwindigkeit
  des Teilchens.
\item
  Beschleunigung: \(\mathbf{a}(t) = \mathbf{v}'(t) = \mathbf{r}''(t)\).
\end{itemize}

\subsubsection{Beispiele}\label{beispiele-17}

\begin{enumerate}
\def\labelenumi{\arabic{enumi}.}
\tightlist
\item
  Helix
\end{enumerate}

\[
\mathbf{r}(t) = \langle \cos t, \sin t, t \rangle.
\]

\begin{itemize}
\tightlist
\item
  Geschwindigkeit:
  \(\mathbf{v}(t) = \langle -\sin t, \cos t, 1 \rangle\).
\item
  Geschwindigkeit:
  \(|\mathbf{v}(t)| = \sqrt{(-\sin t)^2 + (\cos t)^2 + 1^2} = \sqrt{2}\).
\item
  Beschleunigung:
  \(\mathbf{a}(t) = \langle -\cos t, -\sin t, 0 \rangle\).
\end{itemize}

\begin{enumerate}
\def\labelenumi{\arabic{enumi}.}
\setcounter{enumi}{1}
\tightlist
\item
  Projektilbewegung
\end{enumerate}

\[
\mathbf{r}(t) = \langle v_0 \cos\theta \cdot t, \; v_0 \sin\theta \cdot t - \tfrac{1}{2}gt^2 \rangle.
\]

Dies modelliert die parabolische Bahn eines Projektils unter der
Schwerkraft.

\subsubsection{Integral einer
Vektorfunktion}\label{integral-einer-vektorfunktion}

Wenn

\[
\mathbf{r}(t) = \langle x(t), y(t), z(t) \rangle,
\]

dann

\[
\int \mathbf{r}(t)\,dt = \left\langle \int x(t)\,dt, \; \int y(t)\,dt, \; \int z(t)\,dt \right\rangle + \mathbf{C},
\]

wobei \(\mathbf{C}\) ein konstanter Vektor ist.

\subsubsection{Beispiel}\label{beispiel}

\[
\mathbf{r}(t) = \langle t, t^2, t^3 \rangle.
\]

\begin{itemize}
\tightlist
\item
  Derivat: \(\mathbf{r}'(t) = \langle 1, 2t, 3t^2 \rangle\).
\item
  Integral:
\end{itemize}

\[
\int \mathbf{r}(t)\,dt = \langle \tfrac{1}{2}t^2, \tfrac{1}{3}t^3, \tfrac{1}{4}t^4 \rangle + \mathbf{C}.
\]

\subsubsection{Warum das wichtig ist}\label{warum-das-wichtig-ist-9}

\begin{itemize}
\tightlist
\item
  Ableitungen von Vektorfunktionen beschreiben Bewegungen und Kräfte im
  Raum.
\item
  Integrale geben Verschiebung, Arbeit und akkumulierte Größen an.
\item
  Diese Tools verbinden die Analysis direkt mit der Physik und den
  Ingenieurwissenschaften.
\end{itemize}

\subsubsection{Übungen}\label{uxfcbungen-26}

\begin{enumerate}
\def\labelenumi{\arabic{enumi}.}
\tightlist
\item
  Ermitteln Sie für
  \(\mathbf{r}(t) = \langle t, \cos t, \sin t \rangle\) Geschwindigkeit,
  Geschwindigkeit und Beschleunigung.2. Berechnen Sie \(\mathbf{r}'(t)\)
  für \(\mathbf{r}(t) = \langle e^t, \ln t, t^2 \rangle\).
\item
  Integrieren Sie \(\mathbf{r}(t) = \langle 1, t, t^2 \rangle\).
\item
  Ein Teilchen hat die Geschwindigkeit
  \(\mathbf{v}(t) = \langle t, 2, 0 \rangle\). Finden Sie seinen
  Positionsvektor, wenn \(\mathbf{r}(0) = \langle 1, 0, 0 \rangle\).
\item
  Zeigen Sie, dass die Geschwindigkeit von
  \(\mathbf{r}(t) = \langle \cos t, \sin t, 0 \rangle\) konstant ist.
\end{enumerate}

\subsection{7.3 Bogenlänge und
Krümmung}\label{bogenluxe4nge-und-kruxfcmmung}

Die Vektorrechnung bietet Werkzeuge, um nicht nur den von einer Kurve
verfolgten Weg zu messen, sondern auch, wie stark sie sich biegt. Diese
werden durch Bogenlänge und Krümmung ausgedrückt.

\subsubsection{Bogenlänge einer
Raumkurve}\label{bogenluxe4nge-einer-raumkurve}

Wenn eine Kurve gegeben ist durch

\[
\mathbf{r}(t) = \langle x(t), y(t), z(t) \rangle, \quad a \leq t \leq b,
\]

dann beträgt die Bogenlänge

\[
L = \int_a^b |\mathbf{r}'(t)|\,dt,
\]

wo

\[
|\mathbf{r}'(t)| = \sqrt{(x'(t))^2 + (y'(t))^2 + (z'(t))^2}.
\]

Beispiel: Für die Helix
\(\mathbf{r}(t) = \langle \cos t, \sin t, t \rangle, \, 0 \leq t \leq 2\pi\):

\begin{itemize}
\tightlist
\item
  Geschwindigkeit:
  \(\mathbf{r}'(t) = \langle -\sin t, \cos t, 1 \rangle\).
\item
  Geschwindigkeit:
  \(|\mathbf{r}'(t)| = \sqrt{(-\sin t)^2 + (\cos t)^2 + 1^2} = \sqrt{2}\).
\item
  Bogenlänge:
\end{itemize}

\[
L = \int_0^{2\pi} \sqrt{2}\,dt = 2\pi\sqrt{2}.
\]

\subsubsection{Curvature}\label{curvature}

Die Krümmung misst, wie schnell eine Kurve ihre Richtung ändert.

Für eine glatte Kurve \(\mathbf{r}(t)\):

\[
\kappa(t) = \frac{|\mathbf{r}'(t) \times \mathbf{r}''(t)|}{|\mathbf{r}'(t)|^3}.
\]

\begin{itemize}
\tightlist
\item
  \(\kappa = 0\): gerade Linie.
\item
  Größer \(\kappa\): Kurve knickt stärker ab.
\end{itemize}

Beispiel: Für einen Kreis mit dem Radius \(r\):

\[
\mathbf{r}(t) = \langle r\cos t, r\sin t \rangle.
\]

Dann \(\kappa = \tfrac{1}{r}\). Die Krümmung ist also konstant und
umgekehrt proportional zum Radius.

\subsubsection{Einheitstangenten- und
Normalenvektoren}\label{einheitstangenten--und-normalenvektoren}

\begin{itemize}
\tightlist
\item
  Tangentenvektor:
\end{itemize}

\[
\mathbf{T}(t) = \frac{\mathbf{r}'(t)}{|\mathbf{r}'(t)|}.
\]

\begin{itemize}
\tightlist
\item
  Normalenvektor: zeigt auf den Krümmungsmittelpunkt, definiert als
\end{itemize}

\[
\mathbf{N}(t) = \frac{\mathbf{T}'(t)}{|\mathbf{T}'(t)|}.
\]

Diese Vektoren beschreiben die Bewegungsgeometrie: Fahrtrichtung und
Drehrichtung.

\subsubsection{Warum das wichtig ist}\label{warum-das-wichtig-ist-10}

\begin{itemize}
\tightlist
\item
  Die Bogenlänge verallgemeinert das Konzept der Entfernung zu Kurven im
  Raum.
\item
  Krümmung beschreibt Biegung, die in der Physik
  (Zentripetalbeschleunigung), im Ingenieurwesen (Straßen, Achterbahnen)
  und in der Computergrafik von entscheidender Bedeutung ist.
\end{itemize}

\subsubsection{Übungen}\label{uxfcbungen-27}

\begin{enumerate}
\def\labelenumi{\arabic{enumi}.}
\tightlist
\item
  Ermitteln Sie die Bogenlänge von
  \(\mathbf{r}(t) = \langle t, t^2, 0 \rangle\) von \(t=0\) bis \(t=1\).
\item
  Berechnen Sie die Krümmung des Kreises
  \(\mathbf{r}(t) = \langle \cos t, \sin t \rangle\).
\item
  Berechnen Sie für
  \(\mathbf{r}(t) = \langle t, \cos t, \sin t \rangle\)
  \(|\mathbf{r}'(t)|\).
\item
  Zeigen Sie, dass eine Gerade die Krümmung \(\kappa = 0\) hat.5. Finden
  Sie den Tangentenvektor an
  \(\mathbf{r}(t) = \langle e^t, e^{-t}, t \rangle\) bei \(t=0\).
\end{enumerate}

\subsection{7.4 Motion in Space}\label{motion-in-space}

Vektorfunktionen eignen sich besonders gut zur Beschreibung von
Bewegungen in zwei oder drei Dimensionen. Position, Geschwindigkeit und
Beschleunigung werden natürlich durch Ableitungen und Integrale
vektorwertiger Funktionen ausgedrückt.

\subsubsection{Position, Geschwindigkeit und
Beschleunigung}\label{position-geschwindigkeit-und-beschleunigung}

\begin{itemize}
\tightlist
\item
  Position vector:
\end{itemize}

\[
\mathbf{r}(t) = \langle x(t), y(t), z(t) \rangle
\]

\begin{itemize}
\tightlist
\item
  Geschwindigkeitsvektor (Ableitung der Position):
\end{itemize}

\[
\mathbf{v}(t) = \mathbf{r}'(t) = \langle x'(t), y'(t), z'(t) \rangle
\]

\begin{itemize}
\tightlist
\item
  Geschwindigkeit (Größe der Geschwindigkeit):
\end{itemize}

\[
|\mathbf{v}(t)| = \sqrt{(x'(t))^2 + (y'(t))^2 + (z'(t))^2}
\]

\begin{itemize}
\tightlist
\item
  Beschleunigungsvektor (Ableitung der Geschwindigkeit):
\end{itemize}

\[
\mathbf{a}(t) = \mathbf{v}'(t) = \mathbf{r}''(t).
\]

\subsubsection{Tangential- und
Normalkomponenten}\label{tangential--und-normalkomponenten}

Die Beschleunigung kann in zwei Komponenten zerlegt werden:

\[
\mathbf{a}(t) = a_T \mathbf{T}(t) + a_N \mathbf{N}(t),
\]

where:

\begin{itemize}
\tightlist
\item
  \(\mathbf{T}(t)\) = Einheitstangensvektor,
\item
  \(\mathbf{N}(t)\) = Hauptnormalenvektor,
\item
  \(a_T = \frac{d}{dt}|\mathbf{v}(t)|\) = Tangentialbeschleunigung
  (Geschwindigkeitsänderung),
\item
  \(a_N = \kappa |\mathbf{v}(t)|^2\) = normale Beschleunigung
  (Richtungsänderung).
\end{itemize}

\subsubsection{Projektilbewegung in 3D}\label{projektilbewegung-in-3d}

Bei Schwerkraftwirkung in der \(-z\)-Richtung:

\[
\mathbf{r}(t) = \langle v_0 \cos\theta \cos\phi \cdot t,\; v_0 \cos\theta \sin\phi \cdot t,\; v_0 \sin\theta \cdot t - \tfrac{1}{2}gt^2 \rangle,
\]

Dabei ist \(v_0\) die Anfangsgeschwindigkeit, \(\theta\) Startwinkel und
\(\phi\) die Azimutrichtung.

\subsubsection{Beispiel: Spiralbewegung}\label{beispiel-spiralbewegung}

\[
\mathbf{r}(t) = \langle \cos t, \sin t, t \rangle
\]

\begin{itemize}
\tightlist
\item
  Velocity: \(\mathbf{v}(t) = \langle -\sin t, \cos t, 1 \rangle\).
\item
  Speed: \(|\mathbf{v}(t)| = \sqrt{2}\).
\item
  Beschleunigung:
  \(\mathbf{a}(t) = \langle -\cos t, -\sin t, 0 \rangle\).
\item
  Die Bewegung hat eine gleichmäßige Geschwindigkeit und verläuft
  spiralförmig nach oben.
\end{itemize}

\subsubsection{Warum das wichtig ist}\label{warum-das-wichtig-ist-11}

\begin{itemize}
\tightlist
\item
  Bietet mathematische Sprache für Bewegungen in der realen Welt.
\item
  Wesentlich in der Physik (Kräfte, Flugbahnen, Kreisbewegung).
\item
  Grundlage für fortgeschrittene Mechanik und technische Modelle.
\end{itemize}

\subsubsection{Übungen}\label{uxfcbungen-28}

\begin{enumerate}
\def\labelenumi{\arabic{enumi}.}
\tightlist
\item
  Ein Teilchen bewegt sich entlang
  \(\mathbf{r}(t) = \langle t, t^2, t^3 \rangle\). Finden Sie
  Geschwindigkeit und Beschleunigung bei \(t=1\).
\item
  Zeigen Sie, dass die Geschwindigkeit der Helix
  \(\mathbf{r}(t) = \langle \cos t, \sin t, t \rangle\) konstant ist.
\item
  Ein Projektil wird mit \(v_0 = 20 \,\text{m/s}\) im Winkel
  \(45^\circ\) abgefeuert. Schreiben Sie seinen Positionsvektor unter
  der Annahme einer Bewegung in einer vertikalen Ebene.
\item
  Für \(\mathbf{r}(t) = \langle e^t, e^{-t}, t \rangle\) finden Sie
  \(\mathbf{v}(t)\) und \(\mathbf{a}(t)\).
\item
  Zerlegen Sie den Beschleunigungsvektor in Tangential- und
  Normalkomponenten für die Bewegung entlang eines Kreises mit dem
  Radius \(r\).\# Kapitel 8. Funktionen mehrerer Variablen
\end{enumerate}

\subsection{8.1 Grenzen und Kontinuität in mehreren
Variablen}\label{grenzen-und-kontinuituxe4t-in-mehreren-variablen}

In der Multivariablenrechnung können Funktionen von zwei oder mehr
Variablen abhängen, beispielsweise \(f(x,y)\) oder \(f(x,y,z)\). Die
Konzepte von Grenzen und Kontinuität gehen auf natürliche Weise aus der
Einzelvariablenrechnung hervor, sind jedoch subtiler, da wir alle
möglichen Herangehensweisen berücksichtigen müssen.

\subsubsection{Grenzwerte in zwei
Variablen}\label{grenzwerte-in-zwei-variablen}

Für eine Funktion sagen wir \(f(x,y)\)

\[
\lim_{(x,y) \to (a,b)} f(x,y) = L
\]

wenn \(f(x,y)\) willkürlich nahe an \(L\) herankommt, während sich
\((x,y)\) entlang eines beliebigen Pfades \((a,b)\) nähert.

Wenn unterschiedliche Pfade unterschiedliche Grenzwerte ergeben, dann
existiert der Grenzwert nicht.

Beispiel 1 (Limit vorhanden):

\[
f(x,y) = x^2 + y^2, \quad \lim_{(x,y) \to (0,0)} f(x,y) = 0.
\]

Beispiel 2 (Limit existiert nicht):

\[
f(x,y) = \frac{xy}{x^2+y^2}, \quad (x,y) \to (0,0).
\]

\begin{itemize}
\tightlist
\item
  Entlang \(y=0\) ist die Funktion 0.
\item
  Neben \(y=x\) ist die Funktion \(\tfrac{1}{2}\). Unterschiedliche
  Ergebnisse → Grenzwert existiert nicht.
\end{itemize}

\subsubsection{Continuity}\label{continuity}

Eine Funktion \(f(x,y)\) ist stetig bei \((a,b)\) wenn

\[
\lim_{(x,y)\to(a,b)} f(x,y) = f(a,b).
\]

Polynome und rationale Funktionen (mit Nenner ≠ 0) sind überall in ihren
Definitionsbereichen stetig.

\subsubsection{Erweiterung auf drei oder mehr
Variablen}\label{erweiterung-auf-drei-oder-mehr-variablen}

Für \(f(x,y,z)\) werden Grenzen und Kontinuität auf die gleiche Weise
definiert, aber der Punkt \((a,b,c)\) muss aus unendlich vielen
Richtungen im Raum angefahren werden.

\subsubsection{Warum das wichtig ist}\label{warum-das-wichtig-ist-12}

\begin{itemize}
\tightlist
\item
  Kontinuität stellt sicher, dass es in multivariablen Funktionen keine
  Sprünge, Lücken oder Asymptoten gibt.
\item
  Grenzwerte sind von grundlegender Bedeutung für die Definition
  partieller Ableitungen und mehrerer Integrale.
\item
  Diese Konzepte sind Bausteine \hspace{0pt}\hspace{0pt}für die
  Multivariablenrechnung.
\end{itemize}

\subsubsection{Übungen}\label{uxfcbungen-29}

\begin{enumerate}
\def\labelenumi{\arabic{enumi}.}
\tightlist
\item
  Stellen Sie fest, ob \(\lim_{(x,y)\to(0,0)} (x^2+y^2)\) vorhanden ist.
\item
  Zeigen Sie, dass \(\lim_{(x,y)\to(0,0)} \frac{x^2y}{x^2+y^2} = 0\)
  entlang aller geraden Wege \(y=mx\) ist.
\item
  Gibt es das Limit für \(f(x,y) = \frac{x^2-y^2}{x^2+y^2}\) wie
  \((x,y)\to(0,0)\)?
\item
  Erklären Sie, warum Polynome in zwei Variablen überall stetig sind.
\item
  Geben Sie ein Beispiel für eine Funktion zweier Variablen, die an
  einem Punkt unstetig ist, und erklären Sie, warum.
\end{enumerate}

\subsection{8.2 Partielle Ableitungen}\label{partielle-ableitungen}

Bei Funktionen mehrerer Variablen möchten wir häufig messen, wie sich
die Funktion ändert, wenn sich nur eine Variable ändert, während die
anderen konstant gehalten werden. Dies führt zur Idee der partiellen
Ableitungen.

\subsubsection{Definition}\label{definition-7}

Für eine Funktion \(f(x,y)\) beträgt die partielle Ableitung nach \(x\)
an einem Punkt \((a,b)\)

\[
\frac{\partial f}{\partial x}(a,b) = \lim_{h \to 0} \frac{f(a+h, b) - f(a,b)}{h}.
\]

Ebenso beträgt die partielle Ableitung bezüglich \(y\)

\[\frac{\partial f}{\partial y}(a,b) = \lim_{h \to 0} \frac{f(a, b+h) - f(a,b)}{h}.
\]

We treat all other variables as constants when differentiating.

\subsubsection{Notation}\label{notation-1}

\begin{itemize}
\tightlist
\item
  \(\frac{\partial f}{\partial x}\), \(f_x\), \(\partial_x f\).
\item
  \(\frac{\partial f}{\partial y}\), \(f_y\), \(\partial_y f\).
\end{itemize}

For three variables \(f(x,y,z)\), we also have \(f_x, f_y, f_z\).

\subsubsection{Examples}\label{examples-2}

\begin{enumerate}
\def\labelenumi{\arabic{enumi}.}
\tightlist
\item
  \(f(x,y) = x^2y + y^3\)
\end{enumerate}

\begin{itemize}
\tightlist
\item
  \(f_x = 2xy\).
\item
  \(f_y = x^2 + 3y^2\).
\end{itemize}

\begin{enumerate}
\def\labelenumi{\arabic{enumi}.}
\setcounter{enumi}{1}
\tightlist
\item
  \(f(x,y) = e^{xy}\)
\end{enumerate}

\begin{itemize}
\tightlist
\item
  \(f_x = y e^{xy}\).
\item
  \(f_y = x e^{xy}\).
\end{itemize}

\begin{enumerate}
\def\labelenumi{\arabic{enumi}.}
\setcounter{enumi}{2}
\tightlist
\item
  \(f(x,y,z) = x^2 + yz\)
\end{enumerate}

\begin{itemize}
\tightlist
\item
  \(f_x = 2x\).
\item
  \(f_y = z\).
\item
  \(f_z = y\).
\end{itemize}

\subsubsection{Higher-Order Partial
Derivatives}\label{higher-order-partial-derivatives}

We can take partial derivatives repeatedly:

\begin{itemize}
\tightlist
\item
  \(f_{xx} = \frac{\partial}{\partial x}\Big(f_x\Big)\).
\item
  \(f_{yy}, f_{xy}, f_{yx}\), etc.
\end{itemize}

Clairaut's Theorem: If \(f\) has continuous second partial derivatives,
then

\[
f_{xy} = f_{yx}.
\]

\subsubsection{Geometric Meaning}\label{geometric-meaning}

\begin{itemize}
\tightlist
\item
  \(f_x\): slope of the surface in the \(x\)-direction.
\item
  \(f_y\): slope of the surface in the \(y\)-direction.
\item
  Together they describe how the surface tilts.
\end{itemize}

\subsubsection{Why This Matters}\label{why-this-matters}

\begin{itemize}
\tightlist
\item
  Partial derivatives are the foundation of gradients, tangent planes,
  and optimization in multiple variables.
\item
  They are widely used in physics, engineering, and economics to model
  systems with several inputs.
\end{itemize}

\subsubsection{Exercises}\label{exercises-3}

\begin{enumerate}
\def\labelenumi{\arabic{enumi}.}
\tightlist
\item
  Find \(f_x\) and \(f_y\) for \(f(x,y) = x^3y^2\).
\item
  Compute \(f_x, f_y, f_z\) for \(f(x,y,z) = xyz + x^2\).
\item
  Verify Clairaut's theorem for \(f(x,y) = x^2y + y^3\).
\item
  Interpret geometrically what \(f_x\) and \(f_y\) mean for
  \(f(x,y) = \sqrt{x^2+y^2}\).
\item
  Find all second-order partial derivatives of \(f(x,y) = e^{x^2+y^2}\).
\end{enumerate}

\subsection{8.3 Gradient and Directional
Derivatives}\label{gradient-and-directional-derivatives}

Partial derivatives measure change along the coordinate axes, but
sometimes we want to know the rate of change of a function in any
direction. This leads to the concepts of the gradient and directional
derivatives.

\subsubsection{Gradient Vector}\label{gradient-vector}

For a function \(f(x,y)\), the gradient is the vector

\[
\nabla f(x,y) = \left\langle \frac{\partial f}{\partial x}, \frac{\partial f}{\partial y} \right\rangle.
\]

For three variables \(f(x,y,z)\):

\[
\nabla f(x,y,z) = \left\langle f_x, f_y, f_z \right\rangle.
\]

The gradient points in the direction of maximum increase of the
function, and its magnitude gives the steepest slope.

\subsubsection{Directional Derivatives}\label{directional-derivatives}

The rate of change of \(f(x,y)\) at a point in the direction of a unit
vector \(\mathbf{u} = \langle u_1, u_2 \rangle\) is

\[
D_{\mathbf{u}} f(x,y) = \nabla f(x,y)\cdot\mathbf{u}.
\]

Dies ist das Skalarprodukt des Gradienten mit dem Richtungsvektor.

\subsubsection{Beispiele}\label{beispiele-18}

\begin{enumerate}
\def\labelenumi{\arabic{enumi}.}
\tightlist
\item
  \(f(x,y) = x^2 + y^2\)
\end{enumerate}

\begin{itemize}
\tightlist
\item
  Gradient: \(\nabla f = \langle 2x, 2y \rangle\).- At (1,2):
  \(\nabla f = \langle 2,4 \rangle\).
\item
  Richtungsableitung entlang
  \(\mathbf{u} = \langle \tfrac{3}{5}, \tfrac{4}{5} \rangle\):
\end{itemize}

\[
D_{\mathbf{u}} f(1,2) = \langle 2,4 \rangle \cdot \langle \tfrac{3}{5}, \tfrac{4}{5} \rangle = \tfrac{26}{5}.
\]

\begin{enumerate}
\def\labelenumi{\arabic{enumi}.}
\setcounter{enumi}{1}
\tightlist
\item
  \(f(x,y,z) = x y z\)
\end{enumerate}

\begin{itemize}
\tightlist
\item
  Gradient: \(\nabla f = \langle yz, xz, xy \rangle\).
\item
  At (1,1,1): \(\nabla f = \langle 1,1,1 \rangle\).
\item
  Die maximale Anstiegsrichtung ist entlang \(\langle 1,1,1 \rangle\).
\end{itemize}

\subsubsection{Geometrische
Interpretation}\label{geometrische-interpretation-1}

\begin{itemize}
\tightlist
\item
  Der Steigungsvektor steht senkrecht (normal) auf Niveaukurven oder
  Niveauflächen von \(f\).
\item
  Richtungsableitungen verallgemeinern die Steigung in beliebige
  Richtungen.
\end{itemize}

\subsubsection{Warum das wichtig ist}\label{warum-das-wichtig-ist-13}

\begin{itemize}
\tightlist
\item
  Bei der Optimierung gibt uns der Gradient die Bewegungsrichtung für
  den steilsten Anstieg oder Abstieg an.
\item
  In der Physik beschreiben Gradienten Felder wie Wärmefluss und
  elektrisches Potenzial.
\item
  Richtungsableitungen vereinheitlichen einvariable und multivariable
  Änderungsraten.
\end{itemize}

\subsubsection{Übungen}\label{uxfcbungen-30}

\begin{enumerate}
\def\labelenumi{\arabic{enumi}.}
\tightlist
\item
  Berechnen Sie \(\nabla f(x,y)\) für \(f(x,y) = e^{xy}\).
\item
  Finden Sie den Gradienten von \(f(x,y,z) = x^2+y^2+z^2\) und werten
  Sie ihn bei (1,1,1) aus.
\item
  Berechnen Sie die Richtungsableitung von \(f(x,y) = x^2-y\) bei (2,1)
  in Richtung von \(\mathbf{u} = \langle 0,1 \rangle\).
\item
  Zeigen Sie, dass die Steigung von \(f(x,y) = x^2+y^2\) senkrecht zum
  Kreis \(x^2+y^2=1\) verläuft.
\item
  Finden Sie die Einheitsvektorrichtung, die die Richtungsableitung von
  \(f(x,y) = xy\) bei (1,2) maximiert.
\end{enumerate}

\subsection{8.4 Tangentenebenen und lineare
Approximationen}\label{tangentenebenen-und-lineare-approximationen}

In der Einzelvariablenrechnung nähert sich die Tangente einer Kurve in
der Nähe eines Punktes an. In der Multivariablenrechnung ist das analoge
Konzept die Tangentenebene, die eine lineare Annäherung an eine
Oberfläche in der Nähe eines Punktes liefert.

\subsubsection{Tangente Ebene zu einer
Oberfläche}\label{tangente-ebene-zu-einer-oberfluxe4che}

Angenommen, \(z = f(x,y)\) ist zu \((a,b)\) differenzierbar. Die
Tangentenebene bei \((a,b,f(a,b))\) ist gegeben durch

\[
z = f(a,b) + f_x(a,b)(x-a) + f_y(a,b)(y-b).
\]

Diese Ebene berührt die Oberfläche an diesem Punkt und nähert sich ihr
in der Nähe an.

\subsubsection{Beispiel 1: Paraboloid}\label{beispiel-1-paraboloid}

Für \(f(x,y) = x^2 + y^2\) bei \((1,2)\):

\begin{itemize}
\tightlist
\item
  \(f(1,2) = 1^2+2^2=5\).
\item
  \(f_x = 2x\), also \(f_x(1,2) = 2\).
\item
  \(f_y = 2y\), also \(f_y(1,2) = 4\).
\end{itemize}

Gleichung der Tangentenebene:

\[
z = 5 + 2(x-1) + 4(y-2).
\]

\subsubsection{Lineare Näherung}\label{lineare-nuxe4herung}

Die Tangentialebene kann zur Annäherung an \(f(x,y)\) in der Nähe von
\((a,b)\) verwendet werden:

\[
f(x,y) \approx f(a,b) + f_x(a,b)(x-a) + f_y(a,b)(y-b).
\]

Dies ist die Linearisierung von \(f\) zu \((a,b)\).

\subsubsection{Beispiel 2: Lineare
Näherung}\label{beispiel-2-lineare-nuxe4herung}

Ungefähr \(f(x,y) = \sqrt{x+y}\), ungefähr \((4,5)\).

\begin{itemize}
\tightlist
\item
  \(f(4,5) = \sqrt{9} = 3\).
\item
  \(f_x = \frac{1}{2\sqrt{x+y}}, \quad f_y = \frac{1}{2\sqrt{x+y}}\).
\item
  Bei (4,5): \(f_x = f_y = \tfrac{1}{6}\).
\end{itemize}

Also,

\[f(x,y) \ungefähr 3 + \tfrac{1}{6}(x-4) + \tfrac{1}{6}(y-5).
\]

\subsubsection{Why This Matters}\label{why-this-matters-1}

\begin{itemize}
\tightlist
\item
  Tangent planes give the best linear approximation to a surface.
\item
  Linearization simplifies complex functions for computation.
\item
  Widely used in numerical methods, physics, and economics.
\end{itemize}

\subsubsection{Exercises}\label{exercises-4}

\begin{enumerate}
\def\labelenumi{\arabic{enumi}.}
\tightlist
\item
  Find the tangent plane to \(z = x^2y + y^2\) at \((1,1)\).
\item
  Approximate \(f(x,y) = e^{x+y}\) near \((0,0)\).
\item
  Derive the tangent plane equation for \(z = \ln(x^2+y^2)\) at
  \((1,1)\).
\item
  Use linear approximation to estimate \(\sqrt{10.1}\) using
  \(f(x,y) = \sqrt{x+y}\) near (4,6).
\item
  Explain why the tangent plane approximation improves as \((x,y)\) gets
  closer to \((a,b)\).
\end{enumerate}

\subsection{8.5 Optimization in Several
Variables}\label{optimization-in-several-variables}

Optimization in multivariable calculus extends the ideas of maxima and
minima from single-variable functions to functions of two or more
variables.

\subsubsection{Critical Points}\label{critical-points}

For \(f(x,y)\), a critical point occurs where

\[
f_x(x,y) = 0 \quad \text{and} \quad f_y(x,y) = 0,
\]

or where the partial derivatives do not exist.

\subsubsection{Second Derivative Test}\label{second-derivative-test}

To classify critical points, compute the second partial derivatives:

\[
D = f_{xx}(a,b) f_{yy}(a,b) - \big(f_{xy}(a,b)\big)^2.
\]

\begin{itemize}
\tightlist
\item
  If \(D > 0\) and \(f_{xx}(a,b) > 0\): local minimum.
\item
  If \(D > 0\) and \(f_{xx}(a,b) < 0\): local maximum.
\item
  If \(D < 0\): saddle point.
\item
  If \(D = 0\): test is inconclusive.
\end{itemize}

\subsubsection{Example 1: Paraboloid}\label{example-1-paraboloid}

\(f(x,y) = x^2 + y^2\).

\begin{itemize}
\tightlist
\item
  \(f_x = 2x, f_y = 2y\). Critical point at (0,0).
\item
  \(f_{xx} = 2, f_{yy} = 2, f_{xy} = 0\).
\item
  \(D = (2)(2) - 0 = 4 > 0\), and \(f_{xx} > 0\).
\item
  So (0,0) is a local minimum.
\end{itemize}

\subsubsection{Example 2: Saddle Point}\label{example-2-saddle-point}

\(f(x,y) = x^2 - y^2\).

\begin{itemize}
\tightlist
\item
  \(f_x = 2x, f_y = -2y\). Critical point at (0,0).
\item
  \(f_{xx} = 2, f_{yy} = -2, f_{xy} = 0\).
\item
  \(D = (2)(-2) - 0 = -4 < 0\).
\item
  So (0,0) is a saddle point.
\end{itemize}

\subsubsection{Constrained Optimization and Lagrange
Multipliers}\label{constrained-optimization-and-lagrange-multipliers}

Sometimes, we want to optimize \(f(x,y)\) subject to a constraint
\(g(x,y) = c\).

Method of Lagrange multipliers: solve

\[
\nabla f(x,y) = \lambda \nabla g(x,y).
\]

Beispiel: \(f(x,y) = xy\) abhängig von \(x^2+y^2=1\) maximieren.

\begin{itemize}
\tightlist
\item
  Steigungen:
  \(\nabla f = \langle y,x \rangle, \quad \nabla g = \langle 2x,2y \rangle\).
\item
  Gleichungen: \(y = 2\lambda x, \, x = 2\lambda y\).
\item
  Lösungen führen zu max. bei
  \((\pm \tfrac{1}{\sqrt{2}}, \pm \tfrac{1}{\sqrt{2}})\).
\end{itemize}

\subsubsection{Warum das wichtig ist}\label{warum-das-wichtig-ist-14}

\begin{itemize}
\tightlist
\item
  Optimierung ist in den Bereichen Wirtschaft, Ingenieurwesen,
  maschinelles Lernen und Physik von entscheidender Bedeutung.
\item
  Lagrange-Multiplikatoren ermöglichen die Optimierung mit
  Einschränkungen, einem Schlüsselwerkzeug in der angewandten
  Mathematik.
\end{itemize}

\subsubsection{Übungen}\label{uxfcbungen-31}

\begin{enumerate}
\def\labelenumi{\arabic{enumi}.}
\tightlist
\item
  Finden und klassifizieren Sie die kritischen Punkte von
  \(f(x,y) = x^2+xy+y^2\).
\item
  Klassifizieren Sie den Punkt (0,0) für \(f(x,y) = x^3-y^3\).3.
  Verwenden Sie den zweiten Ableitungstest für \(f(x,y) = x^4+y^4-4xy\).
\item
  Maximieren Sie \(f(x,y) = x+y\) abhängig von \(x^2+y^2=1\).
\item
  \(f(x,y) = x^2+2y^2\) vorbehaltlich \(x+y=1\) minimieren.
\end{enumerate}

\section{Kapitel 9. Mehrere
Integrale}\label{kapitel-9.-mehrere-integrale}

\subsection{9.1 Doppelte Integrale}\label{doppelte-integrale}

In der Einvariablenrechnung gibt ein bestimmtes Integral die Fläche
unter einer Kurve an. Bei zwei Variablen berechnet ein Doppelintegral
das Volumen unter einer Oberfläche (oder allgemeiner die Anhäufung von
Werten über eine Region).

\subsubsection{Definition}\label{definition-8}

Wenn \(f(x,y)\) auf einer Region \(R\) stetig ist, ist das
Doppelintegral

\[
\iint_R f(x,y)\, dA = \lim_{m,n \to \infty} \sum_{i=1}^m \sum_{j=1}^n f(x_{ij}^-, y_{ij}^-) \Delta A,
\]

wobei \(R\) in kleine Rechtecke mit der Fläche \(\Delta A\) unterteilt
ist.

\subsubsection{Iterierte Integrale}\label{iterierte-integrale}

Mit dem Satz von Fubini können wir ein Doppelintegral als iteriertes
Integral berechnen:

\[
\iint_R f(x,y)\, dA = \int_a^b \int_c^d f(x,y)\, dy\, dx,
\]

wenn \(R\) ein Rechteck ist \([a,b] \times [c,d]\).

Die Reihenfolge der Integration kann häufig geändert werden:

\[
\int_a^b \int_c^d f(x,y)\,dy\,dx = \int_c^d \int_a^b f(x,y)\,dx\,dy.
\]

\subsubsection{Beispiele}\label{beispiele-19}

\begin{enumerate}
\def\labelenumi{\arabic{enumi}.}
\tightlist
\item
  Rechteckbereich
\end{enumerate}

\[
\iint_R (x+y)\, dA, \quad R=[0,1]\times[0,2].
\]

\[
= \int_0^1 \int_0^2 (x+y)\,dy\,dx = \int_0^1 \Big[xy+\tfrac{1}{2}y^2\Big]_0^2 dx
= \int_0^1 (2x+2)dx = 3.
\]

\begin{enumerate}
\def\labelenumi{\arabic{enumi}.}
\setcounter{enumi}{1}
\tightlist
\item
  Dreieckiger Bereich
\end{enumerate}

\[
R = \{(x,y): 0 \leq x \leq 1, 0 \leq y \leq x\}.
\]

\[
\iint_R (x+y)\, dA = \int_0^1 \int_0^x (x+y)\,dy\,dx.
\]

Die Bewertung ergibt \(\tfrac{2}{3}\).

\subsubsection{Anwendungen}\label{anwendungen-1}

\begin{itemize}
\tightlist
\item
  Volumen unter einer Oberfläche:
\end{itemize}

\[
V = \iint_R f(x,y)\, dA.
\]

\begin{itemize}
\tightlist
\item
  Durchschnittswert einer Funktion über eine Region:
\end{itemize}

\[
f_{\text{avg}} = \frac{1}{A(R)} \iint_R f(x,y)\, dA.
\]

\subsubsection{Warum das wichtig ist}\label{warum-das-wichtig-ist-15}

Doppelintegrale erweitern die Integration auf zwei Dimensionen. Sie sind
in der Physik (Masse, Wahrscheinlichkeitsverteilungen), der Ökonomie
(erwartete Werte) und dem Ingenieurwesen (Schwerpunkte, Fluss) von
wesentlicher Bedeutung.

\subsubsection{Übungen}\label{uxfcbungen-32}

\begin{enumerate}
\def\labelenumi{\arabic{enumi}.}
\tightlist
\item
  Werten Sie \(\iint_R (x^2+y^2)\, dA\) mit \(R=[0,1]\times[0,1]\) aus.
\item
  Berechnen Sie \(\iint_R xy\, dA\) mit
  \(R=\{(x,y):0\leq x\leq2,0\leq y\leq x\}\).
\item
  Ermitteln Sie den Durchschnittswert von \(f(x,y) = x+y\) über dem
  Einheitsquadrat \([0,1]\times[0,1]\).
\item
  Interpretieren Sie \(\iint_R f(x,y)\, dA\) als Wahrscheinlichkeit,
  wenn \(f(x,y)\) eine Wahrscheinlichkeitsdichtefunktion ist.
\item
  Zeigen Sie, dass die Änderung der Integrationsreihenfolge für
  \(\iint_{[0,1]\times[0,2]} (x+y)\,dA\) das gleiche Ergebnis liefert.
\end{enumerate}

\subsection{9.2 Dreifache Integrale}\label{dreifache-integrale}

Dreifache Integrale erweitern die Idee der Integration auf drei
Variablen und ermöglichen uns die Berechnung von Volumina, Massen und
anderen Größen in dreidimensionalen Regionen.

\subsubsection{Definition}\label{definition-9}

Wenn \(f(x,y,z)\) auf einem festen Bereich \(E\) stetig ist, ist das
Dreifachintegral

\[\iiint_E f(x,y,z)\, dV = \lim_{m,n,p \to \infty} \sum f(x_{ijk}^-, y_{ijk}^-, z_{ijk}^-) \Delta V,
\]

where the region is subdivided into boxes of volume \(\Delta V\).

\subsubsection{Iterated Integrals}\label{iterated-integrals}

By Fubini's Theorem, a triple integral can be computed as an iterated
integral:

\[
\iiint_E f(x,y,z)\, dV = \int_a^b \int_c^d \int_e^f f(x,y,z)\, dz\, dy\, dx,
\]

for a rectangular box \(E = [a,b]\times[c,d]\times[e,f]\).

The order of integration can be chosen for convenience.

\subsubsection{Examples}\label{examples-3}

\begin{enumerate}
\def\labelenumi{\arabic{enumi}.}
\tightlist
\item
  Rectangular box
\end{enumerate}

\[
\iiint_E xyz\, dV, \quad E=[0,1]\times[0,2]\times[0,3].
\]

\[
= \int_0^1 \int_0^2 \int_0^3 xyz\,dz\,dy\,dx.
\]

First integrate over \(z\):

\[
\int_0^3 xyz\,dz = xy \left[\tfrac{1}{2}z^2\right]_0^3 = \tfrac{9}{2}xy.
\]

Now integrate over \(y\):

\[
\int_0^2 \tfrac{9}{2}xy\,dy = \tfrac{9}{2}x \cdot \left[\tfrac{1}{2}y^2\right]_0^2 = 9x.
\]

Finally integrate over \(x\):

\[
\int_0^1 9x\,dx = \tfrac{9}{2}.
\]

\begin{enumerate}
\def\labelenumi{\arabic{enumi}.}
\setcounter{enumi}{1}
\tightlist
\item
  Region bounded by planes Let
  \(E = \{(x,y,z) \mid 0 \leq x \leq 1, 0 \leq y \leq x, 0 \leq z \leq y\}\).
\end{enumerate}

\[
\iiint_E 1\,dV = \int_0^1 \int_0^x \int_0^y 1\,dz\,dy\,dx.
\]

Evaluate:

\[
= \int_0^1 \int_0^x y\,dy\,dx = \int_0^1 \tfrac{1}{2}x^2\,dx = \tfrac{1}{6}.
\]

So the volume of this triangular region is \(\tfrac{1}{6}\).

\subsubsection{Applications}\label{applications}

\begin{itemize}
\item
  Volume: \(V = \iiint_E 1 \, dV\).
\item
  Mass: If density is \(\rho(x,y,z)\), then

  \[
  M = \iiiint_E \rho(x,y,z)\, dV.
  \]
\item
  Average value:

  \[
  f_{\text{avg}} = \frac{1}{V(E)} \iiint_E f(x,y,z)\,dV.
  \]
\end{itemize}

\subsubsection{Why This Matters}\label{why-this-matters-2}

Triple integrals generalize area and volume calculations to arbitrary
solids. They are used in physics (mass distributions, center of mass,
gravitational fields), engineering, and probability.

\subsubsection{Exercises}\label{exercises-5}

\begin{enumerate}
\def\labelenumi{\arabic{enumi}.}
\tightlist
\item
  Compute \(\iiint_E (x+y+z)\,dV\) over the cube
  \(E=[0,1]\times[0,1]\times[0,1]\).
\item
  Find the volume of the tetrahedron bounded by
  \(x=0, y=0, z=0, x+y+z=1\).
\item
  Evaluate \(\iiint_E x^2 \, dV\) where
  \(E=[0,2]\times[0,1]\times[0,1]\).
\item
  Show that \(\iiint_E 1\,dV\) equals the geometric volume of \(E\).
\item
  If density is \(\rho(x,y,z)=x+y+z\), compute the mass of the unit
  cube.
\end{enumerate}

\subsection{9.3 Applications: Volume, Mass,
Probability}\label{applications-volume-mass-probability}

Triple integrals are powerful because they allow us to compute
quantities in three dimensions by accumulating values over a solid
region.

\subsubsection{Volume}\label{volume}

The simplest application is finding the volume of a region \(E\):

\[
V = \iiiint_E 1 \, dV.
\]

Example: Find the volume of the solid bounded by the coordinate planes
and the plane \(x+y+z=1\).

\[
V = \iiint_E 1 \, dV = \int_0^1 \int_0^{1-x} \int_0^{1-x-y} 1 \, dz\, dy\, dx.
\]

Die Bewertung ergibt \(V = \tfrac{1}{6}\).\#\#\# Masse und Dichte

Wenn ein Festkörper die Dichtefunktion \(\rho(x,y,z)\) hat, beträgt
seine Masse

\[
M = \iiint_E \rho(x,y,z)\, dV.
\]

Der Schwerpunkt ist gegeben durch

\[
\bar{x} = \frac{1}{M}\iiint_E x\rho(x,y,z)\,dV, \quad
\bar{y} = \frac{1}{M}\iiint_E y\rho(x,y,z)\,dV, \quad
\bar{z} = \frac{1}{M}\iiint_E z\rho(x,y,z)\,dV.
\]

Beispiel: Für einen Einheitswürfel mit konstanter Dichte \(\rho=1\)
liegt der Schwerpunkt bei \((0.5,0.5,0.5)\).

\subsubsection{Wahrscheinlichkeit}\label{wahrscheinlichkeit}

Wenn \(f(x,y,z)\) eine Wahrscheinlichkeitsdichtefunktion in 3D ist, dann
ist die Wahrscheinlichkeit, dass die Zufallsvariable in einem Bereich
liegt, \(E\)

\[
P(E) = \iiint_E f(x,y,z)\, dV,
\]

wo \(f(x,y,z) \geq 0\) und

\[
\iiint_{\mathbb{R}^3} f(x,y,z)\,dV = 1.
\]

Beispiel: Wenn \(f(x,y,z) = \tfrac{3}{4}z^2\) für \(0 \leq z \leq 1\),
einheitlich in \(x,y\), dann

\[
P(0 \leq z \leq 0.5) = \int_0^{0.5} \tfrac{3}{4}z^2 \, dz = \tfrac{1}{32}.
\]

\subsubsection{Warum das wichtig ist}\label{warum-das-wichtig-ist-16}

\begin{itemize}
\tightlist
\item
  Volumina verallgemeinern die Geometrie auf unregelmäßige Körper.
\item
  Massen- und Dichteintegrale verbinden Analysis mit Physik und
  Ingenieurwesen.
\item
  Wahrscheinlichkeitsdichtefunktionen in höheren Dimensionen werden in
  der Statistik und Datenwissenschaft häufig verwendet.
\end{itemize}

\subsubsection{Übungen}\label{uxfcbungen-33}

\begin{enumerate}
\def\labelenumi{\arabic{enumi}.}
\tightlist
\item
  Ermitteln Sie das Volumen des durch \(x^2+y^2+z^2 \leq 1\) (die
  Einheitskugel) begrenzten Festkörpers.
\item
  Berechnen Sie die Masse eines Kegels mit einer Dichte proportional zu
  \(z\).
\item
  Finden Sie den Massenschwerpunkt eines gleichförmigen Tetraeders, der
  durch \(x=0, y=0, z=0, x+y+z=1\) begrenzt wird.
\item
  Wenn \(f(x,y,z) = \frac{1}{8}\) auf dem Würfel
  \([0,2]\times[0,2]\times[0,2]\) ist, überprüfen Sie, ob es sich um
  eine Wahrscheinlichkeitsdichtefunktion handelt.
\item
  Verwenden Sie ein Dreifachintegral, um die Wahrscheinlichkeit zu
  berechnen, dass ein zufällig ausgewählter Punkt in der Einheitssphäre
  \(z > 0\) hat.
\end{enumerate}

\subsection{9.4 Änderung der Variablen: Polar-, Zylinder-,
Kugelkoordinaten}\label{uxe4nderung-der-variablen-polar--zylinder--kugelkoordinaten}

Viele Integrale werden einfacher, wenn sie in Koordinatensystemen
ausgedrückt werden, die der Symmetrie der Region entsprechen. Anstelle
der kartesischen Koordinaten \((x,y,z)\) können wir Polar-, Zylinder-
oder Kugelkoordinaten verwenden.

\subsubsection{Polarkoordinaten (2D)}\label{polarkoordinaten-2d}

Für Funktionen zweier Variablen können wir auf Polarkoordinaten
umsteigen:

\[
x = r\cos\theta, \quad y = r\sin\theta, \quad r \geq 0, \; 0 \leq \theta < 2\pi.
\]

Das Flächenelement transformiert sich als

\[
dA = r\,dr\,d\theta.
\]

Beispiel: Finden Sie die Fläche des Einheitskreises.

\[
A = \iint_{x^2+y^2\leq 1} 1\,dA = \int_0^{2\pi}\int_0^1 r\,dr\,d\theta = \pi.
\]

\subsubsection{Zylinderkoordinaten (3D)}\label{zylinderkoordinaten-3d}

In 3D erweitern Zylinderkoordinaten Polarkoordinaten um \(z\):

\[
x = r\cos\theta, \quad y = r\sin\theta, \quad z = z.
\]

Das Volumenelement ist

\[
dV = r\,dr\,d\theta\,dz.
\]

Beispiel: Volumen eines Zylinders mit Radius \(R\) und Höhe \(h\):

\[
V = \int_0^h \int_0^{2\pi} \int_0^R r\,dr\,d\theta\,dz = \pi R^2 h.
\]\#\#\# Sphärische Koordinaten (3D)

Für sphärische Symmetrie verwenden Sie:

\[
x = \rho \sin\phi \cos\theta, \quad y = \rho \sin\phi \sin\theta, \quad z = \rho \cos\phi,
\]

wo

\begin{itemize}
\tightlist
\item
  \(\rho \geq 0\) ist der Abstand vom Ursprung,
\item
  \(0 \leq \phi \leq \pi\) ist der Winkel von der positiven \(z\)-Achse,
\item
  \(0 \leq \theta < 2\pi\) ist der Winkel in der \(xy\)-Ebene.
\end{itemize}

Das Volumenelement ist

\[
dV = \rho^2 \sin\phi \, d\rho\, d\phi\, d\theta.
\]

Beispiel: Volumen der Einheitskugel:

\[
V = \int_0^{2\pi} \int_0^\pi \int_0^1 \rho^2 \sin\phi \, d\rho\, d\phi\, d\theta.
\]

Bewerten:

\[
V = \left(\int_0^1 \rho^2 d\rho\right)\left(\int_0^\pi \sin\phi d\phi\right)\left(\int_0^{2\pi} d\theta\right) = \tfrac{1}{3}(2)(2\pi) = \tfrac{4\pi}{3}.
\]

\subsubsection{Warum das wichtig ist}\label{warum-das-wichtig-ist-17}

\begin{itemize}
\tightlist
\item
  Polarkoordinaten vereinfachen kreisförmige Bereiche.
\item
  Zylinderkoordinaten behandeln Zylinder und Rotationssymmetrie.
\item
  Kugelkoordinaten vereinfachen Kugel-, Kegel- und Radialprobleme.
\item
  Diese Variablenänderungen machen ansonsten unmögliche Integrale
  beherrschbar.
\end{itemize}

\subsubsection{Übungen}\label{uxfcbungen-34}

\begin{enumerate}
\def\labelenumi{\arabic{enumi}.}
\tightlist
\item
  Berechnen Sie \(\iint_{x^2+y^2\leq 4} (x^2+y^2)\,dA\) mithilfe von
  Polarkoordinaten.
\item
  Ermitteln Sie mithilfe von Zylinderkoordinaten das Volumen eines
  Kegels mit der Höhe \(h\) und dem Radius \(R\).
\item
  Verwenden Sie Kugelkoordinaten, um das Volumen einer Kugel mit dem
  Radius \(R\) zu ermitteln.
\item
  Zeigen Sie, dass der Jacobi-Faktor für Polarkoordinaten \(r\) beträgt.
\item
  Ermitteln Sie mithilfe von Kugelkoordinaten die Masse einer festen
  Kugel mit einem Radius von \(R\) und einer Dichte, die proportional
  zum Abstand vom Ursprung ist.
\end{enumerate}

\section{Kapitel 10. Vektorrechnung}\label{kapitel-10.-vektorrechnung}

\subsection{10.1 Vektorfelder}\label{vektorfelder}

Ein Vektorfeld weist jedem Punkt im Raum einen Vektor zu, ähnlich wie
eine Skalarfunktion eine Zahl zuweist. Vektorfelder werden zur
Modellierung von Strömungen, Kräften und anderen Richtungsgrößen
verwendet.

\subsubsection{Definition}\label{definition-10}

In zwei Dimensionen ist ein Vektorfeld eine Funktion

\[
\mathbf{F}(x,y) = \langle P(x,y), Q(x,y) \rangle,
\]

wobei \(P\) und \(Q\) Skalarfunktionen sind.

In drei Dimensionen,

\[
\mathbf{F}(x,y,z) = \langle P(x,y,z), Q(x,y,z), R(x,y,z) \rangle.
\]

\subsubsection{Beispiele}\label{beispiele-20}

\begin{enumerate}
\def\labelenumi{\arabic{enumi}.}
\tightlist
\item
  Radiales Feld
\end{enumerate}

\[
\mathbf{F}(x,y) = \langle x, y \rangle.
\]

Vektoren zeigen vom Ursprung nach außen.

\begin{enumerate}
\def\labelenumi{\arabic{enumi}.}
\setcounter{enumi}{1}
\tightlist
\item
  Rotationsfeld
\end{enumerate}

\[
\mathbf{F}(x,y) = \langle -y, x \rangle.
\]

Vektoren zirkulieren um den Ursprung.

\begin{enumerate}
\def\labelenumi{\arabic{enumi}.}
\setcounter{enumi}{2}
\tightlist
\item
  Gravitationsfeld
\end{enumerate}

\[
\mathbf{F}(x,y,z) = -\frac{GM}{r^3}\langle x,y,z \rangle, \quad r=\sqrt{x^2+y^2+z^2}.
\]

\subsubsection{Vektorfelder
visualisieren}\label{vektorfelder-visualisieren}

\begin{itemize}
\tightlist
\item
  Zeichnen Sie kleine Pfeile an den Beispielpunkten, um Richtung und
  Größe anzuzeigen.
\item
  Dichtere Pfeile, bei denen die Beträge größer sind.
\item
  Nützlich für die Interpretation von Strömungslinien, Flugbahnen und
  Kräften.
\end{itemize}

\subsubsection{\texorpdfstring{FlusslinienEine Flusslinie (oder
Integralkurve) eines Vektorfeldes ist eine Kurve \(\mathbf{r}(t)\),
deren Tangentenvektor an jedem Punkt mit dem Feld
übereinstimmt:}{FlusslinienEine Flusslinie (oder Integralkurve) eines Vektorfeldes ist eine Kurve \textbackslash mathbf\{r\}(t), deren Tangentenvektor an jedem Punkt mit dem Feld übereinstimmt:}}\label{flusslinieneine-flusslinie-oder-integralkurve-eines-vektorfeldes-ist-eine-kurve-mathbfrt-deren-tangentenvektor-an-jedem-punkt-mit-dem-feld-uxfcbereinstimmt}

\[
\mathbf{r}'(t) = \mathbf{F}(\mathbf{r}(t)).
\]

Strömungslinien beschreiben Partikelpfade in einem Geschwindigkeitsfeld.

\subsubsection{Warum das wichtig ist}\label{warum-das-wichtig-ist-18}

\begin{itemize}
\tightlist
\item
  Vektorfelder sind in der Physik von grundlegender Bedeutung
  (Flüssigkeitsströmung, Elektromagnetismus, Gravitation).
\item
  Sie bilden die Grundlage für Linienintegrale, Flächenintegrale und die
  großen Sätze der Vektorrechnung (Green, Stokes, Divergenz).
\item
  Bereitstellung einer geometrischen Möglichkeit zur Darstellung von
  Richtungsgrößen.
\end{itemize}

\subsubsection{Übungen}\label{uxfcbungen-35}

\begin{enumerate}
\def\labelenumi{\arabic{enumi}.}
\tightlist
\item
  Skizzieren Sie das Vektorfeld
  \(\mathbf{F}(x,y) = \langle y, -x \rangle\).
\item
  Bestimmen Sie, ob die Vektoren von
  \(\mathbf{F}(x,y) = \langle x,y \rangle\) zum Ursprung hin oder von
  diesem weg zeigen.
\item
  Berechnen Sie für \(\mathbf{F}(x,y,z) = \langle y, z, x \rangle\)
  \(\mathbf{F}(1,2,3)\).
\item
  Beschreiben Sie die Flusslinien von
  \(\mathbf{F}(x,y) = \langle -y, x \rangle\).
\item
  Erklären Sie, warum Gravitations- und elektrische Felder Beispiele für
  radiale Vektorfelder sind.
\end{enumerate}

\subsection{10.2 Linienintegrale}\label{linienintegrale}

Ein Linienintegral erweitert die Idee eines Integrals auf Funktionen,
die entlang einer Kurve ausgewertet werden. Anstatt über ein Intervall
oder eine Region zu integrieren, integrieren wir über einen Weg im Raum.

\subsubsection{Definition:
Skalarlinienintegral}\label{definition-skalarlinienintegral}

Wenn \(f(x,y)\) eine Skalarfunktion und \(C\) eine durch
\(\mathbf{r}(t) = \langle x(t), y(t) \rangle, \; a \leq t \leq b\)
parametrisierte Kurve ist, dann ist das Linienintegral

\[
\int_C f(x,y)\, ds = \int_a^b f(x(t),y(t)) \, |\mathbf{r}'(t)|\, dt,
\]

wobei \(ds\) die Bogenlänge ist.

Dies misst die Anhäufung von \(f\) entlang der Kurve.

\subsubsection{Definition:
Vektorlinienintegral}\label{definition-vektorlinienintegral}

Für ein Vektorfeld \(\mathbf{F}(x,y) = \langle P(x,y), Q(x,y) \rangle\)
beträgt das Linienintegral entlang \(C\)

\[
\int_C \mathbf{F} \cdot d\mathbf{r} = \int_a^b \mathbf{F}(\mathbf{r}(t)) \cdot \mathbf{r}'(t)\, dt.
\]

Dies misst die vom Feld entlang der Kurve geleistete Arbeit.

\subsubsection{Beispiele}\label{beispiele-21}

\begin{enumerate}
\def\labelenumi{\arabic{enumi}.}
\tightlist
\item
  Skalarlinienintegral
\end{enumerate}

\[
f(x,y) = x+y, \quad C: \mathbf{r}(t) = \langle t, t^2 \rangle, \; 0 \leq t \leq 1.
\]

Dann

\[
\int_C f(x,y)\, ds = \int_0^1 (t+t^2)\sqrt{(1)^2+(2t)^2}\, dt.
\]

\begin{enumerate}
\def\labelenumi{\arabic{enumi}.}
\setcounter{enumi}{1}
\tightlist
\item
  Von einer Kraft geleistete Arbeit
\end{enumerate}

\[
\mathbf{F}(x,y) = \langle y, x \rangle, \quad C: \mathbf{r}(t) = \langle t, t^2 \rangle, \; 0 \leq t \leq 1.
\]

\[
\int_C \mathbf{F} \cdot d\mathbf{r} = \int_0^1 \langle t^2, t \rangle \cdot \langle 1, 2t \rangle\, dt = \int_0^1 (t^2 + 2t^2)\, dt = \int_0^1 3t^2\, dt = 1.
\]

\subsubsection{Physikalische
Interpretation}\label{physikalische-interpretation}

\begin{itemize}
\tightlist
\item
  Skalarlinienintegral: Dichteakkumulation entlang eines Drahtes.
\item
  Vektorlinienintegral: Arbeit, die von einer Kraft verrichtet wird, die
  ein Objekt entlang einer Bahn bewegt.
\end{itemize}

\subsubsection{Warum das wichtig ist- Linienintegrale verbinden
Vektorfelder mit physikalischen Größen wie Arbeit und
Zirkulation.}\label{warum-das-wichtig-ist--linienintegrale-verbinden-vektorfelder-mit-physikalischen-gruxf6uxdfen-wie-arbeit-und-zirkulation.}

\begin{itemize}
\tightlist
\item
  Sie sind Bausteine \hspace{0pt}\hspace{0pt}für den Satz von Green und
  den Satz von Stokes.
\item
  Erscheinen in der Physik (elektrisches Potential,
  Flüssigkeitsströmung, Mechanik).
\end{itemize}

\subsubsection{Übungen}\label{uxfcbungen-36}

\begin{enumerate}
\def\labelenumi{\arabic{enumi}.}
\tightlist
\item
  Berechnen Sie \(\int_C (x^2+y^2)\, ds\), wobei \(C\) das Liniensegment
  von (0,0) bis (1,1) ist.
\item
  Berechnen Sie \(\int_C \mathbf{F}\cdot d\mathbf{r}\) für
  \(\mathbf{F}(x,y) = \langle -y, x \rangle\) entlang des
  Einheitskreises \(x^2+y^2=1\).
\item
  Interpretieren Sie die Bedeutung von \(\int_C 1\,ds\).
\item
  Berechnen Sie für \(\mathbf{F}(x,y,z) = \langle z,0,x \rangle\) das
  Linienintegral entlang
  \(\mathbf{r}(t) = \langle t,t,1 \rangle, 0 \leq t \leq 1\).
\item
  Erklären Sie den Unterschied zwischen Skalar- und
  Vektorlinienintegralen.
\end{enumerate}

\subsection{10.3 Oberflächenintegrale}\label{oberfluxe4chenintegrale}

Ein Flächenintegral verallgemeinert Linienintegrale auf zweidimensionale
Flächen im dreidimensionalen Raum. Sie ermöglichen uns die Berechnung
des Flusses durch Oberflächen und der Akkumulation von Skalarfeldern
über gekrümmten Oberflächen.

\subsubsection{Skalares
Oberflächenintegral}\label{skalares-oberfluxe4chenintegral}

Wenn eine Fläche \(S\) parametrisiert wird durch

\[
\mathbf{r}(u,v) = \langle x(u,v), y(u,v), z(u,v) \rangle,
\]

dann beträgt das Flächenintegral einer Skalarfunktion \(f(x,y,z)\)

\[
\iint_S f(x,y,z)\, dS = \iint_D f(\mathbf{r}(u,v)) \, |\mathbf{r}_u \times \mathbf{r}_v| \, du\,dv,
\]

wobei \(\mathbf{r}_u\) und \(\mathbf{r}_v\) partielle Ableitungen von
\(\mathbf{r}(u,v)\) sind und \(D\) der Parameterbereich ist.

\subsubsection{Vektorflächenintegral
(Fluss)}\label{vektorfluxe4chenintegral-fluss}

Für ein Vektorfeld beträgt \(\mathbf{F}(x,y,z)\) der Fluss durch eine
Fläche \(S\)

\[
\iint_S \mathbf{F}\cdot d\mathbf{S} = \iint_S \mathbf{F}\cdot \mathbf{n}\, dS,
\]

wobei \(\mathbf{n}\) der Einheitsnormalenvektor ist. Mithilfe der
Parametrisierung

\[
\iint_S \mathbf{F}\cdot d\mathbf{S} = \iint_D \mathbf{F}(\mathbf{r}(u,v)) \cdot (\mathbf{r}_u \times \mathbf{r}_v)\,du\,dv.
\]

\subsubsection{Beispiele}\label{beispiele-22}

\begin{enumerate}
\def\labelenumi{\arabic{enumi}.}
\tightlist
\item
  Skalares Oberflächenintegral Oberfläche: Ebene \(z=1\) über
  Einheitsscheibe \(x^2+y^2 \leq 1\).
\end{enumerate}

\[
\iint_S 1\, dS = \text{area of the disk} = \pi.
\]

\begin{enumerate}
\def\labelenumi{\arabic{enumi}.}
\setcounter{enumi}{1}
\tightlist
\item
  Fluss durch eine Kugel Sei
  \(\mathbf{F}(x,y,z) = \langle x,y,z \rangle\) und \(S\) = Kugel mit
  dem Radius \(R\). Der Normalvektor beträgt
  \(\mathbf{n} = \frac{1}{R}\langle x,y,z \rangle\).
\end{enumerate}

\[
\mathbf{F}\cdot \mathbf{n} = \frac{x^2+y^2+z^2}{R} = R.
\]

Also

\[
\iint_S \mathbf{F}\cdot d\mathbf{S} = \iint_S R\, dS = R \cdot 4\pi R^2 = 4\pi R^3.
\]

\subsubsection{Warum das wichtig ist}\label{warum-das-wichtig-ist-19}

\begin{itemize}
\tightlist
\item
  Skalare Oberflächenintegrale messen Flächen- und
  Oberflächenverteilungen.
\item
  Vektoroberflächenintegrale messen den Fluss: die Stärke eines Feldes,
  das durch eine Oberfläche geht.
\item
  Anwendungen: Elektromagnetismus, Flüssigkeitsströmung,
  Wärmeübertragung und mehr.
\end{itemize}

\subsubsection{Übungen}\label{uxfcbungen-37}

\begin{enumerate}
\def\labelenumi{\arabic{enumi}.}
\tightlist
\item
  Berechnen Sie \(\iint_S 1\, dS\) für die Oberfläche eines Würfels mit
  der Seitenlänge 2.2. Finden Sie den Fluss von
  \(\mathbf{F}(x,y,z) = \langle x,y,z \rangle\) durch die Einheitskugel.
\item
  Berechnen Sie \(\iint_S z\, dS\) für das Paraboloid
  \(z = x^2+y^2, \, z \leq 1\).
\item
  Berechnen Sie für \(\mathbf{F}(x,y,z) = \langle y,0,0 \rangle\) den
  Fluss durch die Ebene \(x=1\), \(0 \leq y,z \leq 1\).
\item
  Erklären Sie physikalisch, was es bedeutet, wenn der Fluss eines
  Vektorfeldes durch eine geschlossene Oberfläche Null ist.
\end{enumerate}

\subsection{10.4 Satz von Green}\label{satz-von-green}

Der Satz von Green ist ein grundlegendes Ergebnis der Vektorrechnung,
das ein Linienintegral um eine geschlossene Kurve mit einem
Doppelintegral über den von ihr umschlossenen Bereich verbindet. Es
handelt sich um eine zweidimensionale Version des Satzes von Stokes.

\subsubsection{Aussage des Satzes von
Green}\label{aussage-des-satzes-von-green}

Sei \(C\) eine positiv orientierte, einfache, geschlossene Kurve in der
Ebene und sei \(R\) der von ihr umschlossene Bereich. Wenn
\(\mathbf{F}(x,y) = \langle P(x,y), Q(x,y) \rangle\) kontinuierliche
partielle Ableitungen auf einen offenen Bereich hat, der \(R\) enthält,
dann

\[
\oint_C \mathbf{F} \cdot d\mathbf{r} = \oint_C P\,dx + Q\,dy = \iint_R \left( \frac{\partial Q}{\partial x} - \frac{\partial P}{\partial y} \right)\, dA.
\]

\subsubsection{Interpretation}\label{interpretation-2}

\begin{itemize}
\tightlist
\item
  Das Linienintegral um \(C\) misst die Zirkulation des Vektorfeldes
  entlang der Grenze.
\item
  Das Doppelintegral über \(R\) misst die Gesamtkrümmung (Rotation) des
  Feldes innerhalb der Region.
\end{itemize}

\subsubsection{Beispiel 1:
Flächenformel}\label{beispiel-1-fluxe4chenformel}

Wenn \(\mathbf{F} = \langle -y/2, x/2 \rangle\), dann

\[
\frac{\partial Q}{\partial x} - \frac{\partial P}{\partial y} = 1.
\]

Somit ergibt sich aus dem Satz von Green

\[
\text{Area}(R) = \iint_R 1\,dA = \oint_C \left(-\tfrac{y}{2}\,dx + \tfrac{x}{2}\,dy\right).
\]

Dies bietet eine Möglichkeit, die Fläche mithilfe eines Linienintegrals
zu berechnen.

\subsubsection{Beispiel 2: Zirkulation}\label{beispiel-2-zirkulation}

Sei \(\mathbf{F}(x,y) = \langle -y, x \rangle\) und \(C\) der
Einheitskreis.

\begin{itemize}
\tightlist
\item
  \(P=-y, Q=x\).
\item
  \(Q_x - P_y = 1 - (-1) = 2\).
\item
  Doppeltes Integral über die Einheitsscheibe:
\end{itemize}

\[
\iint_R 2\,dA = 2\pi (1^2) = 2\pi.
\]

Die Zirkulation um den Kreis beträgt also \(2\pi\).

\subsubsection{Warum das wichtig ist}\label{warum-das-wichtig-ist-20}

\begin{itemize}
\tightlist
\item
  Konvertiert schwierige Linienintegrale in Doppelintegrale oder
  umgekehrt. -- Bietet eine Brücke zwischen lokalen Eigenschaften (Curl)
  und globalen Eigenschaften (Zirkulation).
\item
  Weit verbreitet in der Physik für Flüssigkeitsströmungen,
  Elektromagnetismus und planare Vektorfelder.
\end{itemize}

\subsubsection{Übungen}\label{uxfcbungen-38}

\begin{enumerate}
\def\labelenumi{\arabic{enumi}.}
\tightlist
\item
  Verwenden Sie den Satz von Green, um die Fläche innerhalb der Ellipse
  \(\frac{x^2}{a^2} + \frac{y^2}{b^2} = 1\) zu berechnen.
\item
  Überprüfen Sie den Satz von Green für
  \(\mathbf{F}(x,y) = \langle -y, x \rangle\) entlang des Quadrats mit
  den Eckpunkten (0,0), (1,0), (1,1), (0,1).
\item
  Berechnen Sie die Zirkulation von
  \(\mathbf{F}(x,y) = \langle -y, x \rangle\) um den Einheitskreis.4.
  Zeigen Sie, dass bei \(\nabla \times \mathbf{F} = 0\) das
  Linienintegral von \(\mathbf{F}\) um jede geschlossene Kurve Null ist.
\item
  Verwenden Sie den Satz von Green, um das zu zeigen
\end{enumerate}

\[
\oint_C x\,dy = -\oint_C y\,dx
\]

für jede geschlossene Kurve \(C\).

\subsection{10.5 Satz von Stokes}\label{satz-von-stokes}

Der Satz von Stokes verallgemeinert den Satz von Green auf drei
Dimensionen. Es setzt ein Oberflächenintegral der Krümmung eines
Vektorfeldes über einer Oberfläche in Beziehung zu einem Linienintegral
des Vektorfeldes um die Grenze dieser Oberfläche.

\subsubsection{Aussage zum Satz von
Stokes}\label{aussage-zum-satz-von-stokes}

Sei \(S\) eine orientierte, glatte Fläche mit Randkurve \(C\) (positiv
orientiert). Wenn \(\mathbf{F}(x,y,z)\) ein Vektorfeld mit stetigen
partiellen Ableitungen ist, dann

\[
\iint_S (\nabla \times \mathbf{F}) \cdot d\mathbf{S} = \oint_C \mathbf{F} \cdot d\mathbf{r}.
\]

\begin{itemize}
\tightlist
\item
  Linke Seite: Fluss der Locke von \(\mathbf{F}\) durch die Oberfläche.
\item
  Rechte Seite: Zirkulation von \(\mathbf{F}\) entlang der Randkurve.
\end{itemize}

\subsubsection{Interpretation}\label{interpretation-3}

\begin{itemize}
\tightlist
\item
  Das Linienintegral um den Rand entspricht der gesamten „Rotation``
  innerhalb der Oberfläche.
\item
  Erweitert den Satz von Green (ein Sonderfall, wenn die Oberfläche in
  der Ebene liegt).
\end{itemize}

\subsubsection{Beispiel 1: Theorem von Green als
Sonderfall}\label{beispiel-1-theorem-von-green-als-sonderfall}

Wenn \(S\) eine flache Region in der \(xy\)-Ebene ist, reduziert sich
der Satz von Stokes auf den Satz von Green.

\subsubsection{Beispiel 2: Zirkulation auf einer
Hemisphäre}\label{beispiel-2-zirkulation-auf-einer-hemisphuxe4re}

Sei \(\mathbf{F}(x,y,z) = \langle -y, x, 0 \rangle\) und \(S\) die obere
Halbkugel mit Radius 1.

\begin{itemize}
\tightlist
\item
  Rand \(C\): Einheitskreis in der \(xy\)-Ebene.
\item
  Nach dem Satz von Stokes:
\end{itemize}

\[
\oint_C \mathbf{F}\cdot d\mathbf{r} = \iint_S (\nabla \times \mathbf{F})\cdot d\mathbf{S}.
\]

\begin{itemize}
\tightlist
\item
  Locken: \(\nabla \times \mathbf{F} = \langle 0,0,2 \rangle\).
\item
  Normal zur Halbkugel zeigt nach außen:
  \(\mathbf{n} = \langle 0,0,1 \rangle\).
\item
  Also Integrand = 2.
\item
  Fläche der Hemisphäre = \(2\pi (1^2)\).
\end{itemize}

\[
\iint_S 2\, dS = 2 \cdot 2\pi = 4\pi.
\]

Somit beträgt die Zirkulation um den Äquator \(4\pi\).

\subsubsection{Warum das wichtig ist}\label{warum-das-wichtig-ist-21}

\begin{itemize}
\tightlist
\item
  Bietet eine tiefe Verbindung zwischen Oberflächenintegralen und
  Linienintegralen.
\item
  Vereinfacht Berechnungen durch die Auswahl geeigneter Oberflächen.
\item
  Weit verbreitet im Elektromagnetismus (Faradaysches Gesetz) und in der
  Fluiddynamik.
\end{itemize}

\subsubsection{Übungen}\label{uxfcbungen-39}

\begin{enumerate}
\def\labelenumi{\arabic{enumi}.}
\tightlist
\item
  Überprüfen Sie den Satz von Stokes für
  \(\mathbf{F}(x,y,z) = \langle y, -x, 0 \rangle\) über der
  Einheitsscheibe in der \(xy\)-Ebene.
\item
  Berechnen Sie \(\oint_C \mathbf{F}\cdot d\mathbf{r}\), wobei
  \(\mathbf{F}(x,y,z) = \langle z, 0, x \rangle\) und \(C\) die Grenze
  des Dreiecks mit den Eckpunkten (0,0,0), (1,0,0), (0,1,0) ist.
\item
  Zeigen Sie, dass bei \(\nabla \times \mathbf{F} = 0\) die Zirkulation
  um jede geschlossene Kurve Null ist.4. Wenden Sie den Satz von Stokes
  an, um die Zirkulation von
  \(\mathbf{F}(x,y,z) = \langle -y, x, z \rangle\) um die Grenze des
  Einheitsquadrats in der Ebene \(z=0\) zu berechnen.
\item
  Erklären Sie, wie der Satz von Stokes den Satz von Green
  verallgemeinert.
\end{enumerate}

\subsection{10.6 Divergenzsatz}\label{divergenzsatz}

Der Divergenzsatz (auch Gauß-Satz genannt) setzt den Fluss eines
Vektorfeldes durch eine geschlossene Oberfläche in Beziehung zum
Dreifachintegral der Divergenz des Feldes innerhalb der Oberfläche.

\subsubsection{Aussage des
Divergenzsatzes}\label{aussage-des-divergenzsatzes}

Sei \(E\) ein fester Bereich in \(\mathbb{R}^3\) mit der Grenzfläche
\(S\) (nach außen gerichtet). Wenn \(\mathbf{F}(x,y,z)\) ein Vektorfeld
mit stetigen partiellen Ableitungen auf \(E\) ist, dann

\[
\iint_S \mathbf{F} \cdot d\mathbf{S} = \iiint_E (\nabla \cdot \mathbf{F}) \, dV.
\]

\begin{itemize}
\tightlist
\item
  Linke Seite: Fluss von \(\mathbf{F}\) über die geschlossene Fläche
  \(S\).
\item
  Rechte Seite: Dreifaches Integral der Divergenz innerhalb der Region.
\end{itemize}

\subsubsection{Divergenz}\label{divergenz}

Die Divergenz eines Vektorfeldes beträgt
\(\mathbf{F}(x,y,z) = \langle P, Q, R \rangle\)

\[
\nabla \cdot \mathbf{F} = \frac{\partial P}{\partial x} + \frac{\partial Q}{\partial y} + \frac{\partial R}{\partial z}.
\]

Es misst den „Nettoabfluss`` pro Volumeneinheit an jedem Punkt.

\subsubsection{Beispiel 1: Fluss eines radialen
Feldes}\label{beispiel-1-fluss-eines-radialen-feldes}

Sei \(\mathbf{F}(x,y,z) = \langle x, y, z \rangle\) und sei \(E\) die
Einheitskugel \(x^2+y^2+z^2 \leq 1\).

\begin{itemize}
\tightlist
\item
  Abweichung: \(\nabla \cdot \mathbf{F} = 1+1+1 = 3\).
\item
  Volumen der Einheitskugel: \(\tfrac{4}{3}\pi\). Also
\end{itemize}

\[
\iiint_E (\nabla \cdot \mathbf{F})\, dV = 3 \cdot \tfrac{4}{3}\pi = 4\pi.
\]

Somit beträgt der Fluss über die Einheitssphäre \(4\pi\).

\subsubsection{Beispiel 2:
Konstantenfeld}\label{beispiel-2-konstantenfeld}

Lassen Sie \(\mathbf{F}(x,y,z) = \langle 1, 0, 0 \rangle\).

\begin{itemize}
\tightlist
\item
  Abweichung: \(\nabla \cdot \mathbf{F} = 0\).
\item
  Der Fluss durch jede geschlossene Oberfläche ist also Null, was mit
  der Intuition übereinstimmt (kein Nettoabfluss).
\end{itemize}

\subsubsection{Warum das wichtig ist}\label{warum-das-wichtig-ist-22}

\begin{itemize}
\item
  Konvertiert Oberflächenintegrale in einfachere Volumenintegrale.
\item
  In der Physik verwendet: Gaußsches Gesetz für Elektromagnetismus,
  Flüssigkeitsströmung und Wärmeübertragung.
\item
  Vervollständigt den einheitlichen Rahmen:

  \begin{itemize}
  \tightlist
  \item
    Satz von Green (2D-Curl ↔ Zirkulation)
  \item
    Satz von Stokes (3D-Curl ↔ Zirkulation auf Oberflächen)
  \item
    Divergenzsatz (3D-Divergenz ↔ Fluss auf geschlossenen Flächen)
  \end{itemize}
\end{itemize}

\subsubsection{Übungen}\label{uxfcbungen-40}

\begin{enumerate}
\def\labelenumi{\arabic{enumi}.}
\tightlist
\item
  Verwenden Sie den Divergenzsatz, um den Fluss von
  \(\mathbf{F}(x,y,z) = \langle x,y,z \rangle\) über die Oberfläche
  einer Kugel mit dem Radius \(R\) zu berechnen.
\item
  Überprüfen Sie den Divergenzsatz für
  \(\mathbf{F}(x,y,z) = \langle y, z, x \rangle\) auf dem Einheitswürfel
  \([0,1]^3\).
\item
  Zeigen Sie, dass bei \(\nabla \cdot \mathbf{F} = 0\) der Gesamtfluss
  durch jede geschlossene Oberfläche Null ist.
\item
  Berechnen Sie den Fluss von
  \(\mathbf{F}(x,y,z) = \langle x^2, y^2, z^2 \rangle\) durch die
  Einheitskugel.5. Erklären Sie, wie der Divergenzsatz den
  eindimensionalen Fundamentalsatz der Analysis verallgemeinert.
\end{enumerate}

\section{Teil IV. Unendliche
Prozesse}\label{teil-iv.-unendliche-prozesse}

\section{Kapitel 11. Folgen und
Konvergenz}\label{kapitel-11.-folgen-und-konvergenz}

\subsection{11.1 Definitionen und
Beispiele}\label{definitionen-und-beispiele}

Eine Folge ist eine geordnete Liste von Zahlen, die normalerweise als
geschrieben wird

\[
a_1, a_2, a_3, \dots
\]

oder allgemeiner \((a_n)_{n=1}^\infty\). Jedes \(a_n\) wird als n-tes
Glied der Folge bezeichnet.

\subsubsection{Definieren einer Sequenz}\label{definieren-einer-sequenz}

Eine Sequenz kann auf zwei Arten definiert werden:

\begin{enumerate}
\def\labelenumi{\arabic{enumi}.}
\item
  Explizite Formel -- gibt eine direkte Regel für den n-ten Term an.

  \begin{itemize}
  \item
    Beispiel: \(a_n = \frac{1}{n}\) definiert die Reihenfolge

    \[
    1, \tfrac{1}{2}, \tfrac{1}{3}, \tfrac{1}{4}, \dots
    \]
  \end{itemize}
\item
  Rekursive Definition -- definiert Begriffe unter Verwendung früherer
  Begriffe.

  \begin{itemize}
  \item
    Beispiel: Fibonacci-Folge:

    \[
    a_1 = 1, \quad a_2 = 1, \quad a_{n} = a_{n-1} + a_{n-2} \quad (n \geq 3).
    \]
  \end{itemize}
\end{enumerate}

\subsubsection{Beispiele für
Sequenzen}\label{beispiele-fuxfcr-sequenzen}

\begin{enumerate}
\def\labelenumi{\arabic{enumi}.}
\item
  Arithmetische Folge:

  \[
  a_n = a_1 + (n-1)d.
  \]

  Beispiel: \(a_n = 2n+1\) → Folge ungerader Zahlen.
\item
  Geometrische Reihenfolge:

  \[
  a_n = a_1 r^{n-1}.
  \]

  Beispiel: \(a_n = 2^n\) → Zweierpotenzen.
\item
  Harmonische Folge:

  \[
  a_n = \frac{1}{n}.
  \]
\item
  Abwechselnde Reihenfolge:

  \[
  a_n = (-1)^n.
  \]
\end{enumerate}

\subsubsection{Folgen in der Analysis}\label{folgen-in-der-analysis}

Sequenzen sind die Grundlage für unendliche Prozesse:

\begin{itemize}
\tightlist
\item
  Grenzen von Folgen → Konvergenz definieren.
\item
  Reihen → unendliche Summen, die aus Folgen gebildet werden.
\item
  Durch Folgen und Reihen angenäherte Funktionen.
\end{itemize}

\subsubsection{Warum das wichtig ist}\label{warum-das-wichtig-ist-23}

\begin{itemize}
\tightlist
\item
  Folgen liefern die Bausteine für unendliche Reihen und Näherungen.
\item
  Sie ermöglichen es uns, „Annäherung an die Unendlichkeit`` und
  Konvergenz genau zu definieren.
\item
  Viele wichtige Funktionen (exponentiell, trigonometrisch) können durch
  Folgen und Reihen ausgedrückt werden.
\end{itemize}

\subsubsection{Übungen}\label{uxfcbungen-41}

\begin{enumerate}
\def\labelenumi{\arabic{enumi}.}
\tightlist
\item
  Schreiben Sie die ersten fünf Terme der Sequenz
  \(a_n = \frac{n}{n+1}\).
\item
  Bestimmen Sie, ob \(a_n = (-1)^n n\) beschränkt ist.
\item
  Geben Sie eine rekursive Definition für die Sequenz \(2,4,8,16,\dots\)
  an.
\item
  Finden Sie das 10. Glied der arithmetischen Folge mit \(a_1=3\) und
  \(d=5\).
\item
  Schreiben Sie eine explizite Formel für die durch \(a_1=1\),
  \(a_{n+1}=2a_n\) definierte Sequenz.
\end{enumerate}

\subsection{11.2 Monotone und begrenzte
Folgen}\label{monotone-und-begrenzte-folgen}

Um zu verstehen, ob eine Folge konvergiert, müssen wir ihr Verhalten
untersuchen: Nimmt sie zu, ab, bleibt sie innerhalb der Grenzen oder
wächst sie unbegrenzt? Zwei wichtige Konzepte sind Monotonie und
Beschränktheit.

\subsubsection{Monotone Sequenzen}\label{monotone-sequenzen}

Eine Folge \((a_n)\) heißt monoton, wenn sie immer steigend oder immer
fallend ist.

\begin{itemize}
\item
  Monoton ansteigend:

  \[
  a_{n+1} \geq a_n \quad \text{for all } n.
  \]
\item
  Monoton abnehmend:

  \[
  a_{n+1} \leq a_n \quad \text{for all } n.
  \]
\end{itemize}

Beispiele:1. \(a_n = n\) ist monoton steigend. 2. \(a_n = \frac{1}{n}\)
ist monoton abnehmend.

\subsubsection{Begrenzte Sequenzen}\label{begrenzte-sequenzen}

Eine Folge ist nach oben beschränkt, wenn es eine Zahl \(M\) gibt, so
dass \(a_n \leq M\) für alle \(n\) gilt. Es ist nach unten beschränkt,
wenn es \(m\) gibt, so dass \(a_n \geq m\) für alle \(n\) gilt.

Wenn beide Bedingungen zutreffen, ist die Folge beschränkt.

Beispiele:

\begin{enumerate}
\def\labelenumi{\arabic{enumi}.}
\tightlist
\item
  \(a_n = \frac{1}{n}\) ist zwischen 0 und 1 begrenzt.
\item
  \(a_n = (-1)^n\) liegt zwischen -1 und 1.
\item
  \(a_n = n\) ist nicht begrenzt.
\end{enumerate}

\subsubsection{Satz der monotonen
Konvergenz}\label{satz-der-monotonen-konvergenz}

Ein grundlegendes Ergebnis der Analyse:

\begin{itemize}
\tightlist
\item
  Jede nach oben beschränkte monoton wachsende Folge konvergiert.
\item
  Jede nach unten beschränkte monoton fallende Folge konvergiert.
\end{itemize}

Dieser Satz garantiert Konvergenz, ohne den Grenzwert explizit zu
finden.

\subsubsection{Beispiel}\label{beispiel-1}

\begin{enumerate}
\def\labelenumi{\arabic{enumi}.}
\item
  Reihenfolge: \(a_n = 1 - \frac{1}{n}\).

  \begin{itemize}
  \tightlist
  \item
    Steigend: seit \(a_{n+1} - a_n = \frac{1}{n} - \frac{1}{n+1} > 0\).
  \item
    Oben durch 1 begrenzt.
  \item
    Daher konvergiert es.
  \item
    Limit: \(\lim_{n\to\infty} a_n = 1\).
  \end{itemize}
\end{enumerate}

\subsubsection{Warum das wichtig ist}\label{warum-das-wichtig-ist-24}

\begin{itemize}
\tightlist
\item
  Monotonie und Beschränktheit ermöglichen schnelle Konvergenztests.
\item
  Sie sind für Beweise und die rigorose Konstruktion von Grenzwerten von
  wesentlicher Bedeutung.
\item
  Diese Ideen erstrecken sich natürlich auf Funktionen und Serien.
\end{itemize}

\subsubsection{Übungen}\label{uxfcbungen-42}

\begin{enumerate}
\def\labelenumi{\arabic{enumi}.}
\tightlist
\item
  Bestimmen Sie, ob \(a_n = \frac{n}{n+1}\) monoton und beschränkt ist.
\item
  Zeigen Sie, dass \(a_n = \sqrt{n}\) monoton wachsend, aber nicht
  beschränkt ist.
\item
  Beweisen Sie, dass \(a_n = 2 - \frac{1}{n}\) konvergiert, und finden
  Sie seinen Grenzwert.
\item
  Geben Sie ein Beispiel für eine beschränkte Folge an, die nicht
  monoton ist.
\item
  Wenden Sie den Satz der monotonen Konvergenz auf
  \(a_n = \ln\!\big(1+\frac{1}{n}\big)\) an.
\end{enumerate}

\subsection{11.3 Grenzen von Sequenzen}\label{grenzen-von-sequenzen}

Die zentrale Frage einer Sequenz ist, ob sich ihre Terme einem einzigen
Wert nähern, wenn \(n\) wächst. Dies führt zum Konzept des Grenzwertes
einer Folge.

\subsubsection{Definition}\label{definition-11}

Eine Folge \((a_n)\) hat einen Grenzwert \(L\), wenn für jedes
\(\varepsilon > 0\) eine ganze Zahl \(N\) existiert, so dass

\[
|a_n - L| < \varepsilon \quad \text{whenever } n > N.
\]

Wir schreiben dann

\[
\lim_{n\to\infty} a_n = L.
\]

Existiert kein solches \(L\), divergiert die Reihenfolge.

\subsubsection{Intuition}\label{intuition}

\begin{itemize}
\tightlist
\item
  Die Terme der Folge nähern sich willkürlich \(L\) an, wenn \(n\) groß
  wird.
\item
  Über einen Index von \(N\) hinaus bleiben alle Begriffe innerhalb
  einer winzigen Bandbreite um \(L\).
\end{itemize}

\subsubsection{Beispiele}\label{beispiele-23}

\begin{enumerate}
\def\labelenumi{\arabic{enumi}.}
\item
  \(a_n = \frac{1}{n}\). Wenn \(n\) wächst, schrumpfen die Terme in
  Richtung 0.

  \[
  \lim_{n\to\infty} \frac{1}{n} = 0.
  \]
\item
  \(a_n = (-1)^n\). Die Terme wechseln zwischen -1 und 1, daher gibt es
  keinen einheitlichen Grenzwert. Die Reihenfolge divergiert.
\item
  \(a_n = \frac{n}{n+1}\). Da \(n \to \infty\), sind Zähler und Nenner
  nahezu gleich, also

  \[
  \lim_{n\to\infty} \frac{n}{n+1} = 1.
  \]
\end{enumerate}

\subsubsection{\texorpdfstring{Eigenschaften von GrenzwertenWenn
\(\lim a_n = A\) und
\(\lim b_n = B\):}{Eigenschaften von GrenzwertenWenn \textbackslash lim a\_n = A und \textbackslash lim b\_n = B:}}\label{eigenschaften-von-grenzwertenwenn-lim-a_n-a-und-lim-b_n-b}

\begin{itemize}
\item
  \(\lim (a_n+b_n) = A+B\).
\item
  \(\lim (a_n b_n) = AB\).
\item
  \(\lim (c a_n) = cA\) für konstant \(c\).
\item
  Wenn \(b_n \neq 0\) und \(B \neq 0\), dann

  \[
  \lim \frac{a_n}{b_n} = \frac{A}{B}.
  \]
\end{itemize}

\subsubsection{Satz: Squeeze-Prinzip}\label{satz-squeeze-prinzip}

Wenn \(a_n \leq b_n \leq c_n\) für alle großen \(n\), und

\[
\lim_{n\to\infty} a_n = \lim_{n\to\infty} c_n = L,
\]

dann

\[
\lim_{n\to\infty} b_n = L.
\]

Beispiel:

\[
a_n = -\tfrac{1}{n}, \quad b_n = \tfrac{\sin n}{n}, \quad c_n = \tfrac{1}{n}.
\]

Da \(-\tfrac{1}{n} \leq \tfrac{\sin n}{n} \leq \tfrac{1}{n}\) und beide
Begrenzungsfolgen gegen 0 gehen,

\[
\lim_{n\to\infty} \frac{\sin n}{n} = 0.
\]

\subsubsection{Warum das wichtig ist}\label{warum-das-wichtig-ist-25}

\begin{itemize}
\tightlist
\item
  Grenzwerte verschärfen die Vorstellung, dass sich Sequenzen einem Wert
  „annähern``.
\item
  Konvergenz von Folgen untermauert unendliche Reihen und Kontinuität.
\item
  Diese Konzepte sind für die Definition reeller Zahlen über Grenzwerte
  von wesentlicher Bedeutung.
\end{itemize}

\subsubsection{Übungen}\label{uxfcbungen-43}

\begin{enumerate}
\def\labelenumi{\arabic{enumi}.}
\tightlist
\item
  Finden Sie \(\lim_{n\to\infty} \frac{2n+1}{3n+4}\).
\item
  Bestimmen Sie, ob \(a_n = \sqrt{n+1} - \sqrt{n}\) konvergiert.
\item
  Konvergiert \(a_n = \cos n\)? Warum oder warum nicht?
\item
  Verwenden Sie das Squeeze-Prinzip, um
  \(\lim_{n\to\infty} \frac{\sin n}{n} = 0\) anzuzeigen.
\item
  Beweisen Sie, dass wenn \(\lim a_n = L\), dann \(\lim |a_n| = |L|\).
\end{enumerate}

\section{Kapitel 12. Unendliche
Reihe}\label{kapitel-12.-unendliche-reihe}

\subsection{12.1 Reihen und Konvergenz}\label{reihen-und-konvergenz}

Eine Reihe ist die Summe der Glieder einer Folge. Anstatt nur Zahlen
aufzuzählen, addieren wir sie und untersuchen, ob sich die unendliche
Summe einem endlichen Wert annähert.

\subsubsection{Definition}\label{definition-12}

Bei einer gegebenen Sequenz \((a_n)\) ist die entsprechende Serie

\[
\sum_{n=1}^\infty a_n = a_1 + a_2 + a_3 + \dots
\]

Wir definieren die n-te Teilsumme als

\[
S_n = \sum_{k=1}^n a_k.
\]

Wenn die Folge \((S_n)\) gegen einen endlichen Grenzwert \(S\)
konvergiert, dann konvergiert die Reihe und

\[
\sum_{n=1}^\infty a_n = S.
\]

Wenn \((S_n)\) divergiert, dann divergiert die Reihe.

\subsubsection{Beispiele}\label{beispiele-24}

\begin{enumerate}
\def\labelenumi{\arabic{enumi}.}
\tightlist
\item
  Geometrische Reihe
\end{enumerate}

\[
\sum_{n=0}^\infty ar^n = \frac{a}{1-r}, \quad |r| < 1.
\]

Beispiel:

\[
1 + \tfrac{1}{2} + \tfrac{1}{4} + \tfrac{1}{8} + \dots = 2.
\]

\begin{enumerate}
\def\labelenumi{\arabic{enumi}.}
\setcounter{enumi}{1}
\tightlist
\item
  Harmonische Reihe
\end{enumerate}

\[
\sum_{n=1}^\infty \frac{1}{n}.
\]

Diese Reihe divergiert, auch wenn die Terme gegen 0 gehen.

\begin{enumerate}
\def\labelenumi{\arabic{enumi}.}
\setcounter{enumi}{2}
\tightlist
\item
  p-Reihe
\end{enumerate}

\[
\sum_{n=1}^\infty \frac{1}{n^p}.
\]

\begin{itemize}
\tightlist
\item
  Konvergiert, wenn \(p > 1\).
\item
  Divergiert, wenn \(p \leq 1\).
\end{itemize}

\subsubsection{Notwendige Bedingung für
Konvergenz}\label{notwendige-bedingung-fuxfcr-konvergenz}

Wenn \(\sum a_n\) konvergiert, dann unbedingt

\[
\lim_{n\to\infty} a_n = 0.
\]

Bei \(\lim a_n \neq 0\) divergiert die Reihe. Aber das Gegenteil ist
nicht der Fall: \(\lim a_n = 0\) garantiert keine Konvergenz (z. B.
harmonische Reihen).

\subsubsection{Warum das wichtig ist}\label{warum-das-wichtig-ist-26}

\begin{itemize}
\tightlist
\item
  Reihen erweitern die endliche Addition auf unendliche Prozesse.
\item
  Konvergente Reihen werden verwendet, um Funktionen anzunähern, Flächen
  zu berechnen und physikalische Prozesse zu modellieren.- Die
  Untersuchung von Reihen führt zu leistungsstarken Konvergenztests.
\end{itemize}

\subsubsection{Übungen}\label{uxfcbungen-44}

\begin{enumerate}
\def\labelenumi{\arabic{enumi}.}
\tightlist
\item
  Bestimmen Sie, ob \(\sum_{n=1}^\infty \frac{2}{3^n}\) konvergiert, und
  ermitteln Sie seine Summe.
\item
  Zeigen Sie, dass \(\sum_{n=1}^\infty \frac{1}{n^2}\) konvergiert.
\item
  Konvergiert \(\sum_{n=1}^\infty \frac{1}{\sqrt{n}}\)?
\item
  Schreiben Sie die ersten vier Teilsummen der Reihe
  \(\sum_{n=1}^\infty \frac{1}{2^n}\).
\item
  Erklären Sie, warum \(\lim a_n = 0\) für die Konvergenz notwendig,
  aber nicht ausreichend ist.
\end{enumerate}

\subsection{12.2 Konvergenztests}\label{konvergenztests}

Da viele Reihen nicht direkt summiert werden können, haben Mathematiker
Tests entwickelt, um zu entscheiden, ob eine Reihe konvergiert oder
divergiert. Diese Tests sind Werkzeuge zur Analyse unendlicher Summen.

\subsubsection{1. Der Divergenztest im n-ten
Semester}\label{der-divergenztest-im-n-ten-semester}

Wenn

\[
\lim_{n\to\infty} a_n \neq 0 \quad \text{or does not exist},
\]

dann

\[
\sum a_n
\]

divergiert.

Bei \(\lim a_n = 0\) ist der Test nicht schlüssig.

\subsubsection{2. Vergleichstest}\label{vergleichstest}

Angenommen \(0 \leq a_n \leq b_n\) für alle \(n\).

\begin{itemize}
\tightlist
\item
  Wenn \(\sum b_n\) konvergiert, dann konvergiert auch \(\sum a_n\).
\item
  Wenn \(\sum a_n\) divergiert, dann divergiert auch \(\sum b_n\).
\end{itemize}

\subsubsection{3. Limit-Vergleichstest}\label{limit-vergleichstest}

Wenn \(a_n, b_n > 0\) und

\[
\lim_{n\to\infty} \frac{a_n}{b_n} = c,
\]

wobei \(0 < c < \infty\), dann \(\sum a_n\) und \(\sum b_n\) entweder
beide konvergieren oder beide divergieren.

\subsubsection{4. Verhältnistest}\label{verhuxe4ltnistest}

Berechnen Sie für \(\sum a_n\)

\[
L = \lim_{n\to\infty} \left| \frac{a_{n+1}}{a_n} \right|.
\]

\begin{itemize}
\tightlist
\item
  Wenn \(L < 1\), konvergiert die Reihe absolut.
\item
  Bei \(L > 1\) oder \(L = \infty\) divergiert die Reihe.
\item
  Bei \(L = 1\) ist der Test nicht schlüssig.
\end{itemize}

\subsubsection{5. Root-Test}\label{root-test}

Berechnen Sie für \(\sum a_n\)

\[
L = \lim_{n\to\infty} \sqrt[n]{|a_n|}.
\]

\begin{itemize}
\tightlist
\item
  Bei \(L < 1\) konvergiert die Reihe absolut.
\item
  Bei \(L > 1\) divergiert die Reihe.
\item
  Bei \(L = 1\) ist der Test nicht schlüssig.
\end{itemize}

\subsubsection{6. Wechselreihentest
(Leibniz-Test)}\label{wechselreihentest-leibniz-test}

Für Serien der Form

\[
\sum (-1)^n b_n \quad \text{or} \quad \sum (-1)^{n+1} b_n,
\]

wenn

\begin{enumerate}
\def\labelenumi{\arabic{enumi}.}
\tightlist
\item
  \(b_{n+1} \leq b_n\) (absteigend) und
\item
  \(\lim_{n\to\infty} b_n = 0\),
\end{enumerate}

dann konvergiert die Reihe.

\subsubsection{Beispiele}\label{beispiele-25}

\begin{enumerate}
\def\labelenumi{\arabic{enumi}.}
\tightlist
\item
  \(\sum \frac{1}{n^2}\): Vergleichstest → konvergiert.
\item
  \(\sum \frac{1}{n}\): Harmonische Reihe → divergiert.
\item
  \(\sum \frac{(-1)^n}{n}\): Wechselreihentest → konvergiert.
\item
  \(\sum \frac{n!}{n^n}\): Verhältnistest → konvergiert.
\item
  \(\sum \frac{2^n}{n}\): Root-Test → divergiert.
\end{enumerate}

\subsubsection{Warum das wichtig ist}\label{warum-das-wichtig-ist-27}

\begin{itemize}
\tightlist
\item
  Mit Konvergenztests können wir Reihen klassifizieren, ohne dass
  explizite Summen erforderlich sind.
\item
  Sie bieten systematische Möglichkeiten zur Handhabung unendlicher
  Prozesse in der Analysis.
\item
  Sie sind für spätere Themen wie Potenzreihen und Fourierreihen von
  entscheidender Bedeutung.
\end{itemize}

\subsubsection{Übungen}\label{uxfcbungen-45}

\begin{enumerate}
\def\labelenumi{\arabic{enumi}.}
\tightlist
\item
  Testkonvergenz von \(\sum \frac{1}{n^3}\).
\item
  Nutzen Sie den Verhältnistest für \(\sum \frac{3^n}{n!}\).3. Wenden
  Sie den Root-Test auf \(\sum \left(\frac{1}{2}\right)^n\) an.
\item
  Bestimmen Sie die Konvergenz von \(\sum \frac{(-1)^n}{\sqrt{n}}\).
\item
  Verwenden Sie den Grenzwertvergleichstest mit \(\frac{1}{n^2}\), um
  \(\sum \frac{1}{n^2+1}\) zu testen.
\end{enumerate}

\subsection{12.3 Absolute vs.~bedingte
Konvergenz}\label{absolute-vs.-bedingte-konvergenz}

Nicht alle Serien verhalten sich beim Vorzeichenwechsel gleich. Um dies
zu handhaben, unterscheiden wir zwischen absoluter Konvergenz und
bedingter Konvergenz.

\subsubsection{Absolute Konvergenz}\label{absolute-konvergenz}

Eine Reihe \(\sum a_n\) ist absolut konvergent, wenn

\[
\sum |a_n|
\]

konvergiert.

Satz: Wenn eine Reihe absolut konvergiert, dann konvergiert sie auch.

Beispiel:

\[
\sum \frac{(-1)^n}{n^2}.
\]

Hier konvergiert
\(\sum \left|\frac{(-1)^n}{n^2}\right| = \sum \frac{1}{n^2}\) (p-Reihe,
\(p=2\)). Die Reihe ist also absolut konvergent.

\subsubsection{Bedingte Konvergenz}\label{bedingte-konvergenz}

Eine Reihe \(\sum a_n\) ist bedingt konvergent, wenn sie konvergiert,
aber nicht absolut.

Beispiel:

\[
\sum \frac{(-1)^n}{n}.
\]

\begin{itemize}
\tightlist
\item
  Wechselreihentest → konvergiert.
\item
  Aber \(\sum \left|\frac{(-1)^n}{n}\right| = \sum \frac{1}{n}\)
  divergiert (harmonische Reihe). Die Reihe ist also bedingt konvergent.
\end{itemize}

\subsubsection{Umlagerungssatz}\label{umlagerungssatz}

Bei bedingt konvergenten Reihen kann eine Neuanordnung der Terme die
Summe verändern -- sogar dazu führen, dass sie zu einem anderen Wert
divergiert oder konvergiert.

Dieses überraschende Ergebnis zeigt die heikle Natur der bedingten
Konvergenz.

\subsubsection{Warum das wichtig ist}\label{warum-das-wichtig-ist-28}

\begin{itemize}
\tightlist
\item
  Die absolute Konvergenz ist stärker und garantiert Stabilität.
\item
  Bedingte Konvergenz unterstreicht die Bedeutung der Ordnung in
  unendlichen Summen.
\item
  Viele in der Praxis vorkommende alternierende Reihen sind nur bedingt
  konvergent.
\end{itemize}

\subsubsection{Übungen}\label{uxfcbungen-46}

\begin{enumerate}
\def\labelenumi{\arabic{enumi}.}
\tightlist
\item
  Zeigen Sie, dass \(\sum \frac{(-1)^n}{n^3}\) absolut konvergiert.
\item
  Zeigen Sie, dass \(\sum \frac{(-1)^n}{n}\) bedingt konvergent ist.
\item
  Testen Sie \(\sum \frac{(-1)^n}{\sqrt{n}}\) auf absolute und bedingte
  Konvergenz.
\item
  Erklären Sie, warum absolute Konvergenz Konvergenz impliziert, das
  Gegenteil jedoch nicht der Fall ist.
\item
  Recherchieren Sie den Satz der Riemannschen Umlagerung und fassen Sie
  ihn in Ihren eigenen Worten zusammen.
\end{enumerate}

\section{Kapitel 13. Potenzreihen und
Erweiterungen}\label{kapitel-13.-potenzreihen-und-erweiterungen}

\subsection{13.1 Potenzreihe}\label{potenzreihe}

Eine Potenzreihe ist eine unendliche Reihe, in der jeder Term eine
Potenz der Variablen beinhaltet. Potenzreihen sind in der Analysis von
zentraler Bedeutung, da sie es uns ermöglichen, Funktionen als
unendliche Polynome darzustellen.

\subsubsection{Allgemeines Formular}\label{allgemeines-formular}

Eine bei \(a\) zentrierte Potenzreihe hat die Form

\[
\sum_{n=0}^\infty c_n (x-a)^n,
\]

wobei \(c_n\) Konstanten sind, die als Koeffizienten bezeichnet werden.

\begin{itemize}
\item
  Bei \(a=0\) wird die Reihe am Ursprung zentriert:

  \[
  \sum_{n=0}^\infty c_n x^n.
  \]
\end{itemize}

\subsubsection{Beispiele}\label{beispiele-26}

\begin{enumerate}
\def\labelenumi{\arabic{enumi}.}
\tightlist
\item
  Geometrische Reihe
\end{enumerate}

\[
\sum_{n=0}^\infty x^n = \frac{1}{1-x}, \quad |x|<1.
\]

\begin{enumerate}
\def\labelenumi{\arabic{enumi}.}
\setcounter{enumi}{1}
\tightlist
\item
  Exponentialfunktion
\end{enumerate}

\[e^x = \sum_{n=0}^\infty \frac{x^n}{n!}.
\]

\begin{enumerate}
\def\labelenumi{\arabic{enumi}.}
\setcounter{enumi}{2}
\tightlist
\item
  Sine and cosine
\end{enumerate}

\[
\sin x = \sum_{n=0}^\infty (-1)^n \frac{x^{2n+1}}{(2n+1)!}, \quad  
\cos x = \sum_{n=0}^\infty (-1)^n \frac{x^{2n}}{(2n)!}.
\]

\subsubsection{Interval of Convergence}\label{interval-of-convergence}

For each power series, there exists a radius of convergence \(R\) such
that:

\begin{itemize}
\tightlist
\item
  The series converges if \(|x-a| < R\).
\item
  The series diverges if \(|x-a| > R\).
\item
  At \(|x-a| = R\), convergence must be checked separately.
\end{itemize}

\subsubsection{Why This Matters}\label{why-this-matters-3}

\begin{itemize}
\tightlist
\item
  Power series allow us to approximate functions by polynomials.
\item
  They connect calculus with analysis and differential equations.
\item
  Many special functions in mathematics and physics are defined by their
  power series.
\end{itemize}

\subsubsection{Exercises}\label{exercises-6}

\begin{enumerate}
\def\labelenumi{\arabic{enumi}.}
\tightlist
\item
  Write the power series for \(\sum_{n=0}^\infty \frac{(x-2)^n}{n!}\).
\item
  Find the first four terms of the power series for \(e^x\).
\item
  Express \(\frac{1}{1+x}\) as a power series centered at 0.
\item
  Determine whether the series \(\sum_{n=0}^\infty n! x^n\) converges at
  \(x=0.1\).
\item
  Explain why power series are sometimes called ``infinite
  polynomials.''
\end{enumerate}

\subsection{13.2 Radius of Convergence}\label{radius-of-convergence}

Every power series converges for some values of \(x\) and diverges for
others. The boundary between these two behaviors is described by the
radius of convergence.

\subsubsection{Definition}\label{definition-13}

For a power series

\[
\sum_{n=0}^\infty c_n (x-a)^n,
\]

there exists a number \(R \geq 0\) (possibly infinite) such that:

\begin{itemize}
\tightlist
\item
  The series converges absolutely if \(|x-a| < R\).
\item
  The series diverges if \(|x-a| > R\).
\item
  At \(|x-a| = R\), convergence must be checked separately.
\end{itemize}

This number \(R\) is called the radius of convergence.

\subsubsection{Finding the Radius of
Convergence}\label{finding-the-radius-of-convergence}

Two common methods:

\begin{enumerate}
\def\labelenumi{\arabic{enumi}.}
\tightlist
\item
  Ratio Test
\end{enumerate}

\[
R = \lim_{n\to\infty} \left| \frac{c_n}{c_{n+1}} \right|.
\]

\begin{enumerate}
\def\labelenumi{\arabic{enumi}.}
\setcounter{enumi}{1}
\tightlist
\item
  Root Test
\end{enumerate}

\[
R = \frac{1}{\limsup_{n\to\infty} \sqrt[n]{|c_n|}}.
\]

\subsubsection{Examples}\label{examples-4}

\begin{enumerate}
\def\labelenumi{\arabic{enumi}.}
\tightlist
\item
  Series:
\end{enumerate}

\[
\sum_{n=0}^\infty \frac{x^n}{n!}.
\]

Using ratio test:

\[
\lim_{n\to\infty} \frac{1/(n!)}{1/((n+1)!)} = \infty.
\]

So \(R = \infty\) (converges for all real \(x\)).

\begin{enumerate}
\def\labelenumi{\arabic{enumi}.}
\setcounter{enumi}{1}
\tightlist
\item
  Series:
\end{enumerate}

\[
\sum_{n=0}^\infty x^n.
\]

Here \(c_n = 1\).

\[
R = 1.
\]

Converges for \(|x| < 1\).

\begin{enumerate}
\def\labelenumi{\arabic{enumi}.}
\setcounter{enumi}{2}
\tightlist
\item
  Series:
\end{enumerate}

\[
\sum_{n=1}^\infty \frac{x^n}{n}.
\]

Ratio test:

\[
\lim_{n\to\infty} \left|\frac{(x^{n+1}/(n+1))}{(x^n/n)}\right| = |x|.
\]

Also \(R = 1\). Konvergiert für \(|x| < 1\), divergiert für \(|x| > 1\).
Für \(x=\pm 1\), separat testen.

\subsubsection{Konvergenzintervall}\label{konvergenzintervall}

Die Menge der \(x\)-Werte, bei der die Reihe konvergiert, wird als
Konvergenzintervall bezeichnet.

\begin{itemize}
\tightlist
\item
  Immer zentriert bei \(a\).
\item
  Verlängert \(R\)-Einheiten in beide Richtungen.
\item
  Endpunkte \(x=a\pm R\) müssen einzeln geprüft werden.
\end{itemize}

\subsubsection{Warum das wichtig ist- Der Konvergenzradius sagt uns, wo
sich Potenzreihen wie Funktionen
verhalten.}\label{warum-das-wichtig-ist--der-konvergenzradius-sagt-uns-wo-sich-potenzreihen-wie-funktionen-verhalten.}

\begin{itemize}
\tightlist
\item
  Unverzichtbar für die praktische Anwendung von
  Taylor-Reihenentwicklungen.
\item
  Bestimmt den Gültigkeitsbereich von Reihenlösungen in Physik und
  Ingenieurwesen.
\end{itemize}

\subsubsection{Übungen}\label{uxfcbungen-47}

\begin{enumerate}
\def\labelenumi{\arabic{enumi}.}
\tightlist
\item
  Finden Sie den Konvergenzradius von
  \(\sum_{n=0}^\infty \frac{(x-3)^n}{n!}\).
\item
  Berechnen Sie den Konvergenzradius von
  \(\sum_{n=1}^\infty \frac{x^n}{n^2}\).
\item
  Verwenden Sie den Verhältnistest, um \(R\) für
  \(\sum_{n=0}^\infty n!x^n\) zu finden.
\item
  Bestimmen Sie das Konvergenzintervall für
  \(\sum_{n=1}^\infty \frac{(x+1)^n}{n}\).
\item
  Erklären Sie, warum die Exponentialreihe für alle \(x\) konvergiert,
  während die geometrische Reihe nur für \(|x|<1\) konvergiert.
\end{enumerate}

\subsection{13.3 Taylor- und
Maclaurin-Reihe}\label{taylor--und-maclaurin-reihe}

Potenzreihen werden besonders aussagekräftig, wenn sie zur Darstellung
bekannter Funktionen verwendet werden. Dies geschieht durch
Taylor-Reihen, und der bei 0 zentrierte Sonderfall wird Maclaurin-Reihe
genannt.

\subsubsection{Taylor-Serie}\label{taylor-serie}

Wenn eine Funktion \(f(x)\) bei \(x=a\) unendlich differenzierbar ist,
ist ihre Taylor-Reihe um \(a\) unendlich differenzierbar

\[
f(x) = \sum_{n=0}^\infty \frac{f^{(n)}(a)}{n!}(x-a)^n.
\]

Dabei bezeichnet \(f^{(n)}(a)\) die \(n\)-te Ableitung von \(f\) zu
\(a\).

\subsubsection{Maclaurin-Serie}\label{maclaurin-serie}

Eine Taylor-Reihe mit Schwerpunkt auf \(a=0\):

\[
f(x) = \sum_{n=0}^\infty \frac{f^{(n)}(0)}{n!} x^n.
\]

\subsubsection{Beispiele}\label{beispiele-27}

\begin{enumerate}
\def\labelenumi{\arabic{enumi}.}
\tightlist
\item
  Exponentialfunktion
\end{enumerate}

\[
e^x = 1 + x + \frac{x^2}{2!} + \frac{x^3}{3!} + \cdots
\]

\begin{enumerate}
\def\labelenumi{\arabic{enumi}.}
\setcounter{enumi}{1}
\tightlist
\item
  Sinus und Cosinus
\end{enumerate}

\[
\sin x = x - \frac{x^3}{3!} + \frac{x^5}{5!} - \cdots
\]

\[
\cos x = 1 - \frac{x^2}{2!} + \frac{x^4}{4!} - \cdots
\]

\begin{enumerate}
\def\labelenumi{\arabic{enumi}.}
\setcounter{enumi}{2}
\tightlist
\item
  Natürlicher Logarithmus (für \(|x|<1\))
\end{enumerate}

\[
\ln(1+x) = x - \frac{x^2}{2} + \frac{x^3}{3} - \frac{x^4}{4} + \cdots
\]

\subsubsection{Taylor-Polynom-Approximation}\label{taylor-polynom-approximation}

Die endliche Summe der ersten \(n\)-Terme ist das Taylor-Polynom vom
Grad \(n\):

\[
P_n(x) = \sum_{k=0}^n \frac{f^{(k)}(a)}{k!}(x-a)^k.
\]

Dieses Polynom nähert sich \(f(x)\) in der Nähe von \(x=a\) an.

\subsubsection{Rest (Fehlerbegriff)}\label{rest-fehlerbegriff}

Der Unterschied zwischen der Funktion und ihrem Taylor-Polynom ist der
Rest:

\[
R_n(x) = f(x) - P_n(x).
\]

Eine Form (Lagranges Form) ist

\[
R_n(x) = \frac{f^{(n+1)}(c)}{(n+1)!}(x-a)^{n+1},
\]

für einige \(c\) zwischen \(a\) und \(x\).

\subsubsection{Warum das wichtig ist}\label{warum-das-wichtig-ist-29}

\begin{itemize}
\tightlist
\item
  Taylor-Reihen liefern polynomielle Approximationen für komplizierte
  Funktionen.
\item
  Sie sind in der numerischen Analyse, Physik und Technik von
  wesentlicher Bedeutung.
\item
  Maclaurin-Reihenentwicklungen liefern einfache Formeln für
  exponentielle, trigonometrische und logarithmische Funktionen.
\end{itemize}

\subsubsection{Übungen}\label{uxfcbungen-48}

\begin{enumerate}
\def\labelenumi{\arabic{enumi}.}
\tightlist
\item
  Finden Sie die Maclaurin-Serie für
  \(f(x)=\cosh x = \tfrac{e^x+e^{-x}}{2}\).
\item
  Schreiben Sie die Taylor-Reihe für \(f(x)=e^x\) mit der Mitte bei
  \(a=2\).
\item
  Berechnen Sie das Taylor-Polynom Grad 3 für \(f(x)=\ln(1+x)\) bei
  \(a=0\).4. Verwenden Sie die Maclaurin-Reihe für \(\sin x\), um
  \(\sin(0.1)\) anzunähern.
\item
  Erklären Sie, warum Taylor-Reihen oft gute lokale Näherungen liefern,
  aber für große \(|x|\) divergieren können.
\end{enumerate}

\subsection{13.4 Anwendungen der
Taylor-Reihe}\label{anwendungen-der-taylor-reihe}

Taylor-Reihen sind nicht nur theoretische Werkzeuge -- sie werden auch
zur Approximation von Funktionen, zur Lösung von Gleichungen und zur
Analyse physikalischer Systeme verwendet. Ihre Anwendungen umfassen
Mathematik, Naturwissenschaften und Ingenieurwissenschaften.

\subsubsection{Funktionsnäherung}\label{funktionsnuxe4herung}

Komplizierte Funktionen können durch Polynome in der Nähe eines Punktes
angenähert werden.

Beispiel: Ungefähr \(e^x\) nahe \(x=0\) unter Verwendung des
Maclaurin-Polynoms Grad 3:

\[
P_3(x) = 1 + x + \tfrac{x^2}{2} + \tfrac{x^3}{6}.
\]

Für kleine \(x\) ergibt dies genaue Schätzungen von \(e^x\).

\subsubsection{Numerische Methoden}\label{numerische-methoden}

Taylor-Reihen bilden die Grundlage für numerische Algorithmen:

\begin{itemize}
\tightlist
\item
  Näherung von Quadratwurzeln, Logarithmen und trigonometrischen Werten.
\item
  Fehlerschätzung über die Restlaufzeit. -- Wird in iterativen Methoden
  wie der Newton-Methode verwendet (wobei die lokale Linearisierung aus
  der Taylor-Reihe stammt).
\end{itemize}

\subsubsection{Differentialgleichungen
lösen}\label{differentialgleichungen-luxf6sen}

Viele Differentialgleichungen haben Lösungen, die als Taylor-Reihe (oder
Potenzreihe) ausgedrückt werden.

Beispiel: Die Lösung zu \(y'' + y = 0\) mit \(y(0)=0, y'(0)=1\) ist
\(\sin x\), was natürlich aus seiner Maclaurin-Reihe hervorgeht.

\subsubsection{Physik und
Ingenieurwesen}\label{physik-und-ingenieurwesen}

\begin{itemize}
\item
  Kleinwinkelnäherung:

  \[
  \sin x \approx x, \quad \cos x \approx 1 - \tfrac{x^2}{2}, \quad |x| \ll 1.
  \]

  Wird in der Pendelbewegung, Optik und Wellenmechanik verwendet.
\item
  Relativitätstheorie und Quantenmechanik: Taylor-Entwicklungen
  vereinfachen nichtlineare Ausdrücke für den praktischen Gebrauch.
\item
  Approximierende Energiefunktionen: In der Mechanik werden potentielle
  Energiefunktionen in der Nähe von Gleichgewichtspunkten entwickelt.
\end{itemize}

\subsubsection{Wahrscheinlichkeit und
Statistik}\label{wahrscheinlichkeit-und-statistik}

\begin{itemize}
\tightlist
\item
  Momentenerzeugende Funktionen und charakteristische Funktionen
  verwenden Potenzreihen.
\item
  Approximationen von Wahrscheinlichkeitsverteilungen (z. B.
  Normalnäherung an Binomiale) verwenden Taylor-Entwicklungen.
\end{itemize}

\subsubsection{Warum das wichtig ist}\label{warum-das-wichtig-ist-30}

\begin{itemize}
\tightlist
\item
  Taylor-Reihen schlagen eine Brücke zwischen exakten Formeln und
  praktischer Berechnung.
\item
  Sie ermöglichen es uns, komplexe Probleme auf handhabbare
  Polynomnäherungen zu reduzieren.
\item
  Anwendungen machen sie zu einem der wichtigsten Werkzeuge in der
  angewandten Mathematik.
\end{itemize}

\subsubsection{Übungen}\label{uxfcbungen-49}

\begin{enumerate}
\def\labelenumi{\arabic{enumi}.}
\tightlist
\item
  Verwenden Sie die Maclaurin-Reihe für \(e^x\), um \(e^{0.1}\) bis zu
  vier Dezimalstellen anzunähern.
\item
  Wenden Sie die Kleinwinkelnäherung an, um \(\sin(5^\circ)\) zu
  schätzen.
\item
  Lösen Sie die Differentialgleichung \(y'' = -y\) mit einem
  Potenzreihenansatz.
\item
  Erweitern Sie \(\ln(1+x)\) bis zum 4. Grad und verwenden Sie es, um
  \(\ln(1.1)\) anzunähern.
\item
  Erklären Sie, warum Polynomnäherungen besonders nützlich für Computer
  und Taschenrechner sind.\# Anhänge
\end{enumerate}

\subsection{Anhang A. Grundlagen der
Vorkalkulation}\label{anhang-a.-grundlagen-der-vorkalkulation}

\subsubsection{A.1 Algebra-Auffrischung}\label{a.1-algebra-auffrischung}

Bevor Sie in die Analysis eintauchen, ist es hilfreich, einige
algebraische Fähigkeiten zu wiederholen, die immer wieder auftauchen.
Dies sind die „Werkzeuge``, die Sie zum Bearbeiten von Ausdrücken, Lösen
von Gleichungen und Vereinfachen von Ergebnissen benötigen.

\paragraph{Exponenten und Potenzen}\label{exponenten-und-potenzen}

\begin{itemize}
\item
  Grundregeln:

  \[
  a^m \cdot a^n = a^{m+n}, \quad \frac{a^m}{a^n} = a^{m-n}, \quad (a^m)^n = a^{mn}.
  \]
\item
  Negative Exponenten:

  \[
  a^{-n} = \frac{1}{a^n}, \quad a \neq 0.
  \]
\item
  Bruchexponenten:

  \[
  a^{1/n} = \sqrt[n]{a}, \quad a^{m/n} = \sqrt[n]{a^m}.
  \]
\end{itemize}

\paragraph{Factoring}\label{factoring}

Faktorisieren vereinfacht Ausdrücke und hilft beim Lösen von
Gleichungen.

\begin{enumerate}
\def\labelenumi{\arabic{enumi}.}
\item
  Gemeinsamer Faktor:

  \[
  6x^2+9x = 3x(2x+3).
  \]
\item
  Quadratdifferenz:

  \[
  a^2-b^2 = (a-b)(a+b).
  \]
\item
  Quadratische Trinome:

  \[
  x^2+5x+6 = (x+2)(x+3).
  \]
\end{enumerate}

\paragraph{Polynome}\label{polynome}

\begin{itemize}
\tightlist
\item
  Standardformular: \(P(x) = a_nx^n + a_{n-1}x^{n-1} + \cdots + a_0\).
\item
  Grad: die größte Leistung von \(x\).
\item
  Lange Division und synthetische Division eignen sich zur Vereinfachung
  rationaler Funktionen.
\end{itemize}

\paragraph{Rationale Ausdrücke}\label{rationale-ausdruxfccke}

Vereinfachen Sie, indem Sie Zähler und Nenner faktorisieren:

\[
\frac{x^2-1}{x^2-2x+1} = \frac{(x-1)(x+1)}{(x-1)^2} = \frac{x+1}{x-1}, \quad x \neq 1.
\]

\paragraph{Logarithmen}\label{logarithmen}

\begin{itemize}
\item
  Definition: \(\log_a b = c\) bedeutet \(a^c = b\).
\item
  Gemeinsame Basen: Naturlog (\(\ln x = \log_e x\)) und Basis 10
  (\(\log x\)).
\item
  Regeln:

  \[
  \log(ab) = \log a + \log b, \quad \log\left(\frac{a}{b}\right) = \log a - \log b, \quad \log(a^n) = n\log a.
  \]
\end{itemize}

\paragraph{Gleichungen}\label{gleichungen}

\begin{itemize}
\item
  Linear: \(ax+b=0\) → \(x=-b/a\) lösen.
\item
  Quadratisch: \(ax^2+bx+c=0\) hat Lösungen

  \[
  x=\frac{-b\pm \sqrt{b^2-4ac}}{2a}.
  \]
\item
  Exponentiell: \(e^x = k\) → \(x = \ln k\).
\end{itemize}

\subsubsection{A.2 Grundlagen der
Trigonometrie}\label{a.2-grundlagen-der-trigonometrie}

Die Trigonometrie liefert die Sprache der Winkel und periodischen
Phänomene. Da es in der Infinitesimalrechnung oft um Schwingungen,
Bewegungen und Wellen geht, ist ein solides Verständnis der
trigonometrischen Funktionen und ihrer Eigenschaften unerlässlich.

\paragraph{Der Einheitskreis}\label{der-einheitskreis}

\begin{itemize}
\item
  Definiert als Kreis mit Radius 1, der im Ursprung in der
  Koordinatenebene zentriert ist.
\item
  Für einen Winkel \(\theta\) gemessen von der positiven \(x\)-Achse:

  \[
  (\cos \theta, \sin \theta)
  \]

  gibt die Koordinaten des Punktes auf dem Kreis an.
\end{itemize}

Besondere Werte:

\begin{longtable}[]{@{}
  >{\raggedright\arraybackslash}p{(\linewidth - 6\tabcolsep) * \real{0.3333}}
  >{\raggedright\arraybackslash}p{(\linewidth - 6\tabcolsep) * \real{0.1667}}
  >{\raggedright\arraybackslash}p{(\linewidth - 6\tabcolsep) * \real{0.1667}}
  >{\raggedright\arraybackslash}p{(\linewidth - 6\tabcolsep) * \real{0.3333}}@{}}
\toprule\noalign{}
\begin{minipage}[b]{\linewidth}\raggedright
\(\theta\)
\end{minipage} & \begin{minipage}[b]{\linewidth}\raggedright
\(\sin \theta\)
\end{minipage} & \begin{minipage}[b]{\linewidth}\raggedright
\(\cos \theta\)
\end{minipage} & \begin{minipage}[b]{\linewidth}\raggedright
\(\tan \theta = \frac{\sin \theta}{\cos \theta}\)
\end{minipage} \\
\midrule\noalign{}
\endhead
\bottomrule\noalign{}
\endlastfoot
\(0\) & 0 & 1 & 0 \\
\(\pi/6\) & 1/2 & \(\sqrt{3}/2\) & \(1/\sqrt{3}\) \\
\(\pi/3\) & \(\sqrt{3}/2\) & 1/2 & \(\sqrt{3}\) \\
\(\pi/2\) & 1 & 0 & undefiniert \\
\end{longtable}

\paragraph{Grundlegende Identitäten}\label{grundlegende-identituxe4ten}

\begin{enumerate}
\def\labelenumi{\arabic{enumi}.}
\tightlist
\item
  Pythagoräische Identität
\end{enumerate}

\[
\sin^2\theta + \cos^2\theta = 1.
\]

\begin{enumerate}
\def\labelenumi{\arabic{enumi}.}
\setcounter{enumi}{1}
\tightlist
\item
  Quotientenidentitäten
\end{enumerate}

\[
\tan\theta = \frac{\sin\theta}{\cos\theta}, \quad \cot\theta = \frac{\cos\theta}{\sin\theta}.
\]

\begin{enumerate}
\def\labelenumi{\arabic{enumi}.}
\setcounter{enumi}{2}
\tightlist
\item
  Gegenseitige Identitäten
\end{enumerate}

\[
\sec\theta = \frac{1}{\cos\theta}, \quad \csc\theta = \frac{1}{\sin\theta}.
\]

\paragraph{Winkeladditionsformeln}\label{winkeladditionsformeln}

\[
\sin(\alpha+\beta) = \sin\alpha\cos\beta + \cos\alpha\sin\beta,
\]

\[
\cos(\alpha+\beta) = \cos\alpha\cos\beta - \sin\alpha\sin\beta.
\]

Sonderfälle:

\begin{itemize}
\item
  Doppelwinkel:

  \[
  \sin(2\theta) = 2\sin\theta\cos\theta, \quad
  \cos(2\theta) = \cos^2\theta - \sin^2\theta.
  \]
\end{itemize}

\paragraph{Diagramme}\label{diagramme}

\begin{itemize}
\tightlist
\item
  \(\sin x\): Welle beginnend bei 0, Amplitude 1, Periode \(2\pi\).
\item
  \(\cos x\): Welle beginnend bei 1, Amplitude 1, Periode \(2\pi\).
\item
  \(\tan x\): Wiederholungen alle \(\pi\), undefiniert bei ungeraden
  Vielfachen von \(\pi/2\).
\end{itemize}

\subsubsection{A.3 Koordinatengeometrie}\label{a.3-koordinatengeometrie}

Die Koordinatengeometrie verknüpft Algebra und Geometrie, indem sie
geometrische Objekte (Linien, Kreise, Kurven) mithilfe von Gleichungen
beschreibt. Die Infinitesimalrechnung stützt sich in hohem Maße auf
dieses Framework, um Funktionen grafisch darzustellen, Steigungen zu
finden und Kurven zu analysieren.

\paragraph{Die kartesische Ebene}\label{die-kartesische-ebene}

\begin{itemize}
\item
  Ein Punkt wird durch die Koordinaten \((x,y)\) dargestellt.
\item
  Abstand zwischen zwei Punkten \((x_1,y_1)\) und \((x_2,y_2)\):

  \[
  d = \sqrt{(x_2-x_1)^2 + (y_2-y_1)^2}.
  \]
\item
  Mittelpunkt eines Liniensegments:

  \[
  M = \left(\frac{x_1+x_2}{2}, \frac{y_1+y_2}{2}\right).
  \]
\end{itemize}

\paragraph{Linien}\label{linien}

\begin{enumerate}
\def\labelenumi{\arabic{enumi}.}
\item
  Steigungsformel

  \[
  m = \frac{y_2-y_1}{x_2-x_1}.
  \]
\item
  Gleichung einer Geraden

  \begin{itemize}
  \item
    Punkt-Steigungsform:

    \[
    y-y_1 = m(x-x_1).
    \]
  \item
    Steigungsabschnittsform:

    \[
    y = mx+b.
    \]
  \end{itemize}
\item
  Parallele und senkrechte Linien

  \begin{itemize}
  \tightlist
  \item
    Parallele Linien: gleiche Steigung.
  \item
    Senkrechte Linien: Steigungen genügen \(m_1m_2 = -1\).
  \end{itemize}
\end{enumerate}

\paragraph{Kreise}\label{kreise}

Gleichung eines Kreises mit Mittelpunkt \((h,k)\) und Radius \(r\):

\[
(x-h)^2+(y-k)^2 = r^2.
\]

Sonderfall: Einheitskreis mit Mittelpunkt im Ursprung:

\[
x^2+y^2=1.
\]

\paragraph{Konische Abschnitte}\label{konische-abschnitte}

\begin{enumerate}
\def\labelenumi{\arabic{enumi}.}
\item
  Parabel:

  \begin{itemize}
  \item
    Standardform (Öffnung nach oben/unten):

    \[
    y = ax^2+bx+c.
    \]
  \end{itemize}
\item
  Ellipse (im Ursprung zentriert):

  \[
  \frac{x^2}{a^2}+\frac{y^2}{b^2}=1.
  \]
\item
  Hyperbel (zentriert im Ursprung):

  \[
  \frac{x^2}{a^2}-\frac{y^2}{b^2}=1.
  \]
\end{enumerate}

\subsection{Anhang B. Wichtige Formeln und
Tabellen}\label{anhang-b.-wichtige-formeln-und-tabellen}

\subsubsection{B.1 AbleitungstabelleAbleitungen messen Änderungsraten
und Steigungen von Funktionen. Eine Schnellreferenztabelle hilft den
Lernenden, Formeln nicht jedes Mal neu
abzuleiten.}\label{b.1-ableitungstabelleableitungen-messen-uxe4nderungsraten-und-steigungen-von-funktionen.-eine-schnellreferenztabelle-hilft-den-lernenden-formeln-nicht-jedes-mal-neu-abzuleiten.}

\paragraph{Grundregeln}\label{grundregeln-1}

\begin{enumerate}
\def\labelenumi{\arabic{enumi}.}
\tightlist
\item
  Konstante Regel
\end{enumerate}

\[
\frac{d}{dx}[c] = 0
\]

\begin{enumerate}
\def\labelenumi{\arabic{enumi}.}
\setcounter{enumi}{1}
\tightlist
\item
  Machtregel
\end{enumerate}

\[
\frac{d}{dx}[x^n] = nx^{n-1}, \quad (n \in \mathbb{R})
\]

\begin{enumerate}
\def\labelenumi{\arabic{enumi}.}
\setcounter{enumi}{2}
\tightlist
\item
  Konstante Mehrfachregel
\end{enumerate}

\[
\frac{d}{dx}[c f(x)] = c f'(x)
\]

\begin{enumerate}
\def\labelenumi{\arabic{enumi}.}
\setcounter{enumi}{3}
\tightlist
\item
  Summen- und Differenzregel
\end{enumerate}

\[
\frac{d}{dx}[f(x)\pm g(x)] = f'(x)\pm g'(x)
\]

\paragraph{Trigonometrische
Funktionen}\label{trigonometrische-funktionen}

\[
\frac{d}{dx}[\sin x] = \cos x
\]

\[
\frac{d}{dx}[\cos x] = -\sin x
\]

\[
\frac{d}{dx}[\tan x] = \sec^2 x, \quad x \neq \tfrac{\pi}{2}+k\pi
\]

\[
\frac{d}{dx}[\cot x] = -\csc^2 x
\]

\[
\frac{d}{dx}[\sec x] = \sec x \tan x
\]

\[
\frac{d}{dx}[\csc x] = -\csc x \cot x
\]

\paragraph{Exponentielle und logarithmische
Funktionen}\label{exponentielle-und-logarithmische-funktionen}

\[
\frac{d}{dx}[e^x] = e^x
\]

\[
\frac{d}{dx}[a^x] = a^x \ln a, \quad a>0, a\neq 1
\]

\[
\frac{d}{dx}[\ln x] = \frac{1}{x}, \quad x>0
\]

\[
\frac{d}{dx}[\log_a x] = \frac{1}{x\ln a}, \quad a>0, a\neq 1
\]

\paragraph{Inverse trigonometrische
Funktionen}\label{inverse-trigonometrische-funktionen}

\[
\frac{d}{dx}[\arcsin x] = \frac{1}{\sqrt{1-x^2}}, \quad |x|<1
\]

\[
\frac{d}{dx}[\arccos x] = -\frac{1}{\sqrt{1-x^2}}, \quad |x|<1
\]

\[
\frac{d}{dx}[\arctan x] = \frac{1}{1+x^2}, \quad x \in \mathbb{R}
\]

\paragraph{Produkt-, Quotienten- und
Kettenregeln}\label{produkt--quotienten--und-kettenregeln}

\begin{enumerate}
\def\labelenumi{\arabic{enumi}.}
\tightlist
\item
  Produktregel
\end{enumerate}

\[
\frac{d}{dx}[f(x)g(x)] = f'(x)g(x)+f(x)g'(x)
\]

\begin{enumerate}
\def\labelenumi{\arabic{enumi}.}
\setcounter{enumi}{1}
\tightlist
\item
  Quotientenregel
\end{enumerate}

\[
\frac{d}{dx}\left[\frac{f(x)}{g(x)}\right] = \frac{f'(x)g(x)-f(x)g'(x)}{[g(x)]^2}, \quad g(x)\neq 0
\]

\begin{enumerate}
\def\labelenumi{\arabic{enumi}.}
\setcounter{enumi}{2}
\tightlist
\item
  Kettenregel
\end{enumerate}

\[
\frac{d}{dx}[f(g(x))] = f'(g(x))\cdot g'(x)
\]

\subsubsection{B.3 Gemeinsame
Serienerweiterungen}\label{b.3-gemeinsame-serienerweiterungen}

Mit Potenzreihen können wir Funktionen als unendliche Polynome
ausdrücken. Diese Erweiterungen sind für Approximationen, das Lösen von
Differentialgleichungen und den Aufbau einer Intuition über Funktionen
in der Analysis von wesentlicher Bedeutung.

\paragraph{Geometrische Serie}\label{geometrische-serie}

\[
\frac{1}{1-x} = \sum_{n=0}^\infty x^n, \quad |x| < 1
\]

\paragraph{Exponentialfunktion}\label{exponentialfunktion}

\[
e^x = \sum_{n=0}^\infty \frac{x^n}{n!}
= 1 + x + \frac{x^2}{2!} + \frac{x^3}{3!} + \cdots
\]

Gültig für alle \(x\).

\paragraph{Trigonometrische
Funktionen}\label{trigonometrische-funktionen-1}

\[
\sin x = \sum_{n=0}^\infty (-1)^n \frac{x^{2n+1}}{(2n+1)!}
= x - \frac{x^3}{3!} + \frac{x^5}{5!} - \cdots
\]

\[
\cos x = \sum_{n=0}^\infty (-1)^n \frac{x^{2n}}{(2n)!}
= 1 - \frac{x^2}{2!} + \frac{x^4}{4!} - \cdots
\]

\[
\tan^{-1} x = \sum_{n=0}^\infty (-1)^n \frac{x^{2n+1}}{2n+1}, \quad |x|\leq 1
\]

\paragraph{Logarithmus}\label{logarithmus}

\[
\ln(1+x) = \sum_{n=1}^\infty (-1)^{n+1} \frac{x^n}{n}, \quad -1 < x \leq 1
\]

\paragraph{Binomialentwicklung
(verallgemeinert)}\label{binomialentwicklung-verallgemeinert}

\[
(1+x)^r = \sum_{n=0}^\infty \binom{r}{n} x^n, \quad |x|<1
\]

wo

\[\binom{r}{n} = \frac{r(r-1)(r-2)\cdots(r-n+1)}{n!}.
\]

\subsection{Appendix C. Proof
Sketches}\label{appendix-c.-proof-sketches}

\subsubsection{\texorpdfstring{C.1 Limit Laws and the
\(\varepsilon\)--\(\delta\)
Definition}{C.1 Limit Laws and the \textbackslash varepsilon--\textbackslash delta Definition}}\label{c.1-limit-laws-and-the-varepsilondelta-definition}

Calculus rests on the precise meaning of a limit. While intuition
(``values get closer and closer'') is helpful, a formal definition
ensures rigor and avoids paradoxes.

\paragraph{Intuitive Idea}\label{intuitive-idea}

We write

\[
\lim_{x \to a} f(x) = L
\]

to mean that as \(x\) gets arbitrarily close to \(a\), the values of
\(f(x)\) get arbitrarily close to \(L\).

\paragraph{\texorpdfstring{Formal (\(\varepsilon\)--\(\delta\))
Definition}{Formal (\textbackslash varepsilon--\textbackslash delta) Definition}}\label{formal-varepsilondelta-definition}

We say that

\[
\lim_{x \to a} f(x) = L
\]

if for every \(\varepsilon > 0\), there exists a \(\delta > 0\) such
that whenever

\[
0 < |x-a| < \delta,
\]

we have

\[
|f(x) - L| < \varepsilon.
\]

\begin{itemize}
\tightlist
\item
  \(\varepsilon\): how close we want \(f(x)\) to be to \(L\).
\item
  \(\delta\): how close \(x\) must be to \(a\) to achieve that.
\end{itemize}

\paragraph{Example}\label{example}

Show that

\[
\lim_{x \to 2} (3x+1) = 7.
\]

\begin{itemize}
\tightlist
\item
  Let \(\varepsilon > 0\).
\item
  We want \(|(3x+1)-7| < \varepsilon\).
\item
  Simplify: \(|3x-6| = 3|x-2| < \varepsilon\).
\item
  This holds if we choose \(\delta = \varepsilon/3\).
\end{itemize}

Thus, by the definition, the limit is 7.

\paragraph{Limit Laws}\label{limit-laws}

If \(\lim_{x \to a} f(x) = L\) and \(\lim_{x \to a} g(x) = M\), then:

\begin{enumerate}
\def\labelenumi{\arabic{enumi}.}
\tightlist
\item
  Sum/Difference
\end{enumerate}

\[
\lim_{x \to a} [f(x) \pm g(x)] = L \pm M
\]

\begin{enumerate}
\def\labelenumi{\arabic{enumi}.}
\setcounter{enumi}{1}
\tightlist
\item
  Constant Multiple
\end{enumerate}

\[
\lim_{x \to a} [c f(x)] = cL
\]

\begin{enumerate}
\def\labelenumi{\arabic{enumi}.}
\setcounter{enumi}{2}
\tightlist
\item
  Product
\end{enumerate}

\[
\lim_{x \to a} [f(x)g(x)] = LM
\]

\begin{enumerate}
\def\labelenumi{\arabic{enumi}.}
\setcounter{enumi}{3}
\tightlist
\item
  Quotient (if \(M \neq 0\))
\end{enumerate}

\[
\lim_{x \to a} \frac{f(x)}{g(x)} = \frac{L}{M}
\]

\begin{enumerate}
\def\labelenumi{\arabic{enumi}.}
\setcounter{enumi}{4}
\tightlist
\item
  Powers and Roots
\end{enumerate}

\[
\lim_{x \to a} [f(x)]^n = L^n, \quad \lim_{x \to a} \sqrt[n]{f(x)} = \sqrt[n]{L} \ (\text{falls definiert}).
\]

\subsubsection{C.2 Proof Sketch: The Fundamental Theorem of
Calculus}\label{c.2-proof-sketch-the-fundamental-theorem-of-calculus}

The Fundamental Theorem of Calculus (FTC) links the two central
operations of calculus: differentiation and integration. It shows that
they are, in fact, inverse processes.

\paragraph{Statement of the Theorem}\label{statement-of-the-theorem}

Part I (Differentiation of an Integral): If \(f\) is continuous on
\([a,b]\) and we define

\[
F(x) = \int_a^x f(t)\,dt,
\]

then \(F\) is differentiable on \((a,b)\) and

\[
F'(x) = f(x).
\]

Part II (Evaluation of a Definite Integral): If \(F\) is any
antiderivative of \(f\) on \([a,b]\), then

\[
\int_a^b f(x)\,dx = F(b)-F(a).
\]

\paragraph{Proof Sketch of Part I}\label{proof-sketch-of-part-i}

\begin{enumerate}
\def\labelenumi{\arabic{enumi}.}
\item
  Start with the definition of the derivative:

  \[
  F'(x) = \lim_{h\to 0} \frac{F(x+h)-F(x)}{h}.
  \]
\item
  Substituting \(F(x) = \int_a^x f(t)\,dt\):

  \[
  F(x+h)-F(x) = \int_a^{x+h} f(t)\,dt - \int_a^x f(t)\,dt.
  \]
\item
  By the additivity of integrals:

  \[
  F(x+h)-F(x) = \int_x^{x+h} f(t)\,dt.
  \]
\item
  Therefore:

  \[
  \frac{F(x+h)-F(x)}{h} = \frac{1}{h}\int_x^{x+h} f(t)\,dt.
  \]5. Nach dem Mittelwertsatz für Integrale gibt es \(c \in [x,x+h]\),
  so dass

  \[
  \frac{1}{h}\int_x^{x+h} f(t)\,dt = f(c).
  \]
\item
  Da \(h \to 0\), \(c \to x\) und da \(f\) stetig ist:

  \[
  \lim_{h\to 0} f(c) = f(x).
  \]
\end{enumerate}

Also \(F'(x) = f(x)\).

\paragraph{Beweisskizze von Teil II}\label{beweisskizze-von-teil-ii}

\begin{enumerate}
\def\labelenumi{\arabic{enumi}.}
\item
  Sei \(F\) eine Stammfunktion von \(f\), also \(F'(x) = f(x)\).
\item
  Nach Teil I die Funktion

  \[
  G(x) = \int_a^x f(t)\,dt
  \]

  ist auch eine Stammfunktion von \(f\).
\item
  Da sich \(F\) und \(G\) nur durch eine Konstante unterscheiden,

  \[
  F(x) = G(x) + C.
  \]
\item
  Evaluierung an den Endpunkten:

  \[
  \int_a^b f(x)\,dx = G(b)-G(a) = (F(b)+C)-(F(a)+C) = F(b)-F(a).
  \]
\end{enumerate}

\subsubsection{C.3 Beweisskizze: Konvergenz der geometrischen
Reihe}\label{c.3-beweisskizze-konvergenz-der-geometrischen-reihe}

Die geometrische Reihe ist eine der einfachsten und wichtigsten
unendlichen Reihen. Es dient als Modell zum Verständnis der Konvergenz
und ist die Grundlage für viele spätere Ergebnisse in der Analysis.

\paragraph{Die Serie}\label{die-serie}

\[
\sum_{n=0}^\infty ar^n = a + ar + ar^2 + ar^3 + \cdots
\]

wobei \(a\) der erste Term und \(r\) das gemeinsame Verhältnis ist.

\paragraph{Teilsummenformel}\label{teilsummenformel}

Die \(n\)-te Teilsumme beträgt

\[
S_n = a + ar + ar^2 + \cdots + ar^n.
\]

Beide Seiten mit \(r\) multiplizieren:

\[
rS_n = ar + ar^2 + \cdots + ar^{n+1}.
\]

Subtrahieren Sie die beiden Gleichungen:

\[
S_n - rS_n = a - ar^{n+1}.
\]

\[
S_n(1-r) = a(1-r^{n+1}).
\]

Also

\[
S_n = \frac{a(1-r^{n+1})}{1-r}, \quad r \neq 1.
\]

\paragraph{Konvergenz}\label{konvergenz}

Nehmen Sie das Limit als \(n \to \infty\):

\begin{itemize}
\item
  Wenn \(|r| < 1\), dann \(r^{n+1} \to 0\).

  \[
  \lim_{n\to\infty} S_n = \frac{a}{1-r}.
  \]
\item
  Wenn \(|r| \geq 1\), dann geht \(r^{n+1}\) nicht auf 0. Die Reihe
  divergiert.
\end{itemize}

\paragraph{Ergebnis}\label{ergebnis}

\[
\sum_{n=0}^\infty ar^n =
\begin{cases}
\dfrac{a}{1-r}, & |r|<1, \\[6pt]
\text{diverges}, & |r|\geq 1.
\end{cases}
\]

\subsection{Anhang D. Anwendungen und
Verbindungen}\label{anhang-d.-anwendungen-und-verbindungen}

\subsubsection{D.1 Physikalische Zusammenhänge: Geschwindigkeit,
Beschleunigung und
Arbeit}\label{d.1-physikalische-zusammenhuxe4nge-geschwindigkeit-beschleunigung-und-arbeit}

Die Infinitesimalrechnung wurde ursprünglich entwickelt, um
physikalische Probleme zu lösen -- insbesondere Bewegung und
Veränderung. Hier sind einige der wichtigsten Verbindungen.

\paragraph{Position, Geschwindigkeit und
Beschleunigung}\label{position-geschwindigkeit-und-beschleunigung-1}

\begin{itemize}
\item
  Positionsfunktion: \(s(t)\) gibt den Standort eines Objekts zum
  Zeitpunkt \(t\) an.
\item
  Geschwindigkeit: die Ableitung der Position.

  \[
  v(t) = s'(t) = \frac{ds}{dt}
  \]
\item
  Beschleunigung: die Ableitung der Geschwindigkeit (oder zweite
  Ableitung der Position).

  \[
  a(t) = v'(t) = s''(t) = \frac{d^2s}{dt^2}
  \]
\end{itemize}

Beispiel: Wenn \(s(t) = 4t^2\) Meter, dann:

\[
v(t) = 8t, \quad a(t) = 8.
\]

Das Objekt bewegt sich also bei konstanter Beschleunigung linear mit der
Zeit schneller.

\paragraph{Arbeit und Kraft}\label{arbeit-und-kraft}

In der Physik ist Arbeit das Produkt aus Kraft und Weg. Wenn die Kraft
mit der Position variiert, ergibt die Infinitesimalrechnung:

\[W = \int_a^b F(x)\, dx
\]

where \(F(x)\) is the force at position \(x\), and the object moves from
\(x=a\) to \(x=b\).

Example: A spring with Hooke's law force \(F(x) = kx\) requires work

\[
W = \int_0^d kx\, dx = \frac{1}{2}kd^2
\]

to stretch the spring a distance \(d\).

\paragraph{Energy and Areas Under
Curves}\label{energy-and-areas-under-curves}

\begin{itemize}
\tightlist
\item
  Kinetic energy: \(E_k = \tfrac{1}{2}mv^2\).
\item
  Potential energy often involves integrals (e.g., gravitational
  potential energy from force of gravity).
\item
  In general, integrating a force function gives energy stored or work
  done.
\end{itemize}

\paragraph{Quick Practice}\label{quick-practice}

\begin{enumerate}
\def\labelenumi{\arabic{enumi}.}
\tightlist
\item
  If \(s(t) = t^3 - 3t\), find \(v(t)\) and \(a(t)\).
\item
  Compute the work done by a constant force of 10 N moving an object 5
  m.
\item
  A spring has constant \(k=200\). How much work is needed to stretch it
  0.1 m?
\item
  Show that acceleration is the second derivative of position.
\item
  Explain how the integral \(\int v(t)\, dt\) relates to displacement.
\end{enumerate}

\subsubsection{D.2 Probability and Statistics
Connections}\label{d.2-probability-and-statistics-connections}

Calculus is deeply connected with probability and statistics, especially
when dealing with continuous random variables. Integrals become
essential for defining probabilities, averages, and expectations.

\paragraph{Probability Density Functions
(PDFs)}\label{probability-density-functions-pdfs}

For a continuous random variable \(X\), probabilities are described by a
probability density function \(f(x)\):

\begin{enumerate}
\def\labelenumi{\arabic{enumi}.}
\item
  \(f(x) \geq 0\) for all \(x\).
\item
  Total probability equals 1:

  \[
  \int_{-\infty}^{\infty} f(x)\, dx = 1.
  \]
\end{enumerate}

The probability that \(X\) lies in an interval \([a,b]\) is

\[
P(a \leq X \leq b) = \int_a^b f(x)\, dx.
\]

\paragraph{Expected Value (Mean)}\label{expected-value-mean}

The expected value (average outcome) is

\[
E[X] = \int_{-\infty}^{\infty} x f(x)\, dx.
\]

This is the calculus version of a weighted average.

\paragraph{Variance}\label{variance}

Variance measures spread:

\[
\text{Var}(X) = E[(X-\mu)^2] = \int_{-\infty}^{\infty} (x-\mu)^2 f(x)\, dx,
\]

where \(\mu = E[X]\).

\paragraph{Common Distributions}\label{common-distributions}

\begin{enumerate}
\def\labelenumi{\arabic{enumi}.}
\item
  Uniform distribution on \([a,b]\):

  \[
  f(x) = \frac{1}{b-a}, \quad a \leq x \leq b.
  \]

  Mean: \(\frac{a+b}{2}\).
\item
  Exponential distribution with parameter \(\lambda > 0\):

  \[
  f(x) = \lambda e^{-\lambda x}, \quad x \geq 0.
  \]

  Mean: \(1/\lambda\).
\item
  Normal (Gaussian) distribution:

  \[
  f(x) = \frac{1}{\sqrt{2\pi\sigma^2}} e^{-(x-\mu)^2/(2\sigma^2)}.
  \]

  Integrale dieser Verteilung verbinden sich mit der Fehlerfunktion.
\end{enumerate}

\paragraph{Warum das wichtig ist}\label{warum-das-wichtig-ist-31}

\begin{itemize}
\tightlist
\item
  Integrale wandeln Wahrscheinlichkeiten in Flächen unter Kurven um.
\item
  Erwartung und Varianz verknüpfen die Berechnung mit
  Durchschnittswerten und Variabilität. -- Die meisten realen
  Datenmodelle (Finanzen, Physik, Biologie, KI) verwenden diese
  kontinuierlichen Wahrscheinlichkeitsverteilungen.
\end{itemize}

\paragraph{\texorpdfstring{Schnelle Übung1. Berechnen Sie für
\(f(x) = \tfrac{1}{2}\) auf \([0,2]\)
\(P(0.5 \leq X \leq 1.5)\).}{Schnelle Übung1. Berechnen Sie für f(x) = \textbackslash tfrac\{1\}\{2\} auf {[}0,2{]} P(0.5 \textbackslash leq X \textbackslash leq 1.5).}}\label{schnelle-uxfcbung1.-berechnen-sie-fuxfcr-fx-tfrac12-auf-02-p0.5-leq-x-leq-1.5.}

\begin{enumerate}
\def\labelenumi{\arabic{enumi}.}
\setcounter{enumi}{1}
\tightlist
\item
  Berechnen Sie für die Exponentialverteilung mit \(\lambda = 2\)
  \(E[X]\).
\item
  Zeigen Sie, dass die Gesamtfläche unter der Standardnormalkurve gleich
  1 ist.
\item
  Ermitteln Sie den Mittelwert einer Gleichverteilung auf \([3,7]\).
\item
  Erklären Sie, warum Wahrscheinlichkeiten für kontinuierliche Variablen
  mit Integralen und nicht mit Summen berechnet werden.
\end{enumerate}

\subsubsection{D.3 Informatikverbindungen: Taylor-Approximationen in
Algorithmen}\label{d.3-informatikverbindungen-taylor-approximationen-in-algorithmen}

Infinitesimalrechnung ist nicht nur für die Physik wichtig, sie bildet
auch die Grundlage für viele Werkzeuge und Techniken in der Informatik.
Eine der klarsten Brücken sind Taylor-Reihen, die effiziente
Möglichkeiten zur Approximation von Funktionen in numerischen
Berechnungen und Algorithmen bieten.

\paragraph{Funktionsnäherung für die
Datenverarbeitung}\label{funktionsnuxe4herung-fuxfcr-die-datenverarbeitung}

Computer können die meisten Funktionen nicht direkt speichern oder genau
berechnen (wie \(e^x\), \(\sin x\) oder \(\ln x\)). Stattdessen
verwenden sie Polynomnäherungen, die aus Taylor-Entwicklungen abgeleitet
sind.

Beispiel: Um ungefähr \(e^x\) zu erreichen, kürzen Sie die
Maclaurin-Reihe:

\[
e^x \approx 1 + x + \frac{x^2}{2!} + \frac{x^3}{3!}.
\]

Für kleine \(x\) liefert dieses Polynom mit nur wenigen Termen genaue
Ergebnisse.

\paragraph{Effizienz in Algorithmen}\label{effizienz-in-algorithmen}

\begin{itemize}
\tightlist
\item
  Trigonometrische Funktionen: Algorithmen für Taschenrechner und CPUs
  verwenden häufig Reihenentwicklungen (oder Variationen wie
  Tschebyscheff-Polynome).
\item
  Exponential/Logarithmus: Taylor-Entwicklungen sind die Grundlage für
  schnelle Approximationen in numerischen Bibliotheken.
\item
  Wurzelfindung: Newtons Methode basiert auf linearer Näherung, einer
  direkten Anwendung der Taylor-Reihe (erste Ableitung).
\end{itemize}

\paragraph{Numerische Analyse}\label{numerische-analyse}

Taylor-Entwicklungen sind von zentraler Bedeutung in der Fehleranalyse:

\begin{itemize}
\item
  Approximation des Fehlerterms mit der Restformel:

  \[
  R_n(x) = \frac{f^{(n+1)}(c)}{(n+1)!}(x-a)^{n+1}.
  \]
\item
  Dies sagt uns, wie viele Terme für eine bestimmte Genauigkeit benötigt
  werden.
\end{itemize}

\paragraph{Verbindungen zum maschinellen
Lernen}\label{verbindungen-zum-maschinellen-lernen}

\begin{itemize}
\tightlist
\item
  Gradientenbasierte Optimierung (wie Gradientenabstieg) verwendet
  Ableitungen, um Parameter effizient zu aktualisieren.
\item
  Aktivierungsfunktionen (wie \(\tanh x\) oder
  \(\sigma(x)=1/(1+e^{-x})\)) werden häufig durch Polynome oder
  stückweise Geschwindigkeitsfunktionen angenähert.
\item
  Reihennäherungen können das Training und die Inferenz in
  eingeschränkten Umgebungen beschleunigen.
\end{itemize}

\paragraph{Warum das wichtig ist}\label{warum-das-wichtig-ist-32}

\begin{itemize}
\tightlist
\item
  Taylor-Approximationen verbinden kontinuierliche Mathematik mit
  diskretem Rechnen.
\item
  Sie zeigen, wie Kalkülkonzepte in Algorithmen, numerischen Methoden
  und maschinellem Lernen verwendet werden.
\item
  Das Verständnis der Näherungswerte hilft, Fallstricke zu vermeiden,
  wenn man sich bei Berechnungen auf Computer verlässt.
\end{itemize}

\paragraph{Schnelle Übung}\label{schnelle-uxfcbung}

\begin{enumerate}
\def\labelenumi{\arabic{enumi}.}
\tightlist
\item
  Ungefähr \(\sin(0.1)\) unter Verwendung der ersten drei Begriffe der
  Maclaurin-Reihe.2. Verwenden Sie den Restterm, um den Fehler bei der
  Approximation von \(e^1\) mit einem Polynom Grad 3 abzuschätzen.
\item
  Erklären Sie, wie Newtons Methode den Satz von Taylor verwendet.
\item
  Warum bevorzugen Computer möglicherweise Polynomnäherungen gegenüber
  exakten Formeln für Funktionen?
\item
  Warum ist beim maschinellen Lernen die Ableitung (Gradient) so wichtig
  für die Optimierung?
\end{enumerate}




\end{document}
