% Options for packages loaded elsewhere
\PassOptionsToPackage{unicode}{hyperref}
\PassOptionsToPackage{hyphens}{url}
\PassOptionsToPackage{dvipsnames,svgnames,x11names}{xcolor}
%
\documentclass[
  letterpaper,
  DIV=11,
  numbers=noendperiod]{scrartcl}

\usepackage{amsmath,amssymb}
\usepackage{iftex}
\ifPDFTeX
  \usepackage[T1]{fontenc}
  \usepackage[utf8]{inputenc}
  \usepackage{textcomp} % provide euro and other symbols
\else % if luatex or xetex
  \usepackage{unicode-math}
  \defaultfontfeatures{Scale=MatchLowercase}
  \defaultfontfeatures[\rmfamily]{Ligatures=TeX,Scale=1}
\fi
\usepackage{lmodern}
\ifPDFTeX\else  
    % xetex/luatex font selection
\fi
% Use upquote if available, for straight quotes in verbatim environments
\IfFileExists{upquote.sty}{\usepackage{upquote}}{}
\IfFileExists{microtype.sty}{% use microtype if available
  \usepackage[]{microtype}
  \UseMicrotypeSet[protrusion]{basicmath} % disable protrusion for tt fonts
}{}
\makeatletter
\@ifundefined{KOMAClassName}{% if non-KOMA class
  \IfFileExists{parskip.sty}{%
    \usepackage{parskip}
  }{% else
    \setlength{\parindent}{0pt}
    \setlength{\parskip}{6pt plus 2pt minus 1pt}}
}{% if KOMA class
  \KOMAoptions{parskip=half}}
\makeatother
\usepackage{xcolor}
\setlength{\emergencystretch}{3em} % prevent overfull lines
\setcounter{secnumdepth}{-\maxdimen} % remove section numbering
% Make \paragraph and \subparagraph free-standing
\makeatletter
\ifx\paragraph\undefined\else
  \let\oldparagraph\paragraph
  \renewcommand{\paragraph}{
    \@ifstar
      \xxxParagraphStar
      \xxxParagraphNoStar
  }
  \newcommand{\xxxParagraphStar}[1]{\oldparagraph*{#1}\mbox{}}
  \newcommand{\xxxParagraphNoStar}[1]{\oldparagraph{#1}\mbox{}}
\fi
\ifx\subparagraph\undefined\else
  \let\oldsubparagraph\subparagraph
  \renewcommand{\subparagraph}{
    \@ifstar
      \xxxSubParagraphStar
      \xxxSubParagraphNoStar
  }
  \newcommand{\xxxSubParagraphStar}[1]{\oldsubparagraph*{#1}\mbox{}}
  \newcommand{\xxxSubParagraphNoStar}[1]{\oldsubparagraph{#1}\mbox{}}
\fi
\makeatother


\providecommand{\tightlist}{%
  \setlength{\itemsep}{0pt}\setlength{\parskip}{0pt}}\usepackage{longtable,booktabs,array}
\usepackage{calc} % for calculating minipage widths
% Correct order of tables after \paragraph or \subparagraph
\usepackage{etoolbox}
\makeatletter
\patchcmd\longtable{\par}{\if@noskipsec\mbox{}\fi\par}{}{}
\makeatother
% Allow footnotes in longtable head/foot
\IfFileExists{footnotehyper.sty}{\usepackage{footnotehyper}}{\usepackage{footnote}}
\makesavenoteenv{longtable}
\usepackage{graphicx}
\makeatletter
\newsavebox\pandoc@box
\newcommand*\pandocbounded[1]{% scales image to fit in text height/width
  \sbox\pandoc@box{#1}%
  \Gscale@div\@tempa{\textheight}{\dimexpr\ht\pandoc@box+\dp\pandoc@box\relax}%
  \Gscale@div\@tempb{\linewidth}{\wd\pandoc@box}%
  \ifdim\@tempb\p@<\@tempa\p@\let\@tempa\@tempb\fi% select the smaller of both
  \ifdim\@tempa\p@<\p@\scalebox{\@tempa}{\usebox\pandoc@box}%
  \else\usebox{\pandoc@box}%
  \fi%
}
% Set default figure placement to htbp
\def\fps@figure{htbp}
\makeatother

\KOMAoption{captions}{tableheading}
\makeatletter
\@ifpackageloaded{caption}{}{\usepackage{caption}}
\AtBeginDocument{%
\ifdefined\contentsname
  \renewcommand*\contentsname{Tabla de contenidos}
\else
  \newcommand\contentsname{Tabla de contenidos}
\fi
\ifdefined\listfigurename
  \renewcommand*\listfigurename{Listado de Figuras}
\else
  \newcommand\listfigurename{Listado de Figuras}
\fi
\ifdefined\listtablename
  \renewcommand*\listtablename{Listado de Tablas}
\else
  \newcommand\listtablename{Listado de Tablas}
\fi
\ifdefined\figurename
  \renewcommand*\figurename{Figura}
\else
  \newcommand\figurename{Figura}
\fi
\ifdefined\tablename
  \renewcommand*\tablename{Tabla}
\else
  \newcommand\tablename{Tabla}
\fi
}
\@ifpackageloaded{float}{}{\usepackage{float}}
\floatstyle{ruled}
\@ifundefined{c@chapter}{\newfloat{codelisting}{h}{lop}}{\newfloat{codelisting}{h}{lop}[chapter]}
\floatname{codelisting}{Listado}
\newcommand*\listoflistings{\listof{codelisting}{Listado de Listados}}
\makeatother
\makeatletter
\makeatother
\makeatletter
\@ifpackageloaded{caption}{}{\usepackage{caption}}
\@ifpackageloaded{subcaption}{}{\usepackage{subcaption}}
\makeatother

\ifLuaTeX
\usepackage[bidi=basic]{babel}
\else
\usepackage[bidi=default]{babel}
\fi
\babelprovide[main,import]{spanish}
% get rid of language-specific shorthands (see #6817):
\let\LanguageShortHands\languageshorthands
\def\languageshorthands#1{}
\usepackage{bookmark}

\IfFileExists{xurl.sty}{\usepackage{xurl}}{} % add URL line breaks if available
\urlstyle{same} % disable monospaced font for URLs
\hypersetup{
  pdftitle={El pequeño libro del cálculo},
  pdflang={es},
  colorlinks=true,
  linkcolor={blue},
  filecolor={Maroon},
  citecolor={Blue},
  urlcolor={Blue},
  pdfcreator={LaTeX via pandoc}}


\title{El pequeño libro del cálculo}
\author{}
\date{}

\begin{document}
\maketitle


\section{El librito de cálculo}\label{el-librito-de-cuxe1lculo}

Una introducción concisa y amigable para principiantes a las ideas
centrales del cálculo.

\subsection{Formatos}\label{formatos}

\begin{itemize}
\tightlist
\item
  \href{../artifacts/es/book.pdf}{Download PDF} -- versión lista para
  imprimir
\item
  \href{../artifacts/es/book.epub}{Download EPUB} -- compatible con
  lectores electrónicos
\item
  \href{../artifacts/es/book.tex}{View LaTeX} -- Fuente de látex
\end{itemize}

\section{Parte 1. Límites y
derivadas}\label{parte-1.-luxedmites-y-derivadas}

\section{Capítulo 1. Funciones y
límites}\label{capuxedtulo-1.-funciones-y-luxedmites}

\subsection{1.1 Funciones}\label{funciones}

Una función es uno de los objetos más básicos de las matemáticas. En
esencia, una función es una regla que toma una entrada y produce
exactamente una salida. Las funciones nos permiten describir relaciones,
modelar fenómenos del mundo real y construir toda la maquinaria del
cálculo.

\subsubsection{Definición}\label{definiciuxf3n}

Formalmente, se escribe una función \(f\) desde un conjunto \(X\)
(llamado dominio) hasta un conjunto \(Y\) (llamado codominio)

\[
f : X \to Y.
\]

Por cada elemento \(x \in X\), existe un elemento único \(f(x) \in Y\).
El valor \(f(x)\) se llama imagen de \(x\) bajo \(f\).

Si \(y = f(x)\), entonces \(y\) es la salida correspondiente a la
entrada \(x\). El conjunto de todas las salidas que realmente aparecen
se denomina rango (un subconjunto del codominio).

\subsubsection{Ejemplos}\label{ejemplos}

\begin{enumerate}
\def\labelenumi{\arabic{enumi}.}
\item
  La función \(f(x) = x^2\) asigna cada número real \(x\) a su cuadrado.

  \begin{itemize}
  \tightlist
  \item
    Dominio: todos los números reales \(\mathbb{R}\).
  \item
    Codominio: todos los números reales \(\mathbb{R}\).
  \item
    Rango: todos los números reales no negativos \([0, \infty)\).
  \end{itemize}
\item
  La función \(g(x) = \dfrac{1}{x}\) asigna a cada número real distinto
  de cero su recíproco.

  \begin{itemize}
  \tightlist
  \item
    Dominio: \(\mathbb{R} \setminus \{0\}\).
  \item
    Rango: \(\mathbb{R} \setminus \{0\}\).
  \end{itemize}
\item
  Un ejemplo del mundo real: Sea \(T(t)\) la temperatura exterior (en
  °C) en el momento \(t\) (en horas). Esta es una función desde ``hora
  del día'' hasta ``temperatura''.
\end{enumerate}

\subsubsection{Formas de representar
funciones}\label{formas-de-representar-funciones}

Las funciones se pueden representar de varias formas útiles:

\begin{itemize}
\tightlist
\item
  Fórmulas: ej., \(f(x) = \sin x + x^2\).
\item
  Gráficas: trazar todos los puntos \((x, f(x))\) en el plano
  coordenado.
\item
  Tablas: emparejamiento de entradas y salidas para conjuntos discretos
  de datos.
\item
  Descripciones verbales: ``Asignar a cada alumno su calificación''.
\end{itemize}

Cada representación resalta diferentes aspectos de una misma función.

\subsubsection{Terminología}\label{terminologuxeda}

\begin{itemize}
\tightlist
\item
  Variable independiente: la entrada (normalmente se escribe \(x\)).
\item
  Variable dependiente: la salida (generalmente escrita \(y\), donde
  \(y = f(x)\)).
\item
  Notación de funciones: \(f(x)\) se lee ``\(f\) de \(x\)''.
\end{itemize}

\subsubsection{Por qué las funciones son importantes en
cálculo}\label{por-quuxe9-las-funciones-son-importantes-en-cuxe1lculo}

El cálculo es el estudio de cómo cambian las funciones. Los derivados
miden tasas de cambio instantáneas, mientras que las integrales miden
los efectos acumulados. Para dominar estas ideas, primero necesitamos
una comprensión sólida de qué son las funciones y cómo se comportan.

\subsubsection{Ejercicios}\label{ejercicios}

\begin{enumerate}
\def\labelenumi{\arabic{enumi}.}
\item
  Para la función \(f(x) = 3x - 2\):- Encuentra el dominio, codominio y
  rango.
\item
  ¿Para qué entradas está definida la función \(h(x) = \sqrt{x-1}\)?
  ¿Cuál es su alcance?
\item
  Da un ejemplo del mundo real de una función de tu vida diaria. Indique
  claramente el dominio y el codominio.
\item
  Dibuja la gráfica de \(f(x) = |x|\). ¿Cuál es el rango?
\item
  Supongamos \(g(x) = \dfrac{1}{x^2+1}\). Explica por qué su rango es el
  intervalo \((0, 1]\).
\end{enumerate}

\subsection{1.2 Gráficos y
Transformaciones}\label{gruxe1ficos-y-transformaciones}

Una función puede entenderse no sólo por fórmulas sino también por su
gráfica. La gráfica de una función \(f\) es el conjunto de todos los
pares ordenados \((x, f(x))\), donde \(x\) pertenece al dominio de
\(f\). Trazar estos pares en el plano de coordenadas da una idea de cómo
se comporta la función.

\subsubsection{Gráficos básicos}\label{gruxe1ficos-buxe1sicos}

Algunos gráficos son tan fundamentales que conviene memorizarlos:

\begin{itemize}
\tightlist
\item
  \(f(x) = x\): recta que pasa por el origen.
\item
  \(f(x) = x^2\): una parábola que se abre hacia arriba.
\item
  \(f(x) = |x|\): gráfico en forma de ``V''.
\item
  \(f(x) = \frac{1}{x}\): una hipérbola con dos ramas.
\item
  \(f(x) = \sin x\): una curva periódica en forma de onda.
\end{itemize}

Estos sirven como componentes básicos para funciones más complicadas.

\subsubsection{Transformaciones}\label{transformaciones}

Los gráficos se pueden desplazar, estirar o reflejar usando reglas
simples:

\begin{enumerate}
\def\labelenumi{\arabic{enumi}.}
\item
  Desplazamientos verticales: agregar una constante mueve el gráfico
  hacia arriba o hacia abajo.

  \[
  y = f(x) + c \quad \text{is } f(x) \text{ shifted upward by } c.
  \]
\item
  Desplazamientos horizontales: Agregar dentro del argumento mueve el
  gráfico hacia la izquierda o hacia la derecha.

  \[
  y = f(x - c) \quad \text{is } f(x) \text{ shifted right by } c.
  \]
\item
  Escala vertical: multiplicar por una constante estira o comprime el
  gráfico verticalmente.

  \[
  y = a f(x), \quad a > 1 \text{ stretches; } 0 < a < 1 \text{ compresses.}
  \]
\item
  Escala horizontal: multiplicar dentro del argumento estira o comprime
  el gráfico horizontalmente.

  \[
  y = f(bx), \quad b > 1 \text{ compresses toward the } y\text{-axis}.
  \]
\item
  Reflexiones:

  \begin{itemize}
  \tightlist
  \item
    \(y = -f(x)\): reflexión sobre el eje \(x\).
  \item
    \(y = f(-x)\): reflexión sobre el eje \(y\).
  \end{itemize}
\end{enumerate}

\subsubsection{Combinando
transformaciones}\label{combinando-transformaciones}

Los gráficos complejos a menudo surgen de la combinación de varias
transformaciones en secuencia. Por ejemplo:

\[
y = 2(x-1)^2 + 3
\]

se obtiene tomando la parábola \(y = x^2\), desplazándola 1 hacia la
derecha, estirándola verticalmente 2 y desplazándola hacia arriba 3.

\subsubsection{Ejercicios}\label{ejercicios-1}

\begin{enumerate}
\def\labelenumi{\arabic{enumi}.}
\tightlist
\item
  Dibuja la gráfica de \(y = (x+2)^2 - 1\). Identifica la secuencia de
  transformaciones de \(y = x^2\).
\item
  ¿Qué pasa con la gráfica de \(y = f(x)\) si reemplazamos \(x\) por
  \(-x\)? Pruébalo con \(f(x) = \sqrt{x}\).
\item
  Describe las transformaciones que convierten \(y = \sin x\) en
  \(y = 3\sin(x - \pi/4)\).4. Dibuja la gráfica de \(y = |x-1| + 2\).
  Indique su vértice y pendiente de cada rama.
\item
  Para \(y = \frac{1}{x-2}\), explica cómo se ha transformado la gráfica
  de \(y = \frac{1}{x}\).
\end{enumerate}

\subsection{1.3 Idea intuitiva de
límites}\label{idea-intuitiva-de-luxedmites}

En muchas situaciones, el valor de una función en un punto es menos
importante que los valores que toma cerca de ese punto. El concepto de
límite captura esta idea.

\subsubsection{Acercándose a un valor}\label{acercuxe1ndose-a-un-valor}

Imagínese caminar hacia una pared. Incluso antes de tocarlo, te acercas
cada vez más. De la misma manera, cuando \(x\) se acerca a un número
\(a\), los valores de \(f(x)\) pueden acercarse a algún número \(L\).
Entonces decimos:

\[
\lim_{x \to a} f(x) = L.
\]

Esto expresa la idea de que \(f(x)\) se puede acercar tanto como
queramos a \(L\), simplemente acercando \(x\) lo suficiente a \(a\).

\subsubsection{Ejemplos}\label{ejemplos-1}

\begin{enumerate}
\def\labelenumi{\arabic{enumi}.}
\item
  Por \(f(x) = 2x + 3\): Como \(x \to 1\), \(f(x) \to 5\).
\item
  Por \(f(x) = \dfrac{\sin x}{x}\): Como \(x \to 0\), la función tiende
  a 1, aunque \(f(0)\) no esté definida.
\item
  Por \(f(x) = \dfrac{1}{x}\): Como \(x \to 0^+\) (acercándose por la
  derecha), \(f(x) \to +\infty\). Como \(x \to 0^-\) (acercándose desde
  la izquierda), \(f(x) \to -\infty\). Dado que los comportamientos
  izquierdo y derecho difieren, el límite en 0 no existe.
\end{enumerate}

\subsubsection{Importancia de los
límites}\label{importancia-de-los-luxedmites}

\begin{itemize}
\tightlist
\item
  Nos permiten definir funciones en puntos donde originalmente no están
  definidas.
\item
  Captan comportamientos cercanos a discontinuidades y singularidades.
\item
  Forman la base de las derivadas (tasas de cambio instantáneas) y las
  integrales (áreas como límites de sumas).
\end{itemize}

\subsubsection{Límites unilaterales}\label{luxedmites-unilaterales}

A veces el comportamiento de la izquierda y de la derecha debe
estudiarse por separado:

\[
\lim_{x \to a^-} f(x), \quad \lim_{x \to a^+} f(x).
\]

Si ambos están de acuerdo, entonces existe el límite bilateral.

\subsubsection{Ejercicios}\label{ejercicios-2}

\begin{enumerate}
\def\labelenumi{\arabic{enumi}.}
\tightlist
\item
  Calcular \(\lim_{x \to 2} (3x^2 - x)\).
\item
  ¿Qué es \(\lim_{x \to 0} \frac{\sin x}{x}\)? Usa la intuición de la
  gráfica de \(\sin x\).
\item
  Valorar \(\lim_{x \to 0} |x|/x\). ¿Existe el límite bilateral?
\item
  Calcula \(\lim_{x \to \infty} \frac{1}{x}\). Interprete este resultado
  en palabras.
\item
  Para \(f(x) = \frac{x^2-1}{x-1}\), ¿cuánto es \(\lim_{x \to 1} f(x)\)?
  Comparar con el valor de \(f(1)\).
\end{enumerate}

\subsection{1.4 Definición formal de
límites}\label{definiciuxf3n-formal-de-luxedmites}

La idea intuitiva de límite se puede precisar utilizando la definición
épsilon-delta. Esto nos da una forma rigurosa de decir que \(f(x)\) se
acerca a un valor \(L\) a medida que \(x\) se acerca a \(a\).

\subsubsection{La definición}\label{la-definiciuxf3n}

escribimos

\[
\lim_{x \to a} f(x) = L
\]

si se cumple la siguiente condición:

Por cada \(\varepsilon > 0\) (por pequeño que sea), existe un
\(\delta > 0\) tal que siempre que

\[
0 < |x - a| < \delta,
\]

se deduce que

\[
|f(x) - L| < \varepsilon.
\]En palabras: podemos hacer que \(f(x)\) sea lo más cercano que
queramos a \(L\), siempre que \(x\) esté lo suficientemente cerca de
\(a\) (pero no sea igual a \(a\)).

\subsubsection{Ejemplo 1: función
lineal}\label{ejemplo-1-funciuxf3n-lineal}

Para \(f(x) = 2x + 1\), demuestra que \(\lim_{x \to 3} f(x) = 7\).

\begin{itemize}
\tightlist
\item
  Queremos \(|f(x) - 7| < \varepsilon\).
\item
  Pero \(f(x) - 7 = 2x + 1 - 7 = 2(x - 3)\).
\item
  Entonces \(|f(x) - 7| = 2|x - 3|\).
\item
  Si elegimos \(\delta = \varepsilon / 2\), entonces siempre que
  \(|x - 3| < \delta\), tenemos \(|f(x) - 7| < \varepsilon\). Esto
  demuestra el límite.
\end{itemize}

\subsubsection{Ejemplo 2: función
recíproca}\label{ejemplo-2-funciuxf3n-recuxedproca}

Por \(f(x) = \frac{1}{x}\), considere
\(\lim_{x \to 2} f(x) = \tfrac{1}{2}\).

\begin{itemize}
\tightlist
\item
  Queremos \(\left|\frac{1}{x} - \frac{1}{2}\right| < \varepsilon\).
\item
  Esta desigualdad requiere manipulación algebraica, pero se puede
  satisfacer eligiendo \(\delta\) dependiendo de \(\varepsilon\). El
  proceso es más complicado, pero el principio es el mismo.
\end{itemize}

\subsubsection{Por qué esto es
importante}\label{por-quuxe9-esto-es-importante}

\begin{itemize}
\tightlist
\item
  La definición épsilon-delta garantiza que los límites no sean vagos ni
  se basen únicamente en la intuición.
\item
  Es la base de la continuidad, las derivadas y las integrales.
\item
  Aunque los principiantes pueden encontrarlo abstracto, trabajar con
  ejemplos simples genera familiaridad.
\end{itemize}

\subsubsection{Ejercicios}\label{ejercicios-3}

\begin{enumerate}
\def\labelenumi{\arabic{enumi}.}
\tightlist
\item
  Utilizando la definición épsilon-delta, demuestra que
  \(\lim_{x \to 4} (x+1) = 5\).
\item
  Demuestra que \(\lim_{x \to 0} 5x = 0\) usando la definición formal.
\item
  Explica por qué \(\lim_{x \to 0} \frac{1}{x}\) no existe.
\item
  Para \(f(x) = x^2\), demuestra que \(\lim_{x \to 2} f(x) = 4\).
\item
  En tus propias palabras, explica el papel de \(\varepsilon\) y
  \(\delta\) en la definición de límite.
\end{enumerate}

\subsection{1.5 Continuidad}\label{continuidad}

Una función es continua si su gráfica se puede dibujar sin levantar el
lápiz del papel. Más precisamente, la continuidad asegura que pequeños
cambios en la entrada produzcan pequeños cambios en la salida.

\subsubsection{Definición}\label{definiciuxf3n-1}

Una función \(f\) es continua en un punto \(a\) si se cumplen tres
condiciones:

\begin{enumerate}
\def\labelenumi{\arabic{enumi}.}
\tightlist
\item
  Se define \(f(a)\).
\item
  Existe \(\lim_{x \to a} f(x)\).
\item
  \(\lim_{x \to a} f(x) = f(a)\).
\end{enumerate}

Si una función es continua en todo punto de un intervalo, decimos que es
continua en ese intervalo.

\subsubsection{Ejemplos}\label{ejemplos-2}

\begin{enumerate}
\def\labelenumi{\arabic{enumi}.}
\item
  Funciones polinómicas: Funciones como \(f(x) = x^2 + 3x - 5\) son
  continuas en todas partes de \(\mathbb{R}\).
\item
  Funciones racionales: \(f(x) = \frac{1}{x-1}\) es continua en todas
  partes excepto en \(x = 1\), donde no está definida.
\item
  Funciones por partes:

  \[
  f(x) =
  \begin{cases}
  x^2 & x < 1, \\
  2 & x = 1, \\
  x+1 & x > 1,
  \end{cases}
  \]

  Esta función tiene un ``salto'' en \(x = 1\), por lo que no es
  continua allí.
\end{enumerate}

\subsubsection{Tipos de
discontinuidades}\label{tipos-de-discontinuidades}

\begin{enumerate}
\def\labelenumi{\arabic{enumi}.}
\tightlist
\item
  Discontinuidad removible: Un ``agujero'' en el gráfico. Ejemplo:
  \(f(x) = \frac{x^2-1}{x-1}\) a \(x=1\).2. Discontinuidad de salto: los
  límites izquierdo y derecho son diferentes.
\item
  Discontinuidad infinita: La función va a \(\pm\infty\) cerca de un
  punto, como ocurre con \(f(x) = 1/x\) cerca de \(x = 0\).
\end{enumerate}

\subsubsection{El teorema del valor
intermedio}\label{el-teorema-del-valor-intermedio}

Si una función es continua en un intervalo \([a, b]\), entonces para
cualquier número \(N\) entre \(f(a)\) y \(f(b)\), existe algún
\(c \in [a, b]\) tal que \(f(c) = N\).

Esta propiedad es crucial para demostrar la existencia de raíces y
soluciones de ecuaciones.

\subsubsection{Ejercicios}\label{ejercicios-4}

\begin{enumerate}
\def\labelenumi{\arabic{enumi}.}
\tightlist
\item
  Decide si la función \(f(x) = |x|\) es continua en \(x = 0\).
\item
  Identifica los puntos de discontinuidad para
  \(f(x) = \frac{x+2}{x^2-1}\).
\item
  Explica por qué toda función polinómica es continua en todas partes.
\item
  Da un ejemplo de una función con una discontinuidad de salto. Dibuja
  su gráfica.
\item
  Usa el Teorema del Valor Intermedio para demostrar que la ecuación
  \(x^3 + x - 1 = 0\) tiene una solución entre 0 y 1.
\end{enumerate}

\section{Capítulo 2. Derivados}\label{capuxedtulo-2.-derivados}

\subsection{2.1 La derivada como tasa de
cambio}\label{la-derivada-como-tasa-de-cambio}

La derivada es una de las ideas centrales del cálculo. Mide cómo cambia
una función a medida que cambia su entrada; en otras palabras, la tasa
de cambio de la salida con respecto a la entrada.

\subsubsection{Tasa de cambio promedio}\label{tasa-de-cambio-promedio}

Para una función \(f(x)\), la tasa de cambio promedio entre dos puntos
\(x = a\) y \(x = b\) es

\[
\frac{f(b) - f(a)}{b - a}.
\]

Esta es la pendiente de la recta secante que pasa por los puntos
\((a, f(a))\) y \((b, f(b))\).

\subsubsection{Tasa de cambio
instantánea}\label{tasa-de-cambio-instantuxe1nea}

Para medir qué tan rápido cambia \(f(x)\) en un solo punto, dejamos que
el intervalo se reduzca:

\[
f'(a) = \lim_{h \to 0} \frac{f(a+h) - f(a)}{h}.
\]

Este límite, si existe, se llama derivada de \(f\) en \(a\).
Geométricamente es la pendiente de la recta tangente a la gráfica de
\(f\) en el punto \((a, f(a))\).

\subsubsection{Notación}\label{notaciuxf3n}

\begin{itemize}
\tightlist
\item
  \(f'(x)\): notación prima.
\item
  \(\dfrac{dy}{dx}\): Notación de Leibniz, utilizada cuando
  \(y = f(x)\).
\item
  \(Df(x)\): notación de operador.
\end{itemize}

Todos estos símbolos hacen referencia al mismo concepto.

\subsubsection{Ejemplos}\label{ejemplos-3}

\begin{enumerate}
\def\labelenumi{\arabic{enumi}.}
\item
  Por \(f(x) = x^2\):

  \[
  f'(x) = \lim_{h \to 0} \frac{(x+h)^2 - x^2}{h} = \lim_{h \to 0} \frac{2xh + h^2}{h} = 2x.
  \]

  La pendiente de la parábola en \(x\) es \(2x\).
\item
  Por \(f(x) = \sin x\):

  \[
  f'(x) = \cos x.
  \]
\item
  Por \(f(x) = c\) (una constante):

  \[
  f'(x) = 0.
  \]

  Una función constante nunca cambia.
\end{enumerate}

\subsubsection{Interpretación}\label{interpretaciuxf3n}

\begin{itemize}
\tightlist
\item
  En física: Si \(s(t)\) es posición, entonces \(s'(t)\) es velocidad.
\item
  En economía: Si \(C(x)\) es costo, entonces \(C'(x)\) es costo
  marginal.
\item
  En biología: Si \(P(t)\) es población, entonces \(P'(t)\) es tasa de
  crecimiento.
\end{itemize}

La derivada hace que el ``cambio'' sea preciso en muchos contextos.

\subsubsection{Ejercicios}\label{ejercicios-5}

\begin{enumerate}
\def\labelenumi{\arabic{enumi}.}
\tightlist
\item
  Calcular \(f'(x)\) para \(f(x) = 3x^2 - 2x + 1\).2. Calcula la
  pendiente de la recta tangente a \(f(x) = x^3\) en \(x = 2\).
\item
  Si \(s(t) = t^2 + 2t\) representa la distancia en metros, ¿cuál es la
  velocidad en \(t = 5\)?
\item
  Utilice la definición de límite para calcular la derivada de
  \(f(x) = \frac{1}{x}\).
\item
  Dibuja la gráfica de \(y = x^2\) y traza la recta tangente en
  \(x = 1\).
\end{enumerate}

\subsection{2.2 Reglas de
diferenciación}\label{reglas-de-diferenciaciuxf3n}

Una vez definida la derivada, necesitamos formas eficientes de
calcularla. Las reglas de diferenciación son atajos que nos salvan de
aplicar repetidamente la definición de límite.

\subsubsection{La regla constante}\label{la-regla-constante}

Si \(f(x) = c\) donde \(c\) es una constante, entonces

\[
f'(x) = 0.
\]

\subsubsection{La regla del poder}\label{la-regla-del-poder}

Para \(f(x) = x^n\) donde \(n\) es un número real,

\[
\frac{d}{dx} \big( x^n \big) = n x^{n-1}.
\]

Ejemplos:

\begin{itemize}
\tightlist
\item
  \(\frac{d}{dx}(x^2) = 2x\).
\item
  \(\frac{d}{dx}(x^5) = 5x^4\).
\item
  \(\frac{d}{dx}(\sqrt{x}) = \frac{1}{2\sqrt{x}}\).
\end{itemize}

\subsubsection{La regla del múltiplo
constante}\label{la-regla-del-muxfaltiplo-constante}

Si \(f(x) = c \cdot g(x)\), entonces

\[
f'(x) = c \cdot g'(x).
\]

\subsubsection{Las reglas de la suma y la
diferencia}\label{las-reglas-de-la-suma-y-la-diferencia}

\begin{itemize}
\tightlist
\item
  \((f + g)' = f' + g'\).
\item
  \((f - g)' = f' - g'\).
\end{itemize}

\subsubsection{La regla del producto}\label{la-regla-del-producto}

Por \(f(x)\) y \(g(x)\):

\[
(fg)' = f'g + fg'.
\]

Ejemplo: Si \(f(x) = x^2\), \(g(x) = \sin x\):

\[
(fg)' = (2x)(\sin x) + (x^2)(\cos x).
\]

\subsubsection{La regla del cociente}\label{la-regla-del-cociente}

Por \(f(x)\) y \(g(x)\):

\[
\left(\frac{f}{g}\right)' = \frac{f'g - fg'}{g^2}, \quad g(x) \neq 0.
\]

Ejemplo: Si \(f(x) = x^2\), \(g(x) = x+1\):

\[
\left(\frac{x^2}{x+1}\right)' = \frac{(2x)(x+1) - (x^2)(1)}{(x+1)^2}.
\]

\subsubsection{Derivadas de funciones
comunes}\label{derivadas-de-funciones-comunes}

\begin{itemize}
\tightlist
\item
  \(\frac{d}{dx}(\sin x) = \cos x\).
\item
  \(\frac{d}{dx}(\cos x) = -\sin x\).
\item
  \(\frac{d}{dx}(e^x) = e^x\).
\item
  \(\frac{d}{dx}(\ln x) = \frac{1}{x}, \quad x > 0\).
\end{itemize}

\subsubsection{Ejercicios}\label{ejercicios-6}

\begin{enumerate}
\def\labelenumi{\arabic{enumi}.}
\tightlist
\item
  Diferenciar \(f(x) = 7x^3 - 4x + 9\).
\item
  Usa la regla del producto para encontrar la derivada de
  \(f(x) = x^2 e^x\).
\item
  Aplica la regla del cociente a \(f(x) = \frac{\sin x}{x}\).
\item
  Calcula \(\frac{d}{dx}(\ln(x^2))\) usando la cadena de reglas.
\item
  Demuestra que la derivada de \(f(x) = \frac{1}{x}\) es
  \(-\frac{1}{x^2}\).
\end{enumerate}

\subsection{2.3 La regla de la cadena}\label{la-regla-de-la-cadena}

A menudo, las funciones se construyen combinando funciones más simples.
Para diferenciar este tipo de funciones compuestas, utilizamos la regla
de la cadena.

\subsubsection{La regla}\label{la-regla}

Si \(y = f(g(x))\), entonces

\[
\frac{dy}{dx} = f'(g(x)) \cdot g'(x).
\]

En palabras: diferencia la función exterior, mantén la interior sin
cambios y luego multiplica por la derivada de la interior.

\subsubsection{Ejemplos}\label{ejemplos-4}

\begin{enumerate}
\def\labelenumi{\arabic{enumi}.}
\item
  Cuadrado de una función lineal

  \[
  y = (3x+2)^2
  \]

  Función exterior: \(f(u) = u^2\), función interior: \(g(x) = 3x+2\).

  \[
  y' = 2(3x+2) \cdot 3 = 6(3x+2).
  \]
\item
  Exponencial con interior cuadrático

  \[
  y = e^{x^2}
  \]

  Función exterior: \(f(u) = e^u\), función interior: \(g(x) = x^2\).

  \[y' = e^{x^2} \cdot 2x = 2x e^{x^2}.
  \]
\item
  Logarithm with root inside

  \[
  y = \ln(\sqrt{x})
  \]

  Outer: \(f(u) = \ln u\), inner: \(g(x) = \sqrt{x}\).

  \[
  y' = \frac{1}{\sqrt{x}} \cdot \frac{1}{2\sqrt{x}} = \frac{1}{2x}.
  \]
\end{enumerate}

\subsubsection{Generalized Chain Rule}\label{generalized-chain-rule}

For multiple nested functions \(y = f(g(h(x)))\):

\[
\frac{dy}{dx} = f'(g(h(x))) \cdot g'(h(x)) \cdot h'(x).
\]

This extends naturally to deeper compositions.

\subsubsection{Why the Chain Rule
Matters}\label{why-the-chain-rule-matters}

\begin{itemize}
\tightlist
\item
  It handles nearly all real-world models where one quantity depends on
  another indirectly.
\item
  It connects calculus with physics (e.g., velocity depending on time
  through position).
\item
  It is essential in implicit differentiation and advanced topics.
\end{itemize}

\subsubsection{Exercises}\label{exercises}

\begin{enumerate}
\def\labelenumi{\arabic{enumi}.}
\tightlist
\item
  Differentiate \(y = (5x^2 + 1)^3\).
\item
  Find \(\frac{d}{dx}(\sin(3x))\).
\item
  Compute \(\frac{d}{dx}(\ln(1+x^2))\).
\item
  Differentiate \(y = \cos^2(x)\).
\item
  Apply the generalized chain rule to \(y = e^{\sin(x^2)}\).
\end{enumerate}

\subsection{2.4 Implicit
Differentiation}\label{implicit-differentiation}

Not all functions are given in the form \(y = f(x)\). Sometimes \(x\)
and \(y\) are related by an equation, and solving explicitly for \(y\)
is difficult or impossible. In such cases, we use implicit
differentiation.

\subsubsection{The Idea}\label{the-idea}

If an equation involves both \(x\) and \(y\), we can differentiate both
sides with respect to \(x\), treating \(y\) as a function of \(x\). Each
time we differentiate a term involving \(y\), we multiply by
\(\frac{dy}{dx}\).

\subsubsection{Example 1: A Circle}\label{example-1-a-circle}

Equation:

\[
x^2 + y^2 = 25
\]

Differentiate with respect to \(x\):

\[
2x + 2y \frac{dy}{dx} = 0.
\]

Solve for \(\frac{dy}{dx}\):

\[
\frac{dy}{dx} = -\frac{x}{y}.
\]

This gives the slope of the tangent to the circle at any point.

\subsubsection{Example 2: A Product of
Variables}\label{example-2-a-product-of-variables}

Equation:

\[
xy = 1
\]

Differentiate:

\[
x \frac{dy}{dx} + y = 0.
\]

So,

\[
\frac{dy}{dx} = -\frac{y}{x}.
\]

\subsubsection{Example 3: Trigonometric
Relation}\label{example-3-trigonometric-relation}

Equation:

\[
\sin(xy) = x
\]

Differentiate:

\[
\cos(xy) \cdot \Big(y + x\frac{dy}{dx}\Big) = 1.
\]

Solve for \(\frac{dy}{dx}\):

\[
\frac{dy}{dx} = \frac{1 - y\cos(xy)}{x\cos(xy)}.
\]

\subsubsection{Por qué es útil la diferenciación
implícita}\label{por-quuxe9-es-uxfatil-la-diferenciaciuxf3n-impluxedcita}

\begin{itemize}
\tightlist
\item
  Muchas curvas importantes (círculos, elipses, hipérbolas) se definen
  naturalmente de forma implícita.
\item
  Nos permite diferenciar ecuaciones sin resolver primero \(y\).
\item
  Es un paso clave en temas más avanzados como tasas relacionadas y
  ecuaciones diferenciales.
\end{itemize}

\subsubsection{Ejercicios}\label{ejercicios-7}

\begin{enumerate}
\def\labelenumi{\arabic{enumi}.}
\tightlist
\item
  Para la curva \(x^2 + xy + y^2 = 7\), encuentre \(\frac{dy}{dx}\).
\item
  Diferenciar \(\cos(x) + \cos(y) = 1\) implícitamente.
\item
  Calcula la pendiente de la recta tangente a \(x^3 + y^3 = 9\) en el
  punto \((1, 2)\).4. Dado \(x^2 + y^2 = 10\), calcula \(\frac{dy}{dx}\)
  cuando \((x, y) = (1, 3)\).
\item
  Diferencia \(e^{xy} = x + y\) para encontrar \(\frac{dy}{dx}\).
\end{enumerate}

\subsection{2.5 Derivados de orden
superior}\label{derivados-de-orden-superior}

Hasta ahora hemos estudiado la primera derivada, que mide la tasa de
cambio de una función. Pero los propios derivados también pueden
diferenciarse, dando lugar a derivados de orden superior.

\subsubsection{Definición}\label{definiciuxf3n-2}

\begin{itemize}
\item
  La segunda derivada de \(f\) es la derivada de la derivada:

  \[
  f''(x) = \frac{d}{dx}\left(f'(x)\right).
  \]
\item
  De manera más general, la derivada \(n\)-ésima se escribe como

  \[
  f^{(n)}(x) = \frac{d^n}{dx^n} f(x).
  \]
\end{itemize}

\subsubsection{Ejemplos}\label{ejemplos-5}

\begin{enumerate}
\def\labelenumi{\arabic{enumi}.}
\item
  \(f(x) = x^3\)

  \begin{itemize}
  \tightlist
  \item
    Primera derivada: \(f'(x) = 3x^2\).
  \item
    Segunda derivada: \(f''(x) = 6x\).
  \item
    Tercera derivada: \(f^{(3)}(x) = 6\).
  \item
    Cuarta derivada: \(f^{(4)}(x) = 0\).
  \end{itemize}
\item
  \(f(x) = \sin x\)

  \begin{itemize}
  \tightlist
  \item
    \(f'(x) = \cos x\).
  \item
    \(f''(x) = -\sin x\).
  \item
    \(f^{(3)}(x) = -\cos x\).
  \item
    \(f^{(4)}(x) = \sin x\). Las derivadas se repiten en un ciclo de
    longitud 4.
  \end{itemize}
\item
  \(f(x) = e^x\)

  \begin{itemize}
  \tightlist
  \item
    Cada derivada es \(e^x\).
  \end{itemize}
\end{enumerate}

\subsubsection{Aplicaciones}\label{aplicaciones}

\begin{itemize}
\item
  Concavidad: El signo de \(f''(x)\) indica si la gráfica de \(f\) es
  cóncava hacia arriba (\(f'' > 0\)) o hacia abajo (\(f'' < 0\)).
\item
  Puntos de inflexión: Puntos donde \(f''(x) = 0\) y cambia la
  concavidad.
\item
  Movimiento: En física, si \(s(t)\) es posición:

  \begin{itemize}
  \tightlist
  \item
    \(s'(t)\) = velocidad,
  \item
    \(s''(t)\) = aceleración,
  \item
    \(s^{(3)}(t)\) = tirón (tasa de cambio de aceleración).
  \end{itemize}
\item
  Aproximaciones: Las derivadas de orden superior aparecen en las series
  de Taylor, utilizadas para aproximar funciones.
\end{itemize}

\subsubsection{Ejercicios}\label{ejercicios-8}

\begin{enumerate}
\def\labelenumi{\arabic{enumi}.}
\tightlist
\item
  Calcula las primeras cuatro derivadas de \(f(x) = \cos x\).
\item
  Calcula \(f''(x)\) para \(f(x) = x^4 - 2x^2 + 3\).
\item
  Para \(f(x) = e^{2x}\), demuestra que \(f^{(n)}(x) = 2^n e^{2x}\).
\item
  Determina los intervalos donde \(f(x) = x^3 - 3x\) es cóncavo hacia
  arriba y cóncavo hacia abajo.
\item
  Si \(s(t) = t^3 - 6t^2 + 9t\), encuentre la velocidad y la aceleración
  en \(t = 2\).
\end{enumerate}

\section{Capítulo 3. Aplicaciones de los
Derivados}\label{capuxedtulo-3.-aplicaciones-de-los-derivados}

\subsection{3.1 Tangentes y Normales}\label{tangentes-y-normales}

Una de las primeras aplicaciones de las derivadas es encontrar las
ecuaciones de las rectas tangentes y normales a una curva. Estas líneas
capturan la geometría local de una función en un punto dado.

\subsubsection{Línea tangente}\label{luxednea-tangente}

La recta tangente a una curva \(y = f(x)\) en un punto \((a, f(a))\) es
la recta que simplemente ``toca'' la gráfica allí y tiene la misma
pendiente que la curva.

La pendiente de la recta tangente viene dada por la derivada:

\[
m_{\text{tangent}} = f'(a).
\]

Así, la ecuación de la recta tangente en \((a, f(a))\) es

\[
y - f(a) = f'(a)(x - a).
\]

\subsubsection{Línea normal}\label{luxednea-normal}

La recta normal es perpendicular a la recta tangente en el mismo punto.
Su pendiente es el recíproco negativo de la pendiente tangente:

\[m_{\text{normal}} = -\frac{1}{f'(a)}.
\]

So the equation of the normal line is

\[
y - f(a) = -\frac{1}{f'(a)} (x - a), \quad f'(a) \neq 0.
\]

\subsubsection{Examples}\label{examples}

\begin{enumerate}
\def\labelenumi{\arabic{enumi}.}
\item
  \(f(x) = x^2\) at \(x = 1\).

  \begin{itemize}
  \tightlist
  \item
    \(f(1) = 1\), \(f'(x) = 2x\), so \(f'(1) = 2\).
  \item
    Tangent: \(y - 1 = 2(x - 1)\), or \(y = 2x - 1\).
  \item
    Normal: slope = \(-\tfrac{1}{2}\), so equation is
    \(y - 1 = -\tfrac{1}{2}(x - 1)\).
  \end{itemize}
\item
  \(f(x) = \sin x\) at \(x = \tfrac{\pi}{4}\).

  \begin{itemize}
  \tightlist
  \item
    \(f(\tfrac{\pi}{4}) = \tfrac{\sqrt{2}}{2}\),
    \(f'(\tfrac{\pi}{4}) = \cos(\tfrac{\pi}{4}) = \tfrac{\sqrt{2}}{2}\).
  \item
    Tangent:
    \(y - \tfrac{\sqrt{2}}{2} = \tfrac{\sqrt{2}}{2}(x - \tfrac{\pi}{4})\).
  \end{itemize}
\end{enumerate}

\subsubsection{Why Tangents and Normals
Matter}\label{why-tangents-and-normals-matter}

\begin{itemize}
\tightlist
\item
  Tangents approximate the curve locally (linear approximation).
\item
  Normals are useful in geometry, optics (reflection/refraction), and
  mechanics (force directions).
\item
  Both play a role in optimization and curvature studies.
\end{itemize}

\subsubsection{Exercises}\label{exercises-1}

\begin{enumerate}
\def\labelenumi{\arabic{enumi}.}
\tightlist
\item
  Find the tangent and normal lines to \(y = x^3\) at \(x = 2\).
\item
  Determine the tangent and normal lines to \(y = e^x\) at \(x = 0\).
\item
  For \(y = \ln x\), compute the tangent line at \(x = 1\).
\item
  A circle is given by \(x^2 + y^2 = 9\). Use implicit differentiation
  to find the slope of the tangent at \((0,3)\).
\item
  Sketch the graph of \(y = \sqrt{x}\) and draw the tangent and normal
  lines at \(x = 4\).
\end{enumerate}

\subsection{3.2 Related Rates}\label{related-rates}

In many real-world problems, two or more quantities change with respect
to time, and their rates of change are connected. Related rates problems
use derivatives to describe these relationships.

\subsubsection{General Approach}\label{general-approach}

\begin{enumerate}
\def\labelenumi{\arabic{enumi}.}
\tightlist
\item
  Identify the variables that depend on time \(t\).
\item
  Write an equation relating the variables.
\item
  Differentiate both sides with respect to \(t\), applying the chain
  rule.
\item
  Substitute the known values at the given instant.
\item
  Solve for the unknown rate.
\end{enumerate}

\subsubsection{Example 1: Expanding
Circle}\label{example-1-expanding-circle}

A circle has radius \(r\), which increases at the rate of
\(\frac{dr}{dt} = 2 \,\text{cm/s}\). Find the rate at which the area
\(A = \pi r^2\) increases when \(r = 5\).

Differentiate:

\[
\frac{dA}{dt} = 2\pi r \frac{dr}{dt}.
\]

Substitute:

\[
\frac{dA}{dt} = 2\pi (5)(2) = 20\pi \,\text{cm}^2/\text{s}.
\]

\subsubsection{Example 2: Sliding
Ladder}\label{example-2-sliding-ladder}

A 10 ft ladder leans against a wall. The bottom slides away at
\(\frac{dx}{dt} = 1 \,\text{ft/s}\). How fast is the top sliding down
when the bottom is 6 ft from the wall?

Equation: \(x^2 + y^2 = 100\), where \(y\) is the height.

Differentiate:

\[
2x \frac{dx}{dt} + 2y \frac{dy}{dt} = 0.
\]

At \(x = 6\), \(y = 8\). Substitute:

\[
2(6)(1) + 2(8)\frac{dy}{dt} = 0 \quad \Rightarrow \quad \frac{dy}{dt} = -\tfrac{6}{8} = -\tfrac{3}{4}.
\]Entonces la parte superior se desliza hacia abajo en
\(0.75 \,\text{ft/s}\).

\subsubsection{Ejemplo 3: Agua en un
cono}\label{ejemplo-3-agua-en-un-cono}

Se vierte agua en un cono de 12 cm de altura y 6 cm de radio. Cuando el
agua tiene 4 cm de profundidad, el nivel del agua sube a
\(2 \,\text{cm/s}\). ¿A qué velocidad aumenta el volumen?

Ecuación: \(V = \tfrac{1}{3}\pi r^2 h\). Usando similitud,
\(r = \tfrac{h}{2}\). Sustituyendo:

\[
V = \tfrac{1}{12}\pi h^3.
\]

Diferenciar:

\[
\frac{dV}{dt} = \tfrac{1}{4}\pi h^2 \frac{dh}{dt}.
\]

A \(h = 4\), \(\frac{dh}{dt} = 2\):

\[
\frac{dV}{dt} = \tfrac{1}{4}\pi (16)(2) = 8\pi \,\text{cm}^3/\text{s}.
\]

\subsubsection{Por qué son importantes las tarifas
relacionadas}\label{por-quuxe9-son-importantes-las-tarifas-relacionadas}

\begin{itemize}
\tightlist
\item
  Describen el movimiento y el cambio en física, ingeniería y biología.
\item
  Relacionan la geometría con el cálculo mediante procesos dependientes
  del tiempo.
\item
  Nos entrenan para modelar matemáticamente sistemas dinámicos.
\end{itemize}

\subsubsection{Ejercicios}\label{ejercicios-9}

\begin{enumerate}
\def\labelenumi{\arabic{enumi}.}
\tightlist
\item
  Se infla un globo de modo que su radio aumenta en
  \(0.5 \,\text{cm/s}\). Encuentre qué tan rápido aumenta su volumen
  cuando el radio es de 10 cm.
\item
  Un coche circula hacia el norte a 40 km/h y otro hacia el este a 30
  km/h. ¿A qué velocidad aumenta la distancia entre ellos 2 horas
  después?
\item
  Un foco a 20 m de una pared ilumina a un hombre de 2 m de altura que
  se aleja a 1,5 m/s. ¿Qué tan rápido cambia la longitud de su sombra en
  la pared cuando está a 5 m de la luz?
\item
  La longitud de los lados de un cubo crece a 2 cm/s. ¿A qué velocidad
  aumenta el área de la superficie cuando el lado mide 3 cm?
\item
  Se vierte arena sobre un montón formando un cono de radio siempre
  igual a la altura. Si la altura aumenta a 5 cm/s, ¿a qué tasa aumenta
  el volumen cuando la altura es de 10 cm?
\end{enumerate}

\subsection{3.3 Problemas de
optimización}\label{problemas-de-optimizaciuxf3n}

Los problemas de optimización utilizan derivadas para encontrar los
valores máximos o mínimos de una función, a menudo bajo ciertas
restricciones. Estos problemas modelan situaciones en las que queremos
maximizar la eficiencia, las ganancias o el área, o minimizar el costo,
la distancia o el tiempo.

\subsubsection{Pasos generales}\label{pasos-generales}

\begin{enumerate}
\def\labelenumi{\arabic{enumi}.}
\tightlist
\item
  Comprenda el problema: identifique la cantidad a optimizar.
\item
  Modelo con una función: Escribe la función objetivo en términos de una
  variable.
\item
  Aplicar restricciones: utilizar condiciones dadas para reducir las
  variables.
\item
  Derivar: Calcular la derivada de la función objetivo.
\item
  Encuentra puntos críticos: Resuelve \(f'(x) = 0\) o donde \(f'(x)\) no
  está definido.
\item
  Prueba de máximos/mínimos: utilice la prueba de la segunda derivada o
  verifique los puntos finales.
\item
  Interpretar el resultado: Indique la respuesta en el contexto
  original.
\end{enumerate}

\subsubsection{Ejemplo 1: Área máxima de un
rectángulo}\label{ejemplo-1-uxe1rea-muxe1xima-de-un-rectuxe1ngulo}

Un rectángulo tiene perímetro 40. ¿Qué dimensiones maximizan su área?

\begin{itemize}
\tightlist
\item
  Sea largo \(x\), ancho \(y\). Restricción:
  \(2x + 2y = 40 \Rightarrow y = 20 - x\).
\item
  Superficie: \(A = xy = x(20 - x) = 20x - x^2\).- Derivado:
  \(A'(x) = 20 - 2x\). Establecer igual a 0: \(x = 10\).
\item
  Entonces \(y = 10\).
\item
  Superficie máxima: \(100\). El rectángulo es un cuadrado.
\end{itemize}

\subsubsection{Ejemplo 2: Minimizar la
distancia}\label{ejemplo-2-minimizar-la-distancia}

Encuentra el punto de la parábola \(y = x^2\) más cercano a \((0,3)\).

\begin{itemize}
\tightlist
\item
  Distancia al cuadrado: \(D(x) = (x-0)^2 + (x^2 - 3)^2\).
\item
  Ampliar:
  \(D(x) = x^2 + (x^2 - 3)^2 = x^2 + x^4 - 6x^2 + 9 = x^4 - 5x^2 + 9\).
\item
  Derivado: \(D'(x) = 4x^3 - 10x\). Resuelve: \(x(4x^2 - 10) = 0\).
\item
  Soluciones: \(x = 0\), \(x = \pm \sqrt{2.5}\).
\item
  La comprobación da la distancia mínima en \(x = \pm \sqrt{2.5}\).
\end{itemize}

\subsubsection{Ejemplo 3: Caja con volumen
máximo}\label{ejemplo-3-caja-con-volumen-muxe1ximo}

Se va a hacer una caja sin tapa con un trozo de cartón cuadrado de 20 cm
de lado, cortando cuadrados iguales de las esquinas y doblando los
lados. Encuentra el tamaño del corte que maximiza el volumen.

\begin{itemize}
\tightlist
\item
  Dejar tamaño de corte = \(x\). Luego dimensiones:
  \((20 - 2x) \times (20 - 2x) \times x\).
\item
  Volumen: \(V(x) = x(20 - 2x)^2\).
\item
  Derivado: \(V'(x) = (20 - 2x)(20 - 6x)\).
\item
  Puntos críticos: \(x = 10\) (da volumen cero) o
  \(x = \tfrac{20}{6} \approx 3.33\).
\item
  A \(x \approx 3.33\), se maximiza el volumen.
\end{itemize}

\subsubsection{Por qué es importante la
optimización}\label{por-quuxe9-es-importante-la-optimizaciuxf3n}

\begin{itemize}
\tightlist
\item
  Los ingenieros lo utilizan para diseñar estructuras eficientes.
\item
  Las empresas lo utilizan para maximizar las ganancias o minimizar los
  costos.
\item
  Los científicos lo utilizan para modelar sistemas naturales que buscan
  el equilibrio.
\end{itemize}

\subsubsection{Ejercicios}\label{ejercicios-10}

\begin{enumerate}
\def\labelenumi{\arabic{enumi}.}
\tightlist
\item
  Un agricultor tiene 100 m de cerca para cercar un campo rectangular a
  lo largo de un río (por lo que solo se necesitan cercas en 3 lados).
  Encuentre dimensiones que maximicen el área.
\item
  Encuentra dos números positivos cuya suma sea 20 y cuyo producto sea
  lo más grande posible.
\item
  Se va a fabricar un cilindro con 100 cm\(^2\) de material. Encuentre
  las dimensiones del volumen máximo.
\item
  Un cable de 10 m de largo se corta en dos trozos, uno se dobla
  formando un cuadrado y el otro formando un círculo. ¿Cómo se debe
  cortar para maximizar el área total encerrada?
\item
  Se va a construir una caja cerrada de base cuadrada y volumen 32
  m\(^3\). Encuentre dimensiones minimizando el área de superficie.
\end{enumerate}

\subsection{3.4 Concavidad y puntos de
inflexión}\label{concavidad-y-puntos-de-inflexiuxf3n}

Las derivadas no sólo nos informan sobre las pendientes sino también
sobre la forma de una gráfica. La segunda derivada es especialmente útil
para comprender la concavidad e identificar puntos de inflexión.

\subsubsection{Concavidad}\label{concavidad}

\begin{itemize}
\item
  Una función \(f(x)\) es cóncava hacia arriba en un intervalo si
  \(f''(x) > 0\). El gráfico se inclina hacia arriba, como una taza.
\item
  Una función \(f(x)\) es cóncava hacia abajo en un intervalo si
  \(f''(x) < 0\). El gráfico se inclina hacia abajo, como si frunciera
  el ceño.
\end{itemize}

La concavidad describe cómo cambia la pendiente de una función: si las
pendientes aumentan, la gráfica es cóncava hacia arriba; si las
pendientes disminuyen, la gráfica es cóncava hacia abajo.

\subsubsection{Puntos de inflexión}\label{puntos-de-inflexiuxf3n}

Un punto de inflexión es un punto del gráfico donde cambia la
concavidad.- Si \(f''(x) = 0\) o \(f''(x)\) no está definido, el punto
es candidato a punto de inflexión. - Para confirmar, la concavidad debe
cambiar de signo a ambos lados de la punta.

\subsubsection{Ejemplos}\label{ejemplos-6}

\begin{enumerate}
\def\labelenumi{\arabic{enumi}.}
\item
  \(f(x) = x^3\)

  \begin{itemize}
  \tightlist
  \item
    \(f''(x) = 6x\).
  \item
    A \(x = 0\), \(f''(0) = 0\).
  \item
    Para \(x < 0\), \(f''(x) < 0\) → cóncavo hacia abajo.
  \item
    Para \(x > 0\), \(f''(x) > 0\) → cóncavo hacia arriba.
  \item
    Por tanto, \((0,0)\) es un punto de inflexión.
  \end{itemize}
\item
  \(f(x) = x^4\)

  \begin{itemize}
  \tightlist
  \item
    \(f''(x) = 12x^2\).
  \item
    En \(x = 0\), \(f''(0) = 0\), pero la concavidad no cambia de signo
    (siempre ≥ 0).
  \item
    Sin punto de inflexión.
  \end{itemize}
\end{enumerate}

\subsubsection{Bosquejo de concavidades y
curvas}\label{bosquejo-de-concavidades-y-curvas}

\begin{itemize}
\tightlist
\item
  Si \(f'(x) = 0\) y \(f''(x) > 0\), entonces \(f\) tiene un mínimo
  local.
\item
  Si \(f'(x) = 0\) y \(f''(x) < 0\), entonces \(f\) tiene un máximo
  local.
\item
  Esto se conoce como prueba de la segunda derivada.
\end{itemize}

\subsubsection{Por qué esto es
importante}\label{por-quuxe9-esto-es-importante-1}

La concavidad y los puntos de inflexión nos ayudan a comprender la
``forma'' de las gráficas: dónde se doblan, aplanan o giran. Estas ideas
son fundamentales en el trazado de curvas, la física (aceleración) y la
economía (rendimientos decrecientes).

\subsubsection{Ejercicios}\label{ejercicios-11}

\begin{enumerate}
\def\labelenumi{\arabic{enumi}.}
\tightlist
\item
  Determinar intervalos de concavidad para \(f(x) = x^3 - 3x\).
  Encuentra sus puntos de inflexión.
\item
  Para \(f(x) = \ln(x)\), identifique la concavidad y los posibles
  puntos de inflexión.
\item
  Aplicar la prueba de la segunda derivada a \(f(x) = x^2 e^{-x}\) para
  clasificar los puntos críticos.
\item
  Dibujar \(f(x) = \sin x\), marcando intervalos de concavidad y puntos
  de inflexión.
\item
  Explica por qué \(f(x) = e^x\) no tiene puntos de inflexión.
\end{enumerate}

\subsection{3.5 Croquis de curvas}\label{croquis-de-curvas}

El trazado de curvas es el proceso de dibujar la gráfica de una función
utilizando información de sus derivadas. En lugar de trazar muchos
puntos, analizamos características clave: intersecciones, asíntotas,
intervalos crecientes/decrecientes y concavidad.

\subsubsection{Pasos para dibujar
curvas}\label{pasos-para-dibujar-curvas}

\begin{enumerate}
\def\labelenumi{\arabic{enumi}.}
\item
  Dominio: Identifique dónde está definida la función.
\item
  Intersecciones: Encuentra dónde la gráfica cruza los ejes.
\item
  Asíntotas:

  \begin{itemize}
  \tightlist
  \item
    Las asíntotas verticales ocurren cuando la función no está definida
    y tiende al infinito.
  \item
    Las asíntotas horizontales o inclinadas describen el comportamiento
    final como \(x \to \pm\infty\).
  \end{itemize}
\item
  Primera derivada \(f'(x)\):

  \begin{itemize}
  \tightlist
  \item
    Positivo → la función está aumentando.
  \item
    Negativo → la función es decreciente.
  \item
    Ceros de \(f'(x)\) → puntos críticos (máximos/mínimos posibles).
  \end{itemize}
\item
  Segunda derivada \(f''(x)\):

  \begin{itemize}
  \tightlist
  \item
    Positivo → cóncavo hacia arriba.
  \item
    Negativo → cóncavo hacia abajo.
  \item
    Ceros o indefinidos → posibles puntos de inflexión.
  \end{itemize}
\item
  Combine información: utilice todos los resultados para dibujar un
  gráfico claro y preciso.
\end{enumerate}

\subsubsection{\texorpdfstring{Ejemplo 1:
\(f(x) = x^3 - 3x\)}{Ejemplo 1: f(x) = x\^{}3 - 3x}}\label{ejemplo-1-fx-x3---3x}

\begin{itemize}
\item
  Dominio: todos los números reales.
\item
  Intercepciones: a \((0,0)\).
\item
  Derivado: \(f'(x) = 3x^2 - 3 = 3(x-1)(x+1)\).

  \begin{itemize}
  \item
    Creciente: \((-\infty, -1) \cup (1, \infty)\).
  \item
    Decreciente: \((-1, 1)\).- Segunda derivada: \(f''(x) = 6x\).
  \item
    Cóncavo hacia abajo por \(x < 0\), cóncavo hacia arriba por
    \(x > 0\).
  \item
    Punto de inflexión en \((0,0)\).
  \end{itemize}
\item
  Forma: una curva en S con un máximo local en \((-1, 2)\), un mínimo
  local en \((1, -2)\).
\end{itemize}

\subsubsection{\texorpdfstring{Ejemplo 2:
\(f(x) = \frac{1}{x}\)}{Ejemplo 2: f(x) = \textbackslash frac\{1\}\{x\}}}\label{ejemplo-2-fx-frac1x}

\begin{itemize}
\item
  Dominio: \(x \neq 0\).
\item
  Asíntota vertical: \(x = 0\).
\item
  Asíntota horizontal: \(y = 0\).
\item
  Derivado: \(f'(x) = -\frac{1}{x^2}\) (siempre negativo). La función
  siempre es decreciente.
\item
  Segunda derivada: \(f''(x) = \frac{2}{x^3}\).

  \begin{itemize}
  \tightlist
  \item
    Cóncavo hacia arriba por \(x > 0\).
  \item
    Cóncavo hacia abajo por \(x < 0\).
  \end{itemize}
\item
  Gráfica: hipérbola con dos ramas.
\end{itemize}

\subsubsection{Por qué es útil dibujar
curvas}\label{por-quuxe9-es-uxfatil-dibujar-curvas}

\begin{itemize}
\tightlist
\item
  Proporciona información sobre el comportamiento general de funciones
  sin un cálculo exhaustivo.
\item
  Imprescindible en exámenes de cálculo y problemas aplicados.
\item
  Une el análisis algebraico y la comprensión geométrica.
\end{itemize}

\subsubsection{Ejercicios}\label{ejercicios-12}

\begin{enumerate}
\def\labelenumi{\arabic{enumi}.}
\tightlist
\item
  Dibuja la curva de \(f(x) = x^4 - 2x^2\). Identificar máximos, mínimos
  y puntos de inflexión.
\item
  Analiza y dibuja \(f(x) = \ln(x)\). Muestre intersecciones, asíntotas
  y concavidad.
\item
  Para \(f(x) = e^{-x}\), describa crecimiento/decadencia, asíntotas y
  concavidad.
\item
  Dibuja la gráfica de \(f(x) = \tan x\) en el intervalo
  \((- \pi, \pi)\). Marque asíntotas.
\item
  Utilice las pruebas de la primera y segunda derivada para clasificar
  los puntos críticos de \(f(x) = x^3 - 6x^2 + 9x\).
\end{enumerate}

\section{Parte II. Integrales}\label{parte-ii.-integrales}

\section{Capítulo 4. Antiderivadas e Integrales
Definidas}\label{capuxedtulo-4.-antiderivadas-e-integrales-definidas}

\subsection{4.1 Integrales Indefinidas}\label{integrales-indefinidas}

Una integral indefinida es el proceso inverso de diferenciación. Si una
derivada mide el cambio, entonces una integral recupera la función
original a partir de su tasa de cambio.

\subsubsection{Definición}\label{definiciuxf3n-3}

Si \(F'(x) = f(x)\), entonces

\[
\int f(x)\,dx = F(x) + C,
\]

donde \(C\) es la constante de integración.

Cada integral indefinida representa una familia de funciones que
difieren sólo por una constante, ya que la diferenciación elimina
constantes.

\subsubsection{Reglas básicas}\label{reglas-buxe1sicas}

\begin{enumerate}
\def\labelenumi{\arabic{enumi}.}
\tightlist
\item
  Regla constante
\end{enumerate}

\[
\int c\,dx = cx + C.
\]

\begin{enumerate}
\def\labelenumi{\arabic{enumi}.}
\setcounter{enumi}{1}
\tightlist
\item
  Regla de poder
\end{enumerate}

\[
\int x^n\,dx = \frac{x^{n+1}}{n+1} + C, \quad n \neq -1.
\]

\begin{enumerate}
\def\labelenumi{\arabic{enumi}.}
\setcounter{enumi}{2}
\tightlist
\item
  Regla de la suma
\end{enumerate}

\[
\int \big(f(x) + g(x)\big)\,dx = \int f(x)\,dx + \int g(x)\,dx.
\]

\begin{enumerate}
\def\labelenumi{\arabic{enumi}.}
\setcounter{enumi}{3}
\tightlist
\item
  Regla múltiple constante
\end{enumerate}

\[
\int c f(x)\,dx = c \int f(x)\,dx.
\]

\subsubsection{Integrales comunes}\label{integrales-comunes}

\begin{itemize}
\tightlist
\item
  \(\int e^x dx = e^x + C\)
\item
  \(\int \sin x dx = -\cos x + C\)
\item
  \(\int \cos x dx = \sin x + C\)
\item
  \(\int \frac{1}{x} dx = \ln|x| + C\)
\end{itemize}

\subsubsection{Ejemplos}\label{ejemplos-7}

\begin{enumerate}
\def\labelenumi{\arabic{enumi}.}
\item
  \(\int (3x^2 - 4)\,dx = x^3 - 4x + C\).
\item
  \(\int \cos(2x)\,dx = \tfrac{1}{2}\sin(2x) + C\).
\item
  \(\int \frac{1}{x}\,dx = \ln|x| + C\).
\end{enumerate}

\subsubsection{Interpretación}\label{interpretaciuxf3n-1}

\begin{itemize}
\tightlist
\item
  Las integrales indefinidas son antiderivadas.
\item
  Son la base de las integrales definidas, que miden cantidades
  acumuladas como área, distancia y masa.- En contextos aplicados, la
  integración nos permite pasar de las tasas a los totales.
\end{itemize}

\subsubsection{Ejercicios}\label{ejercicios-13}

\begin{enumerate}
\def\labelenumi{\arabic{enumi}.}
\tightlist
\item
  Encuentra \(\int (5x^4 + 2x)\,dx\).
\item
  Calcula \(\int (e^x + 3)\,dx\).
\item
  Encuentra la solución general de \(f'(x) = 6x\) usando integración.
\item
  Evalúe \(\int \frac{2}{x}\,dx\).
\item
  Si la velocidad es \(v(t) = 4t\), encuentre la función de posición
  \(s(t)\).
\end{enumerate}

\subsection{4.2 La integral definida como
área}\label{la-integral-definida-como-uxe1rea}

Mientras que las integrales indefinidas representan familias de
primitivas, la integral definida da un valor numérico: el área acumulada
bajo una curva entre dos puntos.

\subsubsection{Definición}\label{definiciuxf3n-4}

Para una función \(f(x)\) definida en \([a, b]\), la integral definida
es

\[
\int_a^b f(x)\,dx = \lim_{n \to \infty} \sum_{i=1}^n f(x_i^-) \,\Delta x,
\]

donde el intervalo \([a, b]\) se divide en \(n\) subintervalos de ancho
\(\Delta x\), y \(x_i^-\) es un punto de muestra en cada subintervalo.

Este es el límite de las sumas de Riemann.

\subsubsection{Interpretación
geométrica}\label{interpretaciuxf3n-geomuxe9trica}

\begin{itemize}
\tightlist
\item
  Si \(f(x) \geq 0\) sobre \([a, b]\), entonces \(\int_a^b f(x)\,dx\) es
  igual al área bajo la curva \(y = f(x)\) de \(x=a\) a \(x=b\).
\item
  Si \(f(x)\) cae por debajo del eje \(x\), la integral calcula el área
  con signo: las regiones debajo del eje cuentan como negativas.
\end{itemize}

\subsubsection{Propiedades de la integral
definida}\label{propiedades-de-la-integral-definida}

\begin{enumerate}
\def\labelenumi{\arabic{enumi}.}
\tightlist
\item
  Aditividad en intervalos
\end{enumerate}

\[
\int_a^c f(x)\,dx = \int_a^b f(x)\,dx + \int_b^c f(x)\,dx.
\]

\begin{enumerate}
\def\labelenumi{\arabic{enumi}.}
\setcounter{enumi}{1}
\tightlist
\item
  Límites de inversión
\end{enumerate}

\[
\int_a^b f(x)\,dx = -\int_b^a f(x)\,dx.
\]

\begin{enumerate}
\def\labelenumi{\arabic{enumi}.}
\setcounter{enumi}{2}
\tightlist
\item
  Intervalo de ancho cero
\end{enumerate}

\[
\int_a^a f(x)\,dx = 0.
\]

\begin{enumerate}
\def\labelenumi{\arabic{enumi}.}
\setcounter{enumi}{3}
\tightlist
\item
  Linealidad
\end{enumerate}

\[
\int_a^b \big( cf(x) + g(x)\big)\,dx = c\int_a^b f(x)\,dx + \int_a^b g(x)\,dx.
\]

\subsubsection{Ejemplos}\label{ejemplos-8}

\begin{enumerate}
\def\labelenumi{\arabic{enumi}.}
\item
  \(\int_0^2 x\,dx = \left[\tfrac{1}{2}x^2\right]_0^2 = 2.\) Esta es el
  área de un triángulo rectángulo bajo la línea \(y=x\).
\item
  \(\int_{-1}^1 x^3\,dx = 0.\) La función impar \(x^3\) tiene áreas
  simétricas que se cancelan.
\item
  \(\int_0^\pi \sin x\,dx = 2.\) Esto equivale al área bajo un arco de
  la curva sinusoidal.
\end{enumerate}

\subsubsection{Por qué esto es
importante}\label{por-quuxe9-esto-es-importante-2}

\begin{itemize}
\tightlist
\item
  Las integrales definidas miden cantidades acumuladas: distancia, masa,
  energía, probabilidad.
\item
  Unen el cálculo algebraico con la intuición geométrica.
\item
  El siguiente paso es el Teorema Fundamental del Cálculo, que conecta
  integrales definidas con antiderivadas.
\end{itemize}

\subsubsection{Ejercicios}\label{ejercicios-14}

\begin{enumerate}
\def\labelenumi{\arabic{enumi}.}
\tightlist
\item
  Calcular \(\int_0^3 (2x+1)\,dx\).
\item
  Encuentra el área entre \(y = x^2\) y el eje \(x\) desde \(x = 0\)
  hasta \(x = 2\).
\item
  Tasar \(\int_{-2}^2 (x^2 - 1)\,dx\).
\item
  Demuestra que \(\int_{-a}^a f(x)\,dx = 0\) si \(f(x)\) es impar.
\item
  Calcule \(\int_0^1 e^x\,dx\) usando una suma de Riemann con \(n=4\)
  subintervalos y extremos derechos.
\end{enumerate}

\subsection{4.3 El teorema fundamental del cálculoEl Teorema Fundamental
del Cálculo (FTC) une las dos ideas principales del cálculo:
diferenciación e integración. Muestra que encontrar áreas y encontrar
tasas de cambio son dos caras de la misma
moneda.}\label{el-teorema-fundamental-del-cuxe1lculoel-teorema-fundamental-del-cuxe1lculo-ftc-une-las-dos-ideas-principales-del-cuxe1lculo-diferenciaciuxf3n-e-integraciuxf3n.-muestra-que-encontrar-uxe1reas-y-encontrar-tasas-de-cambio-son-dos-caras-de-la-misma-moneda.}

\subsubsection{Parte 1: Diferenciación de una
integral}\label{parte-1-diferenciaciuxf3n-de-una-integral}

Si \(f\) es continua en \([a, b]\), defina

\[
F(x) = \int_a^x f(t)\,dt.
\]

Entonces \(F\) es derivable, y

\[
F'(x) = f(x).
\]

En palabras: la derivada de la función de área acumulada es la función
original misma.

\subsubsection{Parte 2: Evaluación de integrales
definidas}\label{parte-2-evaluaciuxf3n-de-integrales-definidas}

Si \(f\) es continua en \([a, b]\) y \(F\) es cualquier antiderivada de
\(f\), entonces

\[
\int_a^b f(x)\,dx = F(b) - F(a).
\]

Esto nos dice que podemos evaluar integrales definidas simplemente
encontrando una primitiva, en lugar de calcular los límites de las sumas
de Riemann.

\subsubsection{Ejemplos}\label{ejemplos-9}

\begin{enumerate}
\def\labelenumi{\arabic{enumi}.}
\item
  \(\int_0^2 x^2\,dx\).

  \begin{itemize}
  \tightlist
  \item
    Antiderivada: \(F(x) = \tfrac{1}{3}x^3\).
  \item
    Aplicar FTC: \(F(2) - F(0) = \tfrac{8}{3} - 0 = \tfrac{8}{3}.\)
  \end{itemize}
\item
  Si \(F(x) = \int_1^x \cos t \, dt\), entonces \(F'(x) = \cos x\).
\item
  \(\int_1^4 \frac{1}{x}\,dx\).

  \begin{itemize}
  \tightlist
  \item
    Antiderivada: \(\ln|x|\).
  \item
    Aplicar FTC: \(\ln 4 - \ln 1 = \ln 4.\)
  \end{itemize}
\end{enumerate}

\subsubsection{Por qué es importante la
FTC}\label{por-quuxe9-es-importante-la-ftc}

\begin{itemize}
\tightlist
\item
  Transforma la integración de un proceso límite a un cálculo práctico.
\item
  Confirma que diferenciación e integración son operaciones inversas.
\item
  Es el teorema central que hace que el cálculo sea útil en matemáticas,
  ciencias e ingeniería.
\end{itemize}

\subsubsection{Ejercicios}\label{ejercicios-15}

\begin{enumerate}
\def\labelenumi{\arabic{enumi}.}
\tightlist
\item
  Evalúe \(\int_0^3 (2x+1)\,dx\) utilizando la FTC.
\item
  Si \(F(x) = \int_0^x e^t\,dt\), encuentre \(F'(x)\).
\item
  Calcular \(\int_0^\pi \sin x \, dx\).
\item
  Demuestre que si \(f'(x) = g(x)\), entonces
  \(\int_a^b g(x)\,dx = f(b) - f(a)\).
\item
  Utilice la FTC para explicar por qué el área bajo \(y = \cos x\) desde
  \(0\) hasta \(\pi/2\) es igual a 1.
\end{enumerate}

\subsection{4.4 Propiedades de las
integrales}\label{propiedades-de-las-integrales}

La integral definida tiene varias propiedades importantes que la hacen
flexible y potente en aplicaciones. Estas propiedades se derivan de la
definición como límite de sumas y del Teorema Fundamental del Cálculo.

\subsubsection{Linealidad}\label{linealidad}

Para funciones \(f(x)\) y \(g(x)\), y constantes \(c, d\):

\[
\int_a^b \big(c f(x) + d g(x)\big)\,dx = c \int_a^b f(x)\,dx + d \int_a^b g(x)\,dx.
\]

Esto nos permite dividir integrales complicadas en partes más simples.

\subsubsection{Aditividad en intervalos}\label{aditividad-en-intervalos}

Si \(a < c < b\), entonces

\[
\int_a^b f(x)\,dx = \int_a^c f(x)\,dx + \int_c^b f(x)\,dx.
\]

Podemos calcular integrales pieza por pieza.

\subsubsection{Reversión de límites}\label{reversiuxf3n-de-luxedmites}

\[
\int_a^b f(x)\,dx = -\int_b^a f(x)\,dx.
\]

Al intercambiar los límites se cambia el signo de la integral.

\subsubsection{Propiedad de
comparación}\label{propiedad-de-comparaciuxf3n}

Si \(f(x) \leq g(x)\) para todos los \(x\) en \([a, b]\), entonces

\[
\int_a^b f(x)\,dx \leq \int_a^b g(x)\,dx.
\]Esto nos permite comparar áreas sin cálculo directo.

\subsubsection{Desigualdad de valor
absoluto}\label{desigualdad-de-valor-absoluto}

\[
\left| \int_a^b f(x)\,dx \right| \leq \int_a^b |f(x)|\,dx.
\]

Esta propiedad es esencial en análisis y pruebas de convergencia.

\subsubsection{Simetría}\label{simetruxeda}

\begin{itemize}
\item
  Si \(f(x)\) es par (simétrico respecto al eje \(y\)):

  \[
  \int_{-a}^a f(x)\,dx = 2\int_0^a f(x)\,dx.
  \]
\item
  Si \(f(x)\) es impar (simétrico respecto al origen):

  \[
  \int_{-a}^a f(x)\,dx = 0.
  \]
\end{itemize}

\subsubsection{Ejemplos}\label{ejemplos-10}

\begin{enumerate}
\def\labelenumi{\arabic{enumi}.}
\item
  \(\int_0^2 (3x^2 + 4)\,dx = \int_0^2 3x^2\,dx + \int_0^2 4\,dx = 8 + 8 = 16.\)
\item
  Como \(f(x) = x^3\) es impar, \(\int_{-1}^1 x^3\,dx = 0.\)
\item
  Como \(f(x) = x^2\) es par,
  \(\int_{-2}^2 x^2\,dx = 2\int_0^2 x^2\,dx = 2\cdot \tfrac{8}{3} = \tfrac{16}{3}.\)
\end{enumerate}

\subsubsection{Por qué son importantes estas
propiedades}\label{por-quuxe9-son-importantes-estas-propiedades}

\begin{itemize}
\tightlist
\item
  Simplifican los cálculos.
\item
  Revelan características geométricas y de simetría de funciones.
\item
  Proporcionan herramientas teóricas para análisis más avanzados.
\end{itemize}

\subsubsection{Ejercicios}\label{ejercicios-16}

\begin{enumerate}
\def\labelenumi{\arabic{enumi}.}
\tightlist
\item
  Usa la simetría para evaluar \(\int_{-5}^5 (x^4 - x^3)\,dx\).
\item
  Demuestra que
  \(\int_1^4 (2x+3)\,dx = \int_1^2 (2x+3)\,dx + \int_2^4 (2x+3)\,dx\).
\item
  Evalúe \(\int_0^\pi \sin(x)\,dx\) y compárelo con
  \(\int_{-\pi}^\pi \sin(x)\,dx\).
\item
  Demuestre que si \(f(x) \geq 0\) sobre \([a, b]\), entonces
  \(\int_a^b f(x)\,dx \geq 0\).
\item
  Calcula \(\int_{-3}^3 (x^2 + 1)\,dx\) usando propiedades
  pares/impares.
\end{enumerate}

\section{Capítulo 5. Técnicas de
Integración}\label{capuxedtulo-5.-tuxe9cnicas-de-integraciuxf3n}

\subsection{5.1 Sustitución}\label{sustituciuxf3n}

Una de las técnicas de integración más útiles es el método de
sustitución, también llamado -u-sustitución-. Es el proceso inverso de
la regla de la cadena para los derivados.

\subsubsection{La idea}\label{la-idea}

Si una integral contiene una función compuesta, podemos simplificarla
cambiando variables.

Formalmente, si \(u = g(x)\) es una función derivable, entonces

\[
\int f(g(x)) g'(x)\,dx = \int f(u)\,du.
\]

Esta sustitución hace que la integral sea más fácil de evaluar.

\subsubsection{Pasos para la
sustitución}\label{pasos-para-la-sustituciuxf3n}

\begin{enumerate}
\def\labelenumi{\arabic{enumi}.}
\tightlist
\item
  Identifica una función interna \(u = g(x)\) cuya derivada también
  aparece en el integrando.
\item
  Calcular \(du = g'(x)\,dx\).
\item
  Reescribe la integral en términos de \(u\).
\item
  Integrar respecto de \(u\).
\item
  Sustituir nuevamente \(u = g(x)\).
\end{enumerate}

\subsubsection{Ejemplos}\label{ejemplos-11}

\begin{enumerate}
\def\labelenumi{\arabic{enumi}.}
\item
  Sustitución sencilla

  \[
  \int 2x \cos(x^2)\,dx
  \]

  Sea \(u = x^2\), entonces \(du = 2x\,dx\). Entonces la integral se
  convierte en \(\int \cos u \,du = \sin u + C = \sin(x^2) + C\).
\item
  Caso logarítmico

  \[
  \int \frac{2x}{x^2+1}\,dx
  \]

  Sea \(u = x^2 + 1\), entonces \(du = 2x\,dx\). Entonces la integral se
  convierte en \(\int \frac{1}{u}\,du = \ln|u| + C = \ln(x^2+1) + C\).
\item
  Sustitución trigonométrica

  \[
  \int \sin(3x)\,dx
  \]

  Sea \(u = 3x\), entonces \(du = 3\,dx\), por lo tanto
  \(dx = \frac{du}{3}\).La integral se convierte en
  \(\tfrac{1}{3}\int \sin u\,du = -\tfrac{1}{3}\cos u + C = -\tfrac{1}{3}\cos(3x) + C\).
\end{enumerate}

\subsubsection{Integrales definidas con
sustitución}\label{integrales-definidas-con-sustituciuxf3n}

Al evaluar integrales definidas, también debemos cambiar los límites:

\[
\int_a^b f(g(x)) g'(x)\,dx = \int_{g(a)}^{g(b)} f(u)\,du.
\]

Ejemplo:

\[
\int_0^1 2x e^{x^2}\,dx.
\]

Sea \(u = x^2\), \(du = 2x\,dx\). Límites: cuando \(x=0, u=0\); cuando
\(x=1, u=1\). Entonces la integral se convierte en

\[
\int_0^1 e^u\,du = e - 1.
\]

\subsubsection{Ejercicios}\label{ejercicios-17}

\begin{enumerate}
\def\labelenumi{\arabic{enumi}.}
\tightlist
\item
  Evalúe \(\int (x^2+1)^5 (2x)\,dx\).
\item
  Calcula \(\int \frac{\cos x}{\sin x}\,dx\).
\item
  Calcula \(\int_0^\pi \sin(2x)\,dx\) usando sustitución.
\item
  Encuentra \(\int e^{3x}\,dx\).
\item
  Calcula \(\int \frac{1}{\sqrt{1+x^2}}\,dx\) dejando \(u = 1+x^2\).
\end{enumerate}

\subsection{5.2 Integración por Partes}\label{integraciuxf3n-por-partes}

La integración por partes es una técnica que surge de la regla del
producto para las derivadas. Ayuda a evaluar integrales que involucran
productos de funciones que no se manejan fácilmente mediante sustitución
únicamente.

\subsubsection{La fórmula}\label{la-fuxf3rmula}

De la regla del producto:

\[
\frac{d}{dx}[u(x)v(x)] = u'(x)v(x) + u(x)v'(x).
\]

Al integrar ambos lados se obtiene la fórmula de integración por partes:

\[
\int u\,dv = uv - \int v\,du.
\]

Aquí:

\begin{itemize}
\tightlist
\item
  \(u\) = una función elegida para ser diferenciada,
\item
  \(dv\) = la parte restante del integrando a integrar.
\end{itemize}

\subsubsection{\texorpdfstring{Eligiendo \(u\) y
\(dv\)}{Eligiendo u y dv}}\label{eligiendo-u-y-dv}

Una pauta común es LIATE (logarítmica, trigonométrica inversa,
algebraica, trigonométrica, exponencial).

\begin{itemize}
\tightlist
\item
  Elija \(u\) de la categoría más antigua presente.
\item
  Elige \(dv\) como el resto.
\end{itemize}

\subsubsection{Ejemplos}\label{ejemplos-12}

\begin{enumerate}
\def\labelenumi{\arabic{enumi}.}
\tightlist
\item
  Polinomio × Exponencial
\end{enumerate}

\[
\int x e^x\,dx
\]

Sea \(u = x\), \(dv = e^x dx\). Entonces \(du = dx\), \(v = e^x\).

\[
\int x e^x\,dx = x e^x - \int e^x dx = x e^x - e^x + C.
\]

\begin{enumerate}
\def\labelenumi{\arabic{enumi}.}
\setcounter{enumi}{1}
\tightlist
\item
  Polinomio × Trig.
\end{enumerate}

\[
\int x \cos x\,dx
\]

Sea \(u = x\), \(dv = \cos x dx\). Entonces \(du = dx\), \(v = \sin x\).

\[
\int x \cos x\,dx = x \sin x - \int \sin x dx = x \sin x + \cos x + C.
\]

\begin{enumerate}
\def\labelenumi{\arabic{enumi}.}
\setcounter{enumi}{2}
\tightlist
\item
  Logaritmo
\end{enumerate}

\[
\int \ln x\,dx
\]

Sea \(u = \ln x\), \(dv = dx\). Entonces \(du = \frac{1}{x}dx\),
\(v = x\).

\[
\int \ln x\,dx = x \ln x - \int 1 dx = x \ln x - x + C.
\]

\subsubsection{Ejemplo de integral
definida}\label{ejemplo-de-integral-definida}

\[
\int_0^1 x e^x\,dx
\]

Usando el resultado anterior: \(\int x e^x dx = (x-1)e^x\). Evaluar:

\[
\big[(x-1)e^x\big]_0^1 = (0)e^1 - (-1)e^0 = 0 + 1 = 1.
\]

\subsubsection{Por qué esto es
importante}\label{por-quuxe9-esto-es-importante-3}

La integración por partes es crucial cuando falla la sustitución,
especialmente con logaritmos, funciones trigonométricas inversas y
productos que involucran polinomios con exponenciales o funciones
trigonométricas.

\subsubsection{Ejercicios}\label{ejercicios-18}

\begin{enumerate}
\def\labelenumi{\arabic{enumi}.}
\tightlist
\item
  Valorar \(\int x \sin x\,dx\).
\item
  Calcula \(\int e^x \cos x\,dx\).
\item
  Calcular \(\int_1^2 \ln x\,dx\).
\item
  Valorar \(\int x^2 e^x\,dx\).5. Usa la integración por partes para
  mostrar
  \(\int \arctan x\,dx = x\arctan x - \tfrac{1}{2}\ln(1+x^2) + C\).
\end{enumerate}

\subsection{5.3 Integrales y sustituciones
trigonométricas}\label{integrales-y-sustituciones-trigonomuxe9tricas}

Muchas integrales involucran funciones trigonométricas. A menudo, estos
se pueden simplificar utilizando identidades o realizando sustituciones
especiales.

\subsubsection{Integrales
trigonométricas}\label{integrales-trigonomuxe9tricas}

\begin{enumerate}
\def\labelenumi{\arabic{enumi}.}
\tightlist
\item
  Potencias del seno y el coseno
\end{enumerate}

\begin{itemize}
\tightlist
\item
  Si la potencia del seno es impar: guarda un \(\sin x\), convierte el
  resto por \(\sin^2x = 1 - \cos^2x\), y sustituye \(u = \cos x\).
\item
  Si la potencia del coseno es impar: guarda un \(\cos x\), convierte el
  resto con \(\cos^2x = 1 - \sin^2x\) y sustituye \(u = \sin x\).
\item
  Si ambos son pares: use identidades de medio ángulo.
\end{itemize}

Ejemplo:

\[
\int \sin^3x \cos x \, dx
\]

Sean \(u = \sin x\), \(du = \cos x\,dx\):

\[
\int u^3\,du = \tfrac{u^4}{4} + C = \tfrac{\sin^4x}{4} + C.
\]

\begin{enumerate}
\def\labelenumi{\arabic{enumi}.}
\setcounter{enumi}{1}
\tightlist
\item
  Productos de seno y coseno con diferentes ángulos. Utilice fórmulas de
  producto a suma:
\end{enumerate}

\[
\sin A \cos B = \tfrac{1}{2}[\sin(A+B) + \sin(A-B)].
\]

Ejemplo:

\[
\int \sin(2x)\cos(3x)\,dx = \tfrac{1}{2}\int [\sin(5x) - \sin(x)]\,dx.
\]

\begin{enumerate}
\def\labelenumi{\arabic{enumi}.}
\setcounter{enumi}{2}
\tightlist
\item
  Potencias de secante y tangente
\end{enumerate}

\begin{itemize}
\tightlist
\item
  Si la potencia de la secante es par: guarda \(\sec^2x\), convierte el
  resto con \(\sec^2x = 1 + \tan^2x\), y sustituye \(u = \tan x\).
\item
  Si la potencia de la tangente es impar: guarda \(\sec^2x\), convierte
  el resto con \(\tan^2x = \sec^2x - 1\), y sustituye \(u = \tan x\).
\end{itemize}

Ejemplo:

\[
\int \tan^3x \sec^2x \, dx
\]

Sean \(u = \tan x\), \(du = \sec^2x\,dx\):

\[
\int u^3\,du = \tfrac{u^4}{4} + C = \tfrac{\tan^4x}{4} + C.
\]

\subsubsection{Sustituciones
trigonométricas}\label{sustituciones-trigonomuxe9tricas}

Para integrales que involucran \(\sqrt{a^2 - x^2}\),
\(\sqrt{a^2 + x^2}\) o \(\sqrt{x^2 - a^2}\), use sustituciones
especiales:

\begin{enumerate}
\def\labelenumi{\arabic{enumi}.}
\tightlist
\item
  \(x = a \sin \theta\), por \(\sqrt{a^2 - x^2}\).
\item
  \(x = a \tan \theta\), por \(\sqrt{a^2 + x^2}\).
\item
  \(x = a \sec \theta\), por \(\sqrt{x^2 - a^2}\).
\end{enumerate}

Ejemplo:

\[
\int \sqrt{a^2 - x^2}\,dx
\]

Sea \(x = a\sin\theta\), entonces \(dx = a\cos\theta\,d\theta\):

\[
\int \sqrt{a^2 - a^2\sin^2\theta}(a\cos\theta\,d\theta) = \int a^2 \cos^2\theta \, d\theta.
\]

Simplifique usando identidades de medio ángulo.

\subsubsection{Por qué son importantes estas
técnicas}\label{por-quuxe9-son-importantes-estas-tuxe9cnicas}

\begin{itemize}
\tightlist
\item
  Convierten formas algebraicas difíciles en trigonométricas manejables.
\item
  Son especialmente útiles en problemas que involucran áreas, volúmenes
  y longitudes de arco.
\item
  Sientan las bases para métodos de integración avanzados.
\end{itemize}

\subsubsection{Ejercicios}\label{ejercicios-19}

\begin{enumerate}
\def\labelenumi{\arabic{enumi}.}
\tightlist
\item
  Valorar \(\int \sin^4x \cos^2x \, dx\).
\item
  Calcular \(\int \sin(5x)\cos(2x)\,dx\).
\item
  Valorar \(\int \tan^2x \sec^2x \, dx\).
\item
  Calcula \(\int \sqrt{9 - x^2}\,dx\) usando sustitución.
\item
  Demuestra que
  \(\int \frac{dx}{\sqrt{x^2 + a^2}} = \ln|x + \sqrt{x^2 + a^2}| + C\)
  usando \(x = a\tan\theta\).
\end{enumerate}

\subsection{5.4 Fracciones ParcialesAl integrar funciones racionales
(razones de polinomios), un método poderoso es la descomposición en
fracciones parciales. Esta técnica expresa una fracción complicada como
una suma de fracciones más simples que son más fáciles de
integrar.}\label{fracciones-parcialesal-integrar-funciones-racionales-razones-de-polinomios-un-muxe9todo-poderoso-es-la-descomposiciuxf3n-en-fracciones-parciales.-esta-tuxe9cnica-expresa-una-fracciuxf3n-complicada-como-una-suma-de-fracciones-muxe1s-simples-que-son-muxe1s-fuxe1ciles-de-integrar.}

\subsubsection{La idea}\label{la-idea-1}

Si \(R(x) = \frac{P(x)}{Q(x)}\) es una función racional, donde el grado
de \(P(x)\) es menor que el grado de \(Q(x)\), podemos descomponer
\(R(x)\) en fracciones más simples.

Estas piezas más simples corresponden a los factores del denominador
\(Q(x)\).

\subsubsection{Formularios comunes}\label{formularios-comunes}

\begin{enumerate}
\def\labelenumi{\arabic{enumi}.}
\tightlist
\item
  Factores lineales distintos si
\end{enumerate}

\[
\frac{1}{(x-a)(x-b)},
\]

luego descomponerse como

\[
\frac{A}{x-a} + \frac{B}{x-b}.
\]

\begin{enumerate}
\def\labelenumi{\arabic{enumi}.}
\setcounter{enumi}{1}
\tightlist
\item
  Factores lineales repetidos Si el denominador tiene \((x-a)^n\),
  entonces los términos son
\end{enumerate}

\[
\frac{A_1}{x-a} + \frac{A_2}{(x-a)^2} + \dots + \frac{A_n}{(x-a)^n}.
\]

\begin{enumerate}
\def\labelenumi{\arabic{enumi}.}
\setcounter{enumi}{2}
\tightlist
\item
  Factores cuadráticos irreducibles Si el denominador tiene
  \((x^2+bx+c)\), entonces el numerador es lineal:
\end{enumerate}

\[
\frac{Ax+B}{x^2+bx+c}.
\]

\subsubsection{Ejemplo 1: factores lineales
distintos}\label{ejemplo-1-factores-lineales-distintos}

\[
\int \frac{1}{x^2 - 1}\,dx
\]

Denominador del factor: \((x-1)(x+1)\). Descomponer:

\[
\frac{1}{x^2-1} = \frac{1}{2}\left(\frac{1}{x-1} - \frac{1}{x+1}\right).
\]

Integrar:

\[
\int \frac{1}{x^2 - 1}\,dx = \tfrac{1}{2}\ln\left|\frac{x-1}{x+1}\right| + C.
\]

\subsubsection{Ejemplo 2: factor lineal
repetido}\label{ejemplo-2-factor-lineal-repetido}

\[
\int \frac{1}{(x-1)^2}\,dx
\]

Esto ya es sencillo:

\[
\int (x-1)^{-2}\,dx = -\frac{1}{x-1} + C.
\]

\subsubsection{Ejemplo 3: Factor cuadrático
irreducible}\label{ejemplo-3-factor-cuadruxe1tico-irreducible}

\[
\int \frac{x}{x^2+1}\,dx
\]

Sustituye \(u = x^2+1\), o reconoce que el numerador es derivado del
denominador.

\[
\int \frac{x}{x^2+1}\,dx = \tfrac{1}{2}\ln(x^2+1) + C.
\]

\subsubsection{Pasos en la descomposición de fracciones
parciales}\label{pasos-en-la-descomposiciuxf3n-de-fracciones-parciales}

\begin{enumerate}
\def\labelenumi{\arabic{enumi}.}
\tightlist
\item
  Factoriza el denominador.
\item
  Escribe la forma de fracción parcial general.
\item
  Multiplica por el denominador para borrar fracciones.
\item
  Resuelva para constantes desconocidas.
\item
  Integra cada término.
\end{enumerate}

\subsubsection{Por qué esto es
importante}\label{por-quuxe9-esto-es-importante-4}

\begin{itemize}
\tightlist
\item
  Convierte funciones racionales complejas en formas logarítmicas o
  arctangentes simples.
\item
  Especialmente útil en ecuaciones diferenciales y transformadas de
  Laplace.
\item
  Fundamental en cálculo avanzado e ingeniería.
\end{itemize}

\subsubsection{Ejercicios}\label{ejercicios-20}

\begin{enumerate}
\def\labelenumi{\arabic{enumi}.}
\tightlist
\item
  Descomponer e integrar \(\int \frac{3x+5}{x^2-1}\,dx\).
\item
  Valorar \(\int \frac{1}{x^2(x+1)}\,dx\).
\item
  Calcula \(\int \frac{2x+1}{x^2+2x+2}\,dx\).
\item
  Calcula \(\int \frac{1}{x^3 - x}\,dx\).
\item
  Demuestra que \(\int \frac{dx}{x^2+1} = \arctan x + C\) usando
  fracciones parciales o sustitución.
\end{enumerate}

\subsection{5.5 Integrales impropias}\label{integrales-impropias}

Algunas integrales no se pueden evaluar directamente porque el intervalo
es infinito o el integrando se vuelve ilimitado. Éstas se llaman
integrales impropias. Se definen mediante límites.

\subsubsection{Definición}\label{definiciuxf3n-5}

\begin{enumerate}
\def\labelenumi{\arabic{enumi}.}
\tightlist
\item
  Intervalo infinito
\end{enumerate}

\[\int_a^\infty f(x)\,dx = \lim_{b \to \infty} \int_a^b f(x)\,dx.
\]

\[
\int_{-\infty}^a f(x)\,dx = \lim_{b \to -\infty} \int_b^a f(x)\,dx.
\]

\begin{enumerate}
\def\labelenumi{\arabic{enumi}.}
\setcounter{enumi}{1}
\tightlist
\item
  Unbounded integrand If \(f(x)\) has a vertical asymptote at \(c\),
  then
\end{enumerate}

\[
\int_a^c f(x)\,dx = \lim_{t \to c^-} \int_a^t f(x)\,dx,
\]

\[
\int_c^b f(x)\,dx = \lim_{t \to c^+} \int_t^b f(x)\,dx.
\]

\subsubsection{Convergence and
Divergence}\label{convergence-and-divergence}

\begin{itemize}
\tightlist
\item
  If the limit exists and is finite, the improper integral converges.
\item
  If the limit does not exist or is infinite, the improper integral
  diverges.
\end{itemize}

\subsubsection{Examples}\label{examples-1}

\begin{enumerate}
\def\labelenumi{\arabic{enumi}.}
\tightlist
\item
  Exponential decay
\end{enumerate}

\[
\int_1^\infty \frac{1}{x^2}\,dx = \lim_{b \to \infty} \Big[-\tfrac{1}{x}\Big]_1^b = 1.
\]

This converges.

\begin{enumerate}
\def\labelenumi{\arabic{enumi}.}
\setcounter{enumi}{1}
\tightlist
\item
  Harmonic function
\end{enumerate}

\[
\int_1^\infty \frac{1}{x}\,dx = \lim_{b \to \infty} \ln b.
\]

This diverges to infinity.

\begin{enumerate}
\def\labelenumi{\arabic{enumi}.}
\setcounter{enumi}{2}
\tightlist
\item
  Asymptote at 0
\end{enumerate}

\[
\int_0^1 \frac{1}{\sqrt{x}}\,dx = \lim_{t \to 0^+} \int_t^1 x^{-1/2}\,dx.
\]

\[
= \lim_{t \to 0^+} [2\sqrt{x}]_t^1 = 2.
\]

This converges.

\begin{enumerate}
\def\labelenumi{\arabic{enumi}.}
\setcounter{enumi}{3}
\tightlist
\item
  Asymptote at 0 (divergent)
\end{enumerate}

\[
\int_0^1 \frac{1}{x}\,dx = \lim_{t \to 0^+} \ln(1) - \ln(t).
\]

This diverges since \(\ln(t) \to -\infty\).

\subsubsection{Comparison Test for Improper
Integrals}\label{comparison-test-for-improper-integrals}

\begin{itemize}
\tightlist
\item
  If \(0 \leq f(x) \leq g(x)\) for large \(x\), and \(\int g(x)\,dx\)
  converges, then \(\int f(x)\,dx\) also converges.
\item
  If \(\int f(x)\,dx\) diverges and \(f(x) \geq g(x) \geq 0\), then
  \(\int g(x)\,dx\) also diverges.
\end{itemize}

\subsubsection{Why Improper Integrals
Matter}\label{why-improper-integrals-matter}

\begin{itemize}
\tightlist
\item
  They extend integration to infinite domains and unbounded functions.
\item
  They are essential in probability (continuous distributions), physics
  (gravitational/electric fields), and Fourier analysis.
\end{itemize}

\subsubsection{Exercises}\label{exercises-2}

\begin{enumerate}
\def\labelenumi{\arabic{enumi}.}
\tightlist
\item
  Determine whether \(\int_1^\infty \frac{1}{x^p}\,dx\) converges for
  various values of \(p\).
\item
  Evaluate \(\int_0^\infty e^{-x}\,dx\).
\item
  Test convergence of \(\int_0^1 \frac{1}{x^p}\,dx\) depending on \(p\).
\item
  Compute \(\int_{-\infty}^\infty \frac{1}{1+x^2}\,dx\).
\item
  Use the comparison test to show that
  \(\int_1^\infty \frac{1}{x^2+1}\,dx\) converges.
\end{enumerate}

\section{Chapter 6. Applications of
Integration}\label{chapter-6.-applications-of-integration}

\subsection{6.1 Areas and Volumes}\label{areas-and-volumes}

One of the most important applications of integration is finding areas
under curves and volumes of solids.

\subsubsection{Area Between Curves}\label{area-between-curves}

If \(f(x) \geq g(x)\) on \([a, b]\), then the area between the curves
\(y=f(x)\) and \(y=g(x)\) is

\[
A = \int_a^b \big(f(x) - g(x)\big)\,dx.
\]

Example: Find the area between \(y=x^2\) and \(y=x\) on \([0,1]\).

\[
A = \int_0^1 (x - x^2)\,dx = \left[\tfrac{1}{2}x^2 - \tfrac{1}{3}x^3\right]_0^1 = \tfrac{1}{6}.
\]

\subsubsection{Volumes by Slicing}\label{volumes-by-slicing}

If a solid has cross-sectional area \(A(x)\) at position \(x\), then the
volume is

\[
V = \int_a^b A(x)\,dx.
\]\#\#\# Volúmenes de Revolución

Cuando una región gira alrededor de un eje, el volumen del sólido
resultante se puede encontrar mediante integración.

\begin{enumerate}
\def\labelenumi{\arabic{enumi}.}
\tightlist
\item
  Método de disco Si la región bajo \(y=f(x)\), \(x\in[a,b]\), gira
  alrededor del eje \(x\):
\end{enumerate}

\[
V = \pi \int_a^b [f(x)]^2\,dx.
\]

\begin{enumerate}
\def\labelenumi{\arabic{enumi}.}
\setcounter{enumi}{1}
\tightlist
\item
  Método de la lavadora Si la región entre \(y=f(x)\) y \(y=g(x)\) gira
  alrededor del eje \(x\):
\end{enumerate}

\[
V = \pi \int_a^b \Big([f(x)]^2 - [g(x)]^2\Big)\,dx.
\]

\begin{enumerate}
\def\labelenumi{\arabic{enumi}.}
\setcounter{enumi}{2}
\tightlist
\item
  Método de concha Si la región bajo \(y=f(x)\) gira alrededor del eje
  \(y\):
\end{enumerate}

\[
V = 2\pi \int_a^b x f(x)\,dx.
\]

\subsubsection{Ejemplos}\label{ejemplos-13}

\begin{enumerate}
\def\labelenumi{\arabic{enumi}.}
\tightlist
\item
  método de disco Girar \(y=\sqrt{x}\), \(0 \leq x \leq 4\), alrededor
  del eje \(x\):
\end{enumerate}

\[
V = \pi \int_0^4 (\sqrt{x})^2\,dx = \pi \int_0^4 x\,dx = \pi \left[\tfrac{1}{2}x^2\right]_0^4 = 8\pi.
\]

\begin{enumerate}
\def\labelenumi{\arabic{enumi}.}
\setcounter{enumi}{1}
\tightlist
\item
  Método de la lavadora Región de revolución entre \(y=\sqrt{x}\) y
  \(y=1\), \(0 \leq x \leq 1\), alrededor del eje \(x\):
\end{enumerate}

\[
V = \pi \int_0^1 \big((\sqrt{x})^2 - (1)^2\big)\,dx = \pi \int_0^1 (x-1)\,dx = -\tfrac{\pi}{2}.
\]

(Tomar valor absoluto para el volumen: \(V = \tfrac{\pi}{2}\)).

\begin{enumerate}
\def\labelenumi{\arabic{enumi}.}
\setcounter{enumi}{2}
\tightlist
\item
  Método de concha Región de revolución bajo \(y=x\),
  \(0 \leq x \leq 1\), alrededor del eje \(y\):
\end{enumerate}

\[
V = 2\pi \int_0^1 x(x)\,dx = 2\pi \int_0^1 x^2\,dx = 2\pi \cdot \tfrac{1}{3} = \tfrac{2\pi}{3}.
\]

\subsubsection{Por qué esto es
importante}\label{por-quuxe9-esto-es-importante-5}

\begin{itemize}
\tightlist
\item
  Proporciona formas exactas de calcular áreas y volúmenes en geometría.
\item
  Imprescindible en física, ingeniería y probabilidad.
\item
  Introduce el pensamiento geométrico con integración.
\end{itemize}

\subsubsection{Ejercicios}\label{ejercicios-21}

\begin{enumerate}
\def\labelenumi{\arabic{enumi}.}
\tightlist
\item
  Calcula el área entre \(y=\cos x\) y \(y=\sin x\) en \([0, \pi/2]\).
\item
  Calcula el volumen del sólido que se forma al hacer girar \(y=x^2\),
  \(0 \leq x \leq 1\), alrededor del eje \(x\).
\item
  Encuentra el volumen del sólido formado al hacer girar la región entre
  \(y=x\) y \(y=\sqrt{x}\) en \([0,1]\) alrededor del eje \(y\).
\item
  Utilice el método de la arandela para calcular el volumen del sólido
  formado al hacer girar \(y=\sqrt{1-x^2}\) (un semicírculo) alrededor
  del eje \(x\).
\item
  Calcula el área encerrada entre \(y=x^2+1\) y \(y=3x\).
\end{enumerate}

\subsection{6.2 Longitud del arco y área de
superficie}\label{longitud-del-arco-y-uxe1rea-de-superficie}

La integración también se puede utilizar para medir la longitud de las
curvas y el área de superficie de los sólidos generados por curvas
giratorias.

\subsubsection{Longitud del arco}\label{longitud-del-arco}

Para una curva suave \(y=f(x)\) en el intervalo \([a,b]\), la longitud
de la curva es

\[
L = \int_a^b \sqrt{1 + \big(f'(x)\big)^2}\,dx.
\]

Esto surge de aproximar la curva con segmentos de recta y tomar el
límite.

Ejemplo: Encuentra la longitud de \(y=\tfrac{1}{2}x^{3/2}\) desde
\(x=0\) hasta \(x=4\).

\begin{itemize}
\tightlist
\item
  Derivado: \(f'(x) = \tfrac{3}{4}\sqrt{x}\).
\item
  Fórmula:
\end{itemize}

\[
L = \int_0^4 \sqrt{1 + \Big(\tfrac{3}{4}\sqrt{x}\Big)^2}\,dx
= \int_0^4 \sqrt{1 + \tfrac{9}{16}x}\,dx.
\]

Esta integral se puede evaluar mediante sustitución.\#\#\# Área de
superficie de revolución

Si una curva \(y=f(x)\), \(a \leq x \leq b\), gira alrededor del eje
\(x\), el área de la superficie del sólido resultante es

\[
S = 2\pi \int_a^b f(x)\sqrt{1 + \big(f'(x)\big)^2}\,dx.
\]

Si gira alrededor del eje \(y\):

\[
S = 2\pi \int_a^b x \sqrt{1 + \big(f'(x)\big)^2}\,dx.
\]

\subsubsection{Ejemplos}\label{ejemplos-14}

\begin{enumerate}
\def\labelenumi{\arabic{enumi}.}
\tightlist
\item
  Longitud del arco de una línea Por \(y=x\), \(0 \leq x \leq 3\):
\end{enumerate}

\[
L = \int_0^3 \sqrt{1+(1)^2}\,dx = \int_0^3 \sqrt{2}\,dx = 3\sqrt{2}.
\]

\begin{enumerate}
\def\labelenumi{\arabic{enumi}.}
\setcounter{enumi}{1}
\tightlist
\item
  Área de superficie de una esfera Tome \(y = \sqrt{r^2 - x^2}\),
  \(-r \leq x \leq r\) y gire alrededor del eje \(x\).
\end{enumerate}

\[
S = 2\pi \int_{-r}^r \sqrt{r^2 - x^2}\sqrt{1+\left(\frac{-x}{\sqrt{r^2-x^2}}\right)^2}\,dx.
\]

La simplificación da \(S = 4\pi r^2\), la conocida fórmula para el área
de superficie de una esfera.

\subsubsection{Por qué esto es
importante}\label{por-quuxe9-esto-es-importante-6}

\begin{itemize}
\tightlist
\item
  La longitud del arco extiende la idea de distancia a caminos curvos.
\item
  El área de superficie de revolución tiene aplicaciones en física,
  ingeniería y diseño.
\item
  Proporciona un puente entre el cálculo y la geometría.
\end{itemize}

\subsubsection{Ejercicios}\label{ejercicios-22}

\begin{enumerate}
\def\labelenumi{\arabic{enumi}.}
\tightlist
\item
  Encuentra la longitud del arco de \(y=\sqrt{x}\) desde \(x=0\) hasta
  \(x=4\).
\item
  Calcula el área de la superficie del sólido obtenido al hacer girar
  \(y=x^2\), \(0 \leq x \leq 1\), alrededor del eje \(x\).
\item
  Calcula la longitud del arco de \(y=\ln(\cosh x)\) desde \(x=0\) hasta
  \(x=1\).
\item
  Demuestra que al hacer girar \(y=\sqrt{r^2 - x^2}\) de \(0\) a \(r\)
  alrededor del eje \(x\) se obtiene la mitad del área de superficie de
  una esfera.
\item
  Derive la fórmula para el área de superficie de un cono haciendo girar
  una línea.
\end{enumerate}

\subsection{6.3 Trabajo y Promedios}\label{trabajo-y-promedios}

La integración no se limita a la geometría. También ayuda a calcular el
trabajo realizado por una fuerza y \hspace{0pt}\hspace{0pt}el valor
promedio de una función durante un intervalo.

\subsubsection{Trabajo}\label{trabajo}

Si una fuerza variable \(F(x)\) mueve un objeto en línea recta desde
\(x=a\) hasta \(x=b\), entonces el trabajo total es

\[
W = \int_a^b F(x)\,dx.
\]

Esta fórmula generaliza el caso simple \(W = F \cdot d\) para fuerza
constante.

Ejemplo 1: Fuerza del resorte (Ley de Hooke) Para un resorte estirado
desde una longitud \(a\) hasta \(b\), con fuerza \(F(x) = kx\):

\[
W = \int_a^b kx\,dx = \tfrac{1}{2}k(b^2-a^2).
\]

Ejemplo 2: bombeo de agua Si se bombea agua de un tanque, el trabajo
requerido es igual

\[
W = \int_a^b \text{(weight density)} \times \text{(cross-sectional area)} \times \text{(distance lifted)} \, dx.
\]

\subsubsection{Valor promedio de una
función}\label{valor-promedio-de-una-funciuxf3n}

El valor medio de una función continua \(f(x)\) sobre \([a,b]\) es

\[
f_{\text{avg}} = \frac{1}{b-a} \int_a^b f(x)\,dx.
\]

Este es el análogo continuo de promediar una lista de números.

Ejemplo 1: Por \(f(x)=x^2\) sobre \([0,2]\):

\[
f_{\text{avg}} = \tfrac{1}{2-0}\int_0^2 x^2 dx = \tfrac{1}{2}\cdot \tfrac{8}{3} = \tfrac{4}{3}.
\]

Ejemplo 2:Si la velocidad de una partícula es \(v(t)\), entonces la
velocidad promedio sobre \([a,b]\) es

\[
v_{\text{avg}} = \frac{1}{b-a}\int_a^b v(t)\,dt.
\]

\subsubsection{Por qué esto es
importante}\label{por-quuxe9-esto-es-importante-7}

\begin{itemize}
\tightlist
\item
  Las integrales de trabajo aparecen en física, ingeniería y cálculos de
  energía.
\item
  El valor medio proporciona un único número representativo para
  cantidades variables.
\item
  Ambos conectan el cálculo con problemas de movimiento, fuerza y
  \hspace{0pt}\hspace{0pt}eficiencia del mundo real.
\end{itemize}

\subsubsection{Ejercicios}\label{ejercicios-23}

\begin{enumerate}
\def\labelenumi{\arabic{enumi}.}
\tightlist
\item
  Calcula el trabajo necesario para estirar un resorte de 2 ma 5 m si
  \(k=10\).
\item
  Un objeto de 100 kg se eleva verticalmente 5 m en un campo
  gravitacional (\(g=9.8 \,\text{m/s}^2\)). Expresar el trabajo como una
  integral y evaluar.
\item
  Calcula el valor medio de \(f(x)=\sin x\) sobre \([0,\pi]\).
\item
  Calcula la temperatura promedio si
  \(T(t)=20+5\cos(\tfrac{\pi t}{12})\) durante un día de 24 horas.
\item
  Un tanque de 10 m de profundidad está lleno de agua. Calcula el
  trabajo necesario para bombear toda el agua hasta la superficie, dado
  que el agua pesa \(9800 \,\text{N/m}^3\).
\end{enumerate}

\subsection{6.4 Densidades de probabilidad y distribuciones
continuas}\label{densidades-de-probabilidad-y-distribuciones-continuas}

La integración también juega un papel central en la teoría de la
probabilidad, especialmente para variables aleatorias continuas. En
lugar de resultados discretos, describimos probabilidades con funciones
llamadas funciones de densidad de probabilidad (pdf).

\subsubsection{Funciones de densidad de
probabilidad}\label{funciones-de-densidad-de-probabilidad}

Una función de densidad de probabilidad \(f(x)\) debe cumplir dos
condiciones:

\begin{enumerate}
\def\labelenumi{\arabic{enumi}.}
\item
  \(f(x) \geq 0\) para todos \(x\).
\item
  El área total bajo la curva es 1:

  \[
  \int_{-\infty}^\infty f(x)\,dx = 1.
  \]
\end{enumerate}

Si \(X\) es una variable aleatoria continua con función de densidad de
probabilidad \(f(x)\), entonces la probabilidad de que \(X\) se
encuentre entre \(a\) y \(b\) es

\[
P(a \leq X \leq b) = \int_a^b f(x)\,dx.
\]

\subsubsection{Función de distribución
acumulativa}\label{funciuxf3n-de-distribuciuxf3n-acumulativa}

La función de distribución acumulativa (cdf) se define como

\[
F(x) = \int_{-\infty}^x f(t)\,dt.
\]

Da la probabilidad de que la variable aleatoria sea menor o igual a
\(x\).

\subsubsection{Valor esperado (media)}\label{valor-esperado-media}

El valor esperado de una variable aleatoria continua es el promedio
ponderado:

\[
E[X] = \int_{-\infty}^\infty x f(x)\,dx.
\]

\subsubsection{Ejemplos}\label{ejemplos-15}

\begin{enumerate}
\def\labelenumi{\arabic{enumi}.}
\tightlist
\item
  Distribución uniforme Por \(f(x) = \tfrac{1}{b-a}\) a \([a,b]\):
\end{enumerate}

\begin{itemize}
\item
  Probabilidad del intervalo \([c,d]\):

  \[
  P(c \leq X \leq d) = \frac{d-c}{b-a}.
  \]
\item
  Valor esperado: \(E[X] = \tfrac{a+b}{2}\).
\end{itemize}

\begin{enumerate}
\def\labelenumi{\arabic{enumi}.}
\setcounter{enumi}{1}
\tightlist
\item
  Distribución exponencial Por \(f(x) = \lambda e^{-\lambda x}\),
  \(x \geq 0\):
\end{enumerate}

\begin{itemize}
\tightlist
\item
  \(\int_0^\infty \lambda e^{-\lambda x}\,dx = 1\).
\item
  Media: \(E[X] = \tfrac{1}{\lambda}\).
\end{itemize}

\begin{enumerate}
\def\labelenumi{\arabic{enumi}.}
\setcounter{enumi}{2}
\tightlist
\item
  Distribución normal La curva de campana:
\end{enumerate}

\[
f(x) = \frac{1}{\sqrt{2\pi\sigma^2}} e^{-\frac{(x-\mu)^2}{2\sigma^2}}.
\]

Se integra a 1, pero requiere técnicas avanzadas.

\subsubsection{Por qué esto es importante- Las densidades de
probabilidad describen la incertidumbre en ciencia, ingeniería y
estadística.}\label{por-quuxe9-esto-es-importante--las-densidades-de-probabilidad-describen-la-incertidumbre-en-ciencia-ingenieruxeda-y-estaduxedstica.}

\begin{itemize}
\tightlist
\item
  Las integrales conectan áreas bajo curvas con probabilidades.
\item
  Las distribuciones continuas generalizan la idea de contar resultados
  para medir probabilidades en intervalos.
\end{itemize}

\subsubsection{Ejercicios}\label{ejercicios-24}

\begin{enumerate}
\def\labelenumi{\arabic{enumi}.}
\tightlist
\item
  Demuestre que la densidad uniforme \(f(x) = \tfrac{1}{b-a}\) en
  \([a,b]\) se integra a 1.
\item
  Para la distribución exponencial con \(\lambda = 2\), calcula
  \(P(0 \leq X \leq 1)\).
\item
  Calcula el valor esperado de \(X\) si \(f(x) = 3x^2\) sobre \([0,1]\).
\item
  Verifique que la distribución normal con media 0 y varianza 1 tenga
  probabilidad total 1 (no es necesario realizar una prueba completa,
  pero explique por qué se cumple).
\item
  Calcule la CDF de la distribución uniforme en \([0,1]\).
\end{enumerate}

\section{Parte III. Cálculo
multivariable}\label{parte-iii.-cuxe1lculo-multivariable}

\section{Capítulo 7. Funciones vectoriales y
curvas}\label{capuxedtulo-7.-funciones-vectoriales-y-curvas}

\subsection{7.1 Funciones vectoriales y curvas
espaciales}\label{funciones-vectoriales-y-curvas-espaciales}

En el cálculo multivariable, las funciones pueden generar vectores en
lugar de números. Éstas se denominan funciones con valores vectoriales y
son esenciales para describir curvas en el espacio.

\subsubsection{Definición}\label{definiciuxf3n-6}

Una función vectorial es una función de la forma

\[
\mathbf{r}(t) = \langle x(t), y(t), z(t) \rangle,
\]

donde \(x(t), y(t), z(t)\) son funciones de valor real.

\begin{itemize}
\tightlist
\item
  La entrada \(t\) suele denominarse parámetro.
\item
  La salida es un vector en un espacio 2D o 3D.
\item
  La gráfica de una función vectorial en 3D es una curva espacial.
\end{itemize}

\subsubsection{Ejemplos}\label{ejemplos-16}

\begin{enumerate}
\def\labelenumi{\arabic{enumi}.}
\tightlist
\item
  Línea
\end{enumerate}

\[
\mathbf{r}(t) = \langle 1+2t, \; 3-t, \; 4+5t \rangle.
\]

Describe una línea recta que pasa por el punto \((1,3,4)\) con el vector
director \(\langle 2,-1,5 \rangle\).

\begin{enumerate}
\def\labelenumi{\arabic{enumi}.}
\setcounter{enumi}{1}
\tightlist
\item
  Círculo en el avión.
\end{enumerate}

\[
\mathbf{r}(t) = \langle \cos t, \; \sin t, \; 0 \rangle, \quad 0 \leq t < 2\pi.
\]

\begin{enumerate}
\def\labelenumi{\arabic{enumi}.}
\setcounter{enumi}{2}
\tightlist
\item
  hélice
\end{enumerate}

\[
\mathbf{r}(t) = \langle \cos t, \; \sin t, \; t \rangle.
\]

Se trata de una espiral que se eleva alrededor del eje \(z\).

\subsubsection{Límites y Continuidad}\label{luxedmites-y-continuidad}

Una función vectorial es continua en \(t=a\) si cada componente
\(x(t), y(t), z(t)\) es continua en \(t=a\).

\[
\lim_{t \to a} \mathbf{r}(t) = \langle \lim_{t \to a} x(t), \; \lim_{t \to a} y(t), \; \lim_{t \to a} z(t) \rangle.
\]

\subsubsection{Geometría de las curvas
espaciales}\label{geometruxeda-de-las-curvas-espaciales}

\begin{itemize}
\tightlist
\item
  Cada curva tiene una dirección tangente dada por la derivada.
\item
  Las curvas espaciales pueden modelar trayectorias de movimiento,
  trayectorias de partículas y formas geométricas.
\end{itemize}

\subsubsection{Por qué esto es
importante}\label{por-quuxe9-esto-es-importante-8}

Las funciones vectoriales son la base del cálculo multivariable y nos
permiten extender las ideas de derivadas e integrales a dimensiones
superiores. También aparecen de forma natural en física (movimiento en
3D, electromagnetismo, dinámica de fluidos).

\subsubsection{Ejercicios}\label{ejercicios-25}

\begin{enumerate}
\def\labelenumi{\arabic{enumi}.}
\tightlist
\item
  Escribe una función vectorial para una recta que pasa por \((0,1,2)\)
  paralela al vector \(\langle 3,-2,1 \rangle\).2. Describe la curva
  dada por
  \(\mathbf{r}(t) = \langle 2\cos t, \; 2\sin t, \; 3 \rangle\).
\item
  Determina si \(\mathbf{r}(t) = \langle e^t, \; \ln t, \; t^2 \rangle\)
  es continua en \(t=1\).
\item
  Dibuja la hélice
  \(\mathbf{r}(t) = \langle \cos t, \; \sin t, \; 2t \rangle\).
\item
  Encuentra el punto de la curva
  \(\mathbf{r}(t) = \langle t, \; t^2, \; t^3 \rangle\) cuando \(t=2\).
\end{enumerate}

\subsection{7.2 Derivadas e integrales de funciones
vectoriales}\label{derivadas-e-integrales-de-funciones-vectoriales}

Las funciones vectoriales se pueden diferenciar e integrar como
funciones ordinarias: simplemente aplicamos la operación a cada
componente. Esto nos permite estudiar el movimiento, la velocidad, la
aceleración y la acumulación en dimensiones superiores.

\subsubsection{Derivada de una función
vectorial}\label{derivada-de-una-funciuxf3n-vectorial}

si

\[
\mathbf{r}(t) = \langle x(t), y(t), z(t) \rangle,
\]

entonces

\[
\mathbf{r}'(t) = \langle x'(t), y'(t), z'(t) \rangle.
\]

Este vector derivada apunta en la dirección tangente a la curva en el
parámetro \(t\).

\begin{itemize}
\tightlist
\item
  Velocidad: Si \(\mathbf{r}(t)\) da la posición de una partícula en el
  tiempo \(t\), entonces \(\mathbf{v}(t) = \mathbf{r}'(t)\) es su vector
  velocidad.
\item
  Velocidad: La magnitud \(|\mathbf{v}(t)|\) es la velocidad de la
  partícula.
\item
  Aceleración: \(\mathbf{a}(t) = \mathbf{v}'(t) = \mathbf{r}''(t)\).
\end{itemize}

\subsubsection{Ejemplos}\label{ejemplos-17}

\begin{enumerate}
\def\labelenumi{\arabic{enumi}.}
\tightlist
\item
  hélice
\end{enumerate}

\[
\mathbf{r}(t) = \langle \cos t, \sin t, t \rangle.
\]

\begin{itemize}
\tightlist
\item
  Velocidad: \(\mathbf{v}(t) = \langle -\sin t, \cos t, 1 \rangle\).
\item
  Velocidad:
  \(|\mathbf{v}(t)| = \sqrt{(-\sin t)^2 + (\cos t)^2 + 1^2} = \sqrt{2}\).
\item
  Aceleración: \(\mathbf{a}(t) = \langle -\cos t, -\sin t, 0 \rangle\).
\end{itemize}

\begin{enumerate}
\def\labelenumi{\arabic{enumi}.}
\setcounter{enumi}{1}
\tightlist
\item
  Movimiento del proyectil
\end{enumerate}

\[
\mathbf{r}(t) = \langle v_0 \cos\theta \cdot t, \; v_0 \sin\theta \cdot t - \tfrac{1}{2}gt^2 \rangle.
\]

Esto modela la trayectoria parabólica de un proyectil bajo gravedad.

\subsubsection{Integral de una función
vectorial}\label{integral-de-una-funciuxf3n-vectorial}

si

\[
\mathbf{r}(t) = \langle x(t), y(t), z(t) \rangle,
\]

entonces

\[
\int \mathbf{r}(t)\,dt = \left\langle \int x(t)\,dt, \; \int y(t)\,dt, \; \int z(t)\,dt \right\rangle + \mathbf{C},
\]

donde \(\mathbf{C}\) es un vector constante.

\subsubsection{Ejemplo}\label{ejemplo}

\[
\mathbf{r}(t) = \langle t, t^2, t^3 \rangle.
\]

\begin{itemize}
\tightlist
\item
  Derivado: \(\mathbf{r}'(t) = \langle 1, 2t, 3t^2 \rangle\).
\item
  Integrales:
\end{itemize}

\[
\int \mathbf{r}(t)\,dt = \langle \tfrac{1}{2}t^2, \tfrac{1}{3}t^3, \tfrac{1}{4}t^4 \rangle + \mathbf{C}.
\]

\subsubsection{Por qué esto es
importante}\label{por-quuxe9-esto-es-importante-9}

\begin{itemize}
\tightlist
\item
  Las derivadas de funciones vectoriales describen el movimiento y las
  fuerzas en el espacio.
\item
  Las integrales dan desplazamiento, trabajo y cantidades acumuladas.
\item
  Estas herramientas conectan el cálculo directamente con la física y la
  ingeniería.
\end{itemize}

\subsubsection{Ejercicios}\label{ejercicios-26}

\begin{enumerate}
\def\labelenumi{\arabic{enumi}.}
\tightlist
\item
  Para \(\mathbf{r}(t) = \langle t, \cos t, \sin t \rangle\), encuentra
  la velocidad, la rapidez y la aceleración.2. Calcula
  \(\mathbf{r}'(t)\) para
  \(\mathbf{r}(t) = \langle e^t, \ln t, t^2 \rangle\).
\item
  Integra \(\mathbf{r}(t) = \langle 1, t, t^2 \rangle\).
\item
  Una partícula tiene velocidad
  \(\mathbf{v}(t) = \langle t, 2, 0 \rangle\). Encuentre su vector de
  posición si \(\mathbf{r}(0) = \langle 1, 0, 0 \rangle\).
\item
  Demuestra que la rapidez de
  \(\mathbf{r}(t) = \langle \cos t, \sin t, 0 \rangle\) es constante.
\end{enumerate}

\subsection{7.3 Longitud y curvatura del
arco}\label{longitud-y-curvatura-del-arco}

El cálculo vectorial proporciona herramientas para medir no sólo la
trayectoria trazada por una curva sino también la intensidad con la que
se curva. Estos se expresan mediante la longitud del arco y la
curvatura.

\subsubsection{Longitud de arco de una curva
espacial}\label{longitud-de-arco-de-una-curva-espacial}

Si una curva está dada por

\[
\mathbf{r}(t) = \langle x(t), y(t), z(t) \rangle, \quad a \leq t \leq b,
\]

entonces la longitud del arco es

\[
L = \int_a^b |\mathbf{r}'(t)|\,dt,
\]

donde

\[
|\mathbf{r}'(t)| = \sqrt{(x'(t))^2 + (y'(t))^2 + (z'(t))^2}.
\]

Ejemplo: Para la hélice
\(\mathbf{r}(t) = \langle \cos t, \sin t, t \rangle, \, 0 \leq t \leq 2\pi\):

\begin{itemize}
\tightlist
\item
  Velocidad: \(\mathbf{r}'(t) = \langle -\sin t, \cos t, 1 \rangle\).
\item
  Velocidad:
  \(|\mathbf{r}'(t)| = \sqrt{(-\sin t)^2 + (\cos t)^2 + 1^2} = \sqrt{2}\).
\item
  Longitud del arco:
\end{itemize}

\[
L = \int_0^{2\pi} \sqrt{2}\,dt = 2\pi\sqrt{2}.
\]

\subsubsection{Curvatura}\label{curvatura}

La curvatura mide la rapidez con la que una curva cambia de dirección.

Para una curva suave \(\mathbf{r}(t)\):

\[
\kappa(t) = \frac{|\mathbf{r}'(t) \times \mathbf{r}''(t)|}{|\mathbf{r}'(t)|^3}.
\]

\begin{itemize}
\tightlist
\item
  \(\kappa = 0\): línea recta.
\item
  Más grande \(\kappa\): la curva se curva más bruscamente.
\end{itemize}

Ejemplo: Para un círculo de radio \(r\):

\[
\mathbf{r}(t) = \langle r\cos t, r\sin t \rangle.
\]

Luego \(\kappa = \tfrac{1}{r}\). Entonces la curvatura es constante e
inversamente proporcional al radio.

\subsubsection{Unidades tangentes y vectores
normales}\label{unidades-tangentes-y-vectores-normales}

\begin{itemize}
\tightlist
\item
  Vector tangente:
\end{itemize}

\[
\mathbf{T}(t) = \frac{\mathbf{r}'(t)}{|\mathbf{r}'(t)|}.
\]

\begin{itemize}
\tightlist
\item
  Vector normal: apunta hacia el centro de curvatura, definido como
\end{itemize}

\[
\mathbf{N}(t) = \frac{\mathbf{T}'(t)}{|\mathbf{T}'(t)|}.
\]

Estos vectores describen la geometría del movimiento: dirección de
desplazamiento y dirección de giro.

\subsubsection{Por qué esto es
importante}\label{por-quuxe9-esto-es-importante-10}

\begin{itemize}
\tightlist
\item
  La longitud del arco generaliza el concepto de distancia a las curvas
  en el espacio.
\item
  La curvatura describe la flexión, crucial en física (aceleración
  centrípeta), ingeniería (carreteras, montañas rusas) y gráficos por
  computadora.
\end{itemize}

\subsubsection{Ejercicios}\label{ejercicios-27}

\begin{enumerate}
\def\labelenumi{\arabic{enumi}.}
\tightlist
\item
  Encuentra la longitud del arco de
  \(\mathbf{r}(t) = \langle t, t^2, 0 \rangle\) desde \(t=0\) hasta
  \(t=1\).
\item
  Calcula la curvatura del círculo
  \(\mathbf{r}(t) = \langle \cos t, \sin t \rangle\).
\item
  Para \(\mathbf{r}(t) = \langle t, \cos t, \sin t \rangle\), calcula
  \(|\mathbf{r}'(t)|\).
\item
  Demuestra que una recta tiene curvatura \(\kappa = 0\).5. Encuentra el
  vector tangente a \(\mathbf{r}(t) = \langle e^t, e^{-t}, t \rangle\)
  en \(t=0\).
\end{enumerate}

\subsection{7.4 Movimiento en el
espacio}\label{movimiento-en-el-espacio}

Las funciones vectoriales son especialmente poderosas para describir el
movimiento en dos o tres dimensiones. La posición, la velocidad y la
aceleración se expresan naturalmente mediante derivadas e integrales de
funciones con valores vectoriales.

\subsubsection{Posición, velocidad y
aceleración}\label{posiciuxf3n-velocidad-y-aceleraciuxf3n}

\begin{itemize}
\tightlist
\item
  Vector de posición:
\end{itemize}

\[
\mathbf{r}(t) = \langle x(t), y(t), z(t) \rangle
\]

\begin{itemize}
\tightlist
\item
  Vector de velocidad (derivada de la posición):
\end{itemize}

\[
\mathbf{v}(t) = \mathbf{r}'(t) = \langle x'(t), y'(t), z'(t) \rangle
\]

\begin{itemize}
\tightlist
\item
  Velocidad (magnitud de la velocidad):
\end{itemize}

\[
|\mathbf{v}(t)| = \sqrt{(x'(t))^2 + (y'(t))^2 + (z'(t))^2}
\]

\begin{itemize}
\tightlist
\item
  Vector de aceleración (derivada de la velocidad):
\end{itemize}

\[
\mathbf{a}(t) = \mathbf{v}'(t) = \mathbf{r}''(t).
\]

\subsubsection{Componentes tangenciales y
normales}\label{componentes-tangenciales-y-normales}

La aceleración se puede descomponer en dos componentes:

\[
\mathbf{a}(t) = a_T \mathbf{T}(t) + a_N \mathbf{N}(t),
\]

donde:

\begin{itemize}
\tightlist
\item
  \(\mathbf{T}(t)\) = vector unitario tangente,
\item
  \(\mathbf{N}(t)\) = vector normal principal,
\item
  \(a_T = \frac{d}{dt}|\mathbf{v}(t)|\) = aceleración tangencial (cambio
  de velocidad),
\item
  \(a_N = \kappa |\mathbf{v}(t)|^2\) = aceleración normal (cambio de
  dirección).
\end{itemize}

\subsubsection{Movimiento de proyectil en
3D}\label{movimiento-de-proyectil-en-3d}

Con gravedad actuando en la dirección \(-z\):

\[
\mathbf{r}(t) = \langle v_0 \cos\theta \cos\phi \cdot t,\; v_0 \cos\theta \sin\phi \cdot t,\; v_0 \sin\theta \cdot t - \tfrac{1}{2}gt^2 \rangle,
\]

donde \(v_0\) es la velocidad inicial, \(\theta\) el ángulo de
lanzamiento y \(\phi\) la dirección azimutal.

\subsubsection{Ejemplo: movimiento
helicoidal}\label{ejemplo-movimiento-helicoidal}

\[
\mathbf{r}(t) = \langle \cos t, \sin t, t \rangle
\]

\begin{itemize}
\tightlist
\item
  Velocidad: \(\mathbf{v}(t) = \langle -\sin t, \cos t, 1 \rangle\).
\item
  Velocidad: \(|\mathbf{v}(t)| = \sqrt{2}\).
\item
  Aceleración: \(\mathbf{a}(t) = \langle -\cos t, -\sin t, 0 \rangle\).
\item
  El movimiento es uniforme en velocidad, en espiral hacia arriba.
\end{itemize}

\subsubsection{Por qué esto es
importante}\label{por-quuxe9-esto-es-importante-11}

\begin{itemize}
\tightlist
\item
  Proporciona lenguaje matemático para el movimiento del mundo real.
\item
  Imprescindible en física (fuerzas, trayectorias, movimiento circular).
\item
  Fundamentos de mecánica avanzada y modelos de ingeniería.
\end{itemize}

\subsubsection{Ejercicios}\label{ejercicios-28}

\begin{enumerate}
\def\labelenumi{\arabic{enumi}.}
\tightlist
\item
  Una partícula se mueve a lo largo de
  \(\mathbf{r}(t) = \langle t, t^2, t^3 \rangle\). Encuentre la
  velocidad y la aceleración en \(t=1\).
\item
  Demuestre que la velocidad es constante para la hélice
  \(\mathbf{r}(t) = \langle \cos t, \sin t, t \rangle\).
\item
  Se lanza un proyectil con \(v_0 = 20 \,\text{m/s}\) en un ángulo de
  \(45^\circ\). Escriba su vector de posición suponiendo movimiento en
  un plano vertical.
\item
  Para \(\mathbf{r}(t) = \langle e^t, e^{-t}, t \rangle\), encuentre
  \(\mathbf{v}(t)\) y \(\mathbf{a}(t)\).
\item
  Descomponga el vector de aceleración en componentes tangencial y
  normal para el movimiento a lo largo de un círculo de radio \(r\).\#
  Capítulo 8. Funciones de varias variables
\end{enumerate}

\subsection{8.1 Límites y continuidad en varias
variables}\label{luxedmites-y-continuidad-en-varias-variables}

En cálculo multivariable, las funciones pueden depender de dos o más
variables, como \(f(x,y)\) o \(f(x,y,z)\). Los conceptos de límites y
continuidad se extienden naturalmente del cálculo de una sola variable,
pero son más sutiles porque debemos considerar todos los caminos
posibles de aproximación.

\subsubsection{Límites en dos
variables}\label{luxedmites-en-dos-variables}

Para una función \(f(x,y)\), decimos

\[
\lim_{(x,y) \to (a,b)} f(x,y) = L
\]

si \(f(x,y)\) se acerca arbitrariamente a \(L\) cuando \((x,y)\) se
acerca a \((a,b)\) en cualquier camino.

Si diferentes caminos dan valores límite diferentes, entonces el límite
no existe.

Ejemplo 1 (existe límite):

\[
f(x,y) = x^2 + y^2, \quad \lim_{(x,y) \to (0,0)} f(x,y) = 0.
\]

Ejemplo 2 (el límite no existe):

\[
f(x,y) = \frac{xy}{x^2+y^2}, \quad (x,y) \to (0,0).
\]

\begin{itemize}
\tightlist
\item
  A lo largo de \(y=0\), la función es 0.
\item
  A lo largo de \(y=x\), la función es \(\tfrac{1}{2}\). Resultados
  diferentes → el límite no existe.
\end{itemize}

\subsubsection{Continuidad}\label{continuidad-1}

Una función \(f(x,y)\) es continua en \((a,b)\) si

\[
\lim_{(x,y)\to(a,b)} f(x,y) = f(a,b).
\]

Los polinomios y las funciones racionales (donde denominador ≠ 0) son
continuos en todos sus dominios.

\subsubsection{Extensión a tres o más
variables}\label{extensiuxf3n-a-tres-o-muxe1s-variables}

Para \(f(x,y,z)\), los límites y la continuidad se definen de la misma
manera, pero el punto \((a,b,c)\) debe acercarse desde infinitas
direcciones en el espacio.

\subsubsection{Por qué esto es
importante}\label{por-quuxe9-esto-es-importante-12}

\begin{itemize}
\tightlist
\item
  La continuidad garantiza que no haya saltos, agujeros o asíntotas en
  funciones multivariables.
\item
  Los límites son fundamentales para definir derivadas parciales e
  integrales múltiples.
\item
  Estos conceptos son componentes básicos del cálculo multivariable.
\end{itemize}

\subsubsection{Ejercicios}\label{ejercicios-29}

\begin{enumerate}
\def\labelenumi{\arabic{enumi}.}
\tightlist
\item
  Determinar si existe \(\lim_{(x,y)\to(0,0)} (x^2+y^2)\).
\item
  Demuestre que \(\lim_{(x,y)\to(0,0)} \frac{x^2y}{x^2+y^2} = 0\) a lo
  largo de todos los caminos rectilíneos \(y=mx\).
\item
  ¿Existe el límite para \(f(x,y) = \frac{x^2-y^2}{x^2+y^2}\) como
  \((x,y)\to(0,0)\)?
\item
  Explique por qué los polinomios de dos variables son continuos en
  todas partes.
\item
  Da un ejemplo de una función de dos variables que sea discontinua en
  un punto y explica por qué.
\end{enumerate}

\subsection{8.2 Derivados Parciales}\label{derivados-parciales}

En funciones de varias variables, a menudo queremos medir cómo cambia la
función cuando solo cambia una variable mientras las demás se mantienen
constantes. Esto lleva a la idea de derivadas parciales.

\subsubsection{Definición}\label{definiciuxf3n-7}

Para una función \(f(x,y)\), la derivada parcial respecto de \(x\) en un
punto \((a,b)\) es

\[
\frac{\partial f}{\partial x}(a,b) = \lim_{h \to 0} \frac{f(a+h, b) - f(a,b)}{h}.
\]

De manera similar, la derivada parcial con respecto a \(y\) es

\[\frac{\partial f}{\partial y}(a,b) = \lim_{h \to 0} \frac{f(a, b+h) - f(a,b)}{h}.
\]

We treat all other variables as constants when differentiating.

\subsubsection{Notation}\label{notation}

\begin{itemize}
\tightlist
\item
  \(\frac{\partial f}{\partial x}\), \(f_x\), \(\partial_x f\).
\item
  \(\frac{\partial f}{\partial y}\), \(f_y\), \(\partial_y f\).
\end{itemize}

For three variables \(f(x,y,z)\), we also have \(f_x, f_y, f_z\).

\subsubsection{Examples}\label{examples-2}

\begin{enumerate}
\def\labelenumi{\arabic{enumi}.}
\tightlist
\item
  \(f(x,y) = x^2y + y^3\)
\end{enumerate}

\begin{itemize}
\tightlist
\item
  \(f_x = 2xy\).
\item
  \(f_y = x^2 + 3y^2\).
\end{itemize}

\begin{enumerate}
\def\labelenumi{\arabic{enumi}.}
\setcounter{enumi}{1}
\tightlist
\item
  \(f(x,y) = e^{xy}\)
\end{enumerate}

\begin{itemize}
\tightlist
\item
  \(f_x = y e^{xy}\).
\item
  \(f_y = x e^{xy}\).
\end{itemize}

\begin{enumerate}
\def\labelenumi{\arabic{enumi}.}
\setcounter{enumi}{2}
\tightlist
\item
  \(f(x,y,z) = x^2 + yz\)
\end{enumerate}

\begin{itemize}
\tightlist
\item
  \(f_x = 2x\).
\item
  \(f_y = z\).
\item
  \(f_z = y\).
\end{itemize}

\subsubsection{Higher-Order Partial
Derivatives}\label{higher-order-partial-derivatives}

We can take partial derivatives repeatedly:

\begin{itemize}
\tightlist
\item
  \(f_{xx} = \frac{\partial}{\partial x}\Big(f_x\Big)\).
\item
  \(f_{yy}, f_{xy}, f_{yx}\), etc.
\end{itemize}

Clairaut's Theorem: If \(f\) has continuous second partial derivatives,
then

\[
f_{xy} = f_{yx}.
\]

\subsubsection{Geometric Meaning}\label{geometric-meaning}

\begin{itemize}
\tightlist
\item
  \(f_x\): slope of the surface in the \(x\)-direction.
\item
  \(f_y\): slope of the surface in the \(y\)-direction.
\item
  Together they describe how the surface tilts.
\end{itemize}

\subsubsection{Why This Matters}\label{why-this-matters}

\begin{itemize}
\tightlist
\item
  Partial derivatives are the foundation of gradients, tangent planes,
  and optimization in multiple variables.
\item
  They are widely used in physics, engineering, and economics to model
  systems with several inputs.
\end{itemize}

\subsubsection{Exercises}\label{exercises-3}

\begin{enumerate}
\def\labelenumi{\arabic{enumi}.}
\tightlist
\item
  Find \(f_x\) and \(f_y\) for \(f(x,y) = x^3y^2\).
\item
  Compute \(f_x, f_y, f_z\) for \(f(x,y,z) = xyz + x^2\).
\item
  Verify Clairaut's theorem for \(f(x,y) = x^2y + y^3\).
\item
  Interpret geometrically what \(f_x\) and \(f_y\) mean for
  \(f(x,y) = \sqrt{x^2+y^2}\).
\item
  Find all second-order partial derivatives of \(f(x,y) = e^{x^2+y^2}\).
\end{enumerate}

\subsection{8.3 Gradient and Directional
Derivatives}\label{gradient-and-directional-derivatives}

Partial derivatives measure change along the coordinate axes, but
sometimes we want to know the rate of change of a function in any
direction. This leads to the concepts of the gradient and directional
derivatives.

\subsubsection{Gradient Vector}\label{gradient-vector}

For a function \(f(x,y)\), the gradient is the vector

\[
\nabla f(x,y) = \left\langle \frac{\partial f}{\partial x}, \frac{\partial f}{\partial y} \right\rangle.
\]

For three variables \(f(x,y,z)\):

\[
\nabla f(x,y,z) = \left\langle f_x, f_y, f_z \right\rangle.
\]

The gradient points in the direction of maximum increase of the
function, and its magnitude gives the steepest slope.

\subsubsection{Directional Derivatives}\label{directional-derivatives}

The rate of change of \(f(x,y)\) at a point in the direction of a unit
vector \(\mathbf{u} = \langle u_1, u_2 \rangle\) is

\[
D_{\mathbf{u}} f(x,y) = \nabla f(x,y) \cdot \mathbf{u}.
\]

Este es el producto escalar del gradiente con el vector de dirección.

\subsubsection{Ejemplos}\label{ejemplos-18}

\begin{enumerate}
\def\labelenumi{\arabic{enumi}.}
\tightlist
\item
  \(f(x,y) = x^2 + y^2\)
\end{enumerate}

\begin{itemize}
\tightlist
\item
  Degradado: \(\nabla f = \langle 2x, 2y \rangle\).- En (1,2):
  \(\nabla f = \langle 2,4 \rangle\).
\item
  Derivada direccional según
  \(\mathbf{u} = \langle \tfrac{3}{5}, \tfrac{4}{5} \rangle\):
\end{itemize}

\[
D_{\mathbf{u}} f(1,2) = \langle 2,4 \rangle \cdot \langle \tfrac{3}{5}, \tfrac{4}{5} \rangle = \tfrac{26}{5}.
\]

\begin{enumerate}
\def\labelenumi{\arabic{enumi}.}
\setcounter{enumi}{1}
\tightlist
\item
  \(f(x,y,z) = x y z\)
\end{enumerate}

\begin{itemize}
\tightlist
\item
  Degradado: \(\nabla f = \langle yz, xz, xy \rangle\).
\item
  En (1,1,1): \(\nabla f = \langle 1,1,1 \rangle\).
\item
  La dirección de aumento máximo es a lo largo de
  \(\langle 1,1,1 \rangle\).
\end{itemize}

\subsubsection{Interpretación
geométrica}\label{interpretaciuxf3n-geomuxe9trica-1}

\begin{itemize}
\tightlist
\item
  El vector gradiente es perpendicular (normal) a las curvas de nivel o
  superficies de nivel de \(f\).
\item
  Las derivadas direccionales generalizan la pendiente en direcciones
  arbitrarias.
\end{itemize}

\subsubsection{Por qué esto es
importante}\label{por-quuxe9-esto-es-importante-13}

\begin{itemize}
\tightlist
\item
  En optimización, el gradiente nos indica la dirección a seguir para el
  ascenso o descenso más pronunciado.
\item
  En física, los gradientes describen campos como el flujo de calor y el
  potencial eléctrico.
\item
  Las derivadas direccionales unifican tasas de cambio de una y varias
  variables.
\end{itemize}

\subsubsection{Ejercicios}\label{ejercicios-30}

\begin{enumerate}
\def\labelenumi{\arabic{enumi}.}
\tightlist
\item
  Calcula \(\nabla f(x,y)\) para \(f(x,y) = e^{xy}\).
\item
  Encuentra el gradiente de \(f(x,y,z) = x^2+y^2+z^2\) y evalúa en
  (1,1,1).
\item
  Calcula la derivada direccional de \(f(x,y) = x^2-y\) en (2,1) en la
  dirección de \(\mathbf{u} = \langle 0,1 \rangle\).
\item
  Demuestra que la pendiente de \(f(x,y) = x^2+y^2\) es perpendicular al
  círculo \(x^2+y^2=1\).
\item
  Encuentre la dirección del vector unitario que maximiza la derivada
  direccional de \(f(x,y) = xy\) en (1,2).
\end{enumerate}

\subsection{8.4 Planos tangentes y aproximaciones
lineales}\label{planos-tangentes-y-aproximaciones-lineales}

En el cálculo de una sola variable, la recta tangente se aproxima a una
curva cerca de un punto. En cálculo multivariable, el concepto análogo
es el plano tangente, que proporciona una aproximación lineal a una
superficie cercana a un punto.

\subsubsection{Plano tangente a una
superficie}\label{plano-tangente-a-una-superficie}

Supongamos que \(z = f(x,y)\) es diferenciable en \((a,b)\). El plano
tangente en \((a,b,f(a,b))\) viene dado por

\[
z = f(a,b) + f_x(a,b)(x-a) + f_y(a,b)(y-b).
\]

Este plano toca la superficie en el punto y se aproxima a él cerca.

\subsubsection{Ejemplo 1: Paraboloide}\label{ejemplo-1-paraboloide}

Por \(f(x,y) = x^2 + y^2\) a \((1,2)\):

\begin{itemize}
\tightlist
\item
  \(f(1,2) = 1^2+2^2=5\).
\item
  \(f_x = 2x\), entonces \(f_x(1,2) = 2\).
\item
  \(f_y = 2y\), entonces \(f_y(1,2) = 4\).
\end{itemize}

Ecuación del plano tangente:

\[
z = 5 + 2(x-1) + 4(y-2).
\]

\subsubsection{Aproximación lineal}\label{aproximaciuxf3n-lineal}

El plano tangente se puede utilizar para aproximar \(f(x,y)\) cerca de
\((a,b)\):

\[
f(x,y) \approx f(a,b) + f_x(a,b)(x-a) + f_y(a,b)(y-b).
\]

Esta es la linealización de \(f\) en \((a,b)\).

\subsubsection{Ejemplo 2: Aproximación
lineal}\label{ejemplo-2-aproximaciuxf3n-lineal}

Aproximadamente \(f(x,y) = \sqrt{x+y}\) cerca de \((4,5)\).

\begin{itemize}
\tightlist
\item
  \(f(4,5) = \sqrt{9} = 3\).
\item
  \(f_x = \frac{1}{2\sqrt{x+y}}, \quad f_y = \frac{1}{2\sqrt{x+y}}\).
\item
  En (4,5): \(f_x = f_y = \tfrac{1}{6}\).
\end{itemize}

Entonces,

\[f(x,y) \approx 3 + \tfrac{1}{6}(x-4) + \tfrac{1}{6}(y-5).
\]

\subsubsection{Why This Matters}\label{why-this-matters-1}

\begin{itemize}
\tightlist
\item
  Tangent planes give the best linear approximation to a surface.
\item
  Linearization simplifies complex functions for computation.
\item
  Widely used in numerical methods, physics, and economics.
\end{itemize}

\subsubsection{Exercises}\label{exercises-4}

\begin{enumerate}
\def\labelenumi{\arabic{enumi}.}
\tightlist
\item
  Find the tangent plane to \(z = x^2y + y^2\) at \((1,1)\).
\item
  Approximate \(f(x,y) = e^{x+y}\) near \((0,0)\).
\item
  Derive the tangent plane equation for \(z = \ln(x^2+y^2)\) at
  \((1,1)\).
\item
  Use linear approximation to estimate \(\sqrt{10.1}\) using
  \(f(x,y) = \sqrt{x+y}\) near (4,6).
\item
  Explain why the tangent plane approximation improves as \((x,y)\) gets
  closer to \((a,b)\).
\end{enumerate}

\subsection{8.5 Optimization in Several
Variables}\label{optimization-in-several-variables}

Optimization in multivariable calculus extends the ideas of maxima and
minima from single-variable functions to functions of two or more
variables.

\subsubsection{Critical Points}\label{critical-points}

For \(f(x,y)\), a critical point occurs where

\[
f_x(x,y) = 0 \quad \text{y} \quad f_y(x,y) = 0,
\]

or where the partial derivatives do not exist.

\subsubsection{Second Derivative Test}\label{second-derivative-test}

To classify critical points, compute the second partial derivatives:

\[
D = f_{xx}(a,b) f_{yy}(a,b) - \big(f_{xy}(a,b)\big)^2.
\]

\begin{itemize}
\tightlist
\item
  If \(D > 0\) and \(f_{xx}(a,b) > 0\): local minimum.
\item
  If \(D > 0\) and \(f_{xx}(a,b) < 0\): local maximum.
\item
  If \(D < 0\): saddle point.
\item
  If \(D = 0\): test is inconclusive.
\end{itemize}

\subsubsection{Example 1: Paraboloid}\label{example-1-paraboloid}

\(f(x,y) = x^2 + y^2\).

\begin{itemize}
\tightlist
\item
  \(f_x = 2x, f_y = 2y\). Critical point at (0,0).
\item
  \(f_{xx} = 2, f_{yy} = 2, f_{xy} = 0\).
\item
  \(D = (2)(2) - 0 = 4 > 0\), and \(f_{xx} > 0\).
\item
  So (0,0) is a local minimum.
\end{itemize}

\subsubsection{Example 2: Saddle Point}\label{example-2-saddle-point}

\(f(x,y) = x^2 - y^2\).

\begin{itemize}
\tightlist
\item
  \(f_x = 2x, f_y = -2y\). Critical point at (0,0).
\item
  \(f_{xx} = 2, f_{yy} = -2, f_{xy} = 0\).
\item
  \(D = (2)(-2) - 0 = -4 < 0\).
\item
  So (0,0) is a saddle point.
\end{itemize}

\subsubsection{Constrained Optimization and Lagrange
Multipliers}\label{constrained-optimization-and-lagrange-multipliers}

Sometimes, we want to optimize \(f(x,y)\) subject to a constraint
\(g(x,y) = c\).

Method of Lagrange multipliers: solve

\[
\nabla f(x,y) = \lambda \nabla g(x,y).
\]

Ejemplo: Maximizar \(f(x,y) = xy\) sujeto a \(x^2+y^2=1\).

\begin{itemize}
\tightlist
\item
  Degradados:
  \(\nabla f = \langle y,x \rangle, \quad \nabla g = \langle 2x,2y \rangle\).
\item
  Ecuaciones: \(y = 2\lambda x, \, x = 2\lambda y\).
\item
  Las soluciones llevan al máximo a
  \((\pm \tfrac{1}{\sqrt{2}}, \pm \tfrac{1}{\sqrt{2}})\).
\end{itemize}

\subsubsection{Por qué esto es
importante}\label{por-quuxe9-esto-es-importante-14}

\begin{itemize}
\tightlist
\item
  La optimización es esencial en economía, ingeniería, aprendizaje
  automático y física.
\item
  Los multiplicadores de Lagrange permiten la optimización con
  restricciones, una herramienta clave en matemáticas aplicadas.
\end{itemize}

\subsubsection{Ejercicios}\label{ejercicios-31}

\begin{enumerate}
\def\labelenumi{\arabic{enumi}.}
\tightlist
\item
  Encuentra y clasifica los puntos críticos de \(f(x,y) = x^2+xy+y^2\).
\item
  Clasifica el punto (0,0) para \(f(x,y) = x^3-y^3\).3. Utilice la
  prueba de la segunda derivada para \(f(x,y) = x^4+y^4-4xy\).
\item
  Maximizar \(f(x,y) = x+y\) sujeto a \(x^2+y^2=1\).
\item
  Minimizar \(f(x,y) = x^2+2y^2\) sujeto a \(x+y=1\).
\end{enumerate}

\section{Capítulo 9. Integrales
Múltiples}\label{capuxedtulo-9.-integrales-muxfaltiples}

\subsection{9.1 Integrales dobles}\label{integrales-dobles}

En cálculo de una sola variable, una integral definida da el área bajo
una curva. En dos variables, una integral doble calcula el volumen bajo
una superficie (o más generalmente, la acumulación de valores sobre una
región).

\subsubsection{Definición}\label{definiciuxf3n-8}

Si \(f(x,y)\) es continua en una región \(R\), la integral doble es

\[
\iint_R f(x,y)\, dA = \lim_{m,n \to \infty} \sum_{i=1}^m \sum_{j=1}^n f(x_{ij}^-, y_{ij}^-) \Delta A,
\]

donde \(R\) se divide en pequeños rectángulos de área \(\Delta A\).

\subsubsection{Integrales iteradas}\label{integrales-iteradas}

Según el teorema de Fubini, podemos calcular una integral doble como una
integral iterada:

\[
\iint_R f(x,y)\, dA = \int_a^b \int_c^d f(x,y)\, dy\, dx,
\]

si \(R\) es un rectángulo \([a,b] \times [c,d]\).

El orden de integración a menudo se puede cambiar:

\[
\int_a^b \int_c^d f(x,y)\,dy\,dx = \int_c^d \int_a^b f(x,y)\,dx\,dy.
\]

\subsubsection{Ejemplos}\label{ejemplos-19}

\begin{enumerate}
\def\labelenumi{\arabic{enumi}.}
\tightlist
\item
  Región rectangular
\end{enumerate}

\[
\iint_R (x+y)\, dA, \quad R=[0,1]\times[0,2].
\]

\[
= \int_0^1 \int_0^2 (x+y)\,dy\,dx = \int_0^1 \Big[xy+\tfrac{1}{2}y^2\Big]_0^2 dx
= \int_0^1 (2x+2)dx = 3.
\]

\begin{enumerate}
\def\labelenumi{\arabic{enumi}.}
\setcounter{enumi}{1}
\tightlist
\item
  Región triangular
\end{enumerate}

\[
R = \{(x,y): 0 \leq x \leq 1, 0 \leq y \leq x\}.
\]

\[
\iint_R (x+y)\, dA = \int_0^1 \int_0^x (x+y)\,dy\,dx.
\]

Evaluando da \(\tfrac{2}{3}\).

\subsubsection{Aplicaciones}\label{aplicaciones-1}

\begin{itemize}
\tightlist
\item
  Volumen bajo una superficie:
\end{itemize}

\[
V = \iint_R f(x,y)\, dA.
\]

\begin{itemize}
\tightlist
\item
  Valor medio de una función sobre una región:
\end{itemize}

\[
f_{\text{avg}} = \frac{1}{A(R)} \iint_R f(x,y)\, dA.
\]

\subsubsection{Por qué esto es
importante}\label{por-quuxe9-esto-es-importante-15}

Las integrales dobles extienden la integración a dos dimensiones. Son
esenciales en física (masa, distribuciones de probabilidad), economía
(valores esperados) e ingeniería (centroides, flujo).

\subsubsection{Ejercicios}\label{ejercicios-32}

\begin{enumerate}
\def\labelenumi{\arabic{enumi}.}
\tightlist
\item
  Evalúe \(\iint_R (x^2+y^2)\, dA\) donde \(R=[0,1]\times[0,1]\).
\item
  Calcula \(\iint_R xy\, dA\) donde
  \(R=\{(x,y):0\leq x\leq2,0\leq y\leq x\}\).
\item
  Calcula el valor promedio de \(f(x,y) = x+y\) sobre el cuadrado
  unitario \([0,1]\times[0,1]\).
\item
  Interpretar \(\iint_R f(x,y)\, dA\) en términos de probabilidad si
  \(f(x,y)\) es una función de densidad de probabilidad.
\item
  Demuestre que cambiar el orden de integración da el mismo resultado
  para \(\iint_{[0,1]\times[0,2]} (x+y)\,dA\).
\end{enumerate}

\subsection{9.2 Integrales triples}\label{integrales-triples}

Las integrales triples amplían la idea de integración a tres variables,
permitiéndonos calcular volúmenes, masas y otras cantidades en regiones
tridimensionales.

\subsubsection{Definición}\label{definiciuxf3n-9}

Si \(f(x,y,z)\) es continua en una región sólida \(E\), la integral
triple es

\[\iiint_E f(x,y,z)\, dV = \lim_{m,n,p \to \infty} \sum f(x_{ijk}^-, y_{ijk}^-, z_{ijk}^-) \Delta V,
\]

where the region is subdivided into boxes of volume \(\Delta V\).

\subsubsection{Iterated Integrals}\label{iterated-integrals}

By Fubini's Theorem, a triple integral can be computed as an iterated
integral:

\[
\iiint_E f(x,y,z)\, dV = \int_a^b \int_c^d \int_e^f f(x,y,z)\, dz\, dy\, dx,
\]

for a rectangular box \(E = [a,b]\times[c,d]\times[e,f]\).

The order of integration can be chosen for convenience.

\subsubsection{Examples}\label{examples-3}

\begin{enumerate}
\def\labelenumi{\arabic{enumi}.}
\tightlist
\item
  Rectangular box
\end{enumerate}

\[
\iiint_E xyz\, dV, \quad E=[0,1]\times[0,2]\times[0,3].
\]

\[
= \int_0^1 \int_0^2 \int_0^3 xyz\,dz\,dy\,dx.
\]

First integrate over \(z\):

\[
\int_0^3 xyz\,dz = xy \left[\tfrac{1}{2}z^2\right]_0^3 = \tfrac{9}{2}xy.
\]

Now integrate over \(y\):

\[
\int_0^2 \tfrac{9}{2}xy\,dy = \tfrac{9}{2}x \cdot \left[\tfrac{1}{2}y^2\right]_0^2 = 9x.
\]

Finally integrate over \(x\):

\[
\int_0^1 9x\,dx = \tfrac{9}{2}.
\]

\begin{enumerate}
\def\labelenumi{\arabic{enumi}.}
\setcounter{enumi}{1}
\tightlist
\item
  Region bounded by planes Let
  \(E = \{(x,y,z) \mid 0 \leq x \leq 1, 0 \leq y \leq x, 0 \leq z \leq y\}\).
\end{enumerate}

\[
\iiint_E 1\,dV = \int_0^1 \int_0^x \int_0^y 1\,dz\,dy\,dx.
\]

Evaluate:

\[
= \int_0^1 \int_0^x y\,dy\,dx = \int_0^1 \tfrac{1}{2}x^2\,dx = \tfrac{1}{6}.
\]

So the volume of this triangular region is \(\tfrac{1}{6}\).

\subsubsection{Applications}\label{applications}

\begin{itemize}
\item
  Volume: \(V = \iiint_E 1 \, dV\).
\item
  Mass: If density is \(\rho(x,y,z)\), then

  \[
  M = \iiint_E \rho(x,y,z)\, dV.
  \]
\item
  Average value:

  \[
  f_{\text{promedio}} = \frac{1}{V(E)} \iiint_E f(x,y,z)\,dV.
  \]
\end{itemize}

\subsubsection{Why This Matters}\label{why-this-matters-2}

Triple integrals generalize area and volume calculations to arbitrary
solids. They are used in physics (mass distributions, center of mass,
gravitational fields), engineering, and probability.

\subsubsection{Exercises}\label{exercises-5}

\begin{enumerate}
\def\labelenumi{\arabic{enumi}.}
\tightlist
\item
  Compute \(\iiint_E (x+y+z)\,dV\) over the cube
  \(E=[0,1]\times[0,1]\times[0,1]\).
\item
  Find the volume of the tetrahedron bounded by
  \(x=0, y=0, z=0, x+y+z=1\).
\item
  Evaluate \(\iiint_E x^2 \, dV\) where
  \(E=[0,2]\times[0,1]\times[0,1]\).
\item
  Show that \(\iiint_E 1\,dV\) equals the geometric volume of \(E\).
\item
  If density is \(\rho(x,y,z)=x+y+z\), compute the mass of the unit
  cube.
\end{enumerate}

\subsection{9.3 Applications: Volume, Mass,
Probability}\label{applications-volume-mass-probability}

Triple integrals are powerful because they allow us to compute
quantities in three dimensions by accumulating values over a solid
region.

\subsubsection{Volume}\label{volume}

The simplest application is finding the volume of a region \(E\):

\[
V = \iiint_E 1 \, dV.
\]

Example: Find the volume of the solid bounded by the coordinate planes
and the plane \(x+y+z=1\).

\[
V = \iiint_E 1 \, dV = \int_0^1 \int_0^{1-x} \int_0^{1-x-y} 1 \, dz\, dy\, dx.
\]

Evaluando da \(V = \tfrac{1}{6}\).\#\#\# Masa y densidad

Si un sólido tiene función de densidad \(\rho(x,y,z)\), su masa es

\[
M = \iiint_E \rho(x,y,z)\, dV.
\]

El centro de masa está dado por

\[
\bar{x} = \frac{1}{M}\iiint_E x\rho(x,y,z)\,dV, \quad
\bar{y} = \frac{1}{M}\iiint_E y\rho(x,y,z)\,dV, \quad
\bar{z} = \frac{1}{M}\iiint_E z\rho(x,y,z)\,dV.
\]

Ejemplo: Para un cubo unitario con densidad constante \(\rho=1\), el
centro de masa está en \((0.5,0.5,0.5)\).

\subsubsection{Probabilidad}\label{probabilidad}

Si \(f(x,y,z)\) es una función de densidad de probabilidad en 3D,
entonces la probabilidad de que la variable aleatoria se encuentre en
una región \(E\) es

\[
P(E) = \iiint_E f(x,y,z)\, dV,
\]

donde \(f(x,y,z) \geq 0\) y

\[
\iiint_{\mathbb{R}^3} f(x,y,z)\,dV = 1.
\]

Ejemplo: Si \(f(x,y,z) = \tfrac{3}{4}z^2\) para \(0 \leq z \leq 1\),
uniformemente en \(x,y\), entonces

\[
P(0 \leq z \leq 0.5) = \int_0^{0.5} \tfrac{3}{4}z^2 \, dz = \tfrac{1}{32}.
\]

\subsubsection{Por qué esto es
importante}\label{por-quuxe9-esto-es-importante-16}

\begin{itemize}
\tightlist
\item
  Los volúmenes generalizan la geometría a sólidos irregulares.
\item
  Las integrales de masa y densidad conectan el cálculo con la física y
  la ingeniería.
\item
  Las funciones de densidad de probabilidad en dimensiones superiores se
  utilizan ampliamente en estadística y ciencia de datos.
\end{itemize}

\subsubsection{Ejercicios}\label{ejercicios-33}

\begin{enumerate}
\def\labelenumi{\arabic{enumi}.}
\tightlist
\item
  Encuentra el volumen del sólido acotado por \(x^2+y^2+z^2 \leq 1\) (la
  esfera unitaria).
\item
  Calcula la masa de un cono con densidad proporcional a \(z\).
\item
  Encuentra el centro de masa de un tetraedro uniforme acotado por
  \(x=0, y=0, z=0, x+y+z=1\).
\item
  Si \(f(x,y,z) = \frac{1}{8}\) en el cubo
  \([0,2]\times[0,2]\times[0,2]\), verifica que es una función de
  densidad de probabilidad.
\item
  Usa una integral triple para calcular la probabilidad de que un punto
  elegido al azar en la esfera unitaria tenga \(z > 0\).
\end{enumerate}

\subsection{9.4 Cambio de Variables: Coordenadas Polares, Cilíndricas,
Esféricas}\label{cambio-de-variables-coordenadas-polares-ciluxedndricas-esfuxe9ricas}

Muchas integrales se vuelven más fáciles cuando se expresan en sistemas
de coordenadas que coinciden con la simetría de la región. En lugar de
coordenadas cartesianas \((x,y,z)\), podemos utilizar coordenadas
polares, cilíndricas o esféricas.

\subsubsection{Coordenadas polares (2D)}\label{coordenadas-polares-2d}

Para funciones de dos variables, podemos cambiar a coordenadas polares:

\[
x = r\cos\theta, \quad y = r\sin\theta, \quad r \geq 0, \; 0 \leq \theta < 2\pi.
\]

El elemento área se transforma como

\[
dA = r\,dr\,d\theta.
\]

Ejemplo: Encuentra el área del círculo unitario.

\[
A = \iint_{x^2+y^2\leq 1} 1\,dA = \int_0^{2\pi}\int_0^1 r\,dr\,d\theta = \pi.
\]

\subsubsection{Coordenadas cilíndricas
(3D)}\label{coordenadas-ciluxedndricas-3d}

En 3D, las coordenadas cilíndricas extienden las coordenadas polares con
\(z\):

\[
x = r\cos\theta, \quad y = r\sin\theta, \quad z = z.
\]

El elemento de volumen es

\[
dV = r\,dr\,d\theta\,dz.
\]

Ejemplo: Volumen de un cilindro de radio \(R\) y altura \(h\):

\[
V = \int_0^h \int_0^{2\pi} \int_0^R r\,dr\,d\theta\,dz = \pi R^2 h.
\]\#\#\# Coordenadas esféricas (3D)

Para simetría esférica, utilice:

\[
x = \rho \sin\phi \cos\theta, \quad y = \rho \sin\phi \sin\theta, \quad z = \rho \cos\phi,
\]

donde

\begin{itemize}
\tightlist
\item
  \(\rho \geq 0\) es la distancia desde el origen,
\item
  \(0 \leq \phi \leq \pi\) es el ángulo formado por el eje positivo
  \(z\),
\item
  \(0 \leq \theta < 2\pi\) es el ángulo en el plano \(xy\).
\end{itemize}

El elemento de volumen es

\[
dV = \rho^2 \sin\phi \, d\rho\, d\phi\, d\theta.
\]

Ejemplo: Volumen de la esfera unitaria:

\[
V = \int_0^{2\pi} \int_0^\pi \int_0^1 \rho^2 \sin\phi \, d\rho\, d\phi\, d\theta.
\]

Evaluando:

\[
V = \left(\int_0^1 \rho^2 d\rho\right)\left(\int_0^\pi \sin\phi d\phi\right)\left(\int_0^{2\pi} d\theta\right) = \tfrac{1}{3}(2)(2\pi) = \tfrac{4\pi}{3}.
\]

\subsubsection{Por qué esto es
importante}\label{por-quuxe9-esto-es-importante-17}

\begin{itemize}
\tightlist
\item
  Las coordenadas polares simplifican las regiones circulares.
\item
  Las coordenadas cilíndricas manejan cilindros y simetría rotacional.
\item
  Las coordenadas esféricas simplifican problemas de esferas, conos y
  radiales.
\item
  Estos cambios de variables hacen manejables integrales que de otro
  modo serían imposibles.
\end{itemize}

\subsubsection{Ejercicios}\label{ejercicios-34}

\begin{enumerate}
\def\labelenumi{\arabic{enumi}.}
\tightlist
\item
  Calcula \(\iint_{x^2+y^2\leq 4} (x^2+y^2)\,dA\) usando coordenadas
  polares.
\item
  Encuentra el volumen de un cono de altura \(h\) y radio \(R\) usando
  coordenadas cilíndricas.
\item
  Usa coordenadas esféricas para evaluar el volumen de una bola de radio
  \(R\).
\item
  Demuestra que el factor jacobiano para las coordenadas polares es
  \(r\).
\item
  Encuentra la masa de una esfera sólida de radio \(R\) con densidad
  proporcional a la distancia desde el origen usando coordenadas
  esféricas.
\end{enumerate}

\section{Capítulo 10. Cálculo
vectorial}\label{capuxedtulo-10.-cuxe1lculo-vectorial}

\subsection{10.1 Campos vectoriales}\label{campos-vectoriales}

Un campo vectorial asigna un vector a cada punto en el espacio, de forma
muy similar a como una función escalar asigna un número. Los campos
vectoriales se utilizan para modelar flujos, fuerzas y otras cantidades
direccionales.

\subsubsection{Definición}\label{definiciuxf3n-10}

En dos dimensiones, un campo vectorial es una función.

\[
\mathbf{F}(x,y) = \langle P(x,y), Q(x,y) \rangle,
\]

donde \(P\) y \(Q\) son funciones escalares.

En tres dimensiones,

\[
\mathbf{F}(x,y,z) = \langle P(x,y,z), Q(x,y,z), R(x,y,z) \rangle.
\]

\subsubsection{Ejemplos}\label{ejemplos-20}

\begin{enumerate}
\def\labelenumi{\arabic{enumi}.}
\tightlist
\item
  Campo radial
\end{enumerate}

\[
\mathbf{F}(x,y) = \langle x, y \rangle.
\]

Los vectores apuntan hacia afuera desde el origen.

\begin{enumerate}
\def\labelenumi{\arabic{enumi}.}
\setcounter{enumi}{1}
\tightlist
\item
  Campo rotacional
\end{enumerate}

\[
\mathbf{F}(x,y) = \langle -y, x \rangle.
\]

Los vectores circulan alrededor del origen.

\begin{enumerate}
\def\labelenumi{\arabic{enumi}.}
\setcounter{enumi}{2}
\tightlist
\item
  Campo gravitacional
\end{enumerate}

\[
\mathbf{F}(x,y,z) = -\frac{GM}{r^3}\langle x,y,z \rangle, \quad r=\sqrt{x^2+y^2+z^2}.
\]

\subsubsection{Visualización de campos
vectoriales}\label{visualizaciuxf3n-de-campos-vectoriales}

\begin{itemize}
\tightlist
\item
  Dibujar pequeñas flechas en puntos de muestra para indicar dirección y
  magnitud.
\item
  Flechas más densas donde las magnitudes son mayores.
\item
  Útil para interpretar líneas de flujo, trayectorias y fuerzas.
\end{itemize}

\subsubsection{\texorpdfstring{Líneas de flujoUna línea de flujo (o
curva integral) de un campo vectorial es una curva \(\mathbf{r}(t)\)
cuyo vector tangente en cada punto coincide con el
campo:}{Líneas de flujoUna línea de flujo (o curva integral) de un campo vectorial es una curva \textbackslash mathbf\{r\}(t) cuyo vector tangente en cada punto coincide con el campo:}}\label{luxedneas-de-flujouna-luxednea-de-flujo-o-curva-integral-de-un-campo-vectorial-es-una-curva-mathbfrt-cuyo-vector-tangente-en-cada-punto-coincide-con-el-campo}

\[
\mathbf{r}'(t) = \mathbf{F}(\mathbf{r}(t)).
\]

Las líneas de flujo describen las trayectorias de las partículas en un
campo de velocidad.

\subsubsection{Por qué esto es
importante}\label{por-quuxe9-esto-es-importante-18}

\begin{itemize}
\tightlist
\item
  Los campos vectoriales son fundamentales en física (flujo de fluidos,
  electromagnetismo, gravitación).
\item
  Forman la base de las integrales de línea, las integrales de
  superficie y los grandes teoremas del cálculo vectorial (Green,
  Stokes, Divergencia).
\item
  Proporcionar una forma geométrica de representar cantidades
  direccionales.
\end{itemize}

\subsubsection{Ejercicios}\label{ejercicios-35}

\begin{enumerate}
\def\labelenumi{\arabic{enumi}.}
\tightlist
\item
  Dibuja el campo vectorial \(\mathbf{F}(x,y) = \langle y, -x \rangle\).
\item
  Determina si los vectores de \(\mathbf{F}(x,y) = \langle x,y \rangle\)
  apuntan hacia o lejos del origen.
\item
  Para \(\mathbf{F}(x,y,z) = \langle y, z, x \rangle\), calcule
  \(\mathbf{F}(1,2,3)\).
\item
  Describe las líneas de flujo de
  \(\mathbf{F}(x,y) = \langle -y, x \rangle\).
\item
  Explique por qué los campos gravitacionales y eléctricos son ejemplos
  de campos vectoriales radiales.
\end{enumerate}

\subsection{10.2 Integrales de línea}\label{integrales-de-luxednea}

Una integral de línea extiende la idea de una integral a funciones
evaluadas a lo largo de una curva. En lugar de integrarnos en un
intervalo o región, nos integramos en un camino en el espacio.

\subsubsection{Definición: Integral de línea
escalar}\label{definiciuxf3n-integral-de-luxednea-escalar}

Si \(f(x,y)\) es una función escalar y \(C\) es una curva parametrizada
por \(\mathbf{r}(t) = \langle x(t), y(t) \rangle, \; a \leq t \leq b\),
entonces la integral de línea es

\[
\int_C f(x,y)\, ds = \int_a^b f(x(t),y(t)) \, |\mathbf{r}'(t)|\, dt,
\]

donde \(ds\) es la longitud del arco.

Mide la acumulación de \(f\) a lo largo de la curva.

\subsubsection{Definición: Integral de línea
vectorial}\label{definiciuxf3n-integral-de-luxednea-vectorial}

Para un campo vectorial
\(\mathbf{F}(x,y) = \langle P(x,y), Q(x,y) \rangle\), la integral de
línea a lo largo de \(C\) es

\[
\int_C \mathbf{F} \cdot d\mathbf{r} = \int_a^b \mathbf{F}(\mathbf{r}(t)) \cdot \mathbf{r}'(t)\, dt.
\]

Mide el trabajo realizado por el campo a lo largo de la curva.

\subsubsection{Ejemplos}\label{ejemplos-21}

\begin{enumerate}
\def\labelenumi{\arabic{enumi}.}
\tightlist
\item
  Integral de línea escalar
\end{enumerate}

\[
f(x,y) = x+y, \quad C: \mathbf{r}(t) = \langle t, t^2 \rangle, \; 0 \leq t \leq 1.
\]

entonces

\[
\int_C f(x,y)\, ds = \int_0^1 (t+t^2)\sqrt{(1)^2+(2t)^2}\, dt.
\]

\begin{enumerate}
\def\labelenumi{\arabic{enumi}.}
\setcounter{enumi}{1}
\tightlist
\item
  Trabajo realizado por una fuerza
\end{enumerate}

\[
\mathbf{F}(x,y) = \langle y, x \rangle, \quad C: \mathbf{r}(t) = \langle t, t^2 \rangle, \; 0 \leq t \leq 1.
\]

\[
\int_C \mathbf{F} \cdot d\mathbf{r} = \int_0^1 \langle t^2, t \rangle \cdot \langle 1, 2t \rangle\, dt = \int_0^1 (t^2 + 2t^2)\, dt = \int_0^1 3t^2\, dt = 1.
\]

\subsubsection{Interpretación física}\label{interpretaciuxf3n-fuxedsica}

\begin{itemize}
\tightlist
\item
  Integral de línea escalar: acumulación de densidad a lo largo de un
  alambre.
\item
  Integral de línea vectorial: trabajo realizado por una fuerza que
  mueve un objeto a lo largo de una trayectoria.
\end{itemize}

\subsubsection{Por qué esto es importante- Las integrales de línea
conectan campos vectoriales con cantidades físicas como el trabajo y la
circulación.}\label{por-quuxe9-esto-es-importante--las-integrales-de-luxednea-conectan-campos-vectoriales-con-cantidades-fuxedsicas-como-el-trabajo-y-la-circulaciuxf3n.}

\begin{itemize}
\tightlist
\item
  Son componentes básicos del teorema de Green y del teorema de Stokes.
\item
  Aparecer en física (potencial eléctrico, flujo de fluidos, mecánica).
\end{itemize}

\subsubsection{Ejercicios}\label{ejercicios-36}

\begin{enumerate}
\def\labelenumi{\arabic{enumi}.}
\tightlist
\item
  Calcula \(\int_C (x^2+y^2)\, ds\) donde \(C\) es el segmento de recta
  de (0,0) a (1,1).
\item
  Calcula \(\int_C \mathbf{F}\cdot d\mathbf{r}\) para
  \(\mathbf{F}(x,y) = \langle -y, x \rangle\) a lo largo del círculo
  unitario \(x^2+y^2=1\).
\item
  Interpretar el significado de \(\int_C 1\,ds\).
\item
  Para \(\mathbf{F}(x,y,z) = \langle z,0,x \rangle\), calcula la
  integral de línea a lo largo de
  \(\mathbf{r}(t) = \langle t,t,1 \rangle, 0 \leq t \leq 1\).
\item
  Explique la diferencia entre integrales de línea escalares y
  vectoriales.
\end{enumerate}

\subsection{10.3 Integrales de
superficie}\label{integrales-de-superficie}

Una integral de superficie generaliza integrales de línea a superficies
bidimensionales en un espacio tridimensional. Nos permiten calcular el
flujo a través de superficies y la acumulación de campos escalares sobre
superficies curvas.

\subsubsection{Integral de superficie
escalar}\label{integral-de-superficie-escalar}

Si una superficie \(S\) está parametrizada por

\[
\mathbf{r}(u,v) = \langle x(u,v), y(u,v), z(u,v) \rangle,
\]

entonces la integral de superficie de una función escalar \(f(x,y,z)\)
es

\[
\iint_S f(x,y,z)\, dS = \iint_D f(\mathbf{r}(u,v)) \, |\mathbf{r}_u \times \mathbf{r}_v| \, du\,dv,
\]

donde \(\mathbf{r}_u\) y \(\mathbf{r}_v\) son derivadas parciales de
\(\mathbf{r}(u,v)\), y \(D\) es el dominio de parámetros.

\subsubsection{Integral de superficie vectorial
(flujo)}\label{integral-de-superficie-vectorial-flujo}

Para un campo vectorial \(\mathbf{F}(x,y,z)\), el flujo a través de una
superficie \(S\) es

\[
\iint_S \mathbf{F}\cdot d\mathbf{S} = \iint_S \mathbf{F}\cdot \mathbf{n}\, dS,
\]

donde \(\mathbf{n}\) es el vector normal unitario. Usando
parametrización,

\[
\iint_S \mathbf{F}\cdot d\mathbf{S} = \iint_D \mathbf{F}(\mathbf{r}(u,v)) \cdot (\mathbf{r}_u \times \mathbf{r}_v)\,du\,dv.
\]

\subsubsection{Ejemplos}\label{ejemplos-22}

\begin{enumerate}
\def\labelenumi{\arabic{enumi}.}
\tightlist
\item
  Integral de superficie escalar Superficie: plano \(z=1\) sobre disco
  unitario \(x^2+y^2 \leq 1\).
\end{enumerate}

\[
\iint_S 1\, dS = \text{area of the disk} = \pi.
\]

\begin{enumerate}
\def\labelenumi{\arabic{enumi}.}
\setcounter{enumi}{1}
\tightlist
\item
  Flujo a través de una esfera Sean
  \(\mathbf{F}(x,y,z) = \langle x,y,z \rangle\), y \(S\) = esfera de
  radio \(R\). El vector normal es
  \(\mathbf{n} = \frac{1}{R}\langle x,y,z \rangle\).
\end{enumerate}

\[
\mathbf{F}\cdot \mathbf{n} = \frac{x^2+y^2+z^2}{R} = R.
\]

entonces

\[
\iint_S \mathbf{F}\cdot d\mathbf{S} = \iint_S R\, dS = R \cdot 4\pi R^2 = 4\pi R^3.
\]

\subsubsection{Por qué esto es
importante}\label{por-quuxe9-esto-es-importante-19}

\begin{itemize}
\tightlist
\item
  Las integrales de superficie escalares miden distribuciones de área y
  superficie.
\item
  Las integrales de superficie vectoriales miden el flujo: la cantidad
  de un campo que pasa a través de una superficie.
\item
  Aplicaciones: electromagnetismo, flujo de fluidos, transferencia de
  calor y más.
\end{itemize}

\subsubsection{Ejercicios}\label{ejercicios-37}

\begin{enumerate}
\def\labelenumi{\arabic{enumi}.}
\tightlist
\item
  Calcula \(\iint_S 1\, dS\) para la superficie de un cubo de lado 2.2.
  Encuentra el flujo de \(\mathbf{F}(x,y,z) = \langle x,y,z \rangle\) a
  través de la esfera unitaria.
\item
  Calcula \(\iint_S z\, dS\) para el paraboloide
  \(z = x^2+y^2, \, z \leq 1\).
\item
  Para \(\mathbf{F}(x,y,z) = \langle y,0,0 \rangle\), calcule el flujo a
  través del plano \(x=1\), \(0 \leq y,z \leq 1\).
\item
  Explique físicamente qué significa que el flujo de un campo vectorial
  a través de una superficie cerrada sea cero.
\end{enumerate}

\subsection{10.4 Teorema de Green}\label{teorema-de-green}

El teorema de Green es un resultado fundamental en el cálculo vectorial
que conecta una integral de línea alrededor de una curva cerrada con una
integral doble sobre la región que encierra. Es una versión
bidimensional del teorema de Stokes.

\subsubsection{Declaración del teorema de
Green}\label{declaraciuxf3n-del-teorema-de-green}

Sea \(C\) una curva cerrada, simple y orientada positivamente en el
plano, y sea \(R\) la región que encierra. Si
\(\mathbf{F}(x,y) = \langle P(x,y), Q(x,y) \rangle\) tiene derivadas
parciales continuas en una región abierta que contiene \(R\), entonces

\[
\oint_C \mathbf{F} \cdot d\mathbf{r} = \oint_C P\,dx + Q\,dy = \iint_R \left( \frac{\partial Q}{\partial x} - \frac{\partial P}{\partial y} \right)\, dA.
\]

\subsubsection{Interpretación}\label{interpretaciuxf3n-2}

\begin{itemize}
\tightlist
\item
  La integral de recta alrededor de \(C\) mide la circulación del campo
  vectorial a lo largo de la frontera.
\item
  La integral doble sobre \(R\) mide la curvatura (rotación) total del
  campo dentro de la región.
\end{itemize}

\subsubsection{Ejemplo 1: Fórmula de
área}\label{ejemplo-1-fuxf3rmula-de-uxe1rea}

Si \(\mathbf{F} = \langle -y/2, x/2 \rangle\), entonces

\[
\frac{\partial Q}{\partial x} - \frac{\partial P}{\partial y} = 1.
\]

Por tanto, el teorema de Green da

\[
\text{Area}(R) = \iint_R 1\,dA = \oint_C \left(-\tfrac{y}{2}\,dx + \tfrac{x}{2}\,dy\right).
\]

Esto proporciona una manera de calcular el área usando una integral de
línea.

\subsubsection{Ejemplo 2: Circulación}\label{ejemplo-2-circulaciuxf3n}

Sean \(\mathbf{F}(x,y) = \langle -y, x \rangle\) y \(C\) el círculo
unitario.

\begin{itemize}
\tightlist
\item
  \(P=-y, Q=x\).
\item
  \(Q_x - P_y = 1 - (-1) = 2\).
\item
  Integral doble sobre el disco unitario:
\end{itemize}

\[
\iint_R 2\,dA = 2\pi (1^2) = 2\pi.
\]

Entonces la circulación alrededor del círculo es \(2\pi\).

\subsubsection{Por qué esto es
importante}\label{por-quuxe9-esto-es-importante-20}

\begin{itemize}
\tightlist
\item
  Convierte integrales de línea difíciles en integrales dobles, o
  viceversa.
\item
  Proporciona un puente entre las propiedades locales (rizo) y las
  propiedades globales (circulación).
\item
  Ampliamente utilizado en física para flujo de fluidos,
  electromagnetismo y campos vectoriales planos.
\end{itemize}

\subsubsection{Ejercicios}\label{ejercicios-38}

\begin{enumerate}
\def\labelenumi{\arabic{enumi}.}
\tightlist
\item
  Usa el teorema de Green para calcular el área dentro de la elipse
  \(\frac{x^2}{a^2} + \frac{y^2}{b^2} = 1\).
\item
  Verifica el teorema de Green para
  \(\mathbf{F}(x,y) = \langle -y, x \rangle\) a lo largo del cuadrado
  con vértices (0,0), (1,0), (1,1), (0,1).
\item
  Calcula la circulación de \(\mathbf{F}(x,y) = \langle -y, x \rangle\)
  alrededor del círculo unitario.4. Demuestre que si
  \(\nabla \times \mathbf{F} = 0\), entonces la integral de línea de
  \(\mathbf{F}\) alrededor de cualquier curva cerrada es cero.
\item
  Utilice el teorema de Green para demostrar que
\end{enumerate}

\[
\oint_C x\,dy = -\oint_C y\,dx
\]

para cualquier curva cerrada \(C\).

\subsection{10.5 Teorema de Stokes}\label{teorema-de-stokes}

El teorema de Stokes generaliza el teorema de Green a tres dimensiones.
Relaciona una integral de superficie de la curvatura de un campo
vectorial sobre una superficie con una integral de línea del campo
vectorial alrededor del límite de esa superficie.

\subsubsection{Declaración del teorema de
Stokes}\label{declaraciuxf3n-del-teorema-de-stokes}

Sea \(S\) una superficie lisa y orientada con curva límite \(C\)
(orientada positivamente). Si \(\mathbf{F}(x,y,z)\) es un campo
vectorial con derivadas parciales continuas, entonces

\[
\iint_S (\nabla \times \mathbf{F}) \cdot d\mathbf{S} = \oint_C \mathbf{F} \cdot d\mathbf{r}.
\]

\begin{itemize}
\tightlist
\item
  Lado izquierdo: flujo del rizo de \(\mathbf{F}\) a través de la
  superficie.
\item
  Lado derecho: circulación de \(\mathbf{F}\) por la curva límite.
\end{itemize}

\subsubsection{Interpretación}\label{interpretaciuxf3n-3}

\begin{itemize}
\tightlist
\item
  La integral de línea alrededor del límite es igual a la ``rotación''
  total dentro de la superficie.
\item
  Extiende el teorema de Green (un caso especial cuando la superficie se
  encuentra en el plano).
\end{itemize}

\subsubsection{Ejemplo 1: El teorema de Green como caso
especial}\label{ejemplo-1-el-teorema-de-green-como-caso-especial}

Si \(S\) es una región plana en el plano \(xy\), el teorema de Stokes se
reduce al teorema de Green.

\subsubsection{Ejemplo 2: Circulación en un
hemisferio}\label{ejemplo-2-circulaciuxf3n-en-un-hemisferio}

Sean \(\mathbf{F}(x,y,z) = \langle -y, x, 0 \rangle\) y \(S\) el
hemisferio superior de radio 1.

\begin{itemize}
\tightlist
\item
  Límite \(C\): círculo unitario en el plano \(xy\).
\item
  Según el teorema de Stokes:
\end{itemize}

\[
\oint_C \mathbf{F}\cdot d\mathbf{r} = \iint_S (\nabla \times \mathbf{F})\cdot d\mathbf{S}.
\]

\begin{itemize}
\tightlist
\item
  Rizado: \(\nabla \times \mathbf{F} = \langle 0,0,2 \rangle\).
\item
  Normal al hemisferio apunta hacia afuera:
  \(\mathbf{n} = \langle 0,0,1 \rangle\).
\item
  Entonces integrando = 2.
\item
  Área del hemisferio = \(2\pi (1^2)\).
\end{itemize}

\[
\iint_S 2\, dS = 2 \cdot 2\pi = 4\pi.
\]

Por tanto, la circulación alrededor del ecuador es de \(4\pi\).

\subsubsection{Por qué esto es
importante}\label{por-quuxe9-esto-es-importante-21}

\begin{itemize}
\tightlist
\item
  Proporciona una conexión profunda entre integrales de superficie e
  integrales de línea.
\item
  Simplifica los cálculos al permitir la elección de superficies
  convenientes.
\item
  Ampliamente utilizado en electromagnetismo (Ley de Faraday) y dinámica
  de fluidos.
\end{itemize}

\subsubsection{Ejercicios}\label{ejercicios-39}

\begin{enumerate}
\def\labelenumi{\arabic{enumi}.}
\tightlist
\item
  Verifique el teorema de Stokes para
  \(\mathbf{F}(x,y,z) = \langle y, -x, 0 \rangle\) sobre el disco
  unitario en el plano \(xy\).
\item
  Calcula \(\oint_C \mathbf{F}\cdot d\mathbf{r}\) donde
  \(\mathbf{F}(x,y,z) = \langle z, 0, x \rangle\) y \(C\) es el límite
  del triángulo con vértices (0,0,0), (1,0,0), (0,1,0).
\item
  Demuestre que si \(\nabla \times \mathbf{F} = 0\), entonces la
  circulación alrededor de cualquier curva cerrada es cero.4. Aplicar el
  teorema de Stokes para calcular la circulación de
  \(\mathbf{F}(x,y,z) = \langle -y, x, z \rangle\) alrededor del límite
  del cuadrado unitario en el plano \(z=0\).
\item
  Explique cómo el teorema de Stokes generaliza el teorema de Green.
\end{enumerate}

\subsection{10.6 Teorema de la
divergencia}\label{teorema-de-la-divergencia}

El teorema de la divergencia (también llamado teorema de Gauss)
relaciona el flujo de un campo vectorial a través de una superficie
cerrada con la integral triple de la divergencia del campo dentro de la
superficie.

\subsubsection{Declaración del teorema de la
divergencia}\label{declaraciuxf3n-del-teorema-de-la-divergencia}

Sea \(E\) una región sólida en \(\mathbb{R}^3\) con superficie límite
\(S\) (orientada hacia afuera). Si \(\mathbf{F}(x,y,z)\) es un campo
vectorial con derivadas parciales continuas en \(E\), entonces

\[
\iint_S \mathbf{F} \cdot d\mathbf{S} = \iiint_E (\nabla \cdot \mathbf{F}) \, dV.
\]

\begin{itemize}
\tightlist
\item
  Lado izquierdo: flujo de \(\mathbf{F}\) a través de la superficie
  cerrada \(S\).
\item
  Lado derecho: integral triple de la divergencia dentro de la región.
\end{itemize}

\subsubsection{Divergencia}\label{divergencia}

La divergencia de un campo vectorial
\(\mathbf{F}(x,y,z) = \langle P, Q, R \rangle\) es

\[
\nabla \cdot \mathbf{F} = \frac{\partial P}{\partial x} + \frac{\partial Q}{\partial y} + \frac{\partial R}{\partial z}.
\]

Mide la ``salida neta'' por unidad de volumen en cada punto.

\subsubsection{Ejemplo 1: flujo de un campo
radial}\label{ejemplo-1-flujo-de-un-campo-radial}

Sea \(\mathbf{F}(x,y,z) = \langle x, y, z \rangle\), y sea \(E\) la bola
unitaria \(x^2+y^2+z^2 \leq 1\).

\begin{itemize}
\tightlist
\item
  Divergencia: \(\nabla \cdot \mathbf{F} = 1+1+1 = 3\).
\item
  Volumen de bola unitaria: \(\tfrac{4}{3}\pi\). entonces
\end{itemize}

\[
\iiint_E (\nabla \cdot \mathbf{F})\, dV = 3 \cdot \tfrac{4}{3}\pi = 4\pi.
\]

Por tanto, el flujo a través de la esfera unitaria es \(4\pi\).

\subsubsection{Ejemplo 2: campo
constante}\label{ejemplo-2-campo-constante}

Sea \(\mathbf{F}(x,y,z) = \langle 1, 0, 0 \rangle\).

\begin{itemize}
\tightlist
\item
  Divergencia: \(\nabla \cdot \mathbf{F} = 0\).
\item
  Entonces, el flujo a través de cualquier superficie cerrada es cero,
  lo que es consistente con la intuición (no hay salida neta).
\end{itemize}

\subsubsection{Por qué esto es
importante}\label{por-quuxe9-esto-es-importante-22}

\begin{itemize}
\item
  Convierte integrales de superficie en integrales de volumen más
  simples.
\item
  Utilizado en física: Ley de Gauss en electromagnetismo, flujo de
  fluidos y transferencia de calor.
\item
  Completa el marco unificador:

  \begin{itemize}
  \tightlist
  \item
    Teorema de Green (curvatura 2D ↔ circulación)
  \item
    Teorema de Stokes (curvatura 3D ↔ circulación en superficies)
  \item
    Teorema de la divergencia (divergencia 3D ↔ flujo en superficies
    cerradas)
  \end{itemize}
\end{itemize}

\subsubsection{Ejercicios}\label{ejercicios-40}

\begin{enumerate}
\def\labelenumi{\arabic{enumi}.}
\tightlist
\item
  Utilice el teorema de la divergencia para calcular el flujo de
  \(\mathbf{F}(x,y,z) = \langle x,y,z \rangle\) a través de la
  superficie de una esfera de radio \(R\).
\item
  Verifica el Teorema de la Divergencia para
  \(\mathbf{F}(x,y,z) = \langle y, z, x \rangle\) en el cubo unitario
  \([0,1]^3\).
\item
  Demuestre que si \(\nabla \cdot \mathbf{F} = 0\), entonces el flujo
  total a través de cualquier superficie cerrada es cero.
\item
  Calcula el flujo de
  \(\mathbf{F}(x,y,z) = \langle x^2, y^2, z^2 \rangle\) a través de la
  esfera unitaria.5. Explique cómo el teorema de la divergencia
  generaliza el teorema fundamental unidimensional del cálculo.
\end{enumerate}

\section{Parte IV. Procesos
infinitos}\label{parte-iv.-procesos-infinitos}

\section{Capítulo 11. Secuencias y
convergencia.}\label{capuxedtulo-11.-secuencias-y-convergencia.}

\subsection{11.1 Definiciones y ejemplos}\label{definiciones-y-ejemplos}

Una secuencia es una lista ordenada de números, generalmente escrita
como

\[
a_1, a_2, a_3, \dots
\]

o más generalmente \((a_n)_{n=1}^\infty\). Cada \(a_n\) se llama enésimo
término de la secuencia.

\subsubsection{Definiendo una secuencia}\label{definiendo-una-secuencia}

Una secuencia se puede definir de dos maneras:

\begin{enumerate}
\def\labelenumi{\arabic{enumi}.}
\item
  Fórmula explícita: proporciona una regla directa para el enésimo
  término.

  \begin{itemize}
  \item
    Ejemplo: \(a_n = \frac{1}{n}\) define la secuencia

    \[
    1, \tfrac{1}{2}, \tfrac{1}{3}, \tfrac{1}{4}, \dots
    \]
  \end{itemize}
\item
  Definición recursiva: define términos utilizando términos anteriores.

  \begin{itemize}
  \item
    Ejemplo: secuencia de Fibonacci:

    \[
    a_1 = 1, \quad a_2 = 1, \quad a_{n} = a_{n-1} + a_{n-2} \quad (n \geq 3).
    \]
  \end{itemize}
\end{enumerate}

\subsubsection{Ejemplos de secuencias}\label{ejemplos-de-secuencias}

\begin{enumerate}
\def\labelenumi{\arabic{enumi}.}
\item
  Secuencia aritmética:

  \[
  a_n = a_1 + (n-1)d.
  \]

  Ejemplo: \(a_n = 2n+1\) → secuencia de números impares.
\item
  Secuencia geométrica:

  \[
  a_n = a_1 r^{n-1}.
  \]

  Ejemplo: \(a_n = 2^n\) → potencias de 2.
\item
  Secuencia armónica:

  \[
  a_n = \frac{1}{n}.
  \]
\item
  Secuencia alterna:

  \[
  a_n = (-1)^n.
  \]
\end{enumerate}

\subsubsection{Secuencias en Cálculo}\label{secuencias-en-cuxe1lculo}

Las secuencias son la base de procesos infinitos:

\begin{itemize}
\tightlist
\item
  Límites de secuencias → definen la convergencia.
\item
  Series → sumas infinitas construidas a partir de secuencias.
\item
  Funciones aproximadas por secuencias y series.
\end{itemize}

\subsubsection{Por qué esto es
importante}\label{por-quuxe9-esto-es-importante-23}

\begin{itemize}
\tightlist
\item
  Las secuencias proporcionan los componentes básicos para series y
  aproximaciones infinitas.
\item
  Nos permiten definir rigurosamente ``infinito próximo'' y
  convergencia.
\item
  Muchas funciones importantes (exponenciales, trigonométricas) se
  pueden expresar mediante secuencias y series.
\end{itemize}

\subsubsection{Ejercicios}\label{ejercicios-41}

\begin{enumerate}
\def\labelenumi{\arabic{enumi}.}
\tightlist
\item
  Escribe los primeros cinco términos de la secuencia
  \(a_n = \frac{n}{n+1}\).
\item
  Determina si \(a_n = (-1)^n n\) está acotado.
\item
  Da una definición recursiva para la secuencia \(2,4,8,16,\dots\).
\item
  Encuentra el décimo término de la secuencia aritmética con \(a_1=3\) y
  \(d=5\).
\item
  Escribe una fórmula explícita para la secuencia definida por
  \(a_1=1\), \(a_{n+1}=2a_n\).
\end{enumerate}

\subsection{11.2 Secuencias monótonas y
acotadas}\label{secuencias-monuxf3tonas-y-acotadas}

Para entender si una secuencia converge, necesitamos estudiar su
comportamiento: ¿aumenta, disminuye, se mantiene dentro de límites o
crece sin límite? Dos conceptos importantes son monotonía y limitación.

\subsubsection{Secuencias monótonas}\label{secuencias-monuxf3tonas}

Una secuencia \((a_n)\) se llama monótona si siempre es creciente o
siempre decreciente.

\begin{itemize}
\item
  Monótono creciente:

  \[
  a_{n+1} \geq a_n \quad \text{for all } n.
  \]
\item
  Monótono decreciente:

  \[
  a_{n+1} \leq a_n \quad \text{for all } n.
  \]
\end{itemize}

Ejemplos:1. \(a_n = n\) es monótono creciente. 2. \(a_n = \frac{1}{n}\)
es monótono decreciente.

\subsubsection{Secuencias acotadas}\label{secuencias-acotadas}

Una secuencia está acotada arriba si existe un número \(M\) tal que
\(a_n \leq M\) para todo \(n\). Está acotado por debajo si existe \(m\)
tal que \(a_n \geq m\) para todo \(n\).

Si se cumplen ambas condiciones, la secuencia está acotada.

Ejemplos:

\begin{enumerate}
\def\labelenumi{\arabic{enumi}.}
\tightlist
\item
  \(a_n = \frac{1}{n}\) está acotado entre 0 y 1.
\item
  \(a_n = (-1)^n\) está acotado entre -1 y 1.
\item
  \(a_n = n\) no está acotado.
\end{enumerate}

\subsubsection{Teorema de convergencia
monótona}\label{teorema-de-convergencia-monuxf3tona}

Un resultado fundamental en el análisis:

\begin{itemize}
\tightlist
\item
  Toda secuencia creciente monótona que está acotada arriba converge.
\item
  Toda secuencia monótona decreciente que está acotada por debajo
  converge.
\end{itemize}

Este teorema garantiza la convergencia sin encontrar el límite
explícitamente.

\subsubsection{Ejemplo}\label{ejemplo-1}

\begin{enumerate}
\def\labelenumi{\arabic{enumi}.}
\item
  Secuencia: \(a_n = 1 - \frac{1}{n}\).

  \begin{itemize}
  \tightlist
  \item
    Creciente: desde
    \(a_{n+1} - a_n = \frac{1}{n} - \frac{1}{n+1} > 0\).
  \item
    Acotado arriba por 1.
  \item
    Por tanto, converge.
  \item
    Límite: \(\lim_{n\to\infty} a_n = 1\).
  \end{itemize}
\end{enumerate}

\subsubsection{Por qué esto es
importante}\label{por-quuxe9-esto-es-importante-24}

\begin{itemize}
\tightlist
\item
  La monotonicidad y la limitación son pruebas rápidas de convergencia.
\item
  Son imprescindibles en las pruebas y en la construcción de límites con
  rigor.
\item
  Estas ideas se extienden naturalmente a funciones y series.
\end{itemize}

\subsubsection{Ejercicios}\label{ejercicios-42}

\begin{enumerate}
\def\labelenumi{\arabic{enumi}.}
\tightlist
\item
  Determina si \(a_n = \frac{n}{n+1}\) es monótono y acotado.
\item
  Demuestre que \(a_n = \sqrt{n}\) es monótono creciente pero no
  acotado.
\item
  Demuestra que \(a_n = 2 - \frac{1}{n}\) converge y encuentra su
  límite.
\item
  Da un ejemplo de una secuencia acotada que no sea monótona.
\item
  Aplicar el teorema de convergencia monótona a
  \(a_n = \ln\!\big(1+\frac{1}{n}\big)\).
\end{enumerate}

\subsection{11.3 Límites de secuencias}\label{luxedmites-de-secuencias}

La pregunta central acerca de una secuencia es si sus términos se
acercan a un valor único a medida que \(n\) crece. Esto lleva al
concepto de límite de una secuencia.

\subsubsection{Definición}\label{definiciuxf3n-11}

Una secuencia \((a_n)\) tiene un límite \(L\) si para cada
\(\varepsilon > 0\) existe un número entero \(N\) tal que

\[
|a_n - L| < \varepsilon \quad \text{whenever } n > N.
\]

luego escribimos

\[
\lim_{n\to\infty} a_n = L.
\]

Si no existe tal \(L\), la secuencia diverge.

\subsubsection{Intuición}\label{intuiciuxf3n}

\begin{itemize}
\tightlist
\item
  Los términos de la secuencia se acercan arbitrariamente a \(L\) a
  medida que \(n\) se hace grande.
\item
  Más allá de algún índice \(N\), todos los términos se mantienen dentro
  de una pequeña banda alrededor de \(L\).
\end{itemize}

\subsubsection{Ejemplos}\label{ejemplos-23}

\begin{enumerate}
\def\labelenumi{\arabic{enumi}.}
\item
  \(a_n = \frac{1}{n}\). A medida que \(n\) crece, los términos se
  reducen hacia 0.

  \[
  \lim_{n\to\infty} \frac{1}{n} = 0.
  \]
\item
  \(a_n = (-1)^n\). Los términos alternan entre -1 y 1, por lo que no
  existe un límite único. La secuencia diverge.
\item
  \(a_n = \frac{n}{n+1}\). Como \(n \to \infty\), el numerador y el
  denominador son casi iguales, por lo que

  \[
  \lim_{n\to\infty} \frac{n}{n+1} = 1.
  \]
\end{enumerate}

\subsubsection{\texorpdfstring{Propiedades de los límitesSi
\(\lim a_n = A\) y
\(\lim b_n = B\):}{Propiedades de los límitesSi \textbackslash lim a\_n = A y \textbackslash lim b\_n = B:}}\label{propiedades-de-los-luxedmitessi-lim-a_n-a-y-lim-b_n-b}

\begin{itemize}
\item
  \(\lim (a_n+b_n) = A+B\).
\item
  \(\lim (a_n b_n) = AB\).
\item
  \(\lim (c a_n) = cA\) para \(c\) constante.
\item
  Si \(b_n \neq 0\) y \(B \neq 0\), entonces

  \[
  \lim \frac{a_n}{b_n} = \frac{A}{B}.
  \]
\end{itemize}

\subsubsection{Teorema: principio de
compresión}\label{teorema-principio-de-compresiuxf3n}

Si \(a_n \leq b_n \leq c_n\) para todos los grandes \(n\), y

\[
\lim_{n\to\infty} a_n = \lim_{n\to\infty} c_n = L,
\]

entonces

\[
\lim_{n\to\infty} b_n = L.
\]

Ejemplo:

\[
a_n = -\tfrac{1}{n}, \quad b_n = \tfrac{\sin n}{n}, \quad c_n = \tfrac{1}{n}.
\]

Dado que \(-\tfrac{1}{n} \leq \tfrac{\sin n}{n} \leq \tfrac{1}{n}\) y
ambas secuencias delimitadoras van a 0,

\[
\lim_{n\to\infty} \frac{\sin n}{n} = 0.
\]

\subsubsection{Por qué esto es
importante}\label{por-quuxe9-esto-es-importante-25}

\begin{itemize}
\tightlist
\item
  Los límites hacen rigurosa la idea de secuencias que ``se acercan'' a
  un valor.
\item
  La convergencia de secuencias sustenta las series infinitas y la
  continuidad.
\item
  Estos conceptos son esenciales para definir números reales mediante
  límites.
\end{itemize}

\subsubsection{Ejercicios}\label{ejercicios-43}

\begin{enumerate}
\def\labelenumi{\arabic{enumi}.}
\tightlist
\item
  Calcula \(\lim_{n\to\infty} \frac{2n+1}{3n+4}\).
\item
  Determina si \(a_n = \sqrt{n+1} - \sqrt{n}\) converge.
\item
  ¿Converge \(a_n = \cos n\)? ¿Por qué o por qué no?
\item
  Usa el principio de compresión para mostrar
  \(\lim_{n\to\infty} \frac{\sin n}{n} = 0\).
\item
  Demuestre que si \(\lim a_n = L\), entonces \(\lim |a_n| = |L|\).
\end{enumerate}

\section{Capítulo 12. Serie
infinita}\label{capuxedtulo-12.-serie-infinita}

\subsection{12.1 Serie y Convergencia}\label{serie-y-convergencia}

Una serie es la suma de los términos de una secuencia. En lugar de
simplemente enumerar números, los sumamos y estudiamos si la suma
infinita se acerca a un valor finito.

\subsubsection{Definición}\label{definiciuxf3n-12}

Dada una secuencia \((a_n)\), la serie correspondiente es

\[
\sum_{n=1}^\infty a_n = a_1 + a_2 + a_3 + \dots
\]

Definimos la enésima suma parcial como

\[
S_n = \sum_{k=1}^n a_k.
\]

Si la secuencia \((S_n)\) converge a un límite finito \(S\), entonces la
serie converge y

\[
\sum_{n=1}^\infty a_n = S.
\]

Si \((S_n)\) diverge, entonces la serie diverge.

\subsubsection{Ejemplos}\label{ejemplos-24}

\begin{enumerate}
\def\labelenumi{\arabic{enumi}.}
\tightlist
\item
  Serie geométrica
\end{enumerate}

\[
\sum_{n=0}^\infty ar^n = \frac{a}{1-r}, \quad |r| < 1.
\]

Ejemplo:

\[
1 + \tfrac{1}{2} + \tfrac{1}{4} + \tfrac{1}{8} + \dots = 2.
\]

\begin{enumerate}
\def\labelenumi{\arabic{enumi}.}
\setcounter{enumi}{1}
\tightlist
\item
  Serie armónica
\end{enumerate}

\[
\sum_{n=1}^\infty \frac{1}{n}.
\]

Esta serie diverge, aunque los términos lleguen a 0.

\begin{enumerate}
\def\labelenumi{\arabic{enumi}.}
\setcounter{enumi}{2}
\tightlist
\item
  serie p
\end{enumerate}

\[
\sum_{n=1}^\infty \frac{1}{n^p}.
\]

\begin{itemize}
\tightlist
\item
  Converge si \(p > 1\).
\item
  Divergencia si \(p \leq 1\).
\end{itemize}

\subsubsection{Condición necesaria para la
convergencia}\label{condiciuxf3n-necesaria-para-la-convergencia}

Si \(\sum a_n\) converge, entonces necesariamente

\[
\lim_{n\to\infty} a_n = 0.
\]

Si \(\lim a_n \neq 0\), la serie diverge. Pero lo contrario no es
cierto: \(\lim a_n = 0\) no garantiza la convergencia (por ejemplo,
series armónicas).

\subsubsection{Por qué esto es
importante}\label{por-quuxe9-esto-es-importante-26}

\begin{itemize}
\tightlist
\item
  Las series extienden la suma finita a procesos infinitos.
\item
  Las series convergentes se utilizan para aproximar funciones, calcular
  áreas y modelar procesos físicos.- El estudio de series conduce a
  potentes pruebas de convergencia.
\end{itemize}

\subsubsection{Ejercicios}\label{ejercicios-44}

\begin{enumerate}
\def\labelenumi{\arabic{enumi}.}
\tightlist
\item
  Determina si \(\sum_{n=1}^\infty \frac{2}{3^n}\) converge y encuentra
  su suma.
\item
  Demuestre que \(\sum_{n=1}^\infty \frac{1}{n^2}\) converge.
\item
  ¿Converge \(\sum_{n=1}^\infty \frac{1}{\sqrt{n}}\)?
\item
  Escribe las primeras cuatro sumas parciales de la serie
  \(\sum_{n=1}^\infty \frac{1}{2^n}\).
\item
  Explique por qué \(\lim a_n = 0\) es necesario pero no suficiente para
  la convergencia.
\end{enumerate}

\subsection{12.2 Pruebas de convergencia}\label{pruebas-de-convergencia}

Dado que muchas series no se pueden sumar directamente, los matemáticos
desarrollaron pruebas para decidir si una serie converge o diverge.
Estas pruebas son herramientas para analizar sumas infinitas.

\subsubsection{1. La prueba de divergencia del enésimo
término}\label{la-prueba-de-divergencia-del-enuxe9simo-tuxe9rmino}

si

\[
\lim_{n\to\infty} a_n \neq 0 \quad \text{or does not exist},
\]

entonces

\[
\sum a_n
\]

diverge.

Si \(\lim a_n = 0\), la prueba no es concluyente.

\subsubsection{2. Prueba de comparación}\label{prueba-de-comparaciuxf3n}

Supongamos \(0 \leq a_n \leq b_n\) para todos \(n\).

\begin{itemize}
\tightlist
\item
  Si \(\sum b_n\) converge, entonces \(\sum a_n\) también converge.
\item
  Si \(\sum a_n\) diverge, entonces \(\sum b_n\) también diverge.
\end{itemize}

\subsubsection{3. Prueba de comparación de
límites}\label{prueba-de-comparaciuxf3n-de-luxedmites}

Si \(a_n, b_n > 0\) y

\[
\lim_{n\to\infty} \frac{a_n}{b_n} = c,
\]

donde \(0 < c < \infty\), entonces \(\sum a_n\) y \(\sum b_n\) ambos
convergen o ambos divergen.

\subsubsection{4. Prueba de relación}\label{prueba-de-relaciuxf3n}

Para \(\sum a_n\), calcule

\[
L = \lim_{n\to\infty} \left| \frac{a_{n+1}}{a_n} \right|.
\]

\begin{itemize}
\tightlist
\item
  Si \(L < 1\), la serie converge absolutamente.
\item
  Si \(L > 1\) o \(L = \infty\), la serie diverge.
\item
  Si \(L = 1\), la prueba no es concluyente.
\end{itemize}

\subsubsection{5. Prueba de raíz}\label{prueba-de-rauxedz}

Para \(\sum a_n\), calcule

\[
L = \lim_{n\to\infty} \sqrt[n]{|a_n|}.
\]

\begin{itemize}
\tightlist
\item
  Si \(L < 1\), la serie converge absolutamente.
\item
  Si \(L > 1\), la serie diverge.
\item
  Si \(L = 1\), la prueba no es concluyente.
\end{itemize}

\subsubsection{6. Prueba de series alternas (prueba de
Leibniz)}\label{prueba-de-series-alternas-prueba-de-leibniz}

Para series de la forma

\[
\sum (-1)^n b_n \quad \text{or} \quad \sum (-1)^{n+1} b_n,
\]

si

\begin{enumerate}
\def\labelenumi{\arabic{enumi}.}
\tightlist
\item
  \(b_{n+1} \leq b_n\) (decreciente), y
\item
  \(\lim_{n\to\infty} b_n = 0\),
\end{enumerate}

entonces la serie converge.

\subsubsection{Ejemplos}\label{ejemplos-25}

\begin{enumerate}
\def\labelenumi{\arabic{enumi}.}
\tightlist
\item
  \(\sum \frac{1}{n^2}\): Prueba de Comparación → converge.
\item
  \(\sum \frac{1}{n}\): Serie armónica → diverge.
\item
  \(\sum \frac{(-1)^n}{n}\): Prueba de series alternas → converge.
\item
  \(\sum \frac{n!}{n^n}\): Prueba de razón → converge.
\item
  \(\sum \frac{2^n}{n}\): Prueba de raíz → diverge.
\end{enumerate}

\subsubsection{Por qué esto es
importante}\label{por-quuxe9-esto-es-importante-27}

\begin{itemize}
\tightlist
\item
  Las pruebas de convergencia nos permiten clasificar series sin
  necesidad de sumas explícitas.
\item
  Proporcionan formas sistemáticas de manejar infinitos procesos en
  cálculo.
\item
  Son fundamentales para temas posteriores como series de potencia y
  series de Fourier.
\end{itemize}

\subsubsection{Ejercicios}\label{ejercicios-45}

\begin{enumerate}
\def\labelenumi{\arabic{enumi}.}
\tightlist
\item
  Pruebe la convergencia de \(\sum \frac{1}{n^3}\).
\item
  Utilice la prueba de razón para \(\sum \frac{3^n}{n!}\).3. Aplicar la
  prueba de la raíz a \(\sum \left(\frac{1}{2}\right)^n\).
\item
  Determinar la convergencia de \(\sum \frac{(-1)^n}{\sqrt{n}}\).
\item
  Utilice la prueba de comparación de límites con \(\frac{1}{n^2}\) para
  probar \(\sum \frac{1}{n^2+1}\).
\end{enumerate}

\subsection{12.3 Convergencia absoluta versus
condicional}\label{convergencia-absoluta-versus-condicional}

No todas las series se comportan igual cuando se alternan los signos.
Para manejar esto, distinguimos entre convergencia absoluta y
convergencia condicional.

\subsubsection{Convergencia absoluta}\label{convergencia-absoluta}

Una serie \(\sum a_n\) es absolutamente convergente si

\[
\sum |a_n|
\]

converge.

Teorema: si una serie converge absolutamente, entonces también converge.

Ejemplo:

\[
\sum \frac{(-1)^n}{n^2}.
\]

Aquí converge
\(\sum \left|\frac{(-1)^n}{n^2}\right| = \sum \frac{1}{n^2}\) (serie p,
\(p=2\)). Entonces la serie es absolutamente convergente.

\subsubsection{Convergencia condicional}\label{convergencia-condicional}

Una serie \(\sum a_n\) es condicionalmente convergente si converge, pero
no absolutamente.

Ejemplo:

\[
\sum \frac{(-1)^n}{n}.
\]

\begin{itemize}
\tightlist
\item
  Prueba de series alternas → converge.
\item
  Pero \(\sum \left|\frac{(-1)^n}{n}\right| = \sum \frac{1}{n}\) diverge
  (serie armónica). Entonces la serie es condicionalmente convergente.
\end{itemize}

\subsubsection{Teorema de
reordenamiento}\label{teorema-de-reordenamiento}

Para series condicionalmente convergentes, reorganizar los términos
puede cambiar la suma, incluso hacerla divergir o converger a un valor
diferente.

Este sorprendente resultado muestra la delicada naturaleza de la
convergencia condicional.

\subsubsection{Por qué esto es
importante}\label{por-quuxe9-esto-es-importante-28}

\begin{itemize}
\tightlist
\item
  La convergencia absoluta es más fuerte y garantiza la estabilidad.
\item
  La convergencia condicional resalta la importancia del orden en sumas
  infinitas.
\item
  Muchas series alternas que se encuentran en la práctica son sólo
  condicionalmente convergentes.
\end{itemize}

\subsubsection{Ejercicios}\label{ejercicios-46}

\begin{enumerate}
\def\labelenumi{\arabic{enumi}.}
\tightlist
\item
  Demuestre que \(\sum \frac{(-1)^n}{n^3}\) converge absolutamente.
\item
  Demuestre que \(\sum \frac{(-1)^n}{n}\) es condicionalmente
  convergente.
\item
  Pruebe \(\sum \frac{(-1)^n}{\sqrt{n}}\) para convergencia absoluta y
  condicional.
\item
  Explique por qué la convergencia absoluta implica convergencia, pero
  lo contrario no es cierto.
\item
  Investiga y resume el teorema del reordenamiento de Riemann en tus
  propias palabras.
\end{enumerate}

\section{Capítulo 13. Series de potencia y
expansiones.}\label{capuxedtulo-13.-series-de-potencia-y-expansiones.}

\subsection{13.1 Serie de potencia}\label{serie-de-potencia}

Una serie de potencias es una serie infinita en la que cada término
implica una potencia de la variable. Las series de potencias son
fundamentales en el cálculo porque nos permiten representar funciones
como polinomios infinitos.

\subsubsection{Formulario general}\label{formulario-general}

Una serie de potencias centrada en \(a\) tiene la forma

\[
\sum_{n=0}^\infty c_n (x-a)^n,
\]

donde \(c_n\) son constantes llamadas coeficientes.

\begin{itemize}
\item
  Si \(a=0\), la serie está centrada en el origen:

  \[
  \sum_{n=0}^\infty c_n x^n.
  \]
\end{itemize}

\subsubsection{Ejemplos}\label{ejemplos-26}

\begin{enumerate}
\def\labelenumi{\arabic{enumi}.}
\tightlist
\item
  Serie geométrica
\end{enumerate}

\[
\sum_{n=0}^\infty x^n = \frac{1}{1-x}, \quad |x|<1.
\]

\begin{enumerate}
\def\labelenumi{\arabic{enumi}.}
\setcounter{enumi}{1}
\tightlist
\item
  Función exponencial
\end{enumerate}

\[e^x = \sum_{n=0}^\infty \frac{x^n}{n!}.
\]

\begin{enumerate}
\def\labelenumi{\arabic{enumi}.}
\setcounter{enumi}{2}
\tightlist
\item
  Sine and cosine
\end{enumerate}

\[
\sin x = \sum_{n=0}^\infty (-1)^n \frac{x^{2n+1}}{(2n+1)!}, \quad  
\cos x = \sum_{n=0}^\infty (-1)^n \frac{x^{2n}}{(2n)!}.
\]

\subsubsection{Interval of Convergence}\label{interval-of-convergence}

For each power series, there exists a radius of convergence \(R\) such
that:

\begin{itemize}
\tightlist
\item
  The series converges if \(|x-a| < R\).
\item
  The series diverges if \(|x-a| > R\).
\item
  At \(|x-a| = R\), convergence must be checked separately.
\end{itemize}

\subsubsection{Why This Matters}\label{why-this-matters-3}

\begin{itemize}
\tightlist
\item
  Power series allow us to approximate functions by polynomials.
\item
  They connect calculus with analysis and differential equations.
\item
  Many special functions in mathematics and physics are defined by their
  power series.
\end{itemize}

\subsubsection{Exercises}\label{exercises-6}

\begin{enumerate}
\def\labelenumi{\arabic{enumi}.}
\tightlist
\item
  Write the power series for \(\sum_{n=0}^\infty \frac{(x-2)^n}{n!}\).
\item
  Find the first four terms of the power series for \(e^x\).
\item
  Express \(\frac{1}{1+x}\) as a power series centered at 0.
\item
  Determine whether the series \(\sum_{n=0}^\infty n! x^n\) converges at
  \(x=0.1\).
\item
  Explain why power series are sometimes called ``infinite
  polynomials.''
\end{enumerate}

\subsection{13.2 Radius of Convergence}\label{radius-of-convergence}

Every power series converges for some values of \(x\) and diverges for
others. The boundary between these two behaviors is described by the
radius of convergence.

\subsubsection{Definition}\label{definition}

For a power series

\[
\sum_{n=0}^\infty c_n (xa)^n,
\]

there exists a number \(R \geq 0\) (possibly infinite) such that:

\begin{itemize}
\tightlist
\item
  The series converges absolutely if \(|x-a| < R\).
\item
  The series diverges if \(|x-a| > R\).
\item
  At \(|x-a| = R\), convergence must be checked separately.
\end{itemize}

This number \(R\) is called the radius of convergence.

\subsubsection{Finding the Radius of
Convergence}\label{finding-the-radius-of-convergence}

Two common methods:

\begin{enumerate}
\def\labelenumi{\arabic{enumi}.}
\tightlist
\item
  Ratio Test
\end{enumerate}

\[
R = \lim_{n\to\infty} \left| \frac{c_n}{c_{n+1}} \right|.
\]

\begin{enumerate}
\def\labelenumi{\arabic{enumi}.}
\setcounter{enumi}{1}
\tightlist
\item
  Root Test
\end{enumerate}

\[
R = \frac{1}{\limsup_{n\to\infty} \sqrt[n]{|c_n|}}.
\]

\subsubsection{Examples}\label{examples-4}

\begin{enumerate}
\def\labelenumi{\arabic{enumi}.}
\tightlist
\item
  Series:
\end{enumerate}

\[
\sum_{n=0}^\infty \frac{x^n}{n!}.
\]

Using ratio test:

\[
\lim_{n\to\infty} \frac{1/(n!)}{1/((n+1)!)} = \infty.
\]

So \(R = \infty\) (converges for all real \(x\)).

\begin{enumerate}
\def\labelenumi{\arabic{enumi}.}
\setcounter{enumi}{1}
\tightlist
\item
  Series:
\end{enumerate}

\[
\sum_{n=0}^\infty x^n.
\]

Here \(c_n = 1\).

\[
R = 1.
\]

Converges for \(|x| < 1\).

\begin{enumerate}
\def\labelenumi{\arabic{enumi}.}
\setcounter{enumi}{2}
\tightlist
\item
  Series:
\end{enumerate}

\[
\sum_{n=1}^\infty \frac{x^n}{n}.
\]

Ratio test:

\[
\lim_{n\to\infty} \left|\frac{(x^{n+1}/(n+1))}{(x^n/n)}\right| = |x|.
\]

Entonces \(R = 1\). Converge para \(|x| < 1\), diverge para \(|x| > 1\).
A \(x=\pm 1\), prueba por separado.

\subsubsection{Intervalo de
convergencia}\label{intervalo-de-convergencia}

El conjunto de valores de \(x\) donde converge la serie se llama
intervalo de convergencia.

\begin{itemize}
\tightlist
\item
  Siempre centrado en \(a\).
\item
  Extiende \(R\) unidades en ambas direcciones.
\item
  Los puntos finales \(x=a\pm R\) deben comprobarse individualmente.
\end{itemize}

\subsubsection{Por qué esto es importante- El radio de convergencia nos
dice dónde las series de potencias se comportan como
funciones.}\label{por-quuxe9-esto-es-importante--el-radio-de-convergencia-nos-dice-duxf3nde-las-series-de-potencias-se-comportan-como-funciones.}

\begin{itemize}
\tightlist
\item
  Esencial para utilizar las expansiones de series de Taylor en la
  práctica.
\item
  Determina el dominio de validez de soluciones en serie en física e
  ingeniería.
\end{itemize}

\subsubsection{Ejercicios}\label{ejercicios-47}

\begin{enumerate}
\def\labelenumi{\arabic{enumi}.}
\tightlist
\item
  Encuentra el radio de convergencia de
  \(\sum_{n=0}^\infty \frac{(x-3)^n}{n!}\).
\item
  Calcula el radio de convergencia de
  \(\sum_{n=1}^\infty \frac{x^n}{n^2}\).
\item
  Usa la prueba de razón para encontrar \(R\) para
  \(\sum_{n=0}^\infty n!x^n\).
\item
  Determina el intervalo de convergencia para
  \(\sum_{n=1}^\infty \frac{(x+1)^n}{n}\).
\item
  Explica por qué la serie exponencial converge para todos \(x\),
  mientras que la serie geométrica solo converge para \(|x|<1\).
\end{enumerate}

\subsection{13.3 Serie Taylor y
Maclaurin}\label{serie-taylor-y-maclaurin}

Las series de potencias se vuelven especialmente poderosas cuando se
usan para representar funciones familiares. Esto se hace mediante series
de Taylor y el caso especial centrado en 0 se denomina serie de
Maclaurin.

\subsubsection{Serie Taylor}\label{serie-taylor}

Si una función \(f(x)\) es infinitamente diferenciable en \(x=a\), su
serie de Taylor sobre \(a\) es

\[
f(x) = \sum_{n=0}^\infty \frac{f^{(n)}(a)}{n!}(x-a)^n.
\]

Aquí \(f^{(n)}(a)\) denota la \(n\)-ésima derivada de \(f\) en \(a\).

\subsubsection{Serie Maclaurin}\label{serie-maclaurin}

Una serie de Taylor centrada en \(a=0\):

\[
f(x) = \sum_{n=0}^\infty \frac{f^{(n)}(0)}{n!} x^n.
\]

\subsubsection{Ejemplos}\label{ejemplos-27}

\begin{enumerate}
\def\labelenumi{\arabic{enumi}.}
\tightlist
\item
  Función exponencial
\end{enumerate}

\[
e^x = 1 + x + \frac{x^2}{2!} + \frac{x^3}{3!} + \cdots
\]

\begin{enumerate}
\def\labelenumi{\arabic{enumi}.}
\setcounter{enumi}{1}
\tightlist
\item
  Seno y coseno
\end{enumerate}

\[
\sin x = x - \frac{x^3}{3!} + \frac{x^5}{5!} - \cdots
\]

\[
\cos x = 1 - \frac{x^2}{2!} + \frac{x^4}{4!} - \cdots
\]

\begin{enumerate}
\def\labelenumi{\arabic{enumi}.}
\setcounter{enumi}{2}
\tightlist
\item
  Logaritmo natural (por \(|x|<1\))
\end{enumerate}

\[
\ln(1+x) = x - \frac{x^2}{2} + \frac{x^3}{3} - \frac{x^4}{4} + \cdots
\]

\subsubsection{Aproximación del polinomio de
Taylor}\label{aproximaciuxf3n-del-polinomio-de-taylor}

La suma finita de los primeros \(n\) términos es el polinomio de Taylor
de grado \(n\):

\[
P_n(x) = \sum_{k=0}^n \frac{f^{(k)}(a)}{k!}(x-a)^k.
\]

Este polinomio se aproxima a \(f(x)\) cerca de \(x=a\).

\subsubsection{Resto (término de
error)}\label{resto-tuxe9rmino-de-error}

La diferencia entre la función y su polinomio de Taylor es el resto:

\[
R_n(x) = f(x) - P_n(x).
\]

Una forma (la forma de Lagrange) es

\[
R_n(x) = \frac{f^{(n+1)}(c)}{(n+1)!}(x-a)^{n+1},
\]

por unos \(c\) entre \(a\) y \(x\).

\subsubsection{Por qué esto es
importante}\label{por-quuxe9-esto-es-importante-29}

\begin{itemize}
\tightlist
\item
  Las series de Taylor proporcionan aproximaciones polinómicas a
  funciones complicadas.
\item
  Son fundamentales en análisis numérico, física e ingeniería.
\item
  Las expansiones de series de Maclaurin dan fórmulas simples para
  funciones exponenciales, trigonométricas y logarítmicas.
\end{itemize}

\subsubsection{Ejercicios}\label{ejercicios-48}

\begin{enumerate}
\def\labelenumi{\arabic{enumi}.}
\tightlist
\item
  Encuentra la serie de Maclaurin por
  \(f(x)=\cosh x = \tfrac{e^x+e^{-x}}{2}\).
\item
  Escribe la serie de Taylor para \(f(x)=e^x\) centrada en \(a=2\).
\item
  Calcula el polinomio de Taylor de grado 3 para \(f(x)=\ln(1+x)\) en
  \(a=0\).4. Utilice la serie de Maclaurin para \(\sin x\) para
  aproximar \(\sin(0.1)\).
\item
  Explique por qué las series de Taylor a menudo proporcionan buenas
  aproximaciones locales pero pueden divergir para \(|x|\) grandes.
\end{enumerate}

\subsection{13.4 Aplicaciones de las series de
Taylor}\label{aplicaciones-de-las-series-de-taylor}

Las series de Taylor no son sólo herramientas teóricas: se utilizan para
aproximar funciones, resolver ecuaciones y analizar sistemas físicos.
Sus aplicaciones abarcan matemáticas, ciencias e ingeniería.

\subsubsection{Aproximación de
funciones}\label{aproximaciuxf3n-de-funciones}

Las funciones complicadas se pueden aproximar mediante polinomios cerca
de un punto.

Ejemplo: Aproximar \(e^x\) cerca de \(x=0\) usando el polinomio de
Maclaurin de grado 3:

\[
P_3(x) = 1 + x + \tfrac{x^2}{2} + \tfrac{x^3}{6}.
\]

Para pequeños \(x\), esto da estimaciones precisas de \(e^x\).

\subsubsection{Métodos numéricos}\label{muxe9todos-numuxe9ricos}

Las series de Taylor proporcionan la base para los algoritmos numéricos:

\begin{itemize}
\tightlist
\item
  Aproximar raíces cuadradas, logaritmos y valores trigonométricos.
\item
  Estimación del error a través del término restante.
\item
  Se utiliza en métodos iterativos como el método de Newton (donde la
  linealización local proviene de la serie de Taylor).
\end{itemize}

\subsubsection{Resolver ecuaciones
diferenciales}\label{resolver-ecuaciones-diferenciales}

Muchas ecuaciones diferenciales tienen soluciones expresadas como series
de Taylor (o potencias).

Ejemplo: La solución de \(y'' + y = 0\) con \(y(0)=0, y'(0)=1\) es
\(\sin x\), que surge naturalmente de su serie de Maclaurin.

\subsubsection{Física e Ingeniería}\label{fuxedsica-e-ingenieruxeda}

\begin{itemize}
\item
  Aproximación de ángulo pequeño:

  \[
  \sin x \approx x, \quad \cos x \approx 1 - \tfrac{x^2}{2}, \quad |x| \ll 1.
  \]

  Utilizado en movimiento pendular, óptica y mecánica ondulatoria.
\item
  Relatividad y mecánica cuántica: las expansiones de Taylor simplifican
  expresiones no lineales para uso práctico.
\item
  Aproximación de funciones de energía: En mecánica, las funciones de
  energía potencial se expanden cerca de los puntos de equilibrio.
\end{itemize}

\subsubsection{Probabilidad y
Estadística}\label{probabilidad-y-estaduxedstica}

\begin{itemize}
\tightlist
\item
  Las funciones generadoras de momentos y las funciones características
  utilizan series de potencias.
\item
  Las aproximaciones de distribuciones de probabilidad (p.~ej.,
  aproximación normal a binomial) utilizan expansiones de Taylor.
\end{itemize}

\subsubsection{Por qué esto es
importante}\label{por-quuxe9-esto-es-importante-30}

\begin{itemize}
\tightlist
\item
  Las series de Taylor proporcionan un puente entre fórmulas exactas y
  cálculos prácticos.
\item
  Nos permiten reducir problemas complejos a aproximaciones polinomiales
  manejables.
\item
  Las aplicaciones las convierten en una de las herramientas más
  importantes de las matemáticas aplicadas.
\end{itemize}

\subsubsection{Ejercicios}\label{ejercicios-49}

\begin{enumerate}
\def\labelenumi{\arabic{enumi}.}
\tightlist
\item
  Utilice la serie de Maclaurin para \(e^x\) para aproximar \(e^{0.1}\)
  hasta cuatro decimales.
\item
  Aplique la aproximación de ángulo pequeño para estimar
  \(\sin(5^\circ)\).
\item
  Resuelva la ecuación diferencial \(y'' = -y\) usando un enfoque de
  series de potencias.
\item
  Expande \(\ln(1+x)\) hasta el 4to grado y úsalo para aproximar
  \(\ln(1.1)\).
\item
  Explique por qué las aproximaciones polinómicas son especialmente
  útiles para computadoras y calculadoras.\# Apéndices
\end{enumerate}

\subsection{Apéndice A. Conceptos básicos de
precálculo}\label{apuxe9ndice-a.-conceptos-buxe1sicos-de-precuxe1lculo}

\subsubsection{A.1 Actualización de
álgebra}\label{a.1-actualizaciuxf3n-de-uxe1lgebra}

Antes de sumergirnos en el cálculo, es útil repasar algunas habilidades
de álgebra que aparecerán una y otra vez. Estas son las ``herramientas''
que necesitará para manipular expresiones, resolver ecuaciones y
simplificar resultados.

\paragraph{Exponentes y potencias}\label{exponentes-y-potencias}

\begin{itemize}
\item
  Reglas básicas:

  \[
  a^m \cdot a^n = a^{m+n}, \quad \frac{a^m}{a^n} = a^{m-n}, \quad (a^m)^n = a^{mn}.
  \]
\item
  Exponentes negativos:

  \[
  a^{-n} = \frac{1}{a^n}, \quad a \neq 0.
  \]
\item
  Exponentes fraccionarios:

  \[
  a^{1/n} = \sqrt[n]{a}, \quad a^{m/n} = \sqrt[n]{a^m}.
  \]
\end{itemize}

\paragraph{Factorización}\label{factorizaciuxf3n}

La factorización simplifica expresiones y ayuda a resolver ecuaciones.

\begin{enumerate}
\def\labelenumi{\arabic{enumi}.}
\item
  Factor común:

  \[
  6x^2+9x = 3x(2x+3).
  \]
\item
  Diferencia de cuadrados:

  \[
  a^2-b^2 = (a-b)(a+b).
  \]
\item
  Trinomios cuadráticos:

  \[
  x^2+5x+6 = (x+2)(x+3).
  \]
\end{enumerate}

\paragraph{Polinomios}\label{polinomios}

\begin{itemize}
\tightlist
\item
  Formulario estándar:
  \(P(x) = a_nx^n + a_{n-1}x^{n-1} + \cdots + a_0\).
\item
  Grado: la potencia mayor de \(x\).
\item
  La división larga y la división sintética son útiles para simplificar
  funciones racionales.
\end{itemize}

\paragraph{Expresiones racionales}\label{expresiones-racionales}

Simplifica factorizando numerador y denominador:

\[
\frac{x^2-1}{x^2-2x+1} = \frac{(x-1)(x+1)}{(x-1)^2} = \frac{x+1}{x-1}, \quad x \neq 1.
\]

\paragraph{Logaritmos}\label{logaritmos}

\begin{itemize}
\item
  Definición: \(\log_a b = c\) significa \(a^c = b\).
\item
  Bases comunes: tronco natural (\(\ln x = \log_e x\)) y base 10
  (\(\log x\)).
\item
  Reglas:

  \[
  \log(ab) = \log a + \log b, \quad \log\left(\frac{a}{b}\right) = \log a - \log b, \quad \log(a^n) = n\log a.
  \]
\end{itemize}

\paragraph{Ecuaciones}\label{ecuaciones}

\begin{itemize}
\item
  Lineal: resolver \(ax+b=0\) → \(x=-b/a\).
\item
  Cuadrática: \(ax^2+bx+c=0\) tiene soluciones

  \[
  x=\frac{-b\pm \sqrt{b^2-4ac}}{2a}.
  \]
\item
  Exponencial: \(e^x = k\) → \(x = \ln k\).
\end{itemize}

\subsubsection{A.2 Conceptos básicos de
trigonometría}\label{a.2-conceptos-buxe1sicos-de-trigonometruxeda}

La trigonometría proporciona el lenguaje de los ángulos y los fenómenos
periódicos. Dado que el cálculo a menudo se ocupa de oscilaciones,
movimientos y ondas, es esencial tener una comprensión sólida de las
funciones trigonométricas y sus propiedades.

\paragraph{El círculo unitario}\label{el-cuxedrculo-unitario}

\begin{itemize}
\item
  Definido como el círculo de radio 1 con centro en el origen en el
  plano coordenado.
\item
  Para un ángulo \(\theta\) medido desde el eje positivo \(x\):

  \[
  (\cos \theta, \sin \theta)
  \]

  da las coordenadas del punto en el círculo.
\end{itemize}

Valores especiales:

\begin{longtable}[]{@{}
  >{\raggedright\arraybackslash}p{(\linewidth - 6\tabcolsep) * \real{0.3333}}
  >{\raggedright\arraybackslash}p{(\linewidth - 6\tabcolsep) * \real{0.1667}}
  >{\raggedright\arraybackslash}p{(\linewidth - 6\tabcolsep) * \real{0.1667}}
  >{\raggedright\arraybackslash}p{(\linewidth - 6\tabcolsep) * \real{0.3333}}@{}}
\toprule\noalign{}
\begin{minipage}[b]{\linewidth}\raggedright
\(\theta\)
\end{minipage} & \begin{minipage}[b]{\linewidth}\raggedright
\(\sin \theta\)
\end{minipage} & \begin{minipage}[b]{\linewidth}\raggedright
\(\cos \theta\)
\end{minipage} & \begin{minipage}[b]{\linewidth}\raggedright
\(\tan \theta = \frac{\sin \theta}{\cos \theta}\)
\end{minipage} \\
\midrule\noalign{}
\endhead
\bottomrule\noalign{}
\endlastfoot
\(0\) & 0 & 1 & 0 \\
\(\pi/6\) & 1/2 & \(\sqrt{3}/2\) & \(1/\sqrt{3}\) \\
\(\pi/3\) & \(\sqrt{3}/2\) & 1/2 & \(\sqrt{3}\) \\
\(\pi/2\) & 1 & 0 & indefinido \\
\end{longtable}

\paragraph{Identidades fundamentales}\label{identidades-fundamentales}

\begin{enumerate}
\def\labelenumi{\arabic{enumi}.}
\tightlist
\item
  Identidad pitagórica
\end{enumerate}

\[
\sin^2\theta + \cos^2\theta = 1.
\]

\begin{enumerate}
\def\labelenumi{\arabic{enumi}.}
\setcounter{enumi}{1}
\tightlist
\item
  Identidades de cociente
\end{enumerate}

\[
\tan\theta = \frac{\sin\theta}{\cos\theta}, \quad \cot\theta = \frac{\cos\theta}{\sin\theta}.
\]

\begin{enumerate}
\def\labelenumi{\arabic{enumi}.}
\setcounter{enumi}{2}
\tightlist
\item
  Identidades recíprocas
\end{enumerate}

\[
\sec\theta = \frac{1}{\cos\theta}, \quad \csc\theta = \frac{1}{\sin\theta}.
\]

\paragraph{Fórmulas de suma de
ángulos}\label{fuxf3rmulas-de-suma-de-uxe1ngulos}

\[
\sin(\alpha+\beta) = \sin\alpha\cos\beta + \cos\alpha\sin\beta,
\]

\[
\cos(\alpha+\beta) = \cos\alpha\cos\beta - \sin\alpha\sin\beta.
\]

Casos especiales:

\begin{itemize}
\item
  Doble ángulo:

  \[
  \sin(2\theta) = 2\sin\theta\cos\theta, \quad
  \cos(2\theta) = \cos^2\theta - \sin^2\theta.
  \]
\end{itemize}

\paragraph{Gráficos}\label{gruxe1ficos}

\begin{itemize}
\tightlist
\item
  \(\sin x\): onda que comienza en 0, amplitud 1, período \(2\pi\).
\item
  \(\cos x\): onda que comienza en 1, amplitud 1, período \(2\pi\).
\item
  \(\tan x\): se repite cada \(\pi\), indefinido en múltiplos impares de
  \(\pi/2\).
\end{itemize}

\subsubsection{A.3 Geometría de
coordenadas}\label{a.3-geometruxeda-de-coordenadas}

La geometría de coordenadas vincula el álgebra y la geometría al
describir objetos geométricos (líneas, círculos, curvas) usando
ecuaciones. El cálculo se basa en gran medida en este marco para
graficar funciones, encontrar pendientes y analizar curvas.

\paragraph{El plano cartesiano}\label{el-plano-cartesiano}

\begin{itemize}
\item
  Un punto está representado por las coordenadas \((x,y)\).
\item
  Distancia entre dos puntos \((x_1,y_1)\) y \((x_2,y_2)\):

  \[
  d = \sqrt{(x_2-x_1)^2 + (y_2-y_1)^2}.
  \]
\item
  Punto medio de un segmento de recta:

  \[
  M = \left(\frac{x_1+x_2}{2}, \frac{y_1+y_2}{2}\right).
  \]
\end{itemize}

\paragraph{Líneas}\label{luxedneas}

\begin{enumerate}
\def\labelenumi{\arabic{enumi}.}
\item
  Fórmula de pendiente

  \[
  m = \frac{y_2-y_1}{x_2-x_1}.
  \]
\item
  Ecuación de una recta

  \begin{itemize}
  \item
    Forma punto-pendiente:

    \[
    y-y_1 = m(x-x_1).
    \]
  \item
    Forma pendiente-intersección:

    \[
    y = mx+b.
    \]
  \end{itemize}
\item
  Rectas paralelas y perpendiculares

  \begin{itemize}
  \tightlist
  \item
    Rectas paralelas: misma pendiente.
  \item
    Rectas perpendiculares: las pendientes satisfacen \(m_1m_2 = -1\).
  \end{itemize}
\end{enumerate}

\paragraph{Círculos}\label{cuxedrculos}

Ecuación de una circunferencia con centro \((h,k)\) y radio \(r\):

\[
(x-h)^2+(y-k)^2 = r^2.
\]

Caso especial: círculo unitario centrado en el origen:

\[
x^2+y^2=1.
\]

\paragraph{Secciones cónicas}\label{secciones-cuxf3nicas}

\begin{enumerate}
\def\labelenumi{\arabic{enumi}.}
\item
  Parábola:

  \begin{itemize}
  \item
    Forma estándar (apertura arriba/abajo):

    \[
    y = ax^2+bx+c.
    \]
  \end{itemize}
\item
  Elipse (centrada en el origen):

  \[
  \frac{x^2}{a^2}+\frac{y^2}{b^2}=1.
  \]
\item
  Hipérbola (centrada en el origen):

  \[
  \frac{x^2}{a^2}-\frac{y^2}{b^2}=1.
  \]
\end{enumerate}

\subsection{Apéndice B. Fórmulas y tablas
clave}\label{apuxe9ndice-b.-fuxf3rmulas-y-tablas-clave}

\subsubsection{B.1 Tabla de derivadasLas derivadas miden tasas de cambio
y pendientes de funciones. Tener una tabla de referencia rápida ayuda a
los alumnos a evitar volver a derivar fórmulas cada
vez.}\label{b.1-tabla-de-derivadaslas-derivadas-miden-tasas-de-cambio-y-pendientes-de-funciones.-tener-una-tabla-de-referencia-ruxe1pida-ayuda-a-los-alumnos-a-evitar-volver-a-derivar-fuxf3rmulas-cada-vez.}

\paragraph{Reglas básicas}\label{reglas-buxe1sicas-1}

\begin{enumerate}
\def\labelenumi{\arabic{enumi}.}
\tightlist
\item
  Regla constante
\end{enumerate}

\[
\frac{d}{dx}[c] = 0
\]

\begin{enumerate}
\def\labelenumi{\arabic{enumi}.}
\setcounter{enumi}{1}
\tightlist
\item
  Regla de poder
\end{enumerate}

\[
\frac{d}{dx}[x^n] = nx^{n-1}, \quad (n \in \mathbb{R})
\]

\begin{enumerate}
\def\labelenumi{\arabic{enumi}.}
\setcounter{enumi}{2}
\tightlist
\item
  Regla múltiple constante
\end{enumerate}

\[
\frac{d}{dx}[c f(x)] = c f'(x)
\]

\begin{enumerate}
\def\labelenumi{\arabic{enumi}.}
\setcounter{enumi}{3}
\tightlist
\item
  Regla de suma y diferencia
\end{enumerate}

\[
\frac{d}{dx}[f(x)\pm g(x)] = f'(x)\pm g'(x)
\]

\paragraph{Funciones
trigonométricas}\label{funciones-trigonomuxe9tricas}

\[
\frac{d}{dx}[\sin x] = \cos x
\]

\[
\frac{d}{dx}[\cos x] = -\sin x
\]

\[
\frac{d}{dx}[\tan x] = \sec^2 x, \quad x \neq \tfrac{\pi}{2}+k\pi
\]

\[
\frac{d}{dx}[\cot x] = -\csc^2 x
\]

\[
\frac{d}{dx}[\sec x] = \sec x \tan x
\]

\[
\frac{d}{dx}[\csc x] = -\csc x \cot x
\]

\paragraph{Funciones exponenciales y
logarítmicas}\label{funciones-exponenciales-y-logaruxedtmicas}

\[
\frac{d}{dx}[e^x] = e^x
\]

\[
\frac{d}{dx}[a^x] = a^x \ln a, \quad a>0, a\neq 1
\]

\[
\frac{d}{dx}[\ln x] = \frac{1}{x}, \quad x>0
\]

\[
\frac{d}{dx}[\log_a x] = \frac{1}{x\ln a}, \quad a>0, a\neq 1
\]

\paragraph{Funciones trigonométricas
inversas}\label{funciones-trigonomuxe9tricas-inversas}

\[
\frac{d}{dx}[\arcsin x] = \frac{1}{\sqrt{1-x^2}}, \quad |x|<1
\]

\[
\frac{d}{dx}[\arccos x] = -\frac{1}{\sqrt{1-x^2}}, \quad |x|<1
\]

\[
\frac{d}{dx}[\arctan x] = \frac{1}{1+x^2}, \quad x \in \mathbb{R}
\]

\paragraph{Reglas de producto, cociente y
cadena}\label{reglas-de-producto-cociente-y-cadena}

\begin{enumerate}
\def\labelenumi{\arabic{enumi}.}
\tightlist
\item
  Regla del producto
\end{enumerate}

\[
\frac{d}{dx}[f(x)g(x)] = f'(x)g(x)+f(x)g'(x)
\]

\begin{enumerate}
\def\labelenumi{\arabic{enumi}.}
\setcounter{enumi}{1}
\tightlist
\item
  Regla del cociente
\end{enumerate}

\[
\frac{d}{dx}\left[\frac{f(x)}{g(x)}\right] = \frac{f'(x)g(x)-f(x)g'(x)}{[g(x)]^2}, \quad g(x)\neq 0
\]

\begin{enumerate}
\def\labelenumi{\arabic{enumi}.}
\setcounter{enumi}{2}
\tightlist
\item
  Regla de la cadena
\end{enumerate}

\[
\frac{d}{dx}[f(g(x))] = f'(g(x))\cdot g'(x)
\]

\subsubsection{B.3 Ampliaciones de series
comunes}\label{b.3-ampliaciones-de-series-comunes}

Las series de potencias nos permiten expresar funciones como polinomios
infinitos. Estas expansiones son esenciales para aproximaciones,
resolver ecuaciones diferenciales y desarrollar la intuición sobre
funciones en cálculo.

\paragraph{Serie geométrica}\label{serie-geomuxe9trica}

\[
\frac{1}{1-x} = \sum_{n=0}^\infty x^n, \quad |x| < 1
\]

\paragraph{Función exponencial}\label{funciuxf3n-exponencial}

\[
e^x = \sum_{n=0}^\infty \frac{x^n}{n!}
= 1 + x + \frac{x^2}{2!} + \frac{x^3}{3!} + \cdots
\]

Válido para todos \(x\).

\paragraph{Funciones
trigonométricas}\label{funciones-trigonomuxe9tricas-1}

\[
\sin x = \sum_{n=0}^\infty (-1)^n \frac{x^{2n+1}}{(2n+1)!}
= x - \frac{x^3}{3!} + \frac{x^5}{5!} - \cdots
\]

\[
\cos x = \sum_{n=0}^\infty (-1)^n \frac{x^{2n}}{(2n)!}
= 1 - \frac{x^2}{2!} + \frac{x^4}{4!} - \cdots
\]

\[
\tan^{-1} x = \sum_{n=0}^\infty (-1)^n \frac{x^{2n+1}}{2n+1}, \quad |x|\leq 1
\]

\paragraph{Logaritmo}\label{logaritmo}

\[
\ln(1+x) = \sum_{n=1}^\infty (-1)^{n+1} \frac{x^n}{n}, \quad -1 < x \leq 1
\]

\paragraph{Expansión binomial
(generalizada)}\label{expansiuxf3n-binomial-generalizada}

\[
(1+x)^r = \sum_{n=0}^\infty \binom{r}{n} x^n, \quad |x|<1
\]

donde

\[\binom{r}{n} = \frac{r(r-1)(r-2)\cdots(r-n+1)}{n!}.
\]

\subsection{Appendix C. Proof
Sketches}\label{appendix-c.-proof-sketches}

\subsubsection{\texorpdfstring{C.1 Limit Laws and the
\(\varepsilon\)--\(\delta\)
Definition}{C.1 Limit Laws and the \textbackslash varepsilon--\textbackslash delta Definition}}\label{c.1-limit-laws-and-the-varepsilondelta-definition}

Calculus rests on the precise meaning of a limit. While intuition
(``values get closer and closer'') is helpful, a formal definition
ensures rigor and avoids paradoxes.

\paragraph{Intuitive Idea}\label{intuitive-idea}

We write

\[
\lim_{x\to a} f(x) = L
\]

to mean that as \(x\) gets arbitrarily close to \(a\), the values of
\(f(x)\) get arbitrarily close to \(L\).

\paragraph{\texorpdfstring{Formal (\(\varepsilon\)--\(\delta\))
Definition}{Formal (\textbackslash varepsilon--\textbackslash delta) Definition}}\label{formal-varepsilondelta-definition}

We say that

\[
\lim_{x\to a} f(x) = L
\]

if for every \(\varepsilon > 0\), there exists a \(\delta > 0\) such
that whenever

\[
0 < |xa| <\delta,
\]

we have

\[
|f(x) - L| < \varepsilon.
\]

\begin{itemize}
\tightlist
\item
  \(\varepsilon\): how close we want \(f(x)\) to be to \(L\).
\item
  \(\delta\): how close \(x\) must be to \(a\) to achieve that.
\end{itemize}

\paragraph{Example}\label{example}

Show that

\[
\lim_{x\to 2} (3x+1) = 7.
\]

\begin{itemize}
\tightlist
\item
  Let \(\varepsilon > 0\).
\item
  We want \(|(3x+1)-7| < \varepsilon\).
\item
  Simplify: \(|3x-6| = 3|x-2| < \varepsilon\).
\item
  This holds if we choose \(\delta = \varepsilon/3\).
\end{itemize}

Thus, by the definition, the limit is 7.

\paragraph{Limit Laws}\label{limit-laws}

If \(\lim_{x \to a} f(x) = L\) and \(\lim_{x \to a} g(x) = M\), then:

\begin{enumerate}
\def\labelenumi{\arabic{enumi}.}
\tightlist
\item
  Sum/Difference
\end{enumerate}

\[
\lim_{x \to a} [f(x) \pm g(x)] = L \pm M
\]

\begin{enumerate}
\def\labelenumi{\arabic{enumi}.}
\setcounter{enumi}{1}
\tightlist
\item
  Constant Multiple
\end{enumerate}

\[
\lim_{x \to a} [c f(x)] = cL
\]

\begin{enumerate}
\def\labelenumi{\arabic{enumi}.}
\setcounter{enumi}{2}
\tightlist
\item
  Product
\end{enumerate}

\[
\lim_{x\to a} [f(x)g(x)] = LM
\]

\begin{enumerate}
\def\labelenumi{\arabic{enumi}.}
\setcounter{enumi}{3}
\tightlist
\item
  Quotient (if \(M \neq 0\))
\end{enumerate}

\[
\lim_{x \to a} \frac{f(x)}{g(x)} = \frac{L}{M}
\]

\begin{enumerate}
\def\labelenumi{\arabic{enumi}.}
\setcounter{enumi}{4}
\tightlist
\item
  Powers and Roots
\end{enumerate}

\[
\lim_{x \to a} [f(x)]^n = L^n, \quad \lim_{x \to a} \sqrt[n]{f(x)} = \sqrt[n]{L} \ (\text{si está definido}).
\]

\subsubsection{C.2 Proof Sketch: The Fundamental Theorem of
Calculus}\label{c.2-proof-sketch-the-fundamental-theorem-of-calculus}

The Fundamental Theorem of Calculus (FTC) links the two central
operations of calculus: differentiation and integration. It shows that
they are, in fact, inverse processes.

\paragraph{Statement of the Theorem}\label{statement-of-the-theorem}

Part I (Differentiation of an Integral): If \(f\) is continuous on
\([a,b]\) and we define

\[
F(x) = \int_a^x f(t)\,dt,
\]

then \(F\) is differentiable on \((a,b)\) and

\[
F'(x) = f(x).
\]

Part II (Evaluation of a Definite Integral): If \(F\) is any
antiderivative of \(f\) on \([a,b]\), then

\[
\int_a^b f(x)\,dx = F(b)-F(a).
\]

\paragraph{Proof Sketch of Part I}\label{proof-sketch-of-part-i}

\begin{enumerate}
\def\labelenumi{\arabic{enumi}.}
\item
  Start with the definition of the derivative:

  \[
  F'(x) = \lim_{h\to 0} \frac{F(x+h)-F(x)}{h}.
  \]
\item
  Substituting \(F(x) = \int_a^x f(t)\,dt\):

  \[
  F(x+h)-F(x) = \int_a^{x+h} f(t)\,dt - \int_a^x f(t)\,dt.
  \]
\item
  By the additivity of integrals:

  \[
  F(x+h)-F(x) = \int_x^{x+h} f(t)\,dt.
  \]
\item
  Therefore:

  \[
  \frac{F(x+h)-F(x)}{h} = \frac{1}{h}\int_x^{x+h} f(t)\,dt.
  \]5. Según el teorema del valor medio para integrales, existe
  \(c \in [x,x+h]\) tal que

  \[
  \frac{1}{h}\int_x^{x+h} f(t)\,dt = f(c).
  \]
\item
  Como \(h \to 0\), \(c \to x\), y como \(f\) es continua:

  \[
  \lim_{h\to 0} f(c) = f(x).
  \]
\end{enumerate}

Por tanto, \(F'(x) = f(x)\).

\paragraph{Bosquejo de prueba de la Parte
II}\label{bosquejo-de-prueba-de-la-parte-ii}

\begin{enumerate}
\def\labelenumi{\arabic{enumi}.}
\item
  Sea \(F\) una antiderivada de \(f\), entonces \(F'(x) = f(x)\).
\item
  Por la Parte I, la función

  \[
  G(x) = \int_a^x f(t)\,dt
  \]

  también es una antiderivada de \(f\).
\item
  Dado que \(F\) y \(G\) difieren sólo por una constante,

  \[
  F(x) = G(x) + C.
  \]
\item
  Evaluación en los puntos finales:

  \[
  \int_a^b f(x)\,dx = G(b)-G(a) = (F(b)+C)-(F(a)+C) = F(b)-F(a).
  \]
\end{enumerate}

\subsubsection{C.3 Bosquejo de prueba: Convergencia de la serie
geométrica}\label{c.3-bosquejo-de-prueba-convergencia-de-la-serie-geomuxe9trica}

La serie geométrica es una de las series infinitas más simples e
importantes. Sirve como modelo para comprender la convergencia y es la
base de muchos resultados posteriores en cálculo.

\paragraph{La Serie}\label{la-serie}

\[
\sum_{n=0}^\infty ar^n = a + ar + ar^2 + ar^3 + \cdots
\]

donde \(a\) es el primer término y \(r\) es la razón común.

\paragraph{Fórmula de suma parcial}\label{fuxf3rmula-de-suma-parcial}

La suma parcial \(n\)-ésima es

\[
S_n = a + ar + ar^2 + \cdots + ar^n.
\]

Multiplica ambos lados por \(r\):

\[
rS_n = ar + ar^2 + \cdots + ar^{n+1}.
\]

Resta las dos ecuaciones:

\[
S_n - rS_n = a - ar^{n+1}.
\]

\[
S_n(1-r) = a(1-r^{n+1}).
\]

entonces

\[
S_n = \frac{a(1-r^{n+1})}{1-r}, \quad r \neq 1.
\]

\paragraph{Convergencia}\label{convergencia}

Tome el límite como \(n \to \infty\):

\begin{itemize}
\item
  Si \(|r| < 1\), entonces \(r^{n+1} \to 0\).

  \[
  \lim_{n\to\infty} S_n = \frac{a}{1-r}.
  \]
\item
  Si \(|r| \geq 1\), entonces \(r^{n+1}\) no llega a 0. La serie
  diverge.
\end{itemize}

\paragraph{Resultado}\label{resultado}

\[
\sum_{n=0}^\infty ar^n =
\begin{cases}
\dfrac{a}{1-r}, & |r|<1, \\[6pt]
\text{diverges}, & |r|\geq 1.
\end{cases}
\]

\subsection{Apéndice D. Aplicaciones y
conexiones}\label{apuxe9ndice-d.-aplicaciones-y-conexiones}

\subsubsection{D.1 Conexiones físicas: velocidad, aceleración y
trabajo}\label{d.1-conexiones-fuxedsicas-velocidad-aceleraciuxf3n-y-trabajo}

El cálculo se desarrolló originalmente para resolver problemas de
física, especialmente de movimiento y cambio. Estas son algunas de las
conexiones más importantes.

\paragraph{Posición, velocidad y
aceleración}\label{posiciuxf3n-velocidad-y-aceleraciuxf3n-1}

\begin{itemize}
\item
  Función de posición: \(s(t)\) da la ubicación de un objeto en el
  tiempo \(t\).
\item
  Velocidad: la derivada de la posición.

  \[
  v(t) = s'(t) = \frac{ds}{dt}
  \]
\item
  Aceleración: la derivada de la velocidad (o segunda derivada de la
  posición).

  \[
  a(t) = v'(t) = s''(t) = \frac{d^2s}{dt^2}
  \]
\end{itemize}

Ejemplo: Si \(s(t) = 4t^2\) metros, entonces:

\[
v(t) = 8t, \quad a(t) = 8.
\]

Entonces el objeto se mueve más rápido linealmente con el tiempo, bajo
aceleración constante.

\paragraph{Trabajo y Fuerza}\label{trabajo-y-fuerza}

En física, el trabajo es el producto de la fuerza y la distancia. Si la
fuerza varía con la posición, el cálculo da:

\[W = \int_a^b F(x)\,dx
\]

where \(F(x)\) is the force at position \(x\), and the object moves from
\(x=a\) to \(x=b\).

Example: A spring with Hooke's law force \(F(x) = kx\) requires work

\[
W = \int_0^d kx\, dx = \frac{1}{2}kd^2
\]

to stretch the spring a distance \(d\).

\paragraph{Energy and Areas Under
Curves}\label{energy-and-areas-under-curves}

\begin{itemize}
\tightlist
\item
  Kinetic energy: \(E_k = \tfrac{1}{2}mv^2\).
\item
  Potential energy often involves integrals (e.g., gravitational
  potential energy from force of gravity).
\item
  In general, integrating a force function gives energy stored or work
  done.
\end{itemize}

\paragraph{Quick Practice}\label{quick-practice}

\begin{enumerate}
\def\labelenumi{\arabic{enumi}.}
\tightlist
\item
  If \(s(t) = t^3 - 3t\), find \(v(t)\) and \(a(t)\).
\item
  Compute the work done by a constant force of 10 N moving an object 5
  m.
\item
  A spring has constant \(k=200\). How much work is needed to stretch it
  0.1 m?
\item
  Show that acceleration is the second derivative of position.
\item
  Explain how the integral \(\int v(t)\, dt\) relates to displacement.
\end{enumerate}

\subsubsection{D.2 Probability and Statistics
Connections}\label{d.2-probability-and-statistics-connections}

Calculus is deeply connected with probability and statistics, especially
when dealing with continuous random variables. Integrals become
essential for defining probabilities, averages, and expectations.

\paragraph{Probability Density Functions
(PDFs)}\label{probability-density-functions-pdfs}

For a continuous random variable \(X\), probabilities are described by a
probability density function \(f(x)\):

\begin{enumerate}
\def\labelenumi{\arabic{enumi}.}
\item
  \(f(x) \geq 0\) for all \(x\).
\item
  Total probability equals 1:

  \[
  \int_{-\infty}^{\infty} f(x)\, dx = 1.
  \]
\end{enumerate}

The probability that \(X\) lies in an interval \([a,b]\) is

\[
P(a \leq X \leq b) = \int_a^b f(x)\, dx.
\]

\paragraph{Expected Value (Mean)}\label{expected-value-mean}

The expected value (average outcome) is

\[
E[X] = \int_{-\infty}^{\infty} x f(x)\, dx.
\]

This is the calculus version of a weighted average.

\paragraph{Variance}\label{variance}

Variance measures spread:

\[
\text{Var}(X) = E[(X-\mu)^2] = \int_{-\infty}^{\infty} (x-\mu)^2 f(x)\, dx,
\]

where \(\mu = E[X]\).

\paragraph{Common Distributions}\label{common-distributions}

\begin{enumerate}
\def\labelenumi{\arabic{enumi}.}
\item
  Uniform distribution on \([a,b]\):

  \[
  f(x) = \frac{1}{b-a}, \quad a \leq x \leq b.
  \]

  Mean: \(\frac{a+b}{2}\).
\item
  Exponential distribution with parameter \(\lambda > 0\):

  \[
  f(x) = \lambda e^{-\lambda x}, \quad x \geq 0.
  \]

  Mean: \(1/\lambda\).
\item
  Normal (Gaussian) distribution:

  \[
  f(x) = \frac{1}{\sqrt{2\pi\sigma^2}} e^{-(x-\mu)^2/(2\sigma^2)}.
  \]

  Las integrales de esta distribución se conectan a la función de error.
\end{enumerate}

\paragraph{Por qué esto es
importante}\label{por-quuxe9-esto-es-importante-31}

\begin{itemize}
\tightlist
\item
  Las integrales convierten las probabilidades en áreas bajo curvas.
\item
  El cálculo de expectativa y varianza vincula los promedios y la
  variabilidad.
\item
  La mayoría de los modelos de datos del mundo real (finanzas, física,
  biología, inteligencia artificial) utilizan estas distribuciones de
  probabilidad continua.
\end{itemize}

\paragraph{\texorpdfstring{Práctica rápida1. Para
\(f(x) = \tfrac{1}{2}\) sobre \([0,2]\), calcule
\(P(0.5 \leq X \leq 1.5)\).}{Práctica rápida1. Para f(x) = \textbackslash tfrac\{1\}\{2\} sobre {[}0,2{]}, calcule P(0.5 \textbackslash leq X \textbackslash leq 1.5).}}\label{pruxe1ctica-ruxe1pida1.-para-fx-tfrac12-sobre-02-calcule-p0.5-leq-x-leq-1.5.}

\begin{enumerate}
\def\labelenumi{\arabic{enumi}.}
\setcounter{enumi}{1}
\tightlist
\item
  Para una distribución exponencial con \(\lambda = 2\), calcule
  \(E[X]\).
\item
  Demuestre que el área total bajo la curva normal estándar es igual a
  1.
\item
  Calcula la media de una distribución uniforme en \([3,7]\).
\item
  Explique por qué las probabilidades se calculan con integrales, no
  sumas, para variables continuas.
\end{enumerate}

\subsubsection{D.3 Conexiones de la informática: aproximaciones de
Taylor en
algoritmos}\label{d.3-conexiones-de-la-informuxe1tica-aproximaciones-de-taylor-en-algoritmos}

El cálculo no es sólo para la física: también sustenta muchas
herramientas y técnicas en informática. Uno de los puentes más claros es
a través de las series de Taylor, que proporcionan formas eficientes de
aproximar funciones en algoritmos y computación numérica.

\paragraph{Aproximación de funciones para
informática}\label{aproximaciuxf3n-de-funciones-para-informuxe1tica}

Las computadoras no pueden almacenar ni calcular directamente la mayoría
de las funciones con exactitud (como \(e^x\), \(\sin x\) o \(\ln x\)).
En cambio, utilizan aproximaciones polinómicas derivadas de expansiones
de Taylor.

Ejemplo: Para aproximar \(e^x\), trunca la serie de Maclaurin:

\[
e^x \approx 1 + x + \frac{x^2}{2!} + \frac{x^3}{3!}.
\]

Para \(x\) pequeños, este polinomio da resultados precisos con sólo unos
pocos términos.

\paragraph{Eficiencia en algoritmos}\label{eficiencia-en-algoritmos}

\begin{itemize}
\tightlist
\item
  Funciones trigonométricas: los algoritmos para calculadoras y CPU
  suelen utilizar expansiones en serie (o variaciones como los
  polinomios de Chebyshev).
\item
  Exponencial/logaritmo: las expansiones de Taylor son la base de las
  aproximaciones rápidas en bibliotecas numéricas.
\item
  Hallazgo de raíces: el método de Newton se basa en la aproximación
  lineal, una aplicación directa de la serie de Taylor (primera
  derivada).
\end{itemize}

\paragraph{Análisis numérico}\label{anuxe1lisis-numuxe9rico}

Las expansiones de Taylor son fundamentales en el análisis de errores:

\begin{itemize}
\item
  Aproximar el término de error utilizando la fórmula del resto:

  \[
  R_n(x) = \frac{f^{(n+1)}(c)}{(n+1)!}(x-a)^{n+1}.
  \]
\item
  Esto nos dice cuántos términos se necesitan para una precisión
  determinada.
\end{itemize}

\paragraph{Conexiones de aprendizaje
automático}\label{conexiones-de-aprendizaje-automuxe1tico}

\begin{itemize}
\tightlist
\item
  La optimización basada en gradientes (como el descenso de gradientes)
  utiliza derivados para actualizar los parámetros de manera eficiente.
\item
  Las funciones de activación (como \(\tanh x\) o
  \hspace{0pt}\hspace{0pt}\(\sigma(x)=1/(1+e^{-x})\)) a menudo se
  aproximan mediante polinomios o funciones por partes para la
  velocidad.
\item
  Las aproximaciones en serie pueden acelerar el entrenamiento y la
  inferencia en entornos restringidos.
\end{itemize}

\paragraph{Por qué esto es
importante}\label{por-quuxe9-esto-es-importante-32}

\begin{itemize}
\tightlist
\item
  Las aproximaciones de Taylor unen las matemáticas continuas con la
  computación discreta.
\item
  Muestran cómo se utilizan conceptos de cálculo en algoritmos, métodos
  numéricos y aprendizaje automático.
\item
  Comprender las aproximaciones ayuda a evitar problemas al depender de
  computadoras para los cálculos.
\end{itemize}

\paragraph{Práctica rápida}\label{pruxe1ctica-ruxe1pida}

\begin{enumerate}
\def\labelenumi{\arabic{enumi}.}
\tightlist
\item
  Aproxima \(\sin(0.1)\) utilizando los tres primeros términos de su
  serie de Maclaurin.2. Utilice el término restante para estimar el
  error al aproximar \(e^1\) con un polinomio de grado 3.
\item
  Explique cómo el método de Newton utiliza el teorema de Taylor.
\item
  ¿Por qué las computadoras podrían preferir aproximaciones polinómicas
  a fórmulas exactas de funciones?
\item
  En el aprendizaje automático, ¿por qué la derivada (gradiente) es tan
  crítica para la optimización?
\end{enumerate}




\end{document}
