% Options for packages loaded elsewhere
\PassOptionsToPackage{unicode}{hyperref}
\PassOptionsToPackage{hyphens}{url}
\PassOptionsToPackage{dvipsnames,svgnames,x11names}{xcolor}
%
\documentclass[
  letterpaper,
  DIV=11,
  numbers=noendperiod]{scrartcl}

\usepackage{amsmath,amssymb}
\usepackage{iftex}
\ifPDFTeX
  \usepackage[T1]{fontenc}
  \usepackage[utf8]{inputenc}
  \usepackage{textcomp} % provide euro and other symbols
\else % if luatex or xetex
  \usepackage{unicode-math}
  \defaultfontfeatures{Scale=MatchLowercase}
  \defaultfontfeatures[\rmfamily]{Ligatures=TeX,Scale=1}
\fi
\usepackage{lmodern}
\ifPDFTeX\else  
    % xetex/luatex font selection
\fi
% Use upquote if available, for straight quotes in verbatim environments
\IfFileExists{upquote.sty}{\usepackage{upquote}}{}
\IfFileExists{microtype.sty}{% use microtype if available
  \usepackage[]{microtype}
  \UseMicrotypeSet[protrusion]{basicmath} % disable protrusion for tt fonts
}{}
\makeatletter
\@ifundefined{KOMAClassName}{% if non-KOMA class
  \IfFileExists{parskip.sty}{%
    \usepackage{parskip}
  }{% else
    \setlength{\parindent}{0pt}
    \setlength{\parskip}{6pt plus 2pt minus 1pt}}
}{% if KOMA class
  \KOMAoptions{parskip=half}}
\makeatother
\usepackage{xcolor}
\setlength{\emergencystretch}{3em} % prevent overfull lines
\setcounter{secnumdepth}{-\maxdimen} % remove section numbering
% Make \paragraph and \subparagraph free-standing
\makeatletter
\ifx\paragraph\undefined\else
  \let\oldparagraph\paragraph
  \renewcommand{\paragraph}{
    \@ifstar
      \xxxParagraphStar
      \xxxParagraphNoStar
  }
  \newcommand{\xxxParagraphStar}[1]{\oldparagraph*{#1}\mbox{}}
  \newcommand{\xxxParagraphNoStar}[1]{\oldparagraph{#1}\mbox{}}
\fi
\ifx\subparagraph\undefined\else
  \let\oldsubparagraph\subparagraph
  \renewcommand{\subparagraph}{
    \@ifstar
      \xxxSubParagraphStar
      \xxxSubParagraphNoStar
  }
  \newcommand{\xxxSubParagraphStar}[1]{\oldsubparagraph*{#1}\mbox{}}
  \newcommand{\xxxSubParagraphNoStar}[1]{\oldsubparagraph{#1}\mbox{}}
\fi
\makeatother


\providecommand{\tightlist}{%
  \setlength{\itemsep}{0pt}\setlength{\parskip}{0pt}}\usepackage{longtable,booktabs,array}
\usepackage{calc} % for calculating minipage widths
% Correct order of tables after \paragraph or \subparagraph
\usepackage{etoolbox}
\makeatletter
\patchcmd\longtable{\par}{\if@noskipsec\mbox{}\fi\par}{}{}
\makeatother
% Allow footnotes in longtable head/foot
\IfFileExists{footnotehyper.sty}{\usepackage{footnotehyper}}{\usepackage{footnote}}
\makesavenoteenv{longtable}
\usepackage{graphicx}
\makeatletter
\newsavebox\pandoc@box
\newcommand*\pandocbounded[1]{% scales image to fit in text height/width
  \sbox\pandoc@box{#1}%
  \Gscale@div\@tempa{\textheight}{\dimexpr\ht\pandoc@box+\dp\pandoc@box\relax}%
  \Gscale@div\@tempb{\linewidth}{\wd\pandoc@box}%
  \ifdim\@tempb\p@<\@tempa\p@\let\@tempa\@tempb\fi% select the smaller of both
  \ifdim\@tempa\p@<\p@\scalebox{\@tempa}{\usebox\pandoc@box}%
  \else\usebox{\pandoc@box}%
  \fi%
}
% Set default figure placement to htbp
\def\fps@figure{htbp}
\makeatother

\KOMAoption{captions}{tableheading}
\makeatletter
\@ifpackageloaded{caption}{}{\usepackage{caption}}
\AtBeginDocument{%
\ifdefined\contentsname
  \renewcommand*\contentsname{Table des matières}
\else
  \newcommand\contentsname{Table des matières}
\fi
\ifdefined\listfigurename
  \renewcommand*\listfigurename{Liste des Figures}
\else
  \newcommand\listfigurename{Liste des Figures}
\fi
\ifdefined\listtablename
  \renewcommand*\listtablename{Liste des Tables}
\else
  \newcommand\listtablename{Liste des Tables}
\fi
\ifdefined\figurename
  \renewcommand*\figurename{Figure}
\else
  \newcommand\figurename{Figure}
\fi
\ifdefined\tablename
  \renewcommand*\tablename{Table}
\else
  \newcommand\tablename{Table}
\fi
}
\@ifpackageloaded{float}{}{\usepackage{float}}
\floatstyle{ruled}
\@ifundefined{c@chapter}{\newfloat{codelisting}{h}{lop}}{\newfloat{codelisting}{h}{lop}[chapter]}
\floatname{codelisting}{Listing}
\newcommand*\listoflistings{\listof{codelisting}{Liste des Listings}}
\makeatother
\makeatletter
\makeatother
\makeatletter
\@ifpackageloaded{caption}{}{\usepackage{caption}}
\@ifpackageloaded{subcaption}{}{\usepackage{subcaption}}
\makeatother

\ifLuaTeX
\usepackage[bidi=basic]{babel}
\else
\usepackage[bidi=default]{babel}
\fi
\babelprovide[main,import]{french}
% get rid of language-specific shorthands (see #6817):
\let\LanguageShortHands\languageshorthands
\def\languageshorthands#1{}
\usepackage{bookmark}

\IfFileExists{xurl.sty}{\usepackage{xurl}}{} % add URL line breaks if available
\urlstyle{same} % disable monospaced font for URLs
\hypersetup{
  pdftitle={Le petit livre du calcul},
  pdflang={fr},
  colorlinks=true,
  linkcolor={blue},
  filecolor={Maroon},
  citecolor={Blue},
  urlcolor={Blue},
  pdfcreator={LaTeX via pandoc}}


\title{Le petit livre du calcul}
\author{}
\date{}

\begin{document}
\maketitle


\section{Le petit livre de calcul}\label{le-petit-livre-de-calcul}

Une introduction concise et adaptée aux débutants aux idées
fondamentales du calcul.

\subsection{Formats}\label{formats}

\begin{itemize}
\tightlist
\item
  \href{../artifacts/fr/book.pdf}{Download PDF} -- version prête à
  imprimer
\item
  \href{../artifacts/fr/book.epub}{Download EPUB} -- compatible avec les
  lecteurs électroniques
\item
  \href{../artifacts/fr/book.tex}{View LaTeX} -- Source de latex
\end{itemize}

\section{Partie 1. Limites et
dérivés}\label{partie-1.-limites-et-duxe9rivuxe9s}

\section{Chapitre 1. Fonctions et
limites}\label{chapitre-1.-fonctions-et-limites}

\subsection{1.1 Fonctions}\label{fonctions}

Une fonction est l'un des objets les plus fondamentaux en mathématiques.
À la base, une fonction est une règle qui prend une entrée et produit
exactement une sortie. Les fonctions nous permettent de décrire les
relations, de modéliser des phénomènes du monde réel et de construire
toute la machinerie du calcul.

\subsubsection{Définition}\label{duxe9finition}

Formellement, une fonction \(f\) d'un ensemble \(X\) (appelé le domaine)
vers un ensemble \(Y\) (appelé le codomaine) s'écrit

\[
f : X \to Y.
\]

Pour chaque élément \(x \in X\), il existe un élément unique
\(f(x) \in Y\). La valeur \(f(x)\) est appelée l'image de \(x\) sous
\(f\).

Si \(y = f(x)\), alors \(y\) est la sortie correspondant à l'entrée
\(x\). L'ensemble de toutes les sorties qui apparaissent réellement est
appelé plage (un sous-ensemble du codomaine).

\subsubsection{Exemples}\label{exemples}

\begin{enumerate}
\def\labelenumi{\arabic{enumi}.}
\item
  La fonction \(f(x) = x^2\) mappe chaque nombre réel \(x\) à son carré.

  \begin{itemize}
  \tightlist
  \item
    Domaine~: tous les nombres réels \(\mathbb{R}\).
  \item
    Codomaine~: tous les nombres réels \(\mathbb{R}\).
  \item
    Plage~: tous les nombres réels non négatifs \([0, \infty)\).
  \end{itemize}
\item
  La fonction \(g(x) = \dfrac{1}{x}\) attribue à chaque nombre réel non
  nul son inverse.

  \begin{itemize}
  \tightlist
  \item
    Domaine : \(\mathbb{R} \setminus \{0\}\).
  \item
    Plage~: \(\mathbb{R} \setminus \{0\}\).
  \end{itemize}
\item
  Un exemple concret~: Soit \(T(t)\) la température extérieure (en °C) à
  l'heure \(t\) (en heures). Il s'agit d'une fonction allant de «
  l'heure de la journée » à la « température ».
\end{enumerate}

\subsubsection{Façons de représenter les
fonctions}\label{fauxe7ons-de-repruxe9senter-les-fonctions}

Les fonctions peuvent être représentées de plusieurs manières utiles~:

\begin{itemize}
\tightlist
\item
  Formules~: par exemple, \(f(x) = \sin x + x^2\).
\item
  Graphiques~: tracer tous les points \((x, f(x))\) dans le plan de
  coordonnées.
\item
  Tableaux~: appariement d'entrées et de sorties pour des ensembles de
  données discrets.
\item
  Descriptions verbales : « Attribuez à chaque élève sa note. »
\end{itemize}

Chaque représentation met en évidence différents aspects d'une même
fonction.

\subsubsection{Terminologie}\label{terminologie}

\begin{itemize}
\tightlist
\item
  Variable indépendante : l'entrée (généralement écrite \(x\)).
\item
  Variable dépendante : la sortie (généralement écrite \(y\), où
  \(y = f(x)\)).
\item
  Notation de fonction : \(f(x)\) se lit « \(f\) de \(x\) ».
\end{itemize}

\subsubsection{Pourquoi les fonctions sont importantes dans le
calcul}\label{pourquoi-les-fonctions-sont-importantes-dans-le-calcul}

Le calcul est l'étude de la façon dont les fonctions changent. Les
dérivés mesurent les taux de changement instantanés, tandis que les
intégrales mesurent les effets accumulés. Pour maîtriser ces idées, nous
avons d'abord besoin d'une solide compréhension de ce que sont les
fonctions et de la façon dont elles se comportent.

\subsubsection{Exercices}\label{exercices}

\begin{enumerate}
\def\labelenumi{\arabic{enumi}.}
\item
  Pour la fonction \(f(x) = 3x - 2\)~:- Recherchez le domaine, le
  codomaine et la plage.
\item
  La fonction \(h(x) = \sqrt{x-1}\) est définie pour quelles entrées ?
  Quelle est sa portée ?
\item
  Donnez un exemple concret d'une fonction de votre vie quotidienne.
  Indiquez clairement le domaine et le codomaine.
\item
  Esquissez le graphique de \(f(x) = |x|\). Quelle est la portée ?
\item
  Supposons que \(g(x) = \dfrac{1}{x^2+1}\). Expliquez pourquoi sa plage
  est l'intervalle \((0, 1]\).
\end{enumerate}

\subsection{1.2 Graphiques et
transformations}\label{graphiques-et-transformations}

Une fonction peut être comprise non seulement par des formules mais
aussi par son graphique. Le graphe d'une fonction \(f\) est l'ensemble
de toutes les paires ordonnées \((x, f(x))\), où \(x\) appartient au
domaine de \(f\). Tracer ces paires dans le plan de coordonnées donne
une image du comportement de la fonction.

\subsubsection{Graphiques de base}\label{graphiques-de-base}

Certains graphiques sont si fondamentaux qu'ils méritent d'être
mémorisés~:

\begin{itemize}
\tightlist
\item
  \(f(x) = x\) : une ligne droite passant par l'origine.
\item
  \(f(x) = x^2\) : une parabole s'ouvrant vers le haut.
\item
  \(f(x) = |x|\) : un graphique en forme de « V ».
\item
  \(f(x) = \frac{1}{x}\) : une hyperbole à deux branches.
\item
  \(f(x) = \sin x\) : une courbe périodique en forme d'onde.
\end{itemize}

Ceux-ci servent de base à des fonctions plus complexes.

\subsubsection{Transformations}\label{transformations}

Les graphiques peuvent être déplacés, étirés ou reflétés à l'aide de
règles simples~:

\begin{enumerate}
\def\labelenumi{\arabic{enumi}.}
\item
  Décalages verticaux~: l'ajout d'une constante déplace le graphique
  vers le haut ou vers le bas.

  \[
  y = f(x) + c \quad \text{is } f(x) \text{ shifted upward by } c.
  \]
\item
  Décalages horizontaux~: l'ajout à l'intérieur de l'argument déplace le
  graphique vers la gauche ou la droite.

  \[
  y = f(x - c) \quad \text{is } f(x) \text{ shifted right by } c.
  \]
\item
  Mise à l'échelle verticale~: la multiplication par une constante étire
  ou comprime le graphique verticalement.

  \[
  y = a f(x), \quad a > 1 \text{ stretches; } 0 < a < 1 \text{ compresses.}
  \]
\item
  Mise à l'échelle horizontale~: la multiplication à l'intérieur de
  l'argument étire ou compresse le graphique horizontalement.

  \[
  y = f(bx), \quad b > 1 \text{ compresses toward the } y\text{-axis}.
  \]
\item
  Réflexions :

  \begin{itemize}
  \tightlist
  \item
    \(y = -f(x)\)~: réflexion sur l'axe \(x\).
  \item
    \(y = f(-x)\)~: réflexion sur l'axe \(y\).
  \end{itemize}
\end{enumerate}

\subsubsection{Combiner les
transformations}\label{combiner-les-transformations}

Les graphiques complexes résultent souvent de la combinaison de
plusieurs transformations en séquence. Par exemple~:

\[
y = 2(x-1)^2 + 3
\]

est obtenu en prenant la parabole \(y = x^2\), en la décalant vers la
droite de 1, en l'étirant verticalement de 2 et en la déplaçant vers le
haut de 3.

\subsubsection{Exercices}\label{exercices-1}

\begin{enumerate}
\def\labelenumi{\arabic{enumi}.}
\tightlist
\item
  Dessinez le graphique de \(y = (x+2)^2 - 1\). Identifiez la séquence
  de transformations de \(y = x^2\).
\item
  Qu'arrive-t-il au graphique de \(y = f(x)\) si on remplace \(x\) par
  \(-x\) ? Essayez-le avec \(f(x) = \sqrt{x}\).
\item
  Décrivez les transformations qui transforment \(y = \sin x\) en
  \(y = 3\sin(x - \pi/4)\).4. Dessinez le graphique de
  \(y = |x-1| + 2\). Indiquez le sommet et la pente de chaque branche.
\item
  Pour \(y = \frac{1}{x-2}\), expliquez comment le graphique de
  \(y = \frac{1}{x}\) a été transformé.
\end{enumerate}

\subsection{1.3 Idée intuitive des
limites}\label{iduxe9e-intuitive-des-limites}

Dans de nombreuses situations, la valeur d'une fonction en un point est
moins importante que les valeurs qu'elle prend à proximité de ce point.
Le concept de limite capture cette idée.

\subsubsection{Approche d'une valeur}\label{approche-dune-valeur}

Imaginez-vous marcher vers un mur. Avant même de le toucher, vous vous
rapprochez de plus en plus. De la même manière, lorsque \(x\) se
rapproche d'un nombre \(a\), les valeurs de \(f(x)\) peuvent se
rapprocher d'un nombre \(L\). Nous disons alors :

\[
\lim_{x \to a} f(x) = L.
\]

Cela exprime l'idée que \(f(x)\) peut être rendu aussi proche que nous
le souhaitons de \(L\), simplement en rapprochant \(x\) suffisamment de
\(a\).

\subsubsection{Exemples}\label{exemples-1}

\begin{enumerate}
\def\labelenumi{\arabic{enumi}.}
\item
  Pour \(f(x) = 2x + 3\)~: Comme \(x \to 1\), \(f(x) \to 5\).
\item
  Pour \(f(x) = \dfrac{\sin x}{x}\)~: Comme \(x \to 0\), la fonction se
  rapproche de 1, même si \(f(0)\) n'est pas défini.
\item
  Pour \(f(x) = \dfrac{1}{x}\)~: Comme \(x \to 0^+\) (en venant de la
  droite), \(f(x) \to +\infty\). Comme \(x \to 0^-\) (en venant de la
  gauche), \(f(x) \to -\infty\). Les comportements gauche et droite
  étant différents, la limite à 0 n'existe pas.
\end{enumerate}

\subsubsection{Importance des limites}\label{importance-des-limites}

\begin{itemize}
\tightlist
\item
  Ils nous permettent de définir des fonctions là où elles ne sont pas
  définies initialement.
\item
  Ils capturent les comportements à proximité des discontinuités et des
  singularités.
\item
  Ils constituent le fondement des dérivées (taux de variation
  instantanés) et des intégrales (zones comme limites des sommes).
\end{itemize}

\subsubsection{Limites unilatérales}\label{limites-unilatuxe9rales}

Parfois, les comportements de gauche et de droite doivent être étudiés
séparément :

\[
\lim_{x \to a^-} f(x), \quad \lim_{x \to a^+} f(x).
\]

Si les deux sont d'accord, alors la limite bilatérale existe.

\subsubsection{Exercices}\label{exercices-2}

\begin{enumerate}
\def\labelenumi{\arabic{enumi}.}
\tightlist
\item
  Calculez \(\lim_{x \to 2} (3x^2 - x)\).
\item
  Qu'est-ce que \(\lim_{x \to 0} \frac{\sin x}{x}\)~? Utilisez
  l'intuition du graphique de \(\sin x\).
\item
  Évaluez \(\lim_{x \to 0} |x|/x\). La limite bilatérale existe-t-elle ?
\item
  Recherchez \(\lim_{x \to \infty} \frac{1}{x}\). Interprétez ce
  résultat avec des mots.
\item
  Pour \(f(x) = \frac{x^2-1}{x-1}\), qu'est-ce que
  \(\lim_{x \to 1} f(x)\)~? Comparez avec la valeur de \(f(1)\).
\end{enumerate}

\subsection{1.4 Définition formelle des
limites}\label{duxe9finition-formelle-des-limites}

L'idée intuitive d'une limite peut être précisée en utilisant la
définition epsilon -- delta. Cela nous donne une manière rigoureuse de
dire que \(f(x)\) se rapproche d'une valeur \(L\) alors que \(x\) se
rapproche de \(a\).

\subsubsection{La définition}\label{la-duxe9finition}

Nous écrivons

\[
\lim_{x \to a} f(x) = L
\]

si la condition suivante est remplie :

Pour chaque \(\varepsilon > 0\) (aussi petit soit-il), il existe un
\(\delta > 0\) tel que chaque fois que

\[
0 < |x - a| < \delta,
\]

il s'ensuit que

\[
|f(x) - L| < \varepsilon.
\]En mots~: nous pouvons rendre \(f(x)\) aussi proche que nous le
souhaitons de \(L\), à condition que \(x\) soit suffisamment proche de
\(a\) (mais pas égal à \(a\)).

\subsubsection{Exemple 1~: Fonction
linéaire}\label{exemple-1-fonction-linuxe9aire}

Pour \(f(x) = 2x + 1\), montrez que \(\lim_{x \to 3} f(x) = 7\).

\begin{itemize}
\tightlist
\item
  Nous voulons \(|f(x) - 7| < \varepsilon\).
\item
  Mais \(f(x) - 7 = 2x + 1 - 7 = 2(x - 3)\).
\item
  Alors \(|f(x) - 7| = 2|x - 3|\).
\item
  Si nous choisissons \(\delta = \varepsilon / 2\), alors chaque fois
  que \(|x - 3| < \delta\), nous avons \(|f(x) - 7| < \varepsilon\).
  Cela prouve la limite.
\end{itemize}

\subsubsection{Exemple 2~: Fonction
réciproque}\label{exemple-2-fonction-ruxe9ciproque}

Pour \(f(x) = \frac{1}{x}\), considérez
\(\lim_{x \to 2} f(x) = \tfrac{1}{2}\).

\begin{itemize}
\tightlist
\item
  Nous voulons \(\left|\frac{1}{x} - \frac{1}{2}\right| < \varepsilon\).
\item
  Cette inégalité nécessite une manipulation algébrique, mais elle peut
  être satisfaite en choisissant \(\delta\) en fonction de
  \(\varepsilon\). Le processus est plus compliqué, mais le principe est
  le même.
\end{itemize}

\subsubsection{Pourquoi c'est important}\label{pourquoi-cest-important}

\begin{itemize}
\tightlist
\item
  La définition epsilon-delta garantit que les limites ne sont pas
  vagues ou basées uniquement sur l'intuition.
\item
  C'est le fondement de la continuité, des dérivées et des intégrales.
\item
  Même si les débutants peuvent trouver cela abstrait, travailler avec
  des exemples simples renforce la familiarité.
\end{itemize}

\subsubsection{Exercices}\label{exercices-3}

\begin{enumerate}
\def\labelenumi{\arabic{enumi}.}
\tightlist
\item
  En utilisant la définition epsilon-delta, prouvez que
  \(\lim_{x \to 4} (x+1) = 5\).
\item
  Montrez que \(\lim_{x \to 0} 5x = 0\) en utilisant la définition
  formelle.
\item
  Expliquez pourquoi \(\lim_{x \to 0} \frac{1}{x}\) n'existe pas.
\item
  Pour \(f(x) = x^2\), montrez que \(\lim_{x \to 2} f(x) = 4\).
\item
  Dans vos propres mots, expliquez le rôle de \(\varepsilon\) et
  \(\delta\) dans la définition d'une limite.
\end{enumerate}

\subsection{1.5 Continuité}\label{continuituxe9}

Une fonction est continue si son graphique peut être dessiné sans
retirer le crayon du papier. Plus précisément, la continuité garantit
que de petits changements dans l'entrée produisent de petits changements
dans la sortie.

\subsubsection{Définition}\label{duxe9finition-1}

Une fonction \(f\) est continue en un point \(a\) si trois conditions
sont satisfaites :

\begin{enumerate}
\def\labelenumi{\arabic{enumi}.}
\tightlist
\item
  \(f(a)\) est défini.
\item
  \(\lim_{x \to a} f(x)\) existe.
\item
  \(\lim_{x \to a} f(x) = f(a)\).
\end{enumerate}

Si une fonction est continue en tout point d'un intervalle, on dit
qu'elle est continue sur cet intervalle.

\subsubsection{Exemples}\label{exemples-2}

\begin{enumerate}
\def\labelenumi{\arabic{enumi}.}
\item
  Fonctions polynomiales~: des fonctions comme \(f(x) = x^2 + 3x - 5\)
  sont continues partout sur \(\mathbb{R}\).
\item
  Fonctions rationnelles~: \(f(x) = \frac{1}{x-1}\) est continue partout
  sauf en \(x = 1\), où elle n'est pas définie.
\item
  Fonctions par morceaux~:

  \[
  f(x) =
  \begin{cases}
  x^2 & x < 1, \\
  2 & x = 1, \\
  x+1 & x > 1,
  \end{cases}
  \]

  Cette fonction a un « saut » à \(x = 1\), elle n'y est donc pas
  continue.
\end{enumerate}

\subsubsection{Types de
discontinuités}\label{types-de-discontinuituxe9s}

\begin{enumerate}
\def\labelenumi{\arabic{enumi}.}
\tightlist
\item
  Discontinuité amovible : Un « trou » dans le graphique. Exemple~:
  \(f(x) = \frac{x^2-1}{x-1}\) à \(x=1\).2. Discontinuité du saut : Les
  limites gauche et droite sont différentes.
\item
  Discontinuité infinie : La fonction va à \(\pm\infty\) près d'un
  point, comme avec \(f(x) = 1/x\) près de \(x = 0\).
\end{enumerate}

\subsubsection{Le théorème des valeurs
intermédiaires}\label{le-thuxe9oruxe8me-des-valeurs-intermuxe9diaires}

Si une fonction est continue sur un intervalle \([a, b]\), alors pour
tout nombre \(N\) compris entre \(f(a)\) et \(f(b)\), il existe un
\(c \in [a, b]\) tel que \(f(c) = N\).

Cette propriété est cruciale pour prouver l'existence de racines et de
solutions aux équations.

\subsubsection{Exercices}\label{exercices-4}

\begin{enumerate}
\def\labelenumi{\arabic{enumi}.}
\tightlist
\item
  Décidez si la fonction \(f(x) = |x|\) est continue à \(x = 0\).
\item
  Identifiez les points de discontinuité pour
  \(f(x) = \frac{x+2}{x^2-1}\).
\item
  Expliquez pourquoi chaque fonction polynomiale est continue partout.
\item
  Donnez un exemple de fonction avec une discontinuité de saut. Dessinez
  son graphique.
\item
  Utilisez le théorème des valeurs intermédiaires pour montrer que
  l'équation \(x^3 + x - 1 = 0\) a une solution comprise entre 0 et 1.
\end{enumerate}

\section{Chapitre 2. Dérivés}\label{chapitre-2.-duxe9rivuxe9s}

\subsection{2.1 La dérivée comme taux de
variation}\label{la-duxe9rivuxe9e-comme-taux-de-variation}

La dérivée est l'une des idées centrales du calcul. Il mesure la façon
dont une fonction change à mesure que son entrée change - en d'autres
termes, le taux de variation de la sortie par rapport à l'entrée.

\subsubsection{Taux de variation moyen}\label{taux-de-variation-moyen}

Pour une fonction \(f(x)\), le taux de variation moyen entre deux points
\(x = a\) et \(x = b\) est

\[
\frac{f(b) - f(a)}{b - a}.
\]

Il s'agit de la pente de la ligne sécante passant par les points
\((a, f(a))\) et \((b, f(b))\).

\subsubsection{Taux de changement
instantané}\label{taux-de-changement-instantanuxe9}

Pour mesurer la vitesse à laquelle \(f(x)\) change en un seul point,
nous laissons l'intervalle se réduire~:

\[
f'(a) = \lim_{h \to 0} \frac{f(a+h) - f(a)}{h}.
\]

Cette limite, si elle existe, est appelée la dérivée de \(f\) en \(a\).
Géométriquement, c'est la pente de la droite tangente au graphique de
\(f\) au point \((a, f(a))\).

\subsubsection{Notation}\label{notation}

\begin{itemize}
\tightlist
\item
  \(f'(x)\) : notation première.
\item
  \(\dfrac{dy}{dx}\) : notation Leibniz, utilisée lorsque \(y = f(x)\).
\item
  \(Df(x)\) : notation de l'opérateur.
\end{itemize}

Tous ces symboles font référence au même concept.

\subsubsection{Exemples}\label{exemples-3}

\begin{enumerate}
\def\labelenumi{\arabic{enumi}.}
\item
  Pour \(f(x) = x^2\)~:

  \[
  f'(x) = \lim_{h \to 0} \frac{(x+h)^2 - x^2}{h} = \lim_{h \to 0} \frac{2xh + h^2}{h} = 2x.
  \]

  La pente de la parabole à \(x\) est \(2x\).
\item
  Pour \(f(x) = \sin x\)~:

  \[
  f'(x) = \cos x.
  \]
\item
  Pour \(f(x) = c\) (une constante)~:

  \[
  f'(x) = 0.
  \]

  Une fonction constante ne change jamais.
\end{enumerate}

\subsubsection{Interprétation}\label{interpruxe9tation}

\begin{itemize}
\tightlist
\item
  En physique : Si \(s(t)\) est la position, alors \(s'(t)\) est la
  vitesse.
\item
  En économie : Si \(C(x)\) est le coût, alors \(C'(x)\) est le coût
  marginal.
\item
  En biologie : Si \(P(t)\) est la population, alors \(P'(t)\) est le
  taux de croissance.
\end{itemize}

La dérivée rend le « changement » précis dans de nombreux contextes.

\subsubsection{Exercices}\label{exercices-5}

\begin{enumerate}
\def\labelenumi{\arabic{enumi}.}
\tightlist
\item
  Calculez \(f'(x)\) pour \(f(x) = 3x^2 - 2x + 1\).2. Trouvez la pente
  de la ligne tangente à \(f(x) = x^3\) à \(x = 2\).
\item
  Si \(s(t) = t^2 + 2t\) représente la distance en mètres, quelle est la
  vitesse à \(t = 5\)~?
\item
  Utilisez la définition de limite pour calculer la dérivée de
  \(f(x) = \frac{1}{x}\).
\item
  Esquissez le graphique de \(y = x^2\) et tracez la ligne tangente à
  \(x = 1\).
\end{enumerate}

\subsection{2.2 Règles de
différenciation}\label{ruxe8gles-de-diffuxe9renciation}

Une fois la dérivée définie, nous avons besoin de moyens efficaces pour
la calculer. Les règles de différenciation sont des raccourcis qui nous
évitent d'appliquer à plusieurs reprises la définition limite.

\subsubsection{La règle constante}\label{la-ruxe8gle-constante}

Si \(f(x) = c\) où \(c\) est une constante, alors

\[
f'(x) = 0.
\]

\subsubsection{La règle du pouvoir}\label{la-ruxe8gle-du-pouvoir}

Pour \(f(x) = x^n\) où \(n\) est un nombre réel,

\[
\frac{d}{dx} \big( x^n \big) = n x^{n-1}.
\]

Exemples~:

\begin{itemize}
\tightlist
\item
  \(\frac{d}{dx}(x^2) = 2x\).
\item
  \(\frac{d}{dx}(x^5) = 5x^4\).
\item
  \(\frac{d}{dx}(\sqrt{x}) = \frac{1}{2\sqrt{x}}\).
\end{itemize}

\subsubsection{La règle multiple
constante}\label{la-ruxe8gle-multiple-constante}

Si \(f(x) = c \cdot g(x)\), alors

\[
f'(x) = c \cdot g'(x).
\]

\subsubsection{Les règles de somme et de
différence}\label{les-ruxe8gles-de-somme-et-de-diffuxe9rence}

\begin{itemize}
\tightlist
\item
  \((f + g)' = f' + g'\).
\item
  \((f - g)' = f' - g'\).
\end{itemize}

\subsubsection{La règle du produit}\label{la-ruxe8gle-du-produit}

Pour \(f(x)\) et \(g(x)\)~:

\[
(fg)' = f'g + fg'.
\]

Exemple~: Si \(f(x) = x^2\), \(g(x) = \sin x\)~:

\[
(fg)' = (2x)(\sin x) + (x^2)(\cos x).
\]

\subsubsection{La règle du quotient}\label{la-ruxe8gle-du-quotient}

Pour \(f(x)\) et \(g(x)\)~:

\[
\left(\frac{f}{g}\right)' = \frac{f'g - fg'}{g^2}, \quad g(x) \neq 0.
\]

Exemple~: Si \(f(x) = x^2\), \(g(x) = x+1\)~:

\[
\left(\frac{x^2}{x+1}\right)' = \frac{(2x)(x+1) - (x^2)(1)}{(x+1)^2}.
\]

\subsubsection{Dérivés de fonctions
communes}\label{duxe9rivuxe9s-de-fonctions-communes}

\begin{itemize}
\tightlist
\item
  \(\frac{d}{dx}(\sin x) = \cos x\).
\item
  \(\frac{d}{dx}(\cos x) = -\sin x\).
\item
  \(\frac{d}{dx}(e^x) = e^x\).
\item
  \(\frac{d}{dx}(\ln x) = \frac{1}{x}, \quad x > 0\).
\end{itemize}

\subsubsection{Exercices}\label{exercices-6}

\begin{enumerate}
\def\labelenumi{\arabic{enumi}.}
\tightlist
\item
  Différenciez \(f(x) = 7x^3 - 4x + 9\).
\item
  Utilisez la règle du produit pour trouver la dérivée de
  \(f(x) = x^2 e^x\).
\item
  Appliquez la règle du quotient à \(f(x) = \frac{\sin x}{x}\).
\item
  Calculez \(\frac{d}{dx}(\ln(x^2))\) en utilisant la chaîne de règles.
\item
  Montrez que la dérivée de \(f(x) = \frac{1}{x}\) est
  \(-\frac{1}{x^2}\).
\end{enumerate}

\subsection{2.3 La règle de la
chaîne}\label{la-ruxe8gle-de-la-chauxeene}

Souvent, les fonctions sont créées en combinant des fonctions plus
simples. Pour différencier de telles fonctions composites, nous
utilisons la règle de la chaîne.

\subsubsection{La règle}\label{la-ruxe8gle}

Si \(y = f(g(x))\), alors

\[
\frac{dy}{dx} = f'(g(x)) \cdot g'(x).
\]

En mots~: différenciez la fonction extérieure, gardez l'intérieur
inchangé, puis multipliez par la dérivée de l'intérieur.

\subsubsection{Exemples}\label{exemples-4}

\begin{enumerate}
\def\labelenumi{\arabic{enumi}.}
\item
  Carré d'une fonction linéaire

  \[
  y = (3x+2)^2
  \]

  Fonction externe~: \(f(u) = u^2\), fonction interne~: \(g(x) = 3x+2\).

  \[
  y' = 2(3x+2) \cdot 3 = 6(3x+2).
  \]
\item
  Exponentielle avec quadratique à l'intérieur

  \[
  y = e^{x^2}
  \]

  Fonction externe~: \(f(u) = e^u\), fonction interne~: \(g(x) = x^2\).

  \[y' = e^{x^2} \cdot 2x = 2x e^{x^2}.
  \]
\item
  Logarithm with root inside

  \[
  y = \ln(\sqrt{x})
  \]

  Outer: \(f(u) = \ln u\), inner: \(g(x) = \sqrt{x}\).

  \[
  y' = \frac{1}{\sqrt{x}} \cdot \frac{1}{2\sqrt{x}} = \frac{1}{2x}.
  \]
\end{enumerate}

\subsubsection{Generalized Chain Rule}\label{generalized-chain-rule}

For multiple nested functions \(y = f(g(h(x)))\):

\[
\frac{dy}{dx} = f'(g(h(x))) \cdot g'(h(x)) \cdot h'(x).
\]

This extends naturally to deeper compositions.

\subsubsection{Why the Chain Rule
Matters}\label{why-the-chain-rule-matters}

\begin{itemize}
\tightlist
\item
  It handles nearly all real-world models where one quantity depends on
  another indirectly.
\item
  It connects calculus with physics (e.g., velocity depending on time
  through position).
\item
  It is essential in implicit differentiation and advanced topics.
\end{itemize}

\subsubsection{Exercises}\label{exercises}

\begin{enumerate}
\def\labelenumi{\arabic{enumi}.}
\tightlist
\item
  Differentiate \(y = (5x^2 + 1)^3\).
\item
  Find \(\frac{d}{dx}(\sin(3x))\).
\item
  Compute \(\frac{d}{dx}(\ln(1+x^2))\).
\item
  Differentiate \(y = \cos^2(x)\).
\item
  Apply the generalized chain rule to \(y = e^{\sin(x^2)}\).
\end{enumerate}

\subsection{2.4 Implicit
Differentiation}\label{implicit-differentiation}

Not all functions are given in the form \(y = f(x)\). Sometimes \(x\)
and \(y\) are related by an equation, and solving explicitly for \(y\)
is difficult or impossible. In such cases, we use implicit
differentiation.

\subsubsection{The Idea}\label{the-idea}

If an equation involves both \(x\) and \(y\), we can differentiate both
sides with respect to \(x\), treating \(y\) as a function of \(x\). Each
time we differentiate a term involving \(y\), we multiply by
\(\frac{dy}{dx}\).

\subsubsection{Example 1: A Circle}\label{example-1-a-circle}

Equation:

\[
x^2 + y^2 = 25
\]

Differentiate with respect to \(x\):

\[
2x + 2y \frac{dy}{dx} = 0.
\]

Solve for \(\frac{dy}{dx}\):

\[
\frac{dy}{dx} = -\frac{x}{y}.
\]

This gives the slope of the tangent to the circle at any point.

\subsubsection{Example 2: A Product of
Variables}\label{example-2-a-product-of-variables}

Equation:

\[
xy = 1
\]

Differentiate:

\[
x\frac{dy}{dx} + y = 0.
\]

So,

\[
\frac{dy}{dx} = -\frac{y}{x}.
\]

\subsubsection{Example 3: Trigonometric
Relation}\label{example-3-trigonometric-relation}

Equation:

\[
\sin(xy) = x
\]

Differentiate:

\[
\cos(xy) \cdot \Big(y + x\frac{dy}{dx}\Big) = 1.
\]

Solve for \(\frac{dy}{dx}\):

\[
\frac{dy}{dx} = \frac{1 - y\cos(xy)}{x\cos(xy)}.
\]

\subsubsection{Pourquoi la différenciation implicite est
utile}\label{pourquoi-la-diffuxe9renciation-implicite-est-utile}

\begin{itemize}
\tightlist
\item
  De nombreuses courbes importantes (cercles, ellipses, hyperboles) sont
  naturellement définies implicitement.
\item
  Cela nous permet de différencier des équations sans résoudre au
  préalable \(y\).
\item
  C'est une étape clé dans des sujets plus avancés tels que les taux
  connexes et les équations différentielles.
\end{itemize}

\subsubsection{Exercices}\label{exercices-7}

\begin{enumerate}
\def\labelenumi{\arabic{enumi}.}
\tightlist
\item
  Pour la courbe \(x^2 + xy + y^2 = 7\), recherchez \(\frac{dy}{dx}\).
\item
  Différenciez implicitement \(\cos(x) + \cos(y) = 1\).
\item
  Trouvez la pente de la ligne tangente à \(x^3 + y^3 = 9\) au point
  \((1, 2)\).4. Étant donné \(x^2 + y^2 = 10\), calculez
  \(\frac{dy}{dx}\) lorsque \((x, y) = (1, 3)\).
\item
  Différenciez \(e^{xy} = x + y\) pour trouver \(\frac{dy}{dx}\).
\end{enumerate}

\subsection{2.5 Dérivés d'ordre
supérieur}\label{duxe9rivuxe9s-dordre-supuxe9rieur}

Jusqu'à présent, nous avons étudié la dérivée première, qui mesure le
taux de variation d'une fonction. Mais les dérivés eux-mêmes peuvent
également être différenciés, donnant naissance à des dérivés d'ordre
supérieur.

\subsubsection{Définition}\label{duxe9finition-2}

\begin{itemize}
\item
  La dérivée seconde de \(f\) est la dérivée de la dérivée :

  \[
  f''(x) = \frac{d}{dx}\left(f'(x)\right).
  \]
\item
  Plus généralement, la \(n\)-ième dérivée s'écrit

  \[
  f^{(n)}(x) = \frac{d^n}{dx^n} f(x).
  \]
\end{itemize}

\subsubsection{Exemples}\label{exemples-5}

\begin{enumerate}
\def\labelenumi{\arabic{enumi}.}
\item
  \(f(x) = x^3\)

  \begin{itemize}
  \tightlist
  \item
    Dérivée première : \(f'(x) = 3x^2\).
  \item
    Dérivée seconde : \(f''(x) = 6x\).
  \item
    Troisième dérivée : \(f^{(3)}(x) = 6\).
  \item
    Quatrième dérivée : \(f^{(4)}(x) = 0\).
  \end{itemize}
\item
  \(f(x) = \sin x\)

  \begin{itemize}
  \tightlist
  \item
    \(f'(x) = \cos x\).
  \item
    \(f''(x) = -\sin x\).
  \item
    \(f^{(3)}(x) = -\cos x\).
  \item
    \(f^{(4)}(x) = \sin x\). Les dérivées se répètent dans un cycle de
    longueur 4.
  \end{itemize}
\item
  \(f(x) = e^x\)

  \begin{itemize}
  \tightlist
  \item
    Chaque dérivé est \(e^x\).
  \end{itemize}
\end{enumerate}

\#\#\#~Candidatures

\begin{itemize}
\item
  Concavité : Le signe de \(f''(x)\) indique si le graphique de \(f\)
  est concave vers le haut (\(f'' > 0\)) ou concave vers le bas
  (\(f'' < 0\)).
\item
  Points d'inflexion : Points où \(f''(x) = 0\) et la concavité
  changent.
\item
  Mouvement : En physique, si \(s(t)\) est la position :

  \begin{itemize}
  \tightlist
  \item
    \(s'(t)\) = vitesse,
  \item
    \(s''(t)\) = accélération,
  \item
    \(s^{(3)}(t)\) = à-coup (taux de changement d'accélération).
  \end{itemize}
\item
  Approximations : Les dérivées d'ordre supérieur apparaissent dans les
  séries de Taylor, utilisées pour approximer des fonctions.
\end{itemize}

\subsubsection{Exercices}\label{exercices-8}

\begin{enumerate}
\def\labelenumi{\arabic{enumi}.}
\tightlist
\item
  Calculez les quatre premières dérivées de \(f(x) = \cos x\).
\item
  Recherchez \(f''(x)\) pour \(f(x) = x^4 - 2x^2 + 3\).
\item
  Pour \(f(x) = e^{2x}\), montrez que \(f^{(n)}(x) = 2^n e^{2x}\).
\item
  Déterminez les intervalles où \(f(x) = x^3 - 3x\) est concave vers le
  haut et concave vers le bas.
\item
  Si \(s(t) = t^3 - 6t^2 + 9t\), trouvez la vitesse et l'accélération à
  \(t = 2\).
\end{enumerate}

\section{Chapitre 3. Applications des produits
dérivés}\label{chapitre-3.-applications-des-produits-duxe9rivuxe9s}

\subsection{3.1 Tangentes et normales}\label{tangentes-et-normales}

L'une des premières applications des dérivées consiste à trouver les
équations des droites tangentes et normales à une courbe. Ces lignes
capturent la géométrie locale d'une fonction en un point donné.

\subsubsection{Ligne tangente}\label{ligne-tangente}

La ligne tangente à une courbe \(y = f(x)\) en un point \((a, f(a))\)
est la ligne qui « touche » simplement le graphique à cet endroit et a
la même pente que la courbe.

La pente de la tangente est donnée par la dérivée :

\[
m_{\text{tangent}} = f'(a).
\]

Ainsi, l'équation de la tangente à \((a, f(a))\) est

\[
y - f(a) = f'(a)(x - a).
\]

\subsubsection{Ligne normale}\label{ligne-normale}

La droite normale est perpendiculaire à la droite tangente au même
point. Sa pente est l'inverse négatif de la pente tangente :

\[m_{\text{normal}} = -\frac{1}{f'(a)}.
\]

So the equation of the normal line is

\[
y - f(a) = -\frac{1}{f'(a)} (x - a), \quad f'(a) \neq 0.
\]

\subsubsection{Examples}\label{examples}

\begin{enumerate}
\def\labelenumi{\arabic{enumi}.}
\item
  \(f(x) = x^2\) at \(x = 1\).

  \begin{itemize}
  \tightlist
  \item
    \(f(1) = 1\), \(f'(x) = 2x\), so \(f'(1) = 2\).
  \item
    Tangent: \(y - 1 = 2(x - 1)\), or \(y = 2x - 1\).
  \item
    Normal: slope = \(-\tfrac{1}{2}\), so equation is
    \(y - 1 = -\tfrac{1}{2}(x - 1)\).
  \end{itemize}
\item
  \(f(x) = \sin x\) at \(x = \tfrac{\pi}{4}\).

  \begin{itemize}
  \tightlist
  \item
    \(f(\tfrac{\pi}{4}) = \tfrac{\sqrt{2}}{2}\),
    \(f'(\tfrac{\pi}{4}) = \cos(\tfrac{\pi}{4}) = \tfrac{\sqrt{2}}{2}\).
  \item
    Tangent:
    \(y - \tfrac{\sqrt{2}}{2} = \tfrac{\sqrt{2}}{2}(x - \tfrac{\pi}{4})\).
  \end{itemize}
\end{enumerate}

\subsubsection{Why Tangents and Normals
Matter}\label{why-tangents-and-normals-matter}

\begin{itemize}
\tightlist
\item
  Tangents approximate the curve locally (linear approximation).
\item
  Normals are useful in geometry, optics (reflection/refraction), and
  mechanics (force directions).
\item
  Both play a role in optimization and curvature studies.
\end{itemize}

\subsubsection{Exercises}\label{exercises-1}

\begin{enumerate}
\def\labelenumi{\arabic{enumi}.}
\tightlist
\item
  Find the tangent and normal lines to \(y = x^3\) at \(x = 2\).
\item
  Determine the tangent and normal lines to \(y = e^x\) at \(x = 0\).
\item
  For \(y = \ln x\), compute the tangent line at \(x = 1\).
\item
  A circle is given by \(x^2 + y^2 = 9\). Use implicit differentiation
  to find the slope of the tangent at \((0,3)\).
\item
  Sketch the graph of \(y = \sqrt{x}\) and draw the tangent and normal
  lines at \(x = 4\).
\end{enumerate}

\subsection{3.2 Related Rates}\label{related-rates}

In many real-world problems, two or more quantities change with respect
to time, and their rates of change are connected. Related rates problems
use derivatives to describe these relationships.

\subsubsection{General Approach}\label{general-approach}

\begin{enumerate}
\def\labelenumi{\arabic{enumi}.}
\tightlist
\item
  Identify the variables that depend on time \(t\).
\item
  Write an equation relating the variables.
\item
  Differentiate both sides with respect to \(t\), applying the chain
  rule.
\item
  Substitute the known values at the given instant.
\item
  Solve for the unknown rate.
\end{enumerate}

\subsubsection{Example 1: Expanding
Circle}\label{example-1-expanding-circle}

A circle has radius \(r\), which increases at the rate of
\(\frac{dr}{dt} = 2 \,\text{cm/s}\). Find the rate at which the area
\(A = \pi r^2\) increases when \(r = 5\).

Differentiate:

\[
\frac{dA}{dt} = 2\pi r \frac{dr}{dt}.
\]

Substitute:

\[
\frac{dA}{dt} = 2\pi (5)(2) = 20\pi \,\text{cm}^2/\text{s}.
\]

\subsubsection{Example 2: Sliding
Ladder}\label{example-2-sliding-ladder}

A 10 ft ladder leans against a wall. The bottom slides away at
\(\frac{dx}{dt} = 1 \,\text{ft/s}\). How fast is the top sliding down
when the bottom is 6 ft from the wall?

Equation: \(x^2 + y^2 = 100\), where \(y\) is the height.

Differentiate:

\[
2x \frac{dx}{dt} + 2y \frac{dy}{dt} = 0.
\]

At \(x = 6\), \(y = 8\). Substitute:

\[
2(6)(1) + 2(8)\frac{dy}{dt} = 0 \quad \Rightarrow \quad \frac{dy}{dt} = -\tfrac{6}{8} = -\tfrac{3}{4}.
\]Ainsi, le haut glisse vers \(0.75 \,\text{ft/s}\).

\subsubsection{Exemple 3 : De l'eau dans un
cône}\label{exemple-3-de-leau-dans-un-cuxf4ne}

L'eau est versée dans un cône de hauteur 12 cm et de rayon 6 cm. Lorsque
l'eau atteint 4 cm de profondeur, le niveau d'eau monte à
\(2 \,\text{cm/s}\). A quel rythme le volume augmente-t-il ?

Équation~: \(V = \tfrac{1}{3}\pi r^2 h\). En utilisant la similarité,
\(r = \tfrac{h}{2}\). Remplacement~:

\[
V = \tfrac{1}{12}\pi h^3.
\]

Différencier~:

\[
\frac{dV}{dt} = \tfrac{1}{4}\pi h^2 \frac{dh}{dt}.
\]

À \(h = 4\), \(\frac{dh}{dt} = 2\)~:

\[
\frac{dV}{dt} = \tfrac{1}{4}\pi (16)(2) = 8\pi \,\text{cm}^3/\text{s}.
\]

\subsubsection{Pourquoi les taux associés sont
importants}\label{pourquoi-les-taux-associuxe9s-sont-importants}

\begin{itemize}
\tightlist
\item
  Ils décrivent le mouvement et le changement en physique, en ingénierie
  et en biologie.
\item
  Ils relient la géométrie au calcul à travers des processus dépendants
  du temps.
\item
  Ils nous entraînent à modéliser mathématiquement les systèmes
  dynamiques.
\end{itemize}

\subsubsection{Exercices}\label{exercices-9}

\begin{enumerate}
\def\labelenumi{\arabic{enumi}.}
\tightlist
\item
  Un ballon est gonflé de manière à ce que son rayon augmente à
  \(0.5 \,\text{cm/s}\). Trouvez à quelle vitesse son volume augmente
  lorsque le rayon est de 10 cm.
\item
  Une voiture roule vers le nord à 40 km/h et une autre vers l'est à 30
  km/h. À quelle vitesse la distance entre eux augmente-t-elle 2 heures
  plus tard~?
\item
  Un projecteur situé à 20 m d'un mur éclaire un homme de 2 m qui
  s'éloigne à une vitesse de 1,5 m/s. À quelle vitesse la longueur de
  son ombre sur le mur change-t-elle lorsqu'il se trouve à 5 m de la
  lumière ?
\item
  La longueur du côté d'un cube augmente de 2 cm/s. À quelle vitesse la
  surface augmente-t-elle lorsque le côté mesure 3 cm~?
\item
  Le sable est versé sur un tas formant un cône de rayon toujours égal à
  la hauteur. Si la hauteur augmente de 5 cm/s, à quelle vitesse le
  volume augmente-t-il lorsque la hauteur est de 10 cm ?
\end{enumerate}

\subsection{3.3 Problèmes
d'optimisation}\label{probluxe8mes-doptimisation}

Les problèmes d'optimisation utilisent des dérivées pour trouver les
valeurs maximales ou minimales d'une fonction, souvent sous certaines
contraintes. Ces problèmes modélisent des situations dans lesquelles
nous souhaitons maximiser l'efficacité, le profit ou la surface, ou
minimiser les coûts, la distance ou le temps.

\subsubsection{Étapes générales}\label{uxe9tapes-guxe9nuxe9rales}

\begin{enumerate}
\def\labelenumi{\arabic{enumi}.}
\tightlist
\item
  Comprendre le problème : Identifiez la quantité à optimiser.
\item
  Modèle avec une fonction : Écrivez la fonction objectif en termes
  d'une variable.
\item
  Appliquer des contraintes~: utilisez des conditions données pour
  réduire les variables.
\item
  Différencier~: calculez la dérivée de la fonction objectif.
\item
  Trouvez les points critiques~: résolvez \(f'(x) = 0\) ou lorsque
  \(f'(x)\) n'est pas défini.
\item
  Test des maxima/minima~: utilisez le test de la dérivée seconde ou
  vérifiez les paramètres.
\item
  Interprétez le résultat : Énoncez la réponse dans le contexte
  original.
\end{enumerate}

\subsubsection{Exemple 1~:~aire maximale d'un
rectangle}\label{exemple-1-aire-maximale-dun-rectangle}

Un rectangle a un périmètre de 40. Quelles dimensions maximisent son
aire ?

\begin{itemize}
\tightlist
\item
  Soit longueur \(x\), largeur \(y\). Contrainte~:
  \(2x + 2y = 40 \Rightarrow y = 20 - x\).
\item
  Zone : \(A = xy = x(20 - x) = 20x - x^2\).- Dérivé~:
  \(A'(x) = 20 - 2x\). Définir égal à 0~: \(x = 10\).
\item
  Puis \(y = 10\).
\item
  Superficie maximale~: \(100\). Le rectangle est un carré.
\end{itemize}

\subsubsection{Exemple 2~:~Minimiser la
distance}\label{exemple-2-minimiser-la-distance}

Trouvez le point sur la parabole \(y = x^2\) le plus proche de
\((0,3)\).

\begin{itemize}
\tightlist
\item
  Distance au carré : \(D(x) = (x-0)^2 + (x^2 - 3)^2\).
\item
  Développez~:
  \(D(x) = x^2 + (x^2 - 3)^2 = x^2 + x^4 - 6x^2 + 9 = x^4 - 5x^2 + 9\).
\item
  Dérivé~: \(D'(x) = 4x^3 - 10x\). Résolvez~: \(x(4x^2 - 10) = 0\).
  -Solutions~: \(x = 0\), \(x = \pm \sqrt{2.5}\).
\item
  La vérification donne la distance minimale à \(x = \pm \sqrt{2.5}\).
\end{itemize}

\subsubsection{Exemple 3 : Boîte avec volume
maximum}\label{exemple-3-bouxeete-avec-volume-maximum}

Une boîte sans dessus doit être fabriquée à partir d'un morceau de
carton carré de 20 cm de côté en découpant des carrés égaux dans les
coins et en repliant les côtés. Trouvez la taille de coupe qui maximise
le volume.

\begin{itemize}
\tightlist
\item
  Soit taille de coupe = \(x\). Puis dimensions~:
  \((20 - 2x) \times (20 - 2x) \times x\). -Volume :
  \(V(x) = x(20 - 2x)^2\).
\item
  Dérivé~: \(V'(x) = (20 - 2x)(20 - 6x)\).
\item
  Points critiques : \(x = 10\) (donne un volume nul) ou
  \(x = \tfrac{20}{6} \approx 3.33\).
\item
  À \(x \approx 3.33\), le volume est maximisé.
\end{itemize}

\subsubsection{Pourquoi l'optimisation est
importante}\label{pourquoi-loptimisation-est-importante}

\begin{itemize}
\tightlist
\item
  Les ingénieurs l'utilisent pour concevoir des structures efficaces.
\item
  Les entreprises l'utilisent pour maximiser leurs profits ou minimiser
  leurs coûts.
\item
  Les scientifiques l'utilisent pour modéliser des systèmes naturels en
  quête d'équilibre.
\end{itemize}

\subsubsection{Exercices}\label{exercices-10}

\begin{enumerate}
\def\labelenumi{\arabic{enumi}.}
\tightlist
\item
  Un agriculteur dispose de 100 m de clôture pour clôturer un champ
  rectangulaire le long d'une rivière (c'est pourquoi seulement 3 côtés
  ont besoin d'une clôture). Trouvez les dimensions maximisant la zone.
\item
  Trouvez deux nombres positifs dont la somme est 20 et dont le produit
  est le plus grand possible.
\item
  Un cylindre doit être fabriqué à partir de 100 cm\(^2\) de matériau.
  Trouvez les dimensions du volume maximum.
\item
  Un fil de 10 m de long est coupé en deux morceaux, l'un plié en carré,
  l'autre en cercle. Comment doit-il être coupé pour maximiser la
  surface totale clôturée~?
\item
  Une boîte fermée à base carrée et de volume 32 m\(^3\) est à
  construire. Trouvez des dimensions minimisant la surface.
\end{enumerate}

\subsection{3.4 Concavité et points
d'inflexion}\label{concavituxe9-et-points-dinflexion}

Les dérivées nous renseignent non seulement sur les pentes mais aussi
sur la forme d'un graphique. La dérivée seconde est particulièrement
utile pour comprendre la concavité et identifier les points d'inflexion.

\subsubsection{Concavité}\label{concavituxe9}

\begin{itemize}
\item
  Une fonction \(f(x)\) est concave vers le haut sur un intervalle si
  \(f''(x) > 0\). Le graphique se courbe vers le haut, comme une tasse.
\item
  Une fonction \(f(x)\) est concave vers le bas sur un intervalle si
  \(f''(x) < 0\). Le graphique se penche vers le bas, comme un
  froncement de sourcils.
\end{itemize}

La concavité décrit la façon dont la pente d'une fonction change : si
les pentes augmentent, le graphique est concave vers le haut ; si les
pentes diminuent, le graphique est concave vers le bas.

\subsubsection{Points d'inflexion}\label{points-dinflexion}

Un point d'inflexion est un point sur le graphique où la concavité
change.- Si \(f''(x) = 0\) ou \(f''(x)\) n'est pas défini, le point est
candidat pour un point d'inflexion. - Pour confirmer, la concavité doit
changer de signe de part et d'autre de la pointe.

\subsubsection{Exemples}\label{exemples-6}

\begin{enumerate}
\def\labelenumi{\arabic{enumi}.}
\item
  \(f(x) = x^3\)

  \begin{itemize}
  \tightlist
  \item
    \(f''(x) = 6x\).
  \item
    À \(x = 0\), \(f''(0) = 0\).
  \item
    Pour \(x < 0\), \(f''(x) < 0\) → concave vers le bas.
  \item
    Pour \(x > 0\), \(f''(x) > 0\) → concave vers le haut.
  \item
    Ainsi, \((0,0)\) est un point d'inflexion.
  \end{itemize}
\item
  \(f(x) = x^4\)

  \begin{itemize}
  \tightlist
  \item
    \(f''(x) = 12x^2\).
  \item
    A \(x = 0\), \(f''(0) = 0\), mais la concavité ne change pas de
    signe (toujours ≥ 0).
  \item
    Pas de point d'inflexion.
  \end{itemize}
\end{enumerate}

\subsubsection{Esquisse de concavité et de
courbe}\label{esquisse-de-concavituxe9-et-de-courbe}

\begin{itemize}
\tightlist
\item
  Si \(f'(x) = 0\) et \(f''(x) > 0\), alors \(f\) a un minimum local.
\item
  Si \(f'(x) = 0\) et \(f''(x) < 0\), alors \(f\) a un maximum local.
\item
  C'est ce qu'on appelle le test de la dérivée seconde.
\end{itemize}

\subsubsection{Pourquoi c'est
important}\label{pourquoi-cest-important-1}

La concavité et les points d'inflexion nous aident à comprendre la «
forme » des graphiques : où ils se plient, s'aplatissent ou tournent.
Ces idées sont centrales dans l'esquisse de courbes, la physique
(accélération) et l'économie (rendements décroissants).

\subsubsection{Exercices}\label{exercices-11}

\begin{enumerate}
\def\labelenumi{\arabic{enumi}.}
\tightlist
\item
  Déterminez les intervalles de concavité pour \(f(x) = x^3 - 3x\).
  Trouvez ses points d'inflexion.
\item
  Pour \(f(x) = \ln(x)\), identifiez la concavité et les points
  d'inflexion possibles.
\item
  Appliquez le test de dérivée seconde à \(f(x) = x^2 e^{-x}\) pour
  classer les points critiques.
\item
  Croquis \(f(x) = \sin x\), marquant les intervalles de concavité et
  les points d'inflexion.
\item
  Expliquez pourquoi \(f(x) = e^x\) n'a pas de point d'inflexion.
\end{enumerate}

\subsection{3.5 Esquisse de courbe}\label{esquisse-de-courbe}

L'esquisse de courbe est le processus de dessin du graphique d'une
fonction en utilisant les informations de ses dérivées. Plutôt que de
tracer de nombreux points, nous analysons les caractéristiques clés~:
les intersections, les asymptotes, les intervalles
croissants/décroissants et la concavité.

\subsubsection{Étapes pour l'esquisse de
courbes}\label{uxe9tapes-pour-lesquisse-de-courbes}

\begin{enumerate}
\def\labelenumi{\arabic{enumi}.}
\item
  Domaine~: identifiez où la fonction est définie.
\item
  Interceptions~: recherchez l'endroit où le graphique croise les axes.
\item
  Asymptotes~:

  \begin{itemize}
  \tightlist
  \item
    Les asymptotes verticales se produisent lorsque la fonction est
    indéfinie et tend vers l'infini.
  \item
    Les asymptotes horizontales ou inclinées décrivent le comportement
    final comme \(x \to \pm\infty\).
  \end{itemize}
\item
  Dérivé premier \(f'(x)\)~:

  \begin{itemize}
  \tightlist
  \item
    Positif → la fonction augmente.
  \item
    Négatif → la fonction diminue.
  \item
    Zéros de \(f'(x)\) → points critiques (maximas/minima possibles).
  \end{itemize}
\item
  Dérivée seconde \(f''(x)\)~:

  \begin{itemize}
  \tightlist
  \item
    Positif → concave vers le haut.
  \item
    Négatif → concave vers le bas.
  \item
    Zéros ou indéfinis → points d'inflexion possibles.
  \end{itemize}
\item
  Combinez les informations~: utilisez tous les résultats pour tracer un
  graphique clair et précis.
\end{enumerate}

\subsubsection{\texorpdfstring{Exemple 1~:
\(f(x) = x^3 - 3x\)}{Exemple 1~: f(x) = x\^{}3 - 3x}}\label{exemple-1-fx-x3---3x}

\begin{itemize}
\item
  Domaine~: tous les nombres réels.
\item
  Interceptions~: à \((0,0)\).
\item
  Dérivé~: \(f'(x) = 3x^2 - 3 = 3(x-1)(x+1)\).

  \begin{itemize}
  \item
    Augmentation~: \((-\infty, -1) \cup (1, \infty)\).
  \item
    Décroissant : \((-1, 1)\).- Dérivée seconde : \(f''(x) = 6x\).
  \item
    Concave vers le bas pour \(x < 0\), concave vers le haut pour
    \(x > 0\).
  \item
    Point d'inflexion à \((0,0)\).
  \end{itemize}
\item
  Forme~: une courbe en S avec un maximum local à \((-1, 2)\), un
  minimum local à \((1, -2)\).
\end{itemize}

\subsubsection{\texorpdfstring{Exemple 2~:
\(f(x) = \frac{1}{x}\)}{Exemple 2~: f(x) = \textbackslash frac\{1\}\{x\}}}\label{exemple-2-fx-frac1x}

\begin{itemize}
\item
  Domaine : \(x \neq 0\).
\item
  Asymptote verticale : \(x = 0\).
\item
  Asymptote horizontale : \(y = 0\).
\item
  Dérivé : \(f'(x) = -\frac{1}{x^2}\) (toujours négatif). La fonction
  est toujours en diminution.
\item
  Dérivée seconde : \(f''(x) = \frac{2}{x^3}\).

  \begin{itemize}
  \tightlist
  \item
    Concave pour \(x > 0\).
  \item
    Concave vers le bas pour \(x < 0\).
  \end{itemize}
\item
  Graphique : hyperbole à deux branches.
\end{itemize}

\subsubsection{Pourquoi l'esquisse de courbes est
utile}\label{pourquoi-lesquisse-de-courbes-est-utile}

\begin{itemize}
\tightlist
\item
  Fournit un aperçu du comportement global des fonctions sans calcul
  exhaustif.
\item
  Essentiel dans les examens de calcul et les problèmes appliqués.
\item
  Relie l'analyse algébrique et la compréhension géométrique.
\end{itemize}

\subsubsection{Exercices}\label{exercices-12}

\begin{enumerate}
\def\labelenumi{\arabic{enumi}.}
\tightlist
\item
  Esquissez la courbe de \(f(x) = x^4 - 2x^2\). Identifiez les maxima,
  les minima et les points d'inflexion.
\item
  Analysez et dessinez \(f(x) = \ln(x)\). Afficher les interceptions,
  les asymptotes et la concavité.
\item
  Pour \(f(x) = e^{-x}\), décrivez la croissance/décroissance, les
  asymptotes et la concavité.
\item
  Dessinez le graphique de \(f(x) = \tan x\) sur l'intervalle
  \((- \pi, \pi)\). Marquez les asymptotes.
\item
  Utilisez les tests de dérivée première et seconde pour classer les
  points critiques de \(f(x) = x^3 - 6x^2 + 9x\).
\end{enumerate}

\section{Partie II. Intégrales}\label{partie-ii.-intuxe9grales}

\section{Chapitre 4. Primitives et intégrales
définies}\label{chapitre-4.-primitives-et-intuxe9grales-duxe9finies}

\subsection{4.1 Intégrales
indéfinies}\label{intuxe9grales-induxe9finies}

Une intégrale indéfinie est le processus inverse de différenciation. Si
une dérivée mesure un changement, alors une intégrale récupère la
fonction d'origine à partir de son taux de changement.

\subsubsection{Définition}\label{duxe9finition-3}

Si \(F'(x) = f(x)\), alors

\[
\int f(x)\,dx = F(x) + C,
\]

où \(C\) est la constante d'intégration.

Chaque intégrale indéfinie représente une famille de fonctions qui ne
diffèrent que par une constante, puisque la différenciation élimine les
constantes.

\subsubsection{Règles de base}\label{ruxe8gles-de-base}

\begin{enumerate}
\def\labelenumi{\arabic{enumi}.}
\tightlist
\item
  Règle constante
\end{enumerate}

\[
\int c\,dx = cx + C.
\]

\begin{enumerate}
\def\labelenumi{\arabic{enumi}.}
\setcounter{enumi}{1}
\tightlist
\item
  Règle de puissance
\end{enumerate}

\[
\int x^n\,dx = \frac{x^{n+1}}{n+1} + C, \quad n \neq -1.
\]

\begin{enumerate}
\def\labelenumi{\arabic{enumi}.}
\setcounter{enumi}{2}
\tightlist
\item
  Règle de somme
\end{enumerate}

\[
\int \big(f(x) + g(x)\big)\,dx = \int f(x)\,dx + \int g(x)\,dx.
\]

\begin{enumerate}
\def\labelenumi{\arabic{enumi}.}
\setcounter{enumi}{3}
\tightlist
\item
  Règle multiple constante
\end{enumerate}

\[
\int c f(x)\,dx = c \int f(x)\,dx.
\]

\subsubsection{Intégrales communes}\label{intuxe9grales-communes}

\begin{itemize}
\tightlist
\item
  \(\int e^x dx = e^x + C\)
\item
  \(\int \sin x dx = -\cos x + C\)
\item
  \(\int \cos x dx = \sin x + C\)
\item
  \(\int \frac{1}{x} dx = \ln|x| + C\)
\end{itemize}

\subsubsection{Exemples}\label{exemples-7}

\begin{enumerate}
\def\labelenumi{\arabic{enumi}.}
\item
  \(\int (3x^2 - 4)\,dx = x^3 - 4x + C\).
\item
  \(\int \cos(2x)\,dx = \tfrac{1}{2}\sin(2x) + C\).
\item
  \(\int \frac{1}{x}\,dx = \ln|x| + C\).
\end{enumerate}

\subsubsection{Interprétation}\label{interpruxe9tation-1}

\begin{itemize}
\tightlist
\item
  Les intégrales indéfinies sont des primitives.
\item
  Ils constituent le fondement d'intégrales définies, qui mesurent des
  quantités accumulées telles que la surface, la distance et la masse.-
  Dans des contextes appliqués, l'intégration permet de passer des taux
  aux totaux.
\end{itemize}

\subsubsection{Exercices}\label{exercices-13}

\begin{enumerate}
\def\labelenumi{\arabic{enumi}.}
\tightlist
\item
  Recherchez \(\int (5x^4 + 2x)\,dx\).
\item
  Calculez \(\int (e^x + 3)\,dx\).
\item
  Trouvez la solution générale de \(f'(x) = 6x\) en utilisant
  l'intégration.
\item
  Évaluez \(\int \frac{2}{x}\,dx\).
\item
  Si la vitesse est \(v(t) = 4t\), recherchez la fonction de position
  \(s(t)\).
\end{enumerate}

\subsection{4.2 L'intégrale définie comme
aire}\label{lintuxe9grale-duxe9finie-comme-aire}

Alors que les intégrales indéfinies représentent des familles de
primitives, l'intégrale définie donne une valeur numérique : l'aire
accumulée sous une courbe entre deux points.

\subsubsection{Définition}\label{duxe9finition-4}

Pour une fonction \(f(x)\) définie sur \([a, b]\), l'intégrale définie
est

\[
\int_a^b f(x)\,dx = \lim_{n \to \infty} \sum_{i=1}^n f(x_i^-) \,\Delta x,
\]

où l'intervalle \([a, b]\) est divisé en \(n\) sous-intervalles de
largeur \(\Delta x\), et \(x_i^-\) est un point d'échantillonnage dans
chaque sous-intervalle.

C'est la limite des sommes de Riemann.

\subsubsection{Interprétation
géométrique}\label{interpruxe9tation-guxe9omuxe9trique}

\begin{itemize}
\tightlist
\item
  Si \(f(x) \geq 0\) sur \([a, b]\), alors \(\int_a^b f(x)\,dx\) est
  égal à l'aire sous la courbe \(y = f(x)\) de \(x=a\) à \(x=b\).
\item
  Si \(f(x)\) descend en dessous de l'axe \(x\), l'intégrale calcule la
  zone signée~: les régions situées en dessous de l'axe comptent comme
  négatives.
\end{itemize}

\subsubsection{Propriétés de l'intégrale
définie}\label{propriuxe9tuxe9s-de-lintuxe9grale-duxe9finie}

\begin{enumerate}
\def\labelenumi{\arabic{enumi}.}
\tightlist
\item
  Additivité sur les intervalles
\end{enumerate}

\[
\int_a^c f(x)\,dx = \int_a^b f(x)\,dx + \int_b^c f(x)\,dx.
\]

\begin{enumerate}
\def\labelenumi{\arabic{enumi}.}
\setcounter{enumi}{1}
\tightlist
\item
  Inverser les limites
\end{enumerate}

\[
\int_a^b f(x)\,dx = -\int_b^a f(x)\,dx.
\]

\begin{enumerate}
\def\labelenumi{\arabic{enumi}.}
\setcounter{enumi}{2}
\tightlist
\item
  Intervalle de largeur nulle
\end{enumerate}

\[
\int_a^a f(x)\,dx = 0.
\]

\begin{enumerate}
\def\labelenumi{\arabic{enumi}.}
\setcounter{enumi}{3}
\tightlist
\item
  Linéarité
\end{enumerate}

\[
\int_a^b \big( cf(x) + g(x)\big)\,dx = c\int_a^b f(x)\,dx + \int_a^b g(x)\,dx.
\]

\subsubsection{Exemples}\label{exemples-8}

\begin{enumerate}
\def\labelenumi{\arabic{enumi}.}
\item
  \(\int_0^2 x\,dx = \left[\tfrac{1}{2}x^2\right]_0^2 = 2.\) Il s'agit
  de l'aire d'un triangle rectangle sous la ligne \(y=x\).
\item
  \(\int_{-1}^1 x^3\,dx = 0.\) La fonction impaire \(x^3\) a des zones
  symétriques qui s'annulent.
\item
  \(\int_0^\pi \sin x\,dx = 2.\) Cela équivaut à la surface sous un arc
  de la courbe sinusoïdale.
\end{enumerate}

\subsubsection{Pourquoi c'est
important}\label{pourquoi-cest-important-2}

\begin{itemize}
\tightlist
\item
  Les intégrales définies mesurent les quantités accumulées : distance,
  masse, énergie, probabilité.
\item
  Ils relient le calcul algébrique à l'intuition géométrique.
\item
  L'étape suivante est le théorème fondamental du calcul, qui relie les
  intégrales définies aux primitives.
\end{itemize}

\subsubsection{Exercices}\label{exercices-14}

\begin{enumerate}
\def\labelenumi{\arabic{enumi}.}
\tightlist
\item
  Calculez \(\int_0^3 (2x+1)\,dx\).
\item
  Recherchez la zone entre \(y = x^2\) et l'axe \(x\) de \(x = 0\) à
  \(x = 2\).
\item
  Évaluez \(\int_{-2}^2 (x^2 - 1)\,dx\).
\item
  Montrez que \(\int_{-a}^a f(x)\,dx = 0\) si \(f(x)\) est impair.
\item
  Calculez \(\int_0^1 e^x\,dx\) en utilisant une somme de Riemann avec
  des sous-intervalles \(n=4\) et des extrémités droites.
\end{enumerate}

\subsection{4.3 Le théorème fondamental du calculLe théorème fondamental
du calcul (FTC) réunit les deux idées principales du calcul : la
différenciation et l'intégration. Il montre que la recherche de domaines
et la recherche de taux de changement sont les deux faces d'une même
médaille.}\label{le-thuxe9oruxe8me-fondamental-du-calculle-thuxe9oruxe8me-fondamental-du-calcul-ftc-ruxe9unit-les-deux-iduxe9es-principales-du-calcul-la-diffuxe9renciation-et-lintuxe9gration.-il-montre-que-la-recherche-de-domaines-et-la-recherche-de-taux-de-changement-sont-les-deux-faces-dune-muxeame-muxe9daille.}

\subsubsection{Partie 1~:~Différenciation d'une
intégrale}\label{partie-1-diffuxe9renciation-dune-intuxe9grale}

Si \(f\) est continu sur \([a, b]\), définissez

\[
F(x) = \int_a^x f(t)\,dt.
\]

Alors \(F\) est différentiable, et

\[
F'(x) = f(x).
\]

En mots~: la dérivée de la fonction d'aire accumulée est la fonction
originale elle-même.

\subsubsection{Partie 2~: Évaluation des intégrales
définies}\label{partie-2-uxe9valuation-des-intuxe9grales-duxe9finies}

Si \(f\) est continu sur \([a, b]\) et \(F\) est une primitive de \(f\),
alors

\[
\int_a^b f(x)\,dx = F(b) - F(a).
\]

Cela nous indique que nous pouvons évaluer des intégrales définies
simplement en trouvant une primitive, plutôt qu'en calculant les limites
des sommes de Riemann.

\subsubsection{Exemples}\label{exemples-9}

\begin{enumerate}
\def\labelenumi{\arabic{enumi}.}
\item
  \(\int_0^2 x^2\,dx\).

  \begin{itemize}
  \tightlist
  \item
    Primitive : \(F(x) = \tfrac{1}{3}x^3\).
  \item
    Appliquer FTC~: \(F(2) - F(0) = \tfrac{8}{3} - 0 = \tfrac{8}{3}.\)
  \end{itemize}
\item
  Si \(F(x) = \int_1^x \cos t \, dt\), alors \(F'(x) = \cos x\).
\item
  \(\int_1^4 \frac{1}{x}\,dx\).

  \begin{itemize}
  \tightlist
  \item
    Primitive : \(\ln|x|\).
  \item
    Appliquer FTC~: \(\ln 4 - \ln 1 = \ln 4.\)
  \end{itemize}
\end{enumerate}

\subsubsection{Pourquoi la FTC est
importante}\label{pourquoi-la-ftc-est-importante}

\begin{itemize}
\tightlist
\item
  Il transforme l'intégration d'un processus limite en un calcul
  pratique.
\item
  Il confirme que différenciation et intégration sont des opérations
  inverses.
\item
  C'est le théorème central qui rend le calcul utile en mathématiques,
  en sciences et en ingénierie.
\end{itemize}

\subsubsection{Exercices}\label{exercices-15}

\begin{enumerate}
\def\labelenumi{\arabic{enumi}.}
\tightlist
\item
  Évaluez \(\int_0^3 (2x+1)\,dx\) à l'aide du FTC.
\item
  Si \(F(x) = \int_0^x e^t\,dt\), recherchez \(F'(x)\).
\item
  Calculez \(\int_0^\pi \sin x \, dx\).
\item
  Montrez que si \(f'(x) = g(x)\), alors
  \(\int_a^b g(x)\,dx = f(b) - f(a)\).
\item
  Utilisez le FTC pour expliquer pourquoi la zone sous \(y = \cos x\) de
  \(0\) à \(\pi/2\) est égale à 1.
\end{enumerate}

\subsection{4.4 Propriétés des
intégrales}\label{propriuxe9tuxe9s-des-intuxe9grales}

L'intégrale définie possède plusieurs propriétés importantes qui la
rendent flexible et puissante dans les applications. Ces propriétés
découlent de la définition comme limite des sommes et du théorème
fondamental du calcul.

\subsubsection{Linéarité}\label{linuxe9arituxe9}

Pour les fonctions \(f(x)\) et \(g(x)\) et les constantes \(c, d\)~:

\[
\int_a^b \big(c f(x) + d g(x)\big)\,dx = c \int_a^b f(x)\,dx + d \int_a^b g(x)\,dx.
\]

Cela nous permet de diviser des intégrales compliquées en parties plus
simples.

\subsubsection{Additivité sur les
intervalles}\label{additivituxe9-sur-les-intervalles}

Si \(a < c < b\), alors

\[
\int_a^b f(x)\,dx = \int_a^c f(x)\,dx + \int_c^b f(x)\,dx.
\]

Nous pouvons calculer les intégrales pièce par pièce.

\subsubsection{Inversion des limites}\label{inversion-des-limites}

\[
\int_a^b f(x)\,dx = -\int_b^a f(x)\,dx.
\]

L'échange des limites change le signe de l'intégrale.

\subsubsection{Propriété de
comparaison}\label{propriuxe9tuxe9-de-comparaison}

Si \(f(x) \leq g(x)\) pour tous les \(x\) dans \([a, b]\), alors

\[
\int_a^b f(x)\,dx \leq \int_a^b g(x)\,dx.
\]Cela nous permet de comparer les zones sans calcul direct.

\subsubsection{Inégalité de valeur
absolue}\label{inuxe9galituxe9-de-valeur-absolue}

\[
\left| \int_a^b f(x)\,dx \right| \leq \int_a^b |f(x)|\,dx.
\]

Cette propriété est essentielle dans les tests d'analyse et de
convergence.

\subsubsection{Symétrie}\label{symuxe9trie}

\begin{itemize}
\item
  Si \(f(x)\) est pair (symétrique par rapport à l'axe \(y\))~:

  \[
  \int_{-a}^a f(x)\,dx = 2\int_0^a f(x)\,dx.
  \]
\item
  Si \(f(x)\) est impair (symétrique par rapport à l'origine) :

  \[
  \int_{-a}^a f(x)\,dx = 0.
  \]
\end{itemize}

\subsubsection{Exemples}\label{exemples-10}

\begin{enumerate}
\def\labelenumi{\arabic{enumi}.}
\item
  \(\int_0^2 (3x^2 + 4)\,dx = \int_0^2 3x^2\,dx + \int_0^2 4\,dx = 8 + 8 = 16.\)
\item
  Puisque \(f(x) = x^3\) est impair, \(\int_{-1}^1 x^3\,dx = 0.\)
\item
  Puisque \(f(x) = x^2\) est pair,
  \(\int_{-2}^2 x^2\,dx = 2\int_0^2 x^2\,dx = 2\cdot \tfrac{8}{3} = \tfrac{16}{3}.\)
\end{enumerate}

\subsubsection{Pourquoi ces propriétés sont
importantes}\label{pourquoi-ces-propriuxe9tuxe9s-sont-importantes}

\begin{itemize}
\tightlist
\item
  Ils simplifient les calculs.
\item
  Ils révèlent des caractéristiques géométriques et de symétrie des
  fonctions.
\item
  Ils fournissent des outils théoriques pour une analyse plus avancée.
\end{itemize}

\subsubsection{Exercices}\label{exercices-16}

\begin{enumerate}
\def\labelenumi{\arabic{enumi}.}
\tightlist
\item
  Utilisez la symétrie pour évaluer \(\int_{-5}^5 (x^4 - x^3)\,dx\).
\item
  Montrez que
  \(\int_1^4 (2x+3)\,dx = \int_1^2 (2x+3)\,dx + \int_2^4 (2x+3)\,dx\).
\item
  Évaluez \(\int_0^\pi \sin(x)\,dx\) et comparez-le avec
  \(\int_{-\pi}^\pi \sin(x)\,dx\).
\item
  Prouvez que si \(f(x) \geq 0\) sur \([a, b]\), alors
  \(\int_a^b f(x)\,dx \geq 0\).
\item
  Calculez \(\int_{-3}^3 (x^2 + 1)\,dx\) en utilisant les propriétés
  paires/impaires.
\end{enumerate}

\section{Chapitre 5. Techniques
d'intégration}\label{chapitre-5.-techniques-dintuxe9gration}

\subsection{5.1 Remplacement}\label{remplacement}

L'une des techniques d'intégration les plus utiles est la méthode de
substitution, également appelée -u-substitution-. C'est le processus
inverse de la règle de la chaîne pour les produits dérivés.

\subsubsection{L'idée}\label{liduxe9e}

Si une intégrale contient une fonction composite, nous pouvons la
simplifier en changeant les variables.

Formellement, si \(u = g(x)\) est une fonction différentiable, alors

\[
\int f(g(x)) g'(x)\,dx = \int f(u)\,du.
\]

Cette substitution rend l'intégrale plus facile à évaluer.

\subsubsection{Étapes de substitution}\label{uxe9tapes-de-substitution}

\begin{enumerate}
\def\labelenumi{\arabic{enumi}.}
\tightlist
\item
  Identifiez une fonction interne \(u = g(x)\) dont la dérivée apparaît
  également dans l'intégrande.
\item
  Calculez \(du = g'(x)\,dx\).
\item
  Réécrivez l'intégrale en termes de \(u\).
\item
  Intégrez par rapport à \(u\).
\item
  Remplacez \(u = g(x)\).
\end{enumerate}

\subsubsection{Exemples}\label{exemples-11}

\begin{enumerate}
\def\labelenumi{\arabic{enumi}.}
\item
  Remplacement simple

  \[
  \int 2x \cos(x^2)\,dx
  \]

  Laissez \(u = x^2\), donc \(du = 2x\,dx\). L'intégrale devient alors
  \(\int \cos u \,du = \sin u + C = \sin(x^2) + C\).
\item
  Cas logarithmique

  \[
  \int \frac{2x}{x^2+1}\,dx
  \]

  Laissez \(u = x^2 + 1\), donc \(du = 2x\,dx\). L'intégrale devient
  alors \(\int \frac{1}{u}\,du = \ln|u| + C = \ln(x^2+1) + C\).
\item
  Substitution trigonométrique

  \[
  \int \sin(3x)\,dx
  \]

  Soit \(u = 3x\), donc \(du = 3\,dx\), donc
  \(dx = \frac{du}{3}\).L'intégrale devient
  \(\tfrac{1}{3}\int \sin u\,du = -\tfrac{1}{3}\cos u + C = -\tfrac{1}{3}\cos(3x) + C\).
\end{enumerate}

\subsubsection{Intégrales définies avec
substitution}\label{intuxe9grales-duxe9finies-avec-substitution}

Lors de l'évaluation d'intégrales définies, nous devons également
modifier les limites~:

\[
\int_a^b f(g(x)) g'(x)\,dx = \int_{g(a)}^{g(b)} f(u)\,du.
\]

Exemple~:

\[
\int_0^1 2x e^{x^2}\,dx.
\]

Laissez \(u = x^2\), \(du = 2x\,dx\). Limites~: lorsque \(x=0, u=0\)~;
quand \(x=1, u=1\). L'intégrale devient donc

\[
\int_0^1 e^u\,du = e - 1.
\]

\subsubsection{Exercices}\label{exercices-17}

\begin{enumerate}
\def\labelenumi{\arabic{enumi}.}
\tightlist
\item
  Évaluez \(\int (x^2+1)^5 (2x)\,dx\).
\item
  Calculez \(\int \frac{\cos x}{\sin x}\,dx\).
\item
  Évaluez \(\int_0^\pi \sin(2x)\,dx\) en utilisant la substitution.
\item
  Recherchez \(\int e^{3x}\,dx\).
\item
  Calculez \(\int \frac{1}{\sqrt{1+x^2}}\,dx\) en laissant
  \(u = 1+x^2\).
\end{enumerate}

\subsection{5.2 Intégration par
pièces}\label{intuxe9gration-par-piuxe8ces}

L'intégration par parties est une technique issue de la règle du produit
pour les dérivés. Il permet d'évaluer des intégrales impliquant des
produits de fonctions qui ne sont pas facilement gérés par la seule
substitution.

\subsubsection{La formule}\label{la-formule}

De la règle du produit~:

\[
\frac{d}{dx}[u(x)v(x)] = u'(x)v(x) + u(x)v'(x).
\]

L'intégration des deux côtés donne la formule d'intégration par
parties~:

\[
\int u\,dv = uv - \int v\,du.
\]

Ici :

\begin{itemize}
\tightlist
\item
  \(u\) = une fonction choisie pour être différenciée,
\item
  \(dv\) = la partie restante de l'intégrande à intégrer.
\end{itemize}

\subsubsection{\texorpdfstring{Choisir \(u\) et
\(dv\)}{Choisir u et dv}}\label{choisir-u-et-dv}

Une ligne directrice courante est LIATE (Logarithmique, Trigonométrique
Inverse, Algébrique, Trigonométrique, Exponentiel).

\begin{itemize}
\tightlist
\item
  Choisissez \(u\) dans la première catégorie présente.
\item
  Choisissez \(dv\) comme le reste.
\end{itemize}

\subsubsection{Exemples}\label{exemples-12}

\begin{enumerate}
\def\labelenumi{\arabic{enumi}.}
\tightlist
\item
  Polynôme × Exponentiel
\end{enumerate}

\[
\int x e^x\,dx
\]

Laissez \(u = x\), \(dv = e^x dx\). Puis \(du = dx\), \(v = e^x\).

\[
\int x e^x\,dx = x e^x - \int e^x dx = x e^x - e^x + C.
\]

\begin{enumerate}
\def\labelenumi{\arabic{enumi}.}
\setcounter{enumi}{1}
\tightlist
\item
  Polynôme × Trig
\end{enumerate}

\[
\int x \cos x\,dx
\]

Laissez \(u = x\), \(dv = \cos x dx\). Puis \(du = dx\), \(v = \sin x\).

\[
\int x \cos x\,dx = x \sin x - \int \sin x dx = x \sin x + \cos x + C.
\]

\begin{enumerate}
\def\labelenumi{\arabic{enumi}.}
\setcounter{enumi}{2}
\tightlist
\item
  Logarithme
\end{enumerate}

\[
\int \ln x\,dx
\]

Laissez \(u = \ln x\), \(dv = dx\). Puis \(du = \frac{1}{x}dx\),
\(v = x\).

\[
\int \ln x\,dx = x \ln x - \int 1 dx = x \ln x - x + C.
\]

\subsubsection{Exemple d'intégrale
définie}\label{exemple-dintuxe9grale-duxe9finie}

\[
\int_0^1 x e^x\,dx
\]

En utilisant le résultat précédent~: \(\int x e^x dx = (x-1)e^x\).
Évaluer~:

\[
\big[(x-1)e^x\big]_0^1 = (0)e^1 - (-1)e^0 = 0 + 1 = 1.
\]

\subsubsection{Pourquoi c'est
important}\label{pourquoi-cest-important-3}

L'intégration par parties est cruciale lorsque la substitution échoue,
notamment avec les logarithmes, les fonctions trigonométriques inverses
et les produits impliquant des polynômes avec des exponentielles ou des
fonctions trigonométriques.

\subsubsection{Exercices}\label{exercices-18}

\begin{enumerate}
\def\labelenumi{\arabic{enumi}.}
\tightlist
\item
  Évaluez \(\int x \sin x\,dx\).
\item
  Recherchez \(\int e^x \cos x\,dx\).
\item
  Calculez \(\int_1^2 \ln x\,dx\).
\item
  Évaluez \(\int x^2 e^x\,dx\).5. Utilisez l'intégration par parties
  pour afficher
  \(\int \arctan x\,dx = x\arctan x - \tfrac{1}{2}\ln(1+x^2) + C\).
\end{enumerate}

\subsection{5.3 Intégrales trigonométriques et
substitutions}\label{intuxe9grales-trigonomuxe9triques-et-substitutions}

De nombreuses intégrales impliquent des fonctions trigonométriques.
Ceux-ci peuvent souvent être simplifiés à l'aide d'identités ou en
effectuant des substitutions spéciales.

\subsubsection{Intégrales
trigonométriques}\label{intuxe9grales-trigonomuxe9triques}

\begin{enumerate}
\def\labelenumi{\arabic{enumi}.}
\tightlist
\item
  Pouvoirs du sinus et du cosinus
\end{enumerate}

\begin{itemize}
\tightlist
\item
  Si la puissance du sinus est impaire~: sauvegardez un \(\sin x\),
  convertissez le reste avec \(\sin^2x = 1 - \cos^2x\) et remplacez
  \(u = \cos x\).
\item
  Si la puissance du cosinus est impaire~: sauvegardez un \(\cos x\),
  convertissez le reste avec \(\cos^2x = 1 - \sin^2x\) et remplacez
  \(u = \sin x\).
\item
  Si les deux sont pairs : utiliser des identités demi-angle.
\end{itemize}

Exemple~:

\[
\int \sin^3x \cos x \, dx
\]

Soit \(u = \sin x\), \(du = \cos x\,dx\)~:

\[
\int u^3\,du = \tfrac{u^4}{4} + C = \tfrac{\sin^4x}{4} + C.
\]

\begin{enumerate}
\def\labelenumi{\arabic{enumi}.}
\setcounter{enumi}{1}
\tightlist
\item
  Produits du sinus et du cosinus avec des angles différents Utilisez
  des formules produit-somme~:
\end{enumerate}

\[
\sin A \cos B = \tfrac{1}{2}[\sin(A+B) + \sin(A-B)].
\]

Exemple~:

\[
\int \sin(2x)\cos(3x)\,dx = \tfrac{1}{2}\int [\sin(5x) - \sin(x)]\,dx.
\]

\begin{enumerate}
\def\labelenumi{\arabic{enumi}.}
\setcounter{enumi}{2}
\tightlist
\item
  Pouvoirs de sécante et de tangente
\end{enumerate}

\begin{itemize}
\tightlist
\item
  Si le pouvoir de sécante est pair : sauvegardez \(\sec^2x\),
  convertissez le reste avec \(\sec^2x = 1 + \tan^2x\), et remplacez
  \(u = \tan x\).
\item
  Si la puissance de la tangente est impaire~: sauvegardez \(\sec^2x\),
  convertissez le reste avec \(\tan^2x = \sec^2x - 1\) et remplacez
  \(u = \tan x\).
\end{itemize}

Exemple~:

\[
\int \tan^3x \sec^2x \, dx
\]

Soit \(u = \tan x\), \(du = \sec^2x\,dx\)~:

\[
\int u^3\,du = \tfrac{u^4}{4} + C = \tfrac{\tan^4x}{4} + C.
\]

\subsubsection{Substitutions
trigonométriques}\label{substitutions-trigonomuxe9triques}

Pour les intégrales impliquant \(\sqrt{a^2 - x^2}\),
\(\sqrt{a^2 + x^2}\) ou \(\sqrt{x^2 - a^2}\), utilisez des substitutions
spéciales~:

\begin{enumerate}
\def\labelenumi{\arabic{enumi}.}
\tightlist
\item
  \(x = a \sin \theta\), pour \(\sqrt{a^2 - x^2}\).
\item
  \(x = a \tan \theta\), pour \(\sqrt{a^2 + x^2}\).
\item
  \(x = a \sec \theta\), pour \(\sqrt{x^2 - a^2}\).
\end{enumerate}

Exemple~:

\[
\int \sqrt{a^2 - x^2}\,dx
\]

Soit \(x = a\sin\theta\), donc \(dx = a\cos\theta\,d\theta\)~:

\[
\int \sqrt{a^2 - a^2\sin^2\theta}(a\cos\theta\,d\theta) = \int a^2 \cos^2\theta \, d\theta.
\]

Simplifiez en utilisant des identités demi-angle.

\subsubsection{Pourquoi ces techniques sont
importantes}\label{pourquoi-ces-techniques-sont-importantes}

\begin{itemize}
\tightlist
\item
  Ils convertissent des formes algébriques difficiles en formes
  trigonométriques gérables.
\item
  Ils sont particulièrement utiles dans les problèmes impliquant des
  surfaces, des volumes et des longueurs d'arc.
\item
  Ils jettent les bases de méthodes d'intégration avancées.
\end{itemize}

\subsubsection{Exercices}\label{exercices-19}

\begin{enumerate}
\def\labelenumi{\arabic{enumi}.}
\tightlist
\item
  Évaluez \(\int \sin^4x \cos^2x \, dx\).
\item
  Calculez \(\int \sin(5x)\cos(2x)\,dx\).
\item
  Évaluez \(\int \tan^2x \sec^2x \, dx\).
\item
  Recherchez \(\int \sqrt{9 - x^2}\,dx\) en utilisant la substitution.
\item
  Montrez que
  \(\int \frac{dx}{\sqrt{x^2 + a^2}} = \ln|x + \sqrt{x^2 + a^2}| + C\)
  en utilisant \(x = a\tan\theta\).
\end{enumerate}

\subsection{5.4 Fractions partiellesLors de l'intégration de fonctions
rationnelles (rapports de polynômes), une méthode puissante est la
décomposition en fractions partielles. Cette technique exprime une
fraction compliquée comme une somme de fractions plus simples et plus
faciles à
intégrer.}\label{fractions-partielleslors-de-lintuxe9gration-de-fonctions-rationnelles-rapports-de-polynuxf4mes-une-muxe9thode-puissante-est-la-duxe9composition-en-fractions-partielles.-cette-technique-exprime-une-fraction-compliquuxe9e-comme-une-somme-de-fractions-plus-simples-et-plus-faciles-uxe0-intuxe9grer.}

\subsubsection{L'idée}\label{liduxe9e-1}

Si \(R(x) = \frac{P(x)}{Q(x)}\) est une fonction rationnelle, où le
degré de \(P(x)\) est inférieur au degré de \(Q(x)\), nous pouvons
décomposer \(R(x)\) en fractions plus simples.

Ces pièces plus simples correspondent aux facteurs du dénominateur
\(Q(x)\).

\subsubsection{Formulaires courants}\label{formulaires-courants}

\begin{enumerate}
\def\labelenumi{\arabic{enumi}.}
\tightlist
\item
  Facteurs linéaires distincts Si
\end{enumerate}

\[
\frac{1}{(x-a)(x-b)},
\]

puis se décomposer comme

\[
\frac{A}{x-a} + \frac{B}{x-b}.
\]

\begin{enumerate}
\def\labelenumi{\arabic{enumi}.}
\setcounter{enumi}{1}
\tightlist
\item
  Facteurs linéaires répétés Si le dénominateur a \((x-a)^n\), alors les
  termes sont
\end{enumerate}

\[
\frac{A_1}{x-a} + \frac{A_2}{(x-a)^2} + \dots + \frac{A_n}{(x-a)^n}.
\]

\begin{enumerate}
\def\labelenumi{\arabic{enumi}.}
\setcounter{enumi}{2}
\tightlist
\item
  Facteurs quadratiques irréductibles Si le dénominateur a
  \((x^2+bx+c)\), alors le numérateur est linéaire~:
\end{enumerate}

\[
\frac{Ax+B}{x^2+bx+c}.
\]

\subsubsection{Exemple 1~: Facteurs linéaires
distincts}\label{exemple-1-facteurs-linuxe9aires-distincts}

\[
\int \frac{1}{x^2 - 1}\,dx
\]

Dénominateur du facteur~: \((x-1)(x+1)\). Décomposer~:

\[
\frac{1}{x^2-1} = \frac{1}{2}\left(\frac{1}{x-1} - \frac{1}{x+1}\right).
\]

Intégrer~:

\[
\int \frac{1}{x^2 - 1}\,dx = \tfrac{1}{2}\ln\left|\frac{x-1}{x+1}\right| + C.
\]

\subsubsection{Exemple 2~: Facteur linéaire
répété}\label{exemple-2-facteur-linuxe9aire-ruxe9puxe9tuxe9}

\[
\int \frac{1}{(x-1)^2}\,dx
\]

C'est déjà simple :

\[
\int (x-1)^{-2}\,dx = -\frac{1}{x-1} + C.
\]

\subsubsection{Exemple 3~: Facteur quadratique
irréductible}\label{exemple-3-facteur-quadratique-irruxe9ductible}

\[
\int \frac{x}{x^2+1}\,dx
\]

Remplacez \(u = x^2+1\) ou reconnaissez que le numérateur est dérivé du
dénominateur.

\[
\int \frac{x}{x^2+1}\,dx = \tfrac{1}{2}\ln(x^2+1) + C.
\]

\#\#\#~Étapes de la décomposition d'une fraction partielle

\begin{enumerate}
\def\labelenumi{\arabic{enumi}.}
\tightlist
\item
  Factorisez le dénominateur.
\item
  Écrivez la forme générale d'une fraction partielle.
\item
  Multipliez par le dénominateur pour effacer les fractions.
\item
  Résolvez les constantes inconnues.
\item
  Intégrez chaque terme.
\end{enumerate}

\subsubsection{Pourquoi c'est
important}\label{pourquoi-cest-important-4}

\begin{itemize}
\tightlist
\item
  Convertit des fonctions rationnelles complexes en formes
  logarithmiques ou arctangentes simples.
\item
  Particulièrement utile dans les équations différentielles et les
  transformées de Laplace.
\item
  Fondamental en calcul avancé et en ingénierie.
\end{itemize}

\subsubsection{Exercices}\label{exercices-20}

\begin{enumerate}
\def\labelenumi{\arabic{enumi}.}
\tightlist
\item
  Décomposez et intégrez \(\int \frac{3x+5}{x^2-1}\,dx\).
\item
  Évaluez \(\int \frac{1}{x^2(x+1)}\,dx\).
\item
  Calculez \(\int \frac{2x+1}{x^2+2x+2}\,dx\).
\item
  Recherchez \(\int \frac{1}{x^3 - x}\,dx\).
\item
  Montrez que \(\int \frac{dx}{x^2+1} = \arctan x + C\) en utilisant des
  fractions partielles ou une substitution.
\end{enumerate}

\subsection{5.5 Intégrales incorrectes}\label{intuxe9grales-incorrectes}

Certaines intégrales ne peuvent pas être évaluées directement car
l'intervalle est infini ou l'intégrande devient illimitée. C'est ce
qu'on appelle des intégrales impropres. Ils sont définis à l'aide de
limites.

\subsubsection{Définition}\label{duxe9finition-5}

\begin{enumerate}
\def\labelenumi{\arabic{enumi}.}
\tightlist
\item
  Intervalle infini
\end{enumerate}

\[\int_a^\infty f(x)\,dx = \lim_{b \to \infty} \int_a^b f(x)\,dx.
\]

\[
\int_{-\infty}^a f(x)\,dx = \lim_{b \to -\infty} \int_b^a f(x)\,dx.
\]

\begin{enumerate}
\def\labelenumi{\arabic{enumi}.}
\setcounter{enumi}{1}
\tightlist
\item
  Unbounded integrand If \(f(x)\) has a vertical asymptote at \(c\),
  then
\end{enumerate}

\[
\int_a^c f(x)\,dx = \lim_{t \to c^-} \int_a^t f(x)\,dx,
\]

\[
\int_c^b f(x)\,dx = \lim_{t \to c^+} \int_t^b f(x)\,dx.
\]

\subsubsection{Convergence and
Divergence}\label{convergence-and-divergence}

\begin{itemize}
\tightlist
\item
  If the limit exists and is finite, the improper integral converges.
\item
  If the limit does not exist or is infinite, the improper integral
  diverges.
\end{itemize}

\subsubsection{Examples}\label{examples-1}

\begin{enumerate}
\def\labelenumi{\arabic{enumi}.}
\tightlist
\item
  Exponential decay
\end{enumerate}

\[
\int_1^\infty \frac{1}{x^2}\,dx = \lim_{b \to \infty} \Big[-\tfrac{1}{x}\Big]_1^b = 1.
\]

This converges.

\begin{enumerate}
\def\labelenumi{\arabic{enumi}.}
\setcounter{enumi}{1}
\tightlist
\item
  Harmonic function
\end{enumerate}

\[
\int_1^\infty \frac{1}{x}\,dx = \lim_{b \to \infty} \ln b.
\]

This diverges to infinity.

\begin{enumerate}
\def\labelenumi{\arabic{enumi}.}
\setcounter{enumi}{2}
\tightlist
\item
  Asymptote at 0
\end{enumerate}

\[
\int_0^1 \frac{1}{\sqrt{x}}\,dx = \lim_{t \to 0^+} \int_t^1 x^{-1/2}\,dx.
\]

\[
= \lim_{t \to 0^+} [2\sqrt{x}]_t^1 = 2.
\]

This converges.

\begin{enumerate}
\def\labelenumi{\arabic{enumi}.}
\setcounter{enumi}{3}
\tightlist
\item
  Asymptote at 0 (divergent)
\end{enumerate}

\[
\int_0^1 \frac{1}{x}\,dx = \lim_{t \to 0^+} \ln(1) - \ln(t).
\]

This diverges since \(\ln(t) \to -\infty\).

\subsubsection{Comparison Test for Improper
Integrals}\label{comparison-test-for-improper-integrals}

\begin{itemize}
\tightlist
\item
  If \(0 \leq f(x) \leq g(x)\) for large \(x\), and \(\int g(x)\,dx\)
  converges, then \(\int f(x)\,dx\) also converges.
\item
  If \(\int f(x)\,dx\) diverges and \(f(x) \geq g(x) \geq 0\), then
  \(\int g(x)\,dx\) also diverges.
\end{itemize}

\subsubsection{Why Improper Integrals
Matter}\label{why-improper-integrals-matter}

\begin{itemize}
\tightlist
\item
  They extend integration to infinite domains and unbounded functions.
\item
  They are essential in probability (continuous distributions), physics
  (gravitational/electric fields), and Fourier analysis.
\end{itemize}

\subsubsection{Exercises}\label{exercises-2}

\begin{enumerate}
\def\labelenumi{\arabic{enumi}.}
\tightlist
\item
  Determine whether \(\int_1^\infty \frac{1}{x^p}\,dx\) converges for
  various values of \(p\).
\item
  Evaluate \(\int_0^\infty e^{-x}\,dx\).
\item
  Test convergence of \(\int_0^1 \frac{1}{x^p}\,dx\) depending on \(p\).
\item
  Compute \(\int_{-\infty}^\infty \frac{1}{1+x^2}\,dx\).
\item
  Use the comparison test to show that
  \(\int_1^\infty \frac{1}{x^2+1}\,dx\) converges.
\end{enumerate}

\section{Chapter 6. Applications of
Integration}\label{chapter-6.-applications-of-integration}

\subsection{6.1 Areas and Volumes}\label{areas-and-volumes}

One of the most important applications of integration is finding areas
under curves and volumes of solids.

\subsubsection{Area Between Curves}\label{area-between-curves}

If \(f(x) \geq g(x)\) on \([a, b]\), then the area between the curves
\(y=f(x)\) and \(y=g(x)\) is

\[
A = \int_a^b \big(f(x) - g(x)\big)\,dx.
\]

Example: Find the area between \(y=x^2\) and \(y=x\) on \([0,1]\).

\[
A = \int_0^1 (x - x^2)\,dx = \left[\tfrac{1}{2}x^2 - \tfrac{1}{3}x^3\right]_0^1 = \tfrac{1}{6}.
\]

\subsubsection{Volumes by Slicing}\label{volumes-by-slicing}

If a solid has cross-sectional area \(A(x)\) at position \(x\), then the
volume is

\[
V = \int_a^b A(x)\,dx.
\]\#\#\# Volumes de révolution

Lorsqu'une région tourne autour d'un axe, le volume du solide résultant
peut être trouvé par intégration.

\begin{enumerate}
\def\labelenumi{\arabic{enumi}.}
\tightlist
\item
  Méthode du disque Si la région sous \(y=f(x)\), \(x\in[a,b]\), tourne
  autour de l'axe \(x\)~:
\end{enumerate}

\[
V = \pi \int_a^b [f(x)]^2\,dx.
\]

\begin{enumerate}
\def\labelenumi{\arabic{enumi}.}
\setcounter{enumi}{1}
\tightlist
\item
  Méthode de lavage Si la région entre \(y=f(x)\) et \(y=g(x)\) tourne
  autour de l'axe \(x\)~:
\end{enumerate}

\[
V = \pi \int_a^b \Big([f(x)]^2 - [g(x)]^2\Big)\,dx.
\]

\begin{enumerate}
\def\labelenumi{\arabic{enumi}.}
\setcounter{enumi}{2}
\tightlist
\item
  Méthode Shell Si la région sous \(y=f(x)\) tourne autour de l'axe
  \(y\)~:
\end{enumerate}

\[
V = 2\pi \int_a^b x f(x)\,dx.
\]

\subsubsection{Exemples}\label{exemples-13}

\begin{enumerate}
\def\labelenumi{\arabic{enumi}.}
\tightlist
\item
  Méthode du disque Faites pivoter \(y=\sqrt{x}\), \(0 \leq x \leq 4\),
  autour de l'axe \(x\)~:
\end{enumerate}

\[
V = \pi \int_0^4 (\sqrt{x})^2\,dx = \pi \int_0^4 x\,dx = \pi \left[\tfrac{1}{2}x^2\right]_0^4 = 8\pi.
\]

\begin{enumerate}
\def\labelenumi{\arabic{enumi}.}
\setcounter{enumi}{1}
\tightlist
\item
  Méthode de lavage Faites pivoter la région entre \(y=\sqrt{x}\) et
  \(y=1\), \(0 \leq x \leq 1\), autour de l'axe \(x\)~:
\end{enumerate}

\[
V = \pi \int_0^1 \big((\sqrt{x})^2 - (1)^2\big)\,dx = \pi \int_0^1 (x-1)\,dx = -\tfrac{\pi}{2}.
\]

(Prendre la valeur absolue du volume~: \(V = \tfrac{\pi}{2}\)).

\begin{enumerate}
\def\labelenumi{\arabic{enumi}.}
\setcounter{enumi}{2}
\tightlist
\item
  Méthode Shell Faites pivoter la région sous \(y=x\),
  \(0 \leq x \leq 1\), autour de l'axe \(y\)~:
\end{enumerate}

\[
V = 2\pi \int_0^1 x(x)\,dx = 2\pi \int_0^1 x^2\,dx = 2\pi \cdot \tfrac{1}{3} = \tfrac{2\pi}{3}.
\]

\subsubsection{Pourquoi c'est
important}\label{pourquoi-cest-important-5}

\begin{itemize}
\tightlist
\item
  Fournit des moyens exacts de calculer des zones et des volumes en
  géométrie.
\item
  Essentiel en physique, ingénierie et probabilités.
\item
  Introduit la pensée géométrique avec intégration.
\end{itemize}

\subsubsection{Exercices}\label{exercices-21}

\begin{enumerate}
\def\labelenumi{\arabic{enumi}.}
\tightlist
\item
  Recherchez la zone entre \(y=\cos x\) et \(y=\sin x\) sur
  \([0, \pi/2]\).
\item
  Calculez le volume du solide formé en faisant tourner \(y=x^2\),
  \(0 \leq x \leq 1\), autour de l'axe \(x\).
\item
  Trouvez le volume du solide formé en faisant tourner la région entre
  \(y=x\) et \(y=\sqrt{x}\) sur \([0,1]\) autour de l'axe \(y\).
\item
  Utilisez la méthode de la rondelle pour calculer le volume du solide
  formé en faisant tourner \(y=\sqrt{1-x^2}\) (un demi-cercle) autour de
  l'axe \(x\).
\item
  Recherchez la zone délimitée entre \(y=x^2+1\) et \(y=3x\).
\end{enumerate}

\subsection{6.2 Longueur de l'arc et
superficie}\label{longueur-de-larc-et-superficie}

L'intégration peut également être utilisée pour mesurer la longueur des
courbes et la surface des solides générés par les courbes tournantes.

\subsubsection{Longueur de l'arc}\label{longueur-de-larc}

Pour une courbe lisse \(y=f(x)\) sur l'intervalle \([a,b]\), la longueur
de la courbe est

\[
L = \int_a^b \sqrt{1 + \big(f'(x)\big)^2}\,dx.
\]

Cela vient du rapprochement de la courbe avec des segments de ligne et
de la prise de la limite.

Exemple~: Trouvez la longueur de \(y=\tfrac{1}{2}x^{3/2}\) de \(x=0\) à
\(x=4\).

\begin{itemize}
\tightlist
\item
  Dérivé~: \(f'(x) = \tfrac{3}{4}\sqrt{x}\).
\item
  Formule :
\end{itemize}

\[
L = \int_0^4 \sqrt{1 + \Big(\tfrac{3}{4}\sqrt{x}\Big)^2}\,dx
= \int_0^4 \sqrt{1 + \tfrac{9}{16}x}\,dx.
\]

Cette intégrale peut être évaluée par substitution.\#\#\# Superficie de
révolution

Si une courbe \(y=f(x)\), \(a \leq x \leq b\), tourne autour de l'axe
\(x\), la surface du solide résultant est

\[
S = 2\pi \int_a^b f(x)\sqrt{1 + \big(f'(x)\big)^2}\,dx.
\]

S'il tourne autour de l'axe \(y\)~:

\[
S = 2\pi \int_a^b x \sqrt{1 + \big(f'(x)\big)^2}\,dx.
\]

\subsubsection{Exemples}\label{exemples-14}

\begin{enumerate}
\def\labelenumi{\arabic{enumi}.}
\tightlist
\item
  Longueur de l'arc d'une ligne Pour \(y=x\), \(0 \leq x \leq 3\)~:
\end{enumerate}

\[
L = \int_0^3 \sqrt{1+(1)^2}\,dx = \int_0^3 \sqrt{2}\,dx = 3\sqrt{2}.
\]

\begin{enumerate}
\def\labelenumi{\arabic{enumi}.}
\setcounter{enumi}{1}
\tightlist
\item
  Surface d'une sphère Prenez \(y = \sqrt{r^2 - x^2}\),
  \(-r \leq x \leq r\) et tournez autour de l'axe \(x\).
\end{enumerate}

\[
S = 2\pi \int_{-r}^r \sqrt{r^2 - x^2}\sqrt{1+\left(\frac{-x}{\sqrt{r^2-x^2}}\right)^2}\,dx.
\]

La simplification donne \(S = 4\pi r^2\), la formule familière pour la
surface d'une sphère.

\subsubsection{Pourquoi c'est
important}\label{pourquoi-cest-important-6}

\begin{itemize}
\tightlist
\item
  La longueur de l'arc étend l'idée de distance aux chemins courbes.
\item
  La surface de révolution a des applications en physique, en ingénierie
  et en conception.
\item
  Fournit un pont entre le calcul et la géométrie.
\end{itemize}

\subsubsection{Exercices}\label{exercices-22}

\begin{enumerate}
\def\labelenumi{\arabic{enumi}.}
\tightlist
\item
  Trouvez la longueur de l'arc de \(y=\sqrt{x}\) de \(x=0\) à \(x=4\).
\item
  Calculez la surface du solide obtenu en faisant tourner \(y=x^2\),
  \(0 \leq x \leq 1\), autour de l'axe \(x\).
\item
  Trouvez la longueur de l'arc de \(y=\ln(\cosh x)\) de \(x=0\) à
  \(x=1\).
\item
  Montrez que la rotation de \(y=\sqrt{r^2 - x^2}\) de \(0\) à \(r\)
  autour de l'axe \(x\) donne la moitié de la surface d'une sphère.
\item
  Dérivez la formule de l'aire de la surface d'un cône en faisant
  tourner une ligne.
\end{enumerate}

\subsection{6.3 Travail et moyennes}\label{travail-et-moyennes}

L'intégration ne se limite pas à la géométrie. Il permet également de
calculer le travail effectué par une force et la valeur moyenne d'une
fonction sur un intervalle.

\subsubsection{Travail}\label{travail}

Si une force variable \(F(x)\) déplace un objet le long d'une ligne
droite de \(x=a\) à \(x=b\), alors le travail total est

\[
W = \int_a^b F(x)\,dx.
\]

Cette formule généralise le cas simple \(W = F \cdot d\) pour une force
constante.

Exemple 1~: Force de ressort (loi de Hooke) Pour un ressort étiré de la
longueur \(a\) à \(b\), avec une force \(F(x) = kx\)~:

\[
W = \int_a^b kx\,dx = \tfrac{1}{2}k(b^2-a^2).
\]

Exemple 2~:~Pompage de l'eau Si l'eau est pompée hors d'un réservoir, le
travail requis est égal à

\[
W = \int_a^b \text{(weight density)} \times \text{(cross-sectional area)} \times \text{(distance lifted)} \, dx.
\]

\subsubsection{Valeur moyenne d'une
fonction}\label{valeur-moyenne-dune-fonction}

La valeur moyenne d'une fonction continue \(f(x)\) sur \([a,b]\) est

\[
f_{\text{avg}} = \frac{1}{b-a} \int_a^b f(x)\,dx.
\]

C'est l'analogue continu de la moyenne d'une liste de nombres.

Exemple 1~: Pour \(f(x)=x^2\) sur \([0,2]\)~:

\[
f_{\text{avg}} = \tfrac{1}{2-0}\int_0^2 x^2 dx = \tfrac{1}{2}\cdot \tfrac{8}{3} = \tfrac{4}{3}.
\]

Exemple 2~:Si la vitesse d'une particule est \(v(t)\), alors la vitesse
moyenne sur \([a,b]\) est

\[
v_{\text{avg}} = \frac{1}{b-a}\int_a^b v(t)\,dt.
\]

\subsubsection{Pourquoi c'est
important}\label{pourquoi-cest-important-7}

\begin{itemize}
\tightlist
\item
  Les intégrales de travail apparaissent dans les calculs de physique,
  d'ingénierie et d'énergie.
\item
  La valeur moyenne donne un nombre représentatif unique pour des
  quantités variables.
\item
  Les deux relient le calcul à des problèmes réels de mouvement, de
  force et d'efficacité.
\end{itemize}

\subsubsection{Exercices}\label{exercices-23}

\begin{enumerate}
\def\labelenumi{\arabic{enumi}.}
\tightlist
\item
  Calculez le travail nécessaire pour étirer une source de 2 m à 5 m si
  \(k=10\).
\item
  Un objet de 100 kg est soulevé verticalement sur 5 m dans un champ
  gravitationnel (\(g=9.8 \,\text{m/s}^2\)). Exprimez le travail comme
  une intégrale et évaluez-le.
\item
  Trouvez la valeur moyenne de \(f(x)=\sin x\) sur \([0,\pi]\).
\item
  Calculez la température moyenne si
  \(T(t)=20+5\cos(\tfrac{\pi t}{12})\) sur une journée de 24 heures.
\item
  Un réservoir d'une profondeur de 10 m est rempli d'eau. Calculez le
  travail requis pour pomper toute l'eau vers le haut, étant donné que
  l'eau pèse \(9800 \,\text{N/m}^3\).
\end{enumerate}

\subsection{6.4 Densités de probabilité et distributions
continues}\label{densituxe9s-de-probabilituxe9-et-distributions-continues}

L'intégration joue également un rôle central dans la théorie des
probabilités, notamment pour les variables aléatoires continues. Au lieu
de résultats discrets, nous décrivons les probabilités avec des
fonctions appelées fonctions de densité de probabilité (pdfs).

\subsubsection{Fonctions de densité de
probabilité}\label{fonctions-de-densituxe9-de-probabilituxe9}

Une fonction de densité de probabilité \(f(x)\) doit satisfaire deux
conditions~:

\begin{enumerate}
\def\labelenumi{\arabic{enumi}.}
\item
  \(f(x) \geq 0\) pour tous les \(x\).
\item
  L'aire totale sous la courbe est de 1~:

  \[
  \int_{-\infty}^\infty f(x)\,dx = 1.
  \]
\end{enumerate}

Si \(X\) est une variable aléatoire continue avec pdf \(f(x)\), alors la
probabilité que \(X\) se situe entre \(a\) et \(b\) est

\[
P(a \leq X \leq b) = \int_a^b f(x)\,dx.
\]

\subsubsection{Fonction de distribution
cumulative}\label{fonction-de-distribution-cumulative}

La fonction de distribution cumulative (cdf) est définie comme

\[
F(x) = \int_{-\infty}^x f(t)\,dt.
\]

Il donne la probabilité que la variable aléatoire soit inférieure ou
égale à \(x\).

\subsubsection{Valeur attendue (moyenne)}\label{valeur-attendue-moyenne}

La valeur attendue d'une variable aléatoire continue est la moyenne
pondérée~:

\[
E[X] = \int_{-\infty}^\infty x f(x)\,dx.
\]

\subsubsection{Exemples}\label{exemples-15}

\begin{enumerate}
\def\labelenumi{\arabic{enumi}.}
\tightlist
\item
  Distribution uniforme Pour \(f(x) = \tfrac{1}{b-a}\) le \([a,b]\)~:
\end{enumerate}

\begin{itemize}
\item
  Probabilité d'intervalle \([c,d]\)~:

  \[
  P(c \leq X \leq d) = \frac{d-c}{b-a}.
  \]
\item
  Valeur attendue~: \(E[X] = \tfrac{a+b}{2}\).
\end{itemize}

\begin{enumerate}
\def\labelenumi{\arabic{enumi}.}
\setcounter{enumi}{1}
\tightlist
\item
  Distribution exponentielle Pour \(f(x) = \lambda e^{-\lambda x}\),
  \(x \geq 0\)~:
\end{enumerate}

\begin{itemize}
\tightlist
\item
  \(\int_0^\infty \lambda e^{-\lambda x}\,dx = 1\).
\item
  Moyenne : \(E[X] = \tfrac{1}{\lambda}\).
\end{itemize}

\begin{enumerate}
\def\labelenumi{\arabic{enumi}.}
\setcounter{enumi}{2}
\tightlist
\item
  Distribution normale La courbe en cloche :
\end{enumerate}

\[
f(x) = \frac{1}{\sqrt{2\pi\sigma^2}} e^{-\frac{(x-\mu)^2}{2\sigma^2}}.
\]

Il s'intègre à 1, mais nécessite des techniques avancées.

\subsubsection{Pourquoi c'est important- Les densités de probabilité
décrivent l'incertitude dans les domaines de la science, de l'ingénierie
et des
statistiques.}\label{pourquoi-cest-important--les-densituxe9s-de-probabilituxe9-duxe9crivent-lincertitude-dans-les-domaines-de-la-science-de-linguxe9nierie-et-des-statistiques.}

\begin{itemize}
\tightlist
\item
  Les intégrales relient les zones sous les courbes aux probabilités.
\item
  Les distributions continues généralisent l'idée de compter les
  résultats à la mesure des probabilités sur des intervalles.
\end{itemize}

\subsubsection{Exercices}\label{exercices-24}

\begin{enumerate}
\def\labelenumi{\arabic{enumi}.}
\tightlist
\item
  Montrer que la densité uniforme \(f(x) = \tfrac{1}{b-a}\) sur
  \([a,b]\) s'intègre à 1.
\item
  Pour la distribution exponentielle avec \(\lambda = 2\), calculez
  \(P(0 \leq X \leq 1)\).
\item
  Recherchez la valeur attendue de \(X\) si \(f(x) = 3x^2\) sur
  \([0,1]\).
\item
  Vérifiez que la distribution normale de moyenne 0 et de variance 1 a
  une probabilité totale de 1 (pas besoin de preuve complète, mais
  expliquez pourquoi elle est vraie).
\item
  Calculez le cdf de la distribution uniforme sur \([0,1]\).
\end{enumerate}

\section{Partie III. Calcul
multivarié}\label{partie-iii.-calcul-multivariuxe9}

\section{Chapitre 7. Fonctions vectorielles et
courbes}\label{chapitre-7.-fonctions-vectorielles-et-courbes}

\subsection{7.1 Fonctions vectorielles et courbes
spatiales}\label{fonctions-vectorielles-et-courbes-spatiales}

Dans le calcul multivarié, les fonctions peuvent générer des vecteurs au
lieu de nombres. Celles-ci sont appelées fonctions à valeurs
vectorielles et elles sont essentielles pour décrire les courbes dans
l'espace.

\subsubsection{Définition}\label{duxe9finition-6}

Une fonction vectorielle est une fonction de la forme

\[
\mathbf{r}(t) = \langle x(t), y(t), z(t) \rangle,
\]

où \(x(t), y(t), z(t)\) sont des fonctions à valeur réelle.

\begin{itemize}
\tightlist
\item
  L'entrée \(t\) est souvent appelée le paramètre.
\item
  La sortie est un vecteur dans l'espace 2D ou 3D.
\item
  Le graphique d'une fonction vectorielle en 3D est une courbe spatiale.
\end{itemize}

\subsubsection{Exemples}\label{exemples-16}

\begin{enumerate}
\def\labelenumi{\arabic{enumi}.}
\tightlist
\item
  Ligne
\end{enumerate}

\[
\mathbf{r}(t) = \langle 1+2t, \; 3-t, \; 4+5t \rangle.
\]

Ceci décrit une ligne droite passant par le point \((1,3,4)\) avec le
vecteur directeur \(\langle 2,-1,5 \rangle\).

\begin{enumerate}
\def\labelenumi{\arabic{enumi}.}
\setcounter{enumi}{1}
\tightlist
\item
  Cercle dans l'avion
\end{enumerate}

\[
\mathbf{r}(t) = \langle \cos t, \; \sin t, \; 0 \rangle, \quad 0 \leq t < 2\pi.
\]

\begin{enumerate}
\def\labelenumi{\arabic{enumi}.}
\setcounter{enumi}{2}
\tightlist
\item
  Hélice
\end{enumerate}

\[
\mathbf{r}(t) = \langle \cos t, \; \sin t, \; t \rangle.
\]

Il s'agit d'une spirale montant autour de l'axe \(z\).

\subsubsection{Limites et continuité}\label{limites-et-continuituxe9}

Une fonction vectorielle est continue en \(t=a\) si chaque composante
\(x(t), y(t), z(t)\) est continue en \(t=a\).

\[
\lim_{t \to a} \mathbf{r}(t) = \langle \lim_{t \to a} x(t), \; \lim_{t \to a} y(t), \; \lim_{t \to a} z(t) \rangle.
\]

\subsubsection{Géométrie des courbes
spatiales}\label{guxe9omuxe9trie-des-courbes-spatiales}

\begin{itemize}
\tightlist
\item
  Chaque courbe a une direction tangente donnée par la dérivée.
\item
  Les courbes spatiales peuvent modéliser les trajectoires de mouvement,
  les trajectoires des particules et les formes géométriques.
\end{itemize}

\subsubsection{Pourquoi c'est
important}\label{pourquoi-cest-important-8}

Les fonctions vectorielles constituent le fondement du calcul
multivariable, nous permettant d'étendre les idées de dérivées et
d'intégrales à des dimensions supérieures. Ils apparaissent aussi
naturellement en physique (mouvement en 3D, électromagnétisme, dynamique
des fluides).

\subsubsection{Exercices}\label{exercices-25}

\begin{enumerate}
\def\labelenumi{\arabic{enumi}.}
\tightlist
\item
  Écrivez une fonction vectorielle pour une ligne passant par
  \((0,1,2)\) parallèle au vecteur \(\langle 3,-2,1 \rangle\).2.
  Décrivez la courbe donnée par
  \(\mathbf{r}(t) = \langle 2\cos t, \; 2\sin t, \; 3 \rangle\).
\item
  Déterminez si
  \(\mathbf{r}(t) = \langle e^t, \; \ln t, \; t^2 \rangle\) est continu
  à \(t=1\).
\item
  Esquissez l'hélice
  \(\mathbf{r}(t) = \langle \cos t, \; \sin t, \; 2t \rangle\).
\item
  Trouvez le point sur la courbe
  \(\mathbf{r}(t) = \langle t, \; t^2, \; t^3 \rangle\) lorsque \(t=2\).
\end{enumerate}

\subsection{7.2 Dérivées et intégrales des fonctions
vectorielles}\label{duxe9rivuxe9es-et-intuxe9grales-des-fonctions-vectorielles}

Les fonctions vectorielles peuvent être différenciées et intégrées tout
comme les fonctions ordinaires : nous appliquons simplement l'opération
à chaque composant. Cela nous permet d'étudier le mouvement, la vitesse,
l'accélération et l'accumulation dans des dimensions supérieures.

\subsubsection{Dérivée d'une fonction
vectorielle}\label{duxe9rivuxe9e-dune-fonction-vectorielle}

Si

\[
\mathbf{r}(t) = \langle x(t), y(t), z(t) \rangle,
\]

alors

\[
\mathbf{r}'(t) = \langle x'(t), y'(t), z'(t) \rangle.
\]

Ce vecteur dérivé pointe dans la direction tangente à la courbe au
paramètre \(t\).

\begin{itemize}
\tightlist
\item
  Vitesse : Si \(\mathbf{r}(t)\) donne la position d'une particule à
  l'instant \(t\), alors \(\mathbf{v}(t) = \mathbf{r}'(t)\) est son
  vecteur vitesse.
\item
  Vitesse : La magnitude \(|\mathbf{v}(t)|\) est la vitesse de la
  particule.
\item
  Accélération : \(\mathbf{a}(t) = \mathbf{v}'(t) = \mathbf{r}''(t)\).
\end{itemize}

\subsubsection{Exemples}\label{exemples-17}

\begin{enumerate}
\def\labelenumi{\arabic{enumi}.}
\tightlist
\item
  Hélice
\end{enumerate}

\[
\mathbf{r}(t) = \langle \cos t, \sin t, t \rangle.
\]

\begin{itemize}
\tightlist
\item
  Vitesse~: \(\mathbf{v}(t) = \langle -\sin t, \cos t, 1 \rangle\).
\item
  Vitesse~:
  \(|\mathbf{v}(t)| = \sqrt{(-\sin t)^2 + (\cos t)^2 + 1^2} = \sqrt{2}\).
\item
  Accélération :
  \(\mathbf{a}(t) = \langle -\cos t, -\sin t, 0 \rangle\).
\end{itemize}

\begin{enumerate}
\def\labelenumi{\arabic{enumi}.}
\setcounter{enumi}{1}
\tightlist
\item
  Mouvement du projectile
\end{enumerate}

\[
\mathbf{r}(t) = \langle v_0 \cos\theta \cdot t, \; v_0 \sin\theta \cdot t - \tfrac{1}{2}gt^2 \rangle.
\]

Ceci modélise la trajectoire parabolique d'un projectile sous gravité.

\subsubsection{Intégrale d'une fonction
vectorielle}\label{intuxe9grale-dune-fonction-vectorielle}

Si

\[
\mathbf{r}(t) = \langle x(t), y(t), z(t) \rangle,
\]

alors

\[
\int \mathbf{r}(t)\,dt = \left\langle \int x(t)\,dt, \; \int y(t)\,dt, \; \int z(t)\,dt \right\rangle + \mathbf{C},
\]

où \(\mathbf{C}\) est un vecteur constant.

\subsubsection{Exemple}\label{exemple}

\[
\mathbf{r}(t) = \langle t, t^2, t^3 \rangle.
\]

\begin{itemize}
\tightlist
\item
  Dérivé~: \(\mathbf{r}'(t) = \langle 1, 2t, 3t^2 \rangle\).
\item
  Intégrale :
\end{itemize}

\[
\int \mathbf{r}(t)\,dt = \langle \tfrac{1}{2}t^2, \tfrac{1}{3}t^3, \tfrac{1}{4}t^4 \rangle + \mathbf{C}.
\]

\subsubsection{Pourquoi c'est
important}\label{pourquoi-cest-important-9}

\begin{itemize}
\tightlist
\item
  Les dérivées de fonctions vectorielles décrivent le mouvement et les
  forces dans l'espace.
\item
  Les intégrales donnent le déplacement, le travail et les quantités
  accumulées.
\item
  Ces outils relient le calcul directement à la physique et à
  l'ingénierie.
\end{itemize}

\subsubsection{Exercices}\label{exercices-26}

\begin{enumerate}
\def\labelenumi{\arabic{enumi}.}
\tightlist
\item
  Pour \(\mathbf{r}(t) = \langle t, \cos t, \sin t \rangle\), recherchez
  la vitesse, la vitesse et l'accélération.2. Calculez
  \(\mathbf{r}'(t)\) pour
  \(\mathbf{r}(t) = \langle e^t, \ln t, t^2 \rangle\).
\item
  Intégrez \(\mathbf{r}(t) = \langle 1, t, t^2 \rangle\).
\item
  Une particule a une vitesse
  \(\mathbf{v}(t) = \langle t, 2, 0 \rangle\). Trouvez son vecteur de
  position si \(\mathbf{r}(0) = \langle 1, 0, 0 \rangle\).
\item
  Montrez que la vitesse de
  \(\mathbf{r}(t) = \langle \cos t, \sin t, 0 \rangle\) est constante.
\end{enumerate}

\subsection{7.3 Longueur et courbure de
l'arc}\label{longueur-et-courbure-de-larc}

Le calcul vectoriel fournit des outils pour mesurer non seulement le
chemin tracé par une courbe, mais également l'ampleur de sa courbure.
Ceux-ci sont exprimés par la longueur et la courbure de l'arc.

\subsubsection{Longueur de l'arc d'une courbe
spatiale}\label{longueur-de-larc-dune-courbe-spatiale}

Si une courbe est donnée par

\[
\mathbf{r}(t) = \langle x(t), y(t), z(t) \rangle, \quad a \leq t \leq b,
\]

alors la longueur de l'arc est

\[
L = \int_a^b |\mathbf{r}'(t)|\,dt,
\]

où

\[
|\mathbf{r}'(t)| = \sqrt{(x'(t))^2 + (y'(t))^2 + (z'(t))^2}.
\]

Exemple~: Pour l'hélice
\(\mathbf{r}(t) = \langle \cos t, \sin t, t \rangle, \, 0 \leq t \leq 2\pi\)~:

\begin{itemize}
\tightlist
\item
  Vitesse~: \(\mathbf{r}'(t) = \langle -\sin t, \cos t, 1 \rangle\).
\item
  Vitesse~:
  \(|\mathbf{r}'(t)| = \sqrt{(-\sin t)^2 + (\cos t)^2 + 1^2} = \sqrt{2}\).
\item
  Longueur des arcs~:
\end{itemize}

\[
L = \int_0^{2\pi} \sqrt{2}\,dt = 2\pi\sqrt{2}.
\]

\subsubsection{Courbure}\label{courbure}

La courbure mesure la rapidité avec laquelle une courbe change de
direction.

Pour une courbe lisse \(\mathbf{r}(t)\)~:

\[
\kappa(t) = \frac{|\mathbf{r}'(t) \times \mathbf{r}''(t)|}{|\mathbf{r}'(t)|^3}.
\]

\begin{itemize}
\tightlist
\item
  \(\kappa = 0\) : ligne droite.
\item
  Plus grand \(\kappa\)~: la courbe se plie plus fortement.
\end{itemize}

Exemple~: Pour un cercle de rayon \(r\)~:

\[
\mathbf{r}(t) = \langle r\cos t, r\sin t \rangle.
\]

Puis \(\kappa = \tfrac{1}{r}\). La courbure est donc constante et
inversement proportionnelle au rayon.

\subsubsection{Tangente unitaire et vecteurs
normaux}\label{tangente-unitaire-et-vecteurs-normaux}

\begin{itemize}
\tightlist
\item
  Vecteur tangent :
\end{itemize}

\[
\mathbf{T}(t) = \frac{\mathbf{r}'(t)}{|\mathbf{r}'(t)|}.
\]

\begin{itemize}
\tightlist
\item
  Vecteur normal~: pointe vers le centre de courbure, défini comme
\end{itemize}

\[
\mathbf{N}(t) = \frac{\mathbf{T}'(t)}{|\mathbf{T}'(t)|}.
\]

Ces vecteurs décrivent la géométrie du mouvement : sens de déplacement
et sens de virage.

\subsubsection{Pourquoi c'est
important}\label{pourquoi-cest-important-10}

\begin{itemize}
\tightlist
\item
  La longueur d'arc généralise la notion de distance aux courbes dans
  l'espace.
\item
  La courbure décrit la flexion, cruciale en physique (accélération
  centripète), en ingénierie (routes, montagnes russes) et en
  infographie.
\end{itemize}

\subsubsection{Exercices}\label{exercices-27}

\begin{enumerate}
\def\labelenumi{\arabic{enumi}.}
\tightlist
\item
  Trouvez la longueur de l'arc de
  \(\mathbf{r}(t) = \langle t, t^2, 0 \rangle\) de \(t=0\) à \(t=1\).
\item
  Calculez la courbure du cercle
  \(\mathbf{r}(t) = \langle \cos t, \sin t \rangle\).
\item
  Pour \(\mathbf{r}(t) = \langle t, \cos t, \sin t \rangle\), calculez
  \(|\mathbf{r}'(t)|\).
\item
  Montrez qu'une ligne droite a une courbure \(\kappa = 0\).5.
  Recherchez le vecteur tangent à
  \(\mathbf{r}(t) = \langle e^t, e^{-t}, t \rangle\) en \(t=0\).
\end{enumerate}

\subsection{7.4 Mouvement dans l'espace}\label{mouvement-dans-lespace}

Les fonctions vectorielles sont particulièrement puissantes pour décrire
le mouvement en deux ou trois dimensions. La position, la vitesse et
l'accélération sont naturellement exprimées à l'aide de dérivées et
d'intégrales de fonctions à valeurs vectorielles.

\subsubsection{Position, vitesse et
accélération}\label{position-vitesse-et-accuxe9luxe9ration}

\begin{itemize}
\tightlist
\item
  Vecteur de position~:
\end{itemize}

\[
\mathbf{r}(t) = \langle x(t), y(t), z(t) \rangle
\]

\begin{itemize}
\tightlist
\item
  Vecteur vitesse (dérivé de la position) :
\end{itemize}

\[
\mathbf{v}(t) = \mathbf{r}'(t) = \langle x'(t), y'(t), z'(t) \rangle
\]

\begin{itemize}
\tightlist
\item
  Vitesse (amplitude de la vitesse)~:
\end{itemize}

\[
|\mathbf{v}(t)| = \sqrt{(x'(t))^2 + (y'(t))^2 + (z'(t))^2}
\]

\begin{itemize}
\tightlist
\item
  Vecteur accélération (dérivée de la vitesse) :
\end{itemize}

\[
\mathbf{a}(t) = \mathbf{v}'(t) = \mathbf{r}''(t).
\]

\subsubsection{Composants tangentiels et
normaux}\label{composants-tangentiels-et-normaux}

L'accélération peut être décomposée en deux composantes~:

\[
\mathbf{a}(t) = a_T \mathbf{T}(t) + a_N \mathbf{N}(t),
\]

où :

\begin{itemize}
\tightlist
\item
  \(\mathbf{T}(t)\) = vecteur tangent unitaire,
\item
  \(\mathbf{N}(t)\) = vecteur normal principal,
\item
  \(a_T = \frac{d}{dt}|\mathbf{v}(t)|\) = accélération tangentielle
  (changement de vitesse),
\item
  \(a_N = \kappa |\mathbf{v}(t)|^2\) = accélération normale (changement
  de direction).
\end{itemize}

\subsubsection{Mouvement du projectile en
3D}\label{mouvement-du-projectile-en-3d}

Avec la gravité agissant dans la direction \(-z\)~:

\[
\mathbf{r}(t) = \langle v_0 \cos\theta \cos\phi \cdot t,\; v_0 \cos\theta \sin\phi \cdot t,\; v_0 \sin\theta \cdot t - \tfrac{1}{2}gt^2 \rangle,
\]

où \(v_0\) est la vitesse initiale, \(\theta\) l'angle de lancement et
\(\phi\) la direction azimutale.

\subsubsection{Exemple~: mouvement
hélicoïdal}\label{exemple-mouvement-huxe9licouxefdal}

\[
\mathbf{r}(t) = \langle \cos t, \sin t, t \rangle
\]

\begin{itemize}
\tightlist
\item
  Vitesse~: \(\mathbf{v}(t) = \langle -\sin t, \cos t, 1 \rangle\).
\item
  Vitesse~: \(|\mathbf{v}(t)| = \sqrt{2}\).
\item
  Accélération :
  \(\mathbf{a}(t) = \langle -\cos t, -\sin t, 0 \rangle\).
\item
  Le mouvement est uniforme en vitesse, en spirale vers le haut.
\end{itemize}

\subsubsection{Pourquoi c'est
important}\label{pourquoi-cest-important-11}

\begin{itemize}
\tightlist
\item
  Fournit un langage mathématique pour le mouvement du monde réel.
\item
  Indispensable en physique (forces, trajectoires, mouvement
  circulaire).
\item
  Fondation pour la mécanique avancée et les modèles d'ingénierie.
\end{itemize}

\subsubsection{Exercices}\label{exercices-28}

\begin{enumerate}
\def\labelenumi{\arabic{enumi}.}
\tightlist
\item
  Une particule se déplace le long de
  \(\mathbf{r}(t) = \langle t, t^2, t^3 \rangle\). Trouvez la vitesse et
  l'accélération sur \(t=1\).
\item
  Montrez que la vitesse est constante pour l'hélice
  \(\mathbf{r}(t) = \langle \cos t, \sin t, t \rangle\).
\item
  Un projectile est lancé avec \(v_0 = 20 \,\text{m/s}\) à l'angle
  \(45^\circ\). Écrivez son vecteur de position en supposant un
  mouvement dans un plan vertical.
\item
  Pour \(\mathbf{r}(t) = \langle e^t, e^{-t}, t \rangle\), recherchez
  \(\mathbf{v}(t)\) et \(\mathbf{a}(t)\).
\item
  Décomposez le vecteur d'accélération en composantes tangentielles et
  normales pour un mouvement le long d'un cercle de rayon \(r\).\#
  Chapitre 8. Fonctions de plusieurs variables
\end{enumerate}

\subsection{8.1 Limites et continuité dans plusieurs
variables}\label{limites-et-continuituxe9-dans-plusieurs-variables}

Dans le calcul multivarié, les fonctions peuvent dépendre de deux
variables ou plus, telles que \(f(x,y)\) ou \(f(x,y,z)\). Les concepts
de limites et de continuité découlent naturellement du calcul à variable
unique, mais ils sont plus subtils car il faut considérer toutes les
voies d'approche possibles.

\subsubsection{Limites dans deux
variables}\label{limites-dans-deux-variables}

Pour une fonction \(f(x,y)\), on dit

\[
\lim_{(x,y) \to (a,b)} f(x,y) = L
\]

si \(f(x,y)\) se rapproche arbitrairement de \(L\) alors que \((x,y)\)
s'approche de \((a,b)\) le long de n'importe quel chemin.

Si différents chemins donnent des valeurs limites différentes, alors la
limite n'existe pas.

Exemple 1 (une limite existe)~:

\[
f(x,y) = x^2 + y^2, \quad \lim_{(x,y) \to (0,0)} f(x,y) = 0.
\]

Exemple 2 (la limite n'existe pas)~:

\[
f(x,y) = \frac{xy}{x^2+y^2}, \quad (x,y) \to (0,0).
\]

\begin{itemize}
\tightlist
\item
  Le long de \(y=0\), la fonction est 0.
\item
  Le long de \(y=x\), la fonction est \(\tfrac{1}{2}\). Résultats
  différents → la limite n'existe pas.
\end{itemize}

\subsubsection{Continuité}\label{continuituxe9-1}

Une fonction \(f(x,y)\) est continue à \((a,b)\) si

\[
\lim_{(x,y)\to(a,b)} f(x,y) = f(a,b).
\]

Les polynômes et les fonctions rationnelles (où dénominateur ≠ 0) sont
continues partout dans leurs domaines.

\subsubsection{Extension à trois variables ou
plus}\label{extension-uxe0-trois-variables-ou-plus}

Pour \(f(x,y,z)\), les limites et la continuité sont définies de la même
manière, mais le point \((a,b,c)\) doit être approché depuis une
infinité de directions dans l'espace.

\subsubsection{Pourquoi c'est
important}\label{pourquoi-cest-important-12}

\begin{itemize}
\tightlist
\item
  La continuité garantit l'absence de sauts, de trous ou d'asymptotes
  dans les fonctions multivariables.
\item
  Les limites sont fondamentales pour définir les dérivées partielles et
  les intégrales multiples.
\item
  Ces concepts sont des éléments constitutifs du calcul multivarié.
\end{itemize}

\subsubsection{Exercices}\label{exercices-29}

\begin{enumerate}
\def\labelenumi{\arabic{enumi}.}
\tightlist
\item
  Déterminez si \(\lim_{(x,y)\to(0,0)} (x^2+y^2)\) existe.
\item
  Montrez que \(\lim_{(x,y)\to(0,0)} \frac{x^2y}{x^2+y^2} = 0\) le long
  de tous les chemins en ligne droite \(y=mx\).
\item
  La limite existe-t-elle pour \(f(x,y) = \frac{x^2-y^2}{x^2+y^2}\) en
  tant que \((x,y)\to(0,0)\)~?
\item
  Expliquez pourquoi les polynômes à deux variables sont continus
  partout.
\item
  Donnez un exemple d'une fonction de deux variables qui est discontinue
  en un point et expliquez pourquoi.
\end{enumerate}

\subsection{8.2 Dérivées partielles}\label{duxe9rivuxe9es-partielles}

Dans les fonctions de plusieurs variables, nous souhaitons souvent
mesurer la façon dont la fonction change lorsqu'une seule variable
change tandis que les autres restent constantes. Cela conduit à l'idée
de dérivées partielles.

\subsubsection{Définition}\label{duxe9finition-7}

Pour une fonction \(f(x,y)\), la dérivée partielle par rapport à \(x\)
en un point \((a,b)\) est

\[
\frac{\partial f}{\partial x}(a,b) = \lim_{h \to 0} \frac{f(a+h, b) - f(a,b)}{h}.
\]

De même, la dérivée partielle par rapport à \(y\) est

\[\frac{\partial f}{\partial y}(a,b) = \lim_{h \to 0} \frac{f(a, b+h) - f(a,b)}{h}.
\]

We treat all other variables as constants when differentiating.

\subsubsection{Notation}\label{notation-1}

\begin{itemize}
\tightlist
\item
  \(\frac{\partial f}{\partial x}\), \(f_x\), \(\partial_x f\).
\item
  \(\frac{\partial f}{\partial y}\), \(f_y\), \(\partial_y f\).
\end{itemize}

For three variables \(f(x,y,z)\), we also have \(f_x, f_y, f_z\).

\subsubsection{Examples}\label{examples-2}

\begin{enumerate}
\def\labelenumi{\arabic{enumi}.}
\tightlist
\item
  \(f(x,y) = x^2y + y^3\)
\end{enumerate}

\begin{itemize}
\tightlist
\item
  \(f_x = 2xy\).
\item
  \(f_y = x^2 + 3y^2\).
\end{itemize}

\begin{enumerate}
\def\labelenumi{\arabic{enumi}.}
\setcounter{enumi}{1}
\tightlist
\item
  \(f(x,y) = e^{xy}\)
\end{enumerate}

\begin{itemize}
\tightlist
\item
  \(f_x = y e^{xy}\).
\item
  \(f_y = x e^{xy}\).
\end{itemize}

\begin{enumerate}
\def\labelenumi{\arabic{enumi}.}
\setcounter{enumi}{2}
\tightlist
\item
  \(f(x,y,z) = x^2 + yz\)
\end{enumerate}

\begin{itemize}
\tightlist
\item
  \(f_x = 2x\).
\item
  \(f_y = z\).
\item
  \(f_z = y\).
\end{itemize}

\subsubsection{Higher-Order Partial
Derivatives}\label{higher-order-partial-derivatives}

We can take partial derivatives repeatedly:

\begin{itemize}
\tightlist
\item
  \(f_{xx} = \frac{\partial}{\partial x}\Big(f_x\Big)\).
\item
  \(f_{yy}, f_{xy}, f_{yx}\), etc.
\end{itemize}

Clairaut's Theorem: If \(f\) has continuous second partial derivatives,
then

\[
f_{xy} = f_{yx}.
\]

\subsubsection{Geometric Meaning}\label{geometric-meaning}

\begin{itemize}
\tightlist
\item
  \(f_x\): slope of the surface in the \(x\)-direction.
\item
  \(f_y\): slope of the surface in the \(y\)-direction.
\item
  Together they describe how the surface tilts.
\end{itemize}

\subsubsection{Why This Matters}\label{why-this-matters}

\begin{itemize}
\tightlist
\item
  Partial derivatives are the foundation of gradients, tangent planes,
  and optimization in multiple variables.
\item
  They are widely used in physics, engineering, and economics to model
  systems with several inputs.
\end{itemize}

\subsubsection{Exercises}\label{exercises-3}

\begin{enumerate}
\def\labelenumi{\arabic{enumi}.}
\tightlist
\item
  Find \(f_x\) and \(f_y\) for \(f(x,y) = x^3y^2\).
\item
  Compute \(f_x, f_y, f_z\) for \(f(x,y,z) = xyz + x^2\).
\item
  Verify Clairaut's theorem for \(f(x,y) = x^2y + y^3\).
\item
  Interpret geometrically what \(f_x\) and \(f_y\) mean for
  \(f(x,y) = \sqrt{x^2+y^2}\).
\item
  Find all second-order partial derivatives of \(f(x,y) = e^{x^2+y^2}\).
\end{enumerate}

\subsection{8.3 Gradient and Directional
Derivatives}\label{gradient-and-directional-derivatives}

Partial derivatives measure change along the coordinate axes, but
sometimes we want to know the rate of change of a function in any
direction. This leads to the concepts of the gradient and directional
derivatives.

\subsubsection{Gradient Vector}\label{gradient-vector}

For a function \(f(x,y)\), the gradient is the vector

\[
\nabla f(x,y) = \left\langle \frac{\partial f}{\partial x}, \frac{\partial f}{\partial y} \right\rangle.
\]

For three variables \(f(x,y,z)\):

\[
\nabla f(x,y,z) = \left\langle f_x, f_y, f_z \right\rangle.
\]

The gradient points in the direction of maximum increase of the
function, and its magnitude gives the steepest slope.

\subsubsection{Directional Derivatives}\label{directional-derivatives}

The rate of change of \(f(x,y)\) at a point in the direction of a unit
vector \(\mathbf{u} = \langle u_1, u_2 \rangle\) is

\[
D_{\mathbf{u}} f(x,y) = \nabla f(x,y) \cdot \mathbf{u}.
\]

Il s'agit du produit scalaire du dégradé avec le vecteur directeur.

\subsubsection{Exemples}\label{exemples-18}

\begin{enumerate}
\def\labelenumi{\arabic{enumi}.}
\tightlist
\item
  \(f(x,y) = x^2 + y^2\)
\end{enumerate}

\begin{itemize}
\tightlist
\item
  Dégradé~: \(\nabla f = \langle 2x, 2y \rangle\).- En (1,2) :
  \(\nabla f = \langle 2,4 \rangle\).
\item
  Dérivée directionnelle selon
  \(\mathbf{u} = \langle \tfrac{3}{5}, \tfrac{4}{5} \rangle\)~:
\end{itemize}

\[
D_{\mathbf{u}} f(1,2) = \langle 2,4 \rangle \cdot \langle \tfrac{3}{5}, \tfrac{4}{5} \rangle = \tfrac{26}{5}.
\]

\begin{enumerate}
\def\labelenumi{\arabic{enumi}.}
\setcounter{enumi}{1}
\tightlist
\item
  \(f(x,y,z) = x y z\)
\end{enumerate}

\begin{itemize}
\tightlist
\item
  Dégradé~: \(\nabla f = \langle yz, xz, xy \rangle\).
\item
  En (1,1,1) : \(\nabla f = \langle 1,1,1 \rangle\).
\item
  La direction d'augmentation maximale est le long de
  \(\langle 1,1,1 \rangle\).
\end{itemize}

\subsubsection{Interprétation
géométrique}\label{interpruxe9tation-guxe9omuxe9trique-1}

\begin{itemize}
\tightlist
\item
  Le vecteur gradient est perpendiculaire (normal) aux courbes de niveau
  ou aux surfaces planes de \(f\).
\item
  Les dérivées directionnelles généralisent la pente dans des directions
  arbitraires.
\end{itemize}

\subsubsection{Pourquoi c'est
important}\label{pourquoi-cest-important-13}

\begin{itemize}
\tightlist
\item
  En optimisation, la pente nous indique la direction dans laquelle se
  déplacer pour la montée ou la descente la plus raide.
\item
  En physique, les gradients décrivent des champs tels que le flux de
  chaleur et le potentiel électrique.
\item
  Les dérivées directionnelles unifient les taux de changement à une et
  plusieurs variables.
\end{itemize}

\subsubsection{Exercices}\label{exercices-30}

\begin{enumerate}
\def\labelenumi{\arabic{enumi}.}
\tightlist
\item
  Calculez \(\nabla f(x,y)\) pour \(f(x,y) = e^{xy}\).
\item
  Trouvez le gradient de \(f(x,y,z) = x^2+y^2+z^2\) et évaluez-le à
  (1,1,1).
\item
  Calculez la dérivée directionnelle de \(f(x,y) = x^2-y\) en (2,1) dans
  la direction de \(\mathbf{u} = \langle 0,1 \rangle\).
\item
  Montrez que le gradient de \(f(x,y) = x^2+y^2\) est perpendiculaire au
  cercle \(x^2+y^2=1\).
\item
  Trouvez la direction du vecteur unitaire qui maximise la dérivée
  directionnelle de \(f(x,y) = xy\) en (1,2).
\end{enumerate}

\subsection{8.4 Plans tangents et approximations
linéaires}\label{plans-tangents-et-approximations-linuxe9aires}

Dans le calcul à variable unique, la ligne tangente se rapproche d'une
courbe proche d'un point. En calcul multivarié, le concept analogue est
le plan tangent, qui fournit une approximation linéaire d'une surface
proche d'un point.

\subsubsection{Plan tangent à une
surface}\label{plan-tangent-uxe0-une-surface}

Supposons que \(z = f(x,y)\) soit différentiable en \((a,b)\). Le plan
tangent à \((a,b,f(a,b))\) est donné par

\[
z = f(a,b) + f_x(a,b)(x-a) + f_y(a,b)(y-b).
\]

Ce plan touche la surface au point et s'en rapproche à proximité.

\subsubsection{Exemple 1 : Paraboloïde}\label{exemple-1-parabolouxefde}

Pour \(f(x,y) = x^2 + y^2\) à \((1,2)\)~:

\begin{itemize}
\tightlist
\item
  \(f(1,2) = 1^2+2^2=5\).
\item
  \(f_x = 2x\), donc \(f_x(1,2) = 2\).
\item
  \(f_y = 2y\), donc \(f_y(1,2) = 4\).
\end{itemize}

Équation du plan tangent~:

\[
z = 5 + 2(x-1) + 4(y-2).
\]

\subsubsection{Approximation linéaire}\label{approximation-linuxe9aire}

Le plan tangent peut être utilisé pour approximer \(f(x,y)\) près de
\((a,b)\)~:

\[
f(x,y) \approx f(a,b) + f_x(a,b)(x-a) + f_y(a,b)(y-b).
\]

Il s'agit de la linéarisation de \(f\) à \((a,b)\).

\subsubsection{Exemple 2~:~approximation
linéaire}\label{exemple-2-approximation-linuxe9aire}

Environ \(f(x,y) = \sqrt{x+y}\) près de \((4,5)\).

\begin{itemize}
\tightlist
\item
  \(f(4,5) = \sqrt{9} = 3\).
\item
  \(f_x = \frac{1}{2\sqrt{x+y}}, \quad f_y = \frac{1}{2\sqrt{x+y}}\).
\item
  En (4,5) : \(f_x = f_y = \tfrac{1}{6}\).
\end{itemize}

Alors,

\[f(x,y) \approx 3 + \tfrac{1}{6}(x-4) + \tfrac{1}{6}(y-5).
\]

\subsubsection{Why This Matters}\label{why-this-matters-1}

\begin{itemize}
\tightlist
\item
  Tangent planes give the best linear approximation to a surface.
\item
  Linearization simplifies complex functions for computation.
\item
  Widely used in numerical methods, physics, and economics.
\end{itemize}

\subsubsection{Exercises}\label{exercises-4}

\begin{enumerate}
\def\labelenumi{\arabic{enumi}.}
\tightlist
\item
  Find the tangent plane to \(z = x^2y + y^2\) at \((1,1)\).
\item
  Approximate \(f(x,y) = e^{x+y}\) near \((0,0)\).
\item
  Derive the tangent plane equation for \(z = \ln(x^2+y^2)\) at
  \((1,1)\).
\item
  Use linear approximation to estimate \(\sqrt{10.1}\) using
  \(f(x,y) = \sqrt{x+y}\) near (4,6).
\item
  Explain why the tangent plane approximation improves as \((x,y)\) gets
  closer to \((a,b)\).
\end{enumerate}

\subsection{8.5 Optimization in Several
Variables}\label{optimization-in-several-variables}

Optimization in multivariable calculus extends the ideas of maxima and
minima from single-variable functions to functions of two or more
variables.

\subsubsection{Critical Points}\label{critical-points}

For \(f(x,y)\), a critical point occurs where

\[
f_x(x,y) = 0 \quad \text{and} \quad f_y(x,y) = 0,
\]

or where the partial derivatives do not exist.

\subsubsection{Second Derivative Test}\label{second-derivative-test}

To classify critical points, compute the second partial derivatives:

\[
D = f_{xx}(a,b) f_{yy}(a,b) - \big(f_{xy}(a,b)\big)^2.
\]

\begin{itemize}
\tightlist
\item
  If \(D > 0\) and \(f_{xx}(a,b) > 0\): local minimum.
\item
  If \(D > 0\) and \(f_{xx}(a,b) < 0\): local maximum.
\item
  If \(D < 0\): saddle point.
\item
  If \(D = 0\): test is inconclusive.
\end{itemize}

\subsubsection{Example 1: Paraboloid}\label{example-1-paraboloid}

\(f(x,y) = x^2 + y^2\).

\begin{itemize}
\tightlist
\item
  \(f_x = 2x, f_y = 2y\). Critical point at (0,0).
\item
  \(f_{xx} = 2, f_{yy} = 2, f_{xy} = 0\).
\item
  \(D = (2)(2) - 0 = 4 > 0\), and \(f_{xx} > 0\).
\item
  So (0,0) is a local minimum.
\end{itemize}

\subsubsection{Example 2: Saddle Point}\label{example-2-saddle-point}

\(f(x,y) = x^2 - y^2\).

\begin{itemize}
\tightlist
\item
  \(f_x = 2x, f_y = -2y\). Critical point at (0,0).
\item
  \(f_{xx} = 2, f_{yy} = -2, f_{xy} = 0\).
\item
  \(D = (2)(-2) - 0 = -4 < 0\).
\item
  So (0,0) is a saddle point.
\end{itemize}

\subsubsection{Constrained Optimization and Lagrange
Multipliers}\label{constrained-optimization-and-lagrange-multipliers}

Sometimes, we want to optimize \(f(x,y)\) subject to a constraint
\(g(x,y) = c\).

Method of Lagrange multipliers: solve

\[
\nabla f(x,y) = \lambda \nabla g(x,y).
\]

Exemple~: Maximisez \(f(x,y) = xy\) sous réserve de \(x^2+y^2=1\).

\begin{itemize}
\tightlist
\item
  Dégradés~:
  \(\nabla f = \langle y,x \rangle, \quad \nabla g = \langle 2x,2y \rangle\).
\item
  Équations : \(y = 2\lambda x, \, x = 2\lambda y\).
\item
  Les solutions mènent au maximum à
  \((\pm \tfrac{1}{\sqrt{2}}, \pm \tfrac{1}{\sqrt{2}})\).
\end{itemize}

\subsubsection{Pourquoi c'est
important}\label{pourquoi-cest-important-14}

\begin{itemize}
\tightlist
\item
  L'optimisation est essentielle en économie, en ingénierie, en
  apprentissage automatique et en physique.
\item
  Les multiplicateurs de Lagrange permettent une optimisation avec
  contraintes, un outil clé en mathématiques appliquées.
\end{itemize}

\subsubsection{Exercices}\label{exercices-31}

\begin{enumerate}
\def\labelenumi{\arabic{enumi}.}
\tightlist
\item
  Trouvez et classez les points critiques de \(f(x,y) = x^2+xy+y^2\).
\item
  Classez le point (0,0) pour \(f(x,y) = x^3-y^3\).3. Utilisez le test
  de dérivée seconde pour \(f(x,y) = x^4+y^4-4xy\).
\item
  Maximisez \(f(x,y) = x+y\) sous réserve de \(x^2+y^2=1\).
\item
  Réduisez \(f(x,y) = x^2+2y^2\) sous réserve de \(x+y=1\).
\end{enumerate}

\section{Chapitre 9. Intégrales
multiples}\label{chapitre-9.-intuxe9grales-multiples}

\subsection{9.1 Intégrales doubles}\label{intuxe9grales-doubles}

Dans le calcul à variable unique, une intégrale définie donne l'aire
sous une courbe. À deux variables, une intégrale double calcule le
volume sous une surface (ou plus généralement, l'accumulation de valeurs
sur une région).

\subsubsection{Définition}\label{duxe9finition-8}

Si \(f(x,y)\) est continue sur une région \(R\), l'intégrale double est

\[
\iint_R f(x,y)\, dA = \lim_{m,n \to \infty} \sum_{i=1}^m \sum_{j=1}^n f(x_{ij}^-, y_{ij}^-) \Delta A,
\]

où \(R\) est divisé en petits rectangles de zone \(\Delta A\).

\subsubsection{Intégrales itérées}\label{intuxe9grales-ituxe9ruxe9es}

Grâce au théorème de Fubini, nous pouvons calculer une intégrale double
comme une intégrale itérée~:

\[
\iint_R f(x,y)\, dA = \int_a^b \int_c^d f(x,y)\, dy\, dx,
\]

si \(R\) est un rectangle \([a,b] \times [c,d]\).

L'ordre d'intégration peut souvent être inversé~:

\[
\int_a^b \int_c^d f(x,y)\,dy\,dx = \int_c^d \int_a^b f(x,y)\,dx\,dy.
\]

\subsubsection{Exemples}\label{exemples-19}

\begin{enumerate}
\def\labelenumi{\arabic{enumi}.}
\tightlist
\item
  Région rectangulaire
\end{enumerate}

\[
\iint_R (x+y)\, dA, \quad R=[0,1]\times[0,2].
\]

\[
= \int_0^1 \int_0^2 (x+y)\,dy\,dx = \int_0^1 \Big[xy+\tfrac{1}{2}y^2\Big]_0^2 dx
= \int_0^1 (2x+2)dx = 3.
\]

\begin{enumerate}
\def\labelenumi{\arabic{enumi}.}
\setcounter{enumi}{1}
\tightlist
\item
  Région triangulaire
\end{enumerate}

\[
R = \{(x,y): 0 \leq x \leq 1, 0 \leq y \leq x\}.
\]

\[
\iint_R (x+y)\, dA = \int_0^1 \int_0^x (x+y)\,dy\,dx.
\]

L'évaluation donne \(\tfrac{2}{3}\).

\#\#\#~Candidatures

\begin{itemize}
\tightlist
\item
  Volume sous une surface :
\end{itemize}

\[
V = \iint_R f(x,y)\, dA.
\]

\begin{itemize}
\tightlist
\item
  Valeur moyenne d'une fonction sur une région :
\end{itemize}

\[
f_{\text{avg}} = \frac{1}{A(R)} \iint_R f(x,y)\, dA.
\]

\subsubsection{Pourquoi c'est
important}\label{pourquoi-cest-important-15}

Les intégrales doubles étendent l'intégration à deux dimensions. Ils
sont essentiels en physique (masse, distributions de probabilité), en
économie (valeurs attendues) et en ingénierie (centroïdes, flux).

\subsubsection{Exercices}\label{exercices-32}

\begin{enumerate}
\def\labelenumi{\arabic{enumi}.}
\tightlist
\item
  Évaluez \(\iint_R (x^2+y^2)\, dA\) où \(R=[0,1]\times[0,1]\).
\item
  Calculez \(\iint_R xy\, dA\) où
  \(R=\{(x,y):0\leq x\leq2,0\leq y\leq x\}\).
\item
  Trouvez la valeur moyenne de \(f(x,y) = x+y\) sur le carré unitaire
  \([0,1]\times[0,1]\).
\item
  Interprétez \(\iint_R f(x,y)\, dA\) en termes de probabilité si
  \(f(x,y)\) est une fonction de densité de probabilité.
\item
  Montrez que le changement d'ordre d'intégration donne le même résultat
  pour \(\iint_{[0,1]\times[0,2]} (x+y)\,dA\).
\end{enumerate}

\subsection{9.2 Triples intégrales}\label{triples-intuxe9grales}

Les intégrales triples étendent l'idée d'intégration à trois variables,
nous permettant de calculer des volumes, des masses et d'autres
quantités dans des régions tridimensionnelles.

\subsubsection{Définition}\label{duxe9finition-9}

Si \(f(x,y,z)\) est continue sur une région solide \(E\), l'intégrale
triple est

\[\iiint_E f(x,y,z)\, dV = \lim_{m,n,p \to \infty} \sum f(x_{ijk}^-, y_{ijk}^-, z_{ijk}^-) \Delta V,
\]

where the region is subdivided into boxes of volume \(\Delta V\).

\subsubsection{Iterated Integrals}\label{iterated-integrals}

By Fubini's Theorem, a triple integral can be computed as an iterated
integral:

\[
\iiint_E f(x,y,z)\, dV = \int_a^b \int_c^d \int_e^f f(x,y,z)\, dz\, dy\, dx,
\]

for a rectangular box \(E = [a,b]\times[c,d]\times[e,f]\).

The order of integration can be chosen for convenience.

\subsubsection{Examples}\label{examples-3}

\begin{enumerate}
\def\labelenumi{\arabic{enumi}.}
\tightlist
\item
  Rectangular box
\end{enumerate}

\[
\iiint_E xyz\, dV, \quad E=[0,1]\times[0,2]\times[0,3].
\]

\[
= \int_0^1 \int_0^2 \int_0^3 xyz\,dz\,dy\,dx.
\]

First integrate over \(z\):

\[
\int_0^3 xyz\,dz = xy \left[\tfrac{1}{2}z^2\right]_0^3 = \tfrac{9}{2}xy.
\]

Now integrate over \(y\):

\[
\int_0^2 \tfrac{9}{2}xy\,dy = \tfrac{9}{2}x \cdot \left[\tfrac{1}{2}y^2\right]_0^2 = 9x.
\]

Finally integrate over \(x\):

\[
\int_0^1 9x\,dx = \tfrac{9}{2}.
\]

\begin{enumerate}
\def\labelenumi{\arabic{enumi}.}
\setcounter{enumi}{1}
\tightlist
\item
  Region bounded by planes Let
  \(E = \{(x,y,z) \mid 0 \leq x \leq 1, 0 \leq y \leq x, 0 \leq z \leq y\}\).
\end{enumerate}

\[
\iiint_E 1\,dV = \int_0^1 \int_0^x \int_0^y 1\,dz\,dy\,dx.
\]

Evaluate:

\[
= \int_0^1 \int_0^x y\,dy\,dx = \int_0^1 \tfrac{1}{2}x^2\,dx = \tfrac{1}{6}.
\]

So the volume of this triangular region is \(\tfrac{1}{6}\).

\subsubsection{Applications}\label{applications}

\begin{itemize}
\item
  Volume: \(V = \iiint_E 1 \, dV\).
\item
  Mass: If density is \(\rho(x,y,z)\), then

  \[
  M = \iiint_E \rho(x,y,z)\, dV.
  \]
\item
  Average value:

  \[
  f_{\text{avg}} = \frac{1}{V(E)} \iiint_E f(x,y,z)\,dV.
  \]
\end{itemize}

\subsubsection{Why This Matters}\label{why-this-matters-2}

Triple integrals generalize area and volume calculations to arbitrary
solids. They are used in physics (mass distributions, center of mass,
gravitational fields), engineering, and probability.

\subsubsection{Exercises}\label{exercises-5}

\begin{enumerate}
\def\labelenumi{\arabic{enumi}.}
\tightlist
\item
  Compute \(\iiint_E (x+y+z)\,dV\) over the cube
  \(E=[0,1]\times[0,1]\times[0,1]\).
\item
  Find the volume of the tetrahedron bounded by
  \(x=0, y=0, z=0, x+y+z=1\).
\item
  Evaluate \(\iiint_E x^2 \, dV\) where
  \(E=[0,2]\times[0,1]\times[0,1]\).
\item
  Show that \(\iiint_E 1\,dV\) equals the geometric volume of \(E\).
\item
  If density is \(\rho(x,y,z)=x+y+z\), compute the mass of the unit
  cube.
\end{enumerate}

\subsection{9.3 Applications: Volume, Mass,
Probability}\label{applications-volume-mass-probability}

Triple integrals are powerful because they allow us to compute
quantities in three dimensions by accumulating values over a solid
region.

\subsubsection{Volume}\label{volume}

The simplest application is finding the volume of a region \(E\):

\[
V = \iiint_E 1 \, dV.
\]

Example: Find the volume of the solid bounded by the coordinate planes
and the plane \(x+y+z=1\).

\[
V = \iiint_E 1 \, dV = \int_0^1 \int_0^{1-x} \int_0^{1-x-y} 1 \, dz\, dy\, dx.
\]

L'évaluation donne \(V = \tfrac{1}{6}\).\#\#\# Masse et densité

Si un solide a une fonction de densité \(\rho(x,y,z)\), sa masse est

\[
M = \iiint_E \rho(x,y,z)\, dV.
\]

Le centre de masse est donné par

\[
\bar{x} = \frac{1}{M}\iiint_E x\rho(x,y,z)\,dV, \quad
\bar{y} = \frac{1}{M}\iiint_E y\rho(x,y,z)\,dV, \quad
\bar{z} = \frac{1}{M}\iiint_E z\rho(x,y,z)\,dV.
\]

Exemple~: Pour un cube unitaire de densité constante \(\rho=1\), le
centre de masse est à \((0.5,0.5,0.5)\).

\subsubsection{Probabilité}\label{probabilituxe9}

Si \(f(x,y,z)\) est une fonction de densité de probabilité en 3D, alors
la probabilité que la variable aléatoire se trouve dans une région \(E\)
est

\[
P(E) = \iiint_E f(x,y,z)\, dV,
\]

où \(f(x,y,z) \geq 0\) et

\[
\iiint_{\mathbb{R}^3} f(x,y,z)\,dV = 1.
\]

Exemple~: Si \(f(x,y,z) = \tfrac{3}{4}z^2\) pour \(0 \leq z \leq 1\),
uniformément dans \(x,y\), alors

\[
P(0 \leq z \leq 0.5) = \int_0^{0.5} \tfrac{3}{4}z^2 \, dz = \tfrac{1}{32}.
\]

\subsubsection{Pourquoi c'est
important}\label{pourquoi-cest-important-16}

\begin{itemize}
\tightlist
\item
  Les volumes généralisent la géométrie aux solides irréguliers.
\item
  Les intégrales de masse et de densité relient le calcul à la physique
  et à l'ingénierie.
\item
  Les fonctions de densité de probabilité en dimensions supérieures sont
  largement utilisées en statistique et en science des données.
\end{itemize}

\subsubsection{Exercices}\label{exercices-33}

\begin{enumerate}
\def\labelenumi{\arabic{enumi}.}
\tightlist
\item
  Trouvez le volume du solide délimité par \(x^2+y^2+z^2 \leq 1\) (la
  sphère unité).
\item
  Calculez la masse d'un cône de densité proportionnelle à \(z\).
\item
  Trouvez le centre de masse d'un tétraèdre uniforme délimité par
  \(x=0, y=0, z=0, x+y+z=1\).
\item
  Si \(f(x,y,z) = \frac{1}{8}\) sur le cube
  \([0,2]\times[0,2]\times[0,2]\), vérifiez qu'il s'agit d'une fonction
  de densité de probabilité.
\item
  Utilisez une intégrale triple pour calculer la probabilité qu'un point
  choisi au hasard dans la sphère unitaire ait \(z > 0\).
\end{enumerate}

\subsection{9.4 Changement de variables : coordonnées polaires,
cylindriques,
sphériques}\label{changement-de-variables-coordonnuxe9es-polaires-cylindriques-sphuxe9riques}

De nombreuses intégrales deviennent plus faciles lorsqu'elles sont
exprimées dans des systèmes de coordonnées qui correspondent à la
symétrie de la région. Au lieu des coordonnées cartésiennes \((x,y,z)\),
nous pouvons utiliser des coordonnées polaires, cylindriques ou
sphériques.

\subsubsection{Coordonnées polaires
(2D)}\label{coordonnuxe9es-polaires-2d}

Pour les fonctions à deux variables, on peut passer aux coordonnées
polaires :

\[
x = r\cos\theta, \quad y = r\sin\theta, \quad r \geq 0, \; 0 \leq \theta < 2\pi.
\]

L'élément de zone se transforme comme

\[
dA = r\,dr\,d\theta.
\]

Exemple~: Trouvez l'aire du cercle unité.

\[
A = \iint_{x^2+y^2\leq 1} 1\,dA = \int_0^{2\pi}\int_0^1 r\,dr\,d\theta = \pi.
\]

\subsubsection{Coordonnées cylindriques
(3D)}\label{coordonnuxe9es-cylindriques-3d}

En 3D, les coordonnées cylindriques prolongent les coordonnées polaires
avec \(z\)~:

\[
x = r\cos\theta, \quad y = r\sin\theta, \quad z = z.
\]

L'élément de volume est

\[
dV = r\,dr\,d\theta\,dz.
\]

Exemple~: Volume d'un cylindre de rayon \(R\) et de hauteur \(h\)~:

\[
V = \int_0^h \int_0^{2\pi} \int_0^R r\,dr\,d\theta\,dz = \pi R^2 h.
\]\#\#\# Coordonnées sphériques (3D)

Pour la symétrie sphérique, utilisez~:

\[
x = \rho \sin\phi \cos\theta, \quad y = \rho \sin\phi \sin\theta, \quad z = \rho \cos\phi,
\]

où

\begin{itemize}
\tightlist
\item
  \(\rho \geq 0\) est la distance à l'origine,
\item
  \(0 \leq \phi \leq \pi\) est l'angle par rapport à l'axe positif
  \(z\),
\item
  \(0 \leq \theta < 2\pi\) est l'angle dans le plan \(xy\).
\end{itemize}

L'élément de volume est

\[
dV = \rho^2 \sin\phi \, d\rho\, d\phi\, d\theta.
\]

Exemple~: Volume de la sphère unitaire :

\[
V = \int_0^{2\pi} \int_0^\pi \int_0^1 \rho^2 \sin\phi \, d\rho\, d\phi\, d\theta.
\]

Évaluation~:

\[
V = \left(\int_0^1 \rho^2 d\rho\right)\left(\int_0^\pi \sin\phi d\phi\right)\left(\int_0^{2\pi} d\theta\right) = \tfrac{1}{3}(2)(2\pi) = \tfrac{4\pi}{3}.
\]

\subsubsection{Pourquoi c'est
important}\label{pourquoi-cest-important-17}

\begin{itemize}
\tightlist
\item
  Les coordonnées polaires simplifient les régions circulaires.
\item
  Les coordonnées cylindriques gèrent les cylindres et la symétrie de
  rotation.
\item
  Les coordonnées sphériques simplifient les problèmes de sphères, de
  cônes et de rayons.
\item
  Ces changements de variables rendent gérables des intégrales autrement
  impossibles.
\end{itemize}

\subsubsection{Exercices}\label{exercices-34}

\begin{enumerate}
\def\labelenumi{\arabic{enumi}.}
\tightlist
\item
  Calculez \(\iint_{x^2+y^2\leq 4} (x^2+y^2)\,dA\) en utilisant les
  coordonnées polaires.
\item
  Trouvez le volume d'un cône de hauteur \(h\) et de rayon \(R\) en
  utilisant les coordonnées cylindriques.
\item
  Utilisez des coordonnées sphériques pour évaluer le volume d'une boule
  de rayon \(R\).
\item
  Montrez que le facteur jacobien pour les coordonnées polaires est
  \(r\).
\item
  Trouvez la masse d'une sphère solide de rayon \(R\) avec une densité
  proportionnelle à la distance de l'origine à l'aide de coordonnées
  sphériques.
\end{enumerate}

\section{Chapitre 10. Calcul
vectoriel}\label{chapitre-10.-calcul-vectoriel}

\subsection{10.1 Champs vectoriels}\label{champs-vectoriels}

Un champ vectoriel attribue un vecteur à chaque point de l'espace, un
peu comme une fonction scalaire attribue un nombre. Les champs
vectoriels sont utilisés pour modéliser les flux, les forces et d'autres
quantités directionnelles.

\subsubsection{Définition}\label{duxe9finition-10}

En deux dimensions, un champ vectoriel est une fonction

\[
\mathbf{F}(x,y) = \langle P(x,y), Q(x,y) \rangle,
\]

où \(P\) et \(Q\) sont des fonctions scalaires.

En trois dimensions,

\[
\mathbf{F}(x,y,z) = \langle P(x,y,z), Q(x,y,z), R(x,y,z) \rangle.
\]

\subsubsection{Exemples}\label{exemples-20}

\begin{enumerate}
\def\labelenumi{\arabic{enumi}.}
\tightlist
\item
  Champ radial
\end{enumerate}

\[
\mathbf{F}(x,y) = \langle x, y \rangle.
\]

Les vecteurs pointent vers l'extérieur depuis l'origine.

\begin{enumerate}
\def\labelenumi{\arabic{enumi}.}
\setcounter{enumi}{1}
\tightlist
\item
  Champ de rotation
\end{enumerate}

\[
\mathbf{F}(x,y) = \langle -y, x \rangle.
\]

Les vecteurs circulent autour de l'origine.

\begin{enumerate}
\def\labelenumi{\arabic{enumi}.}
\setcounter{enumi}{2}
\tightlist
\item
  Champ gravitationnel
\end{enumerate}

\[
\mathbf{F}(x,y,z) = -\frac{GM}{r^3}\langle x,y,z \rangle, \quad r=\sqrt{x^2+y^2+z^2}.
\]

\subsubsection{Visualisation des champs
vectoriels}\label{visualisation-des-champs-vectoriels}

\begin{itemize}
\tightlist
\item
  Dessinez de petites flèches sur des points d'échantillonnage pour
  indiquer la direction et l'ampleur.
\item
  Des flèches plus denses où les magnitudes sont plus grandes.
\item
  Utile pour interpréter les lignes d'écoulement, les trajectoires et
  les forces.
\end{itemize}

\subsubsection{\texorpdfstring{Lignes de fluxUne ligne de flux (ou
courbe intégrale) d'un champ vectoriel est une courbe \(\mathbf{r}(t)\)
dont le vecteur tangent en chaque point correspond au champ
:}{Lignes de fluxUne ligne de flux (ou courbe intégrale) d'un champ vectoriel est une courbe \textbackslash mathbf\{r\}(t) dont le vecteur tangent en chaque point correspond au champ :}}\label{lignes-de-fluxune-ligne-de-flux-ou-courbe-intuxe9grale-dun-champ-vectoriel-est-une-courbe-mathbfrt-dont-le-vecteur-tangent-en-chaque-point-correspond-au-champ}

\[
\mathbf{r}'(t) = \mathbf{F}(\mathbf{r}(t)).
\]

Les lignes de flux décrivent les chemins des particules dans un champ de
vitesse.

\subsubsection{Pourquoi c'est
important}\label{pourquoi-cest-important-18}

\begin{itemize}
\tightlist
\item
  Les champs vectoriels sont fondamentaux en physique (écoulement des
  fluides, électromagnétisme, gravitation).
\item
  Ils constituent la base des intégrales de lignes, des intégrales de
  surfaces et des grands théorèmes du calcul vectoriel (Green, Stokes,
  Divergence).
\item
  Fournir une manière géométrique de représenter les quantités
  directionnelles.
\end{itemize}

\subsubsection{Exercices}\label{exercices-35}

\begin{enumerate}
\def\labelenumi{\arabic{enumi}.}
\tightlist
\item
  Esquissez le champ vectoriel
  \(\mathbf{F}(x,y) = \langle y, -x \rangle\).
\item
  Déterminez si les vecteurs de
  \(\mathbf{F}(x,y) = \langle x,y \rangle\) pointent vers ou loin de
  l'origine.
\item
  Pour \(\mathbf{F}(x,y,z) = \langle y, z, x \rangle\), calculez
  \(\mathbf{F}(1,2,3)\).
\item
  Décrivez les lignes de flux de
  \(\mathbf{F}(x,y) = \langle -y, x \rangle\).
\item
  Expliquez pourquoi les champs gravitationnels et électriques sont des
  exemples de champs vectoriels radiaux.
\end{enumerate}

\subsection{10.2 Intégrales de ligne}\label{intuxe9grales-de-ligne}

Une intégrale linéaire étend l'idée d'une intégrale aux fonctions
évaluées le long d'une courbe. Au lieu d'intégrer sur un intervalle ou
une région, nous intégrons sur un chemin dans l'espace.

\subsubsection{Définition~: Intégrale de ligne
scalaire}\label{duxe9finition-intuxe9grale-de-ligne-scalaire}

Si \(f(x,y)\) est une fonction scalaire et \(C\) est une courbe
paramétrée par
\(\mathbf{r}(t) = \langle x(t), y(t) \rangle, \; a \leq t \leq b\),
alors l'intégrale de ligne est

\[
\int_C f(x,y)\, ds = \int_a^b f(x(t),y(t)) \, |\mathbf{r}'(t)|\, dt,
\]

où \(ds\) est la longueur de l'arc.

Cela mesure l'accumulation de \(f\) le long de la courbe.

\subsubsection{Définition~: Intégrale de ligne
vectorielle}\label{duxe9finition-intuxe9grale-de-ligne-vectorielle}

Pour un champ vectoriel
\(\mathbf{F}(x,y) = \langle P(x,y), Q(x,y) \rangle\), l'intégrale de
ligne le long de \(C\) est

\[
\int_C \mathbf{F} \cdot d\mathbf{r} = \int_a^b \mathbf{F}(\mathbf{r}(t)) \cdot \mathbf{r}'(t)\, dt.
\]

Celui-ci mesure le travail effectué par le champ le long de la courbe.

\subsubsection{Exemples}\label{exemples-21}

\begin{enumerate}
\def\labelenumi{\arabic{enumi}.}
\tightlist
\item
  Intégrale de ligne scalaire
\end{enumerate}

\[
f(x,y) = x+y, \quad C: \mathbf{r}(t) = \langle t, t^2 \rangle, \; 0 \leq t \leq 1.
\]

Puis

\[
\int_C f(x,y)\, ds = \int_0^1 (t+t^2)\sqrt{(1)^2+(2t)^2}\, dt.
\]

\begin{enumerate}
\def\labelenumi{\arabic{enumi}.}
\setcounter{enumi}{1}
\tightlist
\item
  Travail effectué par une force
\end{enumerate}

\[
\mathbf{F}(x,y) = \langle y, x \rangle, \quad C: \mathbf{r}(t) = \langle t, t^2 \rangle, \; 0 \leq t \leq 1.
\]

\[
\int_C \mathbf{F} \cdot d\mathbf{r} = \int_0^1 \langle t^2, t \rangle \cdot \langle 1, 2t \rangle\, dt = \int_0^1 (t^2 + 2t^2)\, dt = \int_0^1 3t^2\, dt = 1.
\]

\subsubsection{Interprétation
physique}\label{interpruxe9tation-physique}

\begin{itemize}
\tightlist
\item
  Intégrale de ligne scalaire : accumulation de densité le long d'un
  fil.
\item
  Intégrale de ligne vectorielle : travail effectué par une force
  déplaçant un objet le long d'un chemin.
\end{itemize}

\subsubsection{Pourquoi c'est important- Les intégrales de lignes
relient les champs vectoriels à des grandeurs physiques comme le travail
et la
circulation.}\label{pourquoi-cest-important--les-intuxe9grales-de-lignes-relient-les-champs-vectoriels-uxe0-des-grandeurs-physiques-comme-le-travail-et-la-circulation.}

\begin{itemize}
\tightlist
\item
  Ce sont des éléments constitutifs du théorème de Green et du théorème
  de Stokes.
\item
  Apparaître en physique (potentiel électrique, écoulement des fluides,
  mécanique).
\end{itemize}

\subsubsection{Exercices}\label{exercices-36}

\begin{enumerate}
\def\labelenumi{\arabic{enumi}.}
\tightlist
\item
  Calculez \(\int_C (x^2+y^2)\, ds\) où \(C\) est le segment de ligne de
  (0,0) à (1,1).
\item
  Évaluez \(\int_C \mathbf{F}\cdot d\mathbf{r}\) pour
  \(\mathbf{F}(x,y) = \langle -y, x \rangle\) le long du cercle unité
  \(x^2+y^2=1\).
\item
  Interprétez la signification de \(\int_C 1\,ds\).
\item
  Pour \(\mathbf{F}(x,y,z) = \langle z,0,x \rangle\), calculez
  l'intégrale de ligne le long de
  \(\mathbf{r}(t) = \langle t,t,1 \rangle, 0 \leq t \leq 1\).
\item
  Expliquez la différence entre les intégrales scalaires et
  vectorielles.
\end{enumerate}

\subsection{10.3 Intégrales de surface}\label{intuxe9grales-de-surface}

Une intégrale de surface généralise les intégrales de lignes aux
surfaces bidimensionnelles dans un espace tridimensionnel. Ils nous
permettent de calculer le flux à travers des surfaces et l'accumulation
de champs scalaires sur des surfaces courbes.

\subsubsection{Intégrale de surface
scalaire}\label{intuxe9grale-de-surface-scalaire}

Si une surface \(S\) est paramétrée par

\[
\mathbf{r}(u,v) = \langle x(u,v), y(u,v), z(u,v) \rangle,
\]

alors l'intégrale de surface d'une fonction scalaire \(f(x,y,z)\) est

\[
\iint_S f(x,y,z)\, dS = \iint_D f(\mathbf{r}(u,v)) \, |\mathbf{r}_u \times \mathbf{r}_v| \, du\,dv,
\]

où \(\mathbf{r}_u\) et \(\mathbf{r}_v\) sont des dérivées partielles de
\(\mathbf{r}(u,v)\) et \(D\) est le domaine des paramètres.

\subsubsection{Intégrale de surface vectorielle
(Flux)}\label{intuxe9grale-de-surface-vectorielle-flux}

Pour un champ vectoriel \(\mathbf{F}(x,y,z)\), le flux à travers une
surface \(S\) est

\[
\iint_S \mathbf{F}\cdot d\mathbf{S} = \iint_S \mathbf{F}\cdot \mathbf{n}\, dS,
\]

où \(\mathbf{n}\) est le vecteur normal unitaire. En utilisant le
paramétrage,

\[
\iint_S \mathbf{F}\cdot d\mathbf{S} = \iint_D \mathbf{F}(\mathbf{r}(u,v)) \cdot (\mathbf{r}_u \times \mathbf{r}_v)\,du\,dv.
\]

\subsubsection{Exemples}\label{exemples-22}

\begin{enumerate}
\def\labelenumi{\arabic{enumi}.}
\tightlist
\item
  Intégrale de surface scalaire Surface~: planez \(z=1\) sur le disque
  unitaire \(x^2+y^2 \leq 1\).
\end{enumerate}

\[
\iint_S 1\, dS = \text{area of the disk} = \pi.
\]

\begin{enumerate}
\def\labelenumi{\arabic{enumi}.}
\setcounter{enumi}{1}
\tightlist
\item
  Flux à travers une sphère Soit
  \(\mathbf{F}(x,y,z) = \langle x,y,z \rangle\), et \(S\) = sphère de
  rayon \(R\). Le vecteur normal est
  \(\mathbf{n} = \frac{1}{R}\langle x,y,z \rangle\).
\end{enumerate}

\[
\mathbf{F}\cdot \mathbf{n} = \frac{x^2+y^2+z^2}{R} = R.
\]

Alors

\[
\iint_S \mathbf{F}\cdot d\mathbf{S} = \iint_S R\, dS = R \cdot 4\pi R^2 = 4\pi R^3.
\]

\subsubsection{Pourquoi c'est
important}\label{pourquoi-cest-important-19}

\begin{itemize}
\tightlist
\item
  Les intégrales de surface scalaires mesurent les distributions de
  surface et de surface.
\item
  Les intégrales de surface vectorielles mesurent le flux : la quantité
  d'un champ traversant une surface.
\item
  Applications : électromagnétisme, écoulement de fluide, transfert de
  chaleur, etc.
\end{itemize}

\subsubsection{Exercices}\label{exercices-37}

\begin{enumerate}
\def\labelenumi{\arabic{enumi}.}
\tightlist
\item
  Calculez \(\iint_S 1\, dS\) pour la surface d'un cube de longueur de
  côté 2.2. Trouvez le flux de
  \(\mathbf{F}(x,y,z) = \langle x,y,z \rangle\) à travers la sphère
  unitaire.
\item
  Évaluez \(\iint_S z\, dS\) pour le paraboloïde
  \(z = x^2+y^2, \, z \leq 1\).
\item
  Pour \(\mathbf{F}(x,y,z) = \langle y,0,0 \rangle\), calculez le flux à
  travers le plan \(x=1\), \(0 \leq y,z \leq 1\).
\item
  Expliquez physiquement ce que cela signifie si le flux d'un champ
  vectoriel à travers une surface fermée est nul.
\end{enumerate}

\subsection{10.4 Théorème de Green}\label{thuxe9oruxe8me-de-green}

Le théorème de Green est un résultat fondamental du calcul vectoriel qui
relie une intégrale droite autour d'une courbe fermée à une intégrale
double sur la région qu'elle entoure. Il s'agit d'une version
bidimensionnelle du théorème de Stokes.

\subsubsection{Énoncé du théorème de
Green}\label{uxe9noncuxe9-du-thuxe9oruxe8me-de-green}

Soit \(C\) une courbe fermée, simple et orientée positivement dans le
plan, et soit \(R\) la région qu'elle englobe. Si
\(\mathbf{F}(x,y) = \langle P(x,y), Q(x,y) \rangle\) a des dérivées
partielles continues sur une région ouverte contenant \(R\), alors

\[
\oint_C \mathbf{F} \cdot d\mathbf{r} = \oint_C P\,dx + Q\,dy = \iint_R \left( \frac{\partial Q}{\partial x} - \frac{\partial P}{\partial y} \right)\, dA.
\]

\subsubsection{Interprétation}\label{interpruxe9tation-2}

\begin{itemize}
\tightlist
\item
  La droite intégrale autour de \(C\) mesure la circulation du champ
  vectoriel le long de la frontière.
\item
  La double intégrale sur \(R\) mesure la boucle totale (rotation) du
  champ à l'intérieur de la région.
\end{itemize}

\subsubsection{Exemple 1~: Formule de
surface}\label{exemple-1-formule-de-surface}

Si \(\mathbf{F} = \langle -y/2, x/2 \rangle\), alors

\[
\frac{\partial Q}{\partial x} - \frac{\partial P}{\partial y} = 1.
\]

Ainsi, le théorème de Green donne

\[
\text{Area}(R) = \iint_R 1\,dA = \oint_C \left(-\tfrac{y}{2}\,dx + \tfrac{x}{2}\,dy\right).
\]

Cela fournit un moyen de calculer l'aire à l'aide d'une intégrale de
ligne.

\subsubsection{Exemple 2~:~Circulation}\label{exemple-2-circulation}

Soit \(\mathbf{F}(x,y) = \langle -y, x \rangle\) et \(C\) le cercle
unitaire.

\begin{itemize}
\tightlist
\item
  \(P=-y, Q=x\).
\item
  \(Q_x - P_y = 1 - (-1) = 2\).
\item
  Double intégrale sur le disque unité :
\end{itemize}

\[
\iint_R 2\,dA = 2\pi (1^2) = 2\pi.
\]

La circulation autour du cercle est donc \(2\pi\).

\subsubsection{Pourquoi c'est
important}\label{pourquoi-cest-important-20}

\begin{itemize}
\tightlist
\item
  Convertit les intégrales de lignes difficiles en intégrales doubles,
  ou vice versa.
\item
  Fournit un pont entre les propriétés locales (curl) et les propriétés
  globales (circulation).
\item
  Largement utilisé en physique pour l'écoulement des fluides,
  l'électromagnétisme et les champs vectoriels planaires.
\end{itemize}

\subsubsection{Exercices}\label{exercices-38}

\begin{enumerate}
\def\labelenumi{\arabic{enumi}.}
\tightlist
\item
  Utilisez le théorème de Green pour calculer l'aire à l'intérieur de
  l'ellipse \(\frac{x^2}{a^2} + \frac{y^2}{b^2} = 1\).
\item
  Vérifiez le théorème de Green pour
  \(\mathbf{F}(x,y) = \langle -y, x \rangle\) le long du carré de
  sommets (0,0), (1,0), (1,1), (0,1).
\item
  Calculez la circulation de \(\mathbf{F}(x,y) = \langle -y, x \rangle\)
  autour du cercle unité.4. Montrez que si
  \(\nabla \times \mathbf{F} = 0\), alors l'intégrale de droite de
  \(\mathbf{F}\) autour de toute courbe fermée est nulle.
\item
  Utilisez le théorème de Green pour montrer que
\end{enumerate}

\[
\oint_C x\,dy = -\oint_C y\,dx
\]

pour toute courbe fermée \(C\).

\subsection{10.5 Théorème de Stokes}\label{thuxe9oruxe8me-de-stokes}

Le théorème de Stokes généralise le théorème de Green à trois
dimensions. Il relie une intégrale de surface de la boucle d'un champ
vectoriel sur une surface à une intégrale de ligne du champ vectoriel
autour de la limite de cette surface.

\subsubsection{Énoncé du théorème de
Stokes}\label{uxe9noncuxe9-du-thuxe9oruxe8me-de-stokes}

Soit \(S\) une surface lisse et orientée avec une courbe limite \(C\)
(orientée positivement). Si \(\mathbf{F}(x,y,z)\) est un champ vectoriel
à dérivées partielles continues, alors

\[
\iint_S (\nabla \times \mathbf{F}) \cdot d\mathbf{S} = \oint_C \mathbf{F} \cdot d\mathbf{r}.
\]

\begin{itemize}
\tightlist
\item
  Côté gauche : flux de la boucle de \(\mathbf{F}\) à travers la
  surface.
\item
  Côté droit : circulation de \(\mathbf{F}\) le long de la courbe
  frontière.
\end{itemize}

\subsubsection{Interprétation}\label{interpruxe9tation-3}

\begin{itemize}
\tightlist
\item
  La ligne intégrale autour de la frontière est égale à la ``rotation''
  totale à l'intérieur de la surface.
\item
  Étend le théorème de Green (cas particulier où la surface est dans le
  plan).
\end{itemize}

\subsubsection{Exemple 1~: Le théorème de Green comme cas
particulier}\label{exemple-1-le-thuxe9oruxe8me-de-green-comme-cas-particulier}

Si \(S\) est une région plate dans le plan \(xy\), le théorème de Stokes
se réduit au théorème de Green.

\subsubsection{Exemple 2 : Circulation sur un
hémisphère}\label{exemple-2-circulation-sur-un-huxe9misphuxe8re}

Soit \(\mathbf{F}(x,y,z) = \langle -y, x, 0 \rangle\) et \(S\)
l'hémisphère supérieur de rayon 1.

\begin{itemize}
\tightlist
\item
  Limite \(C\) : cercle unité dans le plan \(xy\).
\item
  Par le théorème de Stokes~:
\end{itemize}

\[
\oint_C \mathbf{F}\cdot d\mathbf{r} = \iint_S (\nabla \times \mathbf{F})\cdot d\mathbf{S}.
\]

\begin{itemize}
\tightlist
\item
  Boucle : \(\nabla \times \mathbf{F} = \langle 0,0,2 \rangle\).
\item
  La normale à l'hémisphère pointe vers l'extérieur~:
  \(\mathbf{n} = \langle 0,0,1 \rangle\).
\item
  Donc intégral = 2.
\item
  Superficie de l'hémisphère = \(2\pi (1^2)\).
\end{itemize}

\[
\iint_S 2\, dS = 2 \cdot 2\pi = 4\pi.
\]

Ainsi, la circulation autour de l'équateur est \(4\pi\).

\subsubsection{Pourquoi c'est
important}\label{pourquoi-cest-important-21}

\begin{itemize}
\tightlist
\item
  Fournit une connexion profonde entre les intégrales de surface et les
  intégrales de ligne.
\item
  Simplifie les calculs en permettant le choix de surfaces pratiques.
\item
  Largement utilisé en électromagnétisme (loi de Faraday) et en
  dynamique des fluides.
\end{itemize}

\subsubsection{Exercices}\label{exercices-39}

\begin{enumerate}
\def\labelenumi{\arabic{enumi}.}
\tightlist
\item
  Vérifiez le théorème de Stokes pour
  \(\mathbf{F}(x,y,z) = \langle y, -x, 0 \rangle\) sur le disque unité
  dans le plan \(xy\).
\item
  Calculez \(\oint_C \mathbf{F}\cdot d\mathbf{r}\) où
  \(\mathbf{F}(x,y,z) = \langle z, 0, x \rangle\) et \(C\) est la limite
  du triangle avec les sommets (0,0,0), (1,0,0), (0,1,0).
\item
  Montrez que si \(\nabla \times \mathbf{F} = 0\), alors la circulation
  autour de toute courbe fermée est nulle.4. Appliquez le théorème de
  Stokes pour calculer la circulation de
  \(\mathbf{F}(x,y,z) = \langle -y, x, z \rangle\) autour de la limite
  du carré unité dans le plan \(z=0\).
\item
  Expliquez comment le théorème de Stokes généralise le théorème de
  Green.
\end{enumerate}

\subsection{10.6 Théorème de
divergence}\label{thuxe9oruxe8me-de-divergence}

Le théorème de divergence (également appelé théorème de Gauss) relie le
flux d'un champ vectoriel à travers une surface fermée à la triple
intégrale de la divergence du champ à l'intérieur de la surface.

\subsubsection{Énoncé du théorème de
divergence}\label{uxe9noncuxe9-du-thuxe9oruxe8me-de-divergence}

Soit \(E\) une région solide dans \(\mathbb{R}^3\) avec une surface
limite \(S\) (orientée vers l'extérieur). Si \(\mathbf{F}(x,y,z)\) est
un champ vectoriel à dérivées partielles continues sur \(E\), alors

\[
\iint_S \mathbf{F} \cdot d\mathbf{S} = \iiint_E (\nabla \cdot \mathbf{F}) \, dV.
\]

\begin{itemize}
\tightlist
\item
  Côté gauche : flux de \(\mathbf{F}\) à travers la surface fermée
  \(S\).
\item
  Côté droit : triple intégrale de la divergence à l'intérieur de la
  région.
\end{itemize}

\#\#\#Divergences

La divergence d'un champ vectoriel
\(\mathbf{F}(x,y,z) = \langle P, Q, R \rangle\) est

\[
\nabla \cdot \mathbf{F} = \frac{\partial P}{\partial x} + \frac{\partial Q}{\partial y} + \frac{\partial R}{\partial z}.
\]

Il mesure le « débit net » par unité de volume en chaque point.

\subsubsection{Exemple 1~: Flux d'un champ
radial}\label{exemple-1-flux-dun-champ-radial}

Soit \(\mathbf{F}(x,y,z) = \langle x, y, z \rangle\), et soit \(E\) la
boule unité \(x^2+y^2+z^2 \leq 1\).

-Divergence~: \(\nabla \cdot \mathbf{F} = 1+1+1 = 3\). - Volume de la
boule unitaire : \(\tfrac{4}{3}\pi\). Alors

\[
\iiint_E (\nabla \cdot \mathbf{F})\, dV = 3 \cdot \tfrac{4}{3}\pi = 4\pi.
\]

Ainsi, le flux à travers la sphère unitaire est \(4\pi\).

\subsubsection{Exemple 2~:~Champ
constant}\label{exemple-2-champ-constant}

Laissez \(\mathbf{F}(x,y,z) = \langle 1, 0, 0 \rangle\).

-Divergence~: \(\nabla \cdot \mathbf{F} = 0\). - Le flux à travers toute
surface fermée est donc nul, conformément à l'intuition (pas de sortie
nette).

\subsubsection{Pourquoi c'est
important}\label{pourquoi-cest-important-22}

\begin{itemize}
\item
  Convertit les intégrales de surface en intégrales de volume plus
  simples.
\item
  Utilisé en physique : Loi de Gauss en électromagnétisme, écoulement
  des fluides et transfert de chaleur.
\item
  Complète le cadre fédérateur :

  \begin{itemize}
  \tightlist
  \item
    Théorème de Green (boucle 2D ↔ circulation)
  \item
    Théorème de Stokes (boucle 3D ↔ circulation sur les surfaces)
  \item
    Théorème de Divergence (divergence 3D ↔ flux sur surfaces fermées)
  \end{itemize}
\end{itemize}

\subsubsection{Exercices}\label{exercices-40}

\begin{enumerate}
\def\labelenumi{\arabic{enumi}.}
\tightlist
\item
  Utilisez le théorème de divergence pour calculer le flux de
  \(\mathbf{F}(x,y,z) = \langle x,y,z \rangle\) à travers la surface
  d'une sphère de rayon \(R\).
\item
  Vérifiez le théorème de divergence pour
  \(\mathbf{F}(x,y,z) = \langle y, z, x \rangle\) sur le cube unité
  \([0,1]^3\).
\item
  Montrez que si \(\nabla \cdot \mathbf{F} = 0\), alors le flux total à
  travers toute surface fermée est nul.
\item
  Calculez le flux de
  \(\mathbf{F}(x,y,z) = \langle x^2, y^2, z^2 \rangle\) à travers la
  sphère unitaire.5. Expliquez comment le théorème de divergence
  généralise le théorème fondamental unidimensionnel du calcul.
\end{enumerate}

\section{Partie IV. Processus
infinis}\label{partie-iv.-processus-infinis}

\section{Chapitre 11. Séquences et
convergence}\label{chapitre-11.-suxe9quences-et-convergence}

\subsection{11.1 Définitions et
exemples}\label{duxe9finitions-et-exemples}

Une séquence est une liste ordonnée de nombres, généralement écrite sous
la forme

\[
a_1, a_2, a_3, \dots
\]

ou plus généralement \((a_n)_{n=1}^\infty\). Chaque \(a_n\) est appelé
le nième terme de la séquence.

\subsubsection{Définir une séquence}\label{duxe9finir-une-suxe9quence}

Une séquence peut être définie de deux manières~:

\begin{enumerate}
\def\labelenumi{\arabic{enumi}.}
\item
  Formule explicite -- donne une règle directe pour le nième terme.

  \begin{itemize}
  \item
    Exemple~: \(a_n = \frac{1}{n}\) définit la séquence

    \[
    1, \tfrac{1}{2}, \tfrac{1}{3}, \tfrac{1}{4}, \dots
    \]
  \end{itemize}
\item
  Définition récursive -- définit les termes en utilisant des termes
  antérieurs.

  \begin{itemize}
  \item
    Exemple : Suite de Fibonacci :

    \[
    a_1 = 1, \quad a_2 = 1, \quad a_{n} = a_{n-1} + a_{n-2} \quad (n \geq 3).
    \]
  \end{itemize}
\end{enumerate}

\subsubsection{Exemples de séquences}\label{exemples-de-suxe9quences}

\begin{enumerate}
\def\labelenumi{\arabic{enumi}.}
\item
  Séquence arithmétique :

  \[
  a_n = a_1 + (n-1)d.
  \]

  Exemple~: \(a_n = 2n+1\) → séquence de nombres impairs.
\item
  Séquence géométrique :

  \[
  a_n = a_1 r^{n-1}.
  \]

  Exemple~: \(a_n = 2^n\) → puissances de 2.
\item
  Séquence harmonique :

  \[
  a_n = \frac{1}{n}.
  \]
\item
  Séquence alternée~:

  \[
  a_n = (-1)^n.
  \]
\end{enumerate}

\subsubsection{Séquences en calcul}\label{suxe9quences-en-calcul}

Les séquences sont la base de processus infinis~:

\begin{itemize}
\tightlist
\item
  Limites des séquences → définir la convergence.
\item
  Séries → sommes infinies construites à partir de séquences.
\item
  Fonctions approximées par des séquences et des séries.
\end{itemize}

\subsubsection{Pourquoi c'est
important}\label{pourquoi-cest-important-23}

\begin{itemize}
\tightlist
\item
  Les séquences fournissent les éléments de base pour les séries et
  approximations infinies.
\item
  Ils permettent de définir rigoureusement « l'approche de l'infini » et
  la convergence.
\item
  De nombreuses fonctions importantes (exponentielles, trigonométriques)
  peuvent être exprimées à travers des séquences et des séries.
\end{itemize}

\subsubsection{Exercices}\label{exercices-41}

\begin{enumerate}
\def\labelenumi{\arabic{enumi}.}
\tightlist
\item
  Écrivez les cinq premiers termes de la séquence
  \(a_n = \frac{n}{n+1}\).
\item
  Déterminez si \(a_n = (-1)^n n\) est délimité.
\item
  Donnez une définition récursive de la séquence \(2,4,8,16,\dots\).
\item
  Trouvez le 10ème terme de la suite arithmétique avec \(a_1=3\) et
  \(d=5\).
\item
  Écrivez une formule explicite pour la séquence définie par \(a_1=1\),
  \(a_{n+1}=2a_n\).
\end{enumerate}

\subsection{11.2 Séquences monotones et
limitées}\label{suxe9quences-monotones-et-limituxe9es}

Pour comprendre si une séquence converge, nous devons étudier son
comportement : est-ce qu'elle augmente, diminue, reste dans des limites
ou croît sans limite ? Deux concepts importants sont la monotonie et la
limitation.

\subsubsection{Séquences monotones}\label{suxe9quences-monotones}

Une séquence \((a_n)\) est dite monotone si elle est toujours croissante
ou toujours décroissante.

\begin{itemize}
\item
  Monotone croissant :

  \[
  a_{n+1} \geq a_n \quad \text{for all } n.
  \]
\item
  Monotone décroissant :

  \[
  a_{n+1} \leq a_n \quad \text{for all } n.
  \]
\end{itemize}

Exemples~:1. \(a_n = n\) est monotone croissant. 2.
\(a_n = \frac{1}{n}\) est monotone décroissant.

\subsubsection{Séquences limitées}\label{suxe9quences-limituxe9es}

Une séquence est délimitée au-dessus s'il existe un nombre \(M\) tel que
\(a_n \leq M\) pour tout \(n\). Il est délimité ci-dessous s'il existe
\(m\) tel que \(a_n \geq m\) pour tout \(n\).

Si les deux conditions sont remplies, la suite est bornée.

Exemples~:

\begin{enumerate}
\def\labelenumi{\arabic{enumi}.}
\tightlist
\item
  \(a_n = \frac{1}{n}\) est limité entre 0 et 1.
\item
  \(a_n = (-1)^n\) est limité entre -1 et 1.
\item
  \(a_n = n\) n'est pas limité.
\end{enumerate}

\subsubsection{Théorème de convergence
monotone}\label{thuxe9oruxe8me-de-convergence-monotone}

Un résultat fondamental en analyse :

\begin{itemize}
\tightlist
\item
  Toute séquence croissante monotone délimitée ci-dessus converge.
\item
  Toute séquence décroissante monotone délimitée en dessous converge.
\end{itemize}

Ce théorème garantit la convergence sans trouver explicitement la
limite.

\subsubsection{Exemple}\label{exemple-1}

\begin{enumerate}
\def\labelenumi{\arabic{enumi}.}
\item
  Séquence~: \(a_n = 1 - \frac{1}{n}\).

  \begin{itemize}
  \tightlist
  \item
    En augmentation~: depuis le
    \(a_{n+1} - a_n = \frac{1}{n} - \frac{1}{n+1} > 0\).
  \item
    Borné au dessus par 1.
  \item
    Par conséquent, il converge.
  \item
    Limite~: \(\lim_{n\to\infty} a_n = 1\).
  \end{itemize}
\end{enumerate}

\subsubsection{Pourquoi c'est
important}\label{pourquoi-cest-important-24}

\begin{itemize}
\tightlist
\item
  La monotonie et la limitation donnent des tests rapides de
  convergence.
\item
  Ils sont essentiels dans les preuves et dans la construction
  rigoureuse des limites.
\item
  Ces idées s'étendent naturellement aux fonctions et aux séries.
\end{itemize}

\subsubsection{Exercices}\label{exercices-42}

\begin{enumerate}
\def\labelenumi{\arabic{enumi}.}
\tightlist
\item
  Déterminez si \(a_n = \frac{n}{n+1}\) est monotone et délimité.
\item
  Montrer que \(a_n = \sqrt{n}\) est monotone croissant mais non borné.
\item
  Montrer que \(a_n = 2 - \frac{1}{n}\) converge et trouver sa limite.
\item
  Donnez un exemple de séquence délimitée qui n'est pas monotone.
\item
  Appliquez le théorème de convergence monotone à
  \(a_n = \ln\!\big(1+\frac{1}{n}\big)\).
\end{enumerate}

\subsection{11.3 Limites des séquences}\label{limites-des-suxe9quences}

La question centrale concernant une séquence est de savoir si ses termes
s'approchent d'une valeur unique à mesure que \(n\) grandit. Cela nous
amène à la notion de limite d'une séquence.

\subsubsection{Définition}\label{duxe9finition-11}

Une séquence \((a_n)\) a une limite \(L\) si, pour tout
\(\varepsilon > 0\), il existe un entier \(N\) tel que

\[
|a_n - L| < \varepsilon \quad \text{whenever } n > N.
\]

On écrit alors

\[
\lim_{n\to\infty} a_n = L.
\]

Si aucun \(L\) n'existe, la séquence diverge.

\#\#\#Intuition

\begin{itemize}
\tightlist
\item
  Les termes de la séquence se rapprochent arbitrairement de \(L\) à
  mesure que \(n\) devient grand.
\item
  Au-delà d'un certain index \(N\), tous les termes restent dans une
  toute petite bande autour de \(L\).
\end{itemize}

\subsubsection{Exemples}\label{exemples-23}

\begin{enumerate}
\def\labelenumi{\arabic{enumi}.}
\item
  \(a_n = \frac{1}{n}\). À mesure que \(n\) grandit, les termes
  diminuent vers 0.

  \[
  \lim_{n\to\infty} \frac{1}{n} = 0.
  \]
\item
  \(a_n = (-1)^n\). Les termes alternent entre -1 et 1, il n'existe donc
  pas de limite unique. La séquence diverge.
\item
  \(a_n = \frac{n}{n+1}\). Comme \(n \to \infty\), le numérateur et le
  dénominateur sont presque égaux, donc

  \[
  \lim_{n\to\infty} \frac{n}{n+1} = 1.
  \]
\end{enumerate}

\subsubsection{\texorpdfstring{Propriétés des limitesSi \(\lim a_n = A\)
et
\(\lim b_n = B\)~:}{Propriétés des limitesSi \textbackslash lim a\_n = A et \textbackslash lim b\_n = B~:}}\label{propriuxe9tuxe9s-des-limitessi-lim-a_n-a-et-lim-b_n-b}

\begin{itemize}
\item
  \(\lim (a_n+b_n) = A+B\).
\item
  \(\lim (a_n b_n) = AB\).
\item
  \(\lim (c a_n) = cA\) pour la constante \(c\).
\item
  Si \(b_n \neq 0\) et \(B \neq 0\), alors

  \[
  \lim \frac{a_n}{b_n} = \frac{A}{B}.
  \]
\end{itemize}

\subsubsection{Théorème~: principe de
compression}\label{thuxe9oruxe8me-principe-de-compression}

Si \(a_n \leq b_n \leq c_n\) pour tous les grands \(n\), et

\[
\lim_{n\to\infty} a_n = \lim_{n\to\infty} c_n = L,
\]

alors

\[
\lim_{n\to\infty} b_n = L.
\]

Exemple~:

\[
a_n = -\tfrac{1}{n}, \quad b_n = \tfrac{\sin n}{n}, \quad c_n = \tfrac{1}{n}.
\]

Puisque \(-\tfrac{1}{n} \leq \tfrac{\sin n}{n} \leq \tfrac{1}{n}\) et
les deux séquences englobantes vont à 0,

\[
\lim_{n\to\infty} \frac{\sin n}{n} = 0.
\]

\subsubsection{Pourquoi c'est
important}\label{pourquoi-cest-important-25}

\begin{itemize}
\tightlist
\item
  Les limites rendent rigoureuse l'idée de séquences « se rapprochant »
  d'une valeur.
\item
  La convergence des séquences sous-tend les séries infinies et la
  continuité.
\item
  Ces concepts sont essentiels pour définir des nombres réels via des
  limites.
\end{itemize}

\subsubsection{Exercices}\label{exercices-43}

\begin{enumerate}
\def\labelenumi{\arabic{enumi}.}
\tightlist
\item
  Recherchez \(\lim_{n\to\infty} \frac{2n+1}{3n+4}\).
\item
  Déterminez si \(a_n = \sqrt{n+1} - \sqrt{n}\) converge.
\item
  Est-ce que \(a_n = \cos n\) converge~? Pourquoi ou pourquoi pas ?
\item
  Utilisez le principe de compression pour afficher
  \(\lim_{n\to\infty} \frac{\sin n}{n} = 0\).
\item
  Prouvez que si \(\lim a_n = L\), alors \(\lim |a_n| = |L|\).
\end{enumerate}

\section{Chapitre 12. Séries
infinies}\label{chapitre-12.-suxe9ries-infinies}

\subsection{Série 12.1 et
convergence}\label{suxe9rie-12.1-et-convergence}

Une série est la somme des termes d'une séquence. Au lieu de simplement
énumérer des nombres, nous les additionnons et étudions si la somme
infinie s'approche d'une valeur finie.

\subsubsection{Définition}\label{duxe9finition-12}

Étant donné une séquence \((a_n)\), la série correspondante est

\[
\sum_{n=1}^\infty a_n = a_1 + a_2 + a_3 + \dots
\]

Nous définissons la nième somme partielle comme

\[
S_n = \sum_{k=1}^n a_k.
\]

Si la suite \((S_n)\) converge vers une limite finie \(S\), alors la
série converge et

\[
\sum_{n=1}^\infty a_n = S.
\]

Si \((S_n)\) diverge, alors la série diverge.

\subsubsection{Exemples}\label{exemples-24}

\begin{enumerate}
\def\labelenumi{\arabic{enumi}.}
\tightlist
\item
  Série géométrique
\end{enumerate}

\[
\sum_{n=0}^\infty ar^n = \frac{a}{1-r}, \quad |r| < 1.
\]

Exemple~:

\[
1 + \tfrac{1}{2} + \tfrac{1}{4} + \tfrac{1}{8} + \dots = 2.
\]

\begin{enumerate}
\def\labelenumi{\arabic{enumi}.}
\setcounter{enumi}{1}
\tightlist
\item
  Série harmonique
\end{enumerate}

\[
\sum_{n=1}^\infty \frac{1}{n}.
\]

Cette série diverge, même si les termes tendent vers 0.

\begin{enumerate}
\def\labelenumi{\arabic{enumi}.}
\setcounter{enumi}{2}
\tightlist
\item
  série p
\end{enumerate}

\[
\sum_{n=1}^\infty \frac{1}{n^p}.
\]

\begin{itemize}
\tightlist
\item
  Converge si \(p > 1\).
\item
  Diverge si \(p \leq 1\).
\end{itemize}

\subsubsection{Condition nécessaire à la
convergence}\label{condition-nuxe9cessaire-uxe0-la-convergence}

Si \(\sum a_n\) converge, alors nécessairement

\[
\lim_{n\to\infty} a_n = 0.
\]

Si \(\lim a_n \neq 0\), la série diverge. Mais l'inverse n'est pas
vrai~: \(\lim a_n = 0\) ne garantit pas la convergence (par exemple, les
séries harmoniques).

\subsubsection{Pourquoi c'est
important}\label{pourquoi-cest-important-26}

\begin{itemize}
\tightlist
\item
  Les séries étendent l'addition finie à des processus infinis.
\item
  Les séries convergentes sont utilisées pour approximer des fonctions,
  calculer des zones et modéliser des processus physiques.- L'étude des
  séries conduit à des tests de convergence puissants.
\end{itemize}

\subsubsection{Exercices}\label{exercices-44}

\begin{enumerate}
\def\labelenumi{\arabic{enumi}.}
\tightlist
\item
  Déterminez si \(\sum_{n=1}^\infty \frac{2}{3^n}\) converge et trouvez
  sa somme.
\item
  Montrez que \(\sum_{n=1}^\infty \frac{1}{n^2}\) converge.
\item
  Est-ce que \(\sum_{n=1}^\infty \frac{1}{\sqrt{n}}\) converge~?
\item
  Écrivez les quatre premières sommes partielles de la série
  \(\sum_{n=1}^\infty \frac{1}{2^n}\).
\item
  Expliquez pourquoi \(\lim a_n = 0\) est nécessaire mais pas suffisant
  pour la convergence.
\end{enumerate}

\subsection{12.2 Tests de convergence}\label{tests-de-convergence}

Comme de nombreuses séries ne peuvent pas être additionnées directement,
les mathématiciens ont développé des tests pour décider si une série
converge ou diverge. Ces tests sont des outils d'analyse de sommes
infinies.

\subsubsection{1. Le test de divergence au nième
terme}\label{le-test-de-divergence-au-niuxe8me-terme}

Si

\[
\lim_{n\to\infty} a_n \neq 0 \quad \text{or does not exist},
\]

alors

\[
\sum a_n
\]

diverge.

Si \(\lim a_n = 0\), le test n'est pas concluant.

\subsubsection{2. Test de comparaison}\label{test-de-comparaison}

Supposons \(0 \leq a_n \leq b_n\) pour tous les \(n\).

\begin{itemize}
\tightlist
\item
  Si \(\sum b_n\) converge, alors \(\sum a_n\) converge également.
\item
  Si \(\sum a_n\) diverge, alors \(\sum b_n\) diverge également.
\end{itemize}

\subsubsection{3. Test de comparaison de
limites}\label{test-de-comparaison-de-limites}

Si \(a_n, b_n > 0\) et

\[
\lim_{n\to\infty} \frac{a_n}{b_n} = c,
\]

où \(0 < c < \infty\), puis \(\sum a_n\) et \(\sum b_n\) convergent tous
les deux ou divergent tous les deux.

\subsubsection{4. Test de rapport}\label{test-de-rapport}

Pour \(\sum a_n\), calculez

\[
L = \lim_{n\to\infty} \left| \frac{a_{n+1}}{a_n} \right|.
\]

\begin{itemize}
\tightlist
\item
  Si \(L < 1\), la série converge absolument.
\item
  Si \(L > 1\) ou \(L = \infty\), la série diverge.
\item
  Si \(L = 1\), le test n'est pas concluant.
\end{itemize}

\subsubsection{5. Test racine}\label{test-racine}

Pour \(\sum a_n\), calculez

\[
L = \lim_{n\to\infty} \sqrt[n]{|a_n|}.
\]

\begin{itemize}
\tightlist
\item
  Si \(L < 1\), la série converge absolument.
\item
  Si \(L > 1\), la série diverge.
\item
  Si \(L = 1\), le test n'est pas concluant.
\end{itemize}

\subsubsection{6. Test des séries alternées (test de
Leibniz)}\label{test-des-suxe9ries-alternuxe9es-test-de-leibniz}

Pour les séries de la forme

\[
\sum (-1)^n b_n \quad \text{or} \quad \sum (-1)^{n+1} b_n,
\]

si

\begin{enumerate}
\def\labelenumi{\arabic{enumi}.}
\tightlist
\item
  \(b_{n+1} \leq b_n\) (décroissant), et
\item
  \(\lim_{n\to\infty} b_n = 0\),
\end{enumerate}

alors la série converge.

\subsubsection{Exemples}\label{exemples-25}

\begin{enumerate}
\def\labelenumi{\arabic{enumi}.}
\tightlist
\item
  \(\sum \frac{1}{n^2}\) : Test de comparaison → converge.
\item
  \(\sum \frac{1}{n}\) : Série harmonique → diverge.
\item
  \(\sum \frac{(-1)^n}{n}\) : Test en série alternée → converge.
\item
  \(\sum \frac{n!}{n^n}\) : Test de ratio → converge.
\item
  \(\sum \frac{2^n}{n}\)~: Test racine → diverge.
\end{enumerate}

\subsubsection{Pourquoi c'est
important}\label{pourquoi-cest-important-27}

\begin{itemize}
\tightlist
\item
  Les tests de convergence permettent de classer des séries sans avoir
  besoin de sommes explicites.
\item
  Ils fournissent des moyens systématiques de gérer des processus
  infinis en calcul.
\item
  Ils sont essentiels pour des sujets ultérieurs comme les séries de
  puissances et les séries de Fourier.
\end{itemize}

\subsubsection{Exercices}\label{exercices-45}

\begin{enumerate}
\def\labelenumi{\arabic{enumi}.}
\tightlist
\item
  Testez la convergence de \(\sum \frac{1}{n^3}\).
\item
  Utilisez le test de ratio pour \(\sum \frac{3^n}{n!}\).3. Appliquez le
  test racine à \(\sum \left(\frac{1}{2}\right)^n\).
\item
  Déterminez la convergence de \(\sum \frac{(-1)^n}{\sqrt{n}}\).
\item
  Utilisez le test de comparaison de limites avec \(\frac{1}{n^2}\) pour
  tester \(\sum \frac{1}{n^2+1}\).
\end{enumerate}

\subsection{12.3 Convergence absolue ou
conditionnelle}\label{convergence-absolue-ou-conditionnelle}

Toutes les séries ne se comportent pas de la même manière lorsque les
signes alternent. Pour gérer cela, nous distinguons la convergence
absolue et la convergence conditionnelle.

\subsubsection{Convergence absolue}\label{convergence-absolue}

Une série \(\sum a_n\) est absolument convergente si

\[
\sum |a_n|
\]

converge.

Théorème : Si une série converge absolument, alors elle converge aussi.

Exemple~:

\[
\sum \frac{(-1)^n}{n^2}.
\]

Ici, \(\sum \left|\frac{(-1)^n}{n^2}\right| = \sum \frac{1}{n^2}\)
converge (série p, \(p=2\)). La série est donc absolument convergente.

\subsubsection{Convergence
conditionnelle}\label{convergence-conditionnelle}

Une série \(\sum a_n\) est conditionnellement convergente si elle
converge, mais pas absolument.

Exemple~:

\[
\sum \frac{(-1)^n}{n}.
\]

\begin{itemize}
\tightlist
\item
  Test en série alternée → converge.
\item
  Mais \(\sum \left|\frac{(-1)^n}{n}\right| = \sum \frac{1}{n}\) diverge
  (série harmonique). La série est donc conditionnellement convergente.
\end{itemize}

\subsubsection{Théorème de
réarrangement}\label{thuxe9oruxe8me-de-ruxe9arrangement}

Pour les séries conditionnellement convergentes, la réorganisation des
termes peut modifier la somme, voire la faire diverger ou converger vers
une valeur différente.

Ce résultat surprenant montre la nature délicate de la convergence
conditionnelle.

\subsubsection{Pourquoi c'est
important}\label{pourquoi-cest-important-28}

\begin{itemize}
\tightlist
\item
  La convergence absolue est plus forte et garantit la stabilité.
\item
  La convergence conditionnelle met en évidence l'importance de l'ordre
  dans les sommes infinies.
\item
  De nombreuses séries alternées rencontrées en pratique ne convergent
  que conditionnellement.
\end{itemize}

\subsubsection{Exercices}\label{exercices-46}

\begin{enumerate}
\def\labelenumi{\arabic{enumi}.}
\tightlist
\item
  Montrer que \(\sum \frac{(-1)^n}{n^3}\) converge absolument.
\item
  Montrez que \(\sum \frac{(-1)^n}{n}\) est conditionnellement
  convergent.
\item
  Testez \(\sum \frac{(-1)^n}{\sqrt{n}}\) pour la convergence absolue et
  conditionnelle.
\item
  Expliquez pourquoi la convergence absolue implique la convergence,
  mais l'inverse n'est pas vrai.
\item
  Recherchez et résumez le théorème du réarrangement de Riemann dans vos
  propres mots.
\end{enumerate}

\section{Chapitre 13. Séries Power et
extensions}\label{chapitre-13.-suxe9ries-power-et-extensions}

\subsection{13.1 Série de puissance}\label{suxe9rie-de-puissance}

Une série entière est une série infinie dans laquelle chaque terme
implique une puissance de la variable. Les séries entières sont
centrales en calcul car elles nous permettent de représenter des
fonctions comme des polynômes infinis.

\subsubsection{Formulaire général}\label{formulaire-guxe9nuxe9ral}

Une série entière centrée sur \(a\) a la forme

\[
\sum_{n=0}^\infty c_n (x-a)^n,
\]

où \(c_n\) sont des constantes appelées coefficients.

\begin{itemize}
\item
  Si \(a=0\), la série est centrée à l'origine :

  \[
  \sum_{n=0}^\infty c_n x^n.
  \]
\end{itemize}

\subsubsection{Exemples}\label{exemples-26}

\begin{enumerate}
\def\labelenumi{\arabic{enumi}.}
\tightlist
\item
  Série géométrique
\end{enumerate}

\[
\sum_{n=0}^\infty x^n = \frac{1}{1-x}, \quad |x|<1.
\]

\begin{enumerate}
\def\labelenumi{\arabic{enumi}.}
\setcounter{enumi}{1}
\tightlist
\item
  Fonction exponentielle
\end{enumerate}

\[e^x = \sum_{n=0}^\infty \frac{x^n}{n!}.
\]

\begin{enumerate}
\def\labelenumi{\arabic{enumi}.}
\setcounter{enumi}{2}
\tightlist
\item
  Sine and cosine
\end{enumerate}

\[
\sin x = \sum_{n=0}^\infty (-1)^n \frac{x^{2n+1}}{(2n+1)!}, \quad  
\cos x = \sum_{n=0}^\infty (-1)^n \frac{x^{2n}}{(2n)!}.
\]

\subsubsection{Interval of Convergence}\label{interval-of-convergence}

For each power series, there exists a radius of convergence \(R\) such
that:

\begin{itemize}
\tightlist
\item
  The series converges if \(|x-a| < R\).
\item
  The series diverges if \(|x-a| > R\).
\item
  At \(|x-a| = R\), convergence must be checked separately.
\end{itemize}

\subsubsection{Why This Matters}\label{why-this-matters-3}

\begin{itemize}
\tightlist
\item
  Power series allow us to approximate functions by polynomials.
\item
  They connect calculus with analysis and differential equations.
\item
  Many special functions in mathematics and physics are defined by their
  power series.
\end{itemize}

\subsubsection{Exercises}\label{exercises-6}

\begin{enumerate}
\def\labelenumi{\arabic{enumi}.}
\tightlist
\item
  Write the power series for \(\sum_{n=0}^\infty \frac{(x-2)^n}{n!}\).
\item
  Find the first four terms of the power series for \(e^x\).
\item
  Express \(\frac{1}{1+x}\) as a power series centered at 0.
\item
  Determine whether the series \(\sum_{n=0}^\infty n! x^n\) converges at
  \(x=0.1\).
\item
  Explain why power series are sometimes called ``infinite
  polynomials.''
\end{enumerate}

\subsection{13.2 Radius of Convergence}\label{radius-of-convergence}

Every power series converges for some values of \(x\) and diverges for
others. The boundary between these two behaviors is described by the
radius of convergence.

\subsubsection{Definition}\label{definition}

For a power series

\[
\sum_{n=0}^\infty c_n (x-a)^n,
\]

there exists a number \(R \geq 0\) (possibly infinite) such that:

\begin{itemize}
\tightlist
\item
  The series converges absolutely if \(|x-a| < R\).
\item
  The series diverges if \(|x-a| > R\).
\item
  At \(|x-a| = R\), convergence must be checked separately.
\end{itemize}

This number \(R\) is called the radius of convergence.

\subsubsection{Finding the Radius of
Convergence}\label{finding-the-radius-of-convergence}

Two common methods:

\begin{enumerate}
\def\labelenumi{\arabic{enumi}.}
\tightlist
\item
  Ratio Test
\end{enumerate}

\[
R = \lim_{n\to\infty} \left| \frac{c_n}{c_{n+1}} \right|.
\]

\begin{enumerate}
\def\labelenumi{\arabic{enumi}.}
\setcounter{enumi}{1}
\tightlist
\item
  Root Test
\end{enumerate}

\[
R = \frac{1}{\limsup_{n\to\infty} \sqrt[n]{|c_n|}}.
\]

\subsubsection{Examples}\label{examples-4}

\begin{enumerate}
\def\labelenumi{\arabic{enumi}.}
\tightlist
\item
  Series:
\end{enumerate}

\[
\sum_{n=0}^\infty \frac{x^n}{n!}.
\]

Using ratio test:

\[
\lim_{n\to\infty} \frac{1/(n!)}{1/((n+1)!)} = \infty.
\]

So \(R = \infty\) (converges for all real \(x\)).

\begin{enumerate}
\def\labelenumi{\arabic{enumi}.}
\setcounter{enumi}{1}
\tightlist
\item
  Series:
\end{enumerate}

\[
\sum_{n=0}^\infty x^n.
\]

Here \(c_n = 1\).

\[
R = 1.
\]

Converges for \(|x| < 1\).

\begin{enumerate}
\def\labelenumi{\arabic{enumi}.}
\setcounter{enumi}{2}
\tightlist
\item
  Series:
\end{enumerate}

\[
\sum_{n=1}^\infty \frac{x^n}{n}.
\]

Ratio test:

\[
\lim_{n\to\infty} \left|\frac{(x^{n+1}/(n+1))}{(x^n/n)}\right| = |x|.
\]

Donc \(R = 1\). Converge pour \(|x| < 1\), diverge pour \(|x| > 1\). À
\(x=\pm 1\), testez séparément.

\subsubsection{Intervalle de
convergence}\label{intervalle-de-convergence}

L'ensemble des valeurs \(x\) où la série converge est appelé
l'intervalle de convergence.

\begin{itemize}
\tightlist
\item
  Toujours centré sur \(a\).
\item
  Étend les unités \(R\) dans les deux sens.
\item
  Les points de terminaison \(x=a\pm R\) doivent être vérifiés
  individuellement.
\end{itemize}

\subsubsection{Pourquoi c'est important- Le rayon de convergence nous
indique où les séries entières se comportent comme des
fonctions.}\label{pourquoi-cest-important--le-rayon-de-convergence-nous-indique-ouxf9-les-suxe9ries-entiuxe8res-se-comportent-comme-des-fonctions.}

\begin{itemize}
\tightlist
\item
  Indispensable pour utiliser les extensions de la série Taylor dans la
  pratique.
\item
  Détermine le domaine de validité des solutions séries en physique et
  en ingénierie.
\end{itemize}

\subsubsection{Exercices}\label{exercices-47}

\begin{enumerate}
\def\labelenumi{\arabic{enumi}.}
\tightlist
\item
  Trouvez le rayon de convergence de
  \(\sum_{n=0}^\infty \frac{(x-3)^n}{n!}\).
\item
  Calculez le rayon de convergence de
  \(\sum_{n=1}^\infty \frac{x^n}{n^2}\).
\item
  Utilisez le test de ratio pour trouver \(R\) pour
  \(\sum_{n=0}^\infty n!x^n\).
\item
  Déterminez l'intervalle de convergence pour
  \(\sum_{n=1}^\infty \frac{(x+1)^n}{n}\).
\item
  Expliquez pourquoi la série exponentielle converge pour tous les
  \(x\), alors que la série géométrique ne converge que pour \(|x|<1\).
\end{enumerate}

\subsection{13.3 Série Taylor et
Maclaurin}\label{suxe9rie-taylor-et-maclaurin}

Les séries de puissance deviennent particulièrement puissantes
lorsqu'elles sont utilisées pour représenter des fonctions familières.
Cela se fait via la série de Taylor, et le cas particulier centré en 0
est appelé série de Maclaurin.

\subsubsection{Série Taylor}\label{suxe9rie-taylor}

Si une fonction \(f(x)\) est infiniment différentiable en \(x=a\), sa
série de Taylor sur \(a\) est

\[
f(x) = \sum_{n=0}^\infty \frac{f^{(n)}(a)}{n!}(x-a)^n.
\]

Ici, \(f^{(n)}(a)\) désigne le \(n\)-ème dérivé de \(f\) à \(a\).

\subsubsection{Série Maclaurin}\label{suxe9rie-maclaurin}

Une série de Taylor centrée sur \(a=0\)~:

\[
f(x) = \sum_{n=0}^\infty \frac{f^{(n)}(0)}{n!} x^n.
\]

\subsubsection{Exemples}\label{exemples-27}

\begin{enumerate}
\def\labelenumi{\arabic{enumi}.}
\tightlist
\item
  Fonction exponentielle
\end{enumerate}

\[
e^x = 1 + x + \frac{x^2}{2!} + \frac{x^3}{3!} + \cdots
\]

\begin{enumerate}
\def\labelenumi{\arabic{enumi}.}
\setcounter{enumi}{1}
\tightlist
\item
  Sinus et cosinus
\end{enumerate}

\[
\sin x = x - \frac{x^3}{3!} + \frac{x^5}{5!} - \cdots
\]

\[
\cos x = 1 - \frac{x^2}{2!} + \frac{x^4}{4!} - \cdots
\]

\begin{enumerate}
\def\labelenumi{\arabic{enumi}.}
\setcounter{enumi}{2}
\tightlist
\item
  Logarithme népérien (pour \(|x|<1\))
\end{enumerate}

\[
\ln(1+x) = x - \frac{x^2}{2} + \frac{x^3}{3} - \frac{x^4}{4} + \cdots
\]

\subsubsection{Approximation polynomiale de
Taylor}\label{approximation-polynomiale-de-taylor}

La somme finie des premiers termes \(n\) est le polynôme de Taylor de
degré \(n\)~:

\[
P_n(x) = \sum_{k=0}^n \frac{f^{(k)}(a)}{k!}(x-a)^k.
\]

Ce polynôme se rapproche de \(f(x)\) près de \(x=a\).

\subsubsection{Reste (terme d'erreur)}\label{reste-terme-derreur}

La différence entre la fonction et son polynôme de Taylor est le reste :

\[
R_n(x) = f(x) - P_n(x).
\]

Une forme (la forme de Lagrange) est

\[
R_n(x) = \frac{f^{(n+1)}(c)}{(n+1)!}(x-a)^{n+1},
\]

pour certains \(c\) entre \(a\) et \(x\).

\subsubsection{Pourquoi c'est
important}\label{pourquoi-cest-important-29}

\begin{itemize}
\tightlist
\item
  Les séries de Taylor fournissent des approximations polynomiales de
  fonctions compliquées.
\item
  Ils sont essentiels en analyse numérique, en physique et en
  ingénierie.
\item
  Les extensions de la série Maclaurin donnent des formules simples pour
  les fonctions exponentielles, trigonométriques et logarithmiques.
\end{itemize}

\subsubsection{Exercices}\label{exercices-48}

\begin{enumerate}
\def\labelenumi{\arabic{enumi}.}
\tightlist
\item
  Recherchez la série Maclaurin pour
  \(f(x)=\cosh x = \tfrac{e^x+e^{-x}}{2}\).
\item
  Écrivez la série de Taylor pour \(f(x)=e^x\) centrée sur \(a=2\).
\item
  Calculez le polynôme de Taylor de degré 3 pour \(f(x)=\ln(1+x)\) à
  \(a=0\).4. Utilisez la série Maclaurin pour \(\sin x\) pour vous
  rapprocher de \(\sin(0.1)\).
\item
  Expliquez pourquoi les séries de Taylor fournissent souvent de bonnes
  approximations locales mais peuvent diverger pour de grands \(|x|\).
\end{enumerate}

\subsection{13.4 Applications de la série
Taylor}\label{applications-de-la-suxe9rie-taylor}

Les séries de Taylor ne sont pas seulement des outils théoriques : elles
sont utilisées pour approximer des fonctions, résoudre des équations et
analyser des systèmes physiques. Leurs applications couvrent les
mathématiques, les sciences et l'ingénierie.

\subsubsection{Rapprochement de la
fonction}\label{rapprochement-de-la-fonction}

Les fonctions compliquées peuvent être approchées par des polynômes
proches d'un point.

Exemple~: approximez \(e^x\) près de \(x=0\) en utilisant le polynôme de
Maclaurin de degré 3~:

\[
P_3(x) = 1 + x + \tfrac{x^2}{2} + \tfrac{x^3}{6}.
\]

Pour les petits \(x\), cela donne des estimations précises de \(e^x\).

\subsubsection{Méthodes numériques}\label{muxe9thodes-numuxe9riques}

Les séries de Taylor constituent la base des algorithmes numériques~:

\begin{itemize}
\tightlist
\item
  Rapprochement des racines carrées, des logarithmes et des valeurs
  trigonométriques.
\item
  Estimation de l'erreur à travers le terme restant.
\item
  Utilisé dans les méthodes itératives comme la méthode de Newton (où la
  linéarisation locale provient de la série de Taylor).
\end{itemize}

\subsubsection{Résolution d'équations
différentielles}\label{ruxe9solution-duxe9quations-diffuxe9rentielles}

De nombreuses équations différentielles ont des solutions exprimées sous
forme de série de Taylor (ou puissance).

Exemple~: La solution de \(y'' + y = 0\) avec \(y(0)=0, y'(0)=1\) est
\(\sin x\), qui découle naturellement de sa série Maclaurin.

\subsubsection{Physique et ingénierie}\label{physique-et-inguxe9nierie}

\begin{itemize}
\item
  approximation aux petits angles :

  \[
  \sin x \approx x, \quad \cos x \approx 1 - \tfrac{x^2}{2}, \quad |x| \ll 1.
  \]

  Utilisé dans le mouvement pendulaire, l'optique et la mécanique
  ondulatoire.
\item
  Relativité et mécanique quantique : les expansions de Taylor
  simplifient les expressions non linéaires pour une utilisation
  pratique.
\item
  Approximation des fonctions énergétiques : En mécanique, les fonctions
  énergétiques potentielles sont développées à proximité des points
  d'équilibre.
\end{itemize}

\subsubsection{Probabilités et
statistiques}\label{probabilituxe9s-et-statistiques}

\begin{itemize}
\tightlist
\item
  Les fonctions génératrices de moments et les fonctions
  caractéristiques utilisent des séries de puissances.
\item
  Les approximations des distributions de probabilité (par exemple,
  approximation normale du binôme) utilisent les développements de
  Taylor.
\end{itemize}

\subsubsection{Pourquoi c'est
important}\label{pourquoi-cest-important-30}

\begin{itemize}
\tightlist
\item
  Les séries de Taylor constituent un pont entre les formules exactes et
  le calcul pratique.
\item
  Ils nous permettent de réduire des problèmes complexes à des
  approximations polynomiales gérables.
\item
  Les applications en font l'un des outils les plus importants en
  mathématiques appliquées.
\end{itemize}

\subsubsection{Exercices}\label{exercices-49}

\begin{enumerate}
\def\labelenumi{\arabic{enumi}.}
\tightlist
\item
  Utilisez la série Maclaurin pour \(e^x\) pour approximer \(e^{0.1}\)
  jusqu'à quatre décimales.
\item
  Appliquez l'approximation aux petits angles pour estimer
  \(\sin(5^\circ)\).
\item
  Résolvez l'équation différentielle \(y'' = -y\) en utilisant une
  approche en séries entières.
\item
  Développez \(\ln(1+x)\) jusqu'au 4ème degré et utilisez-le pour vous
  rapprocher de \(\ln(1.1)\).
\item
  Expliquez pourquoi les approximations polynomiales sont
  particulièrement utiles pour les ordinateurs et les calculatrices.\#
  Annexes
\end{enumerate}

\subsection{Annexe A. Éléments essentiels du
pré-calcul}\label{annexe-a.-uxe9luxe9ments-essentiels-du-pruxe9-calcul}

\subsubsection{A.1 Actualisation de
l'algèbre}\label{a.1-actualisation-de-lalguxe8bre}

Avant de plonger dans le calcul, il est utile de revoir certaines
compétences en algèbre qui apparaîtront encore et encore. Ce sont les «
outils » dont vous aurez besoin pour manipuler des expressions, résoudre
des équations et simplifier les résultats.

\paragraph{Exposants et puissances}\label{exposants-et-puissances}

\begin{itemize}
\item
  Règles de base :

  \[
  a^m \cdot a^n = a^{m+n}, \quad \frac{a^m}{a^n} = a^{m-n}, \quad (a^m)^n = a^{mn}.
  \]
\item
  Exposants négatifs :

  \[
  a^{-n} = \frac{1}{a^n}, \quad a \neq 0.
  \]
\item
  Exposants fractionnaires :

  \[
  a^{1/n} = \sqrt[n]{a}, \quad a^{m/n} = \sqrt[n]{a^m}.
  \]
\end{itemize}

\paragraph{Affacturage}\label{affacturage}

La factorisation simplifie les expressions et aide à résoudre les
équations.

\begin{enumerate}
\def\labelenumi{\arabic{enumi}.}
\item
  Facteur commun~:

  \[
  6x^2+9x = 3x(2x+3).
  \]
\item
  Différence des carrés :

  \[
  a^2-b^2 = (a-b)(a+b).
  \]
\item
  Trinômes quadratiques :

  \[
  x^2+5x+6 = (x+2)(x+3).
  \]
\end{enumerate}

\paragraph{Polynômes}\label{polynuxf4mes}

\begin{itemize}
\tightlist
\item
  Formulaire standard :
  \(P(x) = a_nx^n + a_{n-1}x^{n-1} + \cdots + a_0\).
\item
  Degré~: la plus grande puissance de \(x\).
\item
  La division longue et la division synthétique sont utiles pour
  simplifier les fonctions rationnelles.
\end{itemize}

\paragraph{Expressions rationnelles}\label{expressions-rationnelles}

Simplifiez en factorisant le numérateur et le dénominateur~:

\[
\frac{x^2-1}{x^2-2x+1} = \frac{(x-1)(x+1)}{(x-1)^2} = \frac{x+1}{x-1}, \quad x \neq 1.
\]

\paragraph{Logarithmes}\label{logarithmes}

\begin{itemize}
\item
  Définition : \(\log_a b = c\) signifie \(a^c = b\).
\item
  Bases communes : log naturel (\(\ln x = \log_e x\)) et base 10
  (\(\log x\)).
\item
  Règles~:

  \[
  \log(ab) = \log a + \log b, \quad \log\left(\frac{a}{b}\right) = \log a - \log b, \quad \log(a^n) = n\log a.
  \]
\end{itemize}

\paragraph{Équations}\label{uxe9quations}

\begin{itemize}
\item
  Linéaire~: résolvez \(ax+b=0\) → \(x=-b/a\).
\item
  Quadratique~: \(ax^2+bx+c=0\) a des solutions

  \[
  x=\frac{-b\pm \sqrt{b^2-4ac}}{2a}.
  \]
\item
  Exponentiel~: \(e^x = k\) → \(x = \ln k\).
\end{itemize}

\subsubsection{A.2 Bases de la
trigonométrie}\label{a.2-bases-de-la-trigonomuxe9trie}

La trigonométrie fournit le langage des angles et des phénomènes
périodiques. Étant donné que le calcul traite souvent des oscillations,
du mouvement et des ondes, une solide compréhension des fonctions
trigonométriques et de leurs propriétés est essentielle.

\paragraph{Le cercle unitaire}\label{le-cercle-unitaire}

\begin{itemize}
\item
  Défini comme le cercle de rayon 1 centré à l'origine dans le plan de
  coordonnées.
\item
  Pour un angle \(\theta\) mesuré à partir de l'axe \(x\) positif~:

  \[
  (\cos \theta, \sin \theta)
  \]

  donne les coordonnées du point sur le cercle.
\end{itemize}

Valeurs spéciales~:

\begin{longtable}[]{@{}
  >{\raggedright\arraybackslash}p{(\linewidth - 6\tabcolsep) * \real{0.3333}}
  >{\raggedright\arraybackslash}p{(\linewidth - 6\tabcolsep) * \real{0.1667}}
  >{\raggedright\arraybackslash}p{(\linewidth - 6\tabcolsep) * \real{0.1667}}
  >{\raggedright\arraybackslash}p{(\linewidth - 6\tabcolsep) * \real{0.3333}}@{}}
\toprule\noalign{}
\begin{minipage}[b]{\linewidth}\raggedright
\(\theta\)
\end{minipage} & \begin{minipage}[b]{\linewidth}\raggedright
\(\sin \theta\)
\end{minipage} & \begin{minipage}[b]{\linewidth}\raggedright
\(\cos \theta\)
\end{minipage} & \begin{minipage}[b]{\linewidth}\raggedright
\(\tan \theta = \frac{\sin \theta}{\cos \theta}\)
\end{minipage} \\
\midrule\noalign{}
\endhead
\bottomrule\noalign{}
\endlastfoot
\(0\) & 0 & 1 & 0 \\
\(\pi/6\) & 1/2 & \(\sqrt{3}/2\) & \(1/\sqrt{3}\) \\
\(\pi/3\) & \(\sqrt{3}/2\) & 1/2 & \(\sqrt{3}\) \\
\(\pi/2\) & 1 & 0 & indéfini \\
\end{longtable}

\paragraph{Identités fondamentales}\label{identituxe9s-fondamentales}

\begin{enumerate}
\def\labelenumi{\arabic{enumi}.}
\tightlist
\item
  Identité pythagoricienne
\end{enumerate}

\[
\sin^2\theta + \cos^2\theta = 1.
\]

\begin{enumerate}
\def\labelenumi{\arabic{enumi}.}
\setcounter{enumi}{1}
\tightlist
\item
  Identités quotientes
\end{enumerate}

\[
\tan\theta = \frac{\sin\theta}{\cos\theta}, \quad \cot\theta = \frac{\cos\theta}{\sin\theta}.
\]

\begin{enumerate}
\def\labelenumi{\arabic{enumi}.}
\setcounter{enumi}{2}
\tightlist
\item
  Identités réciproques
\end{enumerate}

\[
\sec\theta = \frac{1}{\cos\theta}, \quad \csc\theta = \frac{1}{\sin\theta}.
\]

\paragraph{Formules d'addition d'angle}\label{formules-daddition-dangle}

\[
\sin(\alpha+\beta) = \sin\alpha\cos\beta + \cos\alpha\sin\beta,
\]

\[
\cos(\alpha+\beta) = \cos\alpha\cos\beta - \sin\alpha\sin\beta.
\]

Cas particuliers :

\begin{itemize}
\item
  Double-angle :

  \[
  \sin(2\theta) = 2\sin\theta\cos\theta, \quad
  \cos(2\theta) = \cos^2\theta - \sin^2\theta.
  \]
\end{itemize}

\paragraph{Graphiques}\label{graphiques}

\begin{itemize}
\tightlist
\item
  \(\sin x\) : onde commençant à 0, amplitude 1, période \(2\pi\).
\item
  \(\cos x\) : onde commençant à 1, amplitude 1, période \(2\pi\).
\item
  \(\tan x\)~: se répète tous les \(\pi\), non défini aux multiples
  impairs de \(\pi/2\).
\end{itemize}

\subsubsection{A.3 Géométrie des
coordonnées}\label{a.3-guxe9omuxe9trie-des-coordonnuxe9es}

La géométrie des coordonnées relie l'algèbre et la géométrie en
décrivant des objets géométriques (lignes, cercles, courbes) à l'aide
d'équations. Le calcul s'appuie fortement sur ce cadre pour représenter
graphiquement les fonctions, trouver les pentes et analyser les courbes.

\paragraph{Le plan cartésien}\label{le-plan-cartuxe9sien}

\begin{itemize}
\item
  Un point est représenté par les coordonnées \((x,y)\).
\item
  Distance entre deux points \((x_1,y_1)\) et \((x_2,y_2)\) :

  \[
  d = \sqrt{(x_2-x_1)^2 + (y_2-y_1)^2}.
  \]
\item
  Milieu d'un segment de droite~:

  \[
  M = \left(\frac{x_1+x_2}{2}, \frac{y_1+y_2}{2}\right).
  \]
\end{itemize}

\paragraph{Lignes}\label{lignes}

\begin{enumerate}
\def\labelenumi{\arabic{enumi}.}
\item
  Formule de pente

  \[
  m = \frac{y_2-y_1}{x_2-x_1}.
  \]
\item
  Équation d'une droite

  \begin{itemize}
  \item
    Forme point-pente~:

    \[
    y-y_1 = m(x-x_1).
    \]
  \item
    Formulaire d'intersection de pente~:

    \[
    y = mx+b.
    \]
  \end{itemize}
\item
  Lignes parallèles et perpendiculaires

  \begin{itemize}
  \tightlist
  \item
    Lignes parallèles : même pente.
  \item
    Lignes perpendiculaires~: les pentes satisfont \(m_1m_2 = -1\).
  \end{itemize}
\end{enumerate}

\paragraph{Cercles}\label{cercles}

Équation d'un cercle de centre \((h,k)\) et de rayon \(r\)~:

\[
(x-h)^2+(y-k)^2 = r^2.
\]

Cas particulier : cercle unité centré à l'origine :

\[
x^2+y^2=1.
\]

\paragraph{Sections coniques}\label{sections-coniques}

\begin{enumerate}
\def\labelenumi{\arabic{enumi}.}
\item
  Parabole~:

  \begin{itemize}
  \item
    Forme standard (ouverture haut/bas) :

    \[
    y = ax^2+bx+c.
    \]
  \end{itemize}
\item
  Ellipse (centrée à l'origine)~:

  \[
  \frac{x^2}{a^2}+\frac{y^2}{b^2}=1.
  \]
\item
  Hyperbole (centrée à l'origine)~:

  \[
  \frac{x^2}{a^2}-\frac{y^2}{b^2}=1.
  \]
\end{enumerate}

\subsection{Annexe B. Formules et tableaux
clés}\label{annexe-b.-formules-et-tableaux-cluxe9s}

\subsubsection{B.1 Table dérivéeLes dérivés mesurent les taux de
changement et les pentes des fonctions. Disposer d'un tableau de
référence rapide aide les apprenants à éviter de recréer des formules à
chaque
fois.}\label{b.1-table-duxe9rivuxe9eles-duxe9rivuxe9s-mesurent-les-taux-de-changement-et-les-pentes-des-fonctions.-disposer-dun-tableau-de-ruxe9fuxe9rence-rapide-aide-les-apprenants-uxe0-uxe9viter-de-recruxe9er-des-formules-uxe0-chaque-fois.}

\paragraph{Règles de base}\label{ruxe8gles-de-base-1}

\begin{enumerate}
\def\labelenumi{\arabic{enumi}.}
\tightlist
\item
  Règle constante
\end{enumerate}

\[
\frac{d}{dx}[c] = 0
\]

\begin{enumerate}
\def\labelenumi{\arabic{enumi}.}
\setcounter{enumi}{1}
\tightlist
\item
  Règle de puissance
\end{enumerate}

\[
\frac{d}{dx}[x^n] = nx^{n-1}, \quad (n \in \mathbb{R})
\]

\begin{enumerate}
\def\labelenumi{\arabic{enumi}.}
\setcounter{enumi}{2}
\tightlist
\item
  Règle multiple constante
\end{enumerate}

\[
\frac{d}{dx}[c f(x)] = c f'(x)
\]

\begin{enumerate}
\def\labelenumi{\arabic{enumi}.}
\setcounter{enumi}{3}
\tightlist
\item
  Règle de somme et de différence
\end{enumerate}

\[
\frac{d}{dx}[f(x)\pm g(x)] = f'(x)\pm g'(x)
\]

\paragraph{Fonctions
trigonométriques}\label{fonctions-trigonomuxe9triques}

\[
\frac{d}{dx}[\sin x] = \cos x
\]

\[
\frac{d}{dx}[\cos x] = -\sin x
\]

\[
\frac{d}{dx}[\tan x] = \sec^2 x, \quad x \neq \tfrac{\pi}{2}+k\pi
\]

\[
\frac{d}{dx}[\cot x] = -\csc^2 x
\]

\[
\frac{d}{dx}[\sec x] = \sec x \tan x
\]

\[
\frac{d}{dx}[\csc x] = -\csc x \cot x
\]

\paragraph{Fonctions exponentielles et
logarithmiques}\label{fonctions-exponentielles-et-logarithmiques}

\[
\frac{d}{dx}[e^x] = e^x
\]

\[
\frac{d}{dx}[a^x] = a^x \ln a, \quad a>0, a\neq 1
\]

\[
\frac{d}{dx}[\ln x] = \frac{1}{x}, \quad x>0
\]

\[
\frac{d}{dx}[\log_a x] = \frac{1}{x\ln a}, \quad a>0, a\neq 1
\]

\paragraph{Fonctions trigonométriques
inverses}\label{fonctions-trigonomuxe9triques-inverses}

\[
\frac{d}{dx}[\arcsin x] = \frac{1}{\sqrt{1-x^2}}, \quad |x|<1
\]

\[
\frac{d}{dx}[\arccos x] = -\frac{1}{\sqrt{1-x^2}}, \quad |x|<1
\]

\[
\frac{d}{dx}[\arctan x] = \frac{1}{1+x^2}, \quad x \in \mathbb{R}
\]

\paragraph{Règles de produit, de quotient et de
chaîne}\label{ruxe8gles-de-produit-de-quotient-et-de-chauxeene}

\begin{enumerate}
\def\labelenumi{\arabic{enumi}.}
\tightlist
\item
  Règle du produit
\end{enumerate}

\[
\frac{d}{dx}[f(x)g(x)] = f'(x)g(x)+f(x)g'(x)
\]

\begin{enumerate}
\def\labelenumi{\arabic{enumi}.}
\setcounter{enumi}{1}
\tightlist
\item
  Règle du quotient
\end{enumerate}

\[
\frac{d}{dx}\left[\frac{f(x)}{g(x)}\right] = \frac{f'(x)g(x)-f(x)g'(x)}{[g(x)]^2}, \quad g(x)\neq 0
\]

\begin{enumerate}
\def\labelenumi{\arabic{enumi}.}
\setcounter{enumi}{2}
\tightlist
\item
  Règle de chaîne
\end{enumerate}

\[
\frac{d}{dx}[f(g(x))] = f'(g(x))\cdot g'(x)
\]

\subsubsection{B.3 Extensions de séries
communes}\label{b.3-extensions-de-suxe9ries-communes}

Les séries entières permettent d'exprimer les fonctions sous forme de
polynômes infinis. Ces expansions sont essentielles pour les
approximations, la résolution d'équations différentielles et la
construction d'une intuition sur les fonctions en calcul.

\paragraph{Série géométrique}\label{suxe9rie-guxe9omuxe9trique}

\[
\frac{1}{1-x} = \sum_{n=0}^\infty x^n, \quad |x| < 1
\]

\paragraph{Fonction exponentielle}\label{fonction-exponentielle}

\[
e^x = \sum_{n=0}^\infty \frac{x^n}{n!}
= 1 + x + \frac{x^2}{2!} + \frac{x^3}{3!} + \cdots
\]

Valable pour tous les \(x\).

\paragraph{Fonctions
trigonométriques}\label{fonctions-trigonomuxe9triques-1}

\[
\sin x = \sum_{n=0}^\infty (-1)^n \frac{x^{2n+1}}{(2n+1)!}
= x - \frac{x^3}{3!} + \frac{x^5}{5!} - \cdots
\]

\[
\cos x = \sum_{n=0}^\infty (-1)^n \frac{x^{2n}}{(2n)!}
= 1 - \frac{x^2}{2!} + \frac{x^4}{4!} - \cdots
\]

\[
\tan^{-1} x = \sum_{n=0}^\infty (-1)^n \frac{x^{2n+1}}{2n+1}, \quad |x|\leq 1
\]

\paragraph{Logarithme}\label{logarithme}

\[
\ln(1+x) = \sum_{n=1}^\infty (-1)^{n+1} \frac{x^n}{n}, \quad -1 < x \leq 1
\]

\paragraph{Expansion binomiale
(généralisée)}\label{expansion-binomiale-guxe9nuxe9ralisuxe9e}

\[
(1+x)^r = \sum_{n=0}^\infty \binom{r}{n} x^n, \quad |x|<1
\]

où

\[\binom{r}{n} = \frac{r(r-1)(r-2)\cdots(r-n+1)}{n!}.
\]

\subsection{Appendix C. Proof
Sketches}\label{appendix-c.-proof-sketches}

\subsubsection{\texorpdfstring{C.1 Limit Laws and the
\(\varepsilon\)--\(\delta\)
Definition}{C.1 Limit Laws and the \textbackslash varepsilon--\textbackslash delta Definition}}\label{c.1-limit-laws-and-the-varepsilondelta-definition}

Calculus rests on the precise meaning of a limit. While intuition
(``values get closer and closer'') is helpful, a formal definition
ensures rigor and avoids paradoxes.

\paragraph{Intuitive Idea}\label{intuitive-idea}

We write

\[
\lim_{x \to a} f(x) = L
\]

to mean that as \(x\) gets arbitrarily close to \(a\), the values of
\(f(x)\) get arbitrarily close to \(L\).

\paragraph{\texorpdfstring{Formal (\(\varepsilon\)--\(\delta\))
Definition}{Formal (\textbackslash varepsilon--\textbackslash delta) Definition}}\label{formal-varepsilondelta-definition}

We say that

\[
\lim_{x \to a} f(x) = L
\]

if for every \(\varepsilon > 0\), there exists a \(\delta > 0\) such
that whenever

\[
0 < |x-a| < \delta,
\]

we have

\[
|f(x) -L| < \varepsilon.
\]

\begin{itemize}
\tightlist
\item
  \(\varepsilon\): how close we want \(f(x)\) to be to \(L\).
\item
  \(\delta\): how close \(x\) must be to \(a\) to achieve that.
\end{itemize}

\paragraph{Example}\label{example}

Show that

\[
\lim_{x \to 2} (3x+1) = 7.
\]

\begin{itemize}
\tightlist
\item
  Let \(\varepsilon > 0\).
\item
  We want \(|(3x+1)-7| < \varepsilon\).
\item
  Simplify: \(|3x-6| = 3|x-2| < \varepsilon\).
\item
  This holds if we choose \(\delta = \varepsilon/3\).
\end{itemize}

Thus, by the definition, the limit is 7.

\paragraph{Limit Laws}\label{limit-laws}

If \(\lim_{x \to a} f(x) = L\) and \(\lim_{x \to a} g(x) = M\), then:

\begin{enumerate}
\def\labelenumi{\arabic{enumi}.}
\tightlist
\item
  Sum/Difference
\end{enumerate}

\[
\lim_{x \to a} [f(x) \pm g(x)] = L \pm M
\]

\begin{enumerate}
\def\labelenumi{\arabic{enumi}.}
\setcounter{enumi}{1}
\tightlist
\item
  Constant Multiple
\end{enumerate}

\[
\lim_{x \to a} [c f(x)] = cL
\]

\begin{enumerate}
\def\labelenumi{\arabic{enumi}.}
\setcounter{enumi}{2}
\tightlist
\item
  Product
\end{enumerate}

\[
\lim_{x \to a} [f(x)g(x)] = LM
\]

\begin{enumerate}
\def\labelenumi{\arabic{enumi}.}
\setcounter{enumi}{3}
\tightlist
\item
  Quotient (if \(M \neq 0\))
\end{enumerate}

\[
\lim_{x \to a} \frac{f(x)}{g(x)} = \frac{L}{M}
\]

\begin{enumerate}
\def\labelenumi{\arabic{enumi}.}
\setcounter{enumi}{4}
\tightlist
\item
  Powers and Roots
\end{enumerate}

\[
\lim_{x \to a} [f(x)]^n = L^n, \quad \lim_{x \to a} \sqrt[n]{f(x)} = \sqrt[n]{L} \ (\text{si défini}).
\]

\subsubsection{C.2 Proof Sketch: The Fundamental Theorem of
Calculus}\label{c.2-proof-sketch-the-fundamental-theorem-of-calculus}

The Fundamental Theorem of Calculus (FTC) links the two central
operations of calculus: differentiation and integration. It shows that
they are, in fact, inverse processes.

\paragraph{Statement of the Theorem}\label{statement-of-the-theorem}

Part I (Differentiation of an Integral): If \(f\) is continuous on
\([a,b]\) and we define

\[
F(x) = \int_a^x f(t)\,dt,
\]

then \(F\) is differentiable on \((a,b)\) and

\[
F'(x) = f(x).
\]

Part II (Evaluation of a Definite Integral): If \(F\) is any
antiderivative of \(f\) on \([a,b]\), then

\[
\int_a^b f(x)\,dx = F(b)-F(a).
\]

\paragraph{Proof Sketch of Part I}\label{proof-sketch-of-part-i}

\begin{enumerate}
\def\labelenumi{\arabic{enumi}.}
\item
  Start with the definition of the derivative:

  \[
  F'(x) = \lim_{h\to 0} \frac{F(x+h)-F(x)}{h}.
  \]
\item
  Substituting \(F(x) = \int_a^x f(t)\,dt\):

  \[
  F(x+h)-F(x) = \int_a^{x+h} f(t)\,dt - \int_a^x f(t)\,dt.
  \]
\item
  By the additivity of integrals:

  \[
  F(x+h)-F(x) = \int_x^{x+h} f(t)\,dt.
  \]
\item
  Therefore:

  \[
  \frac{F(x+h)-F(x)}{h} = \frac{1}{h}\int_x^{x+h} f(t)\,dt.
  \]5. D'après le théorème de la valeur moyenne des intégrales, il
  existe \(c \in [x,x+h]\) tel que

  \[
  \frac{1}{h}\int_x^{x+h} f(t)\,dt = f(c).
  \]
\item
  Comme \(h \to 0\), \(c \to x\), et puisque \(f\) est continu :

  \[
  \lim_{h\to 0} f(c) = f(x).
  \]
\end{enumerate}

Ainsi, \(F'(x) = f(x)\).

\paragraph{Croquis de preuve de la partie
II}\label{croquis-de-preuve-de-la-partie-ii}

\begin{enumerate}
\def\labelenumi{\arabic{enumi}.}
\item
  Soit \(F\) une primitive de \(f\), donc \(F'(x) = f(x)\).
\item
  Par la partie I, la fonction

  \[
  G(x) = \int_a^x f(t)\,dt
  \]

  est également une primitive de \(f\).
\item
  Puisque \(F\) et \(G\) ne diffèrent que par une constante,

  \[
  F(x) = G(x) + C.
  \]
\item
  Évaluation aux points finaux~:

  \[
  \int_a^b f(x)\,dx = G(b)-G(a) = (F(b)+C)-(F(a)+C) = F(b)-F(a).
  \]
\end{enumerate}

\subsubsection{C.3 Esquisse de preuve : Convergence des séries
géométriques}\label{c.3-esquisse-de-preuve-convergence-des-suxe9ries-guxe9omuxe9triques}

La série géométrique est l'une des séries infinies les plus simples et
les plus importantes. Il sert de modèle pour comprendre la convergence
et constitue le fondement de nombreux résultats ultérieurs en calcul.

\paragraph{La série}\label{la-suxe9rie}

\[
\sum_{n=0}^\infty ar^n = a + ar + ar^2 + ar^3 + \cdots
\]

où \(a\) est le premier terme et \(r\) est la raison.

\paragraph{Formule de somme partielle}\label{formule-de-somme-partielle}

La \(n\)-ème somme partielle est

\[
S_n = a + ar + ar^2 + \cdots + ar^n.
\]

Multipliez les deux côtés par \(r\)~:

\[
rS_n = ar + ar^2 + \cdots + ar^{n+1}.
\]

Soustrayez les deux équations~:

\[
S_n - rS_n = a - ar^{n+1}.
\]

\[
S_n(1-r) = a(1-r^{n+1}).
\]

Alors

\[
S_n = \frac{a(1-r^{n+1})}{1-r}, \quad r \neq 1.
\]

\paragraph{Convergence}\label{convergence}

Prenez la limite comme \(n \to \infty\)~:

\begin{itemize}
\item
  Si \(|r| < 1\), alors \(r^{n+1} \to 0\).

  \[
  \lim_{n\to\infty} S_n = \frac{a}{1-r}.
  \]
\item
  Si \(|r| \geq 1\), alors \(r^{n+1}\) ne va pas à 0. La série diverge.
\end{itemize}

\paragraph{Résultat}\label{ruxe9sultat}

\[
\sum_{n=0}^\infty ar^n =
\begin{cases}
\dfrac{a}{1-r}, & |r|<1, \\[6pt]
\text{diverges}, & |r|\geq 1.
\end{cases}
\]

\subsection{Annexe D. Applications et
connexions}\label{annexe-d.-applications-et-connexions}

\subsubsection{D.1 Connexions physiques~: vitesse, accélération et
travail}\label{d.1-connexions-physiques-vitesse-accuxe9luxe9ration-et-travail}

Le calcul a été initialement développé pour résoudre des problèmes de
physique, en particulier le mouvement et le changement. Voici
quelques-unes des connexions les plus importantes.

\paragraph{Position, vitesse et
accélération}\label{position-vitesse-et-accuxe9luxe9ration-1}

\begin{itemize}
\item
  Fonction Position : \(s(t)\) donne la localisation d'un objet à
  l'instant \(t\).
\item
  Vitesse : la dérivée de la position.

  \[
  v(t) = s'(t) = \frac{ds}{dt}
  \]
\item
  Accélération : la dérivée de la vitesse (ou dérivée seconde de la
  position).

  \[
  a(t) = v'(t) = s''(t) = \frac{d^2s}{dt^2}
  \]
\end{itemize}

Exemple~: Si \(s(t) = 4t^2\) mètres, alors~:

\[
v(t) = 8t, \quad a(t) = 8.
\]

Ainsi, l'objet se déplace plus rapidement de manière linéaire avec le
temps, sous une accélération constante.

\paragraph{Travail et force}\label{travail-et-force}

En physique, le travail est le produit de la force et de la distance. Si
la force varie avec la position, le calcul donne~:

\[W = \int_a^b F(x)\, dx
\]

where \(F(x)\) is the force at position \(x\), and the object moves from
\(x=a\) to \(x=b\).

Example: A spring with Hooke's law force \(F(x) = kx\) requires work

\[
W = \int_0^d kx\, dx = \frac{1}{2}kd^2
\]

to stretch the spring a distance \(d\).

\paragraph{Energy and Areas Under
Curves}\label{energy-and-areas-under-curves}

\begin{itemize}
\tightlist
\item
  Kinetic energy: \(E_k = \tfrac{1}{2}mv^2\).
\item
  Potential energy often involves integrals (e.g., gravitational
  potential energy from force of gravity).
\item
  In general, integrating a force function gives energy stored or work
  done.
\end{itemize}

\paragraph{Quick Practice}\label{quick-practice}

\begin{enumerate}
\def\labelenumi{\arabic{enumi}.}
\tightlist
\item
  If \(s(t) = t^3 - 3t\), find \(v(t)\) and \(a(t)\).
\item
  Compute the work done by a constant force of 10 N moving an object 5
  m.
\item
  A spring has constant \(k=200\). How much work is needed to stretch it
  0.1 m?
\item
  Show that acceleration is the second derivative of position.
\item
  Explain how the integral \(\int v(t)\, dt\) relates to displacement.
\end{enumerate}

\subsubsection{D.2 Probability and Statistics
Connections}\label{d.2-probability-and-statistics-connections}

Calculus is deeply connected with probability and statistics, especially
when dealing with continuous random variables. Integrals become
essential for defining probabilities, averages, and expectations.

\paragraph{Probability Density Functions
(PDFs)}\label{probability-density-functions-pdfs}

For a continuous random variable \(X\), probabilities are described by a
probability density function \(f(x)\):

\begin{enumerate}
\def\labelenumi{\arabic{enumi}.}
\item
  \(f(x) \geq 0\) for all \(x\).
\item
  Total probability equals 1:

  \[
  \int_{-\infty}^{\infty} f(x)\, dx = 1.
  \]
\end{enumerate}

The probability that \(X\) lies in an interval \([a,b]\) is

\[
P(a \leq X \leq b) = \int_a^b f(x)\, dx.
\]

\paragraph{Expected Value (Mean)}\label{expected-value-mean}

The expected value (average outcome) is

\[
E[X] = \int_{-\infty}^{\infty} x f(x)\, dx.
\]

This is the calculus version of a weighted average.

\paragraph{Variance}\label{variance}

Variance measures spread:

\[
\text{Var}(X) = E[(X-\mu)^2] = \int_{-\infty}^{\infty} (x-\mu)^2 f(x)\, dx,
\]

where \(\mu = E[X]\).

\paragraph{Common Distributions}\label{common-distributions}

\begin{enumerate}
\def\labelenumi{\arabic{enumi}.}
\item
  Uniform distribution on \([a,b]\):

  \[
  f(x) = \frac{1}{b-a}, \quad a \leq x \leq b.
  \]

  Mean: \(\frac{a+b}{2}\).
\item
  Exponential distribution with parameter \(\lambda > 0\):

  \[
  f(x) = \lambda e^{-\lambda x}, \quad x \geq 0.
  \]

  Mean: \(1/\lambda\).
\item
  Normal (Gaussian) distribution:

  \[
  f(x) = \frac{1}{\sqrt{2\pi\sigma^2}} e^{-(x-\mu)^2/(2\sigma^2)}.
  \]

  Les intégrales de cette distribution se connectent à la fonction
  d'erreur.
\end{enumerate}

\paragraph{Pourquoi c'est important}\label{pourquoi-cest-important-31}

\begin{itemize}
\tightlist
\item
  Les intégrales transforment les probabilités en zones sous courbes.
\item
  Les attentes et la variance relient le calcul aux moyennes et à la
  variabilité.
\item
  La plupart des modèles de données du monde réel (finance, physique,
  biologie, IA) utilisent ces distributions de probabilité continues.
\end{itemize}

\paragraph{\texorpdfstring{Pratique rapide1. Pour
\(f(x) = \tfrac{1}{2}\) sur \([0,2]\), calculez
\(P(0.5 \leq X \leq 1.5)\).}{Pratique rapide1. Pour f(x) = \textbackslash tfrac\{1\}\{2\} sur {[}0,2{]}, calculez P(0.5 \textbackslash leq X \textbackslash leq 1.5).}}\label{pratique-rapide1.-pour-fx-tfrac12-sur-02-calculez-p0.5-leq-x-leq-1.5.}

\begin{enumerate}
\def\labelenumi{\arabic{enumi}.}
\setcounter{enumi}{1}
\tightlist
\item
  Pour une distribution exponentielle avec \(\lambda = 2\), calculez
  \(E[X]\).
\item
  Montrez que l'aire totale sous la courbe normale standard est égale à
  1.
\item
  Trouvez la moyenne d'une distribution uniforme sur \([3,7]\).
\item
  Expliquez pourquoi les probabilités sont calculées avec des
  intégrales, et non des sommes, pour les variables continues.
\end{enumerate}

\subsubsection{D.3 Connexions informatiques~: approximations de Taylor
dans les
algorithmes}\label{d.3-connexions-informatiques-approximations-de-taylor-dans-les-algorithmes}

Le calcul ne concerne pas seulement la physique : il sous-tend également
de nombreux outils et techniques en informatique. L'un des ponts les
plus évidents réside dans les séries de Taylor, qui fournissent des
moyens efficaces d'approcher les fonctions du calcul numérique et des
algorithmes.

\paragraph{Approximation des fonctions pour
l'informatique}\label{approximation-des-fonctions-pour-linformatique}

Les ordinateurs ne peuvent pas directement stocker ou calculer
exactement la plupart des fonctions (comme \(e^x\), \(\sin x\) ou
\(\ln x\)). Au lieu de cela, ils utilisent des approximations
polynomiales dérivées des développements de Taylor.

Exemple~: Pour approximer \(e^x\), tronquez la série Maclaurin~:

\[
e^x \approx 1 + x + \frac{x^2}{2!} + \frac{x^3}{3!}.
\]

Pour le petit \(x\), ce polynôme donne des résultats précis avec
seulement quelques termes.

\paragraph{Efficacité des
algorithmes}\label{efficacituxe9-des-algorithmes}

\begin{itemize}
\tightlist
\item
  Fonctions trigonométriques~: les algorithmes pour calculatrices et
  processeurs utilisent souvent des extensions en série (ou des
  variations comme les polynômes de Chebyshev).
\item
  Exponentiel/logarithme : les développements de Taylor sont à la base
  des approximations rapides dans les bibliothèques numériques.
\item
  Recherche de racine : la méthode de Newton est basée sur
  l'approximation linéaire, application directe de la série de Taylor
  (dérivée première).
\end{itemize}

\paragraph{Analyse numérique}\label{analyse-numuxe9rique}

Les développements de Taylor sont essentiels dans l'analyse des
erreurs~:

\begin{itemize}
\item
  Approximation du terme d'erreur à l'aide de la formule du reste~:

  \[
  R_n(x) = \frac{f^{(n+1)}(c)}{(n+1)!}(x-a)^{n+1}.
  \]
\item
  Cela nous indique combien de termes sont nécessaires pour une
  précision donnée.
\end{itemize}

\paragraph{Connexions d'apprentissage
automatique}\label{connexions-dapprentissage-automatique}

\begin{itemize}
\tightlist
\item
  L'optimisation basée sur le gradient (comme la descente de gradient)
  utilise des dérivés pour mettre à jour efficacement les paramètres.
\item
  Les fonctions d'activation (comme \(\tanh x\) ou
  \(\sigma(x)=1/(1+e^{-x})\)) sont souvent approximées par des polynômes
  ou des fonctions par morceaux pour la vitesse.
\item
  Les approximations en série peuvent accélérer la formation et
  l'inférence dans des environnements contraints.
\end{itemize}

\paragraph{Pourquoi c'est important}\label{pourquoi-cest-important-32}

\begin{itemize}
\tightlist
\item
  Les approximations de Taylor relient les mathématiques continues et le
  calcul discret.
\item
  Ils montrent comment les concepts de calcul sont utilisés dans les
  algorithmes, les méthodes numériques et l'apprentissage automatique.
\item
  Comprendre les approximations permet d'éviter les pièges liés au
  recours aux ordinateurs pour les calculs.
\end{itemize}

\paragraph{Pratique rapide}\label{pratique-rapide}

\begin{enumerate}
\def\labelenumi{\arabic{enumi}.}
\tightlist
\item
  Calculez \(\sin(0.1)\) en utilisant les trois premiers termes de sa
  série Maclaurin.2. Utilisez le terme restant pour estimer l'erreur
  d'approximation de \(e^1\) avec un polynôme de degré 3.
\item
  Expliquez comment la méthode de Newton utilise le théorème de Taylor.
\item
  Pourquoi les ordinateurs pourraient-ils préférer les approximations
  polynomiales aux formules exactes des fonctions~?
\item
  En apprentissage automatique, pourquoi la dérivée (gradient) est-elle
  si critique pour l'optimisation~?
\end{enumerate}




\end{document}
