% Options for packages loaded elsewhere
\PassOptionsToPackage{unicode}{hyperref}
\PassOptionsToPackage{hyphens}{url}
\PassOptionsToPackage{dvipsnames,svgnames,x11names}{xcolor}
%
\documentclass[
  letterpaper,
  DIV=11,
  numbers=noendperiod]{scrartcl}

\usepackage{amsmath,amssymb}
\usepackage{iftex}
\ifPDFTeX
  \usepackage[T1]{fontenc}
  \usepackage[utf8]{inputenc}
  \usepackage{textcomp} % provide euro and other symbols
\else % if luatex or xetex
  \usepackage{unicode-math}
  \defaultfontfeatures{Scale=MatchLowercase}
  \defaultfontfeatures[\rmfamily]{Ligatures=TeX,Scale=1}
\fi
\usepackage{lmodern}
\ifPDFTeX\else  
    % xetex/luatex font selection
\fi
% Use upquote if available, for straight quotes in verbatim environments
\IfFileExists{upquote.sty}{\usepackage{upquote}}{}
\IfFileExists{microtype.sty}{% use microtype if available
  \usepackage[]{microtype}
  \UseMicrotypeSet[protrusion]{basicmath} % disable protrusion for tt fonts
}{}
\makeatletter
\@ifundefined{KOMAClassName}{% if non-KOMA class
  \IfFileExists{parskip.sty}{%
    \usepackage{parskip}
  }{% else
    \setlength{\parindent}{0pt}
    \setlength{\parskip}{6pt plus 2pt minus 1pt}}
}{% if KOMA class
  \KOMAoptions{parskip=half}}
\makeatother
\usepackage{xcolor}
\setlength{\emergencystretch}{3em} % prevent overfull lines
\setcounter{secnumdepth}{-\maxdimen} % remove section numbering
% Make \paragraph and \subparagraph free-standing
\makeatletter
\ifx\paragraph\undefined\else
  \let\oldparagraph\paragraph
  \renewcommand{\paragraph}{
    \@ifstar
      \xxxParagraphStar
      \xxxParagraphNoStar
  }
  \newcommand{\xxxParagraphStar}[1]{\oldparagraph*{#1}\mbox{}}
  \newcommand{\xxxParagraphNoStar}[1]{\oldparagraph{#1}\mbox{}}
\fi
\ifx\subparagraph\undefined\else
  \let\oldsubparagraph\subparagraph
  \renewcommand{\subparagraph}{
    \@ifstar
      \xxxSubParagraphStar
      \xxxSubParagraphNoStar
  }
  \newcommand{\xxxSubParagraphStar}[1]{\oldsubparagraph*{#1}\mbox{}}
  \newcommand{\xxxSubParagraphNoStar}[1]{\oldsubparagraph{#1}\mbox{}}
\fi
\makeatother


\providecommand{\tightlist}{%
  \setlength{\itemsep}{0pt}\setlength{\parskip}{0pt}}\usepackage{longtable,booktabs,array}
\usepackage{calc} % for calculating minipage widths
% Correct order of tables after \paragraph or \subparagraph
\usepackage{etoolbox}
\makeatletter
\patchcmd\longtable{\par}{\if@noskipsec\mbox{}\fi\par}{}{}
\makeatother
% Allow footnotes in longtable head/foot
\IfFileExists{footnotehyper.sty}{\usepackage{footnotehyper}}{\usepackage{footnote}}
\makesavenoteenv{longtable}
\usepackage{graphicx}
\makeatletter
\newsavebox\pandoc@box
\newcommand*\pandocbounded[1]{% scales image to fit in text height/width
  \sbox\pandoc@box{#1}%
  \Gscale@div\@tempa{\textheight}{\dimexpr\ht\pandoc@box+\dp\pandoc@box\relax}%
  \Gscale@div\@tempb{\linewidth}{\wd\pandoc@box}%
  \ifdim\@tempb\p@<\@tempa\p@\let\@tempa\@tempb\fi% select the smaller of both
  \ifdim\@tempa\p@<\p@\scalebox{\@tempa}{\usebox\pandoc@box}%
  \else\usebox{\pandoc@box}%
  \fi%
}
% Set default figure placement to htbp
\def\fps@figure{htbp}
\makeatother

\KOMAoption{captions}{tableheading}
\makeatletter
\@ifpackageloaded{caption}{}{\usepackage{caption}}
\AtBeginDocument{%
\ifdefined\contentsname
  \renewcommand*\contentsname{Table of contents}
\else
  \newcommand\contentsname{Table of contents}
\fi
\ifdefined\listfigurename
  \renewcommand*\listfigurename{List of Figures}
\else
  \newcommand\listfigurename{List of Figures}
\fi
\ifdefined\listtablename
  \renewcommand*\listtablename{List of Tables}
\else
  \newcommand\listtablename{List of Tables}
\fi
\ifdefined\figurename
  \renewcommand*\figurename{Figure}
\else
  \newcommand\figurename{Figure}
\fi
\ifdefined\tablename
  \renewcommand*\tablename{Table}
\else
  \newcommand\tablename{Table}
\fi
}
\@ifpackageloaded{float}{}{\usepackage{float}}
\floatstyle{ruled}
\@ifundefined{c@chapter}{\newfloat{codelisting}{h}{lop}}{\newfloat{codelisting}{h}{lop}[chapter]}
\floatname{codelisting}{Listing}
\newcommand*\listoflistings{\listof{codelisting}{List of Listings}}
\makeatother
\makeatletter
\makeatother
\makeatletter
\@ifpackageloaded{caption}{}{\usepackage{caption}}
\@ifpackageloaded{subcaption}{}{\usepackage{subcaption}}
\makeatother

\ifLuaTeX
\usepackage[bidi=basic]{babel}
\else
\usepackage[bidi=default]{babel}
\fi
\babelprovide[main,import]{hindi}
% get rid of language-specific shorthands (see #6817):
\let\LanguageShortHands\languageshorthands
\def\languageshorthands#1{}
\usepackage{bookmark}

\IfFileExists{xurl.sty}{\usepackage{xurl}}{} % add URL line breaks if available
\urlstyle{same} % disable monospaced font for URLs
\hypersetup{
  pdftitle={कैलकुलस की छोटी किताब},
  pdflang={hi},
  colorlinks=true,
  linkcolor={blue},
  filecolor={Maroon},
  citecolor={Blue},
  urlcolor={Blue},
  pdfcreator={LaTeX via pandoc}}


\title{कैलकुलस की छोटी किताब}
\author{}
\date{}

\begin{document}
\maketitle


\section{कैलकुलस की छोटी सी
किताब}\label{ux915ux932ux915ux932ux938-ux915-ux91bux91f-ux938-ux915ux924ux92c}

कैलकुलस के मूल विचारों का एक संक्षिप्त, शुरुआती-अनुकूल परिचय।

\subsection{प्रारूप}\label{ux92aux930ux930ux92a}

\begin{itemize}
\tightlist
\item
  \href{../artifacts/hi/book.pdf}{Download PDF} - प्रिंट-तैयार संस्करण
\item
  \href{../artifacts/hi/book.epub}{Download EPUB} - ई-रीडर फ्रेंडली
\item
  \href{../artifacts/hi/book.tex}{View LaTeX} - लेटेक्स स्रोत
\end{itemize}

\section{भाग 1. सीमाएँ और
व्युत्पन्न}\label{ux92dux917-1.-ux938ux92eux90f-ux914ux930-ux935ux92fux924ux92aux928ux928}

\section{अध्याय 1. कार्य और
सीमाएँ}\label{ux905ux927ux92fux92f-1.-ux915ux930ux92f-ux914ux930-ux938ux92eux90f}

\subsection{1.1 कार्य}\label{ux915ux930ux92f}

फ़ंक्शन गणित में सबसे बुनियादी वस्तुओं में से एक है। इसके मूल में, एक फ़ंक्शन एक नियम है जो एक
इनपुट लेता है और ठीक एक आउटपुट उत्पन्न करता है। फ़ंक्शंस हमें रिश्तों का वर्णन करने,
वास्तविक दुनिया की घटनाओं का मॉडल बनाने और कैलकुलस की पूरी मशीनरी का निर्माण करने
देते हैं।

\subsubsection{परिभाषा}\label{ux92aux930ux92dux937}

औपचारिक रूप से, एक फ़ंक्शन \(f\) को एक सेट \(X\) (जिसे डोमेन कहा जाता है) से एक सेट
\(Y\) (जिसे कोडोमेन कहा जाता है) में लिखा जाता है

\[
f : X \to Y.
\]

प्रत्येक तत्व \(x \in X\) के लिए, एक अद्वितीय तत्व \(f(x) \in Y\) है। \(f(x)\)
मान को \(f\) के अंतर्गत \(x\) की छवि कहा जाता है।

यदि \(y = f(x)\), तो \(y\) इनपुट \(x\) के अनुरूप आउटपुट है। वास्तव में दिखाई देने
वाले सभी आउटपुट के सेट को रेंज (कोडोमेन का एक सबसेट) कहा जाता है।

\subsubsection{उदाहरण}\label{ux909ux926ux939ux930ux923}

\begin{enumerate}
\def\labelenumi{\arabic{enumi}.}
\item
  फ़ंक्शन \(f(x) = x^2\) प्रत्येक वास्तविक संख्या \(x\) को उसके वर्ग में मैप करता है।

  \begin{itemize}
  \tightlist
  \item
    डोमेन: सभी वास्तविक संख्याएँ \(\mathbb{R}\)।
  \item
    कोडोमेन: सभी वास्तविक संख्याएँ \(\mathbb{R}\)।
  \item
    रेंज: सभी गैर-नकारात्मक वास्तविक संख्याएं \([0, \infty)\)।
  \end{itemize}
\item
  फ़ंक्शन \(g(x) = \dfrac{1}{x}\) प्रत्येक गैरशून्य वास्तविक संख्या को उसका व्युत्क्रम
  निर्दिष्ट करता है।

  \begin{itemize}
  \tightlist
  \item
    डोमेन: \(\mathbb{R} \setminus \{0\}\).
  \item
    रेंज: \(\mathbb{R} \setminus \{0\}\)।
  \end{itemize}
\item
  एक वास्तविक दुनिया का उदाहरण: मान लीजिए \(T(t)\) समय \(t\) (घंटों में) पर
  बाहरी तापमान (डिग्री सेल्सियस में) है। यह ``दिन के समय'' से लेकर ``तापमान'' तक
  का कार्य है।
\end{enumerate}

\subsubsection{कार्यों को निरूपित करने के
तरीके}\label{ux915ux930ux92f-ux915-ux928ux930ux92aux924-ux915ux930ux928-ux915-ux924ux930ux915}

कार्यों को कई उपयोगी तरीकों से दर्शाया जा सकता है:

\begin{itemize}
\tightlist
\item
  सूत्र: जैसे, \(f(x) = \sin x + x^2\)।
\item
  ग्राफ़: निर्देशांक तल में सभी बिंदुओं \((x, f(x))\) को आलेखित करना।
\item
  टेबल्स: डेटा के अलग-अलग सेटों के लिए इनपुट और आउटपुट को जोड़ना।
\item
  मौखिक विवरण: ``प्रत्येक छात्र को उनका ग्रेड निर्दिष्ट करें।''
\end{itemize}

प्रत्येक प्रतिनिधित्व एक ही फ़ंक्शन के विभिन्न पहलुओं पर प्रकाश डालता है।

\subsubsection{शब्दावली}\label{ux936ux92cux926ux935ux932}

\begin{itemize}
\tightlist
\item
  स्वतंत्र चर: इनपुट (आमतौर पर \(x\) लिखा जाता है)।
\item
  आश्रित चर: आउटपुट (आमतौर पर \(y\) लिखा जाता है, जहां \(y = f(x)\))।
\item
  फ़ंक्शन नोटेशन: \(f(x)\) को ``\(x\) में से \(f\)'' पढ़ा जाता है।
\end{itemize}

\subsubsection{कैलकुलस में फ़ंक्शंस क्यों मायने रखते
हैं}\label{ux915ux932ux915ux932ux938-ux92e-ux92bux915ux936ux938-ux915ux92f-ux92eux92fux928-ux930ux916ux924-ux939}

कैलकुलस इस बात का अध्ययन है कि कार्य कैसे बदलते हैं। डेरिवेटिव परिवर्तन की तात्कालिक
दरों को मापते हैं, जबकि इंटीग्रल संचित प्रभावों को मापते हैं। इन विचारों पर महारत
हासिल करने के लिए, हमें सबसे पहले इस बात की ठोस समझ की आवश्यकता है कि कार्य क्या हैं
और वे कैसे व्यवहार करते हैं।

\subsubsection{व्यायाम}\label{ux935ux92fux92fux92e}

\begin{enumerate}
\def\labelenumi{\arabic{enumi}.}
\item
  फ़ंक्शन \(f(x) = 3x - 2\) के लिए:- डोमेन, कोडोमेन और रेंज ढूंढें।
\item
  फ़ंक्शन \(h(x) = \sqrt{x-1}\) किस इनपुट के लिए परिभाषित है? इसकी सीमा क्या है?
\item
  अपने दैनिक जीवन से किसी समारोह का वास्तविक दुनिया का उदाहरण दें। डोमेन और
  कोडोमेन को स्पष्ट रूप से बताएं।
\item
  \(f(x) = |x|\) का ग्राफ़ बनाएं। रेंज क्या है?
\item
  मान लीजिए \(g(x) = \dfrac{1}{x^2+1}\)। बताएं कि इसकी सीमा अंतराल
  \((0, 1]\) क्यों है।
\end{enumerate}

\subsection{1.2 ग्राफ़ और
परिवर्तन}\label{ux917ux930ux92b-ux914ux930-ux92aux930ux935ux930ux924ux928}

किसी फ़ंक्शन को न केवल सूत्रों से बल्कि उसके ग्राफ़ से भी समझा जा सकता है। फ़ंक्शन \(f\)
का ग्राफ़ सभी ऑर्डर किए गए जोड़े \((x, f(x))\) का सेट है, जहां \(x\) \(f\) के
डोमेन से संबंधित है। इन जोड़ियों को निर्देशांक तल में आलेखित करने से यह चित्र मिलता है कि
फ़ंक्शन कैसे व्यवहार करता है।

\subsubsection{बुनियादी ग्राफ़}\label{ux92cux928ux92fux926-ux917ux930ux92b}

कुछ ग्राफ़ इतने मौलिक हैं कि उन्हें याद रखना चाहिए:

\begin{itemize}
\tightlist
\item
  \(f(x) = x\): मूल बिंदु से होकर जाने वाली एक सीधी रेखा।
\item
  \(f(x) = x^2\): ऊपर की ओर खुलने वाला एक परवलय।
\item
  \(f(x) = |x|\): एक ``V'' आकार का ग्राफ़।
\item
  \(f(x) = \frac{1}{x}\): दो शाखाओं वाला एक अतिपरवलय।
\item
  \(f(x) = \sin x\): एक तरंग जैसा आवर्त वक्र।
\end{itemize}

ये अधिक जटिल कार्यों के लिए बिल्डिंग ब्लॉक्स के रूप में काम करते हैं।

\subsubsection{परिवर्तन}\label{ux92aux930ux935ux930ux924ux928}

सरल नियमों का उपयोग करके ग्राफ़ को स्थानांतरित, बढ़ाया या प्रतिबिंबित किया जा
सकता है:

\begin{enumerate}
\def\labelenumi{\arabic{enumi}.}
\item
  लंबवत बदलाव: एक स्थिरांक जोड़ने से ग्राफ़ ऊपर या नीचे चला जाता है।

  \[
  y = f(x) + c \quad \text{is } f(x) \text{ shifted upward by } c.
  \]
\item
  क्षैतिज बदलाव: तर्क के अंदर जोड़ने से ग्राफ़ बाएँ या दाएँ चलता है।

  \[
  y = f(x - c) \quad \text{is } f(x) \text{ shifted right by } c.
  \]
\item
  लंबवत स्केलिंग: स्थिरांक से गुणा करना ग्राफ़ को लंबवत रूप से फैलाता या संपीड़ित करता
  है।

  \[
  y = a f(x), \quad a > 1 \text{ stretches; } 0 < a < 1 \text{ compresses.}
  \]
\item
  क्षैतिज स्केलिंग: तर्क के अंदर गुणा करना ग्राफ़ को क्षैतिज रूप से फैलाता या संपीड़ित
  करता है।

  \[
  y = f(bx), \quad b > 1 \text{ compresses toward the } y\text{-axis}.
  \]
\item
  विचार:

  \begin{itemize}
  \tightlist
  \item
    \(y = -f(x)\): \(x\)-अक्ष पर प्रतिबिंब।
  \item
    \(y = f(-x)\): \(y\)-अक्ष पर प्रतिबिंब।
  \end{itemize}
\end{enumerate}

\subsubsection{परिवर्तनों का
संयोजन}\label{ux92aux930ux935ux930ux924ux928-ux915-ux938ux92fux91cux928}

जटिल ग्राफ़ अक्सर कई परिवर्तनों को क्रम में संयोजित करने से आते हैं। उदाहरण के लिए:

\[
y = 2(x-1)^2 + 3
\]

परवलय \(y = x^2\) को लेकर, 1 से दाएं खिसककर, 2 से लंबवत खींचकर और 3 से ऊपर की
ओर खिसकाकर प्राप्त किया जाता है।

\subsubsection{व्यायाम}\label{ux935ux92fux92fux92e-1}

\begin{enumerate}
\def\labelenumi{\arabic{enumi}.}
\tightlist
\item
  \(y = (x+2)^2 - 1\) का ग्राफ़ बनाएं। \(y = x^2\) से परिवर्तनों के अनुक्रम को
  पहचानें।
\item
  यदि हम \(x\) को \(-x\) से बदल दें तो \(y = f(x)\) के ग्राफ़ का क्या होगा? इसे
  \(f(x) = \sqrt{x}\) के साथ आज़माएँ।
\item
  उन परिवर्तनों का वर्णन करें जो \(y = \sin x\) को \(y = 3\sin(x - \pi/4)\) में
  बदल देते हैं।4. \(y = |x-1| + 2\) का ग्राफ़ बनाएं। प्रत्येक शाखा का शीर्ष तथा
  ढलान बताइये।
\item
  \(y = \frac{1}{x-2}\) के लिए, बताएं कि \(y = \frac{1}{x}\) का ग्राफ़ कैसे
  बदल गया है।
\end{enumerate}

\subsection{1.3 सीमाओं का सहज ज्ञान युक्त
विचार}\label{ux938ux92eux913-ux915-ux938ux939ux91c-ux91cux91eux928-ux92fux915ux924-ux935ux91aux930}

कई स्थितियों में, किसी बिंदु पर किसी फ़ंक्शन का मान उस बिंदु के निकट के मानों से कम
महत्वपूर्ण होता है। सीमा की अवधारणा इस विचार को दर्शाती है।

\subsubsection{किसी मान के करीब
पहुंचना}\label{ux915ux938-ux92eux928-ux915-ux915ux930ux92c-ux92aux939ux91aux928}

एक दीवार की ओर चलने की कल्पना करें। इससे पहले कि आप इसे छूएं, आप और भी करीब आ जाते
हैं। उसी तरह, जैसे \(x\) किसी संख्या \(a\) के करीब पहुंचता है, \(f(x)\) का मान
किसी संख्या \(L\) के करीब पहुंच सकता है। फिर हम कहते हैं:

\[
\lim_{x \to a} f(x) = L.
\]

यह इस विचार को व्यक्त करता है कि \(f(x)\) को \(L\) के उतना करीब बनाया जा
सकता है जितना हम \(L\) को चाहते हैं, बस \(x\) को \(a\) के काफी करीब ले जाकर।

\subsubsection{उदाहरण}\label{ux909ux926ux939ux930ux923-1}

\begin{enumerate}
\def\labelenumi{\arabic{enumi}.}
\item
  \(f(x) = 2x + 3\) के लिए: \(x \to 1\), \(f(x) \to 5\) के रूप में।
\item
  \(f(x) = \dfrac{\sin x}{x}\) के लिए: \(x \to 0\) के रूप में, फ़ंक्शन 1 के करीब
  पहुंचता है, भले ही \(f(0)\) परिभाषित नहीं है।
\item
  \(f(x) = \dfrac{1}{x}\) के लिए: जैसे \(x \to 0^+\) (दाईं ओर से आते हुए),
  \(f(x) \to +\infty\)। जैसे \(x \to 0^-\) (बायीं ओर से आते हुए),
  \(f(x) \to -\infty\)। चूँकि बाएँ और दाएँ व्यवहार भिन्न-भिन्न हैं, इसलिए 0 की
  सीमा मौजूद नहीं है।
\end{enumerate}

\subsubsection{सीमाओं का
महत्व}\label{ux938ux92eux913-ux915-ux92eux939ux924ux935}

\begin{itemize}
\tightlist
\item
  वे हमें उन बिंदुओं पर कार्यों को परिभाषित करने की अनुमति देते हैं जहां वे मूल रूप से
  परिभाषित नहीं हैं।
\item
  वे विच्छेदन और विलक्षणताओं के निकट व्यवहार को पकड़ते हैं।
\item
  वे डेरिवेटिव (परिवर्तन की तात्कालिक दर) और इंटीग्रल (योग की सीमा के रूप में क्षेत्र)
  के लिए आधार बनाते हैं।
\end{itemize}

\subsubsection{एकतरफ़ा
सीमाएँ}\label{ux90fux915ux924ux930ux92b-ux938ux92eux90f}

कभी-कभी बाएँ और दाएँ के व्यवहार का अलग-अलग अध्ययन किया जाना चाहिए:

\[
\lim_{x \to a^-} f(x), \quad \lim_{x \to a^+} f(x).
\]

यदि दोनों सहमत हैं, तो दोतरफा सीमा मौजूद है।

\subsubsection{व्यायाम}\label{ux935ux92fux92fux92e-2}

\begin{enumerate}
\def\labelenumi{\arabic{enumi}.}
\tightlist
\item
  \(\lim_{x \to 2} (3x^2 - x)\) की गणना करें।
\item
  \(\lim_{x \to 0} \frac{\sin x}{x}\) क्या है? \(\sin x\) के ग्राफ़ से अंतर्ज्ञान
  का उपयोग करें।
\item
  \(\lim_{x \to 0} |x|/x\) का मूल्यांकन करें। क्या दो तरफा सीमा मौजूद है?
\item
  \(\lim_{x \to \infty} \frac{1}{x}\) खोजें। इस परिणाम की शब्दों में व्याख्या
  करें।
\item
  \(f(x) = \frac{x^2-1}{x-1}\) के लिए, \(\lim_{x \to 1} f(x)\) क्या है?
  \(f(1)\) के मान से तुलना करें।
\end{enumerate}

\subsection{1.4 सीमाओं की औपचारिक
परिभाषा}\label{ux938ux92eux913-ux915-ux914ux92aux91aux930ux915-ux92aux930ux92dux937}

किसी सीमा के सहज विचार को एप्सिलॉन-डेल्टा परिभाषा का उपयोग करके सटीक बनाया जा
सकता है। इससे हमें यह कहने का एक कठोर तरीका मिलता है कि \(f(x)\) मान \(L\) के
करीब पहुंच जाता है क्योंकि \(x\) \(a\) के करीब पहुंच जाता है।

\subsubsection{परिभाषा}\label{ux92aux930ux92dux937-1}

हम लिखते हैं

\[
\lim_{x \to a} f(x) = L
\]

यदि निम्नलिखित शर्त लागू होती है:

प्रत्येक \(\varepsilon > 0\) (चाहे कितना भी छोटा क्यों न हो) के लिए, एक
\(\delta > 0\) मौजूद होता है जैसे कि जब भी

\[
0 < |x - a| < \delta,
\]

यह उसका अनुसरण करता है

\[
|f(x) - L| < \varepsilon.
\]शब्दों में: हम \(f(x)\) को \(L\) के जितना करीब चाहें, बना सकते हैं, बशर्ते \(x\)
\(a\) के काफी करीब हो (लेकिन \(a\) के बराबर नहीं)।

\subsubsection{उदाहरण 1: रैखिक
कार्य}\label{ux909ux926ux939ux930ux923-1-ux930ux916ux915-ux915ux930ux92f}

\(f(x) = 2x + 1\) के लिए, दिखाएँ कि \(\lim_{x \to 3} f(x) = 7\)।

\begin{itemize}
\tightlist
\item
  हम \(|f(x) - 7| < \varepsilon\) चाहते हैं।
\item
  लेकिन \(f(x) - 7 = 2x + 1 - 7 = 2(x - 3)\)।
\item
  तो \(|f(x) - 7| = 2|x - 3|\)।
\item
  यदि हम \(\delta = \varepsilon / 2\) चुनते हैं, तो जब भी
  \(|x - 3| < \delta\) होता है, तो हमारे पास \(|f(x) - 7| < \varepsilon\)
  होता है। इससे सीमा सिद्ध होती है।
\end{itemize}

\subsubsection{उदाहरण 2: पारस्परिक
कार्य}\label{ux909ux926ux939ux930ux923-2-ux92aux930ux938ux92aux930ux915-ux915ux930ux92f}

\(f(x) = \frac{1}{x}\) के लिए, \(\lim_{x \to 2} f(x) = \tfrac{1}{2}\) पर
विचार करें।

\begin{itemize}
\tightlist
\item
  हम \(\left|\frac{1}{x} - \frac{1}{2}\right| < \varepsilon\) चाहते हैं।
\item
  इस असमानता के लिए बीजीय हेरफेर की आवश्यकता होती है, लेकिन इसे \(\varepsilon\)
  के आधार पर \(\delta\) चुनकर संतुष्ट किया जा सकता है। प्रक्रिया अधिक जटिल है,
  लेकिन सिद्धांत वही है।
\end{itemize}

\subsubsection{यह क्यों मायने रखता
है}\label{ux92fux939-ux915ux92f-ux92eux92fux928-ux930ux916ux924-ux939}

\begin{itemize}
\tightlist
\item
  एप्सिलॉन-डेल्टा परिभाषा गारंटी देती है कि सीमाएं अस्पष्ट नहीं हैं या केवल अंतर्ज्ञान
  पर आधारित नहीं हैं।
\item
  यह निरंतरता, व्युत्पन्न और अभिन्नता की नींव है।
\item
  हालांकि शुरुआती लोगों को यह अमूर्त लग सकता है, सरल उदाहरणों के साथ काम करने से
  परिचितता बढ़ती है।
\end{itemize}

\subsubsection{व्यायाम}\label{ux935ux92fux92fux92e-3}

\begin{enumerate}
\def\labelenumi{\arabic{enumi}.}
\tightlist
\item
  एप्सिलॉन-डेल्टा परिभाषा का उपयोग करके सिद्ध करें कि
  \(\lim_{x \to 4} (x+1) = 5\)।
\item
  औपचारिक परिभाषा का उपयोग करके दिखाएँ कि \(\lim_{x \to 0} 5x = 0\)।
\item
  बताएं कि \(\lim_{x \to 0} \frac{1}{x}\) मौजूद क्यों नहीं है।
\item
  \(f(x) = x^2\) के लिए, दिखाएँ कि \(\lim_{x \to 2} f(x) = 4\)।
\item
  किसी सीमा की परिभाषा में \(\varepsilon\) और \(\delta\) की भूमिका को अपने
  शब्दों में समझाइए।
\end{enumerate}

\subsection{1.5 निरंतरता}\label{ux928ux930ux924ux930ux924}

एक फलन सतत है यदि उसका ग्राफ कागज से पेंसिल उठाए बिना खींचा जा सकता है। अधिक
सटीक रूप से, निरंतरता यह सुनिश्चित करती है कि इनपुट में छोटे परिवर्तन आउटपुट में छोटे
परिवर्तन उत्पन्न करते हैं।

\subsubsection{परिभाषा}\label{ux92aux930ux92dux937-2}

एक फ़ंक्शन \(f\) एक बिंदु \(a\) पर निरंतर है यदि तीन शर्तें पूरी होती हैं:

\begin{enumerate}
\def\labelenumi{\arabic{enumi}.}
\tightlist
\item
  \(f(a)\) परिभाषित है।
\item
  \(\lim_{x \to a} f(x)\) मौजूद है।
\item
  \(\lim_{x \to a} f(x) = f(a)\).
\end{enumerate}

यदि कोई फ़ंक्शन किसी अंतराल के प्रत्येक बिंदु पर निरंतर है, तो हम कहते हैं कि यह उस
अंतराल पर निरंतर है।

\subsubsection{उदाहरण}\label{ux909ux926ux939ux930ux923-2}

\begin{enumerate}
\def\labelenumi{\arabic{enumi}.}
\item
  बहुपद फलन: \(f(x) = x^2 + 3x - 5\) जैसे फलन \(\mathbb{R}\) पर हर जगह
  निरंतर होते हैं।
\item
  तर्कसंगत कार्य: \(f(x) = \frac{1}{x-1}\) \(x = 1\) को छोड़कर, जहां यह
  अपरिभाषित है, हर जगह निरंतर है।
\item
  टुकड़े-टुकड़े कार्य:

  \[
  f(x) =
  \begin{cases}
  x^2 & x < 1, \\
  2 & x = 1, \\
  x+1 & x > 1,
  \end{cases}
  \]

  इस फ़ंक्शन में \(x = 1\) पर ``कूद'' है, इसलिए यह वहां निरंतर नहीं है।
\end{enumerate}

\subsubsection{असंततता के
प्रकार}\label{ux905ux938ux924ux924ux924-ux915-ux92aux930ux915ux930}

\begin{enumerate}
\def\labelenumi{\arabic{enumi}.}
\tightlist
\item
  हटाने योग्य असंततता: ग्राफ़ में एक ``छेद''। उदाहरण:
  \(f(x) = \frac{x^2-1}{x-1}\) पर \(x=1\)।2. कूद असंततता: बाएं हाथ और दाएं
  हाथ की सीमाएं अलग-अलग हैं।
\item
  अनंत असंततता: फ़ंक्शन एक बिंदु के पास \(\pm\infty\) पर जाता है, जैसे
  \(f(x) = 1/x\) के पास \(x = 0\) पर जाता है।
\end{enumerate}

\subsubsection{मध्यवर्ती मूल्य
प्रमेय}\label{ux92eux927ux92fux935ux930ux924-ux92eux932ux92f-ux92aux930ux92eux92f}

यदि कोई फ़ंक्शन अंतराल \([a, b]\) पर निरंतर है, तो \(f(a)\) और \(f(b)\) के बीच
किसी भी संख्या \(N\) के लिए, कुछ \(c \in [a, b]\) मौजूद हैं जैसे कि \(f(c) = N\)।

यह गुण समीकरणों के मूलों और समाधानों के अस्तित्व को सिद्ध करने में महत्वपूर्ण है।

\subsubsection{व्यायाम}\label{ux935ux92fux92fux92e-4}

\begin{enumerate}
\def\labelenumi{\arabic{enumi}.}
\tightlist
\item
  तय करें कि फ़ंक्शन \(f(x) = |x|\) \(x = 0\) पर निरंतर है या नहीं।
\item
  \(f(x) = \frac{x+2}{x^2-1}\) के लिए असंततता के बिंदुओं की पहचान करें।
\item
  बताएं कि प्रत्येक बहुपद फलन हर जगह निरंतर क्यों होता है।
\item
  जम्प असंततता वाले फ़ंक्शन का एक उदाहरण दीजिए। इसका ग्राफ बनाइये।
\item
  यह दिखाने के लिए इंटरमीडिएट वैल्यू प्रमेय का उपयोग करें कि समीकरण
  \(x^3 + x - 1 = 0\) का समाधान 0 और 1 के बीच है।
\end{enumerate}

\section{अध्याय 2. संजात}\label{ux905ux927ux92fux92f-2.-ux938ux91cux924}

\subsection{2.1 परिवर्तन की दर के रूप में
व्युत्पन्न}\label{ux92aux930ux935ux930ux924ux928-ux915-ux926ux930-ux915-ux930ux92a-ux92e-ux935ux92fux924ux92aux928ux928}

व्युत्पन्न कलन के केंद्रीय विचारों में से एक है। यह मापता है कि कोई फ़ंक्शन अपने इनपुट में
बदलाव के साथ कैसे बदलता है - दूसरे शब्दों में, इनपुट के संबंध में आउटपुट में परिवर्तन की दर।

\subsubsection{परिवर्तन की औसत
दर}\label{ux92aux930ux935ux930ux924ux928-ux915-ux914ux938ux924-ux926ux930}

एक फ़ंक्शन \(f(x)\) के लिए, दो बिंदुओं \(x = a\) और \(x = b\) के बीच परिवर्तन की
औसत दर है

\[
\frac{f(b) - f(a)}{b - a}.
\]

यह बिंदुओं \((a, f(a))\) और \((b, f(b))\) से होकर जाने वाली छेदक रेखा का ढलान
है।

\subsubsection{परिवर्तन की तात्कालिक
दर}\label{ux92aux930ux935ux930ux924ux928-ux915-ux924ux924ux915ux932ux915-ux926ux930}

यह मापने के लिए कि एक बिंदु पर \(f(x)\) कितनी तेजी से बदल रहा है, हम अंतराल को
सिकुड़ने देते हैं:

\[
f'(a) = \lim_{h \to 0} \frac{f(a+h) - f(a)}{h}.
\]

यह सीमा, यदि मौजूद है, तो \(a\) पर \(f\) का व्युत्पन्न कहलाती है। ज्यामितीय रूप
से, यह बिंदु \((a, f(a))\) पर \(f\) के ग्राफ़ की स्पर्शरेखा रेखा का ढलान है।

\subsubsection{संकेतन}\label{ux938ux915ux924ux928}

\begin{itemize}
\tightlist
\item
  \(f'(x)\): प्राइम नोटेशन।
\item
  \(\dfrac{dy}{dx}\): लीबनिज़ नोटेशन, \(y = f(x)\) के लिए उपयोग किया जाता
  है।
\item
  \(Df(x)\): ऑपरेटर संकेतन।
\end{itemize}

ये सभी प्रतीक एक ही अवधारणा को दर्शाते हैं।

\subsubsection{उदाहरण}\label{ux909ux926ux939ux930ux923-3}

\begin{enumerate}
\def\labelenumi{\arabic{enumi}.}
\item
  \(f(x) = x^2\) के लिए:

  \[
  f'(x) = \lim_{h \to 0} \frac{(x+h)^2 - x^2}{h} = \lim_{h \to 0} \frac{2xh + h^2}{h} = 2x.
  \]

  \(x\) पर परवलय का ढलान \(2x\) है।
\item
  \(f(x) = \sin x\) के लिए:

  \[
  f'(x) = \cos x.
  \]
\item
  \(f(x) = c\) (एक स्थिरांक) के लिए:

  \[
  f'(x) = 0.
  \]

  एक स्थिर कार्य कभी नहीं बदलता।
\end{enumerate}

\subsubsection{व्याख्या}\label{ux935ux92fux916ux92f}

\begin{itemize}
\tightlist
\item
  भौतिकी में: यदि \(s(t)\) स्थिति है, तो \(s'(t)\) वेग है।
\item
  अर्थशास्त्र में: यदि \(C(x)\) लागत है, तो \(C'(x)\) सीमांत लागत है।
\item
  जीव विज्ञान में: यदि \(P(t)\) जनसंख्या है, तो \(P'(t)\) विकास दर है।
\end{itemize}

व्युत्पन्न कई संदर्भों में ``परिवर्तन'' को सटीक बनाता है।

\subsubsection{व्यायाम}\label{ux935ux92fux92fux92e-5}

\begin{enumerate}
\def\labelenumi{\arabic{enumi}.}
\tightlist
\item
  \(f(x) = 3x^2 - 2x + 1\) के लिए \(f'(x)\) की गणना करें।2. \(x = 2\) पर
  \(f(x) = x^3\) की स्पर्श रेखा का ढलान ज्ञात कीजिए।
\item
  यदि \(s(t) = t^2 + 2t\) मीटर में दूरी दर्शाता है, तो \(t = 5\) पर वेग क्या है?
\item
  \(f(x) = \frac{1}{x}\) के व्युत्पन्न की गणना करने के लिए सीमा परिभाषा का
  उपयोग करें।
\item
  \(y = x^2\) का ग्राफ़ बनाएं और \(x = 1\) पर स्पर्शरेखा रेखा खींचें।
\end{enumerate}

\subsection{2.2 विभेदीकरण
नियम}\label{ux935ux92dux926ux915ux930ux923-ux928ux92fux92e}

एक बार जब व्युत्पन्न परिभाषित हो जाता है, तो हमें इसकी गणना करने के लिए कुशल तरीकों
की आवश्यकता होती है। विभेदन नियम शॉर्टकट हैं जो हमें सीमा परिभाषा को बार-बार लागू
करने से बचाते हैं।

\subsubsection{निरंतर नियम}\label{ux928ux930ux924ux930-ux928ux92fux92e}

यदि \(f(x) = c\) जहां \(c\) एक स्थिरांक है, तो

\[
f'(x) = 0.
\]

\subsubsection{शक्ति नियम}\label{ux936ux915ux924-ux928ux92fux92e}

\(f(x) = x^n\) के लिए जहां \(n\) एक वास्तविक संख्या है,

\[
\frac{d}{dx} \big( x^n \big) = n x^{n-1}.
\]

उदाहरण:

\begin{itemize}
\tightlist
\item
  \(\frac{d}{dx}(x^2) = 2x\).
\item
  \(\frac{d}{dx}(x^5) = 5x^4\).
\item
  \(\frac{d}{dx}(\sqrt{x}) = \frac{1}{2\sqrt{x}}\).
\end{itemize}

\subsubsection{लगातार एकाधिक
नियम}\label{ux932ux917ux924ux930-ux90fux915ux927ux915-ux928ux92fux92e}

यदि \(f(x) = c \cdot g(x)\), तो

\[
f'(x) = c \cdot g'(x).
\]

\subsubsection{योग और अंतर
नियम}\label{ux92fux917-ux914ux930-ux905ux924ux930-ux928ux92fux92e}

\begin{itemize}
\tightlist
\item
  \((f + g)' = f' + g'\).
\item
  \((f - g)' = f' - g'\).
\end{itemize}

\subsubsection{उत्पाद नियम}\label{ux909ux924ux92aux926-ux928ux92fux92e}

\(f(x)\) और \(g(x)\) के लिए:

\[
(fg)' = f'g + fg'.
\]

उदाहरण: यदि \(f(x) = x^2\), \(g(x) = \sin x\):

\[
(fg)' = (2x)(\sin x) + (x^2)(\cos x).
\]

\subsubsection{भागफल नियम}\label{ux92dux917ux92bux932-ux928ux92fux92e}

\(f(x)\) और \(g(x)\) के लिए:

\[
\left(\frac{f}{g}\right)' = \frac{f'g - fg'}{g^2}, \quad g(x) \neq 0.
\]

उदाहरण: यदि \(f(x) = x^2\), \(g(x) = x+1\):

\[
\left(\frac{x^2}{x+1}\right)' = \frac{(2x)(x+1) - (x^2)(1)}{(x+1)^2}.
\]

\subsubsection{सामान्य कार्यों के
व्युत्पन्न}\label{ux938ux92eux928ux92f-ux915ux930ux92f-ux915-ux935ux92fux924ux92aux928ux928}

\begin{itemize}
\tightlist
\item
  \(\frac{d}{dx}(\sin x) = \cos x\).
\item
  \(\frac{d}{dx}(\cos x) = -\sin x\).
\item
  \(\frac{d}{dx}(e^x) = e^x\).
\item
  \(\frac{d}{dx}(\ln x) = \frac{1}{x}, \quad x > 0\).
\end{itemize}

\subsubsection{व्यायाम}\label{ux935ux92fux92fux92e-6}

\begin{enumerate}
\def\labelenumi{\arabic{enumi}.}
\tightlist
\item
  \(f(x) = 7x^3 - 4x + 9\) में अंतर बताएं।
\item
  \(f(x) = x^2 e^x\) का अवकलज ज्ञात करने के लिए उत्पाद नियम का उपयोग करें।
\item
  भागफल नियम को \(f(x) = \frac{\sin x}{x}\) पर लागू करें।
\item
  नियमों की श्रृंखला का उपयोग करके \(\frac{d}{dx}(\ln(x^2))\) की गणना करें।
\item
  दिखाएँ कि \(f(x) = \frac{1}{x}\) का अवकलज \(-\frac{1}{x^2}\) है।
\end{enumerate}

\subsection{2.3 श्रृंखला नियम}\label{ux936ux930ux916ux932-ux928ux92fux92e}

अक्सर, फ़ंक्शंस सरल फ़ंक्शंस को एक साथ जोड़कर बनाए जाते हैं। ऐसे समग्र कार्यों को अलग करने
के लिए, हम श्रृंखला नियम का उपयोग करते हैं।

\subsubsection{नियम}\label{ux928ux92fux92e}

यदि \(y = f(g(x))\), तो

\[
\frac{dy}{dx} = f'(g(x)) \cdot g'(x).
\]

शब्दों में: बाहरी फ़ंक्शन को अलग करें, अंदर को अपरिवर्तित रखें, फिर अंदर के व्युत्पन्न से
गुणा करें।

\subsubsection{उदाहरण}\label{ux909ux926ux939ux930ux923-4}

\begin{enumerate}
\def\labelenumi{\arabic{enumi}.}
\item
  एक रैखिक फलन का वर्ग

  \[
  y = (3x+2)^2
  \]

  बाहरी फ़ंक्शन: \(f(u) = u^2\), आंतरिक फ़ंक्शन: \(g(x) = 3x+2\)।

  \[
  y' = 2(3x+2) \cdot 3 = 6(3x+2).
  \]
\item
  अंदर द्विघात के साथ घातांकीय

  \[
  y = e^{x^2}
  \]

  बाहरी फ़ंक्शन: \(f(u) = e^u\), आंतरिक फ़ंक्शन: \(g(x) = x^2\)।

  \[y' = e^{x^2} \cdot 2x = 2x e^{x^2}.
  \]
\item
  Logarithm with root inside

  \[
  y = \ln(\sqrt{x})
  \]

  Outer: \(f(u) = \ln u\), inner: \(g(x) = \sqrt{x}\).

  \[
  y' = \frac{1}{\sqrt{x}
  \]
\end{enumerate}

\subsubsection{Generalized Chain Rule}\label{generalized-chain-rule}

For multiple nested functions \(y = f(g(h(x)))\):

\[
\frac{dy}{dx} = f'(g(h(x))) \cdot g'(h(x)) \cdot h'(x).
\]

This extends naturally to deeper compositions.

\subsubsection{Why the Chain Rule
Matters}\label{why-the-chain-rule-matters}

\begin{itemize}
\tightlist
\item
  It handles nearly all real-world models where one quantity depends on
  another indirectly.
\item
  It connects calculus with physics (e.g., velocity depending on time
  through position).
\item
  It is essential in implicit differentiation and advanced topics.
\end{itemize}

\subsubsection{Exercises}\label{exercises}

\begin{enumerate}
\def\labelenumi{\arabic{enumi}.}
\tightlist
\item
  Differentiate \(y = (5x^2 + 1)^3\).
\item
  Find \(\frac{d}{dx}(\sin(3x))\).
\item
  Compute \(\frac{d}{dx}(\ln(1+x^2))\).
\item
  Differentiate \(y = \cos^2(x)\).
\item
  Apply the generalized chain rule to \(y = e^{\sin(x^2)}\).
\end{enumerate}

\subsection{2.4 Implicit
Differentiation}\label{implicit-differentiation}

Not all functions are given in the form \(y = f(x)\). Sometimes \(x\)
and \(y\) are related by an equation, and solving explicitly for \(y\)
is difficult or impossible. In such cases, we use implicit
differentiation.

\subsubsection{The Idea}\label{the-idea}

If an equation involves both \(x\) and \(y\), we can differentiate both
sides with respect to \(x\), treating \(y\) as a function of \(x\). Each
time we differentiate a term involving \(y\), we multiply by
\(\frac{dy}{dx}\).

\subsubsection{Example 1: A Circle}\label{example-1-a-circle}

Equation:

\[
x^2 + y^2 = 25
\]

Differentiate with respect to \(x\):

\[
2x + 2y \frac{dy}{dx} = 0.
\]

Solve for \(\frac{dy}{dx}\):

\[
\frac{dy}{dx} = -\frac{x}{y}.
\]

This gives the slope of the tangent to the circle at any point.

\subsubsection{Example 2: A Product of
Variables}\label{example-2-a-product-of-variables}

Equation:

\[
एक्सवाई = 1
\]

Differentiate:

\[
x \frac{dy}{dx} + y = 0.
\]

So,

\[
\frac{dy}{dx} = -\frac{y}{x}.
\]

\subsubsection{Example 3: Trigonometric
Relation}\label{example-3-trigonometric-relation}

Equation:

\[
\sin(xy) = x
\]

Differentiate:

\[
\cos(xy) \cdot \Big(y + x\frac{dy}{dx}\Big) = 1.
\]

Solve for \(\frac{dy}{dx}\):

\[
\frac{dy}{dx} = \frac{1 - y\cos(xy)}{x\cos(xy)}.
\]

\subsubsection{अंतर्निहित विभेदीकरण क्यों उपयोगी
है}\label{ux905ux924ux930ux928ux939ux924-ux935ux92dux926ux915ux930ux923-ux915ux92f-ux909ux92aux92fux917-ux939}

\begin{itemize}
\tightlist
\item
  कई महत्वपूर्ण वक्र (वृत्त, दीर्घवृत्त, अतिपरवलय) स्वाभाविक रूप से अंतर्निहित रूप से
  परिभाषित होते हैं।
\item
  यह हमें \(y\) को पहले हल किए बिना समीकरणों में अंतर करने की अनुमति देता है।
\item
  यह संबंधित दरों और अंतर समीकरणों जैसे अधिक उन्नत विषयों में एक महत्वपूर्ण कदम है।
\end{itemize}

\subsubsection{व्यायाम}\label{ux935ux92fux92fux92e-7}

\begin{enumerate}
\def\labelenumi{\arabic{enumi}.}
\tightlist
\item
  वक्र \(x^2 + xy + y^2 = 7\) के लिए, \(\frac{dy}{dx}\) खोजें।
\item
  \(\cos(x) + \cos(y) = 1\) को स्पष्ट रूप से अलग करें।
\item
  बिंदु \((1, 2)\) पर स्पर्शरेखा रेखा का ढलान ज्ञात करें।4. \(x^2 + y^2 = 10\)
  दिया गया है, \((x, y) = (1, 3)\) होने पर \(\frac{dy}{dx}\) की गणना करें।
\item
  \(\frac{dy}{dx}\) ज्ञात करने के लिए \(e^{xy} = x + y\) का अंतर करें।
\end{enumerate}

\subsection{2.5 उच्च-क्रम
डेरिवेटिव}\label{ux909ux91aux91a-ux915ux930ux92e-ux921ux930ux935ux91fux935}

अब तक, हमने पहले व्युत्पन्न का अध्ययन किया है, जो किसी फ़ंक्शन के परिवर्तन की दर को
मापता है। लेकिन स्वयं डेरिवेटिव को भी विभेदित किया जा सकता है, जिससे उच्च-क्रम के
डेरिवेटिव को जन्म मिलता है।

\subsubsection{परिभाषा}\label{ux92aux930ux92dux937-3}

\begin{itemize}
\item
  \(f\) का दूसरा व्युत्पन्न व्युत्पन्न का व्युत्पन्न है:

  \[
  f''(x) = \frac{d}{dx}\left(f'(x)\right).
  \]
\item
  अधिक सामान्यतः, \(n\)-वें व्युत्पन्न को इस प्रकार लिखा जाता है

  \[
  f^{(n)}(x) = \frac{d^n}{dx^n} f(x).
  \]
\end{itemize}

\subsubsection{उदाहरण}\label{ux909ux926ux939ux930ux923-5}

\begin{enumerate}
\def\labelenumi{\arabic{enumi}.}
\item
  \(f(x) = x^3\)

  \begin{itemize}
  \tightlist
  \item
    पहला व्युत्पन्न: \(f'(x) = 3x^2\)।
  \item
    दूसरा व्युत्पन्न: \(f''(x) = 6x\)।
  \item
    तीसरा व्युत्पन्न: \(f^{(3)}(x) = 6\)।
  \item
    चौथा व्युत्पन्न: \(f^{(4)}(x) = 0\)।
  \end{itemize}
\item
  \(f(x) = \sin x\)

  \begin{itemize}
  \tightlist
  \item
    \(f'(x) = \cos x\).
  \item
    \(f''(x) = -\sin x\).
  \item
    \(f^{(3)}(x) = -\cos x\).
  \item
    \(f^{(4)}(x) = \sin x\). डेरिवेटिव लंबाई 4 के चक्र में दोहराए जाते हैं।
  \end{itemize}
\item
  \(f(x) = e^x\)

  \begin{itemize}
  \tightlist
  \item
    प्रत्येक व्युत्पन्न \(e^x\) है।
  \end{itemize}
\end{enumerate}

\subsubsection{अनुप्रयोग}\label{ux905ux928ux92aux930ux92fux917}

\begin{itemize}
\item
  अवतलता: \(f''(x)\) का चिह्न बताता है कि \(f\) का ग्राफ़ अवतल ऊपर
  (\(f'' > 0\)) है या अवतल नीचे (\(f'' < 0\))।
\item
  विभक्ति बिंदु: वे बिंदु जहां \(f''(x) = 0\) और अवतलता बदलती है।
\item
  गति: भौतिकी में, यदि \(s(t)\) स्थिति है:

  \begin{itemize}
  \tightlist
  \item
    \(s'(t)\) = वेग,
  \item
    \(s''(t)\) = त्वरण,
  \item
    \(s^{(3)}(t)\) = झटका (त्वरण परिवर्तन की दर)।
  \end{itemize}
\item
  अनुमान: उच्च-क्रम डेरिवेटिव टेलर श्रृंखला में दिखाई देते हैं, जिनका उपयोग कार्यों का
  अनुमान लगाने के लिए किया जाता है।
\end{itemize}

\subsubsection{व्यायाम}\label{ux935ux92fux92fux92e-8}

\begin{enumerate}
\def\labelenumi{\arabic{enumi}.}
\tightlist
\item
  \(f(x) = \cos x\) के पहले चार डेरिवेटिव की गणना करें।
\item
  \(f(x) = x^4 - 2x^2 + 3\) के लिए \(f''(x)\) खोजें।
\item
  \(f(x) = e^{2x}\) के लिए, दिखाएँ कि \(f^{(n)}(x) = 2^n e^{2x}\)।
\item
  वे अंतराल निर्धारित करें जहां \(f(x) = x^3 - 3x\) अवतल ऊपर और अवतल नीचे है।
\item
  यदि \(s(t) = t^3 - 6t^2 + 9t\), तो \(t = 2\) पर वेग और त्वरण ज्ञात कीजिए।
\end{enumerate}

\section{अध्याय 3. डेरिवेटिव के
अनुप्रयोग}\label{ux905ux927ux92fux92f-3.-ux921ux930ux935ux91fux935-ux915-ux905ux928ux92aux930ux92fux917}

\subsection{3.1 स्पर्श रेखाएँ और
सामान्य}\label{ux938ux92aux930ux936-ux930ux916ux90f-ux914ux930-ux938ux92eux928ux92f}

डेरिवेटिव के पहले अनुप्रयोगों में से एक वक्र की स्पर्शरेखा और सामान्य रेखाओं के समीकरण
ढूंढना है। ये रेखाएँ किसी दिए गए बिंदु पर किसी फ़ंक्शन की स्थानीय ज्यामिति को कैप्चर
करती हैं।

\subsubsection{स्पर्शरेखा
रेखा}\label{ux938ux92aux930ux936ux930ux916-ux930ux916}

एक बिंदु \((a, f(a))\) पर वक्र \(y = f(x)\) की स्पर्शरेखा रेखा वह रेखा है जो वहां
ग्राफ़ को ``स्पर्श'' करती है और वक्र के समान ढलान रखती है।

स्पर्शरेखा रेखा का ढलान व्युत्पन्न द्वारा दिया गया है:

\[
m_{\text{tangent}} = f'(a).
\]

इस प्रकार, \((a, f(a))\) पर स्पर्श रेखा का समीकरण है

\[
y - f(a) = f'(a)(x - a).
\]

\subsubsection{सामान्य रेखा}\label{ux938ux92eux928ux92f-ux930ux916}

सामान्य रेखा एक ही बिंदु पर स्पर्शरेखा रेखा के लंबवत होती है। इसका ढलान स्पर्शरेखा
ढलान का ऋणात्मक व्युत्क्रम है:

\[m_{\text{सामान्य}} = -\frac{1}{f'(a)}.
\]

So the equation of the normal line is

\[
y - f(a) = -\frac{1}{f'(a)} (x - a), \quad f'(a) \neq 0.
\]

\subsubsection{Examples}\label{examples}

\begin{enumerate}
\def\labelenumi{\arabic{enumi}.}
\item
  \(f(x) = x^2\) at \(x = 1\).

  \begin{itemize}
  \tightlist
  \item
    \(f(1) = 1\), \(f'(x) = 2x\), so \(f'(1) = 2\).
  \item
    Tangent: \(y - 1 = 2(x - 1)\), or \(y = 2x - 1\).
  \item
    Normal: slope = \(-\tfrac{1}{2}\), so equation is
    \(y - 1 = -\tfrac{1}{2}(x - 1)\).
  \end{itemize}
\item
  \(f(x) = \sin x\) at \(x = \tfrac{\pi}{4}\).

  \begin{itemize}
  \tightlist
  \item
    \(f(\tfrac{\pi}{4}) = \tfrac{\sqrt{2}}{2}\),
    \(f'(\tfrac{\pi}{4}) = \cos(\tfrac{\pi}{4}) = \tfrac{\sqrt{2}}{2}\).
  \item
    Tangent:
    \(y - \tfrac{\sqrt{2}}{2} = \tfrac{\sqrt{2}}{2}(x - \tfrac{\pi}{4})\).
  \end{itemize}
\end{enumerate}

\subsubsection{Why Tangents and Normals
Matter}\label{why-tangents-and-normals-matter}

\begin{itemize}
\tightlist
\item
  Tangents approximate the curve locally (linear approximation).
\item
  Normals are useful in geometry, optics (reflection/refraction), and
  mechanics (force directions).
\item
  Both play a role in optimization and curvature studies.
\end{itemize}

\subsubsection{Exercises}\label{exercises-1}

\begin{enumerate}
\def\labelenumi{\arabic{enumi}.}
\tightlist
\item
  Find the tangent and normal lines to \(y = x^3\) at \(x = 2\).
\item
  Determine the tangent and normal lines to \(y = e^x\) at \(x = 0\).
\item
  For \(y = \ln x\), compute the tangent line at \(x = 1\).
\item
  A circle is given by \(x^2 + y^2 = 9\). Use implicit differentiation
  to find the slope of the tangent at \((0,3)\).
\item
  Sketch the graph of \(y = \sqrt{x}\) and draw the tangent and normal
  lines at \(x = 4\).
\end{enumerate}

\subsection{3.2 Related Rates}\label{related-rates}

In many real-world problems, two or more quantities change with respect
to time, and their rates of change are connected. Related rates problems
use derivatives to describe these relationships.

\subsubsection{General Approach}\label{general-approach}

\begin{enumerate}
\def\labelenumi{\arabic{enumi}.}
\tightlist
\item
  Identify the variables that depend on time \(t\).
\item
  Write an equation relating the variables.
\item
  Differentiate both sides with respect to \(t\), applying the chain
  rule.
\item
  Substitute the known values at the given instant.
\item
  Solve for the unknown rate.
\end{enumerate}

\subsubsection{Example 1: Expanding
Circle}\label{example-1-expanding-circle}

A circle has radius \(r\), which increases at the rate of
\(\frac{dr}{dt} = 2 \,\text{cm/s}\). Find the rate at which the area
\(A = \pi r^2\) increases when \(r = 5\).

Differentiate:

\[
\frac{dA}{dt} = 2\pi r \frac{dr}{dt}.
\]

Substitute:

\[
\frac{dA}{dt} = 2\pi (5)(2) = 20\pi \,\text{cm}^2/\text{s}.
\]

\subsubsection{Example 2: Sliding
Ladder}\label{example-2-sliding-ladder}

A 10 ft ladder leans against a wall. The bottom slides away at
\(\frac{dx}{dt} = 1 \,\text{ft/s}\). How fast is the top sliding down
when the bottom is 6 ft from the wall?

Equation: \(x^2 + y^2 = 100\), where \(y\) is the height.

Differentiate:

\[
2x \frac{dx}{dt} + 2y \frac{dy}{dt} = 0.
\]

At \(x = 6\), \(y = 8\). Substitute:

\[
2(6)(1) + 2(8)\frac{dy}{dt} = 0 \quad \राइटएरो \quad \frac{dy}{dt} = -\tfrac{6}{8} = -\tfrac{3}{4}.
\]तो शीर्ष \(0.75 \,\text{ft/s}\) पर नीचे की ओर खिसक जाता है।

\subsubsection{उदाहरण 3: एक शंकु में
पानी}\label{ux909ux926ux939ux930ux923-3-ux90fux915-ux936ux915-ux92e-ux92aux928}

12 सेमी ऊंचाई और 6 सेमी त्रिज्या वाले एक शंकु में पानी डाला जाता है। जब पानी 4 सेमी
गहरा होता है, तो जल स्तर \(2 \,\text{cm/s}\) पर बढ़ रहा है। आयतन किस दर से बढ़
रहा है?

समीकरण: \(V = \tfrac{1}{3}\pi r^2 h\). समानता का उपयोग करते हुए,
\(r = \tfrac{h}{2}\)। प्रतिस्थापन:

\[
V = \tfrac{1}{12}\pi h^3.
\]

अंतर करें:

\[
\frac{dV}{dt} = \tfrac{1}{4}\pi h^2 \frac{dh}{dt}.
\]

\(h = 4\), \(\frac{dh}{dt} = 2\) पर:

\[
\frac{dV}{dt} = \tfrac{1}{4}\pi (16)(2) = 8\pi \,\text{cm}^3/\text{s}.
\]

\subsubsection{संबंधित दरें क्यों मायने रखती
हैं}\label{ux938ux92cux927ux924-ux926ux930-ux915ux92f-ux92eux92fux928-ux930ux916ux924-ux939}

\begin{itemize}
\tightlist
\item
  वे भौतिकी, इंजीनियरिंग और जीव विज्ञान में गति और परिवर्तन का वर्णन करते हैं।
\item
  वे समय-निर्भर प्रक्रियाओं के माध्यम से ज्यामिति को कैलकुलस से जोड़ते हैं।
\item
  वे हमें गतिशील प्रणालियों को गणितीय रूप से मॉडल करने के लिए प्रशिक्षित करते हैं।
\end{itemize}

\subsubsection{व्यायाम}\label{ux935ux92fux92fux92e-9}

\begin{enumerate}
\def\labelenumi{\arabic{enumi}.}
\tightlist
\item
  एक गुब्बारा फुलाया जाता है ताकि उसकी त्रिज्या \(0.5 \,\text{cm/s}\) पर बढ़
  जाए। ज्ञात कीजिए कि त्रिज्या 10 सेमी होने पर इसका आयतन कितनी तेजी से बढ़ता है।
\item
  एक कार उत्तर की ओर 40 किमी/घंटा की गति से और दूसरी पूर्व की ओर 30 किमी/घंटा
  की गति से चलती है। 2 घंटे बाद उनके बीच की दूरी कितनी तेजी से बढ़ रही है?
\item
  एक दीवार से 20 मीटर दूर एक स्पॉटलाइट 1.5 मीटर/सेकेंड की गति से चलते हुए 2 मीटर
  लंबे एक व्यक्ति पर चमकती है। जब वह प्रकाश से 5 मीटर दूर है तो दीवार पर उसकी
  छाया की लंबाई कितनी तेजी से बदलती है?
\item
  एक घन की भुजा की लंबाई 2 सेमी/सेकंड की गति से बढ़ती है। जब भुजा 3 सेमी है तो सतह
  का क्षेत्रफल कितनी तेजी से बढ़ रहा है?
\item
  एक ढेर पर रेत डाली जाती है जिससे एक शंकु बनता है जिसकी त्रिज्या हमेशा ऊंचाई के
  बराबर होती है। यदि ऊंचाई 5 सेमी/सेकंड की दर से बढ़ती है, तो ऊंचाई 10 सेमी होने पर
  आयतन किस दर से बढ़ रहा है?
\end{enumerate}

\subsection{3.3 अनुकूलन
समस्याएँ}\label{ux905ux928ux915ux932ux928-ux938ux92eux938ux92fux90f}

अनुकूलन समस्याएं किसी फ़ंक्शन के अधिकतम या न्यूनतम मूल्यों को खोजने के लिए डेरिवेटिव का
उपयोग करती हैं, अक्सर कुछ बाधाओं के तहत। ये समस्याएँ उन स्थितियों को मॉडल करती हैं
जहाँ हम दक्षता, लाभ या क्षेत्र को अधिकतम करना चाहते हैं, या लागत, दूरी या समय को
कम करना चाहते हैं।

\subsubsection{सामान्य कदम}\label{ux938ux92eux928ux92f-ux915ux926ux92e}

\begin{enumerate}
\def\labelenumi{\arabic{enumi}.}
\tightlist
\item
  समस्या को समझें: अनुकूलित करने के लिए मात्रा की पहचान करें।
\item
  एक फ़ंक्शन के साथ मॉडल: उद्देश्य फ़ंक्शन को एक चर के संदर्भ में लिखें।
\item
  बाधाएं लागू करें: चर को कम करने के लिए दी गई शर्तों का उपयोग करें।
\item
  अंतर करें: उद्देश्य फ़ंक्शन के व्युत्पन्न की गणना करें।
\item
  महत्वपूर्ण बिंदु खोजें: \(f'(x) = 0\) को हल करें या जहां \(f'(x)\) अपरिभाषित है।
\item
  मैक्सिमा/मिनिमा के लिए परीक्षण: दूसरे व्युत्पन्न परीक्षण का उपयोग करें या समापन
  बिंदुओं की जांच करें।
\item
  परिणाम की व्याख्या करें: उत्तर को मूल संदर्भ में बताएं।
\end{enumerate}

\subsubsection{उदाहरण 1: एक आयत का अधिकतम
क्षेत्रफल}\label{ux909ux926ux939ux930ux923-1-ux90fux915-ux906ux92fux924-ux915-ux905ux927ux915ux924ux92e-ux915ux937ux924ux930ux92bux932}

एक आयत का परिमाप 40 है। कौन से आयाम इसके क्षेत्रफल को अधिकतम करते हैं?

\begin{itemize}
\tightlist
\item
  माना लंबाई \(x\), चौड़ाई \(y\)। बाधा:
  \(2x + 2y = 40 \Rightarrow y = 20 - x\).
\item
  क्षेत्र: \(A = xy = x(20 - x) = 20x - x^2\).- व्युत्पन्न:
  \(A'(x) = 20 - 2x\). 0 के बराबर सेट करें: \(x = 10\).
\item
  फिर \(y = 10\)।
\item
  अधिकतम क्षेत्र: \(100\). आयत एक वर्ग है.
\end{itemize}

\subsubsection{उदाहरण 2: दूरी कम
करना}\label{ux909ux926ux939ux930ux923-2-ux926ux930-ux915ux92e-ux915ux930ux928}

परवलय \(y = x^2\) पर \((0,3)\) के निकटतम बिंदु का पता लगाएं।

\begin{itemize}
\tightlist
\item
  दूरी का वर्ग: \(D(x) = (x-0)^2 + (x^2 - 3)^2\)।
\item
  विस्तार करें:
  \(D(x) = x^2 + (x^2 - 3)^2 = x^2 + x^4 - 6x^2 + 9 = x^4 - 5x^2 + 9\).
\item
  व्युत्पन्न: \(D'(x) = 4x^3 - 10x\). हल करें: \(x(4x^2 - 10) = 0\).
\item
  समाधान: \(x = 0\), \(x = \pm \sqrt{2.5}\)।
\item
  जाँच करने पर \(x = \pm \sqrt{2.5}\) पर न्यूनतम दूरी मिलती है।
\end{itemize}

\subsubsection{उदाहरण 3: अधिकतम आयतन वाला
बॉक्स}\label{ux909ux926ux939ux930ux923-3-ux905ux927ux915ux924ux92e-ux906ux92fux924ux928-ux935ux932-ux92cux915ux938}

बिना शीर्ष वाला एक बॉक्स कार्डबोर्ड के एक चौकोर टुकड़े से 20 सेमी की दूरी पर कोनों से
समान वर्ग काटकर और किनारों को मोड़कर बनाया जाना है। कट का वह आकार ज्ञात करें जो
आयतन को अधिकतम करता है।

\begin{itemize}
\tightlist
\item
  मान लीजिए कट का आकार = \(x\)। फिर आयाम:
  \((20 - 2x) \times (20 - 2x) \times x\).
\item
  वॉल्यूम: \(V(x) = x(20 - 2x)^2\)।
\item
  व्युत्पन्न: \(V'(x) = (20 - 2x)(20 - 6x)\).
\item
  महत्वपूर्ण बिंदु: \(x = 10\) (शून्य मात्रा देता है) या
  \(x = \tfrac{20}{6} \approx 3.33\)।
\item
  \(x \approx 3.33\) पर, वॉल्यूम अधिकतम होता है।
\end{itemize}

\subsubsection{अनुकूलन क्यों मायने रखता
है}\label{ux905ux928ux915ux932ux928-ux915ux92f-ux92eux92fux928-ux930ux916ux924-ux939}

\begin{itemize}
\tightlist
\item
  इंजीनियर इसका उपयोग कुशल संरचनाओं को डिजाइन करने के लिए करते हैं।
\item
  व्यवसाय इसका उपयोग लाभ को अधिकतम करने या लागत को कम करने के लिए करते हैं।
\item
  वैज्ञानिक इसका उपयोग संतुलन चाहने वाली प्राकृतिक प्रणालियों के मॉडल के लिए करते हैं।
\end{itemize}

\subsubsection{व्यायाम}\label{ux935ux92fux92fux92e-10}

\begin{enumerate}
\def\labelenumi{\arabic{enumi}.}
\tightlist
\item
  एक किसान के पास नदी के किनारे एक आयताकार खेत को घेरने के लिए 100 मीटर की बाड़
  है (इसलिए केवल 3 तरफ बाड़ लगाने की आवश्यकता है)। क्षेत्रफल को अधिकतम करने वाले
  आयाम खोजें।
\item
  दो धनात्मक संख्याएँ ज्ञात कीजिए जिनका योग 20 है और जिनका गुणनफल यथासंभव बड़ा है।
\item
  एक सिलेंडर 100 सेमी\(^2\) सामग्री से बनाया जाना है। अधिकतम आयतन के आयाम ज्ञात
  कीजिए।
\item
  10 मीटर लंबे एक तार को दो टुकड़ों में काटा जाता है, एक को वर्गाकार में मोड़ा जाता
  है, दूसरे को वृत्त में मोड़ा जाता है। संलग्न कुल क्षेत्रफल को अधिकतम करने के लिए इसे कैसे
  काटा जाना चाहिए?
\item
  वर्गाकार आधार और आयतन 32 m\(^3\) वाला एक बंद बॉक्स बनाया जाना है। सतह क्षेत्र
  को न्यूनतम करते हुए आयाम खोजें।
\end{enumerate}

\subsection{3.4 अवतलता और विभक्ति
बिंदु}\label{ux905ux935ux924ux932ux924-ux914ux930-ux935ux92dux915ux924-ux92cux926}

डेरिवेटिव हमें न केवल ढलानों के बारे में बताते हैं बल्कि ग्राफ के आकार के बारे में भी बताते
हैं। दूसरा व्युत्पन्न अवतलता को समझने और विभक्ति बिंदुओं की पहचान करने में विशेष रूप से
उपयोगी है।

\subsubsection{अवतलता}\label{ux905ux935ux924ux932ux924}

\begin{itemize}
\item
  यदि \(f''(x) > 0\) है तो एक फ़ंक्शन \(f(x)\) एक अंतराल पर अवतल होता है। ग्राफ़
  एक कप की तरह ऊपर की ओर झुकता है।
\item
  यदि \(f''(x) < 0\) है तो एक फ़ंक्शन \(f(x)\) एक अंतराल पर नीचे की ओर अवतल
  होता है। ग्राफ़ भौंह की तरह नीचे की ओर झुक जाता है।
\end{itemize}

अवतलता बताती है कि किसी फ़ंक्शन का ढलान कैसे बदल रहा है: यदि ढलान बढ़ रहा है, तो
ग्राफ़ अवतल होता है; यदि ढलान कम हो रही है, तो ग्राफ़ नीचे की ओर अवतल है।

\subsubsection{विभक्ति बिंदु}\label{ux935ux92dux915ux924-ux92cux926}

विभक्ति बिंदु ग्राफ़ पर एक बिंदु है जहां अवतलता बदलती है।- यदि \(f''(x) = 0\) या
\(f''(x)\) अपरिभाषित है, तो बिंदु विभक्ति बिंदु के लिए एक उम्मीदवार है। - पुष्टि
करने के लिए, बिंदु के दोनों ओर अवतलता का चिह्न बदलना होगा।

\subsubsection{उदाहरण}\label{ux909ux926ux939ux930ux923-6}

\begin{enumerate}
\def\labelenumi{\arabic{enumi}.}
\item
  \(f(x) = x^3\)

  \begin{itemize}
  \tightlist
  \item
    \(f''(x) = 6x\).
  \item
    \(x = 0\), \(f''(0) = 0\) पर।
  \item
    \(x < 0\) के लिए, \(f''(x) < 0\) → अवतल नीचे।
  \item
    \(x > 0\) के लिए, \(f''(x) > 0\) → अवतल ऊपर।
  \item
    इस प्रकार, \((0,0)\) एक विभक्ति बिंदु है।
  \end{itemize}
\item
  \(f(x) = x^4\)

  \begin{itemize}
  \tightlist
  \item
    \(f''(x) = 12x^2\).
  \item
    \(x = 0\), \(f''(0) = 0\) पर, लेकिन अवतलता का चिह्न नहीं बदलता (हमेशा ≥
    0)।
  \item
    कोई विभक्ति बिंदु नहीं.
  \end{itemize}
\end{enumerate}

\subsubsection{अवतलता और वक्र
रेखाचित्र}\label{ux905ux935ux924ux932ux924-ux914ux930-ux935ux915ux930-ux930ux916ux91aux924ux930}

\begin{itemize}
\tightlist
\item
  यदि \(f'(x) = 0\) और \(f''(x) > 0\) है, तो \(f\) का स्थानीय न्यूनतम है।
\item
  यदि \(f'(x) = 0\) और \(f''(x) < 0\) है, तो \(f\) का स्थानीय अधिकतम है।
\item
  इसे द्वितीय व्युत्पन्न परीक्षण के रूप में जाना जाता है।
\end{itemize}

\subsubsection{यह क्यों मायने रखता
है}\label{ux92fux939-ux915ux92f-ux92eux92fux928-ux930ux916ux924-ux939-1}

अवतलता और विभक्ति बिंदु हमें ग्राफ़ के ``आकार'' को समझने में मदद करते हैं: जहां वे झुकते
हैं, चपटे होते हैं, या मुड़ते हैं। ये विचार वक्र रेखाचित्र, भौतिकी (त्वरण), और अर्थशास्त्र
(घटते रिटर्न) में केंद्रीय हैं।

\subsubsection{व्यायाम}\label{ux935ux92fux92fux92e-11}

\begin{enumerate}
\def\labelenumi{\arabic{enumi}.}
\tightlist
\item
  \(f(x) = x^3 - 3x\) के लिए अवतलता के अंतराल निर्धारित करें। इसके विभक्ति बिन्दु
  ज्ञात कीजिए।
\item
  \(f(x) = \ln(x)\) के लिए, अवतलता और संभावित विभक्ति बिंदुओं की पहचान करें।
\item
  महत्वपूर्ण बिंदुओं को वर्गीकृत करने के लिए \(f(x) = x^2 e^{-x}\) पर दूसरा व्युत्पन्न
  परीक्षण लागू करें।
\item
  अवतलता और विभक्ति बिंदुओं के अंतराल को चिह्नित करते हुए स्केच \(f(x) = \sin x\)।
\item
  बताएं कि \(f(x) = e^x\) में कोई विभक्ति बिंदु क्यों नहीं है।
\end{enumerate}

\subsection{3.5 वक्र
रेखाचित्र}\label{ux935ux915ux930-ux930ux916ux91aux924ux930}

कर्व स्केचिंग किसी फ़ंक्शन के डेरिवेटिव से जानकारी का उपयोग करके उसका ग्राफ़ खींचने की
प्रक्रिया है। कई बिंदुओं की योजना बनाने के बजाय, हम प्रमुख विशेषताओं का विश्लेषण करते
हैं: अंतःखंड, स्पर्शोन्मुख, बढ़ते/घटते अंतराल, और अवतलता।

\subsubsection{कर्व स्केचिंग के लिए
चरण}\label{ux915ux930ux935-ux938ux915ux91aux917-ux915-ux932ux90f-ux91aux930ux923}

\begin{enumerate}
\def\labelenumi{\arabic{enumi}.}
\item
  डोमेन: पहचानें कि फ़ंक्शन कहां परिभाषित है।
\item
  अवरोधन: पता लगाएं कि ग्राफ़ अक्षों को कहां पार करता है।
\item
  स्पर्शोन्मुख:

  \begin{itemize}
  \tightlist
  \item
    लंबवत अनंतस्पर्शी तब होते हैं जहां फ़ंक्शन अपरिभाषित होता है और अनंत की ओर जाता
    है।
  \item
    क्षैतिज या तिरछे अनंतस्पर्शी अंतिम व्यवहार को \(x \to \pm\infty\) के रूप में
    वर्णित करते हैं।
  \end{itemize}
\item
  पहला व्युत्पन्न \(f'(x)\):

  \begin{itemize}
  \tightlist
  \item
    सकारात्मक → कार्य बढ़ रहा है।
  \item
    नकारात्मक → कार्य कम हो रहा है।
  \item
    \(f'(x)\) के शून्य → महत्वपूर्ण बिंदु (संभव अधिकतम/न्यूनतम)।
  \end{itemize}
\item
  दूसरा व्युत्पन्न \(f''(x)\):

  \begin{itemize}
  \tightlist
  \item
    सकारात्मक → अवतल ऊपर।
  \item
    नकारात्मक → अवतल नीचे।
  \item
    शून्य या अपरिभाषित → संभावित विभक्ति बिंदु।
  \end{itemize}
\item
  जानकारी को संयोजित करें: एक स्पष्ट और सटीक ग्राफ़ बनाने के लिए सभी परिणामों का
  उपयोग करें।
\end{enumerate}

\subsubsection{\texorpdfstring{उदाहरण 1:
\(f(x) = x^3 - 3x\)}{उदाहरण 1: f(x) = x\^{}3 - 3x}}\label{ux909ux926ux939ux930ux923-1-fx-x3---3x}

\begin{itemize}
\item
  डोमेन: सभी वास्तविक संख्याएँ।
\item
  अवरोधन: \((0,0)\) पर।
\item
  व्युत्पन्न: \(f'(x) = 3x^2 - 3 = 3(x-1)(x+1)\).

  \begin{itemize}
  \item
    बढ़ रहा है: \((-\infty, -1) \cup (1, \infty)\)।
  \item
    घट रहा है: \((-1, 1)\)।- दूसरा व्युत्पन्न: \(f''(x) = 6x\)।
  \item
    \(x < 0\) के लिए अवतल नीचे, \(x > 0\) के लिए अवतल ऊपर।
  \item
    \((0,0)\) पर विभक्ति बिंदु।
  \end{itemize}
\item
  आकार: \((-1, 2)\) पर स्थानीय अधिकतम के साथ एक S-वक्र, \((1, -2)\) पर
  स्थानीय न्यूनतम।
\end{itemize}

\subsubsection{\texorpdfstring{उदाहरण 2:
\(f(x) = \frac{1}{x}\)}{उदाहरण 2: f(x) = \textbackslash frac\{1\}\{x\}}}\label{ux909ux926ux939ux930ux923-2-fx-frac1x}

\begin{itemize}
\item
  डोमेन: \(x \neq 0\).
\item
  लंबवत अनंतस्पर्शी: \(x = 0\)।
\item
  क्षैतिज अनंतस्पर्शी: \(y = 0\)।
\item
  व्युत्पन्न: \(f'(x) = -\frac{1}{x^2}\) (हमेशा नकारात्मक)। कार्य सदैव घटता जा
  रहा है।
\item
  दूसरा व्युत्पन्न: \(f''(x) = \frac{2}{x^3}\)।

  \begin{itemize}
  \tightlist
  \item
    \(x > 0\) के लिए अवतल।
  \item
    \(x < 0\) के लिए अवतल नीचे।
  \end{itemize}
\item
  ग्राफ़: दो शाखाओं वाला अतिपरवलय।
\end{itemize}

\subsubsection{कर्व स्केचिंग क्यों उपयोगी
है}\label{ux915ux930ux935-ux938ux915ux91aux917-ux915ux92f-ux909ux92aux92fux917-ux939}

\begin{itemize}
\tightlist
\item
  संपूर्ण गणना के बिना कार्यों के समग्र व्यवहार में अंतर्दृष्टि प्रदान करता है।
\item
  कैलकुलस परीक्षा और व्यावहारिक समस्याओं में आवश्यक।
\item
  बीजगणितीय विश्लेषण और ज्यामितीय समझ को जोड़ता है।
\end{itemize}

\subsubsection{व्यायाम}\label{ux935ux92fux92fux92e-12}

\begin{enumerate}
\def\labelenumi{\arabic{enumi}.}
\tightlist
\item
  \(f(x) = x^4 - 2x^2\) का वक्र रेखाचित्र बनाएं। मैक्सिमा, मिनिमा और विभक्ति
  बिंदुओं को पहचानें।
\item
  विश्लेषण करें और स्केच करें \(f(x) = \ln(x)\)। अंतःखंड, स्पर्शोन्मुख और अवतलता
  दिखाएँ।
\item
  \(f(x) = e^{-x}\) के लिए, वृद्धि/क्षय, स्पर्शोन्मुखता और अवतलता का वर्णन करें।
\item
  अंतराल \((- \pi, \pi)\) पर \(f(x) = \tan x\) का ग्राफ़ बनाएं। स्पर्शोन्मुख
  चिह्न लगाएं.
\item
  \(f(x) = x^3 - 6x^2 + 9x\) के महत्वपूर्ण बिंदुओं को वर्गीकृत करने के लिए पहले और
  दूसरे व्युत्पन्न परीक्षणों का उपयोग करें।
\end{enumerate}

\section{भाग II. इंटीग्रल}\label{ux92dux917-ii.-ux907ux91fux917ux930ux932}

\section{अध्याय 4. प्रतिअवकलज और निश्चित
समाकलन}\label{ux905ux927ux92fux92f-4.-ux92aux930ux924ux905ux935ux915ux932ux91c-ux914ux930-ux928ux936ux91aux924-ux938ux92eux915ux932ux928}

\subsection{4.1 अनिश्चित
इंटीग्रल}\label{ux905ux928ux936ux91aux924-ux907ux91fux917ux930ux932}

अनिश्चितकालीन अभिन्न विभेदन की विपरीत प्रक्रिया है। यदि एक व्युत्पन्न उपाय बदलता
है, तो एक अभिन्न अंग अपने परिवर्तन की दर से मूल कार्य को पुनर्प्राप्त करता है।

\subsubsection{परिभाषा}\label{ux92aux930ux92dux937-4}

यदि \(F'(x) = f(x)\), तो

\[
\int f(x)\,dx = F(x) + C,
\]

जहां \(C\) एकीकरण का स्थिरांक है।

प्रत्येक अनिश्चित अभिन्न अंग कार्यों के एक परिवार का प्रतिनिधित्व करता है जो केवल एक
स्थिरांक से भिन्न होता है, क्योंकि भेदभाव स्थिरांक को समाप्त कर देता है।

\subsubsection{बुनियादी नियम}\label{ux92cux928ux92fux926-ux928ux92fux92e}

\begin{enumerate}
\def\labelenumi{\arabic{enumi}.}
\tightlist
\item
  निरंतर नियम
\end{enumerate}

\[
\int c\,dx = cx + C.
\]

\begin{enumerate}
\def\labelenumi{\arabic{enumi}.}
\setcounter{enumi}{1}
\tightlist
\item
  शक्ति नियम
\end{enumerate}

\[
\int x^n\,dx = \frac{x^{n+1}}{n+1} + C, \quad n \neq -1.
\]

\begin{enumerate}
\def\labelenumi{\arabic{enumi}.}
\setcounter{enumi}{2}
\tightlist
\item
  योग नियम
\end{enumerate}

\[
\int \big(f(x) + g(x)\big)\,dx = \int f(x)\,dx + \int g(x)\,dx.
\]

\begin{enumerate}
\def\labelenumi{\arabic{enumi}.}
\setcounter{enumi}{3}
\tightlist
\item
  लगातार एकाधिक नियम
\end{enumerate}

\[
\int c f(x)\,dx = c \int f(x)\,dx.
\]

\subsubsection{सामान्य
इंटीग्रल}\label{ux938ux92eux928ux92f-ux907ux91fux917ux930ux932}

\begin{itemize}
\tightlist
\item
  \(\int e^x dx = e^x + C\)
\item
  \(\int \sin x dx = -\cos x + C\)
\item
  \(\int \cos x dx = \sin x + C\)
\item
  \(\int \frac{1}{x} dx = \ln|x| + C\)
\end{itemize}

\subsubsection{उदाहरण}\label{ux909ux926ux939ux930ux923-7}

\begin{enumerate}
\def\labelenumi{\arabic{enumi}.}
\item
  \(\int (3x^2 - 4)\,dx = x^3 - 4x + C\).
\item
  \(\int \cos(2x)\,dx = \tfrac{1}{2}\sin(2x) + C\).
\item
  \(\int \frac{1}{x}\,dx = \ln|x| + C\).
\end{enumerate}

\subsubsection{व्याख्या}\label{ux935ux92fux916ux92f-1}

-अनिश्चित समाकलन प्रतिअवकलज होते हैं। - वे निश्चित अभिन्नों की नींव हैं, जो क्षेत्र,
दूरी और द्रव्यमान जैसी संचित मात्राओं को मापते हैं।- लागू संदर्भों में, एकीकरण हमें दरों से
वापस कुल तक जाने की अनुमति देता है।

\subsubsection{व्यायाम}\label{ux935ux92fux92fux92e-13}

\begin{enumerate}
\def\labelenumi{\arabic{enumi}.}
\tightlist
\item
  \(\int (5x^4 + 2x)\,dx\) खोजें।
\item
  \(\int (e^x + 3)\,dx\) की गणना करें।
\item
  एकीकरण का उपयोग करके \(f'(x) = 6x\) का सामान्य समाधान खोजें।
\item
  \(\int \frac{2}{x}\,dx\) का मूल्यांकन करें।
\item
  यदि वेग \(v(t) = 4t\) है, तो स्थिति फलन \(s(t)\) ज्ञात करें।
\end{enumerate}

\subsection{4.2 क्षेत्र के रूप में निश्चित
अभिन्न}\label{ux915ux937ux924ux930-ux915-ux930ux92a-ux92e-ux928ux936ux91aux924-ux905ux92dux928ux928}

जबकि अनिश्चित अभिन्न अंग प्रतिअवकलन के परिवारों का प्रतिनिधित्व करते हैं, निश्चित
अभिन्न एक संख्यात्मक मान देता है: दो बिंदुओं के बीच एक वक्र के नीचे संचित क्षेत्र।

\subsubsection{परिभाषा}\label{ux92aux930ux92dux937-5}

\([a, b]\) पर परिभाषित फ़ंक्शन \(f(x)\) के लिए, निश्चित अभिन्न अंग है

\[
\int_a^b f(x)\,dx = \lim_{n \to \infty} \sum_{i=1}^n f(x_i^-) \,\Delta x,
\]

जहां अंतराल \([a, b]\) को चौड़ाई \(\Delta x\) के \(n\) उपअंतराल में विभाजित
किया गया है, और \(x_i^-\) प्रत्येक उपअंतराल में एक नमूना बिंदु है।

यह रीमैन रकम की सीमा है।

\subsubsection{ज्यामितीय
व्याख्या}\label{ux91cux92fux92eux924ux92f-ux935ux92fux916ux92f}

\begin{itemize}
\tightlist
\item
  यदि \([a, b]\) पर \(f(x) \geq 0\) है, तो \(\int_a^b f(x)\,dx\) \(x=a\)
  से \(x=b\) तक वक्र \(y = f(x)\) के नीचे के क्षेत्र के बराबर है।
\item
  यदि \(f(x)\) \(x\)-अक्ष से नीचे चला जाता है, तो इंटीग्रल हस्ताक्षरित क्षेत्र की
  गणना करता है: अक्ष के नीचे के क्षेत्रों को नकारात्मक माना जाता है।
\end{itemize}

\subsubsection{निश्चित अभिन्न के
गुण}\label{ux928ux936ux91aux924-ux905ux92dux928ux928-ux915-ux917ux923}

\begin{enumerate}
\def\labelenumi{\arabic{enumi}.}
\tightlist
\item
  अंतराल पर संयोजकता
\end{enumerate}

\[
\int_a^c f(x)\,dx = \int_a^b f(x)\,dx + \int_b^c f(x)\,dx.
\]

\begin{enumerate}
\def\labelenumi{\arabic{enumi}.}
\setcounter{enumi}{1}
\tightlist
\item
  उलटने की सीमा
\end{enumerate}

\[
\int_a^b f(x)\,dx = -\int_b^a f(x)\,dx.
\]

\begin{enumerate}
\def\labelenumi{\arabic{enumi}.}
\setcounter{enumi}{2}
\tightlist
\item
  शून्य-चौड़ाई अंतराल
\end{enumerate}

\[
\int_a^a f(x)\,dx = 0.
\]

\begin{enumerate}
\def\labelenumi{\arabic{enumi}.}
\setcounter{enumi}{3}
\tightlist
\item
  रैखिकता
\end{enumerate}

\[
\int_a^b \big( cf(x) + g(x)\big)\,dx = c\int_a^b f(x)\,dx + \int_a^b g(x)\,dx.
\]

\subsubsection{उदाहरण}\label{ux909ux926ux939ux930ux923-8}

\begin{enumerate}
\def\labelenumi{\arabic{enumi}.}
\item
  \(\int_0^2 x\,dx = \left[\tfrac{1}{2}x^2\right]_0^2 = 2.\) यह रेखा
  \(y=x\) के नीचे एक समकोण त्रिभुज का क्षेत्रफल है।
\item
  \(\int_{-1}^1 x^3\,dx = 0.\) विषम फ़ंक्शन \(x^3\) में सममित क्षेत्र हैं जो रद्द
  हो जाते हैं।
\item
  \(\int_0^\pi \sin x\,dx = 2.\) यह साइन वक्र के एक आर्च के नीचे के क्षेत्र के
  बराबर है।
\end{enumerate}

\subsubsection{यह क्यों मायने रखता
है}\label{ux92fux939-ux915ux92f-ux92eux92fux928-ux930ux916ux924-ux939-2}

\begin{itemize}
\tightlist
\item
  निश्चित समाकलन संचित मात्राओं को मापते हैं: दूरी, द्रव्यमान, ऊर्जा, संभाव्यता।
\item
  वे बीजगणितीय गणना को ज्यामितीय अंतर्ज्ञान से जोड़ते हैं।
\item
  अगला चरण कैलकुलस का मौलिक प्रमेय है, जो निश्चित समाकलन को प्रतिअवकलन से जोड़ता
  है।
\end{itemize}

\subsubsection{व्यायाम}\label{ux935ux92fux92fux92e-14}

\begin{enumerate}
\def\labelenumi{\arabic{enumi}.}
\tightlist
\item
  \(\int_0^3 (2x+1)\,dx\) की गणना करें।
\item
  \(y = x^2\) और \(x\)-अक्ष के बीच \(x = 0\) से \(x = 2\) तक का क्षेत्र ज्ञात
  कीजिए।
\item
  \(\int_{-2}^2 (x^2 - 1)\,dx\) का मूल्यांकन करें।
\item
  दिखाएँ कि \(\int_{-a}^a f(x)\,dx = 0\) यदि \(f(x)\) विषम है।
\item
  \(n=4\) उपअंतराल और सही समापन बिंदु के साथ रीमैन योग का उपयोग करके अनुमानित
  \(\int_0^1 e^x\,dx\)।
\end{enumerate}

\subsection{4.3 कैलकुलस का मौलिक प्रमेयकैलकुलस का मौलिक प्रमेय (FTC) कैलकुलस के दो
मुख्य विचारों को जोड़ता है: विभेदन और एकीकरण। यह दर्शाता है कि क्षेत्रों का पता
लगाना और परिवर्तन की दर का पता लगाना एक ही सिक्के के दो पहलू
हैं।}\label{ux915ux932ux915ux932ux938-ux915-ux92eux932ux915-ux92aux930ux92eux92fux915ux932ux915ux932ux938-ux915-ux92eux932ux915-ux92aux930ux92eux92f-ftc-ux915ux932ux915ux932ux938-ux915-ux926-ux92eux916ux92f-ux935ux91aux930-ux915-ux91cux921ux924-ux939-ux935ux92dux926ux928-ux914ux930-ux90fux915ux915ux930ux923-ux92fux939-ux926ux930ux936ux924-ux939-ux915-ux915ux937ux924ux930-ux915-ux92aux924-ux932ux917ux928-ux914ux930-ux92aux930ux935ux930ux924ux928-ux915-ux926ux930-ux915-ux92aux924-ux932ux917ux928-ux90fux915-ux939-ux938ux915ux915-ux915-ux926-ux92aux939ux932-ux939}

\subsubsection{भाग 1: अभिन्न का
भेद}\label{ux92dux917-1-ux905ux92dux928ux928-ux915-ux92dux926}

यदि \(f\) \([a, b]\) पर निरंतर है, तो परिभाषित करें

\[
F(x) = \int_a^x f(t)\,dt.
\]

फिर \(F\) अवकलनीय है, और

\[
F'(x) = f(x).
\]

शब्दों में: संचित क्षेत्र फलन का व्युत्पन्न ही मूल फलन है।

\subsubsection{भाग 2: निश्चित समाकलनों का
मूल्यांकन}\label{ux92dux917-2-ux928ux936ux91aux924-ux938ux92eux915ux932ux928-ux915-ux92eux932ux92fux915ux928}

यदि \(f\) \([a, b]\) पर निरंतर है और \(F\) \(f\) का कोई प्रतिव्युत्पन्न है, तो

\[
\int_a^b f(x)\,dx = F(b) - F(a).
\]

यह हमें बताता है कि हम रीमैन योगों की सीमाओं की गणना करने के बजाय केवल एक
प्रतिअवकलन ढूंढकर निश्चित अभिन्नों का मूल्यांकन कर सकते हैं।

\subsubsection{उदाहरण}\label{ux909ux926ux939ux930ux923-9}

\begin{enumerate}
\def\labelenumi{\arabic{enumi}.}
\item
  \(\int_0^2 x^2\,dx\).

  \begin{itemize}
  \tightlist
  \item
    प्रतिअवकलन: \(F(x) = \tfrac{1}{3}x^3\).
  \item
    एफटीसी लागू करें: \(F(2) - F(0) = \tfrac{8}{3} - 0 = \tfrac{8}{3}.\)
  \end{itemize}
\item
  यदि \(F(x) = \int_1^x \cos t \, dt\), तो \(F'(x) = \cos x\)।
\item
  \(\int_1^4 \frac{1}{x}\,dx\).

  \begin{itemize}
  \tightlist
  \item
    प्रतिव्युत्पन्न: \(\ln|x|\)।
  \item
    एफटीसी लागू करें: \(\ln 4 - \ln 1 = \ln 4.\)
  \end{itemize}
\end{enumerate}

\subsubsection{एफटीसी क्यों मायने रखता
है}\label{ux90fux92bux91fux938-ux915ux92f-ux92eux92fux928-ux930ux916ux924-ux939}

\begin{itemize}
\tightlist
\item
  यह एकीकरण को एक सीमा प्रक्रिया से व्यावहारिक गणना में बदल देता है।
\item
  यह पुष्टि करता है कि विभेदन और एकीकरण व्युत्क्रम संक्रियाएँ हैं।
\item
  यह केंद्रीय प्रमेय है जो कैलकुलस को गणित, विज्ञान और इंजीनियरिंग में उपयोगी बनाता
  है।
\end{itemize}

\subsubsection{व्यायाम}\label{ux935ux92fux92fux92e-15}

\begin{enumerate}
\def\labelenumi{\arabic{enumi}.}
\tightlist
\item
  FTC का उपयोग करके \(\int_0^3 (2x+1)\,dx\) का मूल्यांकन करें।
\item
  यदि \(F(x) = \int_0^x e^t\,dt\), तो \(F'(x)\) खोजें।
\item
  \(\int_0^\pi \sin x \, dx\) की गणना करें।
\item
  दिखाएँ कि यदि \(f'(x) = g(x)\), तो \(\int_a^b g(x)\,dx = f(b) - f(a)\)।
\item
  यह समझाने के लिए FTC का उपयोग करें कि \(y = \cos x\) से \(0\) से \(\pi/2\)
  तक का क्षेत्र 1 के बराबर क्यों है।
\end{enumerate}

\subsection{4.4 इंटीग्रल्स के
गुण}\label{ux907ux91fux917ux930ux932ux938-ux915-ux917ux923}

निश्चित इंटीग्रल में कई महत्वपूर्ण गुण हैं जो इसे अनुप्रयोगों में लचीला और शक्तिशाली बनाते
हैं। ये गुण योगों की सीमा के रूप में परिभाषा और कैलकुलस के मौलिक प्रमेय से अनुसरण करते हैं।

\subsubsection{रैखिकता}\label{ux930ux916ux915ux924}

फ़ंक्शन \(f(x)\) और \(g(x)\), और स्थिरांक \(c, d\) के लिए:

\[
\int_a^b \big(c f(x) + d g(x)\big)\,dx = c \int_a^b f(x)\,dx + d \int_a^b g(x)\,dx.
\]

यह हमें जटिल अभिन्नों को सरल भागों में तोड़ने की अनुमति देता है।

\subsubsection{अंतरालों पर
संवेदनशीलता}\label{ux905ux924ux930ux932-ux92aux930-ux938ux935ux926ux928ux936ux932ux924}

यदि \(a < c < b\), तो

\[
\int_a^b f(x)\,dx = \int_a^c f(x)\,dx + \int_c^b f(x)\,dx.
\]

हम टुकड़े-टुकड़े करके अभिन्नों की गणना कर सकते हैं।

\subsubsection{सीमाओं का उलटा
होना}\label{ux938ux92eux913-ux915-ux909ux932ux91f-ux939ux928}

\[
\int_a^b f(x)\,dx = -\int_b^a f(x)\,dx.
\]

सीमाओं की अदला-बदली से अभिन्न का चिह्न बदल जाता है।

\subsubsection{तुलना संपत्ति}\label{ux924ux932ux928-ux938ux92aux924ux924}

यदि \([a, b]\) में सभी \(x\) के लिए \(f(x) \leq g(x)\) है, तो

\[
\int_a^b f(x)\,dx \leq \int_a^b g(x)\,dx.
\]इससे हम प्रत्यक्ष गणना के बिना क्षेत्रों की तुलना कर सकते हैं।

\subsubsection{पूर्ण मूल्य
असमानता}\label{ux92aux930ux923-ux92eux932ux92f-ux905ux938ux92eux928ux924}

\[
\left| \int_a^b f(x)\,dx \right| \leq \int_a^b |f(x)|\,dx.
\]

यह गुण विश्लेषण और अभिसरण परीक्षणों में आवश्यक है।

\subsubsection{समरूपता}\label{ux938ux92eux930ux92aux924}

\begin{itemize}
\item
  यदि \(f(x)\) सम है (\(y\)-अक्ष के बारे में सममित):

  \[
  \int_{-a}^a f(x)\,dx = 2\int_0^a f(x)\,dx.
  \]
\item
  यदि \(f(x)\) विषम है (मूल के बारे में सममित):

  \[
  \int_{-a}^a f(x)\,dx = 0.
  \]
\end{itemize}

\subsubsection{उदाहरण}\label{ux909ux926ux939ux930ux923-10}

\begin{enumerate}
\def\labelenumi{\arabic{enumi}.}
\item
  \(\int_0^2 (3x^2 + 4)\,dx = \int_0^2 3x^2\,dx + \int_0^2 4\,dx = 8 + 8 = 16.\)
\item
  चूँकि \(f(x) = x^3\) विषम है, \(\int_{-1}^1 x^3\,dx = 0.\)
\item
  चूँकि \(f(x) = x^2\) सम है,
  \(\int_{-2}^2 x^2\,dx = 2\int_0^2 x^2\,dx = 2\cdot \tfrac{8}{3} = \tfrac{16}{3}.\)
\end{enumerate}

\subsubsection{ये गुण क्यों मायने रखते
हैं}\label{ux92f-ux917ux923-ux915ux92f-ux92eux92fux928-ux930ux916ux924-ux939}

\begin{itemize}
\tightlist
\item
  वे गणनाओं को सरल बनाते हैं।
\item
  वे कार्यों की ज्यामितीय और समरूपता विशेषताओं को प्रकट करते हैं।
\item
  वे अधिक उन्नत विश्लेषण के लिए सैद्धांतिक उपकरण प्रदान करते हैं।
\end{itemize}

\subsubsection{व्यायाम}\label{ux935ux92fux92fux92e-16}

\begin{enumerate}
\def\labelenumi{\arabic{enumi}.}
\tightlist
\item
  \(\int_{-5}^5 (x^4 - x^3)\,dx\) का मूल्यांकन करने के लिए समरूपता का उपयोग
  करें।
\item
  दिखाएँ कि
  \(\int_1^4 (2x+3)\,dx = \int_1^2 (2x+3)\,dx + \int_2^4 (2x+3)\,dx\)।
\item
  \(\int_0^\pi \sin(x)\,dx\) का मूल्यांकन करें और
  \(\int_{-\pi}^\pi \sin(x)\,dx\) से तुलना करें।
\item
  सिद्ध करें कि यदि \([a, b]\) पर \(f(x) \geq 0\) है, तो
  \(\int_a^b f(x)\,dx \geq 0\)।
\item
  सम/विषम गुणों का उपयोग करके \(\int_{-3}^3 (x^2 + 1)\,dx\) की गणना करें।
\end{enumerate}

\section{अध्याय 5. एकीकरण की
तकनीकें}\label{ux905ux927ux92fux92f-5.-ux90fux915ux915ux930ux923-ux915-ux924ux915ux928ux915}

\subsection{5.1 प्रतिस्थापन}\label{ux92aux930ux924ux938ux925ux92aux928}

एकीकरण की सबसे उपयोगी तकनीकों में से एक प्रतिस्थापन विधि है, जिसे -यू-प्रतिस्थापन-
भी कहा जाता है। यह डेरिवेटिव के लिए श्रृंखला नियम की विपरीत प्रक्रिया है।

\subsubsection{विचार}\label{ux935ux91aux930}

यदि किसी इंटीग्रल में एक समग्र फ़ंक्शन होता है, तो हम चर को बदलकर इसे सरल बना सकते
हैं।

औपचारिक रूप से, यदि \(u = g(x)\) एक अवकलनीय फलन है, तो

\[
\int f(g(x)) g'(x)\,dx = \int f(u)\,du.
\]

यह प्रतिस्थापन अभिन्न का मूल्यांकन करना आसान बनाता है।

\subsubsection{प्रतिस्थापन के लिए
कदम}\label{ux92aux930ux924ux938ux925ux92aux928-ux915-ux932ux90f-ux915ux926ux92e}

\begin{enumerate}
\def\labelenumi{\arabic{enumi}.}
\tightlist
\item
  एक आंतरिक फ़ंक्शन \(u = g(x)\) की पहचान करें जिसका व्युत्पन्न भी इंटीग्रैंड में दिखाई
  देता है।
\item
  \(du = g'(x)\,dx\) की गणना करें।
\item
  इंटीग्रल को \(u\) के संदर्भ में फिर से लिखें।
\item
  \(u\) के संबंध में एकीकृत करें।
\item
  \(u = g(x)\) को वापस प्रतिस्थापित करें।
\end{enumerate}

\subsubsection{उदाहरण}\label{ux909ux926ux939ux930ux923-11}

\begin{enumerate}
\def\labelenumi{\arabic{enumi}.}
\item
  सरल प्रतिस्थापन

  \[
  \int 2x \cos(x^2)\,dx
  \]

  मान लीजिए \(u = x^2\), तो \(du = 2x\,dx\)। फिर इंटीग्रल
  \(\int \cos u \,du = \sin u + C = \sin(x^2) + C\) हो जाता है।
\item
  लघुगणकीय मामला

  \[
  \int \frac{2x}{x^2+1}\,dx
  \]

  मान लीजिए \(u = x^2 + 1\), तो \(du = 2x\,dx\)। फिर इंटीग्रल
  \(\int \frac{1}{u}\,du = \ln|u| + C = \ln(x^2+1) + C\) हो जाता है।
\item
  त्रिकोणमितीय प्रतिस्थापन

  \[
  \int \sin(3x)\,dx
  \]

  मान लीजिए \(u = 3x\), इसलिए \(du = 3\,dx\), इसलिए
  \(dx = \frac{du}{3}\)।इंटीग्रल
  \(\tfrac{1}{3}\int \sin u\,du = -\tfrac{1}{3}\cos u + C = -\tfrac{1}{3}\cos(3x) + C\)
  बन जाता है।
\end{enumerate}

\subsubsection{प्रतिस्थापन के साथ निश्चित
समाकलन}\label{ux92aux930ux924ux938ux925ux92aux928-ux915-ux938ux925-ux928ux936ux91aux924-ux938ux92eux915ux932ux928}

निश्चित अभिन्नों का मूल्यांकन करते समय, हमें सीमाएँ भी बदलनी होंगी:

\[
\int_a^b f(g(x)) g'(x)\,dx = \int_{g(a)}^{g(b)} f(u)\,du.
\]

उदाहरण:

\[
\int_0^1 2x e^{x^2}\,dx.
\]

चलो \(u = x^2\), \(du = 2x\,dx\)। सीमाएँ: जब \(x=0, u=0\); जब
\(x=1, u=1\). तो अभिन्न हो जाता है

\[
\int_0^1 e^u\,du = e - 1.
\]

\subsubsection{व्यायाम}\label{ux935ux92fux92fux92e-17}

\begin{enumerate}
\def\labelenumi{\arabic{enumi}.}
\tightlist
\item
  \(\int (x^2+1)^5 (2x)\,dx\) का मूल्यांकन करें।
\item
  \(\int \frac{\cos x}{\sin x}\,dx\) की गणना करें।
\item
  प्रतिस्थापन का उपयोग करके \(\int_0^\pi \sin(2x)\,dx\) का मूल्यांकन करें।
\item
  \(\int e^{3x}\,dx\) खोजें।
\item
  \(u = 1+x^2\) देकर \(\int \frac{1}{\sqrt{1+x^2}}\,dx\) की गणना करें।
\end{enumerate}

\subsection{5.2 भागों द्वारा
एकीकरण}\label{ux92dux917-ux926ux935ux930-ux90fux915ux915ux930ux923}

भागों द्वारा एकीकरण एक ऐसी तकनीक है जो डेरिवेटिव के लिए उत्पाद नियम से आती है। यह
उन कार्यों के उत्पादों से जुड़े इंटीग्रल्स का मूल्यांकन करने में मदद करता है जिन्हें अकेले
प्रतिस्थापन द्वारा आसानी से नियंत्रित नहीं किया जाता है।

\subsubsection{सूत्र}\label{ux938ux924ux930}

उत्पाद नियम से:

\[
\frac{d}{dx}[u(x)v(x)] = u'(x)v(x) + u(x)v'(x).
\]

दोनों पक्षों को एकीकृत करने से भागों द्वारा एकीकरण का सूत्र प्राप्त होता है:

\[
\int u\,dv = uv - \int v\,du.
\]

यहाँ:

\begin{itemize}
\tightlist
\item
  \(u\) = विभेदित करने के लिए चुना गया एक फ़ंक्शन,
\item
  \(dv\) = एकीकृत किया जाने वाला इंटीग्रैंड का शेष भाग।
\end{itemize}

\subsubsection{\texorpdfstring{\(u\) और \(dv\)
चुनना}{u और dv चुनना}}\label{u-ux914ux930-dv-ux91aux928ux928}

एक सामान्य दिशानिर्देश LIATE (लघुगणक, व्युत्क्रम त्रिकोणमिति, बीजगणितीय,
त्रिकोणमितीय, घातांक) है।

\begin{itemize}
\tightlist
\item
  वर्तमान श्रेणी में से \(u\) चुनें।
\item
  बाकी के रूप में \(dv\) चुनें।
\end{itemize}

\subsubsection{उदाहरण}\label{ux909ux926ux939ux930ux923-12}

\begin{enumerate}
\def\labelenumi{\arabic{enumi}.}
\tightlist
\item
  बहुपद × घातांक
\end{enumerate}

\[
\int x e^x\,dx
\]

मान लीजिए \(u = x\), \(dv = e^x dx\)। फिर \(du = dx\), \(v = e^x\)।

\[
\int x e^x\,dx = x e^x - \int e^x dx = x e^x - e^x + C.
\]

\begin{enumerate}
\def\labelenumi{\arabic{enumi}.}
\setcounter{enumi}{1}
\tightlist
\item
  बहुपद × त्रिकोणमिति
\end{enumerate}

\[
\int x \cos x\,dx
\]

चलो \(u = x\), \(dv = \cos x dx\)। फिर \(du = dx\), \(v = \sin x\)।

\[
\int x \cos x\,dx = x \sin x - \int \sin x dx = x \sin x + \cos x + C.
\]

\begin{enumerate}
\def\labelenumi{\arabic{enumi}.}
\setcounter{enumi}{2}
\tightlist
\item
  लघुगणक
\end{enumerate}

\[
\int \ln x\,dx
\]

मान लीजिए \(u = \ln x\), \(dv = dx\)। फिर \(du = \frac{1}{x}dx\),
\(v = x\)।

\[
\int \ln x\,dx = x \ln x - \int 1 dx = x \ln x - x + C.
\]

\subsubsection{निश्चित अभिन्न
उदाहरण}\label{ux928ux936ux91aux924-ux905ux92dux928ux928-ux909ux926ux939ux930ux923}

\[
\int_0^1 x e^x\,dx
\]

पिछले परिणाम का उपयोग करना: \(\int x e^x dx = (x-1)e^x\)। मूल्यांकन करें:

\[
\big[(x-1)e^x\big]_0^1 = (0)e^1 - (-1)e^0 = 0 + 1 = 1.
\]

\subsubsection{यह क्यों मायने रखता
है}\label{ux92fux939-ux915ux92f-ux92eux92fux928-ux930ux916ux924-ux939-3}

जब प्रतिस्थापन विफल हो जाता है, तो भागों द्वारा एकीकरण महत्वपूर्ण होता है, विशेष
रूप से लघुगणक, व्युत्क्रम त्रिकोणमितीय कार्यों और घातांक या त्रिकोणमिति कार्यों वाले
बहुपद वाले उत्पादों के साथ।

\subsubsection{व्यायाम}\label{ux935ux92fux92fux92e-18}

\begin{enumerate}
\def\labelenumi{\arabic{enumi}.}
\tightlist
\item
  \(\int x \sin x\,dx\) का मूल्यांकन करें।
\item
  \(\int e^x \cos x\,dx\) खोजें।
\item
  \(\int_1^2 \ln x\,dx\) की गणना करें।
\item
  \(\int x^2 e^x\,dx\) का मूल्यांकन करें।5.
  \(\int \arctan x\,dx = x\arctan x - \tfrac{1}{2}\ln(1+x^2) + C\) दिखाने
  के लिए भागों द्वारा एकीकरण का उपयोग करें।
\end{enumerate}

\subsection{5.3 त्रिकोणमितीय समाकलन और
प्रतिस्थापन}\label{ux924ux930ux915ux923ux92eux924ux92f-ux938ux92eux915ux932ux928-ux914ux930-ux92aux930ux924ux938ux925ux92aux928}

कई अभिन्नों में त्रिकोणमितीय कार्य शामिल होते हैं। इन्हें अक्सर सर्वसमिकाओं का उपयोग
करके या विशेष प्रतिस्थापन करके सरल बनाया जा सकता है।

\subsubsection{त्रिकोणमितीय
इंटीग्रल}\label{ux924ux930ux915ux923ux92eux924ux92f-ux907ux91fux917ux930ux932}

\begin{enumerate}
\def\labelenumi{\arabic{enumi}.}
\tightlist
\item
  ज्या और कोज्या की घातें
\end{enumerate}

\begin{itemize}
\tightlist
\item
  यदि साइन की शक्ति विषम है: एक \(\sin x\) बचाएं, बाकी को
  \(\sin^2x = 1 - \cos^2x\) से बदलें, और \(u = \cos x\) प्रतिस्थापित करें।
\item
  यदि कोसाइन की शक्ति विषम है: एक \(\cos x\) बचाएं, शेष को
  \(\cos^2x = 1 - \sin^2x\) से बदलें, और \(u = \sin x\) प्रतिस्थापित करें।
\item
  यदि दोनों सम हैं: अर्ध-कोण सर्वसमिकाओं का उपयोग करें।
\end{itemize}

उदाहरण:

\[
\int \sin^3x \cos x \, dx
\]

मान लीजिए \(u = \sin x\), \(du = \cos x\,dx\):

\[
\int u^3\,du = \tfrac{u^4}{4} + C = \tfrac{\sin^4x}{4} + C.
\]

\begin{enumerate}
\def\labelenumi{\arabic{enumi}.}
\setcounter{enumi}{1}
\tightlist
\item
  विभिन्न कोणों से ज्या और कोज्या के गुणनफल उत्पाद-से-योग फ़ार्मुलों का उपयोग करें:
\end{enumerate}

\[
\sin A \cos B = \tfrac{1}{2}[\sin(A+B) + \sin(A-B)].
\]

उदाहरण:

\[
\int \sin(2x)\cos(3x)\,dx = \tfrac{1}{2}\int [\sin(5x) - \sin(x)]\,dx.
\]

\begin{enumerate}
\def\labelenumi{\arabic{enumi}.}
\setcounter{enumi}{2}
\tightlist
\item
  छेदक और स्पर्शरेखा की शक्तियाँ
\end{enumerate}

\begin{itemize}
\tightlist
\item
  यदि सेकेंट की शक्ति सम है: \(\sec^2x\) सहेजें, शेष को \(\sec^2x = 1 + \tan^2x\)
  से बदलें, और \(u = \tan x\) प्रतिस्थापित करें।
\item
  यदि स्पर्शरेखा की शक्ति विषम है: \(\sec^2x\) सहेजें, शेष को
  \(\tan^2x = \sec^2x - 1\) से बदलें, और \(u = \tan x\) प्रतिस्थापित करें।
\end{itemize}

उदाहरण:

\[
\int \tan^3x \sec^2x \, dx
\]

चलो \(u = \tan x\), \(du = \sec^2x\,dx\):

\[
\int u^3\,du = \tfrac{u^4}{4} + C = \tfrac{\tan^4x}{4} + C.
\]

\subsubsection{त्रिकोणमितीय
प्रतिस्थापन}\label{ux924ux930ux915ux923ux92eux924ux92f-ux92aux930ux924ux938ux925ux92aux928}

\(\sqrt{a^2 - x^2}\), \(\sqrt{a^2 + x^2}\), या \(\sqrt{x^2 - a^2}\) से जुड़े
इंटीग्रल के लिए, विशेष प्रतिस्थापन का उपयोग करें:

\begin{enumerate}
\def\labelenumi{\arabic{enumi}.}
\tightlist
\item
  \(x = a \sin \theta\), \(\sqrt{a^2 - x^2}\) के लिए।
\item
  \(x = a \tan \theta\), \(\sqrt{a^2 + x^2}\) के लिए।
\item
  \(x = a \sec \theta\), \(\sqrt{x^2 - a^2}\) के लिए।
\end{enumerate}

उदाहरण:

\[
\int \sqrt{a^2 - x^2}\,dx
\]

मान लीजिए \(x = a\sin\theta\), तो \(dx = a\cos\theta\,d\theta\):

\[
\int \sqrt{a^2 - a^2\sin^2\theta}(a\cos\theta\,d\theta) = \int a^2 \cos^2\theta \, d\theta.
\]

अर्ध-कोण सर्वसमिकाओं का उपयोग करके सरल बनाएं।

\subsubsection{ये तकनीकें क्यों मायने रखती
हैं}\label{ux92f-ux924ux915ux928ux915-ux915ux92f-ux92eux92fux928-ux930ux916ux924-ux939}

\begin{itemize}
\tightlist
\item
  वे कठिन बीजगणितीय रूपों को प्रबंधनीय त्रिकोणमितीय रूपों में परिवर्तित करते हैं।
\item
  वे क्षेत्रों, आयतन और चाप लंबाई से संबंधित समस्याओं में विशेष रूप से उपयोगी होते हैं।
\item
  वे उन्नत एकीकरण विधियों के लिए आधार तैयार करते हैं।
\end{itemize}

\subsubsection{व्यायाम}\label{ux935ux92fux92fux92e-19}

\begin{enumerate}
\def\labelenumi{\arabic{enumi}.}
\tightlist
\item
  \(\int \sin^4x \cos^2x \, dx\) का मूल्यांकन करें।
\item
  \(\int \sin(5x)\cos(2x)\,dx\) की गणना करें।
\item
  \(\int \tan^2x \sec^2x \, dx\) का मूल्यांकन करें।
\item
  प्रतिस्थापन का उपयोग करके \(\int \sqrt{9 - x^2}\,dx\) खोजें।
\item
  \(x = a\tan\theta\) का उपयोग करके दिखाएँ कि
  \(\int \frac{dx}{\sqrt{x^2 + a^2}} = \ln|x + \sqrt{x^2 + a^2}| + C\)।
\end{enumerate}

\subsection{5.4 आंशिक भिन्नतर्कसंगत कार्यों (बहुपदों के अनुपात) को एकीकृत करते समय,
एक शक्तिशाली विधि आंशिक अंश अपघटन है। यह तकनीक एक जटिल भिन्न को सरल भिन्नों के
योग के रूप में व्यक्त करती है जिन्हें एकीकृत करना आसान होता
है।}\label{ux906ux936ux915-ux92dux928ux928ux924ux930ux915ux938ux917ux924-ux915ux930ux92f-ux92cux939ux92aux926-ux915-ux905ux928ux92aux924-ux915-ux90fux915ux915ux924-ux915ux930ux924-ux938ux92eux92f-ux90fux915-ux936ux915ux924ux936ux932-ux935ux927-ux906ux936ux915-ux905ux936-ux905ux92aux918ux91fux928-ux939-ux92fux939-ux924ux915ux928ux915-ux90fux915-ux91cux91fux932-ux92dux928ux928-ux915-ux938ux930ux932-ux92dux928ux928-ux915-ux92fux917-ux915-ux930ux92a-ux92e-ux935ux92fux915ux924-ux915ux930ux924-ux939-ux91cux928ux939-ux90fux915ux915ux924-ux915ux930ux928-ux906ux938ux928-ux939ux924-ux939}

\subsubsection{विचार}\label{ux935ux91aux930-1}

यदि \(R(x) = \frac{P(x)}{Q(x)}\) एक तर्कसंगत कार्य है, जहां \(P(x)\) की डिग्री
\(Q(x)\) की डिग्री से कम है, तो हम \(R(x)\) को सरल भिन्नों में विघटित कर सकते हैं।

ये सरल टुकड़े हर \(Q(x)\) के गुणनखंडों के अनुरूप हैं।

\subsubsection{सामान्य
प्रपत्र}\label{ux938ux92eux928ux92f-ux92aux930ux92aux924ux930}

\begin{enumerate}
\def\labelenumi{\arabic{enumi}.}
\tightlist
\item
  विशिष्ट रैखिक कारक यदि
\end{enumerate}

\[
\frac{1}{(x-a)(x-b)},
\]

फिर के रूप में विघटित करें

\[
\frac{A}{x-a} + \frac{B}{x-b}.
\]

\begin{enumerate}
\def\labelenumi{\arabic{enumi}.}
\setcounter{enumi}{1}
\tightlist
\item
  बार-बार रैखिक कारक यदि हर में \((x-a)^n\) है, तो पद हैं
\end{enumerate}

\[
\frac{A_1}{x-a} + \frac{A_2}{(x-a)^2} + \dots + \frac{A_n}{(x-a)^n}.
\]

\begin{enumerate}
\def\labelenumi{\arabic{enumi}.}
\setcounter{enumi}{2}
\tightlist
\item
  अघुलनशील द्विघात कारक यदि हर में \((x^2+bx+c)\) है, तो अंश रैखिक है:
\end{enumerate}

\[
\frac{Ax+B}{x^2+bx+c}.
\]

\subsubsection{उदाहरण 1: विशिष्ट रैखिक
कारक}\label{ux909ux926ux939ux930ux923-1-ux935ux936ux937ux91f-ux930ux916ux915-ux915ux930ux915}

\[
\int \frac{1}{x^2 - 1}\,dx
\]

कारक हर: \((x-1)(x+1)\). विघटित करें:

\[
\frac{1}{x^2-1} = \frac{1}{2}\left(\frac{1}{x-1} - \frac{1}{x+1}\right).
\]

एकीकृत करें:

\[
\int \frac{1}{x^2 - 1}\,dx = \tfrac{1}{2}\ln\left|\frac{x-1}{x+1}\right| + C.
\]

\subsubsection{उदाहरण 2: दोहराया गया रैखिक
कारक}\label{ux909ux926ux939ux930ux923-2-ux926ux939ux930ux92f-ux917ux92f-ux930ux916ux915-ux915ux930ux915}

\[
\int \frac{1}{(x-1)^2}\,dx
\]

यह पहले से ही सरल है:

\[
\int (x-1)^{-2}\,dx = -\frac{1}{x-1} + C.
\]

\subsubsection{उदाहरण 3: इरेड्यूसिबल क्वाड्रैटिक
फ़ैक्टर}\label{ux909ux926ux939ux930ux923-3-ux907ux930ux921ux92fux938ux92cux932-ux915ux935ux921ux930ux91fux915-ux92bux915ux91fux930}

\[
\int \frac{x}{x^2+1}\,dx
\]

\(u = x^2+1\) को प्रतिस्थापित करें, या पहचानें कि अंश हर का व्युत्पन्न है।

\[
\int \frac{x}{x^2+1}\,dx = \tfrac{1}{2}\ln(x^2+1) + C.
\]

\subsubsection{आंशिक भिन्न अपघटन के
चरण}\label{ux906ux936ux915-ux92dux928ux928-ux905ux92aux918ux91fux928-ux915-ux91aux930ux923}

\begin{enumerate}
\def\labelenumi{\arabic{enumi}.}
\tightlist
\item
  हर का गुणनखंड करें।
\item
  सामान्य आंशिक भिन्न रूप लिखिए।
\item
  भिन्नों को साफ़ करने के लिए हर से गुणा करें।
\item
  अज्ञात स्थिरांकों को हल करें।
\item
  प्रत्येक पद को एकीकृत करें.
\end{enumerate}

\subsubsection{यह क्यों मायने रखता
है}\label{ux92fux939-ux915ux92f-ux92eux92fux928-ux930ux916ux924-ux939-4}

\begin{itemize}
\tightlist
\item
  जटिल तर्कसंगत कार्यों को सरल लघुगणक या आर्कटेंजेंट रूपों में परिवर्तित करता है।
\item
  विभेदक समीकरणों और लाप्लास परिवर्तनों में विशेष रूप से उपयोगी।
\item
  उन्नत कैलकुलस और इंजीनियरिंग में मौलिक।
\end{itemize}

\subsubsection{व्यायाम}\label{ux935ux92fux92fux92e-20}

\begin{enumerate}
\def\labelenumi{\arabic{enumi}.}
\tightlist
\item
  \(\int \frac{3x+5}{x^2-1}\,dx\) को विघटित और एकीकृत करें।
\item
  \(\int \frac{1}{x^2(x+1)}\,dx\) का मूल्यांकन करें।
\item
  \(\int \frac{2x+1}{x^2+2x+2}\,dx\) की गणना करें।
\item
  \(\int \frac{1}{x^3 - x}\,dx\) खोजें।
\item
  आंशिक भिन्न या प्रतिस्थापन का उपयोग करके दिखाएँ कि
  \(\int \frac{dx}{x^2+1} = \arctan x + C\)।
\end{enumerate}

\subsection{5.5 अनुचित
इंटीग्रल}\label{ux905ux928ux91aux924-ux907ux91fux917ux930ux932}

कुछ अभिन्नों का सीधे मूल्यांकन नहीं किया जा सकता क्योंकि अंतराल अनंत है या एकीकृत
असीमित हो जाता है। इन्हें अनुचित समाकलन कहा जाता है। उन्हें सीमाओं का उपयोग करके
परिभाषित किया गया है।

\subsubsection{परिभाषा}\label{ux92aux930ux92dux937-6}

\begin{enumerate}
\def\labelenumi{\arabic{enumi}.}
\tightlist
\item
  अनंत अंतराल
\end{enumerate}

\[\int_a^\infty f(x)\,dx = \lim_{b \to \infty} \int_a^b f(x)\,dx.
\]

\[
\int_{-\infty}^a f(x)\,dx = \lim_{b \to -\infty} \int_b^a f(x)\,dx.
\]

\begin{enumerate}
\def\labelenumi{\arabic{enumi}.}
\setcounter{enumi}{1}
\tightlist
\item
  Unbounded integrand If \(f(x)\) has a vertical asymptote at \(c\),
  then
\end{enumerate}

\[
\int_a^c f(x)\,dx = \lim_{t \to c^-} \int_a^t f(x)\,dx,
\]

\[
\int_c^b f(x)\,dx = \lim_{t \to c^+} \int_t^b f(x)\,dx.
\]

\subsubsection{Convergence and
Divergence}\label{convergence-and-divergence}

\begin{itemize}
\tightlist
\item
  If the limit exists and is finite, the improper integral converges.
\item
  If the limit does not exist or is infinite, the improper integral
  diverges.
\end{itemize}

\subsubsection{Examples}\label{examples-1}

\begin{enumerate}
\def\labelenumi{\arabic{enumi}.}
\tightlist
\item
  Exponential decay
\end{enumerate}

\[
\int_1^\infty \frac{1}{x^2}\,dx = \lim_{b \to \infty} \Big[-\tfrac{1}{x}\Big]_1^b = 1.
\]

This converges.

\begin{enumerate}
\def\labelenumi{\arabic{enumi}.}
\setcounter{enumi}{1}
\tightlist
\item
  Harmonic function
\end{enumerate}

\[
\int_1^\infty \frac{1}{x}\,dx = \lim_{b \to \infty} \ln b.
\]

This diverges to infinity.

\begin{enumerate}
\def\labelenumi{\arabic{enumi}.}
\setcounter{enumi}{2}
\tightlist
\item
  Asymptote at 0
\end{enumerate}

\[
\int_0^1 \frac{1}{\sqrt{x}}\,dx = \lim_{t \to 0^+} \int_t^1 x^{-1/2}\,dx.
\]

\[
= \lim_{t \to 0^+} [2\sqrt{x}]_t^1 = 2.
\]

This converges.

\begin{enumerate}
\def\labelenumi{\arabic{enumi}.}
\setcounter{enumi}{3}
\tightlist
\item
  Asymptote at 0 (divergent)
\end{enumerate}

\[
\int_0^1 \frac{1}{x}\,dx = \lim_{t \to 0^+} \ln(1) - \ln(t).
\]

This diverges since \(\ln(t) \to -\infty\).

\subsubsection{Comparison Test for Improper
Integrals}\label{comparison-test-for-improper-integrals}

\begin{itemize}
\tightlist
\item
  If \(0 \leq f(x) \leq g(x)\) for large \(x\), and \(\int g(x)\,dx\)
  converges, then \(\int f(x)\,dx\) also converges.
\item
  If \(\int f(x)\,dx\) diverges and \(f(x) \geq g(x) \geq 0\), then
  \(\int g(x)\,dx\) also diverges.
\end{itemize}

\subsubsection{Why Improper Integrals
Matter}\label{why-improper-integrals-matter}

\begin{itemize}
\tightlist
\item
  They extend integration to infinite domains and unbounded functions.
\item
  They are essential in probability (continuous distributions), physics
  (gravitational/electric fields), and Fourier analysis.
\end{itemize}

\subsubsection{Exercises}\label{exercises-2}

\begin{enumerate}
\def\labelenumi{\arabic{enumi}.}
\tightlist
\item
  Determine whether \(\int_1^\infty \frac{1}{x^p}\,dx\) converges for
  various values of \(p\).
\item
  Evaluate \(\int_0^\infty e^{-x}\,dx\).
\item
  Test convergence of \(\int_0^1 \frac{1}{x^p}\,dx\) depending on \(p\).
\item
  Compute \(\int_{-\infty}^\infty \frac{1}{1+x^2}\,dx\).
\item
  Use the comparison test to show that
  \(\int_1^\infty \frac{1}{x^2+1}\,dx\) converges.
\end{enumerate}

\section{Chapter 6. Applications of
Integration}\label{chapter-6.-applications-of-integration}

\subsection{6.1 Areas and Volumes}\label{areas-and-volumes}

One of the most important applications of integration is finding areas
under curves and volumes of solids.

\subsubsection{Area Between Curves}\label{area-between-curves}

If \(f(x) \geq g(x)\) on \([a, b]\), then the area between the curves
\(y=f(x)\) and \(y=g(x)\) is

\[
ए = \int_a^b \big(f(x) - g(x)\big)\,dx.
\]

Example: Find the area between \(y=x^2\) and \(y=x\) on \([0,1]\).

\[
A = \int_0^1 (x - x^2)\,dx = \left[\tfrac{1}{2}x^2 - \tfrac{1}{3}x^3\right]_0^1 = \tfrac{1}{6}.
\]

\subsubsection{Volumes by Slicing}\label{volumes-by-slicing}

If a solid has cross-sectional area \(A(x)\) at position \(x\), then the
volume is

\[
वी = \int_a^b A(x)\,dx.
\]\#\#\# क्रांति के खंड

जब कोई क्षेत्र किसी अक्ष के चारों ओर घूमता है, तो परिणामी ठोस का आयतन एकीकरण के
साथ पाया जा सकता है।

\begin{enumerate}
\def\labelenumi{\arabic{enumi}.}
\tightlist
\item
  डिस्क विधि यदि \(y=f(x)\), \(x\in[a,b]\) के अंतर्गत क्षेत्र, \(x\)-अक्ष के चारों
  ओर घूमता है:
\end{enumerate}

\[
V = \pi \int_a^b [f(x)]^2\,dx.
\]

\begin{enumerate}
\def\labelenumi{\arabic{enumi}.}
\setcounter{enumi}{1}
\tightlist
\item
  वॉशर विधि यदि \(y=f(x)\) और \(y=g(x)\) के बीच का क्षेत्र \(x\)-अक्ष के चारों
  ओर घूमता है:
\end{enumerate}

\[
V = \pi \int_a^b \Big([f(x)]^2 - [g(x)]^2\Big)\,dx.
\]

\begin{enumerate}
\def\labelenumi{\arabic{enumi}.}
\setcounter{enumi}{2}
\tightlist
\item
  शैल विधि यदि \(y=f(x)\) के अंतर्गत क्षेत्र \(y\)-अक्ष के चारों ओर घूमता है:
\end{enumerate}

\[
V = 2\pi \int_a^b x f(x)\,dx.
\]

\subsubsection{उदाहरण}\label{ux909ux926ux939ux930ux923-13}

\begin{enumerate}
\def\labelenumi{\arabic{enumi}.}
\tightlist
\item
  डिस्क विधि \(y=\sqrt{x}\), \(0 \leq x \leq 4\) को \(x\)-अक्ष के चारों ओर
  घुमाएँ:
\end{enumerate}

\[
V = \pi \int_0^4 (\sqrt{x})^2\,dx = \pi \int_0^4 x\,dx = \pi \left[\tfrac{1}{2}x^2\right]_0^4 = 8\pi.
\]

\begin{enumerate}
\def\labelenumi{\arabic{enumi}.}
\setcounter{enumi}{1}
\tightlist
\item
  वॉशर विधि \(y=\sqrt{x}\) और \(y=1\), \(0 \leq x \leq 1\) के बीच,
  \(x\)-अक्ष के आसपास परिक्रमण क्षेत्र:
\end{enumerate}

\[
V = \pi \int_0^1 \big((\sqrt{x})^2 - (1)^2\big)\,dx = \pi \int_0^1 (x-1)\,dx = -\tfrac{\pi}{2}.
\]

(वॉल्यूम के लिए पूर्ण मान लें: \(V = \tfrac{\pi}{2}\))।

\begin{enumerate}
\def\labelenumi{\arabic{enumi}.}
\setcounter{enumi}{2}
\tightlist
\item
  शैल विधि \(y=x\), \(0 \leq x \leq 1\) के अंतर्गत, \(y\)-अक्ष के चारों ओर
  परिक्रमण क्षेत्र:
\end{enumerate}

\[
V = 2\pi \int_0^1 x(x)\,dx = 2\pi \int_0^1 x^2\,dx = 2\pi \cdot \tfrac{1}{3} = \tfrac{2\pi}{3}.
\]

\subsubsection{यह क्यों मायने रखता
है}\label{ux92fux939-ux915ux92f-ux92eux92fux928-ux930ux916ux924-ux939-5}

\begin{itemize}
\tightlist
\item
  ज्यामिति में क्षेत्रफलों और आयतनों की गणना करने के सटीक तरीके प्रदान करता है।
\item
  भौतिकी, इंजीनियरिंग और संभाव्यता में आवश्यक।
\item
  एकीकरण के साथ ज्यामितीय सोच का परिचय देता है।
\end{itemize}

\subsubsection{व्यायाम}\label{ux935ux92fux92fux92e-21}

\begin{enumerate}
\def\labelenumi{\arabic{enumi}.}
\tightlist
\item
  \([0, \pi/2]\) पर \(y=\cos x\) और \(y=\sin x\) के बीच का क्षेत्र ज्ञात करें।
\item
  \(y=x^2\), \(0 \leq x \leq 1\) को \(x\)-अक्ष के चारों ओर घूमने पर बनने वाले
  ठोस के आयतन की गणना करें।
\item
  \(y=x\) और \(y=\sqrt{x}\) के बीच के क्षेत्र को \([0,1]\) पर \(y\)-अक्ष के
  चारों ओर घुमाने पर बनने वाले ठोस का आयतन ज्ञात कीजिए।
\item
  \(x\)-अक्ष के चारों ओर \(y=\sqrt{1-x^2}\) (एक अर्धवृत्त) घूमने से बनने वाले ठोस के
  आयतन की गणना करने के लिए वॉशर विधि का उपयोग करें।
\item
  \(y=x^2+1\) और \(y=3x\) के बीच घिरा क्षेत्र ज्ञात कीजिए।
\end{enumerate}

\subsection{6.2 चाप की लंबाई और सतह
क्षेत्र}\label{ux91aux92a-ux915-ux932ux92cux908-ux914ux930-ux938ux924ux939-ux915ux937ux924ux930}

एकीकरण का उपयोग वक्रों की लंबाई और घूमने वाले वक्रों द्वारा उत्पन्न ठोस पदार्थों के
सतह क्षेत्र को मापने के लिए भी किया जा सकता है।

\subsubsection{चाप लंबाई}\label{ux91aux92a-ux932ux92cux908}

अंतराल \([a,b]\) पर एक चिकने वक्र \(y=f(x)\) के लिए, वक्र की लंबाई है

\[
L = \int_a^b \sqrt{1 + \big(f'(x)\big)^2}\,dx.
\]

यह रेखाखंडों के साथ वक्र का अनुमान लगाने और सीमा लेने से आता है।

उदाहरण: \(x=0\) से \(x=4\) तक \(y=\tfrac{1}{2}x^{3/2}\) की लंबाई ज्ञात करें।

\begin{itemize}
\tightlist
\item
  व्युत्पन्न: \(f'(x) = \tfrac{3}{4}\sqrt{x}\).
\item
  सूत्र:
\end{itemize}

\[
L = \int_0^4 \sqrt{1 + \Big(\tfrac{3}{4}\sqrt{x}\Big)^2}\,dx
= \int_0^4 \sqrt{1 + \tfrac{9}{16}x}\,dx.
\]

प्रतिस्थापन का उपयोग करके इस अभिन्न का मूल्यांकन किया जा सकता है।\#\#\# क्रांति का
सतही क्षेत्र

यदि एक वक्र \(y=f(x)\), \(a \leq x \leq b\), \(x\)-अक्ष के चारों ओर घूमता है,
तो परिणामी ठोस का सतह क्षेत्र है

\[
S = 2\pi \int_a^b f(x)\sqrt{1 + \big(f'(x)\big)^2}\,dx.
\]

यदि \(y\)-अक्ष के चारों ओर घूमता है:

\[
S = 2\pi \int_a^b x \sqrt{1 + \big(f'(x)\big)^2}\,dx.
\]

\subsubsection{उदाहरण}\label{ux909ux926ux939ux930ux923-14}

\begin{enumerate}
\def\labelenumi{\arabic{enumi}.}
\tightlist
\item
  एक रेखा की चाप लंबाई \(y=x\), \(0 \leq x \leq 3\) के लिए:
\end{enumerate}

\[
L = \int_0^3 \sqrt{1+(1)^2}\,dx = \int_0^3 \sqrt{2}\,dx = 3\sqrt{2}.
\]

\begin{enumerate}
\def\labelenumi{\arabic{enumi}.}
\setcounter{enumi}{1}
\tightlist
\item
  एक गोले का पृष्ठीय क्षेत्रफल \(y = \sqrt{r^2 - x^2}\), \(-r \leq x \leq r\)
  लें और \(x\)-अक्ष के चारों ओर घूमें।
\end{enumerate}

\[
S = 2\pi \int_{-r}^r \sqrt{r^2 - x^2}\sqrt{1+\left(\frac{-x}{\sqrt{r^2-x^2}}\right)^2}\,dx.
\]

सरलीकरण से \(S = 4\pi r^2\) मिलता है, जो गोले के सतह क्षेत्र का परिचित सूत्र है।

\subsubsection{यह क्यों मायने रखता
है}\label{ux92fux939-ux915ux92f-ux92eux92fux928-ux930ux916ux924-ux939-6}

\begin{itemize}
\tightlist
\item
  चाप की लंबाई दूरी के विचार को घुमावदार पथों तक बढ़ाती है।
\item
  क्रांति के सतह क्षेत्र में भौतिकी, इंजीनियरिंग और डिजाइन में अनुप्रयोग हैं।
\item
  कैलकुलस और ज्योमेट्री के बीच एक सेतु प्रदान करता है।
\end{itemize}

\subsubsection{व्यायाम}\label{ux935ux92fux92fux92e-22}

\begin{enumerate}
\def\labelenumi{\arabic{enumi}.}
\tightlist
\item
  \(x=0\) से \(x=4\) तक \(y=\sqrt{x}\) की चाप लंबाई ज्ञात कीजिए।
\item
  \(y=x^2\), \(0 \leq x \leq 1\) को \(x\)-अक्ष के चारों ओर घुमाकर प्राप्त ठोस
  के सतह क्षेत्र की गणना करें।
\item
  \(x=0\) से \(x=1\) तक \(y=\ln(\cosh x)\) की चाप लंबाई ज्ञात कीजिए।
\item
  दिखाएँ कि \(x\)-अक्ष के चारों ओर \(0\) से \(r\) तक घूमने से एक गोले का आधा सतह
  क्षेत्र प्राप्त होता है।
\item
  एक परिक्रमण रेखा द्वारा शंकु के पृष्ठीय क्षेत्रफल का सूत्र प्राप्त करें।
\end{enumerate}

\subsection{6.3 कार्य और
औसत}\label{ux915ux930ux92f-ux914ux930-ux914ux938ux924}

एकीकरण ज्यामिति तक सीमित नहीं है. यह किसी बल द्वारा किए गए कार्य और एक अंतराल
पर किसी फ़ंक्शन के औसत मूल्य की गणना करने में भी मदद करता है।

\subsubsection{कार्य}\label{ux915ux930ux92f-1}

यदि एक परिवर्तनशील बल \(F(x)\) किसी वस्तु को \(x=a\) से \(x=b\) तक एक सीधी
रेखा में ले जाता है, तो कुल कार्य है

\[
W = \int_a^b F(x)\,dx.
\]

यह सूत्र निरंतर बल के लिए सरल मामले \(W = F \cdot d\) को सामान्यीकृत करता है।

उदाहरण 1: स्प्रिंग फोर्स (हुक का नियम) लंबाई \(a\) से \(b\) तक खींचे गए स्प्रिंग के
लिए, बल \(F(x) = kx\) के साथ:

\[
W = \int_a^b kx\,dx = \tfrac{1}{2}k(b^2-a^2).
\]

उदाहरण 2: पानी पम्प करना यदि किसी टैंक से पानी पंप किया जाता है, तो आवश्यक कार्य
बराबर होता है

\[
W = \int_a^b \text{(weight density)} \times \text{(cross-sectional area)} \times \text{(distance lifted)} \, dx.
\]

\subsubsection{किसी फ़ंक्शन का औसत
मान}\label{ux915ux938-ux92bux915ux936ux928-ux915-ux914ux938ux924-ux92eux928}

\([a,b]\) पर एक सतत फ़ंक्शन \(f(x)\) का औसत मान है

\[
f_{\text{avg}} = \frac{1}{b-a} \int_a^b f(x)\,dx.
\]

यह संख्याओं की सूची के औसत का निरंतर एनालॉग है।

उदाहरण 1: \([0,2]\) पर \(f(x)=x^2\) के लिए:

\[
f_{\text{avg}} = \tfrac{1}{2-0}\int_0^2 x^2 dx = \tfrac{1}{2}\cdot \tfrac{8}{3} = \tfrac{4}{3}.
\]

उदाहरण 2:यदि किसी कण का वेग \(v(t)\) है, तो \([a,b]\) पर औसत वेग है

\[
v_{\text{avg}} = \frac{1}{b-a}\int_a^b v(t)\,dt.
\]

\subsubsection{यह क्यों मायने रखता
है}\label{ux92fux939-ux915ux92f-ux92eux92fux928-ux930ux916ux924-ux939-7}

\begin{itemize}
\tightlist
\item
  कार्य अभिन्न अंग भौतिकी, इंजीनियरिंग और ऊर्जा गणना में दिखाई देते हैं।
\item
  औसत मूल्य अलग-अलग मात्राओं के लिए एक ही प्रतिनिधि संख्या देता है।
\item
  दोनों कैलकुलस को गति, बल और दक्षता की वास्तविक दुनिया की समस्याओं से जोड़ते हैं।
\end{itemize}

\subsubsection{व्यायाम}\label{ux935ux92fux92fux92e-23}

\begin{enumerate}
\def\labelenumi{\arabic{enumi}.}
\tightlist
\item
  यदि \(k=10\) है तो एक स्प्रिंग को 2 मीटर से 5 मीटर तक खींचने के लिए आवश्यक कार्य
  की गणना करें।
\item
  एक 100 किलोग्राम की वस्तु को गुरुत्वाकर्षण क्षेत्र (\(g=9.8 \,\text{m/s}^2\)) में
  लंबवत रूप से 5 मीटर ऊपर उठाया जाता है। कार्य को समग्र रूप में व्यक्त करें तथा
  मूल्यांकन करें।
\item
  \([0,\pi]\) पर \(f(x)=\sin x\) का औसत मान ज्ञात कीजिए।
\item
  यदि \(T(t)=20+5\cos(\tfrac{\pi t}{12})\) है तो 24 घंटे के दिन में औसत तापमान
  की गणना करें।
\item
  10 मीटर गहराई का एक टैंक पानी से भरा है। पूरे पानी को ऊपर तक पंप करने के लिए
  आवश्यक कार्य की गणना करें, पानी का वजन \(9800 \,\text{N/m}^3\) है।
\end{enumerate}

\subsection{6.4 संभाव्यता घनत्व और सतत
वितरण}\label{ux938ux92dux935ux92fux924-ux918ux928ux924ux935-ux914ux930-ux938ux924ux924-ux935ux924ux930ux923}

संभाव्यता सिद्धांत में एकीकरण भी एक केंद्रीय भूमिका निभाता है, विशेष रूप से निरंतर
यादृच्छिक चर के लिए। असतत परिणामों के बजाय, हम संभाव्यता घनत्व फ़ंक्शन (पीडीएफ)
नामक कार्यों के साथ संभावनाओं का वर्णन करते हैं।

\subsubsection{संभाव्यता घनत्व
कार्य}\label{ux938ux92dux935ux92fux924-ux918ux928ux924ux935-ux915ux930ux92f}

संभाव्यता घनत्व फ़ंक्शन \(f(x)\) को दो शर्तों को पूरा करना होगा:

\begin{enumerate}
\def\labelenumi{\arabic{enumi}.}
\item
  \(f(x) \geq 0\) सभी \(x\) के लिए।
\item
  वक्र के नीचे का कुल क्षेत्रफल 1 है:

  \[
  \int_{-\infty}^\infty f(x)\,dx = 1.
  \]
\end{enumerate}

यदि \(X\) पीडीएफ \(f(x)\) के साथ एक सतत यादृच्छिक चर है, तो संभावना है कि
\(X\) \(a\) और \(b\) के बीच है

\[
P(a \leq X \leq b) = \int_a^b f(x)\,dx.
\]

\subsubsection{संचयी वितरण
फ़ंक्शन}\label{ux938ux91aux92f-ux935ux924ux930ux923-ux92bux915ux936ux928}

संचयी वितरण फ़ंक्शन (सीडीएफ) को इस प्रकार परिभाषित किया गया है

\[
F(x) = \int_{-\infty}^x f(t)\,dt.
\]

यह संभावना देता है कि यादृच्छिक चर \(x\) से कम या उसके बराबर है।

\subsubsection{अपेक्षित मूल्य
(माध्य)}\label{ux905ux92aux915ux937ux924-ux92eux932ux92f-ux92eux927ux92f}

एक सतत यादृच्छिक चर का अपेक्षित मान भारित औसत है:

\[
E[X] = \int_{-\infty}^\infty x f(x)\,dx.
\]

\subsubsection{उदाहरण}\label{ux909ux926ux939ux930ux923-15}

\begin{enumerate}
\def\labelenumi{\arabic{enumi}.}
\tightlist
\item
  समान वितरण \([a,b]\) पर \(f(x) = \tfrac{1}{b-a}\) के लिए:
\end{enumerate}

\begin{itemize}
\item
  अंतराल की संभावना \([c,d]\):

  \[
  P(c \leq X \leq d) = \frac{d-c}{b-a}.
  \]
\item
  अपेक्षित मूल्य: \(E[X] = \tfrac{a+b}{2}\)।
\end{itemize}

\begin{enumerate}
\def\labelenumi{\arabic{enumi}.}
\setcounter{enumi}{1}
\tightlist
\item
  घातीय वितरण \(f(x) = \lambda e^{-\lambda x}\), \(x \geq 0\) के लिए:
\end{enumerate}

\begin{itemize}
\tightlist
\item
  \(\int_0^\infty \lambda e^{-\lambda x}\,dx = 1\).
\item
  मतलब: \(E[X] = \tfrac{1}{\lambda}\).
\end{itemize}

\begin{enumerate}
\def\labelenumi{\arabic{enumi}.}
\setcounter{enumi}{2}
\tightlist
\item
  सामान्य वितरण घंटी वक्र:
\end{enumerate}

\[
f(x) = \frac{1}{\sqrt{2\pi\sigma^2}} e^{-\frac{(x-\mu)^2}{2\sigma^2}}.
\]

यह 1 से एकीकृत होता है, लेकिन इसके लिए उन्नत तकनीकों की आवश्यकता होती है।

\subsubsection{यह क्यों मायने रखता है- संभाव्यता घनत्व विज्ञान, इंजीनियरिंग और
सांख्यिकी में अनिश्चितता का वर्णन करता
है।}\label{ux92fux939-ux915ux92f-ux92eux92fux928-ux930ux916ux924-ux939--ux938ux92dux935ux92fux924-ux918ux928ux924ux935-ux935ux91cux91eux928-ux907ux91cux928ux92fux930ux917-ux914ux930-ux938ux916ux92fux915-ux92e-ux905ux928ux936ux91aux924ux924-ux915-ux935ux930ux923ux928-ux915ux930ux924-ux939}

\begin{itemize}
\tightlist
\item
  इंटीग्रल वक्रों के नीचे के क्षेत्रों को संभावनाओं से जोड़ते हैं।
\item
  निरंतर वितरण अंतरालों पर संभावनाओं को मापने के लिए परिणामों की गिनती के विचार
  को सामान्यीकृत करता है।
\end{itemize}

\subsubsection{व्यायाम}\label{ux935ux92fux92fux92e-24}

\begin{enumerate}
\def\labelenumi{\arabic{enumi}.}
\tightlist
\item
  दिखाएँ कि \([a,b]\) पर एकसमान घनत्व \(f(x) = \tfrac{1}{b-a}\) 1 में एकीकृत
  हो जाता है।
\item
  \(\lambda = 2\) के साथ घातीय वितरण के लिए, \(P(0 \leq X \leq 1)\) की
  गणना करें।
\item
  यदि \([0,1]\) पर \(f(x) = 3x^2\) है तो \(X\) का अपेक्षित मान ज्ञात कीजिए।
\item
  सत्यापित करें कि माध्य 0 और विचरण 1 के साथ सामान्य वितरण की कुल संभावना 1 है
  (पूर्ण प्रमाण की कोई आवश्यकता नहीं है, लेकिन बताएं कि यह क्यों मान्य है)।
\item
  \([0,1]\) पर समान वितरण की सीडीएफ की गणना करें।
\end{enumerate}

\section{भाग III. बहुपरिवर्तनीय
कलन}\label{ux92dux917-iii.-ux92cux939ux92aux930ux935ux930ux924ux928ux92f-ux915ux932ux928}

\section{अध्याय 7. सदिश फलन और
वक्र}\label{ux905ux927ux92fux92f-7.-ux938ux926ux936-ux92bux932ux928-ux914ux930-ux935ux915ux930}

\subsection{7.1 वेक्टर फ़ंक्शंस और स्पेस
कर्व्स}\label{ux935ux915ux91fux930-ux92bux915ux936ux938-ux914ux930-ux938ux92aux938-ux915ux930ux935ux938}

मल्टीवेरिएबल कैलकुलस में, फ़ंक्शन संख्याओं के बजाय वैक्टर को आउटपुट कर सकते हैं। इन्हें
वेक्टर-वैल्यू फ़ंक्शंस कहा जाता है, और ये अंतरिक्ष में वक्रों का वर्णन करने के लिए आवश्यक हैं।

\subsubsection{परिभाषा}\label{ux92aux930ux92dux937-7}

एक वेक्टर फ़ंक्शन फॉर्म का एक फ़ंक्शन है

\[
\mathbf{r}(t) = \langle x(t), y(t), z(t) \rangle,
\]

जहां \(x(t), y(t), z(t)\) वास्तविक-मूल्यवान फ़ंक्शन हैं।

\begin{itemize}
\tightlist
\item
  इनपुट \(t\) को अक्सर पैरामीटर कहा जाता है।
\item
  आउटपुट 2डी या 3डी स्पेस में एक वेक्टर है।
\item
  3डी में एक वेक्टर फ़ंक्शन का ग्राफ़ एक अंतरिक्ष वक्र है।
\end{itemize}

\subsubsection{उदाहरण}\label{ux909ux926ux939ux930ux923-16}

\begin{enumerate}
\def\labelenumi{\arabic{enumi}.}
\tightlist
\item
  रेखा
\end{enumerate}

\[
\mathbf{r}(t) = \langle 1+2t, \; 3-t, \; 4+5t \rangle.
\]

यह दिशा वेक्टर \(\langle 2,-1,5 \rangle\) के साथ बिंदु \((1,3,4)\) से होकर जाने
वाली एक सीधी रेखा का वर्णन करता है।

\begin{enumerate}
\def\labelenumi{\arabic{enumi}.}
\setcounter{enumi}{1}
\tightlist
\item
  समतल में घेरा
\end{enumerate}

\[
\mathbf{r}(t) = \langle \cos t, \; \sin t, \; 0 \rangle, \quad 0 \leq t < 2\pi.
\]

\begin{enumerate}
\def\labelenumi{\arabic{enumi}.}
\setcounter{enumi}{2}
\tightlist
\item
  हेलिक्स
\end{enumerate}

\[
\mathbf{r}(t) = \langle \cos t, \; \sin t, \; t \rangle.
\]

यह \(z\)-अक्ष के चारों ओर उठता हुआ एक सर्पिल है।

\subsubsection{सीमाएँ और
निरंतरता}\label{ux938ux92eux90f-ux914ux930-ux928ux930ux924ux930ux924}

एक वेक्टर फ़ंक्शन \(t=a\) पर निरंतर है यदि प्रत्येक घटक \(x(t), y(t), z(t)\)
\(t=a\) पर निरंतर है।

\[
\lim_{t \to a} \mathbf{r}(t) = \langle \lim_{t \to a} x(t), \; \lim_{t \to a} y(t), \; \lim_{t \to a} z(t) \rangle.
\]

\subsubsection{अंतरिक्ष वक्रों की
ज्यामिति}\label{ux905ux924ux930ux915ux937-ux935ux915ux930-ux915-ux91cux92fux92eux924}

\begin{itemize}
\tightlist
\item
  प्रत्येक वक्र में व्युत्पन्न द्वारा दी गई एक स्पर्शरेखा दिशा होती है।
\item
  अंतरिक्ष वक्र गति पथ, कण प्रक्षेपवक्र और ज्यामितीय आकृतियों को मॉडल कर सकते हैं।
\end{itemize}

\subsubsection{यह क्यों मायने रखता
है}\label{ux92fux939-ux915ux92f-ux92eux92fux928-ux930ux916ux924-ux939-8}

वेक्टर फ़ंक्शन मल्टीवेरिएबल कैलकुलस की नींव हैं, जो हमें डेरिवेटिव और इंटीग्रल के विचारों को
उच्च आयामों में विस्तारित करने की अनुमति देते हैं। वे भौतिकी (3डी में गति, विद्युत
चुंबकत्व, द्रव गतिकी) में भी स्वाभाविक रूप से दिखाई देते हैं।

\subsubsection{व्यायाम}\label{ux935ux92fux92fux92e-25}

\begin{enumerate}
\def\labelenumi{\arabic{enumi}.}
\tightlist
\item
  वेक्टर \(\langle 3,-2,1 \rangle\) के समानांतर \((0,1,2)\) से गुजरने वाली एक
  रेखा के लिए एक वेक्टर फ़ंक्शन लिखें।2.
  \(\mathbf{r}(t) = \langle 2\cos t, \; 2\sin t, \; 3 \rangle\) द्वारा
  दिए गए वक्र का वर्णन करें।
\item
  निर्धारित करें कि क्या
  \(\mathbf{r}(t) = \langle e^t, \; \ln t, \; t^2 \rangle\) \(t=1\) पर
  निरंतर है।
\item
  हेलिक्स का रेखाचित्र बनाएं
  \(\mathbf{r}(t) = \langle \cos t, \; \sin t, \; 2t \rangle\)।
\item
  वक्र \(\mathbf{r}(t) = \langle t, \; t^2, \; t^3 \rangle\) पर \(t=2\)
  होने पर बिंदु ज्ञात कीजिए।
\end{enumerate}

\subsection{7.2 वेक्टर फ़ंक्शंस के डेरिवेटिव और
इंटीग्रल}\label{ux935ux915ux91fux930-ux92bux915ux936ux938-ux915-ux921ux930ux935ux91fux935-ux914ux930-ux907ux91fux917ux930ux932}

वेक्टर फ़ंक्शंस को सामान्य फ़ंक्शंस की तरह ही विभेदित और एकीकृत किया जा सकता है - हम
बस प्रत्येक घटक पर ऑपरेशन लागू करते हैं। यह हमें उच्च आयामों में गति, वेग, त्वरण और संचय
का अध्ययन करने की अनुमति देता है।

\subsubsection{एक वेक्टर फ़ंक्शन का
व्युत्पन्न}\label{ux90fux915-ux935ux915ux91fux930-ux92bux915ux936ux928-ux915-ux935ux92fux924ux92aux928ux928}

यदि

\[
\mathbf{r}(t) = \langle x(t), y(t), z(t) \rangle,
\]

फिर

\[
\mathbf{r}'(t) = \langle x'(t), y'(t), z'(t) \rangle.
\]

यह व्युत्पन्न वेक्टर पैरामीटर \(t\) पर वक्र की स्पर्शरेखा दिशा में इंगित करता है।

\begin{itemize}
\tightlist
\item
  वेग: यदि \(\mathbf{r}(t)\) समय \(t\) पर एक कण की स्थिति देता है, तो
  \(\mathbf{v}(t) = \mathbf{r}'(t)\) इसका वेग वेक्टर है।
\item
  गति: परिमाण \(|\mathbf{v}(t)|\) कण की गति है।
\item
  त्वरण: \(\mathbf{a}(t) = \mathbf{v}'(t) = \mathbf{r}''(t)\)।
\end{itemize}

\subsubsection{उदाहरण}\label{ux909ux926ux939ux930ux923-17}

\begin{enumerate}
\def\labelenumi{\arabic{enumi}.}
\tightlist
\item
  हेलिक्स
\end{enumerate}

\[
\mathbf{r}(t) = \langle \cos t, \sin t, t \rangle.
\]

\begin{itemize}
\tightlist
\item
  वेग: \(\mathbf{v}(t) = \langle -\sin t, \cos t, 1 \rangle\).
\item
  गति:
  \(|\mathbf{v}(t)| = \sqrt{(-\sin t)^2 + (\cos t)^2 + 1^2} = \sqrt{2}\)।
\item
  त्वरण: \(\mathbf{a}(t) = \langle -\cos t, -\sin t, 0 \rangle\).
\end{itemize}

\begin{enumerate}
\def\labelenumi{\arabic{enumi}.}
\setcounter{enumi}{1}
\tightlist
\item
  प्रक्षेप्य गति
\end{enumerate}

\[
\mathbf{r}(t) = \langle v_0 \cos\theta \cdot t, \; v_0 \sin\theta \cdot t - \tfrac{1}{2}gt^2 \rangle.
\]

यह गुरुत्वाकर्षण के तहत प्रक्षेप्य के परवलयिक पथ को मॉडल करता है।

\subsubsection{एक वेक्टर फ़ंक्शन का अभिन्न
अंग}\label{ux90fux915-ux935ux915ux91fux930-ux92bux915ux936ux928-ux915-ux905ux92dux928ux928-ux905ux917}

यदि

\[
\mathbf{r}(t) = \langle x(t), y(t), z(t) \rangle,
\]

फिर

\[
\int \mathbf{r}(t)\,dt = \left\langle \int x(t)\,dt, \; \int y(t)\,dt, \; \int z(t)\,dt \right\rangle + \mathbf{C},
\]

जहां \(\mathbf{C}\) एक स्थिर वेक्टर है।

\subsubsection{उदाहरण}\label{ux909ux926ux939ux930ux923-18}

\[
\mathbf{r}(t) = \langle t, t^2, t^3 \rangle.
\]

\begin{itemize}
\tightlist
\item
  व्युत्पन्न: \(\mathbf{r}'(t) = \langle 1, 2t, 3t^2 \rangle\)।
\item
  अभिन्न:
\end{itemize}

\[
\int \mathbf{r}(t)\,dt = \langle \tfrac{1}{2}t^2, \tfrac{1}{3}t^3, \tfrac{1}{4}t^4 \rangle + \mathbf{C}.
\]

\subsubsection{यह क्यों मायने रखता
है}\label{ux92fux939-ux915ux92f-ux92eux92fux928-ux930ux916ux924-ux939-9}

\begin{itemize}
\tightlist
\item
  वेक्टर फ़ंक्शंस के व्युत्पन्न अंतरिक्ष में गति और बलों का वर्णन करते हैं।
\item
  इंटीग्रल विस्थापन, कार्य और संचित मात्रा देते हैं।
\item
  ये उपकरण कैलकुलस को सीधे भौतिकी और इंजीनियरिंग से जोड़ते हैं।
\end{itemize}

\subsubsection{व्यायाम}\label{ux935ux92fux92fux92e-26}

\begin{enumerate}
\def\labelenumi{\arabic{enumi}.}
\tightlist
\item
  \(\mathbf{r}(t) = \langle t, \cos t, \sin t \rangle\) के लिए, वेग, गति
  और त्वरण ज्ञात करें।2. \(\mathbf{r}(t) = \langle e^t, \ln t, t^2 \rangle\)
  के लिए \(\mathbf{r}'(t)\) की गणना करें।
\item
  \(\mathbf{r}(t) = \langle 1, t, t^2 \rangle\) को एकीकृत करें।
\item
  एक कण का वेग \(\mathbf{v}(t) = \langle t, 2, 0 \rangle\) है। यदि
  \(\mathbf{r}(0) = \langle 1, 0, 0 \rangle\) है तो इसका स्थिति वेक्टर ज्ञात
  करें।
\item
  दिखाएँ कि \(\mathbf{r}(t) = \langle \cos t, \sin t, 0 \rangle\) की गति
  स्थिर है।
\end{enumerate}

\subsection{7.3 चाप की लंबाई और
वक्रता}\label{ux91aux92a-ux915-ux932ux92cux908-ux914ux930-ux935ux915ux930ux924}

वेक्टर कैलकुलस न केवल वक्र द्वारा अनुरेखित पथ को मापने के लिए उपकरण प्रदान करता है
बल्कि यह भी मापता है कि यह कितनी तेजी से मुड़ता है। इन्हें चाप की लंबाई और वक्रता के
माध्यम से व्यक्त किया जाता है।

\subsubsection{अंतरिक्ष वक्र की चाप
लंबाई}\label{ux905ux924ux930ux915ux937-ux935ux915ux930-ux915-ux91aux92a-ux932ux92cux908}

यदि कोई वक्र दिया गया है

\[
\mathbf{r}(t) = \langle x(t), y(t), z(t) \rangle, \quad a \leq t \leq b,
\]

तो चाप की लंबाई है

\[
L = \int_a^b |\mathbf{r}'(t)|\,dt,
\]

कहाँ

\[
|\mathbf{r}'(t)| = \sqrt{(x'(t))^2 + (y'(t))^2 + (z'(t))^2}.
\]

उदाहरण: हेलिक्स के लिए
\(\mathbf{r}(t) = \langle \cos t, \sin t, t \rangle, \, 0 \leq t \leq 2\pi\):

\begin{itemize}
\tightlist
\item
  वेग: \(\mathbf{r}'(t) = \langle -\sin t, \cos t, 1 \rangle\)।
\item
  गति:
  \(|\mathbf{r}'(t)| = \sqrt{(-\sin t)^2 + (\cos t)^2 + 1^2} = \sqrt{2}\).
\item
  चाप की लंबाई:
\end{itemize}

\[
L = \int_0^{2\pi} \sqrt{2}\,dt = 2\pi\sqrt{2}.
\]

\subsubsection{वक्रता}\label{ux935ux915ux930ux924}

वक्रता मापती है कि वक्र कितनी जल्दी दिशा बदलता है।

एक चिकने वक्र के लिए \(\mathbf{r}(t)\):

\[
\kappa(t) = \frac{|\mathbf{r}'(t) \times \mathbf{r}''(t)|}{|\mathbf{r}'(t)|^3}.
\]

\begin{itemize}
\tightlist
\item
  \(\kappa = 0\): सीधी रेखा।
\item
  बड़ा \(\kappa\): वक्र अधिक तेजी से मुड़ता है।
\end{itemize}

उदाहरण: \(r\) त्रिज्या वाले वृत्त के लिए:

\[
\mathbf{r}(t) = \langle r\cos t, r\sin t \rangle.
\]

फिर \(\kappa = \tfrac{1}{r}\). अतः वक्रता स्थिर है और त्रिज्या के व्युत्क्रमानुपाती
है।

\subsubsection{इकाई स्पर्शरेखा और सामान्य
सदिश}\label{ux907ux915ux908-ux938ux92aux930ux936ux930ux916-ux914ux930-ux938ux92eux928ux92f-ux938ux926ux936}

\begin{itemize}
\tightlist
\item
  स्पर्शरेखा वेक्टर:
\end{itemize}

\[
\mathbf{T}(t) = \frac{\mathbf{r}'(t)}{|\mathbf{r}'(t)|}.
\]

\begin{itemize}
\tightlist
\item
  सामान्य वेक्टर: वक्रता के केंद्र की ओर इंगित करता है, जिसे इस प्रकार परिभाषित किया
  गया है
\end{itemize}

\[
\mathbf{N}(t) = \frac{\mathbf{T}'(t)}{|\mathbf{T}'(t)|}.
\]

ये वेक्टर गति की ज्यामिति का वर्णन करते हैं: यात्रा की दिशा और मोड़ की दिशा।

\subsubsection{यह क्यों मायने रखता
है}\label{ux92fux939-ux915ux92f-ux92eux92fux928-ux930ux916ux924-ux939-10}

\begin{itemize}
\tightlist
\item
  चाप की लंबाई अंतरिक्ष में वक्रों से दूरी की अवधारणा को सामान्य बनाती है।
\item
  वक्रता झुकने का वर्णन करती है, जो भौतिकी (सेंट्रिपेटल एक्सेलेरेशन), इंजीनियरिंग (सड़कें,
  रोलर कोस्टर) और कंप्यूटर ग्राफिक्स में महत्वपूर्ण है।
\end{itemize}

\subsubsection{व्यायाम}\label{ux935ux92fux92fux92e-27}

\begin{enumerate}
\def\labelenumi{\arabic{enumi}.}
\tightlist
\item
  \(t=0\) से \(t=1\) तक \(\mathbf{r}(t) = \langle t, t^2, 0 \rangle\) की
  चाप लंबाई ज्ञात कीजिए।
\item
  वृत्त \(\mathbf{r}(t) = \langle \cos t, \sin t \rangle\) की वक्रता की
  गणना करें।
\item
  \(\mathbf{r}(t) = \langle t, \cos t, \sin t \rangle\) के लिए,
  \(|\mathbf{r}'(t)|\) की गणना करें।
\item
  दिखाएँ कि एक सीधी रेखा में वक्रता \(\kappa = 0\) होती है।5. \(t=0\) पर
  \(\mathbf{r}(t) = \langle e^t, e^{-t}, t \rangle\) का स्पर्शरेखा सदिश
  ज्ञात कीजिए।
\end{enumerate}

\subsection{7.4 अंतरिक्ष में
गति}\label{ux905ux924ux930ux915ux937-ux92e-ux917ux924}

दो या तीन आयामों में गति का वर्णन करने में वेक्टर फ़ंक्शन विशेष रूप से शक्तिशाली होते हैं।
स्थिति, वेग और त्वरण स्वाभाविक रूप से वेक्टर-मूल्य वाले कार्यों के डेरिवेटिव और इंटीग्रल
का उपयोग करके व्यक्त किए जाते हैं।

\subsubsection{स्थिति, वेग, और
त्वरण}\label{ux938ux925ux924-ux935ux917-ux914ux930-ux924ux935ux930ux923}

\begin{itemize}
\tightlist
\item
  स्थिति वेक्टर:
\end{itemize}

\[
\mathbf{r}(t) = \langle x(t), y(t), z(t) \rangle
\]

\begin{itemize}
\tightlist
\item
  वेग वेक्टर (स्थिति का व्युत्पन्न):
\end{itemize}

\[
\mathbf{v}(t) = \mathbf{r}'(t) = \langle x'(t), y'(t), z'(t) \rangle
\]

\begin{itemize}
\tightlist
\item
  गति (वेग का परिमाण):
\end{itemize}

\[
|\mathbf{v}(t)| = \sqrt{(x'(t))^2 + (y'(t))^2 + (z'(t))^2}
\]

\begin{itemize}
\tightlist
\item
  त्वरण वेक्टर (वेग का व्युत्पन्न):
\end{itemize}

\[
\mathbf{a}(t) = \mathbf{v}'(t) = \mathbf{r}''(t).
\]

\subsubsection{स्पर्शरेखा और सामान्य
घटक}\label{ux938ux92aux930ux936ux930ux916-ux914ux930-ux938ux92eux928ux92f-ux918ux91fux915}

त्वरण को दो घटकों में विघटित किया जा सकता है:

\[
\mathbf{a}(t) = a_T \mathbf{T}(t) + a_N \mathbf{N}(t),
\]

कहाँ:

\begin{itemize}
\tightlist
\item
  \(\mathbf{T}(t)\) = इकाई स्पर्शरेखा वेक्टर,
\item
  \(\mathbf{N}(t)\) = प्रमुख सामान्य वेक्टर,
\item
  \(a_T = \frac{d}{dt}|\mathbf{v}(t)|\) = स्पर्शरेखीय त्वरण (गति में परिवर्तन),
\item
  \(a_N = \kappa |\mathbf{v}(t)|^2\) = सामान्य त्वरण (दिशा में परिवर्तन)।
\end{itemize}

\subsubsection{3डी में प्रक्षेप्य
गति}\label{ux921-ux92e-ux92aux930ux915ux937ux92aux92f-ux917ux924}

\(-z\) दिशा में कार्य कर रहे गुरुत्वाकर्षण के साथ:

\[
\mathbf{r}(t) = \langle v_0 \cos\theta \cos\phi \cdot t,\; v_0 \cos\theta \sin\phi \cdot t,\; v_0 \sin\theta \cdot t - \tfrac{1}{2}gt^2 \rangle,
\]

जहां \(v_0\) प्रारंभिक गति, \(\theta\) लॉन्च कोण, और \(\phi\) अज़ीमुथल दिशा है।

\subsubsection{उदाहरण: पेचदार
गति}\label{ux909ux926ux939ux930ux923-ux92aux91aux926ux930-ux917ux924}

\[
\mathbf{r}(t) = \langle \cos t, \sin t, t \rangle
\]

\begin{itemize}
\tightlist
\item
  वेग: \(\mathbf{v}(t) = \langle -\sin t, \cos t, 1 \rangle\).
\item
  गति: \(|\mathbf{v}(t)| = \sqrt{2}\)।
\item
  त्वरण: \(\mathbf{a}(t) = \langle -\cos t, -\sin t, 0 \rangle\)।
\item
  गति गति में एक समान है, ऊपर की ओर सर्पिल है।
\end{itemize}

\subsubsection{यह क्यों मायने रखता
है}\label{ux92fux939-ux915ux92f-ux92eux92fux928-ux930ux916ux924-ux939-11}

\begin{itemize}
\tightlist
\item
  वास्तविक दुनिया की गति के लिए गणितीय भाषा प्रदान करता है।
\item
  भौतिकी में आवश्यक (बल, प्रक्षेप पथ, गोलाकार गति)।
\item
  उन्नत यांत्रिकी और इंजीनियरिंग मॉडल के लिए फाउंडेशन।
\end{itemize}

\subsubsection{व्यायाम}\label{ux935ux92fux92fux92e-28}

\begin{enumerate}
\def\labelenumi{\arabic{enumi}.}
\tightlist
\item
  एक कण \(\mathbf{r}(t) = \langle t, t^2, t^3 \rangle\) के अनुदिश गति करता
  है। \(t=1\) पर वेग और त्वरण ज्ञात करें।
\item
  दिखाएँ कि हेलिक्स \(\mathbf{r}(t) = \langle \cos t, \sin t, t \rangle\) के
  लिए गति स्थिर है।
\item
  एक प्रक्षेप्य को \(v_0 = 20 \,\text{m/s}\) के साथ \(45^\circ\) कोण पर
  प्रक्षेपित किया जाता है। ऊर्ध्वाधर तल में गति मानकर इसका स्थिति सदिश लिखिए।
\item
  \(\mathbf{r}(t) = \langle e^t, e^{-t}, t \rangle\) के लिए,
  \(\mathbf{v}(t)\) और \(\mathbf{a}(t)\) खोजें।
\item
  त्रिज्या \(r\) के एक वृत्त के अनुदिश गति के लिए त्वरण वेक्टर को स्पर्शरेखा और सामान्य
  घटकों में विघटित करें।\# अध्याय 8. अनेक चरों के कार्य
\end{enumerate}

\subsection{8.1 अनेक चरों में सीमाएँ और
निरंतरता}\label{ux905ux928ux915-ux91aux930-ux92e-ux938ux92eux90f-ux914ux930-ux928ux930ux924ux930ux924}

मल्टीवेरिएबल कैलकुलस में, फ़ंक्शन दो या दो से अधिक वेरिएबल्स पर निर्भर हो सकते हैं, जैसे
\(f(x,y)\) या \(f(x,y,z)\)। सीमा और निरंतरता की अवधारणाएँ स्वाभाविक रूप से
एकल-चर कलन से विस्तारित होती हैं, लेकिन वे अधिक सूक्ष्म हैं क्योंकि हमें दृष्टिकोण के सभी
संभावित रास्तों पर विचार करना चाहिए।

\subsubsection{दो चरों में
सीमाएँ}\label{ux926-ux91aux930-ux92e-ux938ux92eux90f}

फ़ंक्शन \(f(x,y)\) के लिए, हम कहते हैं

\[
\lim_{(x,y) \to (a,b)} f(x,y) = L
\]

यदि \(f(x,y)\) किसी भी पथ पर \((x,y)\) के \((a,b)\) के करीब पहुंचने पर मनमाने
ढंग से \(L\) के करीब पहुंच जाता है।

यदि अलग-अलग पथ अलग-अलग सीमा मान देते हैं, तो सीमा मौजूद नहीं है।

उदाहरण 1 (सीमा मौजूद है):

\[
f(x,y) = x^2 + y^2, \quad \lim_{(x,y) \to (0,0)} f(x,y) = 0.
\]

उदाहरण 2 (सीमा मौजूद नहीं है):

\[
f(x,y) = \frac{xy}{x^2+y^2}, \quad (x,y) \to (0,0).
\]

\begin{itemize}
\tightlist
\item
  \(y=0\) के साथ, फ़ंक्शन 0 है।
\item
  \(y=x\) के साथ, फ़ंक्शन \(\tfrac{1}{2}\) है। भिन्न परिणाम → सीमा मौजूद नहीं है।
\end{itemize}

\subsubsection{निरंतरता}\label{ux928ux930ux924ux930ux924-1}

एक फ़ंक्शन \(f(x,y)\) \((a,b)\) पर निरंतर है यदि

\[
\lim_{(x,y)\to(a,b)} f(x,y) = f(a,b).
\]

बहुपद और परिमेय फलन (जहाँ हर ≠ 0) अपने डोमेन में हर जगह निरंतर होते हैं।

\subsubsection{तीन या अधिक वेरिएबल्स का
विस्तार}\label{ux924ux928-ux92f-ux905ux927ux915-ux935ux930ux90fux92cux932ux938-ux915-ux935ux938ux924ux930}

\(f(x,y,z)\) के लिए, सीमाएं और निरंतरता को उसी तरह परिभाषित किया गया है,
लेकिन बिंदु \((a,b,c)\) तक अंतरिक्ष में अनंत दिशाओं से पहुंचना होगा।

\subsubsection{यह क्यों मायने रखता
है}\label{ux92fux939-ux915ux92f-ux92eux92fux928-ux930ux916ux924-ux939-12}

\begin{itemize}
\tightlist
\item
  निरंतरता यह सुनिश्चित करती है कि बहुपरिवर्तनीय कार्यों में कोई छलांग, छेद या
  स्पर्शोन्मुखता न हो।
\item
  आंशिक व्युत्पन्न और एकाधिक अभिन्न को परिभाषित करने के लिए सीमाएं मौलिक हैं।
\item
  ये अवधारणाएँ बहुपरिवर्तनीय कलन के लिए निर्माण खंड हैं।
\end{itemize}

\subsubsection{व्यायाम}\label{ux935ux92fux92fux92e-29}

\begin{enumerate}
\def\labelenumi{\arabic{enumi}.}
\tightlist
\item
  निर्धारित करें कि क्या \(\lim_{(x,y)\to(0,0)} (x^2+y^2)\) मौजूद है।
\item
  दिखाएँ कि \(\lim_{(x,y)\to(0,0)} \frac{x^2y}{x^2+y^2} = 0\) सभी सीधी
  रेखा वाले रास्तों पर \(y=mx\) है।
\item
  क्या \(f(x,y) = \frac{x^2-y^2}{x^2+y^2}\) के लिए सीमा \((x,y)\to(0,0)\)
  के रूप में मौजूद है?
\item
  बताएं कि दो चर वाले बहुपद हर जगह निरंतर क्यों होते हैं।
\item
  दो चरों वाले एक फलन का उदाहरण दीजिए जो एक बिंदु पर असंतत है और इसका कारण
  बताएं।
\end{enumerate}

\subsection{8.2 आंशिक
व्युत्पन्न}\label{ux906ux936ux915-ux935ux92fux924ux92aux928ux928}

कई चर वाले कार्यों में, हम अक्सर यह मापना चाहते हैं कि जब केवल एक चर बदलता है जबकि
अन्य को स्थिर रखा जाता है तो फ़ंक्शन कैसे बदलता है। इससे आंशिक व्युत्पन्न का विचार सामने
आता है।

\subsubsection{परिभाषा}\label{ux92aux930ux92dux937-8}

फ़ंक्शन \(f(x,y)\) के लिए, एक बिंदु \((a,b)\) पर \(x\) के संबंध में आंशिक व्युत्पन्न है

\[
\frac{\partial f}{\partial x}(a,b) = \lim_{h \to 0} \frac{f(a+h, b) - f(a,b)}{h}.
\]

इसी प्रकार, \(y\) के संबंध में आंशिक व्युत्पन्न है

\[\frac{\आंशिक f}{\आंशिक y}(a,b) = \lim_{h \to 0} \frac{f(a, b+h) - f(a,b)}{h}.
\]

We treat all other variables as constants when differentiating.

\subsubsection{Notation}\label{notation}

\begin{itemize}
\tightlist
\item
  \(\frac{\partial f}{\partial x}\), \(f_x\), \(\partial_x f\).
\item
  \(\frac{\partial f}{\partial y}\), \(f_y\), \(\partial_y f\).
\end{itemize}

For three variables \(f(x,y,z)\), we also have \(f_x, f_y, f_z\).

\subsubsection{Examples}\label{examples-2}

\begin{enumerate}
\def\labelenumi{\arabic{enumi}.}
\tightlist
\item
  \(f(x,y) = x^2y + y^3\)
\end{enumerate}

\begin{itemize}
\tightlist
\item
  \(f_x = 2xy\).
\item
  \(f_y = x^2 + 3y^2\).
\end{itemize}

\begin{enumerate}
\def\labelenumi{\arabic{enumi}.}
\setcounter{enumi}{1}
\tightlist
\item
  \(f(x,y) = e^{xy}\)
\end{enumerate}

\begin{itemize}
\tightlist
\item
  \(f_x = y e^{xy}\).
\item
  \(f_y = x e^{xy}\).
\end{itemize}

\begin{enumerate}
\def\labelenumi{\arabic{enumi}.}
\setcounter{enumi}{2}
\tightlist
\item
  \(f(x,y,z) = x^2 + yz\)
\end{enumerate}

\begin{itemize}
\tightlist
\item
  \(f_x = 2x\).
\item
  \(f_y = z\).
\item
  \(f_z = y\).
\end{itemize}

\subsubsection{Higher-Order Partial
Derivatives}\label{higher-order-partial-derivatives}

We can take partial derivatives repeatedly:

\begin{itemize}
\tightlist
\item
  \(f_{xx} = \frac{\partial}{\partial x}\Big(f_x\Big)\).
\item
  \(f_{yy}, f_{xy}, f_{yx}\), etc.
\end{itemize}

Clairaut's Theorem: If \(f\) has continuous second partial derivatives,
then

\[
f_{xy} = f_{yx}.
\]

\subsubsection{Geometric Meaning}\label{geometric-meaning}

\begin{itemize}
\tightlist
\item
  \(f_x\): slope of the surface in the \(x\)-direction.
\item
  \(f_y\): slope of the surface in the \(y\)-direction.
\item
  Together they describe how the surface tilts.
\end{itemize}

\subsubsection{Why This Matters}\label{why-this-matters}

\begin{itemize}
\tightlist
\item
  Partial derivatives are the foundation of gradients, tangent planes,
  and optimization in multiple variables.
\item
  They are widely used in physics, engineering, and economics to model
  systems with several inputs.
\end{itemize}

\subsubsection{Exercises}\label{exercises-3}

\begin{enumerate}
\def\labelenumi{\arabic{enumi}.}
\tightlist
\item
  Find \(f_x\) and \(f_y\) for \(f(x,y) = x^3y^2\).
\item
  Compute \(f_x, f_y, f_z\) for \(f(x,y,z) = xyz + x^2\).
\item
  Verify Clairaut's theorem for \(f(x,y) = x^2y + y^3\).
\item
  Interpret geometrically what \(f_x\) and \(f_y\) mean for
  \(f(x,y) = \sqrt{x^2+y^2}\).
\item
  Find all second-order partial derivatives of \(f(x,y) = e^{x^2+y^2}\).
\end{enumerate}

\subsection{8.3 Gradient and Directional
Derivatives}\label{gradient-and-directional-derivatives}

Partial derivatives measure change along the coordinate axes, but
sometimes we want to know the rate of change of a function in any
direction. This leads to the concepts of the gradient and directional
derivatives.

\subsubsection{Gradient Vector}\label{gradient-vector}

For a function \(f(x,y)\), the gradient is the vector

\[
\nabla f(x,y) = \left\lang \frac{\partial f}{\partial x}, \frac{\partial f}{\partial y} \right\rangel.
\]

For three variables \(f(x,y,z)\):

\[
\नाबला एफ(एक्स,वाई,जेड) = \बाएं\लैंगल एफ_एक्स, एफ_वाई, एफ_जेड \राइट\रंगल।
\]

The gradient points in the direction of maximum increase of the
function, and its magnitude gives the steepest slope.

\subsubsection{Directional Derivatives}\label{directional-derivatives}

The rate of change of \(f(x,y)\) at a point in the direction of a unit
vector \(\mathbf{u} = \langle u_1, u_2 \rangle\) is

\[
D_{\mathbf{u}} f(x,y) = \nabla f(x,y) \cdot \mathbf{u}.
\]

यह दिशा वेक्टर के साथ ग्रेडिएंट का डॉट उत्पाद है।

\subsubsection{उदाहरण}\label{ux909ux926ux939ux930ux923-19}

\begin{enumerate}
\def\labelenumi{\arabic{enumi}.}
\tightlist
\item
  \(f(x,y) = x^2 + y^2\)
\end{enumerate}

\begin{itemize}
\tightlist
\item
  ग्रेडिएंट: \(\nabla f = \langle 2x, 2y \rangle\)।- (1,2) पर:
  \(\nabla f = \langle 2,4 \rangle\)।
\item
  \(\mathbf{u} = \langle \tfrac{3}{5}, \tfrac{4}{5} \rangle\) के साथ
  दिशात्मक व्युत्पन्न:
\end{itemize}

\[
D_{\mathbf{u}} f(1,2) = \langle 2,4 \rangle \cdot \langle \tfrac{3}{5}, \tfrac{4}{5} \rangle = \tfrac{26}{5}.
\]

\begin{enumerate}
\def\labelenumi{\arabic{enumi}.}
\setcounter{enumi}{1}
\tightlist
\item
  \(f(x,y,z) = x y z\)
\end{enumerate}

\begin{itemize}
\tightlist
\item
  ग्रेडिएंट: \(\nabla f = \langle yz, xz, xy \rangle\).
\item
  (1,1,1) पर: \(\nabla f = \langle 1,1,1 \rangle\)।
\item
  अधिकतम वृद्धि दिशा \(\langle 1,1,1 \rangle\) के अनुरूप है।
\end{itemize}

\subsubsection{ज्यामितीय
व्याख्या}\label{ux91cux92fux92eux924ux92f-ux935ux92fux916ux92f-1}

\begin{itemize}
\tightlist
\item
  ग्रेडिएंट वेक्टर \(f\) के समतल वक्रों या समतल सतहों के लंबवत (सामान्य) है।
\item
  दिशात्मक व्युत्पन्न मनमानी दिशाओं में ढलान को सामान्यीकृत करते हैं।
\end{itemize}

\subsubsection{यह क्यों मायने रखता
है}\label{ux92fux939-ux915ux92f-ux92eux92fux928-ux930ux916ux924-ux939-13}

\begin{itemize}
\tightlist
\item
  अनुकूलन में, ढाल हमें सबसे तीव्र चढ़ाई या वंश के लिए आगे बढ़ने की दिशा बताती है।
\item
  भौतिकी में, ग्रेडिएंट ताप प्रवाह और विद्युत क्षमता जैसे क्षेत्रों का वर्णन करते हैं।
\item
  दिशात्मक व्युत्पन्न परिवर्तन की एकल-चर और बहु-परिवर्तनीय दरों को एकीकृत करते हैं।
\end{itemize}

\subsubsection{व्यायाम}\label{ux935ux92fux92fux92e-30}

\begin{enumerate}
\def\labelenumi{\arabic{enumi}.}
\tightlist
\item
  \(f(x,y) = e^{xy}\) के लिए \(\nabla f(x,y)\) की गणना करें।
\item
  \(f(x,y,z) = x^2+y^2+z^2\) का ग्रेडिएंट ज्ञात करें और (1,1,1) पर मूल्यांकन करें।
\item
  \(\mathbf{u} = \langle 0,1 \rangle\) की दिशा में (2,1) पर
  \(f(x,y) = x^2-y\) के दिशात्मक व्युत्पन्न की गणना करें।
\item
  दिखाएँ कि \(f(x,y) = x^2+y^2\) की ढाल वृत्त \(x^2+y^2=1\) पर लंबवत है।
\item
  इकाई वेक्टर दिशा खोजें जो \(f(x,y) = xy\) के दिशात्मक व्युत्पन्न को (1,2) पर
  अधिकतम करती है।
\end{enumerate}

\subsection{8.4 स्पर्शरेखा तल और रैखिक
सन्निकटन}\label{ux938ux92aux930ux936ux930ux916-ux924ux932-ux914ux930-ux930ux916ux915-ux938ux928ux928ux915ux91fux928}

एकल-चर कलन में, स्पर्शरेखा रेखा एक बिंदु के निकट एक वक्र का अनुमान लगाती है।
बहुपरिवर्तनीय कलन में, अनुरूप अवधारणा स्पर्शरेखा तल है, जो एक बिंदु के पास की सतह पर
एक रैखिक सन्निकटन प्रदान करती है।

\subsubsection{किसी सतह पर स्पर्शरेखा
तल}\label{ux915ux938-ux938ux924ux939-ux92aux930-ux938ux92aux930ux936ux930ux916-ux924ux932}

मान लीजिए \(z = f(x,y)\) \((a,b)\) पर भिन्न है। \((a,b,f(a,b))\) पर
स्पर्शरेखा तल द्वारा दिया गया है

\[
z = f(a,b) + f_x(a,b)(x-a) + f_y(a,b)(y-b).
\]

यह तल सतह को बिंदु पर छूता है और उसके निकट आ जाता है।

\subsubsection{उदाहरण 1:
परवलयज}\label{ux909ux926ux939ux930ux923-1-ux92aux930ux935ux932ux92fux91c}

\(f(x,y) = x^2 + y^2\) के लिए \((1,2)\) पर:

\begin{itemize}
\tightlist
\item
  \(f(1,2) = 1^2+2^2=5\).
\item
  \(f_x = 2x\), तो \(f_x(1,2) = 2\)।
\item
  \(f_y = 2y\), तो \(f_y(1,2) = 4\)।
\end{itemize}

स्पर्शरेखा तल का समीकरण:

\[
z = 5 + 2(x-1) + 4(y-2).
\]

\subsubsection{रैखिक
सन्निकटन}\label{ux930ux916ux915-ux938ux928ux928ux915ux91fux928}

स्पर्शरेखा तल का उपयोग \((a,b)\) के निकट \(f(x,y)\) का अनुमान लगाने के लिए किया
जा सकता है:

\[
f(x,y) \approx f(a,b) + f_x(a,b)(x-a) + f_y(a,b)(y-b).
\]

यह \((a,b)\) पर \(f\) का रैखिककरण है।

\subsubsection{उदाहरण 2: रैखिक
सन्निकटन}\label{ux909ux926ux939ux930ux923-2-ux930ux916ux915-ux938ux928ux928ux915ux91fux928}

अनुमानित \(f(x,y) = \sqrt{x+y}\) \((4,5)\) के निकट।

\begin{itemize}
\tightlist
\item
  \(f(4,5) = \sqrt{9} = 3\).
\item
  \(f_x = \frac{1}{2\sqrt{x+y}}, \quad f_y = \frac{1}{2\sqrt{x+y}}\).
\item
  (4,5) पर: \(f_x = f_y = \tfrac{1}{6}\)।
\end{itemize}

तो,

\[f(x,y) \लगभग 3 + \tfrac{1}{6}(x-4) + \tfrac{1}{6}(y-5).
\]

\subsubsection{Why This Matters}\label{why-this-matters-1}

\begin{itemize}
\tightlist
\item
  Tangent planes give the best linear approximation to a surface.
\item
  Linearization simplifies complex functions for computation.
\item
  Widely used in numerical methods, physics, and economics.
\end{itemize}

\subsubsection{Exercises}\label{exercises-4}

\begin{enumerate}
\def\labelenumi{\arabic{enumi}.}
\tightlist
\item
  Find the tangent plane to \(z = x^2y + y^2\) at \((1,1)\).
\item
  Approximate \(f(x,y) = e^{x+y}\) near \((0,0)\).
\item
  Derive the tangent plane equation for \(z = \ln(x^2+y^2)\) at
  \((1,1)\).
\item
  Use linear approximation to estimate \(\sqrt{10.1}\) using
  \(f(x,y) = \sqrt{x+y}\) near (4,6).
\item
  Explain why the tangent plane approximation improves as \((x,y)\) gets
  closer to \((a,b)\).
\end{enumerate}

\subsection{8.5 Optimization in Several
Variables}\label{optimization-in-several-variables}

Optimization in multivariable calculus extends the ideas of maxima and
minima from single-variable functions to functions of two or more
variables.

\subsubsection{Critical Points}\label{critical-points}

For \(f(x,y)\), a critical point occurs where

\[
f_x(x,y) = 0 \quad \text{and} \quad f_y(x,y) = 0,
\]

or where the partial derivatives do not exist.

\subsubsection{Second Derivative Test}\label{second-derivative-test}

To classify critical points, compute the second partial derivatives:

\[
D = f_{xx}(a,b) f_{yy}(a,b) - \big(f_{xy}(a,b)\big)^2.
\]

\begin{itemize}
\tightlist
\item
  If \(D > 0\) and \(f_{xx}(a,b) > 0\): local minimum.
\item
  If \(D > 0\) and \(f_{xx}(a,b) < 0\): local maximum.
\item
  If \(D < 0\): saddle point.
\item
  If \(D = 0\): test is inconclusive.
\end{itemize}

\subsubsection{Example 1: Paraboloid}\label{example-1-paraboloid}

\(f(x,y) = x^2 + y^2\).

\begin{itemize}
\tightlist
\item
  \(f_x = 2x, f_y = 2y\). Critical point at (0,0).
\item
  \(f_{xx} = 2, f_{yy} = 2, f_{xy} = 0\).
\item
  \(D = (2)(2) - 0 = 4 > 0\), and \(f_{xx} > 0\).
\item
  So (0,0) is a local minimum.
\end{itemize}

\subsubsection{Example 2: Saddle Point}\label{example-2-saddle-point}

\(f(x,y) = x^2 - y^2\).

\begin{itemize}
\tightlist
\item
  \(f_x = 2x, f_y = -2y\). Critical point at (0,0).
\item
  \(f_{xx} = 2, f_{yy} = -2, f_{xy} = 0\).
\item
  \(D = (2)(-2) - 0 = -4 < 0\).
\item
  So (0,0) is a saddle point.
\end{itemize}

\subsubsection{Constrained Optimization and Lagrange
Multipliers}\label{constrained-optimization-and-lagrange-multipliers}

Sometimes, we want to optimize \(f(x,y)\) subject to a constraint
\(g(x,y) = c\).

Method of Lagrange multipliers: solve

\[
\nabla f(x,y) = \lambda \nabla g(x,y).
\]

उदाहरण: \(x^2+y^2=1\) के अधीन \(f(x,y) = xy\) को अधिकतम करें।

\begin{itemize}
\tightlist
\item
  स्नातक:
  \(\nabla f = \langle y,x \rangle, \quad \nabla g = \langle 2x,2y \rangle\)।
\item
  समीकरण: \(y = 2\lambda x, \, x = 2\lambda y\).
\item
  समाधान अधिकतम \((\pm \tfrac{1}{\sqrt{2}}, \pm \tfrac{1}{\sqrt{2}})\)
  पर ले जाते हैं।
\end{itemize}

\subsubsection{यह क्यों मायने रखता
है}\label{ux92fux939-ux915ux92f-ux92eux92fux928-ux930ux916ux924-ux939-14}

\begin{itemize}
\tightlist
\item
  अर्थशास्त्र, इंजीनियरिंग, मशीन लर्निंग और भौतिकी में अनुकूलन आवश्यक है।
\item
  लैग्रेंज मल्टीप्लायर बाधाओं के साथ अनुकूलन की अनुमति देते हैं, जो अनुप्रयुक्त गणित में एक
  प्रमुख उपकरण है।
\end{itemize}

\subsubsection{व्यायाम}\label{ux935ux92fux92fux92e-31}

\begin{enumerate}
\def\labelenumi{\arabic{enumi}.}
\tightlist
\item
  \(f(x,y) = x^2+xy+y^2\) के महत्वपूर्ण बिंदुओं को ढूंढें और वर्गीकृत करें।
\item
  \(f(x,y) = x^3-y^3\) के लिए बिंदु (0,0) को वर्गीकृत करें।3.
  \(f(x,y) = x^4+y^4-4xy\) के लिए दूसरे व्युत्पन्न परीक्षण का उपयोग करें।
\item
  \(x^2+y^2=1\) के अधीन \(f(x,y) = x+y\) को अधिकतम करें।
\item
  \(x+y=1\) के अधीन \(f(x,y) = x^2+2y^2\) को न्यूनतम करें।
\end{enumerate}

\section{अध्याय 9. एकाधिक
इंटीग्रल}\label{ux905ux927ux92fux92f-9.-ux90fux915ux927ux915-ux907ux91fux917ux930ux932}

\subsection{9.1 डबल
इंटीग्रल}\label{ux921ux92cux932-ux907ux91fux917ux930ux932}

एकल-चर कलन में, एक निश्चित समाकलन एक वक्र के नीचे का क्षेत्र देता है। दो चर में, एक
डबल इंटीग्रल एक सतह के नीचे की मात्रा (या अधिक सामान्यतः, एक क्षेत्र पर मूल्यों का
संचय) की गणना करता है।

\subsubsection{परिभाषा}\label{ux92aux930ux92dux937-9}

यदि \(f(x,y)\) किसी क्षेत्र \(R\) पर सतत है, तो दोहरा समाकलन है

\[
\iint_R f(x,y)\, dA = \lim_{m,n \to \infty} \sum_{i=1}^m \sum_{j=1}^n f(x_{ij}^-, y_{ij}^-) \Delta A,
\]

जहां \(R\) को \(\Delta A\) क्षेत्रफल के छोटे आयतों में विभाजित किया गया है।

\subsubsection{पुनरावृत्त
इंटीग्रल}\label{ux92aux928ux930ux935ux924ux924-ux907ux91fux917ux930ux932}

फ़ुबिनी के प्रमेय के अनुसार, हम एक दोहरे अभिन्न अंग की गणना एक पुनरावृत्त अभिन्न अंग के
रूप में कर सकते हैं:

\[
\iint_R f(x,y)\, dA = \int_a^b \int_c^d f(x,y)\, dy\, dx,
\]

यदि \(R\) एक आयत \([a,b] \times [c,d]\) है।

एकीकरण का क्रम अक्सर बदला जा सकता है:

\[
\int_a^b \int_c^d f(x,y)\,dy\,dx = \int_c^d \int_a^b f(x,y)\,dx\,dy.
\]

\subsubsection{उदाहरण}\label{ux909ux926ux939ux930ux923-20}

\begin{enumerate}
\def\labelenumi{\arabic{enumi}.}
\tightlist
\item
  आयत क्षेत्र
\end{enumerate}

\[
\iint_R (x+y)\, dA, \quad R=[0,1]\times[0,2].
\]

\[
= \int_0^1 \int_0^2 (x+y)\,dy\,dx = \int_0^1 \Big[xy+\tfrac{1}{2}y^2\Big]_0^2 dx
= \int_0^1 (2x+2)dx = 3.
\]

\begin{enumerate}
\def\labelenumi{\arabic{enumi}.}
\setcounter{enumi}{1}
\tightlist
\item
  त्रिकोणीय क्षेत्र
\end{enumerate}

\[
R = \{(x,y): 0 \leq x \leq 1, 0 \leq y \leq x\}.
\]

\[
\iint_R (x+y)\, dA = \int_0^1 \int_0^x (x+y)\,dy\,dx.
\]

मूल्यांकन करने पर \(\tfrac{2}{3}\) मिलता है।

\subsubsection{अनुप्रयोग}\label{ux905ux928ux92aux930ux92fux917-1}

\begin{itemize}
\tightlist
\item
  किसी सतह के नीचे का आयतन:
\end{itemize}

\[
V = \iint_R f(x,y)\, dA.
\]

\begin{itemize}
\tightlist
\item
  किसी क्षेत्र में किसी फ़ंक्शन का औसत मान:
\end{itemize}

\[
f_{\text{avg}} = \frac{1}{A(R)} \iint_R f(x,y)\, dA.
\]

\subsubsection{यह क्यों मायने रखता
है}\label{ux92fux939-ux915ux92f-ux92eux92fux928-ux930ux916ux924-ux939-15}

डबल इंटीग्रल एकीकरण को दो आयामों तक विस्तारित करते हैं। वे भौतिकी (द्रव्यमान,
संभाव्यता वितरण), अर्थशास्त्र (अपेक्षित मान), और इंजीनियरिंग (केन्द्रक, प्रवाह) में
आवश्यक हैं।

\subsubsection{व्यायाम}\label{ux935ux92fux92fux92e-32}

\begin{enumerate}
\def\labelenumi{\arabic{enumi}.}
\tightlist
\item
  \(\iint_R (x^2+y^2)\, dA\) का मूल्यांकन करें जहां \(R=[0,1]\times[0,1]\) है।
\item
  \(\iint_R xy\, dA\) की गणना करें जहां
  \(R=\{(x,y):0\leq x\leq2,0\leq y\leq x\}\) है।
\item
  इकाई वर्ग \([0,1]\times[0,1]\) पर \(f(x,y) = x+y\) का औसत मान ज्ञात
  कीजिए।
\item
  यदि \(f(x,y)\) एक संभाव्यता घनत्व फ़ंक्शन है, तो संभाव्यता के संदर्भ में
  \(\iint_R f(x,y)\, dA\) की व्याख्या करें।
\item
  दिखाएँ कि एकीकरण का स्विचिंग क्रम \(\iint_{[0,1]\times[0,2]} (x+y)\,dA\) के
  लिए समान परिणाम देता है।
\end{enumerate}

\subsection{9.2 ट्रिपल
इंटीग्रल्स}\label{ux91fux930ux92aux932-ux907ux91fux917ux930ux932ux938}

ट्रिपल इंटीग्रल्स एकीकरण के विचार को तीन चर तक विस्तारित करते हैं, जिससे हमें
त्रि-आयामी क्षेत्रों में वॉल्यूम, द्रव्यमान और अन्य मात्राओं की गणना करने की अनुमति
मिलती है।

\subsubsection{परिभाषा}\label{ux92aux930ux92dux937-10}

यदि \(f(x,y,z)\) एक ठोस क्षेत्र \(E\) पर निरंतर है, तो ट्रिपल इंटीग्रल है

\[\iiint_E f(x,y,z)\, dV = \lim_{m,n,p \to \infty} \sum f(x_{ijk}^-, y_{ijk}^-, z_{ijk}^-) \Delta V,
\]

where the region is subdivided into boxes of volume \(\Delta V\).

\subsubsection{Iterated Integrals}\label{iterated-integrals}

By Fubini's Theorem, a triple integral can be computed as an iterated
integral:

\[
\iiint_E f(x,y,z)\, dV = \int_a^b \int_c^d \int_e^f f(x,y,z)\, dz\, dy\, dx,
\]

for a rectangular box \(E = [a,b]\times[c,d]\times[e,f]\).

The order of integration can be chosen for convenience.

\subsubsection{Examples}\label{examples-3}

\begin{enumerate}
\def\labelenumi{\arabic{enumi}.}
\tightlist
\item
  Rectangular box
\end{enumerate}

\[
\iiint_E xyz\, dV, \quad E=[0,1]\times[0,2]\times[0,3]।
\]

\[
= \int_0^1 \int_0^2 \int_0^3 xyz\,dz\,dy\,dx.
\]

First integrate over \(z\):

\[
\int_0^3 xyz\,dz = xy \left[\tfrac{1}{2}z^2\right]_0^3 = \tfrac{9}{2}xy.
\]

Now integrate over \(y\):

\[
\int_0^2 \tfrac{9}{2}xy\,dy = \tfrac{9}{2}x \cdot \left[\tfrac{1}{2}y^2\right]_0^2 = 9x.
\]

Finally integrate over \(x\):

\[
\int_0^1 9x\,dx = \tfrac{9}{2}.
\]

\begin{enumerate}
\def\labelenumi{\arabic{enumi}.}
\setcounter{enumi}{1}
\tightlist
\item
  Region bounded by planes Let
  \(E = \{(x,y,z) \mid 0 \leq x \leq 1, 0 \leq y \leq x, 0 \leq z \leq y\}\).
\end{enumerate}

\[
\iiint_E 1\,dV = \int_0^1 \int_0^x \int_0^y 1\,dz\,dy\,dx.
\]

Evaluate:

\[
= \int_0^1 \int_0^x y\,dy\,dx = \int_0^1 \tfrac{1}{2}x^2\,dx = \tfrac{1}{6}.
\]

So the volume of this triangular region is \(\tfrac{1}{6}\).

\subsubsection{Applications}\label{applications}

\begin{itemize}
\item
  Volume: \(V = \iiint_E 1 \, dV\).
\item
  Mass: If density is \(\rho(x,y,z)\), then

  \[
  एम = \iiint_E \rho(x,y,z)\, dV.
  \]
\item
  Average value:

  \[
  f_{\text{avg}} = \frac{1}{V(E)} \iiint_E f(x,y,z)\,dV.
  \]
\end{itemize}

\subsubsection{Why This Matters}\label{why-this-matters-2}

Triple integrals generalize area and volume calculations to arbitrary
solids. They are used in physics (mass distributions, center of mass,
gravitational fields), engineering, and probability.

\subsubsection{Exercises}\label{exercises-5}

\begin{enumerate}
\def\labelenumi{\arabic{enumi}.}
\tightlist
\item
  Compute \(\iiint_E (x+y+z)\,dV\) over the cube
  \(E=[0,1]\times[0,1]\times[0,1]\).
\item
  Find the volume of the tetrahedron bounded by
  \(x=0, y=0, z=0, x+y+z=1\).
\item
  Evaluate \(\iiint_E x^2 \, dV\) where
  \(E=[0,2]\times[0,1]\times[0,1]\).
\item
  Show that \(\iiint_E 1\,dV\) equals the geometric volume of \(E\).
\item
  If density is \(\rho(x,y,z)=x+y+z\), compute the mass of the unit
  cube.
\end{enumerate}

\subsection{9.3 Applications: Volume, Mass,
Probability}\label{applications-volume-mass-probability}

Triple integrals are powerful because they allow us to compute
quantities in three dimensions by accumulating values over a solid
region.

\subsubsection{Volume}\label{volume}

The simplest application is finding the volume of a region \(E\):

\[
वी = \iiint_E 1 \, डीवी.
\]

Example: Find the volume of the solid bounded by the coordinate planes
and the plane \(x+y+z=1\).

\[
V = \iiint_E 1 \, dV = \int_0^1 \int_0^{1-x} \int_0^{1-x-y} 1 \, dz\, dy\, dx.
\]

मूल्यांकन करने पर \(V = \tfrac{1}{6}\) मिलता है।\#\#\# द्रव्यमान और घनत्व

यदि किसी ठोस का घनत्व फलन \(\rho(x,y,z)\) है, तो उसका द्रव्यमान है

\[
M = \iiint_E \rho(x,y,z)\, dV.
\]

द्रव्यमान का केंद्र द्वारा दिया गया है

\[
\bar{x} = \frac{1}{M}\iiint_E x\rho(x,y,z)\,dV, \quad
\bar{y} = \frac{1}{M}\iiint_E y\rho(x,y,z)\,dV, \quad
\bar{z} = \frac{1}{M}\iiint_E z\rho(x,y,z)\,dV.
\]

उदाहरण: स्थिर घनत्व \(\rho=1\) वाले इकाई घन के लिए, द्रव्यमान का केंद्र
\((0.5,0.5,0.5)\) पर है।

\subsubsection{संभावना}\label{ux938ux92dux935ux928}

यदि \(f(x,y,z)\) 3D में एक संभाव्यता घनत्व फ़ंक्शन है, तो संभावना है कि यादृच्छिक चर
एक क्षेत्र \(E\) में स्थित है

\[
P(E) = \iiint_E f(x,y,z)\, dV,
\]

जहां \(f(x,y,z) \geq 0\) और

\[
\iiint_{\mathbb{R}^3} f(x,y,z)\,dV = 1.
\]

उदाहरण: यदि \(f(x,y,z) = \tfrac{3}{4}z^2\) के लिए \(0 \leq z \leq 1\),
समान रूप से \(x,y\) में, तो

\[
P(0 \leq z \leq 0.5) = \int_0^{0.5} \tfrac{3}{4}z^2 \, dz = \tfrac{1}{32}.
\]

\subsubsection{यह क्यों मायने रखता
है}\label{ux92fux939-ux915ux92f-ux92eux92fux928-ux930ux916ux924-ux939-16}

\begin{itemize}
\tightlist
\item
  वॉल्यूम अनियमित ठोस पदार्थों के लिए ज्यामिति को सामान्यीकृत करते हैं।
\item
  द्रव्यमान और घनत्व इंटीग्रल्स कैलकुलस को भौतिकी और इंजीनियरिंग से जोड़ते हैं।
\item
  उच्च आयामों में संभाव्यता घनत्व फ़ंक्शन का व्यापक रूप से सांख्यिकी और डेटा विज्ञान में
  उपयोग किया जाता है।
\end{itemize}

\subsubsection{व्यायाम}\label{ux935ux92fux92fux92e-33}

\begin{enumerate}
\def\labelenumi{\arabic{enumi}.}
\tightlist
\item
  \(x^2+y^2+z^2 \leq 1\) (इकाई गोला) से घिरे ठोस का आयतन ज्ञात कीजिए।
\item
  \(z\) के आनुपातिक घनत्व वाले शंकु के द्रव्यमान की गणना करें।
\item
  \(x=0, y=0, z=0, x+y+z=1\) से घिरे एकसमान चतुष्फलक के द्रव्यमान का केंद्र ज्ञात
  कीजिए।
\item
  यदि \(f(x,y,z) = \frac{1}{8}\) घन \([0,2]\times[0,2]\times[0,2]\) पर
  है, तो सत्यापित करें कि यह एक संभाव्यता घनत्व फ़ंक्शन है।
\item
  इस संभावना की गणना करने के लिए ट्रिपल इंटीग्रल का उपयोग करें कि इकाई क्षेत्र में
  यादृच्छिक रूप से चुने गए बिंदु में \(z > 0\) है।
\end{enumerate}

\subsection{9.4 चरों का परिवर्तन: ध्रुवीय, बेलनाकार, गोलाकार
निर्देशांक}\label{ux91aux930-ux915-ux92aux930ux935ux930ux924ux928-ux927ux930ux935ux92f-ux92cux932ux928ux915ux930-ux917ux932ux915ux930-ux928ux930ux926ux936ux915}

जब क्षेत्र की समरूपता से मेल खाने वाली समन्वय प्रणालियों में व्यक्त किया जाता है तो कई
अभिन्न अंग आसान हो जाते हैं। कार्टेशियन निर्देशांक \((x,y,z)\) के बजाय, हम ध्रुवीय,
बेलनाकार या गोलाकार निर्देशांक का उपयोग कर सकते हैं।

\subsubsection{ध्रुवीय निर्देशांक
(2डी)}\label{ux927ux930ux935ux92f-ux928ux930ux926ux936ux915-2ux921}

दो चर के कार्यों के लिए, हम ध्रुवीय निर्देशांक पर स्विच कर सकते हैं:

\[
x = r\cos\theta, \quad y = r\sin\theta, \quad r \geq 0, \; 0 \leq \theta < 2\pi.
\]

क्षेत्र तत्व के रूप में रूपांतरित होता है

\[
dA = r\,dr\,d\theta.
\]

उदाहरण: इकाई वृत्त का क्षेत्रफल ज्ञात कीजिए।

\[
A = \iint_{x^2+y^2\leq 1} 1\,dA = \int_0^{2\pi}\int_0^1 r\,dr\,d\theta = \pi.
\]

\subsubsection{बेलनाकार निर्देशांक
(3डी)}\label{ux92cux932ux928ux915ux930-ux928ux930ux926ux936ux915-3ux921}

3D में, बेलनाकार निर्देशांक \(z\) के साथ ध्रुवीय निर्देशांक का विस्तार करते हैं:

\[
x = r\cos\theta, \quad y = r\sin\theta, \quad z = z.
\]

आयतन तत्व है

\[
dV = r\,dr\,d\theta\,dz.
\]

उदाहरण: त्रिज्या \(R\) और ऊंचाई \(h\) वाले सिलेंडर का आयतन:

\[
V = \int_0^h \int_0^{2\pi} \int_0^R r\,dr\,d\theta\,dz = \pi R^2 h.
\]\#\#\# गोलाकार निर्देशांक (3डी)

गोलाकार समरूपता के लिए, उपयोग करें:

\[
x = \rho \sin\phi \cos\theta, \quad y = \rho \sin\phi \sin\theta, \quad z = \rho \cos\phi,
\]

कहाँ

\begin{itemize}
\tightlist
\item
  \(\rho \geq 0\) मूल बिंदु से दूरी है,
\item
  \(0 \leq \phi \leq \pi\) धनात्मक \(z\)-अक्ष से कोण है,
\item
  \(0 \leq \theta < 2\pi\) \(xy\)-तल में कोण है।
\end{itemize}

आयतन तत्व है

\[
dV = \rho^2 \sin\phi \, d\rho\, d\phi\, d\theta.
\]

उदाहरण: इकाई क्षेत्र का आयतन:

\[
V = \int_0^{2\pi} \int_0^\pi \int_0^1 \rho^2 \sin\phi \, d\rho\, d\phi\, d\theta.
\]

मूल्यांकन:

\[
V = \left(\int_0^1 \rho^2 d\rho\right)\left(\int_0^\pi \sin\phi d\phi\right)\left(\int_0^{2\pi} d\theta\right) = \tfrac{1}{3}(2)(2\pi) = \tfrac{4\pi}{3}.
\]

\subsubsection{यह क्यों मायने रखता
है}\label{ux92fux939-ux915ux92f-ux92eux92fux928-ux930ux916ux924-ux939-17}

\begin{itemize}
\tightlist
\item
  ध्रुवीय निर्देशांक वृत्ताकार क्षेत्रों को सरल बनाते हैं।
\item
  बेलनाकार निर्देशांक सिलेंडर और घूर्णी समरूपता को संभालते हैं।
\item
  गोलाकार निर्देशांक गोले, शंकु और रेडियल समस्याओं को सरल बनाते हैं।
\item
  चरों के ये परिवर्तन अन्यथा असंभव अभिन्नों को प्रबंधनीय बनाते हैं।
\end{itemize}

\subsubsection{व्यायाम}\label{ux935ux92fux92fux92e-34}

\begin{enumerate}
\def\labelenumi{\arabic{enumi}.}
\tightlist
\item
  ध्रुवीय निर्देशांक का उपयोग करके \(\iint_{x^2+y^2\leq 4} (x^2+y^2)\,dA\) की
  गणना करें।
\item
  बेलनाकार निर्देशांक का उपयोग करके ऊंचाई \(h\) और त्रिज्या \(R\) वाले शंकु का आयतन
  ज्ञात करें।
\item
  \(R\) त्रिज्या वाली गेंद के आयतन का मूल्यांकन करने के लिए गोलाकार निर्देशांक का
  उपयोग करें।
\item
  दिखाएँ कि ध्रुवीय निर्देशांक के लिए जैकोबियन कारक \(r\) है।
\item
  गोलाकार निर्देशांक का उपयोग करके मूल बिंदु से दूरी के समानुपाती घनत्व वाले \(R\)
  त्रिज्या वाले एक ठोस गोले का द्रव्यमान ज्ञात करें।
\end{enumerate}

\section{अध्याय 10. वेक्टर
कैलकुलस}\label{ux905ux927ux92fux92f-10.-ux935ux915ux91fux930-ux915ux932ux915ux932ux938}

\subsection{10.1 वेक्टर फ़ील्ड}\label{ux935ux915ux91fux930-ux92bux932ux921}

एक वेक्टर फ़ील्ड अंतरिक्ष में प्रत्येक बिंदु पर एक वेक्टर निर्दिष्ट करता है, ठीक उसी तरह
जैसे एक अदिश फ़ंक्शन एक संख्या निर्दिष्ट करता है। वेक्टर फ़ील्ड का उपयोग प्रवाह, बल और
अन्य दिशात्मक मात्राओं को मॉडल करने के लिए किया जाता है।

\subsubsection{परिभाषा}\label{ux92aux930ux92dux937-11}

दो आयामों में, एक वेक्टर फ़ील्ड एक फ़ंक्शन है

\[
\mathbf{F}(x,y) = \langle P(x,y), Q(x,y) \rangle,
\]

जहां \(P\) और \(Q\) अदिश फलन हैं।

तीन आयामों में,

\[
\mathbf{F}(x,y,z) = \langle P(x,y,z), Q(x,y,z), R(x,y,z) \rangle.
\]

\subsubsection{उदाहरण}\label{ux909ux926ux939ux930ux923-21}

\begin{enumerate}
\def\labelenumi{\arabic{enumi}.}
\tightlist
\item
  रेडियल क्षेत्र
\end{enumerate}

\[
\mathbf{F}(x,y) = \langle x, y \rangle.
\]

सदिश मूल बिंदु से बाहर की ओर इंगित करते हैं।

\begin{enumerate}
\def\labelenumi{\arabic{enumi}.}
\setcounter{enumi}{1}
\tightlist
\item
  घूर्णी क्षेत्र
\end{enumerate}

\[
\mathbf{F}(x,y) = \langle -y, x \rangle.
\]

सदिश मूल के चारों ओर घूमते हैं।

\begin{enumerate}
\def\labelenumi{\arabic{enumi}.}
\setcounter{enumi}{2}
\tightlist
\item
  गुरुत्वाकर्षण क्षेत्र
\end{enumerate}

\[
\mathbf{F}(x,y,z) = -\frac{GM}{r^3}\langle x,y,z \rangle, \quad r=\sqrt{x^2+y^2+z^2}.
\]

\subsubsection{वेक्टर फ़ील्ड्स को विज़ुअलाइज़
करना}\label{ux935ux915ux91fux930-ux92bux932ux921ux938-ux915-ux935ux91cux905ux932ux907ux91c-ux915ux930ux928}

\begin{itemize}
\tightlist
\item
  दिशा और परिमाण को इंगित करने के लिए नमूना बिंदुओं पर छोटे तीर बनाएं।
\item
  सघन तीर जहां परिमाण बड़े होते हैं।
\item
  प्रवाह रेखाओं, प्रक्षेप पथों और बलों की व्याख्या करने के लिए उपयोगी।
\end{itemize}

\subsubsection{\texorpdfstring{प्रवाह रेखाएँएक सदिश क्षेत्र की प्रवाह रेखा (या
अभिन्न वक्र) एक वक्र \(\mathbf{r}(t)\) होती है जिसका प्रत्येक बिंदु पर स्पर्शरेखा
सदिश क्षेत्र से मेल खाता
है:}{प्रवाह रेखाएँएक सदिश क्षेत्र की प्रवाह रेखा (या अभिन्न वक्र) एक वक्र \textbackslash mathbf\{r\}(t) होती है जिसका प्रत्येक बिंदु पर स्पर्शरेखा सदिश क्षेत्र से मेल खाता है:}}\label{ux92aux930ux935ux939-ux930ux916ux90fux90fux915-ux938ux926ux936-ux915ux937ux924ux930-ux915-ux92aux930ux935ux939-ux930ux916-ux92f-ux905ux92dux928ux928-ux935ux915ux930-ux90fux915-ux935ux915ux930-mathbfrt-ux939ux924-ux939-ux91cux938ux915-ux92aux930ux924ux92fux915-ux92cux926-ux92aux930-ux938ux92aux930ux936ux930ux916-ux938ux926ux936-ux915ux937ux924ux930-ux938-ux92eux932-ux916ux924-ux939}

\[
\mathbf{r}'(t) = \mathbf{F}(\mathbf{r}(t)).
\]

प्रवाह रेखाएँ वेग क्षेत्र में कण पथ का वर्णन करती हैं।

\subsubsection{यह क्यों मायने रखता
है}\label{ux92fux939-ux915ux92f-ux92eux92fux928-ux930ux916ux924-ux939-18}

\begin{itemize}
\tightlist
\item
  सदिश क्षेत्र भौतिकी (द्रव प्रवाह, विद्युत चुंबकत्व, गुरुत्वाकर्षण) में मौलिक हैं।
\item
  वे लाइन इंटीग्रल्स, सरफेस इंटीग्रल्स और वेक्टर कैलकुलस (ग्रीन, स्टोक्स, डाइवर्जेंस) के बड़े
  प्रमेयों का आधार बनाते हैं।
\item
  दिशात्मक मात्राओं को दर्शाने के लिए एक ज्यामितीय तरीका प्रदान करें।
\end{itemize}

\subsubsection{व्यायाम}\label{ux935ux92fux92fux92e-35}

\begin{enumerate}
\def\labelenumi{\arabic{enumi}.}
\tightlist
\item
  वेक्टर फ़ील्ड \(\mathbf{F}(x,y) = \langle y, -x \rangle\) को स्केच करें।
\item
  निर्धारित करें कि \(\mathbf{F}(x,y) = \langle x,y \rangle\) के सदिश मूल बिंदु
  की ओर इंगित करते हैं या उससे दूर।
\item
  \(\mathbf{F}(x,y,z) = \langle y, z, x \rangle\) के लिए,
  \(\mathbf{F}(1,2,3)\) की गणना करें।
\item
  \(\mathbf{F}(x,y) = \langle -y, x \rangle\) की प्रवाह रेखाओं का वर्णन करें।
\item
  बताएं कि गुरुत्वाकर्षण और विद्युत क्षेत्र रेडियल वेक्टर क्षेत्रों के उदाहरण क्यों हैं।
\end{enumerate}

\subsection{10.2 लाइन
इंटीग्रल्स}\label{ux932ux907ux928-ux907ux91fux917ux930ux932ux938}

एक लाइन इंटीग्रल एक वक्र के साथ मूल्यांकन किए गए कार्यों के इंटीग्रल के विचार को
विस्तारित करता है। किसी अंतराल या क्षेत्र में एकीकृत होने के बजाय, हम अंतरिक्ष में एक
पथ पर एकीकृत होते हैं।

\subsubsection{परिभाषा: अदिश रेखा
समाकलन}\label{ux92aux930ux92dux937-ux905ux926ux936-ux930ux916-ux938ux92eux915ux932ux928}

यदि \(f(x,y)\) एक अदिश फलन है और \(C\)
\(\mathbf{r}(t) = \langle x(t), y(t) \rangle, \; a \leq t \leq b\) द्वारा
पैरामीटरयुक्त एक वक्र है, तो रेखा समाकलन है

\[
\int_C f(x,y)\, ds = \int_a^b f(x(t),y(t)) \, |\mathbf{r}'(t)|\, dt,
\]

जहां \(ds\) चाप की लंबाई है।

यह वक्र के साथ \(f\) के संचय को मापता है।

\subsubsection{परिभाषा: वेक्टर लाइन
इंटीग्रल}\label{ux92aux930ux92dux937-ux935ux915ux91fux930-ux932ux907ux928-ux907ux91fux917ux930ux932}

एक सदिश क्षेत्र \(\mathbf{F}(x,y) = \langle P(x,y), Q(x,y) \rangle\) के लिए,
\(C\) के साथ अभिन्न रेखा है

\[
\int_C \mathbf{F} \cdot d\mathbf{r} = \int_a^b \mathbf{F}(\mathbf{r}(t)) \cdot \mathbf{r}'(t)\, dt.
\]

यह वक्र के अनुदिश क्षेत्र द्वारा किए गए कार्य को मापता है।

\subsubsection{उदाहरण}\label{ux909ux926ux939ux930ux923-22}

\begin{enumerate}
\def\labelenumi{\arabic{enumi}.}
\tightlist
\item
  अदिश रेखा समाकलन
\end{enumerate}

\[
f(x,y) = x+y, \quad C: \mathbf{r}(t) = \langle t, t^2 \rangle, \; 0 \leq t \leq 1.
\]

फिर

\[
\int_C f(x,y)\, ds = \int_0^1 (t+t^2)\sqrt{(1)^2+(2t)^2}\, dt.
\]

\begin{enumerate}
\def\labelenumi{\arabic{enumi}.}
\setcounter{enumi}{1}
\tightlist
\item
  किसी बल द्वारा किया गया कार्य
\end{enumerate}

\[
\mathbf{F}(x,y) = \langle y, x \rangle, \quad C: \mathbf{r}(t) = \langle t, t^2 \rangle, \; 0 \leq t \leq 1.
\]

\[
\int_C \mathbf{F} \cdot d\mathbf{r} = \int_0^1 \langle t^2, t \rangle \cdot \langle 1, 2t \rangle\, dt = \int_0^1 (t^2 + 2t^2)\, dt = \int_0^1 3t^2\, dt = 1.
\]

\subsubsection{शारीरिक
व्याख्या}\label{ux936ux930ux930ux915-ux935ux92fux916ux92f}

\begin{itemize}
\tightlist
\item
  अदिश रेखा अभिन्न: एक तार के साथ घनत्व का संचय।
\item
  वेक्टर लाइन इंटीग्रल: किसी वस्तु को पथ पर ले जाने वाले बल द्वारा किया गया कार्य।
\end{itemize}

\subsubsection{यह क्यों मायने रखता है- लाइन इंटीग्रल वेक्टर फ़ील्ड को कार्य और
परिसंचरण जैसी भौतिक मात्राओं से जोड़ते
हैं।}\label{ux92fux939-ux915ux92f-ux92eux92fux928-ux930ux916ux924-ux939--ux932ux907ux928-ux907ux91fux917ux930ux932-ux935ux915ux91fux930-ux92bux932ux921-ux915-ux915ux930ux92f-ux914ux930-ux92aux930ux938ux91aux930ux923-ux91cux938-ux92dux924ux915-ux92eux924ux930ux913-ux938-ux91cux921ux924-ux939}

\begin{itemize}
\tightlist
\item
  वे ग्रीन के प्रमेय और स्टोक्स के प्रमेय के लिए ब्लॉक बना रहे हैं।
\item
  भौतिकी में प्रकट (विद्युत क्षमता, द्रव प्रवाह, यांत्रिकी)।
\end{itemize}

\subsubsection{व्यायाम}\label{ux935ux92fux92fux92e-36}

\begin{enumerate}
\def\labelenumi{\arabic{enumi}.}
\tightlist
\item
  \(\int_C (x^2+y^2)\, ds\) की गणना करें जहां \(C\) (0,0) से (1,1) तक का रेखा
  खंड है।
\item
  यूनिट सर्कल \(x^2+y^2=1\) के साथ
  \(\mathbf{F}(x,y) = \langle -y, x \rangle\) के लिए
  \(\int_C \mathbf{F}\cdot d\mathbf{r}\) का मूल्यांकन करें।
\item
  \(\int_C 1\,ds\) का अर्थ स्पष्ट करें।
\item
  \(\mathbf{F}(x,y,z) = \langle z,0,x \rangle\) के लिए,
  \(\mathbf{r}(t) = \langle t,t,1 \rangle, 0 \leq t \leq 1\) के साथ लाइन
  इंटीग्रल की गणना करें।
\item
  अदिश और सदिश रेखा समाकलन के बीच अंतर स्पष्ट करें।
\end{enumerate}

\subsection{10.3 सरफेस
इंटीग्रल्स}\label{ux938ux930ux92bux938-ux907ux91fux917ux930ux932ux938}

एक सतह इंटीग्रल त्रि-आयामी अंतरिक्ष में दो-आयामी सतहों के लिए लाइन इंटीग्रल को
सामान्यीकृत करता है। वे हमें सतहों के माध्यम से प्रवाह और घुमावदार सतहों पर अदिश
क्षेत्रों के संचय की गणना करने की अनुमति देते हैं।

\subsubsection{अदिश सतह
इंटीग्रल}\label{ux905ux926ux936-ux938ux924ux939-ux907ux91fux917ux930ux932}

यदि एक सतह \(S\) द्वारा पैरामीटरयुक्त है

\[
\mathbf{r}(u,v) = \langle x(u,v), y(u,v), z(u,v) \rangle,
\]

तो एक अदिश फलन \(f(x,y,z)\) का सतही समाकलन है

\[
\iint_S f(x,y,z)\, dS = \iint_D f(\mathbf{r}(u,v)) \, |\mathbf{r}_u \times \mathbf{r}_v| \, du\,dv,
\]

जहां \(\mathbf{r}_u\) और \(\mathbf{r}_v\) \(\mathbf{r}(u,v)\) के आंशिक
व्युत्पन्न हैं, और \(D\) पैरामीटर डोमेन है।

\subsubsection{वेक्टर सरफेस इंटीग्रल
(फ्लक्स)}\label{ux935ux915ux91fux930-ux938ux930ux92bux938-ux907ux91fux917ux930ux932-ux92bux932ux915ux938}

एक सदिश क्षेत्र \(\mathbf{F}(x,y,z)\) के लिए, सतह \(S\) के माध्यम से प्रवाह है

\[
\iint_S \mathbf{F}\cdot d\mathbf{S} = \iint_S \mathbf{F}\cdot \mathbf{n}\, dS,
\]

जहां \(\mathbf{n}\) इकाई सामान्य वेक्टर है। पैरामीटरीकरण का उपयोग करना,

\[
\iint_S \mathbf{F}\cdot d\mathbf{S} = \iint_D \mathbf{F}(\mathbf{r}(u,v)) \cdot (\mathbf{r}_u \times \mathbf{r}_v)\,du\,dv.
\]

\subsubsection{उदाहरण}\label{ux909ux926ux939ux930ux923-23}

\begin{enumerate}
\def\labelenumi{\arabic{enumi}.}
\tightlist
\item
  अदिश सतह इंटीग्रल सतह: यूनिट डिस्क \(x^2+y^2 \leq 1\) के ऊपर समतल \(z=1\)।
\end{enumerate}

\[
\iint_S 1\, dS = \text{area of the disk} = \pi.
\]

\begin{enumerate}
\def\labelenumi{\arabic{enumi}.}
\setcounter{enumi}{1}
\tightlist
\item
  एक गोले के माध्यम से प्रवाह माना
  \(\mathbf{F}(x,y,z) = \langle x,y,z \rangle\), और \(S\) = त्रिज्या \(R\)
  का गोला। सामान्य वेक्टर \(\mathbf{n} = \frac{1}{R}\langle x,y,z \rangle\)
  है।
\end{enumerate}

\[
\mathbf{F}\cdot \mathbf{n} = \frac{x^2+y^2+z^2}{R} = R.
\]

तो

\[
\iint_S \mathbf{F}\cdot d\mathbf{S} = \iint_S R\, dS = R \cdot 4\pi R^2 = 4\pi R^3.
\]

\subsubsection{यह क्यों मायने रखता
है}\label{ux92fux939-ux915ux92f-ux92eux92fux928-ux930ux916ux924-ux939-19}

\begin{itemize}
\tightlist
\item
  स्केलर सतह इंटीग्रल्स क्षेत्र और सतह वितरण को मापते हैं।
\item
  वेक्टर सतह इंटीग्रल्स फ्लक्स को मापते हैं: सतह से गुजरने वाले क्षेत्र की मात्रा।
\item
  अनुप्रयोग: विद्युत चुंबकत्व, द्रव प्रवाह, गर्मी हस्तांतरण, और बहुत कुछ।
\end{itemize}

\subsubsection{व्यायाम}\label{ux935ux92fux92fux92e-37}

\begin{enumerate}
\def\labelenumi{\arabic{enumi}.}
\tightlist
\item
  2 भुजा की लंबाई वाले घन की सतह के लिए \(\iint_S 1\, dS\) की गणना करें।2.
  इकाई गोले के माध्यम से \(\mathbf{F}(x,y,z) = \langle x,y,z \rangle\) का
  प्रवाह ज्ञात कीजिए।
\item
  पैराबोलॉइड \(z = x^2+y^2, \, z \leq 1\) के लिए \(\iint_S z\, dS\) का
  मूल्यांकन करें।
\item
  \(\mathbf{F}(x,y,z) = \langle y,0,0 \rangle\) के लिए, समतल \(x=1\),
  \(0 \leq y,z \leq 1\) के माध्यम से फ्लक्स की गणना करें।
\item
  भौतिक रूप से समझाएं कि यदि एक बंद सतह के माध्यम से एक वेक्टर क्षेत्र का प्रवाह शून्य
  है तो इसका क्या मतलब है।
\end{enumerate}

\subsection{10.4 ग्रीन का
प्रमेय}\label{ux917ux930ux928-ux915-ux92aux930ux92eux92f}

ग्रीन का प्रमेय वेक्टर कैलकुलस में एक मौलिक परिणाम है जो एक बंद वक्र के चारों ओर एक
लाइन इंटीग्रल को उस क्षेत्र पर एक डबल इंटीग्रल से जोड़ता है जो इसे घेरता है। यह स्टोक्स
प्रमेय का द्वि-आयामी संस्करण है।

\subsubsection{ग्रीन प्रमेय का
कथन}\label{ux917ux930ux928-ux92aux930ux92eux92f-ux915-ux915ux925ux928}

मान लीजिए \(C\) समतल में एक सकारात्मक रूप से उन्मुख, सरल, बंद वक्र है, और मान
लीजिए कि \(R\) वह क्षेत्र है जो इसे घेरता है। यदि
\(\mathbf{F}(x,y) = \langle P(x,y), Q(x,y) \rangle\) में \(R\) वाले खुले क्षेत्र
पर निरंतर आंशिक व्युत्पन्न हैं, तो

\[
\oint_C \mathbf{F} \cdot d\mathbf{r} = \oint_C P\,dx + Q\,dy = \iint_R \left( \frac{\partial Q}{\partial x} - \frac{\partial P}{\partial y} \right)\, dA.
\]

\subsubsection{व्याख्या}\label{ux935ux92fux916ux92f-2}

\begin{itemize}
\tightlist
\item
  \(C\) के चारों ओर अभिन्न रेखा सीमा के साथ वेक्टर क्षेत्र के परिसंचरण को मापती है।
\item
  \(R\) पर दोहरा इंटीग्रल क्षेत्र के अंदर फ़ील्ड के कुल कर्ल (रोटेशन) को मापता है।
\end{itemize}

\subsubsection{उदाहरण 1: क्षेत्रफल
सूत्र}\label{ux909ux926ux939ux930ux923-1-ux915ux937ux924ux930ux92bux932-ux938ux924ux930}

यदि \(\mathbf{F} = \langle -y/2, x/2 \rangle\), तो

\[
\frac{\partial Q}{\partial x} - \frac{\partial P}{\partial y} = 1.
\]

इस प्रकार, ग्रीन का प्रमेय देता है

\[
\text{Area}(R) = \iint_R 1\,dA = \oint_C \left(-\tfrac{y}{2}\,dx + \tfrac{x}{2}\,dy\right).
\]

यह लाइन इंटीग्रल का उपयोग करके क्षेत्र की गणना करने का एक तरीका प्रदान करता है।

\subsubsection{उदाहरण 2:
सर्कुलेशन}\label{ux909ux926ux939ux930ux923-2-ux938ux930ux915ux932ux936ux928}

मान लीजिए \(\mathbf{F}(x,y) = \langle -y, x \rangle\), और \(C\) इकाई वृत्त
हैं।

\begin{itemize}
\tightlist
\item
  \(P=-y, Q=x\).
\item
  \(Q_x - P_y = 1 - (-1) = 2\).
\item
  यूनिट डिस्क पर दोहरा इंटीग्रल:
\end{itemize}

\[
\iint_R 2\,dA = 2\pi (1^2) = 2\pi.
\]

तो वृत्त के चारों ओर परिसंचरण \(2\pi\) है।

\subsubsection{यह क्यों मायने रखता
है}\label{ux92fux939-ux915ux92f-ux92eux92fux928-ux930ux916ux924-ux939-20}

\begin{itemize}
\tightlist
\item
  कठिन लाइन इंटीग्रल्स को डबल इंटीग्रल्स में परिवर्तित करता है, या इसके विपरीत।
\item
  स्थानीय संपत्तियों (कर्ल) और वैश्विक संपत्तियों (परिसंचरण) के बीच एक पुल प्रदान करता
  है।
\item
  द्रव प्रवाह, विद्युत चुंबकत्व और समतल वेक्टर क्षेत्रों के लिए भौतिकी में व्यापक रूप से
  उपयोग किया जाता है।
\end{itemize}

\subsubsection{व्यायाम}\label{ux935ux92fux92fux92e-38}

\begin{enumerate}
\def\labelenumi{\arabic{enumi}.}
\tightlist
\item
  दीर्घवृत्त \(\frac{x^2}{a^2} + \frac{y^2}{b^2} = 1\) के अंदर के क्षेत्र की गणना
  करने के लिए ग्रीन प्रमेय का उपयोग करें।
\item
  शीर्ष (0,0), (1,0), (1,1), (0,1) वाले वर्ग के साथ
  \(\mathbf{F}(x,y) = \langle -y, x \rangle\) के लिए ग्रीन के प्रमेय को
  सत्यापित करें।
\item
  यूनिट सर्कल के चारों ओर \(\mathbf{F}(x,y) = \langle -y, x \rangle\) के
  परिसंचरण की गणना करें।4. दिखाएँ कि यदि \(\nabla \times \mathbf{F} = 0\),
  तो किसी भी बंद वक्र के चारों ओर \(\mathbf{F}\) का रेखा समाकलन शून्य है।
\item
  इसे दर्शाने के लिए ग्रीन प्रमेय का प्रयोग करें
\end{enumerate}

\[
\oint_C x\,dy = -\oint_C y\,dx
\]

किसी भी बंद वक्र के लिए \(C\)।

\subsection{10.5 स्टोक्स
प्रमेय}\label{ux938ux91fux915ux938-ux92aux930ux92eux92f}

स्टोक्स का प्रमेय ग्रीन के प्रमेय को तीन आयामों में सामान्यीकृत करता है। यह किसी सतह
पर एक सदिश क्षेत्र के कर्ल के सतह अभिन्न अंग को उस सतह की सीमा के चारों ओर सदिश
क्षेत्र के एक रेखा अभिन्न अंग से जोड़ता है।

\subsubsection{स्टोक्स प्रमेय का
कथन}\label{ux938ux91fux915ux938-ux92aux930ux92eux92f-ux915-ux915ux925ux928}

मान लीजिए \(S\) सीमा वक्र \(C\) (सकारात्मक रूप से उन्मुख) के साथ एक उन्मुख, चिकनी
सतह है। यदि \(\mathbf{F}(x,y,z)\) निरंतर आंशिक व्युत्पन्न वाला एक सदिश क्षेत्र है,
तो

\[
\iint_S (\nabla \times \mathbf{F}) \cdot d\mathbf{S} = \oint_C \mathbf{F} \cdot d\mathbf{r}.
\]

\begin{itemize}
\tightlist
\item
  बाईं ओर: सतह के माध्यम से \(\mathbf{F}\) के कर्ल का प्रवाह।
\item
  दाईं ओर: सीमा वक्र के साथ \(\mathbf{F}\) का परिसंचरण।
\end{itemize}

\subsubsection{व्याख्या}\label{ux935ux92fux916ux92f-3}

\begin{itemize}
\tightlist
\item
  सीमा के चारों ओर अभिन्न रेखा सतह के अंदर कुल ``घूर्णन'' के बराबर होती है।
\item
  ग्रीन के प्रमेय का विस्तार करता है (एक विशेष मामला जब सतह समतल में होती है)।
\end{itemize}

\subsubsection{उदाहरण 1: एक विशेष मामले के रूप में ग्रीन का
प्रमेय}\label{ux909ux926ux939ux930ux923-1-ux90fux915-ux935ux936ux937-ux92eux92eux932-ux915-ux930ux92a-ux92e-ux917ux930ux928-ux915-ux92aux930ux92eux92f}

यदि \(S\) \(xy\)-तल में एक समतल क्षेत्र है, तो स्टोक्स प्रमेय ग्रीन के प्रमेय में बदल
जाता है।

\subsubsection{उदाहरण 2: गोलार्ध पर
परिसंचरण}\label{ux909ux926ux939ux930ux923-2-ux917ux932ux930ux927-ux92aux930-ux92aux930ux938ux91aux930ux923}

मान लीजिए \(\mathbf{F}(x,y,z) = \langle -y, x, 0 \rangle\), और \(S\)
त्रिज्या 1 का ऊपरी गोलार्ध है।

\begin{itemize}
\tightlist
\item
  सीमा \(C\): \(xy\)-तल में इकाई वृत्त।
\item
  स्टोक्स प्रमेय द्वारा:
\end{itemize}

\[
\oint_C \mathbf{F}\cdot d\mathbf{r} = \iint_S (\nabla \times \mathbf{F})\cdot d\mathbf{S}.
\]

\begin{itemize}
\tightlist
\item
  कर्ल: \(\nabla \times \mathbf{F} = \langle 0,0,2 \rangle\).
\item
  गोलार्ध के सामान्य बिंदु बाहर की ओर:
  \(\mathbf{n} = \langle 0,0,1 \rangle\)।
\item
  अतः अभिन्न = 2.
\item
  गोलार्ध का क्षेत्रफल = \(2\pi (1^2)\).
\end{itemize}

\[
\iint_S 2\, dS = 2 \cdot 2\pi = 4\pi.
\]

इस प्रकार, भूमध्य रेखा के चारों ओर परिसंचरण \(4\pi\) है।

\subsubsection{यह क्यों मायने रखता
है}\label{ux92fux939-ux915ux92f-ux92eux92fux928-ux930ux916ux924-ux939-21}

\begin{itemize}
\tightlist
\item
  सतह इंटीग्रल्स और लाइन इंटीग्रल्स के बीच गहरा संबंध प्रदान करता है।
\item
  सुविधाजनक सतहों के चयन की अनुमति देकर गणना को सरल बनाता है।
\item
  विद्युत चुंबकत्व (फैराडे का नियम) और द्रव गतिकी में व्यापक रूप से उपयोग किया जाता
  है।
\end{itemize}

\subsubsection{व्यायाम}\label{ux935ux92fux92fux92e-39}

\begin{enumerate}
\def\labelenumi{\arabic{enumi}.}
\tightlist
\item
  \(xy\)-प्लेन में यूनिट डिस्क पर
  \(\mathbf{F}(x,y,z) = \langle y, -x, 0 \rangle\) के लिए स्टोक्स प्रमेय को
  सत्यापित करें।
\item
  \(\oint_C \mathbf{F}\cdot d\mathbf{r}\) की गणना करें जहां
  \(\mathbf{F}(x,y,z) = \langle z, 0, x \rangle\), और \(C\) शीर्ष
  (0,0,0), (1,0,0), (0,1,0) वाले त्रिभुज की सीमा है।
\item
  दिखाएँ कि यदि \(\nabla \times \mathbf{F} = 0\) है, तो किसी भी बंद वक्र के
  चारों ओर परिसंचरण शून्य है।4. समतल \(z=0\) में इकाई वर्ग की सीमा के आसपास
  \(\mathbf{F}(x,y,z) = \langle -y, x, z \rangle\) के परिसंचरण की गणना करने
  के लिए स्टोक्स प्रमेय लागू करें।
\item
  बताएं कि स्टोक्स का प्रमेय ग्रीन के प्रमेय को कैसे सामान्य बनाता है।
\end{enumerate}

\subsection{10.6 विचलन
प्रमेय}\label{ux935ux91aux932ux928-ux92aux930ux92eux92f}

विचलन प्रमेय (जिसे गॉस का प्रमेय भी कहा जाता है) एक बंद सतह के माध्यम से एक वेक्टर
क्षेत्र के प्रवाह को सतह के अंदर क्षेत्र के विचलन के ट्रिपल इंटीग्रल से जोड़ता है।

\subsubsection{विचलन प्रमेय का
कथन}\label{ux935ux91aux932ux928-ux92aux930ux92eux92f-ux915-ux915ux925ux928}

मान लीजिए \(E\) \(\mathbb{R}^3\) में एक ठोस क्षेत्र है जिसकी सीमा सतह \(S\)
(बाहर की ओर उन्मुख) है। यदि \(\mathbf{F}(x,y,z)\) \(E\) पर निरंतर आंशिक
व्युत्पन्न वाला एक सदिश क्षेत्र है, तो

\[
\iint_S \mathbf{F} \cdot d\mathbf{S} = \iiint_E (\nabla \cdot \mathbf{F}) \, dV.
\]

\begin{itemize}
\tightlist
\item
  बाईं ओर: बंद सतह \(S\) पर \(\mathbf{F}\) का प्रवाह।
\item
  दाईं ओर: क्षेत्र के अंदर विचलन का ट्रिपल इंटीग्रल।
\end{itemize}

\subsubsection{विचलन}\label{ux935ux91aux932ux928}

एक सदिश क्षेत्र का विचलन \(\mathbf{F}(x,y,z) = \langle P, Q, R \rangle\) है

\[
\nabla \cdot \mathbf{F} = \frac{\partial P}{\partial x} + \frac{\partial Q}{\partial y} + \frac{\partial R}{\partial z}.
\]

यह प्रत्येक बिंदु पर प्रति इकाई आयतन ``शुद्ध बहिर्वाह'' को मापता है।

\subsubsection{उदाहरण 1: रेडियल क्षेत्र का
फ्लक्स}\label{ux909ux926ux939ux930ux923-1-ux930ux921ux92fux932-ux915ux937ux924ux930-ux915-ux92bux932ux915ux938}

मान लीजिए \(\mathbf{F}(x,y,z) = \langle x, y, z \rangle\), और मान लीजिए
\(E\) यूनिट बॉल \(x^2+y^2+z^2 \leq 1\) है।

\begin{itemize}
\tightlist
\item
  विचलन: \(\nabla \cdot \mathbf{F} = 1+1+1 = 3\)।
\item
  यूनिट बॉल का आयतन: \(\tfrac{4}{3}\pi\)। तो
\end{itemize}

\[
\iiint_E (\nabla \cdot \mathbf{F})\, dV = 3 \cdot \tfrac{4}{3}\pi = 4\pi.
\]

इस प्रकार, इकाई क्षेत्र में प्रवाह \(4\pi\) है।

\subsubsection{उदाहरण 2: स्थिर
क्षेत्र}\label{ux909ux926ux939ux930ux923-2-ux938ux925ux930-ux915ux937ux924ux930}

चलो \(\mathbf{F}(x,y,z) = \langle 1, 0, 0 \rangle\)।

\begin{itemize}
\tightlist
\item
  विचलन: \(\nabla \cdot \mathbf{F} = 0\)।
\item
  तो किसी भी बंद सतह के माध्यम से प्रवाह शून्य है, अंतर्ज्ञान के अनुरूप है (कोई शुद्ध
  बहिर्वाह नहीं)।
\end{itemize}

\subsubsection{यह क्यों मायने रखता
है}\label{ux92fux939-ux915ux92f-ux92eux92fux928-ux930ux916ux924-ux939-22}

\begin{itemize}
\item
  सतह इंटीग्रल्स को सरल वॉल्यूम इंटीग्रल्स में परिवर्तित करता है।
\item
  भौतिकी में प्रयुक्त: विद्युत चुंबकत्व, द्रव प्रवाह और गर्मी हस्तांतरण में गॉस का नियम।
\item
  एकीकृत रूपरेखा को पूरा करता है:

  \begin{itemize}
  \tightlist
  \item
    ग्रीन का प्रमेय (2डी कर्ल ↔ परिसंचरण)
  \item
    स्टोक्स प्रमेय (3डी कर्ल ↔ सतहों पर परिसंचरण)
  \item
    विचलन प्रमेय (3डी विचलन ↔ बंद सतहों पर प्रवाह)
  \end{itemize}
\end{itemize}

\subsubsection{व्यायाम}\label{ux935ux92fux92fux92e-40}

\begin{enumerate}
\def\labelenumi{\arabic{enumi}.}
\tightlist
\item
  त्रिज्या \(R\) के एक गोले की सतह पर
  \(\mathbf{F}(x,y,z) = \langle x,y,z \rangle\) के प्रवाह की गणना करने के
  लिए विचलन प्रमेय का उपयोग करें।
\item
  इकाई घन \([0,1]^3\) पर \(\mathbf{F}(x,y,z) = \langle y, z, x \rangle\)
  के लिए विचलन प्रमेय को सत्यापित करें।
\item
  दिखाएँ कि यदि \(\nabla \cdot \mathbf{F} = 0\) है, तो किसी भी बंद सतह से कुल
  प्रवाह शून्य है।
\item
  इकाई गोले के माध्यम से
  \(\mathbf{F}(x,y,z) = \langle x^2, y^2, z^2 \rangle\) के प्रवाह की गणना
  करें।5. बताएं कि डायवर्जेंस प्रमेय कैलकुलस के एक-आयामी मौलिक प्रमेय को कैसे सामान्यीकृत
  करता है।
\end{enumerate}

\section{भाग IV. अनंत
प्रक्रियाएँ}\label{ux92dux917-iv.-ux905ux928ux924-ux92aux930ux915ux930ux92fux90f}

\section{अध्याय 11. अनुक्रम और
अभिसरण}\label{ux905ux927ux92fux92f-11.-ux905ux928ux915ux930ux92e-ux914ux930-ux905ux92dux938ux930ux923}

\subsection{11.1 परिभाषाएँ और
उदाहरण}\label{ux92aux930ux92dux937ux90f-ux914ux930-ux909ux926ux939ux930ux923}

अनुक्रम संख्याओं की एक क्रमबद्ध सूची है, जिसे आमतौर पर इस प्रकार लिखा जाता है

\[
a_1, a_2, a_3, \dots
\]

या अधिक सामान्यतः \((a_n)_{n=1}^\infty\). प्रत्येक \(a_n\) को अनुक्रम का nवाँ
पद कहा जाता है।

\subsubsection{अनुक्रम को परिभाषित
करना}\label{ux905ux928ux915ux930ux92e-ux915-ux92aux930ux92dux937ux924-ux915ux930ux928}

किसी अनुक्रम को दो प्रकार से परिभाषित किया जा सकता है:

\begin{enumerate}
\def\labelenumi{\arabic{enumi}.}
\item
  स्पष्ट सूत्र - nवें पद के लिए एक सीधा नियम देता है।

  \begin{itemize}
  \item
    उदाहरण: \(a_n = \frac{1}{n}\) अनुक्रम को परिभाषित करता है

    \[
    1, \tfrac{1}{2}, \tfrac{1}{3}, \tfrac{1}{4}, \dots
    \]
  \end{itemize}
\item
  पुनरावर्ती परिभाषा - पहले के शब्दों का उपयोग करके शब्दों को परिभाषित करती है।

  \begin{itemize}
  \item
    उदाहरण: फाइबोनैचि अनुक्रम:

    \[
    a_1 = 1, \quad a_2 = 1, \quad a_{n} = a_{n-1} + a_{n-2} \quad (n \geq 3).
    \]
  \end{itemize}
\end{enumerate}

\subsubsection{अनुक्रमों के
उदाहरण}\label{ux905ux928ux915ux930ux92e-ux915-ux909ux926ux939ux930ux923}

\begin{enumerate}
\def\labelenumi{\arabic{enumi}.}
\item
  अंकगणित अनुक्रम:

  \[
  a_n = a_1 + (n-1)d.
  \]

  उदाहरण: \(a_n = 2n+1\) → विषम संख्याओं का क्रम।
\item
  ज्यामितीय अनुक्रम:

  \[
  a_n = a_1 r^{n-1}.
  \]

  उदाहरण: \(a_n = 2^n\) → 2 की घातें।
\item
  हार्मोनिक अनुक्रम:

  \[
  a_n = \frac{1}{n}.
  \]
\item
  वैकल्पिक क्रम:

  \[
  a_n = (-1)^n.
  \]
\end{enumerate}

\subsubsection{कैलकुलस में
अनुक्रम}\label{ux915ux932ux915ux932ux938-ux92e-ux905ux928ux915ux930ux92e}

अनुक्रम अनंत प्रक्रियाओं की नींव हैं:

\begin{itemize}
\tightlist
\item
  अनुक्रमों की सीमाएँ → अभिसरण को परिभाषित करें।
\item
  श्रृंखला → अनुक्रमों से निर्मित अनंत राशियाँ।
\item
  अनुक्रम और श्रृंखला द्वारा अनुमानित कार्य।
\end{itemize}

\subsubsection{यह क्यों मायने रखता
है}\label{ux92fux939-ux915ux92f-ux92eux92fux928-ux930ux916ux924-ux939-23}

\begin{itemize}
\tightlist
\item
  अनुक्रम अनंत श्रृंखला और सन्निकटन के लिए बिल्डिंग ब्लॉक प्रदान करते हैं।
\item
  वे हमें ``निकट अनंतता'' और अभिसरण को कठोरता से परिभाषित करने की अनुमति देते हैं।
\item
  कई महत्वपूर्ण कार्यों (घातीय, त्रिकोणमितीय) को अनुक्रमों और श्रृंखला के माध्यम से
  व्यक्त किया जा सकता है।
\end{itemize}

\subsubsection{व्यायाम}\label{ux935ux92fux92fux92e-41}

\begin{enumerate}
\def\labelenumi{\arabic{enumi}.}
\tightlist
\item
  अनुक्रम \(a_n = \frac{n}{n+1}\) के पहले पांच पद लिखें।
\item
  निर्धारित करें कि क्या \(a_n = (-1)^n n\) परिबद्ध है।
\item
  अनुक्रम \(2,4,8,16,\dots\) के लिए एक पुनरावर्ती परिभाषा दीजिए।
\item
  \(a_1=3\) और \(d=5\) के साथ अंकगणितीय अनुक्रम का 10वां पद ज्ञात कीजिए।
\item
  \(a_1=1\), \(a_{n+1}=2a_n\) द्वारा परिभाषित अनुक्रम के लिए एक स्पष्ट सूत्र
  लिखें।
\end{enumerate}

\subsection{11.2 मोनोटोन और बंधे हुए
अनुक्रम}\label{ux92eux928ux91fux928-ux914ux930-ux92cux927-ux939ux90f-ux905ux928ux915ux930ux92e}

यह समझने के लिए कि क्या कोई अनुक्रम अभिसरण करता है, हमें उसके व्यवहार का अध्ययन करने
की आवश्यकता है: क्या यह बढ़ता है, घटता है, सीमा के भीतर रहता है, या बिना सीमा के
बढ़ता है? दो महत्वपूर्ण अवधारणाएँ एकरसता और सीमाबद्धता हैं।

\subsubsection{मोनोटोन
अनुक्रम}\label{ux92eux928ux91fux928-ux905ux928ux915ux930ux92e}

एक अनुक्रम \((a_n)\) को मोनोटोन कहा जाता है यदि यह हमेशा बढ़ रहा है या हमेशा घट
रहा है।

\begin{itemize}
\item
  मोनोटोन बढ़ रहा है:

  \[
  a_{n+1} \geq a_n \quad \text{for all } n.
  \]
\item
  नीरसता कम हो रही है:

  \[
  a_{n+1} \leq a_n \quad \text{for all } n.
  \]
\end{itemize}

उदाहरण:1. \(a_n = n\) एकस्वर बढ़ रहा है। 2. \(a_n = \frac{1}{n}\) एकरसता कम
हो रही है।

\subsubsection{बंधे हुए
अनुक्रम}\label{ux92cux927-ux939ux90f-ux905ux928ux915ux930ux92e}

यदि कोई संख्या \(M\) मौजूद है, जैसे कि सभी \(n\) के लिए \(a_n \leq M\) तो ऊपर
एक अनुक्रम परिबद्ध है। यदि \(m\) मौजूद है तो इसे नीचे परिबद्ध किया गया है, ताकि
सभी \(n\) के लिए \(a_n \geq m\) हो।

यदि दोनों स्थितियाँ लागू होती हैं, तो अनुक्रम सीमित हो जाता है।

उदाहरण:

\begin{enumerate}
\def\labelenumi{\arabic{enumi}.}
\tightlist
\item
  \(a_n = \frac{1}{n}\) 0 और 1 के बीच परिबद्ध है।
\item
  \(a_n = (-1)^n\) -1 और 1 के बीच परिबद्ध है।
\item
  \(a_n = n\) परिबद्ध नहीं है।
\end{enumerate}

\subsubsection{मोनोटोन अभिसरण
प्रमेय}\label{ux92eux928ux91fux928-ux905ux92dux938ux930ux923-ux92aux930ux92eux92f}

विश्लेषण में एक मौलिक परिणाम:

\begin{itemize}
\tightlist
\item
  प्रत्येक मोनोटोन बढ़ता क्रम जो ऊपर से घिरा हुआ है, अभिसरण करता है।
\item
  प्रत्येक मोनोटोन घटता क्रम जो नीचे से घिरा हुआ है, अभिसरण करता है।
\end{itemize}

यह प्रमेय स्पष्ट रूप से सीमा ज्ञात किए बिना अभिसरण की गारंटी देता है।

\subsubsection{उदाहरण}\label{ux909ux926ux939ux930ux923-24}

\begin{enumerate}
\def\labelenumi{\arabic{enumi}.}
\item
  अनुक्रम: \(a_n = 1 - \frac{1}{n}\)।

  \begin{itemize}
  \tightlist
  \item
    बढ़ रहा है: \(a_{n+1} - a_n = \frac{1}{n} - \frac{1}{n+1} > 0\) से।
  \item
    ऊपर 1 से घिरा हुआ।
  \item
    इसलिए, यह अभिसरण करता है।
  \item
    सीमा: \(\lim_{n\to\infty} a_n = 1\).
  \end{itemize}
\end{enumerate}

\subsubsection{यह क्यों मायने रखता
है}\label{ux92fux939-ux915ux92f-ux92eux92fux928-ux930ux916ux924-ux939-24}

\begin{itemize}
\tightlist
\item
  एकरसता और सीमाबद्धता अभिसरण के लिए त्वरित परीक्षण देती है।
\item
  वे प्रमाणों में और कठोरता से सीमाएँ बनाने में आवश्यक हैं।
\item
  ये विचार स्वाभाविक रूप से कार्यों और श्रृंखलाओं तक विस्तारित होते हैं।
\end{itemize}

\subsubsection{व्यायाम}\label{ux935ux92fux92fux92e-42}

\begin{enumerate}
\def\labelenumi{\arabic{enumi}.}
\tightlist
\item
  निर्धारित करें कि क्या \(a_n = \frac{n}{n+1}\) एकस्वर और परिबद्ध है।
\item
  दिखाएँ कि \(a_n = \sqrt{n}\) मोनोटोन बढ़ रहा है लेकिन सीमित नहीं है।
\item
  सिद्ध करें कि \(a_n = 2 - \frac{1}{n}\) अभिसरण करता है, और इसकी सीमा ज्ञात
  करें।
\item
  एक बंधे हुए अनुक्रम का उदाहरण दीजिए जो एकसार नहीं है।
\item
  मोनोटोन अभिसरण प्रमेय को \(a_n = \ln\!\big(1+\frac{1}{n}\big)\) पर लागू
  करें।
\end{enumerate}

\subsection{11.3 अनुक्रमों की
सीमाएँ}\label{ux905ux928ux915ux930ux92e-ux915-ux938ux92eux90f}

किसी अनुक्रम के बारे में केंद्रीय प्रश्न यह है कि क्या \(n\) बढ़ने पर इसके पद एकल मान तक
पहुंचते हैं। इससे अनुक्रम की सीमा की अवधारणा सामने आती है।

\subsubsection{परिभाषा}\label{ux92aux930ux92dux937-12}

एक अनुक्रम \((a_n)\) की एक सीमा \(L\) है, यदि, प्रत्येक \(\varepsilon > 0\) के
लिए, एक पूर्णांक \(N\) मौजूद है जैसे कि

\[
|a_n - L| < \varepsilon \quad \text{whenever } n > N.
\]

फिर हम लिखते हैं

\[
\lim_{n\to\infty} a_n = L.
\]

यदि ऐसा कोई \(L\) मौजूद नहीं है, तो अनुक्रम अलग हो जाता है।

\subsubsection{अंतर्ज्ञान}\label{ux905ux924ux930ux91cux91eux928}

\begin{itemize}
\tightlist
\item
  अनुक्रम की शर्तें मनमाने ढंग से \(L\) के करीब हो जाती हैं क्योंकि \(n\) बड़ी हो जाती
  है।
\item
  कुछ सूचकांक \(N\) से परे, सभी पद \(L\) के आसपास एक छोटे बैंड के भीतर रहते हैं।
\end{itemize}

\subsubsection{उदाहरण}\label{ux909ux926ux939ux930ux923-25}

\begin{enumerate}
\def\labelenumi{\arabic{enumi}.}
\item
  \(a_n = \frac{1}{n}\). जैसे-जैसे \(n\) बढ़ता है, पद 0 की ओर सिकुड़ते हैं।

  \[
  \lim_{n\to\infty} \frac{1}{n} = 0.
  \]
\item
  \(a_n = (-1)^n\). पद -1 और 1 के बीच वैकल्पिक होते हैं, इसलिए कोई एकल सीमा
  मौजूद नहीं है। क्रम अलग हो जाता है।
\item
  \(a_n = \frac{n}{n+1}\). चूंकि \(n \to \infty\), अंश और हर लगभग बराबर हैं,
  इसलिए

  \[
  \lim_{n\to\infty} \frac{n}{n+1} = 1.
  \]
\end{enumerate}

\subsubsection{\texorpdfstring{सीमाओं के गुणयदि \(\lim a_n = A\) और
\(\lim b_n = B\):}{सीमाओं के गुणयदि \textbackslash lim a\_n = A और \textbackslash lim b\_n = B:}}\label{ux938ux92eux913-ux915-ux917ux923ux92fux926-lim-a_n-a-ux914ux930-lim-b_n-b}

\begin{itemize}
\item
  \(\lim (a_n+b_n) = A+B\).
\item
  \(\lim (a_n b_n) = AB\).
\item
  \(\lim (c a_n) = cA\) स्थिरांक \(c\) के लिए।
\item
  यदि \(b_n \neq 0\) और \(B \neq 0\), तो

  \[
  \lim \frac{a_n}{b_n} = \frac{A}{B}.
  \]
\end{itemize}

\subsubsection{प्रमेय: निचोड़
सिद्धांत}\label{ux92aux930ux92eux92f-ux928ux91aux921-ux938ux926ux927ux924}

यदि \(a_n \leq b_n \leq c_n\) सभी बड़े \(n\) के लिए, और

\[
\lim_{n\to\infty} a_n = \lim_{n\to\infty} c_n = L,
\]

फिर

\[
\lim_{n\to\infty} b_n = L.
\]

उदाहरण:

\[
a_n = -\tfrac{1}{n}, \quad b_n = \tfrac{\sin n}{n}, \quad c_n = \tfrac{1}{n}.
\]

चूँकि \(-\tfrac{1}{n} \leq \tfrac{\sin n}{n} \leq \tfrac{1}{n}\) और दोनों
बाउंडिंग अनुक्रम 0 पर जाते हैं,

\[
\lim_{n\to\infty} \frac{\sin n}{n} = 0.
\]

\subsubsection{यह क्यों मायने रखता
है}\label{ux92fux939-ux915ux92f-ux92eux92fux928-ux930ux916ux924-ux939-25}

\begin{itemize}
\tightlist
\item
  सीमाएं किसी मूल्य के ``आने'' वाले अनुक्रमों के विचार को कठोर बनाती हैं।
\item
  अनुक्रमों का अभिसरण अनंत श्रृंखला और निरंतरता को रेखांकित करता है।
\item
  सीमा के माध्यम से वास्तविक संख्याओं को परिभाषित करने में ये अवधारणाएँ आवश्यक हैं।
\end{itemize}

\subsubsection{व्यायाम}\label{ux935ux92fux92fux92e-43}

\begin{enumerate}
\def\labelenumi{\arabic{enumi}.}
\tightlist
\item
  \(\lim_{n\to\infty} \frac{2n+1}{3n+4}\) खोजें।
\item
  निर्धारित करें कि क्या \(a_n = \sqrt{n+1} - \sqrt{n}\) अभिसरण करता है।
\item
  क्या \(a_n = \cos n\) अभिसरण होता है? क्यों या क्यों नहीं?
\item
  \(\lim_{n\to\infty} \frac{\sin n}{n} = 0\) दिखाने के लिए स्क्वीज़ सिद्धांत का
  उपयोग करें।
\item
  सिद्ध करें कि यदि \(\lim a_n = L\), तो \(\lim |a_n| = |L|\)।
\end{enumerate}

\#अध्याय 12. अनंत शृंखला

\subsection{12.1 श्रृंखला और
अभिसरण}\label{ux936ux930ux916ux932-ux914ux930-ux905ux92dux938ux930ux923}

एक श्रृंखला एक अनुक्रम के पदों का योग है। केवल संख्याओं को सूचीबद्ध करने के बजाय, हम उन्हें
एक साथ जोड़ते हैं और अध्ययन करते हैं कि क्या अनंत योग एक सीमित मूल्य के करीब पहुंचता है।

\subsubsection{परिभाषा}\label{ux92aux930ux92dux937-13}

एक अनुक्रम \((a_n)\) दिया गया है, संगत श्रृंखला है

\[
\sum_{n=1}^\infty a_n = a_1 + a_2 + a_3 + \dots
\]

हम nवें आंशिक योग को इस प्रकार परिभाषित करते हैं

\[
S_n = \sum_{k=1}^n a_k.
\]

यदि अनुक्रम \((S_n)\) एक सीमित सीमा \(S\) में परिवर्तित हो जाता है, तो श्रृंखला
परिवर्तित हो जाती है और

\[
\sum_{n=1}^\infty a_n = S.
\]

यदि \((S_n)\) विचलन करता है, तो श्रृंखला भिन्न हो जाती है।

\subsubsection{उदाहरण}\label{ux909ux926ux939ux930ux923-26}

\begin{enumerate}
\def\labelenumi{\arabic{enumi}.}
\tightlist
\item
  ज्यामितीय श्रृंखला
\end{enumerate}

\[
\sum_{n=0}^\infty ar^n = \frac{a}{1-r}, \quad |r| < 1.
\]

उदाहरण:

\[
1 + \tfrac{1}{2} + \tfrac{1}{4} + \tfrac{1}{8} + \dots = 2.
\]

\begin{enumerate}
\def\labelenumi{\arabic{enumi}.}
\setcounter{enumi}{1}
\tightlist
\item
  हार्मोनिक श्रृंखला
\end{enumerate}

\[
\sum_{n=1}^\infty \frac{1}{n}.
\]

यह श्रृंखला विचलन करती है, भले ही पद 0 पर जाते हैं।

\begin{enumerate}
\def\labelenumi{\arabic{enumi}.}
\setcounter{enumi}{2}
\tightlist
\item
  पी-श्रृंखला
\end{enumerate}

\[
\sum_{n=1}^\infty \frac{1}{n^p}.
\]

\begin{itemize}
\tightlist
\item
  यदि \(p > 1\) अभिसरण करता है।
\item
  यदि \(p \leq 1\) अलग हो जाता है।
\end{itemize}

\subsubsection{अभिसरण के लिए आवश्यक
शर्त}\label{ux905ux92dux938ux930ux923-ux915-ux932ux90f-ux906ux935ux936ux92fux915-ux936ux930ux924}

यदि \(\sum a_n\) अभिसरण होता है, तो आवश्यक रूप से

\[
\lim_{n\to\infty} a_n = 0.
\]

यदि \(\lim a_n \neq 0\), तो श्रृंखला अलग हो जाती है। लेकिन इसका विपरीत सत्य
नहीं है: \(\lim a_n = 0\) अभिसरण की गारंटी नहीं देता है (उदाहरण के लिए,
हार्मोनिक श्रृंखला)।

\subsubsection{यह क्यों मायने रखता
है}\label{ux92fux939-ux915ux92f-ux92eux92fux928-ux930ux916ux924-ux939-26}

\begin{itemize}
\tightlist
\item
  श्रृंखला अनंत प्रक्रियाओं तक सीमित जोड़ का विस्तार करती है।
\item
  अभिसरण श्रृंखला का उपयोग कार्यों का अनुमान लगाने, क्षेत्रों की गणना करने और भौतिक
  प्रक्रियाओं को मॉडल करने के लिए किया जाता है।- श्रृंखला के अध्ययन से शक्तिशाली
  अभिसरण परीक्षण प्राप्त होते हैं।
\end{itemize}

\subsubsection{व्यायाम}\label{ux935ux92fux92fux92e-44}

\begin{enumerate}
\def\labelenumi{\arabic{enumi}.}
\tightlist
\item
  निर्धारित करें कि क्या \(\sum_{n=1}^\infty \frac{2}{3^n}\) अभिसरण करता है,
  और इसका योग ज्ञात करें।
\item
  दिखाएँ कि \(\sum_{n=1}^\infty \frac{1}{n^2}\) अभिसरण करता है।
\item
  क्या \(\sum_{n=1}^\infty \frac{1}{\sqrt{n}}\) अभिसरण होता है?
\item
  श्रृंखला \(\sum_{n=1}^\infty \frac{1}{2^n}\) के पहले चार आंशिक योग लिखें।
\item
  बताएं कि क्यों \(\lim a_n = 0\) आवश्यक है लेकिन अभिसरण के लिए पर्याप्त नहीं है।
\end{enumerate}

\subsection{12.2 अभिसरण
परीक्षण}\label{ux905ux92dux938ux930ux923-ux92aux930ux915ux937ux923}

चूँकि कई श्रृंखलाओं को सीधे तौर पर सारांशित नहीं किया जा सकता है, गणितज्ञों ने यह तय
करने के लिए परीक्षण विकसित किए हैं कि कोई श्रृंखला अभिसरण करती है या विचलन करती
है। ये परीक्षण अनंत राशियों का विश्लेषण करने के उपकरण हैं।

\subsubsection{1. विचलन के लिए नौवें सत्र का
परीक्षण}\label{ux935ux91aux932ux928-ux915-ux932ux90f-ux928ux935-ux938ux924ux930-ux915-ux92aux930ux915ux937ux923}

यदि

\[
\lim_{n\to\infty} a_n \neq 0 \quad \text{or does not exist},
\]

फिर

\[
\sum a_n
\]

विचलन

यदि \(\lim a_n = 0\), तो परीक्षण अनिर्णायक है।

\subsubsection{2. तुलना
परीक्षण}\label{ux924ux932ux928-ux92aux930ux915ux937ux923}

मान लीजिए \(0 \leq a_n \leq b_n\) सभी \(n\) के लिए।

\begin{itemize}
\tightlist
\item
  यदि \(\sum b_n\) अभिसरण होता है, तो \(\sum a_n\) भी अभिसरण होता है।
\item
  यदि \(\sum a_n\) अलग हो जाता है, तो \(\sum b_n\) भी अलग हो जाता है।
\end{itemize}

\subsubsection{3. सीमा तुलना
परीक्षण}\label{ux938ux92e-ux924ux932ux928-ux92aux930ux915ux937ux923}

यदि \(a_n, b_n > 0\) और

\[
\lim_{n\to\infty} \frac{a_n}{b_n} = c,
\]

जहां \(0 < c < \infty\), फिर \(\sum a_n\) और \(\sum b_n\) या तो दोनों
अभिसरण करते हैं या दोनों अलग हो जाते हैं।

\subsubsection{4. अनुपात
परीक्षण}\label{ux905ux928ux92aux924-ux92aux930ux915ux937ux923}

\(\sum a_n\) के लिए, गणना करें

\[
L = \lim_{n\to\infty} \left| \frac{a_{n+1}}{a_n} \right|.
\]

\begin{itemize}
\tightlist
\item
  यदि \(L < 1\), तो श्रृंखला पूर्णतया अभिसरित हो जाती है।
\item
  यदि \(L > 1\) या \(L = \infty\), तो श्रृंखला अलग हो जाती है।
\item
  यदि \(L = 1\), तो परीक्षण अनिर्णायक है।
\end{itemize}

\subsubsection{5. जड़ परीक्षण}\label{ux91cux921-ux92aux930ux915ux937ux923}

\(\sum a_n\) के लिए, गणना करें

\[
L = \lim_{n\to\infty} \sqrt[n]{|a_n|}.
\]

\begin{itemize}
\tightlist
\item
  यदि \(L < 1\), तो श्रृंखला पूर्णतया अभिसरित हो जाती है।
\item
  यदि \(L > 1\), तो श्रृंखला अलग हो जाती है।
\item
  यदि \(L = 1\), तो परीक्षण अनिर्णायक है।
\end{itemize}

\subsubsection{6. वैकल्पिक श्रृंखला परीक्षण (लीबनिज़ का
परीक्षण)}\label{ux935ux915ux932ux92aux915-ux936ux930ux916ux932-ux92aux930ux915ux937ux923-ux932ux92cux928ux91c-ux915-ux92aux930ux915ux937ux923}

प्रपत्र की श्रृंखला के लिए

\[
\sum (-1)^n b_n \quad \text{or} \quad \sum (-1)^{n+1} b_n,
\]

यदि

\begin{enumerate}
\def\labelenumi{\arabic{enumi}.}
\tightlist
\item
  \(b_{n+1} \leq b_n\) (घटता हुआ), और
\item
  \(\lim_{n\to\infty} b_n = 0\),
\end{enumerate}

फिर शृंखला एकत्रित हो जाती है।

\subsubsection{उदाहरण}\label{ux909ux926ux939ux930ux923-27}

\begin{enumerate}
\def\labelenumi{\arabic{enumi}.}
\tightlist
\item
  \(\sum \frac{1}{n^2}\): तुलना परीक्षण → अभिसरण।
\item
  \(\sum \frac{1}{n}\): हार्मोनिक श्रृंखला → विचलन।
\item
  \(\sum \frac{(-1)^n}{n}\): वैकल्पिक श्रृंखला परीक्षण → अभिसरण।
\item
  \(\sum \frac{n!}{n^n}\): अनुपात परीक्षण → अभिसरण।
\item
  \(\sum \frac{2^n}{n}\): रूट परीक्षण → विचलन।
\end{enumerate}

\subsubsection{यह क्यों मायने रखता
है}\label{ux92fux939-ux915ux92f-ux92eux92fux928-ux930ux916ux924-ux939-27}

\begin{itemize}
\tightlist
\item
  अभिसरण परीक्षण हमें स्पष्ट योगों की आवश्यकता के बिना श्रृंखला को वर्गीकृत करने देते हैं।
\item
  वे कैलकुलस में अनंत प्रक्रियाओं को संभालने के लिए व्यवस्थित तरीके प्रदान करते हैं।
\item
  वे पावर श्रृंखला और फूरियर श्रृंखला जैसे बाद के विषयों के लिए महत्वपूर्ण हैं।
\end{itemize}

\subsubsection{व्यायाम}\label{ux935ux92fux92fux92e-45}

\begin{enumerate}
\def\labelenumi{\arabic{enumi}.}
\tightlist
\item
  \(\sum \frac{1}{n^3}\) का परीक्षण अभिसरण।
\item
  \(\sum \frac{3^n}{n!}\) के लिए अनुपात परीक्षण का उपयोग करें।3. रूट परीक्षण को
  \(\sum \left(\frac{1}{2}\right)^n\) पर लागू करें।
\item
  \(\sum \frac{(-1)^n}{\sqrt{n}}\) का अभिसरण निर्धारित करें।
\item
  \(\sum \frac{1}{n^2+1}\) का परीक्षण करने के लिए \(\frac{1}{n^2}\) के साथ
  सीमा तुलना परीक्षण का उपयोग करें।
\end{enumerate}

\subsection{12.3 निरपेक्ष बनाम सशर्त
अभिसरण}\label{ux928ux930ux92aux915ux937-ux92cux928ux92e-ux938ux936ux930ux924-ux905ux92dux938ux930ux923}

जब संकेत वैकल्पिक होते हैं तो सभी शृंखलाएं एक जैसा व्यवहार नहीं करतीं। इसे संभालने के लिए,
हम पूर्ण अभिसरण और सशर्त अभिसरण के बीच अंतर करते हैं।

\subsubsection{पूर्ण
अभिसरण}\label{ux92aux930ux923-ux905ux92dux938ux930ux923}

एक श्रृंखला \(\sum a_n\) पूर्णतः अभिसारी है यदि

\[
\sum |a_n|
\]

जुटता है.

प्रमेय: यदि कोई श्रृंखला पूर्ण रूप से अभिसरण करती है, तो वह भी अभिसरण करती है।

उदाहरण:

\[
\sum \frac{(-1)^n}{n^2}.
\]

यहां \(\sum \left|\frac{(-1)^n}{n^2}\right| = \sum \frac{1}{n^2}\) अभिसरण
करता है (p-श्रृंखला, \(p=2\))। तो श्रृंखला बिल्कुल अभिसरण है.

\subsubsection{सशर्त
अभिसरण}\label{ux938ux936ux930ux924-ux905ux92dux938ux930ux923}

एक श्रृंखला \(\sum a_n\) यदि अभिसरण होती है तो सशर्त रूप से अभिसरण होती है, लेकिन
पूर्ण रूप से नहीं।

उदाहरण:

\[
\sum \frac{(-1)^n}{n}.
\]

\begin{itemize}
\tightlist
\item
  वैकल्पिक श्रृंखला परीक्षण → अभिसरण।
\item
  लेकिन \(\sum \left|\frac{(-1)^n}{n}\right| = \sum \frac{1}{n}\) विचलन
  (हार्मोनिक श्रृंखला)। इसलिए श्रृंखला सशर्त रूप से अभिसरण है।
\end{itemize}

\subsubsection{पुनर्व्यवस्था
प्रमेय}\label{ux92aux928ux930ux935ux92fux935ux938ux925-ux92aux930ux92eux92f}

सशर्त रूप से अभिसरण श्रृंखला के लिए, शब्दों को पुनर्व्यवस्थित करने से योग बदल सकता है -
यहां तक कि यह अलग हो सकता है या एक अलग मूल्य में परिवर्तित हो सकता है।

यह आश्चर्यजनक परिणाम सशर्त अभिसरण की नाजुक प्रकृति को दर्शाता है।

\subsubsection{यह क्यों मायने रखता
है}\label{ux92fux939-ux915ux92f-ux92eux92fux928-ux930ux916ux924-ux939-28}

\begin{itemize}
\tightlist
\item
  पूर्ण अभिसरण मजबूत है और स्थिरता की गारंटी देता है।
\item
  सशर्त अभिसरण अनंत राशियों में क्रम के महत्व पर प्रकाश डालता है।
\item
  व्यवहार में आने वाली कई वैकल्पिक श्रृंखलाएं केवल सशर्त रूप से अभिसरण होती हैं।
\end{itemize}

\subsubsection{व्यायाम}\label{ux935ux92fux92fux92e-46}

\begin{enumerate}
\def\labelenumi{\arabic{enumi}.}
\tightlist
\item
  दिखाएँ कि \(\sum \frac{(-1)^n}{n^3}\) बिल्कुल अभिसरण करता है।
\item
  दिखाएँ कि \(\sum \frac{(-1)^n}{n}\) सशर्त रूप से अभिसरण है।
\item
  पूर्ण और सशर्त अभिसरण के लिए \(\sum \frac{(-1)^n}{\sqrt{n}}\) का परीक्षण
  करें।
\item
  स्पष्ट करें कि पूर्ण अभिसरण का अर्थ अभिसरण क्यों है, लेकिन इसका विपरीत सत्य नहीं है।
\item
  रीमैन पुनर्व्यवस्था प्रमेय पर शोध करें और उसे अपने शब्दों में सारांशित करें।
\end{enumerate}

\section{अध्याय 13. शक्ति श्रृंखला और
विस्तार}\label{ux905ux927ux92fux92f-13.-ux936ux915ux924-ux936ux930ux916ux932-ux914ux930-ux935ux938ux924ux930}

\subsection{13.1 पावर श्रृंखला}\label{ux92aux935ux930-ux936ux930ux916ux932}

घात श्रृंखला एक अनंत श्रृंखला है जिसमें प्रत्येक पद में चर की एक घात शामिल होती है।
कैलकुलस में घात श्रृंखला केंद्रीय होती है क्योंकि वे हमें कार्यों को अनंत बहुपदों के रूप में
प्रस्तुत करने देती है।

\subsubsection{सामान्य
प्रपत्र}\label{ux938ux92eux928ux92f-ux92aux930ux92aux924ux930-1}

\(a\) पर केन्द्रित एक शक्ति श्रृंखला का रूप है

\[
\sum_{n=0}^\infty c_n (x-a)^n,
\]

जहां \(c_n\) स्थिरांक हैं जिन्हें गुणांक कहा जाता है।

\begin{itemize}
\item
  यदि \(a=0\), तो श्रृंखला मूल बिंदु पर केन्द्रित है:

  \[
  \sum_{n=0}^\infty c_n x^n.
  \]
\end{itemize}

\subsubsection{उदाहरण}\label{ux909ux926ux939ux930ux923-28}

\begin{enumerate}
\def\labelenumi{\arabic{enumi}.}
\tightlist
\item
  ज्यामितीय श्रृंखला
\end{enumerate}

\[
\sum_{n=0}^\infty x^n = \frac{1}{1-x}, \quad |x|<1.
\]

\begin{enumerate}
\def\labelenumi{\arabic{enumi}.}
\setcounter{enumi}{1}
\tightlist
\item
  घातीय फलन
\end{enumerate}

\[e^x = \sum_{n=0}^\infty \frac{x^n}{n!}.
\]

\begin{enumerate}
\def\labelenumi{\arabic{enumi}.}
\setcounter{enumi}{2}
\tightlist
\item
  Sine and cosine
\end{enumerate}

\[
\sin x = \sum_{n=0}^\infty (-1)^n \frac{x^{2n+1}}{(2n+1)!}, \quad  
\cos x = \sum_{n=0}^\infty (-1)^n \frac{x^{2n}}{(2n)!}.
\]

\subsubsection{Interval of Convergence}\label{interval-of-convergence}

For each power series, there exists a radius of convergence \(R\) such
that:

\begin{itemize}
\tightlist
\item
  The series converges if \(|x-a| < R\).
\item
  The series diverges if \(|x-a| > R\).
\item
  At \(|x-a| = R\), convergence must be checked separately.
\end{itemize}

\subsubsection{Why This Matters}\label{why-this-matters-3}

\begin{itemize}
\tightlist
\item
  Power series allow us to approximate functions by polynomials.
\item
  They connect calculus with analysis and differential equations.
\item
  Many special functions in mathematics and physics are defined by their
  power series.
\end{itemize}

\subsubsection{Exercises}\label{exercises-6}

\begin{enumerate}
\def\labelenumi{\arabic{enumi}.}
\tightlist
\item
  Write the power series for \(\sum_{n=0}^\infty \frac{(x-2)^n}{n!}\).
\item
  Find the first four terms of the power series for \(e^x\).
\item
  Express \(\frac{1}{1+x}\) as a power series centered at 0.
\item
  Determine whether the series \(\sum_{n=0}^\infty n! x^n\) converges at
  \(x=0.1\).
\item
  Explain why power series are sometimes called ``infinite
  polynomials.''
\end{enumerate}

\subsection{13.2 Radius of Convergence}\label{radius-of-convergence}

Every power series converges for some values of \(x\) and diverges for
others. The boundary between these two behaviors is described by the
radius of convergence.

\subsubsection{Definition}\label{definition}

For a power series

\[
\sum_{n=0}^\infty c_n (x-a)^n,
\]

there exists a number \(R \geq 0\) (possibly infinite) such that:

\begin{itemize}
\tightlist
\item
  The series converges absolutely if \(|x-a| < R\).
\item
  The series diverges if \(|x-a| > R\).
\item
  At \(|x-a| = R\), convergence must be checked separately.
\end{itemize}

This number \(R\) is called the radius of convergence.

\subsubsection{Finding the Radius of
Convergence}\label{finding-the-radius-of-convergence}

Two common methods:

\begin{enumerate}
\def\labelenumi{\arabic{enumi}.}
\tightlist
\item
  Ratio Test
\end{enumerate}

\[
आर = \lim_{n\to\infty} \left| \frac{c_n}{c_{n+1}} \right|.
\]

\begin{enumerate}
\def\labelenumi{\arabic{enumi}.}
\setcounter{enumi}{1}
\tightlist
\item
  Root Test
\end{enumerate}

\[
आर = \frac{1}{\limsup_{n\to\infty} \sqrt[n]{|c_n|}}.
\]

\subsubsection{Examples}\label{examples-4}

\begin{enumerate}
\def\labelenumi{\arabic{enumi}.}
\tightlist
\item
  Series:
\end{enumerate}

\[
\sum_{n=0}^\infty \frac{x^n}{n!}.
\]

Using ratio test:

\[
\lim_{n\to\infty} \frac{1/(n!)}{1/((n+1)!)} = \infty.
\]

So \(R = \infty\) (converges for all real \(x\)).

\begin{enumerate}
\def\labelenumi{\arabic{enumi}.}
\setcounter{enumi}{1}
\tightlist
\item
  Series:
\end{enumerate}

\[
\sum_{n=0}^\infty x^n.
\]

Here \(c_n = 1\).

\[
आर = 1.
\]

Converges for \(|x| < 1\).

\begin{enumerate}
\def\labelenumi{\arabic{enumi}.}
\setcounter{enumi}{2}
\tightlist
\item
  Series:
\end{enumerate}

\[
\sum_{n=1}^\infty \frac{x^n}{n}.
\]

Ratio test:

\[
\lim_{n\to\infty} \left|\frac{(x^{n+1}/(n+1))}{(x^n/n)}\right| = |x|
\]

तो \(R = 1\). \(|x| < 1\) के लिए अभिसरण, \(|x| > 1\) के लिए विचलन।
\(x=\pm 1\) पर, अलग से परीक्षण करें।

\subsubsection{अभिसरण का
अंतराल}\label{ux905ux92dux938ux930ux923-ux915-ux905ux924ux930ux932}

\(x\)-मानों का सेट जहां श्रृंखला अभिसरण करती है उसे अभिसरण का अंतराल कहा जाता है।

\begin{itemize}
\tightlist
\item
  हमेशा \(a\) पर केंद्रित होता है।
\item
  \(R\) इकाइयों को दोनों दिशाओं में विस्तारित करता है।
\item
  समापन बिंदु \(x=a\pm R\) को व्यक्तिगत रूप से जांचा जाना चाहिए।
\end{itemize}

\subsubsection{यह क्यों मायने रखता है- अभिसरण की त्रिज्या हमें बताती है कि शक्ति
श्रृंखला कहाँ कार्यों की तरह व्यवहार करती
है।}\label{ux92fux939-ux915ux92f-ux92eux92fux928-ux930ux916ux924-ux939--ux905ux92dux938ux930ux923-ux915-ux924ux930ux91cux92f-ux939ux92e-ux92cux924ux924-ux939-ux915-ux936ux915ux924-ux936ux930ux916ux932-ux915ux939-ux915ux930ux92f-ux915-ux924ux930ux939-ux935ux92fux935ux939ux930-ux915ux930ux924-ux939}

\begin{itemize}
\tightlist
\item
  व्यवहार में टेलर श्रृंखला विस्तार का उपयोग करने के लिए आवश्यक।
\item
  भौतिकी और इंजीनियरिंग में श्रृंखला समाधानों की वैधता का क्षेत्र निर्धारित करता है।
\end{itemize}

\subsubsection{व्यायाम}\label{ux935ux92fux92fux92e-47}

\begin{enumerate}
\def\labelenumi{\arabic{enumi}.}
\tightlist
\item
  \(\sum_{n=0}^\infty \frac{(x-3)^n}{n!}\) के अभिसरण की त्रिज्या ज्ञात
  कीजिए।
\item
  \(\sum_{n=1}^\infty \frac{x^n}{n^2}\) के अभिसरण की त्रिज्या की गणना करें।
\item
  \(\sum_{n=0}^\infty n!x^n\) के लिए \(R\) खोजने के लिए अनुपात परीक्षण का
  उपयोग करें।
\item
  \(\sum_{n=1}^\infty \frac{(x+1)^n}{n}\) के लिए अभिसरण का अंतराल निर्धारित
  करें।
\item
  बताएं कि क्यों घातीय श्रृंखला सभी \(x\) के लिए अभिसरण होती है, जबकि ज्यामितीय
  श्रृंखला केवल \(|x|<1\) के लिए अभिसरण होती है।
\end{enumerate}

\subsection{13.3 टेलर और मैकलॉरिन
श्रृंखला}\label{ux91fux932ux930-ux914ux930-ux92eux915ux932ux930ux928-ux936ux930ux916ux932}

पावर श्रृंखला विशेष रूप से तब शक्तिशाली हो जाती है जब उनका उपयोग परिचित कार्यों को
दर्शाने के लिए किया जाता है। यह टेलर श्रृंखला के माध्यम से किया जाता है, और 0 पर
केंद्रित विशेष मामले को मैकलॉरिन श्रृंखला कहा जाता है।

\subsubsection{टेलर सीरीज}\label{ux91fux932ux930-ux938ux930ux91c}

यदि कोई फ़ंक्शन \(f(x)\) \(x=a\) पर असीम रूप से भिन्न है, तो \(a\) के बारे में इसकी
टेलर श्रृंखला है

\[
f(x) = \sum_{n=0}^\infty \frac{f^{(n)}(a)}{n!}(x-a)^n.
\]

यहां \(f^{(n)}(a)\) \(a\) पर \(f\) के \(n\)वें व्युत्पन्न को दर्शाता है।

\subsubsection{मैकलॉरिन
श्रृंखला}\label{ux92eux915ux932ux930ux928-ux936ux930ux916ux932}

\(a=0\) पर केन्द्रित एक टेलर श्रृंखला:

\[
f(x) = \sum_{n=0}^\infty \frac{f^{(n)}(0)}{n!} x^n.
\]

\subsubsection{उदाहरण}\label{ux909ux926ux939ux930ux923-29}

\begin{enumerate}
\def\labelenumi{\arabic{enumi}.}
\tightlist
\item
  घातीय फलन
\end{enumerate}

\[
e^x = 1 + x + \frac{x^2}{2!} + \frac{x^3}{3!} + \cdots
\]

\begin{enumerate}
\def\labelenumi{\arabic{enumi}.}
\setcounter{enumi}{1}
\tightlist
\item
  ज्या और कोज्या
\end{enumerate}

\[
\sin x = x - \frac{x^3}{3!} + \frac{x^5}{5!} - \cdots
\]

\[
\cos x = 1 - \frac{x^2}{2!} + \frac{x^4}{4!} - \cdots
\]

\begin{enumerate}
\def\labelenumi{\arabic{enumi}.}
\setcounter{enumi}{2}
\tightlist
\item
  प्राकृतिक लघुगणक (\(|x|<1\) के लिए)
\end{enumerate}

\[
\ln(1+x) = x - \frac{x^2}{2} + \frac{x^3}{3} - \frac{x^4}{4} + \cdots
\]

\subsubsection{टेलर बहुपद
सन्निकटन}\label{ux91fux932ux930-ux92cux939ux92aux926-ux938ux928ux928ux915ux91fux928}

पहले \(n\) पदों का परिमित योग \(n\) डिग्री का टेलर बहुपद है:

\[
P_n(x) = \sum_{k=0}^n \frac{f^{(k)}(a)}{k!}(x-a)^k.
\]

यह बहुपद \(f(x)\) को \(x=a\) के करीब लाता है।

\subsubsection{शेष (त्रुटि
अवधि)}\label{ux936ux937-ux924ux930ux91f-ux905ux935ux927}

फ़ंक्शन और उसके टेलर बहुपद के बीच का अंतर शेषफल है:

\[
R_n(x) = f(x) - P_n(x).
\]

एक रूप (लैग्रेंज का रूप) है

\[
R_n(x) = \frac{f^{(n+1)}(c)}{(n+1)!}(x-a)^{n+1},
\]

\(a\) और \(x\) के बीच कुछ \(c\) के लिए।

\subsubsection{यह क्यों मायने रखता
है}\label{ux92fux939-ux915ux92f-ux92eux92fux928-ux930ux916ux924-ux939-29}

\begin{itemize}
\tightlist
\item
  टेलर श्रृंखला जटिल कार्यों के लिए बहुपद सन्निकटन प्रदान करती है।
\item
  वे संख्यात्मक विश्लेषण, भौतिकी और इंजीनियरिंग में आवश्यक हैं।
\item
  मैकलॉरिन श्रृंखला विस्तार घातांकीय, त्रिकोणमितीय और लघुगणकीय कार्यों के लिए सरल
  सूत्र देते हैं।
\end{itemize}

\subsubsection{व्यायाम}\label{ux935ux92fux92fux92e-48}

\begin{enumerate}
\def\labelenumi{\arabic{enumi}.}
\tightlist
\item
  \(f(x)=\cosh x = \tfrac{e^x+e^{-x}}{2}\) के लिए मैकलॉरिन श्रृंखला खोजें।
\item
  \(f(x)=e^x\) के लिए टेलर श्रृंखला \(a=2\) पर केंद्रित लिखें।
\item
  \(a=0\) पर \(f(x)=\ln(1+x)\) के लिए डिग्री-3 टेलर बहुपद की गणना करें।4.
  \(\sin(0.1)\) को अनुमानित करने के लिए \(\sin x\) के लिए मैकलॉरिन श्रृंखला का
  उपयोग करें।
\item
  बताएं कि क्यों टेलर श्रृंखला अक्सर अच्छे स्थानीय सन्निकटन प्रदान करती है लेकिन बड़े
  \(|x|\) के लिए भिन्न हो सकती है।
\end{enumerate}

\subsection{13.4 टेलर श्रृंखला के
अनुप्रयोग}\label{ux91fux932ux930-ux936ux930ux916ux932-ux915-ux905ux928ux92aux930ux92fux917}

टेलर श्रृंखला केवल सैद्धांतिक उपकरण नहीं हैं - इनका उपयोग कार्यों का अनुमान लगाने,
समीकरणों को हल करने और भौतिक प्रणालियों का विश्लेषण करने के लिए किया जाता है। उनके
अनुप्रयोग गणित, विज्ञान और इंजीनियरिंग तक फैले हुए हैं।

\subsubsection{फ़ंक्शन
अनुमान}\label{ux92bux915ux936ux928-ux905ux928ux92eux928}

जटिल कार्यों को एक बिंदु के निकट बहुपदों द्वारा अनुमानित किया जा सकता है।

उदाहरण: डिग्री-3 मैकलॉरिन बहुपद का उपयोग करके \(x=0\) के निकट अनुमानित
\(e^x\):

\[
P_3(x) = 1 + x + \tfrac{x^2}{2} + \tfrac{x^3}{6}.
\]

छोटे \(x\) के लिए, यह \(e^x\) का सटीक अनुमान देता है।

\subsubsection{संख्यात्मक
विधियाँ}\label{ux938ux916ux92fux924ux92eux915-ux935ux927ux92f}

टेलर श्रृंखला संख्यात्मक एल्गोरिदम के लिए आधार प्रदान करती है:

\begin{itemize}
\tightlist
\item
  अनुमानित वर्गमूल, लघुगणक और त्रिकोणमितीय मान।
\item
  शेष पद के माध्यम से त्रुटि अनुमान.
\item
  न्यूटन की विधि (जहां स्थानीय रैखिककरण टेलर श्रृंखला से आता है) जैसे पुनरावृत्त तरीकों में
  उपयोग किया जाता है।
\end{itemize}

\subsubsection{विभेदक समीकरणों को हल
करना}\label{ux935ux92dux926ux915-ux938ux92eux915ux930ux923-ux915-ux939ux932-ux915ux930ux928}

कई विभेदक समीकरणों के समाधान टेलर (या शक्ति) श्रृंखला के रूप में व्यक्त किए जाते हैं।

उदाहरण: \(y(0)=0, y'(0)=1\) के साथ \(y'' + y = 0\) का समाधान \(\sin x\)
है, जो स्वाभाविक रूप से इसकी मैकलॉरिन श्रृंखला से उत्पन्न होता है।

\subsubsection{भौतिकी और
इंजीनियरिंग}\label{ux92dux924ux915-ux914ux930-ux907ux91cux928ux92fux930ux917}

\begin{itemize}
\item
  लघु-कोण सन्निकटन:

  \[
  \sin x \approx x, \quad \cos x \approx 1 - \tfrac{x^2}{2}, \quad |x| \ll 1.
  \]

  पेंडुलम गति, प्रकाशिकी और तरंग यांत्रिकी में उपयोग किया जाता है।
\item
  सापेक्षता और क्वांटम यांत्रिकी: टेलर विस्तार व्यावहारिक उपयोग के लिए गैर-रेखीय
  अभिव्यक्तियों को सरल बनाते हैं।
\item
  अनुमानित ऊर्जा कार्य: यांत्रिकी में, संभावित ऊर्जा कार्यों को संतुलन बिंदुओं के निकट
  विस्तारित किया जाता है।
\end{itemize}

\subsubsection{संभाव्यता और
सांख्यिकी}\label{ux938ux92dux935ux92fux924-ux914ux930-ux938ux916ux92fux915}

\begin{itemize}
\tightlist
\item
  क्षण उत्पन्न करने वाले कार्य और विशिष्ट कार्य शक्ति श्रृंखला का उपयोग करते हैं।
\item
  संभाव्यता वितरण का अनुमान (जैसे, द्विपद का सामान्य सन्निकटन) टेलर विस्तार का
  उपयोग करते हैं।
\end{itemize}

\subsubsection{यह क्यों मायने रखता
है}\label{ux92fux939-ux915ux92f-ux92eux92fux928-ux930ux916ux924-ux939-30}

\begin{itemize}
\tightlist
\item
  टेलर श्रृंखला सटीक सूत्रों और व्यावहारिक गणना के बीच एक पुल प्रदान करती है।
\item
  वे हमें जटिल समस्याओं को प्रबंधनीय बहुपद सन्निकटन तक कम करने की अनुमति देते हैं।
\item
  एप्लिकेशन उन्हें व्यावहारिक गणित में सबसे महत्वपूर्ण उपकरणों में से एक बनाते हैं।
\end{itemize}

\subsubsection{व्यायाम}\label{ux935ux92fux92fux92e-49}

\begin{enumerate}
\def\labelenumi{\arabic{enumi}.}
\tightlist
\item
  चार दशमलव स्थानों तक \(e^{0.1}\) का अनुमान लगाने के लिए \(e^x\) के लिए
  मैकलॉरिन श्रृंखला का उपयोग करें।
\item
  \(\sin(5^\circ)\) का अनुमान लगाने के लिए लघु-कोण सन्निकटन लागू करें।
\item
  घात श्रृंखला दृष्टिकोण का उपयोग करके अवकल समीकरण \(y'' = -y\) को हल करें।
\item
  \(\ln(1+x)\) को चौथी डिग्री तक विस्तारित करें और इसका उपयोग लगभग
  \(\ln(1.1)\) तक करें।
\item
  बताएं कि बहुपद सन्निकटन कंप्यूटर और कैलकुलेटर के लिए विशेष रूप से उपयोगी क्यों हैं।\#
  परिशिष्ट
\end{enumerate}

\subsection{परिशिष्ट ए. प्री-कैलकुलस
अनिवार्यताएँ}\label{ux92aux930ux936ux937ux91f-ux90f.-ux92aux930-ux915ux932ux915ux932ux938-ux905ux928ux935ux930ux92fux924ux90f}

\subsubsection{ए.1 बीजगणित
पुनश्चर्या}\label{ux90f.1-ux92cux91cux917ux923ux924-ux92aux928ux936ux91aux930ux92f}

कैलकुलस में गोता लगाने से पहले, कुछ बीजगणित कौशलों की समीक्षा करने में मदद मिलती है जो
बार-बार सामने आएंगे। ये वे ``उपकरण'' हैं जिनकी आपको अभिव्यक्तियों में हेरफेर करने,
समीकरणों को हल करने और परिणामों को सरल बनाने के लिए आवश्यकता होगी।

\paragraph{प्रतिपादक और
शक्तियाँ}\label{ux92aux930ux924ux92aux926ux915-ux914ux930-ux936ux915ux924ux92f}

\begin{itemize}
\item
  बुनियादी नियम:

  \[
  a^m \cdot a^n = a^{m+n}, \quad \frac{a^m}{a^n} = a^{m-n}, \quad (a^m)^n = a^{mn}.
  \]
\item
  नकारात्मक प्रतिपादक:

  \[
  a^{-n} = \frac{1}{a^n}, \quad a \neq 0.
  \]
\item
  भिन्नात्मक घातांक:

  \[
  a^{1/n} = \sqrt[n]{a}, \quad a^{m/n} = \sqrt[n]{a^m}.
  \]
\end{itemize}

\paragraph{फैक्टरिंग}\label{ux92bux915ux91fux930ux917}

फैक्टरिंग अभिव्यक्ति को सरल बनाता है और समीकरणों को हल करने में मदद करता है।

\begin{enumerate}
\def\labelenumi{\arabic{enumi}.}
\item
  सामान्य कारक:

  \[
  6x^2+9x = 3x(2x+3).
  \]
\item
  वर्गों का अंतर:

  \[
  a^2-b^2 = (a-b)(a+b).
  \]
\item
  द्विघात त्रिपद:

  \[
  x^2+5x+6 = (x+2)(x+3).
  \]
\end{enumerate}

\paragraph{बहुपद}\label{ux92cux939ux92aux926}

\begin{itemize}
\tightlist
\item
  मानक प्रपत्र: \(P(x) = a_nx^n + a_{n-1}x^{n-1} + \cdots + a_0\)।
\item
  डिग्री: \(x\) की सबसे बड़ी शक्ति।
\item
  तर्कसंगत कार्यों को सरल बनाने के लिए दीर्घ विभाजन और सिंथेटिक विभाजन उपयोगी हैं।
\end{itemize}

\paragraph{तर्कसंगत
अभिव्यक्तियाँ}\label{ux924ux930ux915ux938ux917ux924-ux905ux92dux935ux92fux915ux924ux92f}

अंश और हर का गुणनखंडन करके सरल बनाएं:

\[
\frac{x^2-1}{x^2-2x+1} = \frac{(x-1)(x+1)}{(x-1)^2} = \frac{x+1}{x-1}, \quad x \neq 1.
\]

\paragraph{लघुगणक}\label{ux932ux918ux917ux923ux915}

\begin{itemize}
\item
  परिभाषा: \(\log_a b = c\) का अर्थ है \(a^c = b\)।
\item
  सामान्य आधार: प्राकृतिक लॉग (\(\ln x = \log_e x\)) और आधार 10
  (\(\log x\))।
\item
  नियम:

  \[
  \log(ab) = \log a + \log b, \quad \log\left(\frac{a}{b}\right) = \log a - \log b, \quad \log(a^n) = n\log a.
  \]
\end{itemize}

\paragraph{समीकरण}\label{ux938ux92eux915ux930ux923}

\begin{itemize}
\item
  रैखिक: \(ax+b=0\) → \(x=-b/a\) को हल करें।
\item
  द्विघात: \(ax^2+bx+c=0\) के समाधान हैं

  \[
  x=\frac{-b\pm \sqrt{b^2-4ac}}{2a}.
  \]
\item
  घातांक: \(e^x = k\) → \(x = \ln k\)।
\end{itemize}

\subsubsection{ए.2 त्रिकोणमिति मूल
बातें}\label{ux90f.2-ux924ux930ux915ux923ux92eux924-ux92eux932-ux92cux924}

त्रिकोणमिति कोणों और आवर्त घटनाओं की भाषा प्रदान करती है। चूँकि कैलकुलस अक्सर
दोलनों, गति और तरंगों से संबंधित होता है, इसलिए त्रिकोणमितीय कार्यों और उनके गुणों की
ठोस समझ आवश्यक है।

\paragraph{यूनिट सर्कल}\label{ux92fux928ux91f-ux938ux930ux915ux932}

\begin{itemize}
\item
  निर्देशांक तल में मूल बिंदु पर केन्द्रित त्रिज्या 1 के वृत्त के रूप में परिभाषित।
\item
  सकारात्मक \(x\)-अक्ष से मापे गए कोण \(\theta\) के लिए:

  \[
  (\cos \theta, \sin \theta)
  \]

  वृत्त पर बिंदु के निर्देशांक देता है।
\end{itemize}

विशेष मूल्य:

\begin{longtable}[]{@{}
  >{\raggedright\arraybackslash}p{(\linewidth - 6\tabcolsep) * \real{0.3333}}
  >{\raggedright\arraybackslash}p{(\linewidth - 6\tabcolsep) * \real{0.1667}}
  >{\raggedright\arraybackslash}p{(\linewidth - 6\tabcolsep) * \real{0.1667}}
  >{\raggedright\arraybackslash}p{(\linewidth - 6\tabcolsep) * \real{0.3333}}@{}}
\toprule\noalign{}
\begin{minipage}[b]{\linewidth}\raggedright
\(\theta\)
\end{minipage} & \begin{minipage}[b]{\linewidth}\raggedright
\(\sin \theta\)
\end{minipage} & \begin{minipage}[b]{\linewidth}\raggedright
\(\cos \theta\)
\end{minipage} & \begin{minipage}[b]{\linewidth}\raggedright
\(\tan \theta = \frac{\sin \theta}{\cos \theta}\)
\end{minipage} \\
\midrule\noalign{}
\endhead
\bottomrule\noalign{}
\endlastfoot
\(0\) & 0 & 1 & 0 \\
\(\pi/6\) & 1/2 & \(\sqrt{3}/2\) & \(1/\sqrt{3}\) \\
\(\pi/3\) & \(\sqrt{3}/2\) & 1/2 & \(\sqrt{3}\) \\
\(\pi/2\) & 1 & 0 & अपरिभाषित \\
\end{longtable}

\paragraph{मौलिक पहचान}\label{ux92eux932ux915-ux92aux939ux91aux928}

\begin{enumerate}
\def\labelenumi{\arabic{enumi}.}
\tightlist
\item
  पायथागॉरियन पहचान
\end{enumerate}

\[
\sin^2\theta + \cos^2\theta = 1.
\]

\begin{enumerate}
\def\labelenumi{\arabic{enumi}.}
\setcounter{enumi}{1}
\tightlist
\item
  भागफल पहचान
\end{enumerate}

\[
\tan\theta = \frac{\sin\theta}{\cos\theta}, \quad \cot\theta = \frac{\cos\theta}{\sin\theta}.
\]

\begin{enumerate}
\def\labelenumi{\arabic{enumi}.}
\setcounter{enumi}{2}
\tightlist
\item
  पारस्परिक पहचान
\end{enumerate}

\[
\sec\theta = \frac{1}{\cos\theta}, \quad \csc\theta = \frac{1}{\sin\theta}.
\]

\paragraph{कोण जोड़ सूत्र}\label{ux915ux923-ux91cux921-ux938ux924ux930}

\[
\sin(\alpha+\beta) = \sin\alpha\cos\beta + \cos\alpha\sin\beta,
\]

\[
\cos(\alpha+\beta) = \cos\alpha\cos\beta - \sin\alpha\sin\beta.
\]

विशेष मामले:

\begin{itemize}
\item
  दोहरा कोण:

  \[
  \sin(2\theta) = 2\sin\theta\cos\theta, \quad
  \cos(2\theta) = \cos^2\theta - \sin^2\theta.
  \]
\end{itemize}

\paragraph{रेखांकन}\label{ux930ux916ux915ux928}

\begin{itemize}
\tightlist
\item
  \(\sin x\): तरंग 0 से शुरू होती है, आयाम 1, अवधि \(2\pi\)।
\item
  \(\cos x\): तरंग 1 से शुरू होती है, आयाम 1, अवधि \(2\pi\)।
\item
  \(\tan x\): प्रत्येक \(\pi\) को दोहराता है, \(\pi/2\) के विषम गुणजों पर
  अपरिभाषित।
\end{itemize}

\subsubsection{ए.3 निर्देशांक
ज्यामिति}\label{ux90f.3-ux928ux930ux926ux936ux915-ux91cux92fux92eux924}

समन्वय ज्यामिति समीकरणों का उपयोग करके ज्यामितीय वस्तुओं (रेखाएं, वृत्त, वक्र) का
वर्णन करके बीजगणित और ज्यामिति को जोड़ती है। रेखांकन कार्यों, ढलानों को खोजने और
वक्रों का विश्लेषण करने के लिए कैलकुलस इस ढांचे पर बहुत अधिक निर्भर करता है।

\paragraph{कार्तीय तल}\label{ux915ux930ux924ux92f-ux924ux932}

\begin{itemize}
\item
  एक बिंदु को निर्देशांक \((x,y)\) द्वारा दर्शाया जाता है।
\item
  दो बिंदुओं \((x_1,y_1)\) और \((x_2,y_2)\) के बीच की दूरी:

  \[
  d = \sqrt{(x_2-x_1)^2 + (y_2-y_1)^2}.
  \]
\item
  एक रेखाखंड का मध्यबिंदु:

  \[
  M = \left(\frac{x_1+x_2}{2}, \frac{y_1+y_2}{2}\right).
  \]
\end{itemize}

\paragraph{पंक्तियाँ}\label{ux92aux915ux924ux92f}

\begin{enumerate}
\def\labelenumi{\arabic{enumi}.}
\item
  ढलान सूत्र

  \[
  m = \frac{y_2-y_1}{x_2-x_1}.
  \]
\item
  एक रेखा का समीकरण

  \begin{itemize}
  \item
    बिंदु-ढलान प्रपत्र:

    \[
    y-y_1 = m(x-x_1).
    \]
  \item
    ढलान-अवरोधन प्रपत्र:

    \[
    y = mx+b.
    \]
  \end{itemize}
\item
  समानान्तर एवं लम्बवत रेखाएँ

  -समानांतर रेखाएँ: समान ढलान।

  \begin{itemize}
  \tightlist
  \item
    लंबवत रेखाएँ: ढलान \(m_1m_2 = -1\) को संतुष्ट करते हैं।
  \end{itemize}
\end{enumerate}

\paragraph{वृत्त}\label{ux935ux924ux924}

केंद्र \((h,k)\) और त्रिज्या \(r\) वाले वृत्त का समीकरण:

\[
(x-h)^2+(y-k)^2 = r^2.
\]

विशेष मामला: मूल पर केन्द्रित इकाई वृत्त:

\[
x^2+y^2=1.
\]

\paragraph{शंक्वाकार
अनुभाग}\label{ux936ux915ux935ux915ux930-ux905ux928ux92dux917}

\begin{enumerate}
\def\labelenumi{\arabic{enumi}.}
\item
  परवलय:

  \begin{itemize}
  \item
    मानक प्रपत्र (ऊपर/नीचे खोलना):

    \[
    y = ax^2+bx+c.
    \]
  \end{itemize}
\item
  दीर्घवृत्त (मूल बिंदु पर केन्द्रित):

  \[
  \frac{x^2}{a^2}+\frac{y^2}{b^2}=1.
  \]
\item
  अतिपरवलय (मूल पर केन्द्रित):

  \[
  \frac{x^2}{a^2}-\frac{y^2}{b^2}=1.
  \]
\end{enumerate}

\subsection{परिशिष्ट बी. मुख्य सूत्र और
तालिकाएँ}\label{ux92aux930ux936ux937ux91f-ux92c.-ux92eux916ux92f-ux938ux924ux930-ux914ux930-ux924ux932ux915ux90f}

\subsubsection{बी.1 व्युत्पन्न तालिकाडेरिवेटिव परिवर्तन की दर और कार्यों के ढलान
को मापते हैं। त्वरित-संदर्भ तालिका होने से शिक्षार्थियों को हर बार सूत्रों को दोबारा
प्राप्त करने से बचने में मदद मिलती
है।}\label{ux92c.1-ux935ux92fux924ux92aux928ux928-ux924ux932ux915ux921ux930ux935ux91fux935-ux92aux930ux935ux930ux924ux928-ux915-ux926ux930-ux914ux930-ux915ux930ux92f-ux915-ux922ux932ux928-ux915-ux92eux92aux924-ux939-ux924ux935ux930ux924-ux938ux926ux930ux92d-ux924ux932ux915-ux939ux928-ux938-ux936ux915ux937ux930ux925ux92f-ux915-ux939ux930-ux92cux930-ux938ux924ux930-ux915-ux926ux92cux930-ux92aux930ux92aux924-ux915ux930ux928-ux938-ux92cux91aux928-ux92e-ux92eux926ux926-ux92eux932ux924-ux939}

\paragraph{बुनियादी नियम}\label{ux92cux928ux92fux926-ux928ux92fux92e-1}

\begin{enumerate}
\def\labelenumi{\arabic{enumi}.}
\tightlist
\item
  निरंतर नियम
\end{enumerate}

\[
\frac{d}{dx}[c] = 0
\]

\begin{enumerate}
\def\labelenumi{\arabic{enumi}.}
\setcounter{enumi}{1}
\tightlist
\item
  शक्ति नियम
\end{enumerate}

\[
\frac{d}{dx}[x^n] = nx^{n-1}, \quad (n \in \mathbb{R})
\]

\begin{enumerate}
\def\labelenumi{\arabic{enumi}.}
\setcounter{enumi}{2}
\tightlist
\item
  लगातार एकाधिक नियम
\end{enumerate}

\[
\frac{d}{dx}[c f(x)] = c f'(x)
\]

\begin{enumerate}
\def\labelenumi{\arabic{enumi}.}
\setcounter{enumi}{3}
\tightlist
\item
  योग और अंतर नियम
\end{enumerate}

\[
\frac{d}{dx}[f(x)\pm g(x)] = f'(x)\pm g'(x)
\]

\paragraph{त्रिकोणमितीय
फलन}\label{ux924ux930ux915ux923ux92eux924ux92f-ux92bux932ux928}

\[
\frac{d}{dx}[\sin x] = \cos x
\]

\[
\frac{d}{dx}[\cos x] = -\sin x
\]

\[
\frac{d}{dx}[\tan x] = \sec^2 x, \quad x \neq \tfrac{\pi}{2}+k\pi
\]

\[
\frac{d}{dx}[\cot x] = -\csc^2 x
\]

\[
\frac{d}{dx}[\sec x] = \sec x \tan x
\]

\[
\frac{d}{dx}[\csc x] = -\csc x \cot x
\]

\paragraph{घातीय और लघुगणकीय
कार्य}\label{ux918ux924ux92f-ux914ux930-ux932ux918ux917ux923ux915ux92f-ux915ux930ux92f}

\[
\frac{d}{dx}[e^x] = e^x
\]

\[
\frac{d}{dx}[a^x] = a^x \ln a, \quad a>0, a\neq 1
\]

\[
\frac{d}{dx}[\ln x] = \frac{1}{x}, \quad x>0
\]

\[
\frac{d}{dx}[\log_a x] = \frac{1}{x\ln a}, \quad a>0, a\neq 1
\]

\paragraph{व्युत्क्रम त्रिकोणमितीय
फलन}\label{ux935ux92fux924ux915ux930ux92e-ux924ux930ux915ux923ux92eux924ux92f-ux92bux932ux928}

\[
\frac{d}{dx}[\arcsin x] = \frac{1}{\sqrt{1-x^2}}, \quad |x|<1
\]

\[
\frac{d}{dx}[\arccos x] = -\frac{1}{\sqrt{1-x^2}}, \quad |x|<1
\]

\[
\frac{d}{dx}[\arctan x] = \frac{1}{1+x^2}, \quad x \in \mathbb{R}
\]

\paragraph{उत्पाद, भागफल और श्रृंखला
नियम}\label{ux909ux924ux92aux926-ux92dux917ux92bux932-ux914ux930-ux936ux930ux916ux932-ux928ux92fux92e}

\begin{enumerate}
\def\labelenumi{\arabic{enumi}.}
\tightlist
\item
  उत्पाद नियम
\end{enumerate}

\[
\frac{d}{dx}[f(x)g(x)] = f'(x)g(x)+f(x)g'(x)
\]

\begin{enumerate}
\def\labelenumi{\arabic{enumi}.}
\setcounter{enumi}{1}
\tightlist
\item
  भागफल नियम
\end{enumerate}

\[
\frac{d}{dx}\left[\frac{f(x)}{g(x)}\right] = \frac{f'(x)g(x)-f(x)g'(x)}{[g(x)]^2}, \quad g(x)\neq 0
\]

\begin{enumerate}
\def\labelenumi{\arabic{enumi}.}
\setcounter{enumi}{2}
\tightlist
\item
  शृंखला नियम
\end{enumerate}

\[
\frac{d}{dx}[f(g(x))] = f'(g(x))\cdot g'(x)
\]

\subsubsection{बी.3 सामान्य श्रृंखला
विस्तार}\label{ux92c.3-ux938ux92eux928ux92f-ux936ux930ux916ux932-ux935ux938ux924ux930}

घात श्रृंखला हमें फलनों को अनंत बहुपदों के रूप में व्यक्त करने देती है। ये विस्तार अनुमान
लगाने, अंतर समीकरणों को हल करने और कैलकुलस में कार्यों के बारे में अंतर्ज्ञान बनाने के लिए
आवश्यक हैं।

\paragraph{ज्यामितीय
श्रृंखला}\label{ux91cux92fux92eux924ux92f-ux936ux930ux916ux932}

\[
\frac{1}{1-x} = \sum_{n=0}^\infty x^n, \quad |x| < 1
\]

\paragraph{घातीय फलन}\label{ux918ux924ux92f-ux92bux932ux928}

\[
e^x = \sum_{n=0}^\infty \frac{x^n}{n!}
= 1 + x + \frac{x^2}{2!} + \frac{x^3}{3!} + \cdots
\]

सभी \(x\) के लिए मान्य।

\paragraph{त्रिकोणमितीय
फलन}\label{ux924ux930ux915ux923ux92eux924ux92f-ux92bux932ux928-1}

\[
\sin x = \sum_{n=0}^\infty (-1)^n \frac{x^{2n+1}}{(2n+1)!}
= x - \frac{x^3}{3!} + \frac{x^5}{5!} - \cdots
\]

\[
\cos x = \sum_{n=0}^\infty (-1)^n \frac{x^{2n}}{(2n)!}
= 1 - \frac{x^2}{2!} + \frac{x^4}{4!} - \cdots
\]

\[
\tan^{-1} x = \sum_{n=0}^\infty (-1)^n \frac{x^{2n+1}}{2n+1}, \quad |x|\leq 1
\]

\paragraph{लघुगणक}\label{ux932ux918ux917ux923ux915-1}

\[
\ln(1+x) = \sum_{n=1}^\infty (-1)^{n+1} \frac{x^n}{n}, \quad -1 < x \leq 1
\]

\paragraph{द्विपद विस्तार
(सामान्यीकृत)}\label{ux926ux935ux92aux926-ux935ux938ux924ux930-ux938ux92eux928ux92fux915ux924}

\[
(1+x)^r = \sum_{n=0}^\infty \binom{r}{n} x^n, \quad |x|<1
\]

कहाँ

\[\binom{r}{n} = \frac{r(r-1)(r-2)\cdots(r-n+1)}{n!}.
\]

\subsection{Appendix C. Proof
Sketches}\label{appendix-c.-proof-sketches}

\subsubsection{\texorpdfstring{C.1 Limit Laws and the
\(\varepsilon\)--\(\delta\)
Definition}{C.1 Limit Laws and the \textbackslash varepsilon--\textbackslash delta Definition}}\label{c.1-limit-laws-and-the-varepsilondelta-definition}

Calculus rests on the precise meaning of a limit. While intuition
(``values get closer and closer'') is helpful, a formal definition
ensures rigor and avoids paradoxes.

\paragraph{Intuitive Idea}\label{intuitive-idea}

We write

\[
\lim_{x \to a} f(x) = L
\]

to mean that as \(x\) gets arbitrarily close to \(a\), the values of
\(f(x)\) get arbitrarily close to \(L\).

\paragraph{\texorpdfstring{Formal (\(\varepsilon\)--\(\delta\))
Definition}{Formal (\textbackslash varepsilon--\textbackslash delta) Definition}}\label{formal-varepsilondelta-definition}

We say that

\[
\lim_{x \to a} f(x) = L
\]

if for every \(\varepsilon > 0\), there exists a \(\delta > 0\) such
that whenever

\[
0 < |एक्स-ए| < \डेल्टा,
\]

we have

\[
|एफ(एक्स) - एल| < \varepsilon.
\]

\begin{itemize}
\tightlist
\item
  \(\varepsilon\): how close we want \(f(x)\) to be to \(L\).
\item
  \(\delta\): how close \(x\) must be to \(a\) to achieve that.
\end{itemize}

\paragraph{Example}\label{example}

Show that

\[
\lim_{x \to 2} (3x+1) = 7.
\]

\begin{itemize}
\tightlist
\item
  Let \(\varepsilon > 0\).
\item
  We want \(|(3x+1)-7| < \varepsilon\).
\item
  Simplify: \(|3x-6| = 3|x-2| < \varepsilon\).
\item
  This holds if we choose \(\delta = \varepsilon/3\).
\end{itemize}

Thus, by the definition, the limit is 7.

\paragraph{Limit Laws}\label{limit-laws}

If \(\lim_{x \to a} f(x) = L\) and \(\lim_{x \to a} g(x) = M\), then:

\begin{enumerate}
\def\labelenumi{\arabic{enumi}.}
\tightlist
\item
  Sum/Difference
\end{enumerate}

\[
\lim_{x \to a} [f(x) \pm g(x)] = L \pm M
\]

\begin{enumerate}
\def\labelenumi{\arabic{enumi}.}
\setcounter{enumi}{1}
\tightlist
\item
  Constant Multiple
\end{enumerate}

\[
\lim_{x \to a} [c f(x)] = cL
\]

\begin{enumerate}
\def\labelenumi{\arabic{enumi}.}
\setcounter{enumi}{2}
\tightlist
\item
  Product
\end{enumerate}

\[
\lim_{x \to a} [f(x)g(x)] = LM
\]

\begin{enumerate}
\def\labelenumi{\arabic{enumi}.}
\setcounter{enumi}{3}
\tightlist
\item
  Quotient (if \(M \neq 0\))
\end{enumerate}

\[
\lim_{x \to a} \frac{f(x)}{g(x)} = \frac{L}{M}
\]

\begin{enumerate}
\def\labelenumi{\arabic{enumi}.}
\setcounter{enumi}{4}
\tightlist
\item
  Powers and Roots
\end{enumerate}

\[
\lim_{x \to a} [f(x)]^n = L^n, \quad \lim_{x \to a} \sqrt[n]{f(x)} = \sqrt[n]{L} \ (\text{यदि परिभाषित})।
\]

\subsubsection{C.2 Proof Sketch: The Fundamental Theorem of
Calculus}\label{c.2-proof-sketch-the-fundamental-theorem-of-calculus}

The Fundamental Theorem of Calculus (FTC) links the two central
operations of calculus: differentiation and integration. It shows that
they are, in fact, inverse processes.

\paragraph{Statement of the Theorem}\label{statement-of-the-theorem}

Part I (Differentiation of an Integral): If \(f\) is continuous on
\([a,b]\) and we define

\[
F(x) = \int_a^x f(t)\,dt,
\]

then \(F\) is differentiable on \((a,b)\) and

\[
एफ'(एक्स) = एफ(एक्स).
\]

Part II (Evaluation of a Definite Integral): If \(F\) is any
antiderivative of \(f\) on \([a,b]\), then

\[
\int_a^b f(x)\,dx = F(b)-F(a).
\]

\paragraph{Proof Sketch of Part I}\label{proof-sketch-of-part-i}

\begin{enumerate}
\def\labelenumi{\arabic{enumi}.}
\item
  Start with the definition of the derivative:

  \[
  F'(x) = \lim_{h\to 0} \frac{F(x+h)-F(x)}{h}.
  \]
\item
  Substituting \(F(x) = \int_a^x f(t)\,dt\):

  \[
  F(x+h)-F(x) = \int_a^{x+h} f(t)\,dt - \int_a^x f(t)\,dt.
  \]
\item
  By the additivity of integrals:

  \[
  F(x+h)-F(x) = \int_x^{x+h} f(t)\,dt.
  \]
\item
  Therefore:

  \[
  \frac{F(x+h)-F(x)}{h} = \frac{1}{h}\int_x^{x+h} f(t)\,dt.
  \]5. अभिन्नों के लिए माध्य मान प्रमेय के अनुसार, \(c \in [x,x+h]\) मौजूद है जैसे
  कि

  \[
  \frac{1}{h}\int_x^{x+h} f(t)\,dt = f(c).
  \]
\item
  चूँकि \(h \to 0\), \(c \to x\), और चूँकि \(f\) निरंतर है:

  \[
  \lim_{h\to 0} f(c) = f(x).
  \]
\end{enumerate}

इस प्रकार, \(F'(x) = f(x)\)।

\paragraph{भाग II का प्रमाण
रेखाचित्र}\label{ux92dux917-ii-ux915-ux92aux930ux92eux923-ux930ux916ux91aux924ux930}

\begin{enumerate}
\def\labelenumi{\arabic{enumi}.}
\item
  मान लीजिए \(F\) \(f\) का प्रतिव्युत्पन्न है, इसलिए \(F'(x) = f(x)\)।
\item
  भाग I द्वारा, फ़ंक्शन

  \[
  G(x) = \int_a^x f(t)\,dt
  \]

  \(f\) का प्रतिव्युत्पन्न भी है।
\item
  चूँकि \(F\) और \(G\) में केवल एक स्थिरांक का अंतर है,

  \[
  F(x) = G(x) + C.
  \]
\item
  अंतिम बिंदुओं पर मूल्यांकन:

  \[
  \int_a^b f(x)\,dx = G(b)-G(a) = (F(b)+C)-(F(a)+C) = F(b)-F(a).
  \]
\end{enumerate}

\subsubsection{सी.3 प्रमाण रेखाचित्र: ज्यामितीय श्रृंखला का
अभिसरण}\label{ux938.3-ux92aux930ux92eux923-ux930ux916ux91aux924ux930-ux91cux92fux92eux924ux92f-ux936ux930ux916ux932-ux915-ux905ux92dux938ux930ux923}

ज्यामितीय श्रृंखला सबसे सरल और सबसे महत्वपूर्ण अनंत श्रृंखलाओं में से एक है। यह अभिसरण को
समझने के लिए एक मॉडल के रूप में कार्य करता है और कैलकुलस में बाद के कई परिणामों की नींव
है।

\paragraph{शृंखला}\label{ux936ux916ux932}

\[
\sum_{n=0}^\infty ar^n = a + ar + ar^2 + ar^3 + \cdots
\]

जहां \(a\) पहला पद है और \(r\) सामान्य अनुपात है।

\paragraph{आंशिक योग
सूत्र}\label{ux906ux936ux915-ux92fux917-ux938ux924ux930}

\(n\)-वाँ आंशिक योग है

\[
S_n = a + ar + ar^2 + \cdots + ar^n.
\]

दोनों पक्षों को \(r\) से गुणा करें:

\[
rS_n = ar + ar^2 + \cdots + ar^{n+1}.
\]

दो समीकरण घटाएँ:

\[
S_n - rS_n = a - ar^{n+1}.
\]

\[
S_n(1-r) = a(1-r^{n+1}).
\]

तो

\[
S_n = \frac{a(1-r^{n+1})}{1-r}, \quad r \neq 1.
\]

\paragraph{अभिसरण}\label{ux905ux92dux938ux930ux923}

सीमा को \(n \to \infty\) के रूप में लें:

\begin{itemize}
\item
  यदि \(|r| < 1\), तो \(r^{n+1} \to 0\)।

  \[
  \lim_{n\to\infty} S_n = \frac{a}{1-r}.
  \]
\item
  यदि \(|r| \geq 1\), तो \(r^{n+1}\) 0 पर नहीं जाता है। श्रृंखला अलग हो जाती
  है।
\end{itemize}

\paragraph{परिणाम}\label{ux92aux930ux923ux92e}

\[
\sum_{n=0}^\infty ar^n =
\begin{cases}
\dfrac{a}{1-r}, & |r|<1, \\[6pt]
\text{diverges}, & |r|\geq 1.
\end{cases}
\]

\subsection{परिशिष्ट डी. अनुप्रयोग और
कनेक्शन}\label{ux92aux930ux936ux937ux91f-ux921.-ux905ux928ux92aux930ux92fux917-ux914ux930-ux915ux928ux915ux936ux928}

\subsubsection{डी.1 भौतिकी संबंध: वेग, त्वरण, और
कार्य}\label{ux921.1-ux92dux924ux915-ux938ux92cux927-ux935ux917-ux924ux935ux930ux923-ux914ux930-ux915ux930ux92f}

कैलकुलस को मूल रूप से भौतिकी में समस्याओं को हल करने के लिए विकसित किया गया था -
विशेष रूप से गति और परिवर्तन। यहां कुछ सबसे महत्वपूर्ण कनेक्शन दिए गए हैं.

\paragraph{स्थिति, वेग, और
त्वरण}\label{ux938ux925ux924-ux935ux917-ux914ux930-ux924ux935ux930ux923-1}

\begin{itemize}
\item
  स्थिति फ़ंक्शन: \(s(t)\) समय \(t\) पर किसी वस्तु का स्थान देता है।
\item
  वेग: स्थिति का व्युत्पन्न.

  \[
  v(t) = s'(t) = \frac{ds}{dt}
  \]
\item
  त्वरण: वेग का व्युत्पन्न (या स्थिति का दूसरा व्युत्पन्न)।

  \[
  a(t) = v'(t) = s''(t) = \frac{d^2s}{dt^2}
  \]
\end{itemize}

उदाहरण: यदि \(s(t) = 4t^2\) मीटर है, तो:

\[
v(t) = 8t, \quad a(t) = 8.
\]

इसलिए वस्तु निरंतर त्वरण के तहत समय के साथ रैखिक रूप से तेजी से चलती है।

\paragraph{कार्य और बल}\label{ux915ux930ux92f-ux914ux930-ux92cux932}

भौतिकी में, कार्य बल और दूरी का उत्पाद है। यदि बल स्थिति के साथ बदलता है, तो
कैलकुलस देता है:

\[डब्ल्यू = \int_a^b F(x)\, dx
\]

where \(F(x)\) is the force at position \(x\), and the object moves from
\(x=a\) to \(x=b\).

Example: A spring with Hooke's law force \(F(x) = kx\) requires work

\[
डब्ल्यू = \int_0^d kx\, dx = \frac{1}{2}kd^2
\]

to stretch the spring a distance \(d\).

\paragraph{Energy and Areas Under
Curves}\label{energy-and-areas-under-curves}

\begin{itemize}
\tightlist
\item
  Kinetic energy: \(E_k = \tfrac{1}{2}mv^2\).
\item
  Potential energy often involves integrals (e.g., gravitational
  potential energy from force of gravity).
\item
  In general, integrating a force function gives energy stored or work
  done.
\end{itemize}

\paragraph{Quick Practice}\label{quick-practice}

\begin{enumerate}
\def\labelenumi{\arabic{enumi}.}
\tightlist
\item
  If \(s(t) = t^3 - 3t\), find \(v(t)\) and \(a(t)\).
\item
  Compute the work done by a constant force of 10 N moving an object 5
  m.
\item
  A spring has constant \(k=200\). How much work is needed to stretch it
  0.1 m?
\item
  Show that acceleration is the second derivative of position.
\item
  Explain how the integral \(\int v(t)\, dt\) relates to displacement.
\end{enumerate}

\subsubsection{D.2 Probability and Statistics
Connections}\label{d.2-probability-and-statistics-connections}

Calculus is deeply connected with probability and statistics, especially
when dealing with continuous random variables. Integrals become
essential for defining probabilities, averages, and expectations.

\paragraph{Probability Density Functions
(PDFs)}\label{probability-density-functions-pdfs}

For a continuous random variable \(X\), probabilities are described by a
probability density function \(f(x)\):

\begin{enumerate}
\def\labelenumi{\arabic{enumi}.}
\item
  \(f(x) \geq 0\) for all \(x\).
\item
  Total probability equals 1:

  \[
  \int_{-\infty}^{\infty} f(x)\, dx = 1.
  \]
\end{enumerate}

The probability that \(X\) lies in an interval \([a,b]\) is

\[
P(a \leq X \leq b) = \int_a^b f(x)\, dx.
\]

\paragraph{Expected Value (Mean)}\label{expected-value-mean}

The expected value (average outcome) is

\[
E[X] = \int_{-\infty}^{\infty} x f(x)\, dx.
\]

This is the calculus version of a weighted average.

\paragraph{Variance}\label{variance}

Variance measures spread:

\[
\text{Var}(X) = E[(X-\mu)^2] = \int_{-\infty}^{\infty} (x-\mu)^2 f(x)\, dx,
\]

where \(\mu = E[X]\).

\paragraph{Common Distributions}\label{common-distributions}

\begin{enumerate}
\def\labelenumi{\arabic{enumi}.}
\item
  Uniform distribution on \([a,b]\):

  \[
  f(x) = \frac{1}{b-a}, \quad a \leq x \leq b.
  \]

  Mean: \(\frac{a+b}{2}\).
\item
  Exponential distribution with parameter \(\lambda > 0\):

  \[
  f(x) = \lambda e^{-\lambda x}, \quad x \geq 0.
  \]

  Mean: \(1/\lambda\).
\item
  Normal (Gaussian) distribution:

  \[
  f(x) = \frac{1}{\sqrt{2\pi\sigma^2}} e^{-(x-\mu)^2/(2\sigma^2)}.
  \]

  इस वितरण के इंटीग्रल त्रुटि फ़ंक्शन से कनेक्ट होते हैं।
\end{enumerate}

\paragraph{यह क्यों मायने रखता
है}\label{ux92fux939-ux915ux92f-ux92eux92fux928-ux930ux916ux924-ux939-31}

\begin{itemize}
\tightlist
\item
  इंटीग्रल संभावनाओं को वक्र के अंतर्गत क्षेत्रों में बदल देते हैं।
\item
  अपेक्षा और विचरण गणना को औसत और परिवर्तनशीलता से जोड़ते हैं।
\item
  अधिकांश वास्तविक दुनिया डेटा मॉडल (वित्त, भौतिकी, जीवविज्ञान, एआई) इन निरंतर
  संभाव्यता वितरण का उपयोग करते हैं।
\end{itemize}

\paragraph{\texorpdfstring{त्वरित अभ्यास1. \([0,2]\) पर
\(f(x) = \tfrac{1}{2}\) के लिए, \(P(0.5 \leq X \leq 1.5)\) की गणना
करें।}{त्वरित अभ्यास1. {[}0,2{]} पर f(x) = \textbackslash tfrac\{1\}\{2\} के लिए, P(0.5 \textbackslash leq X \textbackslash leq 1.5) की गणना करें।}}\label{ux924ux935ux930ux924-ux905ux92dux92fux9381.-02-ux92aux930-fx-tfrac12-ux915-ux932ux90f-p0.5-leq-x-leq-1.5-ux915-ux917ux923ux928-ux915ux930}

\begin{enumerate}
\def\labelenumi{\arabic{enumi}.}
\setcounter{enumi}{1}
\tightlist
\item
  \(\lambda = 2\) के साथ घातीय वितरण के लिए, \(E[X]\) की गणना करें।
\item
  दिखाएँ कि मानक सामान्य वक्र के अंतर्गत कुल क्षेत्रफल 1 के बराबर है।
\item
  \([3,7]\) पर एकसमान वितरण का माध्य ज्ञात कीजिए।
\item
  बताएं कि निरंतर चरों के लिए संभावनाओं की गणना योगों के बजाय अभिन्नों के साथ क्यों
  की जाती है।
\end{enumerate}

\subsubsection{डी.3 कंप्यूटर विज्ञान कनेक्शन: एल्गोरिदम में टेलर
अनुमान}\label{ux921.3-ux915ux92aux92fux91fux930-ux935ux91cux91eux928-ux915ux928ux915ux936ux928-ux90fux932ux917ux930ux926ux92e-ux92e-ux91fux932ux930-ux905ux928ux92eux928}

कैलकुलस न केवल भौतिकी के लिए है - यह कंप्यूटर विज्ञान में कई उपकरणों और तकनीकों का भी
आधार है। सबसे स्पष्ट पुलों में से एक टेलर श्रृंखला के माध्यम से है, जो संख्यात्मक कंप्यूटिंग और
एल्गोरिदम में अनुमानित कार्यों के लिए कुशल तरीके प्रदान करता है।

\paragraph{कंप्यूटिंग के लिए फ़ंक्शन
अनुमान}\label{ux915ux92aux92fux91fux917-ux915-ux932ux90f-ux92bux915ux936ux928-ux905ux928ux92eux928}

कंप्यूटर अधिकांश फ़ंक्शंस को सटीक रूप से संग्रहीत या गणना नहीं कर सकता (जैसे \(e^x\),
\(\sin x\), या \(\ln x\))। इसके बजाय, वे टेलर विस्तार से प्राप्त बहुपद सन्निकटन
का उपयोग करते हैं।

उदाहरण: \(e^x\) को अनुमानित करने के लिए, मैकलॉरिन श्रृंखला को छोटा करें:

\[
e^x \approx 1 + x + \frac{x^2}{2!} + \frac{x^3}{3!}.
\]

छोटे \(x\) के लिए, यह बहुपद केवल कुछ पदों के साथ सटीक परिणाम देता है।

\paragraph{एल्गोरिदम में
दक्षता}\label{ux90fux932ux917ux930ux926ux92e-ux92e-ux926ux915ux937ux924}

\begin{itemize}
\tightlist
\item
  त्रिकोणमितीय कार्य: कैलकुलेटर और सीपीयू के लिए एल्गोरिदम अक्सर श्रृंखला विस्तार (या
  चेबीशेव बहुपद जैसे बदलाव) का उपयोग करते हैं।
\item
  घातांक/लघुगणक: टेलर विस्तार संख्यात्मक पुस्तकालयों में तेजी से अनुमान लगाने की नींव हैं।
\item
  मूल खोज: न्यूटन की विधि रैखिक सन्निकटन पर आधारित है, जो टेलर श्रृंखला (प्रथम
  व्युत्पन्न) का प्रत्यक्ष अनुप्रयोग है।
\end{itemize}

\paragraph{संख्यात्मक
विश्लेषण}\label{ux938ux916ux92fux924ux92eux915-ux935ux936ux932ux937ux923}

त्रुटि विश्लेषण में टेलर विस्तार केंद्रीय हैं:

\begin{itemize}
\item
  शेष सूत्र का उपयोग करके त्रुटि पद का अनुमान लगाना:

  \[
  R_n(x) = \frac{f^{(n+1)}(c)}{(n+1)!}(x-a)^{n+1}.
  \]
\item
  यह हमें बताता है कि किसी दी गई सटीकता के लिए कितने शब्दों की आवश्यकता है।
\end{itemize}

\paragraph{मशीन लर्निंग
कनेक्शन}\label{ux92eux936ux928-ux932ux930ux928ux917-ux915ux928ux915ux936ux928}

\begin{itemize}
\tightlist
\item
  ग्रेडिएंट-आधारित अनुकूलन (ग्रेडिएंट डिसेंट की तरह) मापदंडों को कुशलतापूर्वक अद्यतन करने
  के लिए डेरिवेटिव का उपयोग करता है।
\item
  सक्रियण फ़ंक्शन (जैसे \(\tanh x\) या \(\sigma(x)=1/(1+e^{-x})\)) अक्सर गति के
  लिए बहुपद या टुकड़े-टुकड़े फ़ंक्शन द्वारा अनुमानित होते हैं।
\item
  श्रृंखला सन्निकटन सीमित वातावरण में प्रशिक्षण और अनुमान को गति दे सकते हैं।
\end{itemize}

\paragraph{यह क्यों मायने रखता
है}\label{ux92fux939-ux915ux92f-ux92eux92fux928-ux930ux916ux924-ux939-32}

\begin{itemize}
\tightlist
\item
  टेलर सन्निकटन निरंतर गणित को असतत कंप्यूटिंग के साथ जोड़ता है।
\item
  वे दिखाते हैं कि एल्गोरिदम, संख्यात्मक तरीकों और मशीन लर्निंग में कैलकुलस अवधारणाओं का
  उपयोग कैसे किया जाता है।
\item
  अनुमानों को समझने से गणनाओं के लिए कंप्यूटर पर निर्भर होने पर होने वाले नुकसान से बचने
  में मदद मिलती है।
\end{itemize}

\paragraph{त्वरित अभ्यास}\label{ux924ux935ux930ux924-ux905ux92dux92fux938}

\begin{enumerate}
\def\labelenumi{\arabic{enumi}.}
\tightlist
\item
  मैकलॉरिन श्रृंखला के पहले तीन शब्दों का उपयोग करके अनुमानित \(\sin(0.1)\)।2.
  डिग्री-3 बहुपद के साथ \(e^1\) का अनुमान लगाने में त्रुटि का अनुमान लगाने के लिए शेष
  पद का उपयोग करें।
\item
  बताएं कि न्यूटन की विधि टेलर के प्रमेय का उपयोग कैसे करती है।
\item
  कंप्यूटर फ़ंक्शंस के लिए सटीक सूत्रों के बजाय बहुपद सन्निकटन को क्यों प्राथमिकता दे सकते
  हैं?
\item
  मशीन लर्निंग में, अनुकूलन के लिए व्युत्पन्न (ग्रेडिएंट) इतना महत्वपूर्ण क्यों है?
\end{enumerate}




\end{document}
