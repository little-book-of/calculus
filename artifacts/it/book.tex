% Options for packages loaded elsewhere
\PassOptionsToPackage{unicode}{hyperref}
\PassOptionsToPackage{hyphens}{url}
\PassOptionsToPackage{dvipsnames,svgnames,x11names}{xcolor}
%
\documentclass[
  letterpaper,
  DIV=11,
  numbers=noendperiod]{scrartcl}

\usepackage{amsmath,amssymb}
\usepackage{iftex}
\ifPDFTeX
  \usepackage[T1]{fontenc}
  \usepackage[utf8]{inputenc}
  \usepackage{textcomp} % provide euro and other symbols
\else % if luatex or xetex
  \usepackage{unicode-math}
  \defaultfontfeatures{Scale=MatchLowercase}
  \defaultfontfeatures[\rmfamily]{Ligatures=TeX,Scale=1}
\fi
\usepackage{lmodern}
\ifPDFTeX\else  
    % xetex/luatex font selection
\fi
% Use upquote if available, for straight quotes in verbatim environments
\IfFileExists{upquote.sty}{\usepackage{upquote}}{}
\IfFileExists{microtype.sty}{% use microtype if available
  \usepackage[]{microtype}
  \UseMicrotypeSet[protrusion]{basicmath} % disable protrusion for tt fonts
}{}
\makeatletter
\@ifundefined{KOMAClassName}{% if non-KOMA class
  \IfFileExists{parskip.sty}{%
    \usepackage{parskip}
  }{% else
    \setlength{\parindent}{0pt}
    \setlength{\parskip}{6pt plus 2pt minus 1pt}}
}{% if KOMA class
  \KOMAoptions{parskip=half}}
\makeatother
\usepackage{xcolor}
\setlength{\emergencystretch}{3em} % prevent overfull lines
\setcounter{secnumdepth}{-\maxdimen} % remove section numbering
% Make \paragraph and \subparagraph free-standing
\makeatletter
\ifx\paragraph\undefined\else
  \let\oldparagraph\paragraph
  \renewcommand{\paragraph}{
    \@ifstar
      \xxxParagraphStar
      \xxxParagraphNoStar
  }
  \newcommand{\xxxParagraphStar}[1]{\oldparagraph*{#1}\mbox{}}
  \newcommand{\xxxParagraphNoStar}[1]{\oldparagraph{#1}\mbox{}}
\fi
\ifx\subparagraph\undefined\else
  \let\oldsubparagraph\subparagraph
  \renewcommand{\subparagraph}{
    \@ifstar
      \xxxSubParagraphStar
      \xxxSubParagraphNoStar
  }
  \newcommand{\xxxSubParagraphStar}[1]{\oldsubparagraph*{#1}\mbox{}}
  \newcommand{\xxxSubParagraphNoStar}[1]{\oldsubparagraph{#1}\mbox{}}
\fi
\makeatother


\providecommand{\tightlist}{%
  \setlength{\itemsep}{0pt}\setlength{\parskip}{0pt}}\usepackage{longtable,booktabs,array}
\usepackage{calc} % for calculating minipage widths
% Correct order of tables after \paragraph or \subparagraph
\usepackage{etoolbox}
\makeatletter
\patchcmd\longtable{\par}{\if@noskipsec\mbox{}\fi\par}{}{}
\makeatother
% Allow footnotes in longtable head/foot
\IfFileExists{footnotehyper.sty}{\usepackage{footnotehyper}}{\usepackage{footnote}}
\makesavenoteenv{longtable}
\usepackage{graphicx}
\makeatletter
\newsavebox\pandoc@box
\newcommand*\pandocbounded[1]{% scales image to fit in text height/width
  \sbox\pandoc@box{#1}%
  \Gscale@div\@tempa{\textheight}{\dimexpr\ht\pandoc@box+\dp\pandoc@box\relax}%
  \Gscale@div\@tempb{\linewidth}{\wd\pandoc@box}%
  \ifdim\@tempb\p@<\@tempa\p@\let\@tempa\@tempb\fi% select the smaller of both
  \ifdim\@tempa\p@<\p@\scalebox{\@tempa}{\usebox\pandoc@box}%
  \else\usebox{\pandoc@box}%
  \fi%
}
% Set default figure placement to htbp
\def\fps@figure{htbp}
\makeatother

\KOMAoption{captions}{tableheading}
\makeatletter
\@ifpackageloaded{caption}{}{\usepackage{caption}}
\AtBeginDocument{%
\ifdefined\contentsname
  \renewcommand*\contentsname{Indice}
\else
  \newcommand\contentsname{Indice}
\fi
\ifdefined\listfigurename
  \renewcommand*\listfigurename{Elenco delle Figure}
\else
  \newcommand\listfigurename{Elenco delle Figure}
\fi
\ifdefined\listtablename
  \renewcommand*\listtablename{Elenco delle Tabelle}
\else
  \newcommand\listtablename{Elenco delle Tabelle}
\fi
\ifdefined\figurename
  \renewcommand*\figurename{Figura}
\else
  \newcommand\figurename{Figura}
\fi
\ifdefined\tablename
  \renewcommand*\tablename{Tabella}
\else
  \newcommand\tablename{Tabella}
\fi
}
\@ifpackageloaded{float}{}{\usepackage{float}}
\floatstyle{ruled}
\@ifundefined{c@chapter}{\newfloat{codelisting}{h}{lop}}{\newfloat{codelisting}{h}{lop}[chapter]}
\floatname{codelisting}{Lista}
\newcommand*\listoflistings{\listof{codelisting}{Elenco degli Elenchi}}
\makeatother
\makeatletter
\makeatother
\makeatletter
\@ifpackageloaded{caption}{}{\usepackage{caption}}
\@ifpackageloaded{subcaption}{}{\usepackage{subcaption}}
\makeatother

\ifLuaTeX
\usepackage[bidi=basic]{babel}
\else
\usepackage[bidi=default]{babel}
\fi
\babelprovide[main,import]{italian}
% get rid of language-specific shorthands (see #6817):
\let\LanguageShortHands\languageshorthands
\def\languageshorthands#1{}
\usepackage{bookmark}

\IfFileExists{xurl.sty}{\usepackage{xurl}}{} % add URL line breaks if available
\urlstyle{same} % disable monospaced font for URLs
\hypersetup{
  pdflang={it},
  colorlinks=true,
  linkcolor={blue},
  filecolor={Maroon},
  citecolor={Blue},
  urlcolor={Blue},
  pdfcreator={LaTeX via pandoc}}


\author{}
\date{}

\begin{document}


\section{Il piccolo libro di calcolo}\label{il-piccolo-libro-di-calcolo}

Un'introduzione concisa e adatta ai principianti alle idee fondamentali
del calcolo infinitesimale.

\subsection{Formati}\label{formati}

\begin{itemize}
\tightlist
\item
  \href{../artifacts/it/book.pdf}{Download PDF} -- versione pronta per
  la stampa
\item
  \href{../artifacts/it/book.epub}{Download EPUB} -- compatibile con gli
  e-reader
\item
  \href{../artifacts/it/book.tex}{View LaTeX} -- Origine lattice
\end{itemize}

\section{Parte 1. Limiti e derivati}\label{parte-1.-limiti-e-derivati}

\section{Capitolo 1. Funzioni e
limiti}\label{capitolo-1.-funzioni-e-limiti}

\subsection{1.1 Funzioni}\label{funzioni}

Una funzione è uno degli oggetti più basilari della matematica.
Fondamentalmente, una funzione è una regola che accetta un input e
produce esattamente un output. Le funzioni ci consentono di descrivere
relazioni, modellare fenomeni del mondo reale e costruire l'intero
meccanismo del calcolo.

\subsubsection{Definizione}\label{definizione}

Formalmente, viene scritta una funzione \(f\) da un insieme \(X\)
(chiamato dominio) a un insieme \(Y\) (chiamato codominio)

\[
f : X \to Y.
\]

Per ogni elemento \(x \in X\), esiste un elemento univoco
\(f(x) \in Y\). Il valore \(f(x)\) è chiamato immagine di \(x\) in
\(f\).

Se \(y = f(x)\), allora \(y\) è l'output corrispondente all'input \(x\).
L'insieme di tutti gli output che effettivamente appaiono è chiamato
intervallo (un sottoinsieme del codominio).

\subsubsection{Esempi}\label{esempi}

\begin{enumerate}
\def\labelenumi{\arabic{enumi}.}
\item
  La funzione \(f(x) = x^2\) associa ogni numero reale \(x\) al suo
  quadrato.

  \begin{itemize}
  \tightlist
  \item
    Dominio: tutti i numeri reali \(\mathbb{R}\).
  \item
    Codominio: tutti i numeri reali \(\mathbb{R}\).
  \item
    Intervallo: tutti i numeri reali non negativi \([0, \infty)\).
  \end{itemize}
\item
  La funzione \(g(x) = \dfrac{1}{x}\) assegna a ciascun numero reale
  diverso da zero il suo reciproco.

  \begin{itemize}
  \tightlist
  \item
    Dominio: \(\mathbb{R} \setminus \{0\}\).
  \item
    Intervallo: \(\mathbb{R} \setminus \{0\}\).
  \end{itemize}
\item
  Un esempio reale: Sia \(T(t)\) la temperatura esterna (in °C) al tempo
  \(t\) (in ore). Questa è una funzione che va dall'\,''ora del giorno''
  alla ``temperatura''.
\end{enumerate}

\subsubsection{Modi di rappresentare le
funzioni}\label{modi-di-rappresentare-le-funzioni}

Le funzioni possono essere rappresentate in diversi modi utili:

\begin{itemize}
\tightlist
\item
  Formule: ad esempio, \(f(x) = \sin x + x^2\).- Grafici: tracciamento
  di tutti i punti \((x, f(x))\) nel piano delle coordinate.
\item
  Tabelle: abbinamento di input e output per insiemi discreti di dati.
\item
  Descrizioni verbali: ``Assegna a ciascuno studente il suo voto''.
\end{itemize}

Ciascuna rappresentazione evidenzia aspetti diversi della stessa
funzione.

\subsubsection{Terminologia}\label{terminologia}

\begin{itemize}
\tightlist
\item
  Variabile indipendente: l'input (normalmente scritto \(x\)).
\item
  Variabile dipendente: l'output (solitamente scritto \(y\), dove
  \(y = f(x)\)).
\item
  Notazione della funzione: \(f(x)\) viene letto ``\(f\) di \(x\).''
\end{itemize}

\subsubsection{Perché le funzioni sono importanti nel calcolo
infinitesimale}\label{perchuxe9-le-funzioni-sono-importanti-nel-calcolo-infinitesimale}

Il calcolo infinitesimale è lo studio di come cambiano le funzioni. I
derivati \hspace{0pt}\hspace{0pt}misurano i tassi di cambiamento
istantanei, mentre gli integrali misurano gli effetti accumulati. Per
padroneggiare queste idee, abbiamo prima bisogno di una solida
comprensione di cosa sono le funzioni e come si comportano.

\subsubsection{Esercizi}\label{esercizi}

\begin{enumerate}
\def\labelenumi{\arabic{enumi}.}
\item
  Per la funzione \(f(x) = 3x - 2\):

  \begin{itemize}
  \tightlist
  \item
    Trova il dominio, il codominio e l'intervallo.
  \end{itemize}
\item
  Per quali ingressi è definita la funzione \(h(x) = \sqrt{x-1}\)? Qual
  è la sua portata?
\item
  Fornisci un esempio reale di una funzione della tua vita quotidiana.
  Indicare chiaramente il dominio e il codominio.
\item
  Disegna il grafico di \(f(x) = |x|\). Qual è la portata?
\item
  Supponiamo che \(g(x) = \dfrac{1}{x^2+1}\). Spiega perché il suo
  intervallo è l'intervallo \((0, 1]\).
\end{enumerate}

\subsection{1.2 Grafici e
trasformazioni}\label{grafici-e-trasformazioni}

Una funzione può essere compresa non solo dalle formule ma anche dal suo
grafico. Il grafico di una funzione \(f\) è l'insieme di tutte le coppie
ordinate \((x, f(x))\), dove \(x\) appartiene al dominio di \(f\).
Tracciare queste coppie nel piano delle coordinate fornisce un'immagine
di come si comporta la funzione.

\subsubsection{Grafici di base}\label{grafici-di-base}

Alcuni grafici sono così fondamentali che dovrebbero essere memorizzati:

\begin{itemize}
\tightlist
\item
  \(f(x) = x\): una linea retta passante per l'origine.
\item
  \(f(x) = x^2\): una parabola che si apre verso l'alto.
\item
  \(f(x) = |x|\): un grafico a forma di ``V''.
\item
  \(f(x) = \frac{1}{x}\): un'iperbole con due rami.- \(f(x) = \sin x\):
  una curva periodica a forma di onda.
\end{itemize}

Questi servono come elementi costitutivi per funzioni più complicate.

\subsubsection{Trasformazioni}\label{trasformazioni}

I grafici possono essere spostati, allungati o riflessi utilizzando
semplici regole:

\begin{enumerate}
\def\labelenumi{\arabic{enumi}.}
\item
  Spostamenti verticali: l'aggiunta di una costante sposta il grafico
  verso l'alto o verso il basso.

  \[
  y = f(x) + c \quad \text{is } f(x) \text{ shifted upward by } c.
  \]
\item
  Spostamenti orizzontali: l'aggiunta all'interno dell'argomento sposta
  il grafico a sinistra o a destra.

  \[
  y = f(x - c) \quad \text{is } f(x) \text{ shifted right by } c.
  \]
\item
  Ridimensionamento verticale: moltiplicando per una costante si allunga
  o si comprime il grafico verticalmente.

  \[
  y = a f(x), \quad a > 1 \text{ stretches; } 0 < a < 1 \text{ compresses.}
  \]
\item
  Ridimensionamento orizzontale: la moltiplicazione all'interno
  dell'argomento allunga o comprime il grafico orizzontalmente.

  \[
  y = f(bx), \quad b > 1 \text{ compresses toward the } y\text{-axis}.
  \]
\item
  Riflessioni:

  \begin{itemize}
  \tightlist
  \item
    \(y = -f(x)\): riflessione lungo l'asse \(x\).
  \item
    \(y = f(-x)\): riflessione lungo l'asse \(y\).
  \end{itemize}
\end{enumerate}

\subsubsection{Combinazione di
trasformazioni}\label{combinazione-di-trasformazioni}

I grafici complessi spesso derivano dalla combinazione di diverse
trasformazioni in sequenza. Ad esempio:

\[
y = 2(x-1)^2 + 3
\]

si ottiene prendendo la parabola \(y = x^2\), spostandola verso destra
di 1, allungandola verticalmente di 2 e spostandola verso l'alto di 3.

\subsubsection{Esercizi}\label{esercizi-1}

\begin{enumerate}
\def\labelenumi{\arabic{enumi}.}
\tightlist
\item
  Disegna il grafico di \(y = (x+2)^2 - 1\). Identificare la sequenza di
  trasformazioni da \(y = x^2\).
\item
  Cosa succede al grafico di \(y = f(x)\) se sostituiamo \(x\) con
  \(-x\)? Provalo con \(f(x) = \sqrt{x}\).
\item
  Descrivi le trasformazioni che trasformano \(y = \sin x\) in
  \(y = 3\sin(x - \pi/4)\).
\item
  Disegna il grafico di \(y = |x-1| + 2\). Indica il vertice e la
  pendenza di ciascun ramo.
\item
  Per \(y = \frac{1}{x-2}\), spiegare come è stato trasformato il
  grafico di \(y = \frac{1}{x}\).
\end{enumerate}

\subsection{1.3 Idea intuitiva di limiteIn molte situazioni, il valore
di una funzione in un punto è meno importante dei valori che assume
vicino a quel punto. Il concetto di limite cattura questa
idea.}\label{idea-intuitiva-di-limitein-molte-situazioni-il-valore-di-una-funzione-in-un-punto-uxe8-meno-importante-dei-valori-che-assume-vicino-a-quel-punto.-il-concetto-di-limite-cattura-questa-idea.}

\subsubsection{Avvicinarsi a un valore}\label{avvicinarsi-a-un-valore}

Immagina di camminare verso un muro. Ancor prima di toccarlo, ti
avvicini sempre di più. Allo stesso modo, quando \(x\) si avvicina a un
numero \(a\), i valori di \(f(x)\) possono avvicinarsi a un numero
\(L\). Diciamo allora:

\[
\lim_{x \to a} f(x) = L.
\]

Ciò esprime l'idea che \(f(x)\) può essere avvicinato quanto vogliamo a
\(L\), semplicemente portando \(x\) abbastanza vicino a \(a\).

\subsubsection{Esempi}\label{esempi-1}

\begin{enumerate}
\def\labelenumi{\arabic{enumi}.}
\item
  Per \(f(x) = 2x + 3\): Come \(x \to 1\), \(f(x) \to 5\).
\item
  Per \(f(x) = \dfrac{\sin x}{x}\): Come \(x \to 0\), la funzione si
  avvicina a 1, anche se \(f(0)\) non è definito.
\item
  Per \(f(x) = \dfrac{1}{x}\): Come \(x \to 0^+\) (avvicinandosi da
  destra), \(f(x) \to +\infty\). Come \(x \to 0^-\) (avvicinandosi da
  sinistra), \(f(x) \to -\infty\). Poiché i comportamenti sinistro e
  destro differiscono, il limite a 0 non esiste.
\end{enumerate}

\subsubsection{Importanza dei limiti}\label{importanza-dei-limiti}

\begin{itemize}
\tightlist
\item
  Ci permettono di definire funzioni nei punti in cui non sono
  originariamente definite.
\item
  Catturano il comportamento in prossimità di discontinuità e
  singolarità.
\item
  Costituiscono la base per i derivati \hspace{0pt}\hspace{0pt}(tassi di
  variazione istantanei) e gli integrali (aree come limiti di somme).
\end{itemize}

\subsubsection{Limiti unilaterali}\label{limiti-unilaterali}

A volte il comportamento di sinistra e di destra deve essere studiato
separatamente:

\[
\lim_{x \to a^-} f(x), \quad \lim_{x \to a^+} f(x).
\]

Se entrambi concordano, allora esiste il limite bilaterale.

\subsubsection{Esercizi}\label{esercizi-2}

\begin{enumerate}
\def\labelenumi{\arabic{enumi}.}
\tightlist
\item
  Calcola \(\lim_{x \to 2} (3x^2 - x)\).
\item
  Cos'è \(\lim_{x \to 0} \frac{\sin x}{x}\)? Usa l'intuizione dal
  grafico di \(\sin x\).
\item
  Valuta \(\lim_{x \to 0} |x|/x\). Esiste il limite bilaterale?
\item
  Trova \(\lim_{x \to \infty} \frac{1}{x}\). Interpreta questo risultato
  in parole.5. Per \(f(x) = \frac{x^2-1}{x-1}\), cos'è
  \(\lim_{x \to 1} f(x)\)? Confronta con il valore di \(f(1)\).
\end{enumerate}

\subsection{1.4 Definizione formale dei
limiti}\label{definizione-formale-dei-limiti}

L'idea intuitiva di limite può essere precisata utilizzando la
definizione epsilon-delta. Questo ci fornisce un modo rigoroso per dire
che \(f(x)\) si avvicina a un valore \(L\) quando \(x\) si avvicina a
\(a\).

\subsubsection{La definizione}\label{la-definizione}

Scriviamo

\[
\lim_{x \to a} f(x) = L
\]

se vale la seguente condizione:

Per ogni \(\varepsilon > 0\) (non importa quanto piccolo), esiste un
\(\delta > 0\) tale che ogni volta

\[
0 < |x - a| < \delta,
\]

ne consegue che

\[
|f(x) - L| < \varepsilon.
\]

In parole: possiamo rendere \(f(x)\) il più vicino possibile a \(L\), a
condizione che \(x\) sia abbastanza vicino a \(a\) (ma non uguale a
\(a\)).

\subsubsection{Esempio 1: funzione
lineare}\label{esempio-1-funzione-lineare}

Per \(f(x) = 2x + 1\), mostra che \(\lim_{x \to 3} f(x) = 7\).

\begin{itemize}
\tightlist
\item
  Vogliamo \(|f(x) - 7| < \varepsilon\).
\item
  Ma \(f(x) - 7 = 2x + 1 - 7 = 2(x - 3)\).
\item
  Quindi \(|f(x) - 7| = 2|x - 3|\).
\item
  Se scegliamo \(\delta = \varepsilon / 2\), ogni volta che
  \(|x - 3| < \delta\), avremo \(|f(x) - 7| < \varepsilon\). Ciò
  dimostra il limite.
\end{itemize}

\subsubsection{Esempio 2: Funzione
reciproca}\label{esempio-2-funzione-reciproca}

Per \(f(x) = \frac{1}{x}\), considera
\(\lim_{x \to 2} f(x) = \tfrac{1}{2}\).

\begin{itemize}
\tightlist
\item
  Vogliamo \(\left|\frac{1}{x} - \frac{1}{2}\right| < \varepsilon\).
\item
  Questa disuguaglianza richiede una manipolazione algebrica, ma può
  essere soddisfatta scegliendo \(\delta\) a seconda di \(\varepsilon\).
  Il processo è più complicato, ma il principio è lo stesso.
\end{itemize}

\subsubsection{Perché è importante}\label{perchuxe9-uxe8-importante}

\begin{itemize}
\tightlist
\item
  La definizione epsilon-delta garantisce che i limiti non siano vaghi o
  basati solo sull'intuizione.
\item
  È il fondamento della continuità, delle derivate e degli integrali.
\item
  Sebbene i principianti possano trovarlo astratto, lavorare con esempi
  semplici crea familiarità.
\end{itemize}

\subsubsection{Esercizi}\label{esercizi-3}

\begin{enumerate}
\def\labelenumi{\arabic{enumi}.}
\tightlist
\item
  Utilizzando la definizione epsilon--delta, dimostrare che
  \(\lim_{x \to 4} (x+1) = 5\).2. Mostra che \(\lim_{x \to 0} 5x = 0\)
  utilizzando la definizione formale.
\item
  Spiega perché \(\lim_{x \to 0} \frac{1}{x}\) non esiste.
\item
  Per \(f(x) = x^2\), mostra che \(\lim_{x \to 2} f(x) = 4\).
\item
  Con parole tue, spiega il ruolo di \(\varepsilon\) e \(\delta\) nella
  definizione di un limite.
\end{enumerate}

\#\#1.5 Continuità

Una funzione è continua se il suo grafico può essere disegnato senza
staccare la matita dal foglio. Più precisamente, la continuità
garantisce che piccoli cambiamenti nell'input producano piccoli
cambiamenti nell'output.

\subsubsection{Definizione}\label{definizione-1}

Una funzione \(f\) è continua in un punto \(a\) se tre condizioni sono
soddisfatte:

\begin{enumerate}
\def\labelenumi{\arabic{enumi}.}
\tightlist
\item
  \(f(a)\) è definito.
\item
  \(\lim_{x \to a} f(x)\) esiste.
\item
  \(\lim_{x \to a} f(x) = f(a)\).
\end{enumerate}

Se una funzione è continua in ogni punto di un intervallo, si dice che è
continua in quell'intervallo.

\subsubsection{Esempi}\label{esempi-2}

\begin{enumerate}
\def\labelenumi{\arabic{enumi}.}
\item
  Funzioni polinomiali: funzioni come \(f(x) = x^2 + 3x - 5\) sono
  continue ovunque su \(\mathbb{R}\).
\item
  Funzioni razionali: \(f(x) = \frac{1}{x-1}\) è continuo ovunque tranne
  che in \(x = 1\), dove è indefinito.
\item
  Funzioni a tratti:

  \[
  f(x) =
  \begin{cases}
  x^2 & x < 1, \\
  2 & x = 1, \\
  x+1 & x > 1,
  \end{cases}
  \]

  Questa funzione ha un ``salto'' in \(x = 1\), quindi non è continua
  lì.
\end{enumerate}

\subsubsection{Tipi di discontinuità}\label{tipi-di-discontinuituxe0}

\begin{enumerate}
\def\labelenumi{\arabic{enumi}.}
\tightlist
\item
  Discontinuità rimovibile: un ``buco'' nel grafico. Esempio:
  \(f(x) = \frac{x^2-1}{x-1}\) in \(x=1\).
\item
  Discontinuità del salto: i limiti di sinistra e di destra sono
  diversi.
\item
  Discontinuità infinita: la funzione va a \(\pm\infty\) vicino a un
  punto, come con \(f(x) = 1/x\) vicino a \(x = 0\).
\end{enumerate}

\subsubsection{Il Teorema del Valore
Intermedio}\label{il-teorema-del-valore-intermedio}

Se una funzione è continua su un intervallo \([a, b]\), allora per
qualsiasi numero \(N\) compreso tra \(f(a)\) e \(f(b)\), esiste un
\(c \in [a, b]\) tale che \(f(c) = N\).Questa proprietà è cruciale per
dimostrare l'esistenza di radici e soluzioni alle equazioni.

\subsubsection{Esercizi}\label{esercizi-4}

\begin{enumerate}
\def\labelenumi{\arabic{enumi}.}
\tightlist
\item
  Decidi se la funzione \(f(x) = |x|\) è continua in \(x = 0\).
\item
  Identificare i punti di discontinuità per
  \(f(x) = \frac{x+2}{x^2-1}\).
\item
  Spiega perché ogni funzione polinomiale è continua ovunque.
\item
  Fornisci un esempio di funzione con discontinuità di salto. Disegnane
  il grafico.
\item
  Utilizza il Teorema del Valore Intermedio per dimostrare che
  l'equazione \(x^3 + x - 1 = 0\) ha una soluzione compresa tra 0 e 1.
\end{enumerate}

\section{Capitolo 2. Derivati}\label{capitolo-2.-derivati}

\subsection{2.1 Il derivato come tasso di
cambiamento}\label{il-derivato-come-tasso-di-cambiamento}

La derivata è una delle idee centrali del calcolo infinitesimale. Misura
il modo in cui una funzione cambia al variare del suo input, in altre
parole, il tasso di variazione dell'output rispetto all'input.

\subsubsection{Tasso medio di
variazione}\label{tasso-medio-di-variazione}

Per una funzione \(f(x)\), il tasso medio di variazione tra due punti
\(x = a\) e \(x = b\) è

\[
\frac{f(b) - f(a)}{b - a}.
\]

Questa è la pendenza della linea secante che passa per i punti
\((a, f(a))\) e \((b, f(b))\).

\subsubsection{Tasso di cambiamento
istantaneo}\label{tasso-di-cambiamento-istantaneo}

Per misurare la velocità con cui \(f(x)\) cambia in un singolo punto,
lasciamo che l'intervallo si riduca:

\[
f'(a) = \lim_{h \to 0} \frac{f(a+h) - f(a)}{h}.
\]

Questo limite, se esiste, è chiamato derivata di \(f\) in \(a\).
Geometricamente, è la pendenza della linea tangente al grafico di \(f\)
nel punto \((a, f(a))\).

\subsubsection{Notazione}\label{notazione}

\begin{itemize}
\tightlist
\item
  \(f'(x)\): notazione primaria.
\item
  \(\dfrac{dy}{dx}\): notazione Leibniz, utilizzata quando \(y = f(x)\).
\item
  \(Df(x)\): notazione dell'operatore.
\end{itemize}

Tutti questi simboli si riferiscono allo stesso concetto.

\subsubsection{Esempi}\label{esempi-3}

\begin{enumerate}
\def\labelenumi{\arabic{enumi}.}
\item
  Per \(f(x) = x^2\):

  \[
  f'(x) = \lim_{h \to 0} \frac{(x+h)^2 - x^2}{h} = \lim_{h \to 0} \frac{2xh + h^2}{h} = 2x.
  \]

  La pendenza della parabola in \(x\) è \(2x\).
\item
  Per \(f(x) = \sin x\):

  \[
  f'(x) = \cos x.
  \]3. Per \(f(x) = c\) (una costante):

  \[
  f'(x) = 0.
  \]

  Una funzione costante non cambia mai.
\end{enumerate}

\subsubsection{Interpretazione}\label{interpretazione}

\begin{itemize}
\tightlist
\item
  In fisica: se \(s(t)\) è la posizione, allora \(s'(t)\) è la velocità.
\item
  In economia: se \(C(x)\) è il costo, allora \(C'(x)\) è il costo
  marginale.
\item
  In biologia: se \(P(t)\) è la popolazione, allora \(P'(t)\) è il tasso
  di crescita.
\end{itemize}

Il derivato rende il ``cambiamento'' preciso in molti contesti.

\subsubsection{Esercizi}\label{esercizi-5}

\begin{enumerate}
\def\labelenumi{\arabic{enumi}.}
\tightlist
\item
  Calcola \(f'(x)\) per \(f(x) = 3x^2 - 2x + 1\).
\item
  Trova la pendenza della linea tangente a \(f(x) = x^3\) in \(x = 2\).
\item
  Se \(s(t) = t^2 + 2t\) rappresenta la distanza in metri, qual è la
  velocità a \(t = 5\)?
\item
  Utilizzare la definizione del limite per calcolare la derivata di
  \(f(x) = \frac{1}{x}\).
\item
  Disegna il grafico di \(y = x^2\) e traccia la linea tangente in
  \(x = 1\).
\end{enumerate}

\subsection{2.2 Regole di
differenziazione}\label{regole-di-differenziazione}

Una volta definita la derivata, abbiamo bisogno di metodi efficienti per
calcolarla. Le regole di differenziazione sono scorciatoie che ci
evitano di applicare ripetutamente la definizione di limite.

\subsubsection{La regola costante}\label{la-regola-costante}

Se \(f(x) = c\) dove \(c\) è una costante, allora

\[
f'(x) = 0.
\]

\subsubsection{La regola del potere}\label{la-regola-del-potere}

Per \(f(x) = x^n\) dove \(n\) è un numero reale,

\[
\frac{d}{dx} \big( x^n \big) = n x^{n-1}.
\]

Esempi:

\begin{itemize}
\tightlist
\item
  \(\frac{d}{dx}(x^2) = 2x\).
\item
  \(\frac{d}{dx}(x^5) = 5x^4\).
\item
  \(\frac{d}{dx}(\sqrt{x}) = \frac{1}{2\sqrt{x}}\).
\end{itemize}

\subsubsection{La regola del multiplo
costante}\label{la-regola-del-multiplo-costante}

Se \(f(x) = c \cdot g(x)\), allora

\[
f'(x) = c \cdot g'(x).
\]

\subsubsection{Le regole di somma e
differenza}\label{le-regole-di-somma-e-differenza}

\begin{itemize}
\tightlist
\item
  \((f + g)' = f' + g'\).
\item
  \((f - g)' = f' - g'\).
\end{itemize}

\subsubsection{La regola del prodotto}\label{la-regola-del-prodotto}

Per \(f(x)\) e \(g(x)\):

\[
(fg)' = f'g + fg'.
\]

Esempio: Se \(f(x) = x^2\), \(g(x) = \sin x\):

\[
(fg)' = (2x)(\sin x) + (x^2)(\cos x).
\]

\subsubsection{La regola del quoziente}\label{la-regola-del-quoziente}

Per \(f(x)\) e \(g(x)\):

\[
\left(\frac{f}{g}\right)' = \frac{f'g - fg'}{g^2}, \quad g(x) \neq 0.
\]

Esempio: Se \(f(x) = x^2\), \(g(x) = x+1\):

\[\left(\frac{x^2}{x+1}\right)' = \frac{(2x)(x+1) - (x^2)(1)}{(x+1)^2}.
\]

\subsubsection{Derivatives of Common
Functions}\label{derivatives-of-common-functions}

\begin{itemize}
\tightlist
\item
  \(\frac{d}{dx}(\sin x) = \cos x\).
\item
  \(\frac{d}{dx}(\cos x) = -\sin x\).
\item
  \(\frac{d}{dx}(e^x) = e^x\).
\item
  \(\frac{d}{dx}(\ln x) = \frac{1}{x}, \quad x > 0\).
\end{itemize}

\subsubsection{Exercises}\label{exercises}

\begin{enumerate}
\def\labelenumi{\arabic{enumi}.}
\tightlist
\item
  Differentiate \(f(x) = 7x^3 - 4x + 9\).
\item
  Use the product rule to find the derivative of \(f(x) = x^2 e^x\).
\item
  Apply the quotient rule to \(f(x) = \frac{\sin x}{x}\).
\item
  Compute \(\frac{d}{dx}(\ln(x^2))\) using the chain of rules.
\item
  Show that the derivative of \(f(x) = \frac{1}{x}\) is
  \(-\frac{1}{x^2}\).
\end{enumerate}

\subsection{2.3 The Chain Rule}\label{the-chain-rule}

Often, functions are built by combining simpler functions together. To
differentiate such composite functions, we use the chain rule.

\subsubsection{The Rule}\label{the-rule}

If \(y = f(g(x))\), then

\[
\frac{dy}{dx} = f'(g(x)) \cdot g'(x).
\]

In words: differentiate the outer function, keep the inside unchanged,
then multiply by the derivative of the inside.

\subsubsection{Examples}\label{examples}

\begin{enumerate}
\def\labelenumi{\arabic{enumi}.}
\item
  Square of a linear function

  \[
  y = (3x+2)^2
  \]

  Outer function: \(f(u) = u^2\), inner function: \(g(x) = 3x+2\).

  \[
  y' = 2(3x+2) \cdot 3 = 6(3x+2).
  \]
\item
  Exponential with quadratic inside

  \[
  y = e^{x^2}
  \]

  Outer function: \(f(u) = e^u\), inner function: \(g(x) = x^2\).

  \[
  y' = e^{x^2} \cdot 2x = 2x e^{x^2}.
  \]
\item
  Logarithm with root inside

  \[
  y = \ln(\sqrt{x})
  \]

  Outer: \(f(u) = \ln u\), inner: \(g(x) = \sqrt{x}\).

  \[
  y' = \frac{1}{\sqrt{x}} \cdot \frac{1}{2\sqrt{x}} = \frac{1}{2x}.
  \]
\end{enumerate}

\subsubsection{Generalized Chain Rule}\label{generalized-chain-rule}

For multiple nested functions \(y = f(g(h(x)))\):

\[
\frac{dy}{dx} = f'(g(h(x))) \cdot g'(h(x)) \cdot h'(x).
\]

Ciò si estende naturalmente alle composizioni più profonde.

\subsubsection{Perché la regola della catena è importante- Gestisce
quasi tutti i modelli del mondo reale in cui una quantità dipende
indirettamente da
un'altra.}\label{perchuxe9-la-regola-della-catena-uxe8-importante--gestisce-quasi-tutti-i-modelli-del-mondo-reale-in-cui-una-quantituxe0-dipende-indirettamente-da-unaltra.}

\begin{itemize}
\tightlist
\item
  Collega il calcolo con la fisica (ad esempio, la velocità dipende dal
  tempo attraverso la posizione).
\item
  È essenziale nella differenziazione implicita e negli argomenti
  avanzati.
\end{itemize}

\subsubsection{Esercizi}\label{esercizi-6}

\begin{enumerate}
\def\labelenumi{\arabic{enumi}.}
\tightlist
\item
  Differenziare \(y = (5x^2 + 1)^3\).
\item
  Trova \(\frac{d}{dx}(\sin(3x))\).
\item
  Calcola \(\frac{d}{dx}(\ln(1+x^2))\).
\item
  Differenziare \(y = \cos^2(x)\).
\item
  Applicare la regola della catena generalizzata a
  \(y = e^{\sin(x^2)}\).
\end{enumerate}

\subsection{2.4 Differenziazione
implicita}\label{differenziazione-implicita}

Non tutte le funzioni sono fornite nella forma \(y = f(x)\). A volte
\(x\) e \(y\) sono correlati da un'equazione e risolvere esplicitamente
per \(y\) è difficile o impossibile. In questi casi, usiamo la
differenziazione implicita.

\subsubsection{L'idea}\label{lidea}

Se un'equazione coinvolge sia \(x\) che \(y\), possiamo differenziare
entrambi i lati rispetto a \(x\), trattando \(y\) come una funzione di
\(x\). Ogni volta che differenziamo un termine che coinvolge \(y\),
moltiplichiamo per \(\frac{dy}{dx}\).

\subsubsection{Esempio 1: un cerchio}\label{esempio-1-un-cerchio}

Equazione:

\[
x^2 + y^2 = 25
\]

Differenziare rispetto a \(x\):

\[
2x + 2y \frac{dy}{dx} = 0.
\]

Risolvi per \(\frac{dy}{dx}\):

\[
\frac{dy}{dx} = -\frac{x}{y}.
\]

Questo dà la pendenza della tangente alla circonferenza in ogni punto.

\subsubsection{Esempio 2: un prodotto di
variabili}\label{esempio-2-un-prodotto-di-variabili}

Equazione:

\[
xy = 1
\]

Differenziare:

\[
x \frac{dy}{dx} + y = 0.
\]

quindi,

\[
\frac{dy}{dx} = -\frac{y}{x}.
\]

\subsubsection{Esempio 3: Relazione
trigonometrica}\label{esempio-3-relazione-trigonometrica}

Equazione:

\[
\sin(xy) = x
\]

Differenziare:

\[
\cos(xy) \cdot \Big(y + x\frac{dy}{dx}\Big) = 1.
\]

Risolvi per \(\frac{dy}{dx}\):

\[
\frac{dy}{dx} = \frac{1 - y\cos(xy)}{x\cos(xy)}.
\]

\subsubsection{Perché la differenziazione implicita è
utile}\label{perchuxe9-la-differenziazione-implicita-uxe8-utile}

\begin{itemize}
\tightlist
\item
  Molte curve importanti (cerchi, ellissi, iperboli) sono naturalmente
  definite implicitamente.
\item
  Ci consente di differenziare le equazioni senza prima risolvere
  \(y\).- È un passo fondamentale in argomenti più avanzati come i tassi
  correlati e le equazioni differenziali.
\end{itemize}

\subsubsection{Esercizi}\label{esercizi-7}

\begin{enumerate}
\def\labelenumi{\arabic{enumi}.}
\tightlist
\item
  Per la curva \(x^2 + xy + y^2 = 7\), trovare \(\frac{dy}{dx}\).
\item
  Differenziare \(\cos(x) + \cos(y) = 1\) in modo implicito.
\item
  Trova la pendenza della linea tangente a \(x^3 + y^3 = 9\) nel punto
  \((1, 2)\).
\item
  Dato \(x^2 + y^2 = 10\), calcola \(\frac{dy}{dx}\) quando
  \((x, y) = (1, 3)\).
\item
  Differenziare \(e^{xy} = x + y\) per trovare \(\frac{dy}{dx}\).
\end{enumerate}

\subsection{2.5 Derivati di ordine
superiore}\label{derivati-di-ordine-superiore}

Finora abbiamo studiato la derivata prima, che misura la velocità di
variazione di una funzione. Ma anche i derivati
\hspace{0pt}\hspace{0pt}stessi possono essere differenziati, dando
origine a derivati \hspace{0pt}\hspace{0pt}di ordine superiore.

\subsubsection{Definizione}\label{definizione-2}

\begin{itemize}
\item
  La derivata seconda di \(f\) è la derivata della derivata:

  \[
  f''(x) = \frac{d}{dx}\left(f'(x)\right).
  \]
\item
  Più in generale, la derivata \(n\)-esima è scritta come

  \[
  f^{(n)}(x) = \frac{d^n}{dx^n} f(x).
  \]
\end{itemize}

\subsubsection{Esempi}\label{esempi-4}

\begin{enumerate}
\def\labelenumi{\arabic{enumi}.}
\item
  \(f(x) = x^3\)

  \begin{itemize}
  \tightlist
  \item
    Derivata prima: \(f'(x) = 3x^2\).
  \item
    Derivata seconda: \(f''(x) = 6x\).
  \item
    Derivata terza: \(f^{(3)}(x) = 6\).
  \item
    Derivata quarta: \(f^{(4)}(x) = 0\).
  \end{itemize}
\item
  \(f(x) = \sin x\)

  \begin{itemize}
  \tightlist
  \item
    \(f'(x) = \cos x\).
  \item
    \(f''(x) = -\sin x\).
  \item
    \(f^{(3)}(x) = -\cos x\).
  \item
    \(f^{(4)}(x) = \sin x\). Le derivate si ripetono in un ciclo di
    lunghezza 4.
  \end{itemize}
\item
  \(f(x) = e^x\)

  \begin{itemize}
  \tightlist
  \item
    Ogni derivata è \(e^x\).
  \end{itemize}
\end{enumerate}

\subsubsection{Applicazioni}\label{applicazioni}

\begin{itemize}
\item
  Concavità: il segno di \(f''(x)\) indica se il grafico di \(f\) è
  concavo verso l'alto (\(f'' > 0\)) o concavo verso il basso
  (\(f'' < 0\)).
\item
  Punti di flesso: punti in cui cambia \(f''(x) = 0\) e la concavità.
\item
  Movimento: in fisica, se \(s(t)\) è la posizione:

  \begin{itemize}
  \tightlist
  \item
    \(s'(t)\) = velocità,
  \item
    \(s''(t)\) = accelerazione,
  \item
    \(s^{(3)}(t)\) = jerk (velocità di variazione dell'accelerazione).
  \end{itemize}
\item
  Approssimazioni: le derivate di ordine superiore compaiono nelle serie
  di Taylor, utilizzate per approssimare le funzioni.\#\#\# Esercizi
\end{itemize}

\begin{enumerate}
\def\labelenumi{\arabic{enumi}.}
\tightlist
\item
  Calcola le prime quattro derivate di \(f(x) = \cos x\).
\item
  Trova \(f''(x)\) per \(f(x) = x^4 - 2x^2 + 3\).
\item
  Per \(f(x) = e^{2x}\), mostra che \(f^{(n)}(x) = 2^n e^{2x}\).
\item
  Determinare gli intervalli in cui \(f(x) = x^3 - 3x\) è concavo verso
  l'alto e concavo verso il basso.
\item
  Se \(s(t) = t^3 - 6t^2 + 9t\), trova la velocità e l'accelerazione in
  \(t = 2\).
\end{enumerate}

\section{Capitolo 3. Applicazioni dei
derivati}\label{capitolo-3.-applicazioni-dei-derivati}

\subsection{3.1 Tangenti e normali}\label{tangenti-e-normali}

Una delle prime applicazioni delle derivate è trovare le equazioni delle
rette tangenti e normali a una curva. Queste linee catturano la
geometria locale di una funzione in un dato punto.

\subsubsection{Retta tangente}\label{retta-tangente}

La linea tangente a una curva \(y = f(x)\) in un punto \((a, f(a))\) è
la linea che ``tocca'' semplicemente il grafico in quel punto e ha la
stessa pendenza della curva.

La pendenza della retta tangente è data dalla derivata:

\[
m_{\text{tangent}} = f'(a).
\]

Pertanto, l'equazione della linea tangente in \((a, f(a))\) è

\[
y - f(a) = f'(a)(x - a).
\]

\subsubsection{Linea normale}\label{linea-normale}

La retta normale è perpendicolare alla retta tangente nello stesso
punto. La sua pendenza è il reciproco negativo della pendenza tangente:

\[
m_{\text{normal}} = -\frac{1}{f'(a)}.
\]

Quindi l'equazione della retta normale è

\[
y - f(a) = -\frac{1}{f'(a)} (x - a), \quad f'(a) \neq 0.
\]

\subsubsection{Esempi}\label{esempi-5}

\begin{enumerate}
\def\labelenumi{\arabic{enumi}.}
\item
  \(f(x) = x^2\) a \(x = 1\).

  \begin{itemize}
  \tightlist
  \item
    \(f(1) = 1\), \(f'(x) = 2x\), quindi \(f'(1) = 2\).
  \item
    Tangente: \(y - 1 = 2(x - 1)\) o \(y = 2x - 1\).
  \item
    Normale: pendenza = \(-\tfrac{1}{2}\), quindi l'equazione è
    \(y - 1 = -\tfrac{1}{2}(x - 1)\).
  \end{itemize}
\item
  \(f(x) = \sin x\) a \(x = \tfrac{\pi}{4}\).

  \begin{itemize}
  \tightlist
  \item
    \(f(\tfrac{\pi}{4}) = \tfrac{\sqrt{2}}{2}\),
    \(f'(\tfrac{\pi}{4}) = \cos(\tfrac{\pi}{4}) = \tfrac{\sqrt{2}}{2}\).
  \item
    Tangente:
    \(y - \tfrac{\sqrt{2}}{2} = \tfrac{\sqrt{2}}{2}(x - \tfrac{\pi}{4})\).
  \end{itemize}
\end{enumerate}

\subsubsection{Perché le tangenti e le normali sono importanti- Le
tangenti approssimano localmente la curva (approssimazione
lineare).}\label{perchuxe9-le-tangenti-e-le-normali-sono-importanti--le-tangenti-approssimano-localmente-la-curva-approssimazione-lineare.}

\begin{itemize}
\tightlist
\item
  Le normali sono utili in geometria, ottica (riflessione/rifrazione) e
  meccanica (direzioni della forza).
\item
  Entrambi svolgono un ruolo negli studi di ottimizzazione e curvatura.
\end{itemize}

\subsubsection{Esercizi}\label{esercizi-8}

\begin{enumerate}
\def\labelenumi{\arabic{enumi}.}
\tightlist
\item
  Trova le linee tangente e normale a \(y = x^3\) in \(x = 2\).
\item
  Determinare le linee tangente e normale a \(y = e^x\) in \(x = 0\).
\item
  Per \(y = \ln x\), calcolare la linea tangente in \(x = 1\).
\item
  Un cerchio è dato da \(x^2 + y^2 = 9\). Utilizza la differenziazione
  implicita per trovare la pendenza della tangente in \((0,3)\).
\item
  Disegna il grafico di \(y = \sqrt{x}\) e disegna le linee tangente e
  normale in \(x = 4\).
\end{enumerate}

\subsection{3.2 Tariffe correlate}\label{tariffe-correlate}

In molti problemi del mondo reale, due o più quantità cambiano rispetto
al tempo e i loro tassi di cambiamento sono collegati. I problemi
relativi ai tassi correlati utilizzano i derivati
\hspace{0pt}\hspace{0pt}per descrivere queste relazioni.

\subsubsection{Approccio generale}\label{approccio-generale}

\begin{enumerate}
\def\labelenumi{\arabic{enumi}.}
\tightlist
\item
  Identificare le variabili che dipendono dal tempo \(t\).
\item
  Scrivi un'equazione che metta in relazione le variabili.
\item
  Differenziare entrambi i lati rispetto a \(t\), applicando la regola
  della catena.
\item
  Sostituisci i valori noti nell'istante dato.
\item
  Risolvere per il tasso sconosciuto.
\end{enumerate}

\subsubsection{Esempio 1: cerchio in
espansione}\label{esempio-1-cerchio-in-espansione}

Un cerchio ha un raggio \(r\), che aumenta al ritmo di
\(\frac{dr}{dt} = 2 \,\text{cm/s}\). Trova la velocità con cui l'area
\(A = \pi r^2\) aumenta quando \(r = 5\).

Differenziare:

\[
\frac{dA}{dt} = 2\pi r \frac{dr}{dt}.
\]

Sostituisci:

\[
\frac{dA}{dt} = 2\pi (5)(2) = 20\pi \,\text{cm}^2/\text{s}.
\]

\subsubsection{Esempio 2: scala
scorrevole}\label{esempio-2-scala-scorrevole}

Una scala di 10 piedi è appoggiata a un muro. Il fondo scivola via a
\(\frac{dx}{dt} = 1 \,\text{ft/s}\). Quanto velocemente la parte
superiore scivola verso il basso quando la parte inferiore è a 6 piedi
dal muro?

Equazione: \(x^2 + y^2 = 100\), dove \(y\) è l'altezza.

Differenziare:

\[
2x \frac{dx}{dt} + 2y \frac{dy}{dt} = 0.
\]A \(x = 6\), \(y = 8\). Sostituisci:

\[
2(6)(1) + 2(8)\frac{dy}{dt} = 0 \quad \Rightarrow \quad \frac{dy}{dt} = -\tfrac{6}{8} = -\tfrac{3}{4}.
\]

Quindi la parte superiore scorre verso il basso in
\(0.75 \,\text{ft/s}\).

\subsubsection{Esempio 3: Acqua in un
cono}\label{esempio-3-acqua-in-un-cono}

Si versa l'acqua in un cono di altezza 12 cm e raggio 6 cm. Quando
l'acqua è profonda 4 cm, il livello dell'acqua aumenta a
\(2 \,\text{cm/s}\). A che ritmo aumenta il volume?

Equazione: \(V = \tfrac{1}{3}\pi r^2 h\). Utilizzando la somiglianza,
\(r = \tfrac{h}{2}\). Sostituendo:

\[
V = \tfrac{1}{12}\pi h^3.
\]

Differenziare:

\[
\frac{dV}{dt} = \tfrac{1}{4}\pi h^2 \frac{dh}{dt}.
\]

A \(h = 4\), \(\frac{dh}{dt} = 2\):

\[
\frac{dV}{dt} = \tfrac{1}{4}\pi (16)(2) = 8\pi \,\text{cm}^3/\text{s}.
\]

\subsubsection{Perché le tariffe correlate sono
importanti}\label{perchuxe9-le-tariffe-correlate-sono-importanti}

\begin{itemize}
\tightlist
\item
  Descrivono il movimento e il cambiamento in fisica, ingegneria e
  biologia.
\item
  Connettono la geometria con il calcolo attraverso processi dipendenti
  dal tempo.
\item
  Ci addestrano a modellare matematicamente i sistemi dinamici.
\end{itemize}

\subsubsection{Esercizi}\label{esercizi-9}

\begin{enumerate}
\def\labelenumi{\arabic{enumi}.}
\tightlist
\item
  Un palloncino viene gonfiato in modo che il suo raggio aumenti a
  \(0.5 \,\text{cm/s}\). Trova quanto velocemente aumenta il suo volume
  quando il raggio è 10 cm.
\item
  Un'auto guida verso nord a 40 km/h e un'altra verso est a 30 km/h.
  Quanto velocemente aumenta la distanza tra loro 2 ore dopo?
\item
  Un faretto a 20 m da un muro illumina un uomo alto 2 m che si
  allontana a 1,5 m/s. Quanto velocemente cambia la lunghezza della sua
  ombra sul muro quando si trova a 5 m dalla luce?
\item
  La lunghezza del lato di un cubo cresce di 2 cm/s. Quanto velocemente
  aumenta la superficie quando il lato è 3 cm?
\item
  Si versa la sabbia su un mucchio formando un cono di raggio sempre
  uguale all'altezza. Se l'altezza aumenta a 5 cm/s, a quale velocità
  aumenta il volume quando l'altezza è 10 cm?
\end{enumerate}

\subsection{3.3 Problemi di ottimizzazioneI problemi di ottimizzazione
utilizzano le derivate per trovare i valori massimi o minimi di una
funzione, spesso sotto determinati vincoli. Questi problemi modellano
situazioni in cui vogliamo massimizzare l'efficienza, il profitto o
l'area o minimizzare i costi, la distanza o il
tempo.}\label{problemi-di-ottimizzazionei-problemi-di-ottimizzazione-utilizzano-le-derivate-per-trovare-i-valori-massimi-o-minimi-di-una-funzione-spesso-sotto-determinati-vincoli.-questi-problemi-modellano-situazioni-in-cui-vogliamo-massimizzare-lefficienza-il-profitto-o-larea-o-minimizzare-i-costi-la-distanza-o-il-tempo.}

\subsubsection{Passaggi generali}\label{passaggi-generali}

\begin{enumerate}
\def\labelenumi{\arabic{enumi}.}
\tightlist
\item
  Comprendere il problema: identificare la quantità da ottimizzare.
\item
  Modello con una funzione: Scrivi la funzione obiettivo in termini di
  una variabile.
\item
  Applicare vincoli: utilizzare determinate condizioni per ridurre le
  variabili.
\item
  Differenziare: calcolare la derivata della funzione obiettivo.
\item
  Trova i punti critici: risolvi \(f'(x) = 0\) o dove \(f'(x)\) non è
  definito.
\item
  Test per i massimi/minimi: utilizzare il test della derivata seconda o
  verificare gli endpoint.
\item
  Interpreta il risultato: formula la risposta nel contesto originale.
\end{enumerate}

\subsubsection{Esempio 1: area massima di un
rettangolo}\label{esempio-1-area-massima-di-un-rettangolo}

Un rettangolo ha perimetro 40. Quali dimensioni massimizzano la sua
area?

\begin{itemize}
\tightlist
\item
  Sia la lunghezza \(x\), la larghezza \(y\). Vincolo:
  \(2x + 2y = 40 \Rightarrow y = 20 - x\).
\item
  Area: \(A = xy = x(20 - x) = 20x - x^2\).
\item
  Derivato: \(A'(x) = 20 - 2x\). Imposta uguale a 0: \(x = 10\).
\item
  Quindi \(y = 10\).
\item
  Area massima: \(100\). Il rettangolo è un quadrato.
\end{itemize}

\subsubsection{Esempio 2: minimizzare la
distanza}\label{esempio-2-minimizzare-la-distanza}

Trova il punto sulla parabola \(y = x^2\) più vicino a \((0,3)\).

\begin{itemize}
\tightlist
\item
  Distanza al quadrato: \(D(x) = (x-0)^2 + (x^2 - 3)^2\).
\item
  Espandi:
  \(D(x) = x^2 + (x^2 - 3)^2 = x^2 + x^4 - 6x^2 + 9 = x^4 - 5x^2 + 9\).
\item
  Derivato: \(D'(x) = 4x^3 - 10x\). Risolvi: \(x(4x^2 - 10) = 0\).
\item
  Soluzioni: \(x = 0\), \(x = \pm \sqrt{2.5}\).
\item
  Il controllo fornisce la distanza minima a \(x = \pm \sqrt{2.5}\).
\end{itemize}

\subsubsection{Esempio 3: Scatola con volume
massimo}\label{esempio-3-scatola-con-volume-massimo}

Una scatola senza coperchio deve essere realizzata con un pezzo di
cartone quadrato di 20 cm di lato ritagliando quadrati uguali dagli
angoli e ripiegando i lati. Trova la dimensione del taglio che
massimizza il volume.- Lascia che la dimensione del taglio sia = \(x\).
Quindi dimensioni: \((20 - 2x) \times (20 - 2x) \times x\). - Volume:
\(V(x) = x(20 - 2x)^2\). - Derivato: \(V'(x) = (20 - 2x)(20 - 6x)\). -
Punti critici: \(x = 10\) (riduce il volume a zero) o
\(x = \tfrac{20}{6} \approx 3.33\). - A \(x \approx 3.33\), il volume è
massimizzato.

\subsubsection{Perché l'ottimizzazione è
importante}\label{perchuxe9-lottimizzazione-uxe8-importante}

\begin{itemize}
\tightlist
\item
  Gli ingegneri lo usano per progettare strutture efficienti.
\item
  Le aziende lo utilizzano per massimizzare i profitti o minimizzare i
  costi.
\item
  Gli scienziati lo usano per modellare i sistemi naturali che cercano
  l'equilibrio.
\end{itemize}

\subsubsection{Esercizi}\label{esercizi-10}

\begin{enumerate}
\def\labelenumi{\arabic{enumi}.}
\tightlist
\item
  Un agricoltore ha 100 m di recinzione per recintare un campo
  rettangolare lungo un fiume (quindi solo 3 lati necessitano di
  recinzione). Trova l'area che massimizza le dimensioni.
\item
  Trova due numeri positivi la cui somma sia 20 e il cui prodotto sia il
  più grande possibile.
\item
  Un cilindro deve essere realizzato con 100 cm\(^2\) di materiale.
  Trova le dimensioni del volume massimo.
\item
  Un filo lungo 10 m viene tagliato in due pezzi, uno piegato a forma di
  quadrato, l'altro a forma di cerchio. Come dovrebbe essere tagliato
  per massimizzare l'area totale racchiusa?
\item
  Si dovrà realizzare una scatola chiusa a base quadrata e volume 32
  m\(^3\). Trova le dimensioni che minimizzano la superficie.
\end{enumerate}

\subsection{3.4 Concavità e punti di
flesso}\label{concavituxe0-e-punti-di-flesso}

I derivati non ci parlano solo delle pendenze ma anche della forma di un
grafico. La derivata seconda è particolarmente utile per comprendere la
concavità e identificare i punti di flesso.

\subsubsection{Concavità}\label{concavituxe0}

\begin{itemize}
\item
  Una funzione \(f(x)\) è concava su un intervallo se \(f''(x) > 0\). Il
  grafico si piega verso l'alto, come una tazza.
\item
  Una funzione \(f(x)\) è concava su un intervallo se \(f''(x) < 0\). Il
  grafico si piega verso il basso, come un cipiglio.
\end{itemize}

La concavità descrive come cambia la pendenza di una funzione: se le
pendenze aumentano, il grafico è concavo verso l'alto; se le pendenze
sono decrescenti il \hspace{0pt}\hspace{0pt}grafico è concavo verso il
basso.

\subsubsection{Punti di flessoUn punto di flesso è un punto sul grafico
in cui cambia la
concavità.}\label{punti-di-flessoun-punto-di-flesso-uxe8-un-punto-sul-grafico-in-cui-cambia-la-concavituxe0.}

\begin{itemize}
\tightlist
\item
  Se \(f''(x) = 0\) o \(f''(x)\) non è definito, il punto è un candidato
  per un punto di flesso.
\item
  Per confermare, la concavità deve cambiare segno da una parte e
  dall'altra della punta.
\end{itemize}

\subsubsection{Esempi}\label{esempi-6}

\begin{enumerate}
\def\labelenumi{\arabic{enumi}.}
\item
  \(f(x) = x^3\)

  \begin{itemize}
  \tightlist
  \item
    \(f''(x) = 6x\).
  \item
    A \(x = 0\), \(f''(0) = 0\).
  \item
    Per \(x < 0\), \(f''(x) < 0\) → concavo verso il basso.
  \item
    Per \(x > 0\), \(f''(x) > 0\) → concavo verso l'alto.
  \item
    Pertanto, \((0,0)\) è un punto di flesso.
  \end{itemize}
\item
  \(f(x) = x^4\)

  \begin{itemize}
  \tightlist
  \item
    \(f''(x) = 12x^2\).
  \item
    A \(x = 0\), \(f''(0) = 0\), ma la concavità non cambia segno
    (sempre ≥ 0).
  \item
    Nessun punto di flesso.
  \end{itemize}
\end{enumerate}

\subsubsection{Schizzi di concavità e
curve}\label{schizzi-di-concavituxe0-e-curve}

\begin{itemize}
\tightlist
\item
  Se \(f'(x) = 0\) e \(f''(x) > 0\), allora \(f\) ha un minimo locale.
\item
  Se \(f'(x) = 0\) e \(f''(x) < 0\), allora \(f\) ha un massimo locale.
\item
  Questo è noto come test della derivata seconda.
\end{itemize}

\subsubsection{Perché è importante}\label{perchuxe9-uxe8-importante-1}

La concavità e i punti di flesso ci aiutano a comprendere la ``forma''
dei grafici: dove si piegano, si appiattiscono o girano. Queste idee
sono centrali nel disegno delle curve, nella fisica (accelerazione) e
nell'economia (rendimenti decrescenti).

\subsubsection{Esercizi}\label{esercizi-11}

\begin{enumerate}
\def\labelenumi{\arabic{enumi}.}
\tightlist
\item
  Determinare gli intervalli di concavità per \(f(x) = x^3 - 3x\). Trova
  i suoi punti di flesso.
\item
  Per \(f(x) = \ln(x)\), identificare la concavità e i possibili punti
  di flesso.
\item
  Applicare il test della derivata seconda a \(f(x) = x^2 e^{-x}\) per
  classificare i punti critici.
\item
  Disegna \(f(x) = \sin x\), segnando gli intervalli di concavità e i
  punti di flesso.
\item
  Spiega perché \(f(x) = e^x\) non ha punti di flesso.
\end{enumerate}

\subsection{3.5 Schizzo di curve}\label{schizzo-di-curve}

Lo sketch di curve è il processo di disegno del grafico di una funzione
utilizzando le informazioni delle sue derivate. Invece di tracciare
molti punti, analizziamo le caratteristiche chiave: intercetta,
asintoti, intervalli crescenti/diminuenti e concavità.

\subsubsection{Passaggi per il disegno di curve1. Dominio: identifica
dove è definita la
funzione.}\label{passaggi-per-il-disegno-di-curve1.-dominio-identifica-dove-uxe8-definita-la-funzione.}

\begin{enumerate}
\def\labelenumi{\arabic{enumi}.}
\setcounter{enumi}{1}
\item
  Intercette: trova il punto in cui il grafico incrocia gli assi.
\item
  Asintoti:

  \begin{itemize}
  \tightlist
  \item
    Gli asintoti verticali si verificano dove la funzione non è definita
    e tende all'infinito.
  \item
    Gli asintoti orizzontali o inclinati descrivono il comportamento
    finale come \(x \to \pm\infty\).
  \end{itemize}
\item
  Derivata prima \(f'(x)\):

  \begin{itemize}
  \tightlist
  \item
    Positivo → la funzione è in aumento.
  \item
    Negativo → la funzione diminuisce.
  \item
    Zeri di \(f'(x)\) → punti critici (possibili massimi/minimi).
  \end{itemize}
\item
  Derivata seconda \(f''(x)\):

  \begin{itemize}
  \tightlist
  \item
    Positivo → concavo verso l'alto.
  \item
    Negativo → concavo verso il basso.
  \item
    Zeri o indefiniti → possibili punti di flesso.
  \end{itemize}
\item
  Combina le informazioni: utilizza tutti i risultati per tracciare un
  grafico chiaro e accurato.
\end{enumerate}

\subsubsection{\texorpdfstring{Esempio 1:
\(f(x) = x^3 - 3x\)}{Esempio 1: f(x) = x\^{}3 - 3x}}\label{esempio-1-fx-x3---3x}

\begin{itemize}
\item
  Dominio: tutti i numeri reali.
\item
  Intercettazioni: a \((0,0)\).
\item
  Derivato: \(f'(x) = 3x^2 - 3 = 3(x-1)(x+1)\).

  \begin{itemize}
  \tightlist
  \item
    Crescente: \((-\infty, -1) \cup (1, \infty)\).
  \item
    Decrescente: \((-1, 1)\).
  \end{itemize}
\item
  Derivata seconda: \(f''(x) = 6x\).

  \begin{itemize}
  \tightlist
  \item
    Concavo verso il basso per \(x < 0\), concavo verso l'alto per
    \(x > 0\).
  \item
    Punto di flesso in \((0,0)\).
  \end{itemize}
\item
  Forma: una curva a S con massimo locale in \((-1, 2)\), minimo locale
  in \((1, -2)\).
\end{itemize}

\subsubsection{\texorpdfstring{Esempio 2:
\(f(x) = \frac{1}{x}\)}{Esempio 2: f(x) = \textbackslash frac\{1\}\{x\}}}\label{esempio-2-fx-frac1x}

\begin{itemize}
\item
  Dominio: \(x \neq 0\).
\item
  Asintoto verticale: \(x = 0\).
\item
  Asintoto orizzontale: \(y = 0\).
\item
  Derivato: \(f'(x) = -\frac{1}{x^2}\) (sempre negativo). La funzione è
  sempre decrescente.
\item
  Derivata seconda: \(f''(x) = \frac{2}{x^3}\).

  \begin{itemize}
  \tightlist
  \item
    Concavo per \(x > 0\).
  \item
    Concavo verso il basso per \(x < 0\).
  \end{itemize}
\item
  Grafico: iperbole con due rami.
\end{itemize}

\subsubsection{Perché è utile il disegno di
curve}\label{perchuxe9-uxe8-utile-il-disegno-di-curve}

\begin{itemize}
\tightlist
\item
  Fornisce informazioni dettagliate sul comportamento generale delle
  funzioni senza calcoli esaustivi.
\item
  Indispensabile negli esami di calcolo infinitesimale e nei problemi
  applicativi.
\item
  Collega l'analisi algebrica e la comprensione geometrica.
\end{itemize}

\subsubsection{Esercizi}\label{esercizi-12}

\begin{enumerate}
\def\labelenumi{\arabic{enumi}.}
\tightlist
\item
  Disegna la curva di \(f(x) = x^4 - 2x^2\). Identificare massimi,
  minimi e punti di flesso.2. Analizza e disegna \(f(x) = \ln(x)\).
  Mostra intercetta, asintoti e concavità.
\item
  Per \(f(x) = e^{-x}\), descrivere crescita/decadimento, asintoti e
  concavità.
\item
  Disegna il grafico di \(f(x) = \tan x\) sull'intervallo
  \((- \pi, \pi)\). Segna gli asintoti.
\item
  Utilizzare i test della derivata prima e seconda per classificare i
  punti critici di \(f(x) = x^3 - 6x^2 + 9x\).
\end{enumerate}

\section{Parte II. Integrali}\label{parte-ii.-integrali}

\section{Capitolo 4. Antiderivative e integrali
definiti}\label{capitolo-4.-antiderivative-e-integrali-definiti}

\subsection{4.1 Integrali indefiniti}\label{integrali-indefiniti}

Un integrale indefinito è il processo inverso di differenziazione. Se
una derivata misura il cambiamento, allora un integrale recupera la
funzione originale dal suo tasso di cambiamento.

\subsubsection{Definizione}\label{definizione-3}

Se \(F'(x) = f(x)\), allora

\[
\int f(x)\,dx = F(x) + C,
\]

dove \(C\) è la costante di integrazione.

Ogni integrale indefinito rappresenta una famiglia di funzioni che
differiscono solo per una costante, poiché la differenziazione elimina
le costanti.

\subsubsection{Regole di base}\label{regole-di-base}

\begin{enumerate}
\def\labelenumi{\arabic{enumi}.}
\tightlist
\item
  Regola costante
\end{enumerate}

\[
\int c\,dx = cx + C.
\]

\begin{enumerate}
\def\labelenumi{\arabic{enumi}.}
\setcounter{enumi}{1}
\tightlist
\item
  Regola del potere
\end{enumerate}

\[
\int x^n\,dx = \frac{x^{n+1}}{n+1} + C, \quad n \neq -1.
\]

\begin{enumerate}
\def\labelenumi{\arabic{enumi}.}
\setcounter{enumi}{2}
\tightlist
\item
  Regola della somma
\end{enumerate}

\[
\int \big(f(x) + g(x)\big)\,dx = \int f(x)\,dx + \int g(x)\,dx.
\]

\begin{enumerate}
\def\labelenumi{\arabic{enumi}.}
\setcounter{enumi}{3}
\tightlist
\item
  Regola multipla costante
\end{enumerate}

\[
\int c f(x)\,dx = c \int f(x)\,dx.
\]

\subsubsection{Integrali comuni}\label{integrali-comuni}

\begin{itemize}
\tightlist
\item
  \(\int e^x dx = e^x + C\)
\item
  \(\int \sin x dx = -\cos x + C\)
\item
  \(\int \cos x dx = \sin x + C\)
\item
  \(\int \frac{1}{x} dx = \ln|x| + C\)
\end{itemize}

\subsubsection{Esempi}\label{esempi-7}

\begin{enumerate}
\def\labelenumi{\arabic{enumi}.}
\item
  \(\int (3x^2 - 4)\,dx = x^3 - 4x + C\).
\item
  \(\int \cos(2x)\,dx = \tfrac{1}{2}\sin(2x) + C\).
\item
  \(\int \frac{1}{x}\,dx = \ln|x| + C\).
\end{enumerate}

\subsubsection{Interpretazione}\label{interpretazione-1}

\begin{itemize}
\tightlist
\item
  Gli integrali indefiniti sono antiderivative.
\item
  Sono la base per gli integrali definiti, che misurano quantità
  accumulate come area, distanza e massa.
\item
  Nei contesti applicativi, l'integrazione ci consente di passare dai
  tassi ai totali.
\end{itemize}

\subsubsection{Esercizi}\label{esercizi-13}

\begin{enumerate}
\def\labelenumi{\arabic{enumi}.}
\tightlist
\item
  Trova \(\int (5x^4 + 2x)\,dx\).2. Calcola \(\int (e^x + 3)\,dx\).
\item
  Trova la soluzione generale di \(f'(x) = 6x\) utilizzando
  l'integrazione.
\item
  Valuta \(\int \frac{2}{x}\,dx\).
\item
  Se la velocità è \(v(t) = 4t\), trovare la funzione di posizione
  \(s(t)\).
\end{enumerate}

\subsection{4.2 L'integrale definito come
area}\label{lintegrale-definito-come-area}

Mentre gli integrali indefiniti rappresentano famiglie di
antiderivative, l'integrale definito fornisce un valore numerico: l'area
accumulata sotto una curva tra due punti.

\subsubsection{Definizione}\label{definizione-4}

Per una funzione \(f(x)\) definita su \([a, b]\), l'integrale definito è

\[
\int_a^b f(x)\,dx = \lim_{n \to \infty} \sum_{i=1}^n f(x_i^-) \,\Delta x,
\]

dove l'intervallo \([a, b]\) è diviso in \(n\) sottointervalli di
larghezza \(\Delta x\) e \(x_i^-\) è un punto campione in ciascun
sottointervallo.

Questo è il limite delle somme di Riemann.

\subsubsection{Interpretazione
geometrica}\label{interpretazione-geometrica}

\begin{itemize}
\tightlist
\item
  Se \(f(x) \geq 0\) su \([a, b]\), allora \(\int_a^b f(x)\,dx\) è
  uguale all'area sotto la curva \(y = f(x)\) da \(x=a\) a \(x=b\).
\item
  Se \(f(x)\) scende al di sotto dell'asse \(x\), l'integrale calcola
  l'area con segno: le regioni sotto l'asse contano come negative.
\end{itemize}

\subsubsection{Proprietà dell'integrale
definito}\label{proprietuxe0-dellintegrale-definito}

\begin{enumerate}
\def\labelenumi{\arabic{enumi}.}
\tightlist
\item
  Additività negli intervalli
\end{enumerate}

\[
\int_a^c f(x)\,dx = \int_a^b f(x)\,dx + \int_b^c f(x)\,dx.
\]

\begin{enumerate}
\def\labelenumi{\arabic{enumi}.}
\setcounter{enumi}{1}
\tightlist
\item
  Inversione dei limiti
\end{enumerate}

\[
\int_a^b f(x)\,dx = -\int_b^a f(x)\,dx.
\]

\begin{enumerate}
\def\labelenumi{\arabic{enumi}.}
\setcounter{enumi}{2}
\tightlist
\item
  Intervallo di larghezza zero
\end{enumerate}

\[
\int_a^a f(x)\,dx = 0.
\]

\begin{enumerate}
\def\labelenumi{\arabic{enumi}.}
\setcounter{enumi}{3}
\tightlist
\item
  Linearità
\end{enumerate}

\[
\int_a^b \big( cf(x) + g(x)\big)\,dx = c\int_a^b f(x)\,dx + \int_a^b g(x)\,dx.
\]

\subsubsection{Esempi}\label{esempi-8}

\begin{enumerate}
\def\labelenumi{\arabic{enumi}.}
\item
  \(\int_0^2 x\,dx = \left[\tfrac{1}{2}x^2\right]_0^2 = 2.\) Questa è
  l'area di un triangolo rettangolo sotto la linea \(y=x\).
\item
  \(\int_{-1}^1 x^3\,dx = 0.\) La funzione dispari \(x^3\) ha aree
  simmetriche che si annullano.
\item
  \(\int_0^\pi \sin x\,dx = 2.\) Ciò equivale all'area sotto un arco
  della curva sinusoidale.
\end{enumerate}

\subsubsection{Perché è importante}\label{perchuxe9-uxe8-importante-2}

\begin{itemize}
\tightlist
\item
  Gli integrali definiti misurano quantità accumulate: distanza, massa,
  energia, probabilità.- Collegano il calcolo algebrico con l'intuizione
  geometrica.
\item
  Il passo successivo è il Teorema Fondamentale del Calcolo
  infinitesimale, che collega gli integrali definiti con le
  antiderivative.
\end{itemize}

\subsubsection{Esercizi}\label{esercizi-14}

\begin{enumerate}
\def\labelenumi{\arabic{enumi}.}
\tightlist
\item
  Calcola \(\int_0^3 (2x+1)\,dx\).
\item
  Trova l'area tra \(y = x^2\) e l'asse \(x\) da \(x = 0\) a \(x = 2\).
\item
  Valuta \(\int_{-2}^2 (x^2 - 1)\,dx\).
\item
  Mostra che \(\int_{-a}^a f(x)\,dx = 0\) se \(f(x)\) è dispari.
\item
  Approssima \(\int_0^1 e^x\,dx\) utilizzando una somma di Riemann con
  sottointervalli \(n=4\) ed estremi destri.
\end{enumerate}

\subsection{4.3 Il Teorema Fondamentale del
Calcolo}\label{il-teorema-fondamentale-del-calcolo}

Il Teorema Fondamentale del Calcolo (FTC) unisce le due idee principali
del calcolo: differenziazione e integrazione. Dimostra che individuare
le aree e individuare i tassi di cambiamento sono due facce della stessa
medaglia.

\subsubsection{Parte 1: Differenziazione di un
integrale}\label{parte-1-differenziazione-di-un-integrale}

Se \(f\) è continuo su \([a, b]\), definire

\[
F(x) = \int_a^x f(t)\,dt.
\]

Quindi \(F\) è differenziabile e

\[
F'(x) = f(x).
\]

In parole: la derivata della funzione area accumulata è la funzione
originaria stessa.

\subsubsection{Parte 2: Valutazione degli integrali
definiti}\label{parte-2-valutazione-degli-integrali-definiti}

Se \(f\) è continuo su \([a, b]\) e \(F\) è un qualsiasi antiderivativo
di \(f\), allora

\[
\int_a^b f(x)\,dx = F(b) - F(a).
\]

Questo ci dice che possiamo valutare gli integrali definiti
semplicemente trovando una antiderivativa, piuttosto che calcolando i
limiti delle somme di Riemann.

\subsubsection{Esempi}\label{esempi-9}

\begin{enumerate}
\def\labelenumi{\arabic{enumi}.}
\item
  \(\int_0^2 x^2\,dx\).

  \begin{itemize}
  \tightlist
  \item
    Antiderivativa: \(F(x) = \tfrac{1}{3}x^3\).
  \item
    Applica FTC: \(F(2) - F(0) = \tfrac{8}{3} - 0 = \tfrac{8}{3}.\)
  \end{itemize}
\item
  Se \(F(x) = \int_1^x \cos t \, dt\), allora \(F'(x) = \cos x\).
\item
  \(\int_1^4 \frac{1}{x}\,dx\).

  \begin{itemize}
  \tightlist
  \item
    Antiderivativa: \(\ln|x|\).
  \item
    Applica FTC: \(\ln 4 - \ln 1 = \ln 4.\)
  \end{itemize}
\end{enumerate}

\subsubsection{Perché la FTC è
importante}\label{perchuxe9-la-ftc-uxe8-importante}

\begin{itemize}
\tightlist
\item
  Trasforma l'integrazione da un processo limite in un calcolo pratico.-
  Si conferma che differenziazione e integrazione sono operazioni
  inverse.
\item
  È il teorema centrale che rende il calcolo infinitesimale utile in
  matematica, scienze e ingegneria.
\end{itemize}

\subsubsection{Esercizi}\label{esercizi-15}

\begin{enumerate}
\def\labelenumi{\arabic{enumi}.}
\tightlist
\item
  Valuta \(\int_0^3 (2x+1)\,dx\) utilizzando la FTC.
\item
  Se \(F(x) = \int_0^x e^t\,dt\), trova \(F'(x)\).
\item
  Calcola \(\int_0^\pi \sin x \, dx\).
\item
  Mostra che se \(f'(x) = g(x)\), allora
  \(\int_a^b g(x)\,dx = f(b) - f(a)\).
\item
  Utilizzare la FTC per spiegare perché l'area sotto \(y = \cos x\) da
  \(0\) a \(\pi/2\) è uguale a 1.
\end{enumerate}

\subsection{4.4 Proprietà degli
integrali}\label{proprietuxe0-degli-integrali}

L'integrale definito ha diverse proprietà importanti che lo rendono
flessibile e potente nelle applicazioni. Queste proprietà derivano dalla
definizione di limite di somme e dal Teorema Fondamentale dell'Calcolo.

\subsubsection{Linearità}\label{linearituxe0}

Per le funzioni \(f(x)\) e \(g(x)\) e costanti \(c, d\):

\[
\int_a^b \big(c f(x) + d g(x)\big)\,dx = c \int_a^b f(x)\,dx + d \int_a^b g(x)\,dx.
\]

Ciò ci consente di scomporre gli integrali complicati in parti più
semplici.

\subsubsection{Additività negli
intervalli}\label{additivituxe0-negli-intervalli}

Se \(a < c < b\), allora

\[
\int_a^b f(x)\,dx = \int_a^c f(x)\,dx + \int_c^b f(x)\,dx.
\]

Possiamo calcolare gli integrali pezzo per pezzo.

\subsubsection{Inversione dei limiti}\label{inversione-dei-limiti}

\[
\int_a^b f(x)\,dx = -\int_b^a f(x)\,dx.
\]

Scambiando i limiti si cambia il segno dell'integrale.

\subsubsection{Proprietà di confronto}\label{proprietuxe0-di-confronto}

Se \(f(x) \leq g(x)\) per tutti \(x\) in \([a, b]\), allora

\[
\int_a^b f(x)\,dx \leq \int_a^b g(x)\,dx.
\]

Ciò ci consente di confrontare le aree senza calcolo diretto.

\subsubsection{Disuguaglianza dei valori
assoluti}\label{disuguaglianza-dei-valori-assoluti}

\[
\left| \int_a^b f(x)\,dx \right| \leq \int_a^b |f(x)|\,dx.
\]

Questa proprietà è essenziale nelle analisi e nei test di convergenza.

\subsubsection{Simmetria}\label{simmetria}

\begin{itemize}
\item
  Se \(f(x)\) è pari (simmetrico rispetto all'asse \(y\)):

  \[
  \int_{-a}^a f(x)\,dx = 2\int_0^a f(x)\,dx.
  \]
\item
  Se \(f(x)\) è dispari (simmetrico rispetto all'origine):

  \[
  \int_{-a}^a f(x)\,dx = 0.
  \]\#\#\# Esempi
\end{itemize}

\begin{enumerate}
\def\labelenumi{\arabic{enumi}.}
\item
  \(\int_0^2 (3x^2 + 4)\,dx = \int_0^2 3x^2\,dx + \int_0^2 4\,dx = 8 + 8 = 16.\)
\item
  Poiché \(f(x) = x^3\) è dispari, \(\int_{-1}^1 x^3\,dx = 0.\)
\item
  Poiché \(f(x) = x^2\) è pari,
  \(\int_{-2}^2 x^2\,dx = 2\int_0^2 x^2\,dx = 2\cdot \tfrac{8}{3} = \tfrac{16}{3}.\)
\end{enumerate}

\subsubsection{Perché queste proprietà sono
importanti}\label{perchuxe9-queste-proprietuxe0-sono-importanti}

\begin{itemize}
\tightlist
\item
  Semplificano i calcoli.
\item
  Rivelano caratteristiche geometriche e di simmetria delle funzioni.
\item
  Forniscono strumenti teorici per analisi più avanzate.
\end{itemize}

\subsubsection{Esercizi}\label{esercizi-16}

\begin{enumerate}
\def\labelenumi{\arabic{enumi}.}
\tightlist
\item
  Utilizza la simmetria per valutare \(\int_{-5}^5 (x^4 - x^3)\,dx\).
\item
  Mostra che
  \(\int_1^4 (2x+3)\,dx = \int_1^2 (2x+3)\,dx + \int_2^4 (2x+3)\,dx\).
\item
  Valuta \(\int_0^\pi \sin(x)\,dx\) e confronta con
  \(\int_{-\pi}^\pi \sin(x)\,dx\).
\item
  Dimostra che se \(f(x) \geq 0\) su \([a, b]\), allora
  \(\int_a^b f(x)\,dx \geq 0\).
\item
  Calcola \(\int_{-3}^3 (x^2 + 1)\,dx\) utilizzando le proprietà
  pari/dispari.
\end{enumerate}

\section{Capitolo 5. Tecniche di
integrazione}\label{capitolo-5.-tecniche-di-integrazione}

\subsection{5.1 Sostituzione}\label{sostituzione}

Una delle tecniche di integrazione più utili è il metodo di
sostituzione, chiamato anche -u-substitution-. È il processo inverso
della regola della catena per i derivati.

\subsubsection{L'idea}\label{lidea-1}

Se un integrale contiene una funzione composta, possiamo semplificarla
modificando le variabili.

Formalmente, se \(u = g(x)\) è una funzione differenziabile, allora

\[
\int f(g(x)) g'(x)\,dx = \int f(u)\,du.
\]

Questa sostituzione rende l'integrale più facile da valutare.

\subsubsection{Passaggi per la
sostituzione}\label{passaggi-per-la-sostituzione}

\begin{enumerate}
\def\labelenumi{\arabic{enumi}.}
\tightlist
\item
  Identificare una funzione interna \(u = g(x)\) la cui derivata appare
  anche nell'integrando.
\item
  Calcola \(du = g'(x)\,dx\).
\item
  Riscrivere l'integrale in termini di \(u\).
\item
  Integra rispetto a \(u\).
\item
  Sostituisci con \(u = g(x)\).
\end{enumerate}

\subsubsection{Esempi}\label{esempi-10}

\begin{enumerate}
\def\labelenumi{\arabic{enumi}.}
\item
  Sostituzione semplice

  \[
  \int 2x \cos(x^2)\,dx
  \]

  Sia \(u = x^2\), quindi \(du = 2x\,dx\). Quindi l'integrale diventa
  \(\int \cos u \,du = \sin u + C = \sin(x^2) + C\).
\item
  Caso logaritmico

  \[\int \frac{2x}{x^2+1}\,dx
  \]

  Let \(u = x^2 + 1\), so \(du = 2x\,dx\). Then integral becomes
  \(\int \frac{1}{u}\,du = \ln|u| + C = \ln(x^2+1) + C\).
\item
  Trigonometric substitution

  \[
  \int \sin(3x)\,dx
  \]

  Let \(u = 3x\), so \(du = 3\,dx\), hence \(dx = \frac{du}{3}\).
  Integral becomes
  \(\tfrac{1}{3}\int \sin u\,du = -\tfrac{1}{3}\cos u + C = -\tfrac{1}{3}\cos(3x) + C\).
\end{enumerate}

\subsubsection{Definite Integrals with
Substitution}\label{definite-integrals-with-substitution}

When evaluating definite integrals, we must also change the limits:

\[
\int_a^b f(g(x)) g'(x)\,dx = \int_{g(a)}^{g(b)} f(u)\,du.
\]

Example:

\[
\int_0^1 2x e^{x^2}\,dx.
\]

Let \(u = x^2\), \(du = 2x\,dx\). Limits: when \(x=0, u=0\); when
\(x=1, u=1\). So the integral becomes

\[
\int_0^1 e^u\,du = e - 1.
\]

\subsubsection{Exercises}\label{exercises-1}

\begin{enumerate}
\def\labelenumi{\arabic{enumi}.}
\tightlist
\item
  Evaluate \(\int (x^2+1)^5 (2x)\,dx\).
\item
  Compute \(\int \frac{\cos x}{\sin x}\,dx\).
\item
  Evaluate \(\int_0^\pi \sin(2x)\,dx\) using substitution.
\item
  Find \(\int e^{3x}\,dx\).
\item
  Compute \(\int \frac{1}{\sqrt{1+x^2}}\,dx\) by letting \(u = 1+x^2\).
\end{enumerate}

\subsection{5.2 Integration by Parts}\label{integration-by-parts}

Integration by parts is a technique that comes from the product rule for
derivatives. It helps evaluate integrals involving products of functions
that are not easily handled by substitution alone.

\subsubsection{The Formula}\label{the-formula}

From the product rule:

\[
\frac{d}{dx}[u(x)v(x)] = u'(x)v(x) + u(x)v'(x).
\]

Integrating both sides gives the integration by parts formula:

\[
\int u\,dv = uv - \int v\,du.
\]

Here:

\begin{itemize}
\tightlist
\item
  \(u\) = a function chosen to be differentiated,
\item
  \(dv\) = the remaining part of the integrand to be integrated.
\end{itemize}

\subsubsection{\texorpdfstring{Choosing \(u\) and
\(dv\)}{Choosing u and dv}}\label{choosing-u-and-dv}

A common guideline is LIATE (Logarithmic, Inverse trig, Algebraic,
Trigonometric, Exponential).

\begin{itemize}
\tightlist
\item
  Choose \(u\) from the earliest category present.
\item
  Choose \(dv\) as the rest.
\end{itemize}

\subsubsection{Examples}\label{examples-1}

\begin{enumerate}
\def\labelenumi{\arabic{enumi}.}
\tightlist
\item
  Polynomial × Exponential
\end{enumerate}

\[
\intxe^x\,dx
\]Lascia che \(u = x\), \(dv = e^x dx\). Quindi \(du = dx\),
\(v = e^x\).

\[
\int x e^x\,dx = x e^x - \int e^x dx = x e^x - e^x + C.
\]

\begin{enumerate}
\def\labelenumi{\arabic{enumi}.}
\setcounter{enumi}{1}
\tightlist
\item
  Polinomio × Trig
\end{enumerate}

\[
\int x \cos x\,dx
\]

Lascia che \(u = x\), \(dv = \cos x dx\). Quindi \(du = dx\),
\(v = \sin x\).

\[
\int x \cos x\,dx = x \sin x - \int \sin x dx = x \sin x + \cos x + C.
\]

\begin{enumerate}
\def\labelenumi{\arabic{enumi}.}
\setcounter{enumi}{2}
\tightlist
\item
  Logaritmo
\end{enumerate}

\[
\int \ln x\,dx
\]

Lascia che \(u = \ln x\), \(dv = dx\). Quindi \(du = \frac{1}{x}dx\),
\(v = x\).

\[
\int \ln x\,dx = x \ln x - \int 1 dx = x \ln x - x + C.
\]

\subsubsection{Esempio di integrale
definito}\label{esempio-di-integrale-definito}

\[
\int_0^1 x e^x\,dx
\]

Utilizzando il risultato precedente: \(\int x e^x dx = (x-1)e^x\).
Valutare:

\[
\big[(x-1)e^x\big]_0^1 = (0)e^1 - (-1)e^0 = 0 + 1 = 1.
\]

\subsubsection{Perché è importante}\label{perchuxe9-uxe8-importante-3}

L'integrazione per parti è cruciale quando la sostituzione fallisce,
specialmente con logaritmi, funzioni trigonometriche inverse e prodotti
che coinvolgono polinomi con esponenziali o funzioni trigonometriche.

\subsubsection{Esercizi}\label{esercizi-17}

\begin{enumerate}
\def\labelenumi{\arabic{enumi}.}
\tightlist
\item
  Valuta \(\int x \sin x\,dx\).
\item
  Trova \(\int e^x \cos x\,dx\).
\item
  Calcola \(\int_1^2 \ln x\,dx\).
\item
  Valuta \(\int x^2 e^x\,dx\).
\item
  Utilizzare l'integrazione per parti per mostrare
  \(\int \arctan x\,dx = x\arctan x - \tfrac{1}{2}\ln(1+x^2) + C\).
\end{enumerate}

\subsection{5.3 Integrali trigonometrici e
sostituzioni}\label{integrali-trigonometrici-e-sostituzioni}

Molti integrali coinvolgono funzioni trigonometriche. Questi possono
spesso essere semplificati utilizzando le identità o effettuando
sostituzioni speciali.

\subsubsection{Integrali trigonometrici}\label{integrali-trigonometrici}

\begin{enumerate}
\def\labelenumi{\arabic{enumi}.}
\tightlist
\item
  Potenze di seno e coseno
\end{enumerate}

\begin{itemize}
\tightlist
\item
  Se la potenza del seno è dispari: salva uno \(\sin x\), converti il
  resto con \(\sin^2x = 1 - \cos^2x\) e sostituisci \(u = \cos x\).
\item
  Se la potenza del coseno è dispari: salva uno \(\cos x\), converti il
  \hspace{0pt}\hspace{0pt}resto con \(\cos^2x = 1 - \sin^2x\) e
  sostituisci \(u = \sin x\).
\item
  Se entrambi sono pari: utilizza identità a semiangolo.
\end{itemize}

Esempio:

\[
\int \sin^3x \cos x \, dx
\]

Lascia che \(u = \sin x\), \(du = \cos x\,dx\):

\[
\int u^3\,du = \tfrac{u^4}{4} + C = \tfrac{\sin^4x}{4} + C.\]

\begin{enumerate}
\def\labelenumi{\arabic{enumi}.}
\setcounter{enumi}{1}
\tightlist
\item
  Products of sine and cosine with different angles Use product-to-sum
  formulas:
\end{enumerate}

\[
\sin A \cos B = \tfrac{1}{2}[\sin(A+B) + \sin(A-B)].
\]

Example:

\[
\int \sin(2x)\cos(3x)\,dx = \tfrac{1}{2}\int [\sin(5x) - \sin(x)]\,dx.
\]

\begin{enumerate}
\def\labelenumi{\arabic{enumi}.}
\setcounter{enumi}{2}
\tightlist
\item
  Powers of secant and tangent
\end{enumerate}

\begin{itemize}
\tightlist
\item
  If the power of secant is even: save \(\sec^2x\), convert the rest
  with \(\sec^2x = 1 + \tan^2x\), and substitute \(u = \tan x\).
\item
  If the power of tangent is odd: save \(\sec^2x\), convert the rest
  with \(\tan^2x = \sec^2x - 1\), and substitute \(u = \tan x\).
\end{itemize}

Example:

\[
\int \tan^3x \sec^2x \, dx
\]

Let \(u = \tan x\), \(du = \sec^2x\,dx\):

\[
\int u^3\,du = \tfrac{u^4}{4} + C = \tfrac{\tan^4x}{4} + C.
\]

\subsubsection{Trigonometric
Substitutions}\label{trigonometric-substitutions}

For integrals involving \(\sqrt{a^2 - x^2}\), \(\sqrt{a^2 + x^2}\), or
\(\sqrt{x^2 - a^2}\), use special substitutions:

\begin{enumerate}
\def\labelenumi{\arabic{enumi}.}
\tightlist
\item
  \(x = a \sin \theta\), for \(\sqrt{a^2 - x^2}\).
\item
  \(x = a \tan \theta\), for \(\sqrt{a^2 + x^2}\).
\item
  \(x = a \sec \theta\), for \(\sqrt{x^2 - a^2}\).
\end{enumerate}

Example:

\[
\int \sqrt{a^2 - x^2}\,dx
\]

Let \(x = a\sin\theta\), so \(dx = a\cos\theta\,d\theta\):

\[
\int \sqrt{a^2 - a^2\sin^2\theta}(a\cos\theta\,d\theta) = \int a^2 \cos^2\theta \, d\theta.
\]

Semplificare utilizzando identità a mezzo angolo.

\subsubsection{Perché queste tecniche sono
importanti}\label{perchuxe9-queste-tecniche-sono-importanti}

\begin{itemize}
\tightlist
\item
  Convertono forme algebriche difficili in forme trigonometriche
  gestibili.
\item
  Sono particolarmente utili nei problemi che coinvolgono aree, volumi e
  lunghezze d'arco.
\item
  Gettano le basi per metodi di integrazione avanzati.
\end{itemize}

\subsubsection{Esercizi}\label{esercizi-18}

\begin{enumerate}
\def\labelenumi{\arabic{enumi}.}
\tightlist
\item
  Valuta \(\int \sin^4x \cos^2x \, dx\).
\item
  Calcola \(\int \sin(5x)\cos(2x)\,dx\).
\item
  Valuta \(\int \tan^2x \sec^2x \, dx\).
\item
  Trova \(\int \sqrt{9 - x^2}\,dx\) utilizzando la sostituzione.
\item
  Mostra che
  \(\int \frac{dx}{\sqrt{x^2 + a^2}} = \ln|x + \sqrt{x^2 + a^2}| + C\)
  utilizzando \(x = a\tan\theta\).
\end{enumerate}

\subsection{5.4 Frazioni parzialiQuando si integrano funzioni razionali
(rapporti di polinomi), un metodo potente è la scomposizione di frazioni
parziali. Questa tecnica esprime una frazione complicata come somma di
frazioni più semplici che sono più facili da
integrare.}\label{frazioni-parzialiquando-si-integrano-funzioni-razionali-rapporti-di-polinomi-un-metodo-potente-uxe8-la-scomposizione-di-frazioni-parziali.-questa-tecnica-esprime-una-frazione-complicata-come-somma-di-frazioni-piuxf9-semplici-che-sono-piuxf9-facili-da-integrare.}

\subsubsection{L'idea}\label{lidea-2}

Se \(R(x) = \frac{P(x)}{Q(x)}\) è una funzione razionale, dove il grado
di \(P(x)\) è inferiore al grado di \(Q(x)\), possiamo scomporre
\(R(x)\) in frazioni più semplici.

Questi pezzi più semplici corrispondono ai fattori del denominatore
\(Q(x)\).

\subsubsection{Forme comuni}\label{forme-comuni}

\begin{enumerate}
\def\labelenumi{\arabic{enumi}.}
\tightlist
\item
  Fattori lineari distinti Se
\end{enumerate}

\[
\frac{1}{(x-a)(x-b)},
\]

quindi decomporre come

\[
\frac{A}{x-a} + \frac{B}{x-b}.
\]

\begin{enumerate}
\def\labelenumi{\arabic{enumi}.}
\setcounter{enumi}{1}
\tightlist
\item
  Fattori lineari ripetuti Se il denominatore ha \((x-a)^n\), i termini
  lo sono
\end{enumerate}

\[
\frac{A_1}{x-a} + \frac{A_2}{(x-a)^2} + \dots + \frac{A_n}{(x-a)^n}.
\]

\begin{enumerate}
\def\labelenumi{\arabic{enumi}.}
\setcounter{enumi}{2}
\tightlist
\item
  Fattori quadratici irriducibili Se il denominatore ha \((x^2+bx+c)\),
  il numeratore è lineare:
\end{enumerate}

\[
\frac{Ax+B}{x^2+bx+c}.
\]

\subsubsection{Esempio 1: Fattori lineari
distinti}\label{esempio-1-fattori-lineari-distinti}

\[
\int \frac{1}{x^2 - 1}\,dx
\]

Denominatore del fattore: \((x-1)(x+1)\). Decomporre:

\[
\frac{1}{x^2-1} = \frac{1}{2}\left(\frac{1}{x-1} - \frac{1}{x+1}\right).
\]

Integra:

\[
\int \frac{1}{x^2 - 1}\,dx = \tfrac{1}{2}\ln\left|\frac{x-1}{x+1}\right| + C.
\]

\subsubsection{Esempio 2: fattore lineare
ripetuto}\label{esempio-2-fattore-lineare-ripetuto}

\[
\int \frac{1}{(x-1)^2}\,dx
\]

Questo è già semplice:

\[
\int (x-1)^{-2}\,dx = -\frac{1}{x-1} + C.
\]

\subsubsection{Esempio 3: Fattore quadratico
irriducibile}\label{esempio-3-fattore-quadratico-irriducibile}

\[
\int \frac{x}{x^2+1}\,dx
\]

Sostituisci \(u = x^2+1\) o riconosci che il numeratore è derivato del
denominatore.

\[
\int \frac{x}{x^2+1}\,dx = \tfrac{1}{2}\ln(x^2+1) + C.
\]

\subsubsection{Passaggi nella scomposizione parziale delle
frazioni}\label{passaggi-nella-scomposizione-parziale-delle-frazioni}

\begin{enumerate}
\def\labelenumi{\arabic{enumi}.}
\tightlist
\item
  Fattorizzare il denominatore.
\item
  Scrivi la forma generale delle frazioni parziali.
\item
  Moltiplicare per il denominatore per eliminare le frazioni.
\item
  Risolvere per costanti sconosciute.
\item
  Integra ogni termine.\#\#\# Perché è importante
\end{enumerate}

\begin{itemize}
\tightlist
\item
  Converte funzioni razionali complesse in semplici forme logaritmiche o
  arcotangenti.
\item
  Particolarmente utile nelle equazioni differenziali e nelle
  trasformate di Laplace.
\item
  Fondamentali nel calcolo avanzato e nell'ingegneria.
\end{itemize}

\subsubsection{Esercizi}\label{esercizi-19}

\begin{enumerate}
\def\labelenumi{\arabic{enumi}.}
\tightlist
\item
  Scomporre e integrare \(\int \frac{3x+5}{x^2-1}\,dx\).
\item
  Valuta \(\int \frac{1}{x^2(x+1)}\,dx\).
\item
  Calcola \(\int \frac{2x+1}{x^2+2x+2}\,dx\).
\item
  Trova \(\int \frac{1}{x^3 - x}\,dx\).
\item
  Mostra che \(\int \frac{dx}{x^2+1} = \arctan x + C\) utilizzando
  frazioni parziali o sostituzioni.
\end{enumerate}

\subsection{5.5 Integrali impropri}\label{integrali-impropri}

Alcuni integrali non possono essere valutati direttamente perché
l'intervallo è infinito o l'integrando diventa illimitato. Questi sono
detti integrali impropri. Sono definiti utilizzando i limiti.

\subsubsection{Definizione}\label{definizione-5}

\begin{enumerate}
\def\labelenumi{\arabic{enumi}.}
\tightlist
\item
  Intervallo infinito
\end{enumerate}

\[
\int_a^\infty f(x)\,dx = \lim_{b \to \infty} \int_a^b f(x)\,dx.
\]

\[
\int_{-\infty}^a f(x)\,dx = \lim_{b \to -\infty} \int_b^a f(x)\,dx.
\]

\begin{enumerate}
\def\labelenumi{\arabic{enumi}.}
\setcounter{enumi}{1}
\tightlist
\item
  Integrando illimitato Se \(f(x)\) ha un asintoto verticale in \(c\),
  allora
\end{enumerate}

\[
\int_a^c f(x)\,dx = \lim_{t \to c^-} \int_a^t f(x)\,dx,
\]

\[
\int_c^b f(x)\,dx = \lim_{t \to c^+} \int_t^b f(x)\,dx.
\]

\subsubsection{Convergenza e divergenza}\label{convergenza-e-divergenza}

\begin{itemize}
\tightlist
\item
  Se il limite esiste ed è finito, l'integrale improprio converge.
\item
  Se il limite non esiste o è infinito, l'integrale improprio diverge.
\end{itemize}

\subsubsection{Esempi}\label{esempi-11}

\begin{enumerate}
\def\labelenumi{\arabic{enumi}.}
\tightlist
\item
  Decadimento esponenziale
\end{enumerate}

\[
\int_1^\infty \frac{1}{x^2}\,dx = \lim_{b \to \infty} \Big[-\tfrac{1}{x}\Big]_1^b = 1.
\]

Questo converge.

\begin{enumerate}
\def\labelenumi{\arabic{enumi}.}
\setcounter{enumi}{1}
\tightlist
\item
  Funzione armonica
\end{enumerate}

\[
\int_1^\infty \frac{1}{x}\,dx = \lim_{b \to \infty} \ln b.
\]

Questo diverge all'infinito.

\begin{enumerate}
\def\labelenumi{\arabic{enumi}.}
\setcounter{enumi}{2}
\tightlist
\item
  Asintoto a 0
\end{enumerate}

\[
\int_0^1 \frac{1}{\sqrt{x}}\,dx = \lim_{t \to 0^+} \int_t^1 x^{-1/2}\,dx.
\]

\[
= \lim_{t \to 0^+} [2\sqrt{x}]_t^1 = 2.
\]

Questo converge.

\begin{enumerate}
\def\labelenumi{\arabic{enumi}.}
\setcounter{enumi}{3}
\tightlist
\item
  Asintoto a 0 (divergente)
\end{enumerate}

\[\int_0^1 \frac{1}{x}\,dx = \lim_{t \to 0^+} \ln(1) - \ln(t).
\]

This diverges since \(\ln(t) \to -\infty\).

\subsubsection{Comparison Test for Improper
Integrals}\label{comparison-test-for-improper-integrals}

\begin{itemize}
\tightlist
\item
  If \(0 \leq f(x) \leq g(x)\) for large \(x\), and \(\int g(x)\,dx\)
  converges, then \(\int f(x)\,dx\) also converges.
\item
  If \(\int f(x)\,dx\) diverges and \(f(x) \geq g(x) \geq 0\), then
  \(\int g(x)\,dx\) also diverges.
\end{itemize}

\subsubsection{Why Improper Integrals
Matter}\label{why-improper-integrals-matter}

\begin{itemize}
\tightlist
\item
  They extend integration to infinite domains and unbounded functions.
\item
  They are essential in probability (continuous distributions), physics
  (gravitational/electric fields), and Fourier analysis.
\end{itemize}

\subsubsection{Exercises}\label{exercises-2}

\begin{enumerate}
\def\labelenumi{\arabic{enumi}.}
\tightlist
\item
  Determine whether \(\int_1^\infty \frac{1}{x^p}\,dx\) converges for
  various values of \(p\).
\item
  Evaluate \(\int_0^\infty e^{-x}\,dx\).
\item
  Test convergence of \(\int_0^1 \frac{1}{x^p}\,dx\) depending on \(p\).
\item
  Compute \(\int_{-\infty}^\infty \frac{1}{1+x^2}\,dx\).
\item
  Use the comparison test to show that
  \(\int_1^\infty \frac{1}{x^2+1}\,dx\) converges.
\end{enumerate}

\section{Chapter 6. Applications of
Integration}\label{chapter-6.-applications-of-integration}

\subsection{6.1 Areas and Volumes}\label{areas-and-volumes}

One of the most important applications of integration is finding areas
under curves and volumes of solids.

\subsubsection{Area Between Curves}\label{area-between-curves}

If \(f(x) \geq g(x)\) on \([a, b]\), then the area between the curves
\(y=f(x)\) and \(y=g(x)\) is

\[
A = \int_a^b \big(f(x) - g(x)\big)\,dx.
\]

Example: Find the area between \(y=x^2\) and \(y=x\) on \([0,1]\).

\[
A = \int_0^1 (x - x^2)\,dx = \sinistra[\tfrac{1}{2}x^2 - \tfrac{1}{3}x^3\right]_0^1 = \tfrac{1}{6}.
\]

\subsubsection{Volumes by Slicing}\label{volumes-by-slicing}

If a solid has cross-sectional area \(A(x)\) at position \(x\), then the
volume is

\[
V = \int_a^b A(x)\,dx.
\]

\subsubsection{Volumi della Rivoluzione}\label{volumi-della-rivoluzione}

Quando una regione viene ruotata attorno a un asse, il volume del solido
risultante può essere trovato con l'integrazione.

\begin{enumerate}
\def\labelenumi{\arabic{enumi}.}
\tightlist
\item
  Metodo del discoSe la regione in \(y=f(x)\), \(x\in[a,b]\), ruota
  attorno all'asse \(x\):
\end{enumerate}

\[
V = \pi \int_a^b [f(x)]^2\,dx.
\]

\begin{enumerate}
\def\labelenumi{\arabic{enumi}.}
\setcounter{enumi}{1}
\tightlist
\item
  Metodo della rondella Se la regione tra \(y=f(x)\) e \(y=g(x)\) ruota
  attorno all'asse \(x\):
\end{enumerate}

\[
V = \pi \int_a^b \Big([f(x)]^2 - [g(x)]^2\Big)\,dx.
\]

\begin{enumerate}
\def\labelenumi{\arabic{enumi}.}
\setcounter{enumi}{2}
\tightlist
\item
  Metodo della shell Se la regione in \(y=f(x)\) ruota attorno all'asse
  \(y\):
\end{enumerate}

\[
V = 2\pi \int_a^b x f(x)\,dx.
\]

\subsubsection{Esempi}\label{esempi-12}

\begin{enumerate}
\def\labelenumi{\arabic{enumi}.}
\tightlist
\item
  Metodo del disco Ruota \(y=\sqrt{x}\), \(0 \leq x \leq 4\), attorno
  all'asse \(x\):
\end{enumerate}

\[
V = \pi \int_0^4 (\sqrt{x})^2\,dx = \pi \int_0^4 x\,dx = \pi \left[\tfrac{1}{2}x^2\right]_0^4 = 8\pi.
\]

\begin{enumerate}
\def\labelenumi{\arabic{enumi}.}
\setcounter{enumi}{1}
\tightlist
\item
  Metodo della rondella Ruota la regione tra \(y=\sqrt{x}\) e \(y=1\),
  \(0 \leq x \leq 1\), attorno all'asse \(x\):
\end{enumerate}

\[
V = \pi \int_0^1 \big((\sqrt{x})^2 - (1)^2\big)\,dx = \pi \int_0^1 (x-1)\,dx = -\tfrac{\pi}{2}.
\]

(Prendi il valore assoluto per il volume: \(V = \tfrac{\pi}{2}\)).

\begin{enumerate}
\def\labelenumi{\arabic{enumi}.}
\setcounter{enumi}{2}
\tightlist
\item
  Metodo della shell Ruota la regione sotto \(y=x\),
  \(0 \leq x \leq 1\), attorno all'asse \(y\):
\end{enumerate}

\[
V = 2\pi \int_0^1 x(x)\,dx = 2\pi \int_0^1 x^2\,dx = 2\pi \cdot \tfrac{1}{3} = \tfrac{2\pi}{3}.
\]

\subsubsection{Perché è importante}\label{perchuxe9-uxe8-importante-4}

\begin{itemize}
\tightlist
\item
  Fornisce modi esatti per calcolare aree e volumi in geometria.
\item
  Essenziale in fisica, ingegneria e probabilità.
\item
  Introduce il pensiero geometrico con l'integrazione.
\end{itemize}

\subsubsection{Esercizi}\label{esercizi-20}

\begin{enumerate}
\def\labelenumi{\arabic{enumi}.}
\tightlist
\item
  Trova l'area tra \(y=\cos x\) e \(y=\sin x\) su \([0, \pi/2]\).
\item
  Calcolare il volume del solido formato ruotando \(y=x^2\),
  \(0 \leq x \leq 1\), attorno all'asse \(x\).
\item
  Trova il volume del solido formato ruotando la regione tra \(y=x\) e
  \(y=\sqrt{x}\) su \([0,1]\) attorno all'asse \(y\).
\item
  Utilizzare il metodo della rondella per calcolare il volume del solido
  formato ruotando \(y=\sqrt{1-x^2}\) (un semicerchio) attorno all'asse
  \(x\).
\item
  Trova l'area racchiusa tra \(y=x^2+1\) e \(y=3x\).
\end{enumerate}

\subsection{6.2 Lunghezza dell'arco e area della
superficieL'integrazione può essere utilizzata anche per misurare la
lunghezza delle curve e l'area superficiale dei solidi generati dalle
curve
rotanti.}\label{lunghezza-dellarco-e-area-della-superficielintegrazione-puuxf2-essere-utilizzata-anche-per-misurare-la-lunghezza-delle-curve-e-larea-superficiale-dei-solidi-generati-dalle-curve-rotanti.}

\subsubsection{Lunghezza dell'arco}\label{lunghezza-dellarco}

Per una curva uniforme \(y=f(x)\) nell'intervallo \([a,b]\), la
lunghezza della curva è

\[
L = \int_a^b \sqrt{1 + \big(f'(x)\big)^2}\,dx.
\]

Ciò deriva dall'approssimazione della curva con segmenti di linea e dal
calcolo del limite.

Esempio: Trova la lunghezza di \(y=\tfrac{1}{2}x^{3/2}\) da \(x=0\) a
\(x=4\).

\begin{itemize}
\tightlist
\item
  Derivato: \(f'(x) = \tfrac{3}{4}\sqrt{x}\). -Formula:
\end{itemize}

\[
L = \int_0^4 \sqrt{1 + \Big(\tfrac{3}{4}\sqrt{x}\Big)^2}\,dx
= \int_0^4 \sqrt{1 + \tfrac{9}{16}x}\,dx.
\]

Questo integrale può essere valutato utilizzando la sostituzione.

\subsubsection{Area superficiale di
rivoluzione}\label{area-superficiale-di-rivoluzione}

Se una curva \(y=f(x)\), \(a \leq x \leq b\), viene fatta ruotare
attorno all'asse \(x\), l'area della superficie del solido risultante è

\[
S = 2\pi \int_a^b f(x)\sqrt{1 + \big(f'(x)\big)^2}\,dx.
\]

Se ruotato attorno all'asse \(y\):

\[
S = 2\pi \int_a^b x \sqrt{1 + \big(f'(x)\big)^2}\,dx.
\]

\subsubsection{Esempi}\label{esempi-13}

\begin{enumerate}
\def\labelenumi{\arabic{enumi}.}
\tightlist
\item
  Lunghezza dell'arco di una linea Per \(y=x\), \(0 \leq x \leq 3\):
\end{enumerate}

\[
L = \int_0^3 \sqrt{1+(1)^2}\,dx = \int_0^3 \sqrt{2}\,dx = 3\sqrt{2}.
\]

\begin{enumerate}
\def\labelenumi{\arabic{enumi}.}
\setcounter{enumi}{1}
\tightlist
\item
  Area superficiale di una sfera Prendi \(y = \sqrt{r^2 - x^2}\),
  \(-r \leq x \leq r\) e ruota attorno all'asse \(x\).
\end{enumerate}

\[
S = 2\pi \int_{-r}^r \sqrt{r^2 - x^2}\sqrt{1+\left(\frac{-x}{\sqrt{r^2-x^2}}\right)^2}\,dx.
\]

La semplificazione fornisce \(S = 4\pi r^2\), la formula familiare per
la superficie di una sfera.

\subsubsection{Perché è importante}\label{perchuxe9-uxe8-importante-5}

\begin{itemize}
\tightlist
\item
  La lunghezza dell'arco estende l'idea di distanza ai percorsi curvi.
\item
  L'area superficiale di rivoluzione ha applicazioni in fisica,
  ingegneria e design.
\item
  Fornisce un ponte tra calcolo e geometria.
\end{itemize}

\subsubsection{Esercizi}\label{esercizi-21}

\begin{enumerate}
\def\labelenumi{\arabic{enumi}.}
\tightlist
\item
  Trova la lunghezza dell'arco di \(y=\sqrt{x}\) da \(x=0\) a \(x=4\).2.
  Calcola la superficie del solido ottenuto ruotando \(y=x^2\),
  \(0 \leq x \leq 1\), attorno all'asse \(x\).
\item
  Trova la lunghezza dell'arco di \(y=\ln(\cosh x)\) da \(x=0\) a
  \(x=1\).
\item
  Dimostra che ruotando \(y=\sqrt{r^2 - x^2}\) da \(0\) a \(r\) attorno
  all'asse \(x\) si ottiene metà della superficie di una sfera.
\item
  Deriva la formula per la superficie di un cono ruotando una linea.
\end{enumerate}

\subsection{6.3 Lavoro e medie}\label{lavoro-e-medie}

L'integrazione non si limita alla geometria. Aiuta anche a calcolare il
lavoro svolto da una forza e il valore medio di una funzione in un
intervallo.

\subsubsection{Lavoro}\label{lavoro}

Se una forza variabile \(F(x)\) sposta un oggetto lungo una linea retta
da \(x=a\) a \(x=b\), il lavoro totale è

\[
W = \int_a^b F(x)\,dx.
\]

Questa formula generalizza il caso semplice \(W = F \cdot d\) per forza
costante.

Esempio 1: Forza elastica (Legge di Hooke) Per una molla allungata dalla
lunghezza \(a\) a \(b\), con forza \(F(x) = kx\):

\[
W = \int_a^b kx\,dx = \tfrac{1}{2}k(b^2-a^2).
\]

Esempio 2: pompaggio dell'acqua Se l'acqua viene pompata da un
serbatoio, il lavoro richiesto è uguale

\[
W = \int_a^b \text{(weight density)} \times \text{(cross-sectional area)} \times \text{(distance lifted)} \, dx.
\]

\subsubsection{Valore medio di una
funzione}\label{valore-medio-di-una-funzione}

Il valore medio di una funzione continua \(f(x)\) su \([a,b]\) è

\[
f_{\text{avg}} = \frac{1}{b-a} \int_a^b f(x)\,dx.
\]

Questo è l'analogo continuo della media di un elenco di numeri.

Esempio 1: Per \(f(x)=x^2\) su \([0,2]\):

\[
f_{\text{avg}} = \tfrac{1}{2-0}\int_0^2 x^2 dx = \tfrac{1}{2}\cdot \tfrac{8}{3} = \tfrac{4}{3}.
\]

Esempio 2: Se la velocità di una particella è \(v(t)\), la velocità
media su \([a,b]\) è

\[
v_{\text{avg}} = \frac{1}{b-a}\int_a^b v(t)\,dt.
\]

\subsubsection{Perché è importante}\label{perchuxe9-uxe8-importante-6}

\begin{itemize}
\tightlist
\item
  Gli integrali del lavoro compaiono nei calcoli di fisica, ingegneria e
  energia.- Il valore medio fornisce un unico numero rappresentativo per
  quantità variabili.
\item
  Entrambi collegano il calcolo infinitesimale ai problemi del mondo
  reale di movimento, forza ed efficienza.
\end{itemize}

\subsubsection{Esercizi}\label{esercizi-22}

\begin{enumerate}
\def\labelenumi{\arabic{enumi}.}
\tightlist
\item
  Calcola il lavoro richiesto per allungare una molla da 2 m a 5 m se
  \(k=10\).
\item
  Un oggetto di 100 kg viene sollevato verticalmente per 5 m in un campo
  gravitazionale (\(g=9.8 \,\text{m/s}^2\)). Esprimere il lavoro come
  integrale e valutare.
\item
  Trova il valore medio di \(f(x)=\sin x\) su \([0,\pi]\).
\item
  Calcola la temperatura media se \(T(t)=20+5\cos(\tfrac{\pi t}{12})\)
  nell'arco di 24 ore.
\item
  Un serbatoio profondo 10 m è pieno d'acqua. Calcola il lavoro
  richiesto per pompare tutta l'acqua verso l'alto, dato che l'acqua
  pesa \(9800 \,\text{N/m}^3\).
\end{enumerate}

\subsection{6.4 Densità di probabilità e distribuzioni
continue}\label{densituxe0-di-probabilituxe0-e-distribuzioni-continue}

L'integrazione gioca un ruolo centrale anche nella teoria della
probabilità, soprattutto per le variabili casuali continue. Invece di
risultati discreti, descriviamo le probabilità con funzioni chiamate
funzioni di densità di probabilità (pdf).

\subsubsection{Funzioni di densità di
probabilità}\label{funzioni-di-densituxe0-di-probabilituxe0}

Una funzione di densità di probabilità \(f(x)\) deve soddisfare due
condizioni:

\begin{enumerate}
\def\labelenumi{\arabic{enumi}.}
\item
  \(f(x) \geq 0\) per tutti i \(x\).
\item
  L'area totale sotto la curva è 1:

  \[
  \int_{-\infty}^\infty f(x)\,dx = 1.
  \]
\end{enumerate}

Se \(X\) è una variabile casuale continua con pdf \(f(x)\), allora la
probabilità che \(X\) sia compresa tra \(a\) e \(b\) è

\[
P(a \leq X \leq b) = \int_a^b f(x)\,dx.
\]

\subsubsection{Funzione di distribuzione
cumulativa}\label{funzione-di-distribuzione-cumulativa}

La funzione di distribuzione cumulativa (cdf) è definita come

\[
F(x) = \int_{-\infty}^x f(t)\,dt.
\]

Fornisce la probabilità che la variabile casuale sia inferiore o uguale
a \(x\).

\subsubsection{Valore atteso (media)}\label{valore-atteso-media}

Il valore atteso di una variabile casuale continua è la media ponderata:

\[
E[X] = \int_{-\infty}^\infty x f(x)\,dx.
\]

\subsubsection{Esempi}\label{esempi-14}

\begin{enumerate}
\def\labelenumi{\arabic{enumi}.}
\tightlist
\item
  Distribuzione uniformePer \(f(x) = \tfrac{1}{b-a}\) su \([a,b]\):
\end{enumerate}

\begin{itemize}
\item
  Probabilità dell'intervallo \([c,d]\):

  \[
  P(c \leq X \leq d) = \frac{d-c}{b-a}.
  \]
\item
  Valore previsto: \(E[X] = \tfrac{a+b}{2}\).
\end{itemize}

\begin{enumerate}
\def\labelenumi{\arabic{enumi}.}
\setcounter{enumi}{1}
\tightlist
\item
  Distribuzione esponenziale Per \(f(x) = \lambda e^{-\lambda x}\),
  \(x \geq 0\):
\end{enumerate}

\begin{itemize}
\tightlist
\item
  \(\int_0^\infty \lambda e^{-\lambda x}\,dx = 1\).
\item
  Significa: \(E[X] = \tfrac{1}{\lambda}\).
\end{itemize}

\begin{enumerate}
\def\labelenumi{\arabic{enumi}.}
\setcounter{enumi}{2}
\tightlist
\item
  Distribuzione normale La curva a campana:
\end{enumerate}

\[
f(x) = \frac{1}{\sqrt{2\pi\sigma^2}} e^{-\frac{(x-\mu)^2}{2\sigma^2}}.
\]

Si integra a 1, ma richiede tecniche avanzate.

\subsubsection{Perché è importante}\label{perchuxe9-uxe8-importante-7}

\begin{itemize}
\tightlist
\item
  Le densità di probabilità descrivono l'incertezza nella scienza,
  nell'ingegneria e nella statistica.
\item
  Gli integrali collegano le aree sotto le curve alle probabilità.
\item
  Le distribuzioni continue generalizzano l'idea di contare i risultati
  per misurare le probabilità su intervalli.
\end{itemize}

\subsubsection{Esercizi}\label{esercizi-23}

\begin{enumerate}
\def\labelenumi{\arabic{enumi}.}
\tightlist
\item
  Mostra che la densità uniforme \(f(x) = \tfrac{1}{b-a}\) su \([a,b]\)
  si integra con 1.
\item
  Per la distribuzione esponenziale con \(\lambda = 2\), calcolare
  \(P(0 \leq X \leq 1)\).
\item
  Trova il valore previsto di \(X\) se \(f(x) = 3x^2\) su \([0,1]\).
\item
  Verifica che la distribuzione normale con media 0 e varianza 1 ha
  probabilità totale 1 (non è necessaria una prova completa, ma spiega
  perché vale).
\item
  Calcola il cdf della distribuzione uniforme su \([0,1]\).
\end{enumerate}

\section{Parte III. Calcolo
multivariabile}\label{parte-iii.-calcolo-multivariabile}

\section{Capitolo 7. Funzioni vettoriali e
curve}\label{capitolo-7.-funzioni-vettoriali-e-curve}

\subsection{7.1 Funzioni vettoriali e curve
spaziali}\label{funzioni-vettoriali-e-curve-spaziali}

Nel calcolo multivariabile, le funzioni possono restituire vettori
anziché numeri. Queste sono chiamate funzioni a valori vettoriali e sono
essenziali per descrivere le curve nello spazio.

\subsubsection{Definizione}\label{definizione-6}

Una funzione vettoriale è una funzione della forma

\[
\mathbf{r}(t) = \langle x(t), y(t), z(t) \rangle,
\]

dove \(x(t), y(t), z(t)\) sono funzioni a valori reali.

\begin{itemize}
\tightlist
\item
  L'input \(t\) è spesso chiamato parametro.- L'output è un vettore
  nello spazio 2D o 3D.
\item
  Il grafico di una funzione vettoriale in 3D è una curva spaziale.
\end{itemize}

\subsubsection{Esempi}\label{esempi-15}

\begin{enumerate}
\def\labelenumi{\arabic{enumi}.}
\tightlist
\item
  Linea
\end{enumerate}

\[
\mathbf{r}(t) = \langle 1+2t, \; 3-t, \; 4+5t \rangle.
\]

Descrive una linea retta che passa per il punto \((1,3,4)\) con il
vettore di direzione \(\langle 2,-1,5 \rangle\).

\begin{enumerate}
\def\labelenumi{\arabic{enumi}.}
\setcounter{enumi}{1}
\tightlist
\item
  Cerchio nell'aereo
\end{enumerate}

\[
\mathbf{r}(t) = \langle \cos t, \; \sin t, \; 0 \rangle, \quad 0 \leq t < 2\pi.
\]

\begin{enumerate}
\def\labelenumi{\arabic{enumi}.}
\setcounter{enumi}{2}
\tightlist
\item
  Elica
\end{enumerate}

\[
\mathbf{r}(t) = \langle \cos t, \; \sin t, \; t \rangle.
\]

Questa è una spirale che sale attorno all'asse \(z\).

\subsubsection{Limiti e continuità}\label{limiti-e-continuituxe0}

Una funzione vettoriale è continua in \(t=a\) se ciascun componente
\(x(t), y(t), z(t)\) è continuo in \(t=a\).

\[
\lim_{t \to a} \mathbf{r}(t) = \langle \lim_{t \to a} x(t), \; \lim_{t \to a} y(t), \; \lim_{t \to a} z(t) \rangle.
\]

\subsubsection{Geometria delle curve
spaziali}\label{geometria-delle-curve-spaziali}

\begin{itemize}
\tightlist
\item
  Ogni curva ha una direzione tangente data dalla derivata.
\item
  Le curve spaziali possono modellare percorsi di movimento, traiettorie
  di particelle e forme geometriche.
\end{itemize}

\subsubsection{Perché è importante}\label{perchuxe9-uxe8-importante-8}

Le funzioni vettoriali sono la base del calcolo multivariabile,
consentendoci di estendere le idee di derivate e integrali a dimensioni
più elevate. Appaiono naturalmente anche in fisica (movimento in 3D,
elettromagnetismo, fluidodinamica).

\subsubsection{Esercizi}\label{esercizi-24}

\begin{enumerate}
\def\labelenumi{\arabic{enumi}.}
\tightlist
\item
  Scrivi una funzione vettoriale per una linea passante per \((0,1,2)\)
  parallela al vettore \(\langle 3,-2,1 \rangle\).
\item
  Descrivi la curva data da
  \(\mathbf{r}(t) = \langle 2\cos t, \; 2\sin t, \; 3 \rangle\).
\item
  Determina se \(\mathbf{r}(t) = \langle e^t, \; \ln t, \; t^2 \rangle\)
  è continuo in \(t=1\).
\item
  Disegna l'elica
  \(\mathbf{r}(t) = \langle \cos t, \; \sin t, \; 2t \rangle\).
\item
  Trova il punto sulla curva
  \(\mathbf{r}(t) = \langle t, \; t^2, \; t^3 \rangle\) quando \(t=2\).
\end{enumerate}

\subsection{7.2 Derivate e integrali di funzioni vettorialiLe funzioni
vettoriali possono essere differenziate e integrate proprio come le
funzioni ordinarie: applichiamo semplicemente l'operazione a ciascun
componente. Ciò ci consente di studiare il movimento, la velocità,
l'accelerazione e l'accumulo in dimensioni
superiori.}\label{derivate-e-integrali-di-funzioni-vettorialile-funzioni-vettoriali-possono-essere-differenziate-e-integrate-proprio-come-le-funzioni-ordinarie-applichiamo-semplicemente-loperazione-a-ciascun-componente.-ciuxf2-ci-consente-di-studiare-il-movimento-la-velocituxe0-laccelerazione-e-laccumulo-in-dimensioni-superiori.}

\subsubsection{Derivato di una funzione
vettoriale}\label{derivato-di-una-funzione-vettoriale}

Se

\[
\mathbf{r}(t) = \langle x(t), y(t), z(t) \rangle,
\]

allora

\[
\mathbf{r}'(t) = \langle x'(t), y'(t), z'(t) \rangle.
\]

Questo vettore derivato punta nella direzione tangente alla curva nel
parametro \(t\).

\begin{itemize}
\tightlist
\item
  Velocità: se \(\mathbf{r}(t)\) fornisce la posizione di una particella
  nel momento \(t\), allora \(\mathbf{v}(t) = \mathbf{r}'(t)\) è il suo
  vettore di velocità.
\item
  Velocità: la grandezza \(|\mathbf{v}(t)|\) è la velocità della
  particella.
\item
  Accelerazione: \(\mathbf{a}(t) = \mathbf{v}'(t) = \mathbf{r}''(t)\).
\end{itemize}

\subsubsection{Esempi}\label{esempi-16}

\begin{enumerate}
\def\labelenumi{\arabic{enumi}.}
\tightlist
\item
  Elica
\end{enumerate}

\[
\mathbf{r}(t) = \langle \cos t, \sin t, t \rangle.
\]

\begin{itemize}
\tightlist
\item
  Velocità: \(\mathbf{v}(t) = \langle -\sin t, \cos t, 1 \rangle\).
\item
  Velocità:
  \(|\mathbf{v}(t)| = \sqrt{(-\sin t)^2 + (\cos t)^2 + 1^2} = \sqrt{2}\).
\item
  Accelerazione:
  \(\mathbf{a}(t) = \langle -\cos t, -\sin t, 0 \rangle\).
\end{itemize}

\begin{enumerate}
\def\labelenumi{\arabic{enumi}.}
\setcounter{enumi}{1}
\tightlist
\item
  Movimento del proiettile
\end{enumerate}

\[
\mathbf{r}(t) = \langle v_0 \cos\theta \cdot t, \; v_0 \sin\theta \cdot t - \tfrac{1}{2}gt^2 \rangle.
\]

Questo modella il percorso parabolico di un proiettile soggetto a
gravità.

\subsubsection{Integrale di una funzione
vettoriale}\label{integrale-di-una-funzione-vettoriale}

Se

\[
\mathbf{r}(t) = \langle x(t), y(t), z(t) \rangle,
\]

allora

\[
\int \mathbf{r}(t)\,dt = \left\langle \int x(t)\,dt, \; \int y(t)\,dt, \; \int z(t)\,dt \right\rangle + \mathbf{C},
\]

dove \(\mathbf{C}\) è un vettore costante.

\subsubsection{Esempio}\label{esempio}

\[
\mathbf{r}(t) = \langle t, t^2, t^3 \rangle.
\]

\begin{itemize}
\tightlist
\item
  Derivato: \(\mathbf{r}'(t) = \langle 1, 2t, 3t^2 \rangle\).
\item
  Integrale:
\end{itemize}

\[
\int \mathbf{r}(t)\,dt = \langle \tfrac{1}{2}t^2, \tfrac{1}{3}t^3, \tfrac{1}{4}t^4 \rangle + \mathbf{C}.
\]

\subsubsection{Perché è importante- Le derivate delle funzioni
vettoriali descrivono il movimento e le forze nello
spazio.}\label{perchuxe9-uxe8-importante--le-derivate-delle-funzioni-vettoriali-descrivono-il-movimento-e-le-forze-nello-spazio.}

\begin{itemize}
\tightlist
\item
  Gli integrali danno spostamento, lavoro e quantità accumulate.
\item
  Questi strumenti collegano il calcolo infinitesimale direttamente alla
  fisica e all'ingegneria.
\end{itemize}

\subsubsection{Esercizi}\label{esercizi-25}

\begin{enumerate}
\def\labelenumi{\arabic{enumi}.}
\tightlist
\item
  Per \(\mathbf{r}(t) = \langle t, \cos t, \sin t \rangle\), trova
  velocità, velocità e accelerazione.
\item
  Calcola \(\mathbf{r}'(t)\) per
  \(\mathbf{r}(t) = \langle e^t, \ln t, t^2 \rangle\).
\item
  Integra \(\mathbf{r}(t) = \langle 1, t, t^2 \rangle\).
\item
  Una particella ha velocità
  \(\mathbf{v}(t) = \langle t, 2, 0 \rangle\). Trova il suo vettore di
  posizione se \(\mathbf{r}(0) = \langle 1, 0, 0 \rangle\).
\item
  Mostra che la velocità di
  \(\mathbf{r}(t) = \langle \cos t, \sin t, 0 \rangle\) è costante.
\end{enumerate}

\subsection{7.3 Lunghezza e curvatura
dell'arco}\label{lunghezza-e-curvatura-dellarco}

Il calcolo vettoriale fornisce strumenti per misurare non solo il
percorso tracciato da una curva ma anche la sua curvatura. Questi sono
espressi attraverso la lunghezza dell'arco e la curvatura.

\subsubsection{Lunghezza dell'arco di una curva
spaziale}\label{lunghezza-dellarco-di-una-curva-spaziale}

Se una curva è data da

\[
\mathbf{r}(t) = \langle x(t), y(t), z(t) \rangle, \quad a \leq t \leq b,
\]

allora la lunghezza dell'arco è

\[
L = \int_a^b |\mathbf{r}'(t)|\,dt,
\]

dove

\[
|\mathbf{r}'(t)| = \sqrt{(x'(t))^2 + (y'(t))^2 + (z'(t))^2}.
\]

Esempio: Per l'elica
\(\mathbf{r}(t) = \langle \cos t, \sin t, t \rangle, \, 0 \leq t \leq 2\pi\):

\begin{itemize}
\tightlist
\item
  Velocità: \(\mathbf{r}'(t) = \langle -\sin t, \cos t, 1 \rangle\).
\item
  Velocità:
  \(|\mathbf{r}'(t)| = \sqrt{(-\sin t)^2 + (\cos t)^2 + 1^2} = \sqrt{2}\).
\item
  Lunghezza dell'arco:
\end{itemize}

\[
L = \int_0^{2\pi} \sqrt{2}\,dt = 2\pi\sqrt{2}.
\]

\subsubsection{Curvatura}\label{curvatura}

La curvatura misura la velocità con cui una curva cambia direzione.

Per una curva morbida \(\mathbf{r}(t)\):

\[
\kappa(t) = \frac{|\mathbf{r}'(t) \times \mathbf{r}''(t)|}{|\mathbf{r}'(t)|^3}.
\]

\begin{itemize}
\tightlist
\item
  \(\kappa = 0\): linea retta.
\item
  Più grande \(\kappa\): la curva si piega più bruscamente.
\end{itemize}

Esempio: Per un cerchio di raggio \(r\):\[
\mathbf{r}(t) = \langle r\cos t, r\sin t \rangle.
\]

Quindi \(\kappa = \tfrac{1}{r}\). Quindi la curvatura è costante e
inversamente proporzionale al raggio.

\subsubsection{Unità tangenti e vettori
normali}\label{unituxe0-tangenti-e-vettori-normali}

\begin{itemize}
\tightlist
\item
  Vettore tangente:
\end{itemize}

\[
\mathbf{T}(t) = \frac{\mathbf{r}'(t)}{|\mathbf{r}'(t)|}.
\]

\begin{itemize}
\tightlist
\item
  Vettore normale: punta verso il centro di curvatura, definito come
\end{itemize}

\[
\mathbf{N}(t) = \frac{\mathbf{T}'(t)}{|\mathbf{T}'(t)|}.
\]

Questi vettori descrivono la geometria del movimento: direzione di
viaggio e direzione di svolta.

\subsubsection{Perché è importante}\label{perchuxe9-uxe8-importante-9}

\begin{itemize}
\tightlist
\item
  La lunghezza dell'arco generalizza il concetto di distanza dalle curve
  nello spazio.
\item
  La curvatura descrive la flessione, cruciale in fisica (accelerazione
  centripeta), ingegneria (strade, montagne russe) e grafica
  computerizzata.
\end{itemize}

\subsubsection{Esercizi}\label{esercizi-26}

\begin{enumerate}
\def\labelenumi{\arabic{enumi}.}
\tightlist
\item
  Trova la lunghezza dell'arco di
  \(\mathbf{r}(t) = \langle t, t^2, 0 \rangle\) da \(t=0\) a \(t=1\).
\item
  Calcola la curvatura del cerchio
  \(\mathbf{r}(t) = \langle \cos t, \sin t \rangle\).
\item
  Per \(\mathbf{r}(t) = \langle t, \cos t, \sin t \rangle\), calcolare
  \(|\mathbf{r}'(t)|\).
\item
  Mostra che una linea retta ha curvatura \(\kappa = 0\).
\item
  Trova il vettore tangente a
  \(\mathbf{r}(t) = \langle e^t, e^{-t}, t \rangle\) in \(t=0\).
\end{enumerate}

\subsection{7.4 Movimento nello spazio}\label{movimento-nello-spazio}

Le funzioni vettoriali sono particolarmente potenti nel descrivere il
movimento in due o tre dimensioni. Posizione, velocità e accelerazione
sono naturalmente espresse utilizzando derivate e integrali di funzioni
a valori vettoriali.

\subsubsection{Posizione, velocità e
accelerazione}\label{posizione-velocituxe0-e-accelerazione}

\begin{itemize}
\tightlist
\item
  Vettore di posizione:
\end{itemize}

\[
\mathbf{r}(t) = \langle x(t), y(t), z(t) \rangle
\]

\begin{itemize}
\tightlist
\item
  Vettore velocità (derivata della posizione):
\end{itemize}

\[
\mathbf{v}(t) = \mathbf{r}'(t) = \langle x'(t), y'(t), z'(t) \rangle
\]

\begin{itemize}
\tightlist
\item
  Velocità (entità della velocità):
\end{itemize}

\[
|\mathbf{v}(t)| = \sqrt{(x'(t))^2 + (y'(t))^2 + (z'(t))^2}
\]

\begin{itemize}
\tightlist
\item
  Vettore accelerazione (derivata della velocità):
\end{itemize}

\[\mathbf{a}(t) = \mathbf{v}'(t) = \mathbf{r}''(t).
\]

\subsubsection{Tangential and Normal
Components}\label{tangential-and-normal-components}

Acceleration can be decomposed into two components:

\[
\mathbf{a}(t) = a_T \mathbf{T}(t) + a_N \mathbf{N}(t),
\]

where:

\begin{itemize}
\tightlist
\item
  \(\mathbf{T}(t)\) = unit tangent vector,
\item
  \(\mathbf{N}(t)\) = principal normal vector,
\item
  \(a_T = \frac{d}{dt}|\mathbf{v}(t)|\) = tangential acceleration
  (change in speed),
\item
  \(a_N = \kappa |\mathbf{v}(t)|^2\) = normal acceleration (change in
  direction).
\end{itemize}

\subsubsection{Projectile Motion in 3D}\label{projectile-motion-in-3d}

With gravity acting in the \(-z\) direction:

\[
\mathbf{r}(t) = \langle v_0 \cos\theta \cos\phi \cdot t,\; v_0 \cos\theta \sin\phi \cdot t,\; v_0 \sin\theta \cdot t - \tfrac{1}{2}gt^2 \rangle,
\]

where \(v_0\) is initial speed, \(\theta\) launch angle, and \(\phi\)
azimuthal direction.

\subsubsection{Example: Helical Motion}\label{example-helical-motion}

\[
\mathbf{r}(t) = \langle \cos t, \sin t, t \rangle
\]

\begin{itemize}
\tightlist
\item
  Velocità: \(\mathbf{v}(t) = \langle -\sin t, \cos t, 1 \rangle\).
\item
  Velocità: \(|\mathbf{v}(t)| = \sqrt{2}\).
\item
  Accelerazione:
  \(\mathbf{a}(t) = \langle -\cos t, -\sin t, 0 \rangle\).
\item
  Il movimento ha una velocità uniforme e procede a spirale verso
  l'alto.
\end{itemize}

\subsubsection{Perché è importante}\label{perchuxe9-uxe8-importante-10}

\begin{itemize}
\tightlist
\item
  Fornisce il linguaggio matematico per il movimento nel mondo reale.
\item
  Essenziale in fisica (forze, traiettorie, moto circolare).
\item
  Fondamenti per la meccanica avanzata e modelli ingegneristici.
\end{itemize}

\subsubsection{Esercizi}\label{esercizi-27}

\begin{enumerate}
\def\labelenumi{\arabic{enumi}.}
\tightlist
\item
  Una particella si muove lungo
  \(\mathbf{r}(t) = \langle t, t^2, t^3 \rangle\). Trova velocità e
  accelerazione in \(t=1\).
\item
  Mostra che la velocità è costante per l'elica
  \(\mathbf{r}(t) = \langle \cos t, \sin t, t \rangle\).
\item
  Un proiettile viene lanciato con \(v_0 = 20 \,\text{m/s}\) con un
  angolo \(45^\circ\). Scrivi il suo vettore posizione assumendo il moto
  su un piano verticale.
\item
  Per \(\mathbf{r}(t) = \langle e^t, e^{-t}, t \rangle\), trovare
  \(\mathbf{v}(t)\) e \(\mathbf{a}(t)\).5. Scomporre il vettore
  accelerazione in componenti tangenziali e normali per il movimento
  lungo un cerchio di raggio \(r\).
\end{enumerate}

\section{Capitolo 8. Funzioni di più
variabili}\label{capitolo-8.-funzioni-di-piuxf9-variabili}

\subsection{8.1 Limiti e continuità in più
variabili}\label{limiti-e-continuituxe0-in-piuxf9-variabili}

Nel calcolo multivariabile, le funzioni possono dipendere da due o più
variabili, come \(f(x,y)\) o \(f(x,y,z)\). I concetti di limite e
continuità si estendono naturalmente dal calcolo a variabile singola, ma
sono più sottili perché dobbiamo considerare tutti i possibili percorsi
di approccio.

\subsubsection{Limiti in due variabili}\label{limiti-in-due-variabili}

Per una funzione \(f(x,y)\), diciamo

\[
\lim_{(x,y) \to (a,b)} f(x,y) = L
\]

se \(f(x,y)\) si avvicina arbitrariamente a \(L\) mentre \((x,y)\) si
avvicina a \((a,b)\) lungo qualsiasi percorso.

Se percorsi diversi danno valori limite diversi, allora il limite non
esiste.

Esempio 1 (il limite esiste):

\[
f(x,y) = x^2 + y^2, \quad \lim_{(x,y) \to (0,0)} f(x,y) = 0.
\]

Esempio 2 (il limite non esiste):

\[
f(x,y) = \frac{xy}{x^2+y^2}, \quad (x,y) \to (0,0).
\]

\begin{itemize}
\tightlist
\item
  Insieme a \(y=0\), la funzione è 0.
\item
  Insieme a \(y=x\), la funzione è \(\tfrac{1}{2}\). Risultati diversi →
  il limite non esiste.
\end{itemize}

\subsubsection{Continuità}\label{continuituxe0}

Una funzione \(f(x,y)\) è continua in \((a,b)\) se

\[
\lim_{(x,y)\to(a,b)} f(x,y) = f(a,b).
\]

I polinomi e le funzioni razionali (dove denominatore ≠ 0) sono continui
ovunque nei loro domini.

\subsubsection{Estensione a tre o più
variabili}\label{estensione-a-tre-o-piuxf9-variabili}

Per \(f(x,y,z)\), limiti e continuità sono definiti allo stesso modo, ma
il punto \((a,b,c)\) deve essere avvicinato da infinite direzioni nello
spazio.

\subsubsection{Perché è importante}\label{perchuxe9-uxe8-importante-11}

\begin{itemize}
\tightlist
\item
  La continuità garantisce l'assenza di salti, buchi o asintoti nelle
  funzioni multivariabili.
\item
  I limiti sono fondamentali per definire le derivate parziali e gli
  integrali multipli.
\item
  Questi concetti sono elementi costitutivi del calcolo multivariabile.
\end{itemize}

\subsubsection{\texorpdfstring{Esercizi1. Determina se
\(\lim_{(x,y)\to(0,0)} (x^2+y^2)\)
esiste.}{Esercizi1. Determina se \textbackslash lim\_\{(x,y)\textbackslash to(0,0)\} (x\^{}2+y\^{}2) esiste.}}\label{esercizi1.-determina-se-lim_xyto00-x2y2-esiste.}

\begin{enumerate}
\def\labelenumi{\arabic{enumi}.}
\setcounter{enumi}{1}
\tightlist
\item
  Mostra che \(\lim_{(x,y)\to(0,0)} \frac{x^2y}{x^2+y^2} = 0\) lungo
  tutti i percorsi rettilinei \(y=mx\).
\item
  Esiste il limite per \(f(x,y) = \frac{x^2-y^2}{x^2+y^2}\) come
  \((x,y)\to(0,0)\)?
\item
  Spiega perché i polinomi in due variabili sono continui ovunque.
\item
  Fornisci un esempio di funzione di due variabili che è discontinua in
  un punto e spiega il motivo.
\end{enumerate}

\subsection{8.2 Derivate parziali}\label{derivate-parziali}

Nelle funzioni con più variabili, spesso vogliamo misurare come cambia
la funzione quando cambia solo una variabile mentre le altre vengono
mantenute costanti. Ciò porta all'idea delle derivate parziali.

\subsubsection{Definizione}\label{definizione-7}

Per una funzione \(f(x,y)\), la derivata parziale rispetto a \(x\) in un
punto \((a,b)\) è

\[
\frac{\partial f}{\partial x}(a,b) = \lim_{h \to 0} \frac{f(a+h, b) - f(a,b)}{h}.
\]

Allo stesso modo, la derivata parziale rispetto a \(y\) è

\[
\frac{\partial f}{\partial y}(a,b) = \lim_{h \to 0} \frac{f(a, b+h) - f(a,b)}{h}.
\]

Trattiamo tutte le altre variabili come costanti durante la
differenziazione.

\subsubsection{Notazione}\label{notazione-1}

\begin{itemize}
\tightlist
\item
  \(\frac{\partial f}{\partial x}\), \(f_x\), \(\partial_x f\).
\item
  \(\frac{\partial f}{\partial y}\), \(f_y\), \(\partial_y f\).
\end{itemize}

Per tre variabili \(f(x,y,z)\), abbiamo anche \(f_x, f_y, f_z\).

\subsubsection{Esempi}\label{esempi-17}

\begin{enumerate}
\def\labelenumi{\arabic{enumi}.}
\tightlist
\item
  \(f(x,y) = x^2y + y^3\)
\end{enumerate}

\begin{itemize}
\tightlist
\item
  \(f_x = 2xy\).
\item
  \(f_y = x^2 + 3y^2\).
\end{itemize}

\begin{enumerate}
\def\labelenumi{\arabic{enumi}.}
\setcounter{enumi}{1}
\tightlist
\item
  \(f(x,y) = e^{xy}\)
\end{enumerate}

\begin{itemize}
\tightlist
\item
  \(f_x = y e^{xy}\).
\item
  \(f_y = x e^{xy}\).
\end{itemize}

\begin{enumerate}
\def\labelenumi{\arabic{enumi}.}
\setcounter{enumi}{2}
\tightlist
\item
  \(f(x,y,z) = x^2 + yz\)
\end{enumerate}

\begin{itemize}
\tightlist
\item
  \(f_x = 2x\).
\item
  \(f_y = z\).
\item
  \(f_z = y\).
\end{itemize}

\subsubsection{Derivate parziali di ordine
superiore}\label{derivate-parziali-di-ordine-superiore}

Possiamo prendere ripetutamente le derivate parziali:

\begin{itemize}
\tightlist
\item
  \(f_{xx} = \frac{\partial}{\partial x}\Big(f_x\Big)\).
\item
  \(f_{yy}, f_{xy}, f_{yx}\), ecc.
\end{itemize}

Teorema di Clairaut: se \(f\) ha derivate seconde parziali continue,
allora

\[
f_{xy} = f_{yx}.
\]

\subsubsection{\texorpdfstring{Significato geometrico- \(f_x\): pendenza
della superficie nella direzione
\(x\).}{Significato geometrico- f\_x: pendenza della superficie nella direzione x.}}\label{significato-geometrico--f_x-pendenza-della-superficie-nella-direzione-x.}

\begin{itemize}
\tightlist
\item
  \(f_y\): pendenza della superficie nella direzione \(y\).
\item
  Insieme descrivono come si inclina la superficie.
\end{itemize}

\subsubsection{Perché è importante}\label{perchuxe9-uxe8-importante-12}

\begin{itemize}
\tightlist
\item
  Le derivate parziali sono il fondamento di gradienti, piani tangenti e
  ottimizzazione in più variabili.
\item
  Sono ampiamente utilizzati in fisica, ingegneria ed economia per
  modellare sistemi con diversi input.
\end{itemize}

\subsubsection{Esercizi}\label{esercizi-28}

\begin{enumerate}
\def\labelenumi{\arabic{enumi}.}
\tightlist
\item
  Trova \(f_x\) e \(f_y\) per \(f(x,y) = x^3y^2\).
\item
  Calcola \(f_x, f_y, f_z\) per \(f(x,y,z) = xyz + x^2\).
\item
  Verifica il teorema di Clairaut per \(f(x,y) = x^2y + y^3\).
\item
  Interpretare geometricamente cosa significano \(f_x\) e \(f_y\) per
  \(f(x,y) = \sqrt{x^2+y^2}\).
\item
  Trova tutte le derivate parziali del secondo ordine di
  \(f(x,y) = e^{x^2+y^2}\).
\end{enumerate}

\subsection{8.3 Derivate dei gradienti e
direzionali}\label{derivate-dei-gradienti-e-direzionali}

Le derivate parziali misurano il cambiamento lungo gli assi delle
coordinate, ma a volte vogliamo conoscere la velocità di cambiamento di
una funzione in qualsiasi direzione. Ciò porta ai concetti di gradiente
e derivate direzionali.

\subsubsection{Vettore gradiente}\label{vettore-gradiente}

Per una funzione \(f(x,y)\), il gradiente è il vettore

\[
\nabla f(x,y) = \left\langle \frac{\partial f}{\partial x}, \frac{\partial f}{\partial y} \right\rangle.
\]

Per tre variabili \(f(x,y,z)\):

\[
\nabla f(x,y,z) = \left\langle f_x, f_y, f_z \right\rangle.
\]

Il gradiente punta nella direzione del massimo aumento della funzione e
la sua grandezza dà la pendenza più ripida.

\subsubsection{Derivati direzionali}\label{derivati-direzionali}

Il tasso di variazione di \(f(x,y)\) in un punto nella direzione di un
vettore unitario \(\mathbf{u} = \langle u_1, u_2 \rangle\) è

\[
D_{\mathbf{u}} f(x,y) = \nabla f(x,y) \cdot \mathbf{u}.
\]

Questo è il prodotto scalare del gradiente con il vettore di direzione.

\subsubsection{Esempi}\label{esempi-18}

\begin{enumerate}
\def\labelenumi{\arabic{enumi}.}
\tightlist
\item
  \(f(x,y) = x^2 + y^2\)
\end{enumerate}

\begin{itemize}
\tightlist
\item
  Gradiente: \(\nabla f = \langle 2x, 2y \rangle\).
\item
  A (1,2): \(\nabla f = \langle 2,4 \rangle\).- Derivata direzionale
  lungo \(\mathbf{u} = \langle \tfrac{3}{5}, \tfrac{4}{5} \rangle\):
\end{itemize}

\[
D_{\mathbf{u}} f(1,2) = \langle 2,4 \rangle \cdot \langle \tfrac{3}{5}, \tfrac{4}{5} \rangle = \tfrac{26}{5}.
\]

\begin{enumerate}
\def\labelenumi{\arabic{enumi}.}
\setcounter{enumi}{1}
\tightlist
\item
  \(f(x,y,z) = x y z\)
\end{enumerate}

\begin{itemize}
\tightlist
\item
  Gradiente: \(\nabla f = \langle yz, xz, xy \rangle\).
\item
  A (1,1,1): \(\nabla f = \langle 1,1,1 \rangle\).
\item
  La direzione di aumento massima è lungo \(\langle 1,1,1 \rangle\).
\end{itemize}

\subsubsection{Interpretazione
geometrica}\label{interpretazione-geometrica-1}

\begin{itemize}
\tightlist
\item
  Il vettore del gradiente è perpendicolare (normale) alle curve di
  livello o alle superfici livellate di \(f\).
\item
  Le derivate direzionali generalizzano la pendenza in direzioni
  arbitrarie.
\end{itemize}

\subsubsection{Perché è importante}\label{perchuxe9-uxe8-importante-13}

\begin{itemize}
\tightlist
\item
  Nell'ottimizzazione, la pendenza ci dice la direzione in cui muoversi
  per la salita o la discesa più ripida.
\item
  In fisica, i gradienti descrivono campi come il flusso di calore e il
  potenziale elettrico.
\item
  I derivati \hspace{0pt}\hspace{0pt}direzionali uniscono i tassi di
  cambiamento a variabile singola e multivariabile.
\end{itemize}

\subsubsection{Esercizi}\label{esercizi-29}

\begin{enumerate}
\def\labelenumi{\arabic{enumi}.}
\tightlist
\item
  Calcola \(\nabla f(x,y)\) per \(f(x,y) = e^{xy}\).
\item
  Trova il gradiente di \(f(x,y,z) = x^2+y^2+z^2\) e valuta in (1,1,1).
\item
  Calcola la derivata direzionale di \(f(x,y) = x^2-y\) in (2,1) nella
  direzione di \(\mathbf{u} = \langle 0,1 \rangle\).
\item
  Mostra che il gradiente di \(f(x,y) = x^2+y^2\) è perpendicolare al
  cerchio \(x^2+y^2=1\).
\item
  Trova la direzione del vettore unitario che massimizza la derivata
  direzionale di \(f(x,y) = xy\) in (1,2).
\end{enumerate}

\subsection{8.4 Piani tangenti e approssimazioni
lineari}\label{piani-tangenti-e-approssimazioni-lineari}

Nel calcolo a variabile singola, la linea tangente approssima una curva
vicino a un punto. Nel calcolo multivariabile, il concetto analogo è il
piano tangente, che fornisce un'approssimazione lineare a una superficie
vicino a un punto.

\subsubsection{Piano tangente ad una
superficie}\label{piano-tangente-ad-una-superficie}

Supponiamo che \(z = f(x,y)\) sia differenziabile in \((a,b)\). Il piano
tangente a \((a,b,f(a,b))\) è dato da

\[
z = f(a,b) + f_x(a,b)(x-a) + f_y(a,b)(y-b).
\]Questo piano tocca la superficie in quel punto e la avvicina nelle
vicinanze.

\subsubsection{Esempio 1: paraboloide}\label{esempio-1-paraboloide}

Per \(f(x,y) = x^2 + y^2\) alle \((1,2)\):

\begin{itemize}
\tightlist
\item
  \(f(1,2) = 1^2+2^2=5\).
\item
  \(f_x = 2x\), quindi \(f_x(1,2) = 2\).
\item
  \(f_y = 2y\), quindi \(f_y(1,2) = 4\).
\end{itemize}

Equazione del piano tangente:

\[
z = 5 + 2(x-1) + 4(y-2).
\]

\subsubsection{Approssimazione lineare}\label{approssimazione-lineare}

Il piano tangente può essere utilizzato per approssimare \(f(x,y)\)
vicino a \((a,b)\):

\[
f(x,y) \approx f(a,b) + f_x(a,b)(x-a) + f_y(a,b)(y-b).
\]

Questa è la linearizzazione di \(f\) in \((a,b)\).

\subsubsection{Esempio 2: Approssimazione
lineare}\label{esempio-2-approssimazione-lineare}

Approssimativo \(f(x,y) = \sqrt{x+y}\) vicino a \((4,5)\).

\begin{itemize}
\tightlist
\item
  \(f(4,5) = \sqrt{9} = 3\).
\item
  \(f_x = \frac{1}{2\sqrt{x+y}}, \quad f_y = \frac{1}{2\sqrt{x+y}}\).
\item
  A (4,5): \(f_x = f_y = \tfrac{1}{6}\).
\end{itemize}

quindi,

\[
f(x,y) \approx 3 + \tfrac{1}{6}(x-4) + \tfrac{1}{6}(y-5).
\]

\subsubsection{Perché è importante}\label{perchuxe9-uxe8-importante-14}

\begin{itemize}
\tightlist
\item
  I piani tangenti danno la migliore approssimazione lineare ad una
  superficie.
\item
  La linearizzazione semplifica le funzioni complesse per il calcolo.
\item
  Ampiamente usato nei metodi numerici, nella fisica e nell'economia.
\end{itemize}

\subsubsection{Esercizi}\label{esercizi-30}

\begin{enumerate}
\def\labelenumi{\arabic{enumi}.}
\tightlist
\item
  Trova il piano tangente a \(z = x^2y + y^2\) in \((1,1)\).
\item
  Approssimativo \(f(x,y) = e^{x+y}\) vicino a \((0,0)\).
\item
  Derivare l'equazione del piano tangente per \(z = \ln(x^2+y^2)\) in
  \((1,1)\).
\item
  Utilizzare l'approssimazione lineare per stimare \(\sqrt{10.1}\)
  utilizzando \(f(x,y) = \sqrt{x+y}\) vicino a (4,6).
\item
  Spiega perché l'approssimazione del piano tangente migliora man mano
  che \((x,y)\) si avvicina a \((a,b)\).
\end{enumerate}

\subsection{8.5 Ottimizzazione in più
variabili}\label{ottimizzazione-in-piuxf9-variabili}

L'ottimizzazione nel calcolo multivariabile estende le idee di massimi e
minimi da funzioni a variabile singola a funzioni di due o più
variabili.

\subsubsection{Punti critici}\label{punti-critici}

Per \(f(x,y)\), si verifica un punto critico dove

\[
f_x(x,y) = 0 \quad \text{and} \quad f_y(x,y) = 0,
\]

o dove le derivate parziali non esistono.

\subsubsection{Test della derivata secondaPer classificare i punti
critici, calcolare le derivate parziali
seconde:}\label{test-della-derivata-secondaper-classificare-i-punti-critici-calcolare-le-derivate-parziali-seconde}

\[
D = f_{xx}(a,b) f_{yy}(a,b) - \big(f_{xy}(a,b)\big)^2.
\]

\begin{itemize}
\tightlist
\item
  Se \(D > 0\) e \(f_{xx}(a,b) > 0\): minimo locale.
\item
  Se \(D > 0\) e \(f_{xx}(a,b) < 0\): massimo locale.
\item
  Se \(D < 0\): punto di sella.
\item
  Se \(D = 0\): il test non è conclusivo.
\end{itemize}

\subsubsection{Esempio 1: paraboloide}\label{esempio-1-paraboloide-1}

\(f(x,y) = x^2 + y^2\).

\begin{itemize}
\tightlist
\item
  \(f_x = 2x, f_y = 2y\). Punto critico in (0,0).
\item
  \(f_{xx} = 2, f_{yy} = 2, f_{xy} = 0\).
\item
  \(D = (2)(2) - 0 = 4 > 0\) e \(f_{xx} > 0\).
\item
  Quindi (0,0) è un minimo locale.
\end{itemize}

\subsubsection{Esempio 2: Punto di
sella}\label{esempio-2-punto-di-sella}

\(f(x,y) = x^2 - y^2\).

\begin{itemize}
\tightlist
\item
  \(f_x = 2x, f_y = -2y\). Punto critico in (0,0).
\item
  \(f_{xx} = 2, f_{yy} = -2, f_{xy} = 0\).
\item
  \(D = (2)(-2) - 0 = -4 < 0\).
\item
  Quindi (0,0) è un punto di sella.
\end{itemize}

\subsubsection{Ottimizzazione vincolata e moltiplicatori di
Lagrange}\label{ottimizzazione-vincolata-e-moltiplicatori-di-lagrange}

A volte, vogliamo ottimizzare \(f(x,y)\) soggetto a un vincolo
\(g(x,y) = c\).

Metodo dei moltiplicatori di Lagrange: risolvere

\[
\nabla f(x,y) = \lambda \nabla g(x,y).
\]

Esempio: Massimizza \(f(x,y) = xy\) soggetto a \(x^2+y^2=1\).

\begin{itemize}
\tightlist
\item
  Gradienti:
  \(\nabla f = \langle y,x \rangle, \quad \nabla g = \langle 2x,2y \rangle\).
\item
  Equazioni: \(y = 2\lambda x, \, x = 2\lambda y\).
\item
  Le soluzioni portano al massimo a
  \((\pm \tfrac{1}{\sqrt{2}}, \pm \tfrac{1}{\sqrt{2}})\).
\end{itemize}

\subsubsection{Perché è importante}\label{perchuxe9-uxe8-importante-15}

\begin{itemize}
\tightlist
\item
  L'ottimizzazione è essenziale in economia, ingegneria, apprendimento
  automatico e fisica.
\item
  I moltiplicatori di Lagrange consentono l'ottimizzazione con vincoli,
  uno strumento chiave nella matematica applicata.
\end{itemize}

\subsubsection{Esercizi}\label{esercizi-31}

\begin{enumerate}
\def\labelenumi{\arabic{enumi}.}
\tightlist
\item
  Trova e classifica i punti critici di \(f(x,y) = x^2+xy+y^2\).
\item
  Classificare il punto (0,0) per \(f(x,y) = x^3-y^3\).
\item
  Utilizzare il test della derivata seconda per
  \(f(x,y) = x^4+y^4-4xy\).
\item
  Ingrandisci \(f(x,y) = x+y\) soggetto a \(x^2+y^2=1\).
\item
  Riduci al minimo \(f(x,y) = x^2+2y^2\) soggetto a \(x+y=1\).
\end{enumerate}

\section{Capitolo 9. Integrali
multipli}\label{capitolo-9.-integrali-multipli}

\subsection{9.1 Integrali doppiNel calcolo a variabile singola, un
integrale definito fornisce l'area sotto una curva. In due variabili, un
integrale doppio calcola il volume sotto una superficie (o, più in
generale, l'accumulo di valori su una
regione).}\label{integrali-doppinel-calcolo-a-variabile-singola-un-integrale-definito-fornisce-larea-sotto-una-curva.-in-due-variabili-un-integrale-doppio-calcola-il-volume-sotto-una-superficie-o-piuxf9-in-generale-laccumulo-di-valori-su-una-regione.}

\subsubsection{Definizione}\label{definizione-8}

Se \(f(x,y)\) è continuo su una regione \(R\), il doppio integrale è

\[
\iint_R f(x,y)\, dA = \lim_{m,n \to \infty} \sum_{i=1}^m \sum_{j=1}^n f(x_{ij}^-, y_{ij}^-) \Delta A,
\]

dove \(R\) è diviso in piccoli rettangoli di area \(\Delta A\).

\subsubsection{Integrali iterati}\label{integrali-iterati}

Per il Teorema di Fubini possiamo calcolare un integrale doppio come
integrale iterato:

\[
\iint_R f(x,y)\, dA = \int_a^b \int_c^d f(x,y)\, dy\, dx,
\]

se \(R\) è un rettangolo \([a,b] \times [c,d]\).

L'ordine di integrazione può spesso essere invertito:

\[
\int_a^b \int_c^d f(x,y)\,dy\,dx = \int_c^d \int_a^b f(x,y)\,dx\,dy.
\]

\subsubsection{Esempi}\label{esempi-19}

\begin{enumerate}
\def\labelenumi{\arabic{enumi}.}
\tightlist
\item
  Regione del rettangolo
\end{enumerate}

\[
\iint_R (x+y)\, dA, \quad R=[0,1]\times[0,2].
\]

\[
= \int_0^1 \int_0^2 (x+y)\,dy\,dx = \int_0^1 \Big[xy+\tfrac{1}{2}y^2\Big]_0^2 dx
= \int_0^1 (2x+2)dx = 3.
\]

\begin{enumerate}
\def\labelenumi{\arabic{enumi}.}
\setcounter{enumi}{1}
\tightlist
\item
  Regione triangolare
\end{enumerate}

\[
R = \{(x,y): 0 \leq x \leq 1, 0 \leq y \leq x\}.
\]

\[
\iint_R (x+y)\, dA = \int_0^1 \int_0^x (x+y)\,dy\,dx.
\]

La valutazione dà \(\tfrac{2}{3}\).

\subsubsection{Applicazioni}\label{applicazioni-1}

\begin{itemize}
\tightlist
\item
  Volume sotto una superficie:
\end{itemize}

\[
V = \iint_R f(x,y)\, dA.
\]

\begin{itemize}
\tightlist
\item
  Valore medio di una funzione su una regione:
\end{itemize}

\[
f_{\text{avg}} = \frac{1}{A(R)} \iint_R f(x,y)\, dA.
\]

\subsubsection{Perché è importante}\label{perchuxe9-uxe8-importante-16}

Gli integrali doppi estendono l'integrazione a due dimensioni. Sono
essenziali in fisica (massa, distribuzioni di probabilità), economia
(valori attesi) e ingegneria (centroidi, flusso).

\subsubsection{Esercizi}\label{esercizi-32}

\begin{enumerate}
\def\labelenumi{\arabic{enumi}.}
\tightlist
\item
  Valuta \(\iint_R (x^2+y^2)\, dA\) dove \(R=[0,1]\times[0,1]\).
\item
  Calcola \(\iint_R xy\, dA\) dove
  \(R=\{(x,y):0\leq x\leq2,0\leq y\leq x\}\).3. Trova il valore medio di
  \(f(x,y) = x+y\) sull'unità quadrata \([0,1]\times[0,1]\).
\item
  Interpretare \(\iint_R f(x,y)\, dA\) in termini di probabilità se
  \(f(x,y)\) è una funzione di densità di probabilità.
\item
  Mostra che il cambio dell'ordine di integrazione dà lo stesso
  risultato per \(\iint_{[0,1]\times[0,2]} (x+y)\,dA\).
\end{enumerate}

\subsection{9.2 Integrali tripli}\label{integrali-tripli}

Gli integrali tripli estendono l'idea di integrazione a tre variabili,
permettendoci di calcolare volumi, masse e altre quantità in regioni
tridimensionali.

\subsubsection{Definizione}\label{definizione-9}

Se \(f(x,y,z)\) è continuo su una regione solida \(E\), l'integrale
triplo è

\[
\iiint_E f(x,y,z)\, dV = \lim_{m,n,p \to \infty} \sum f(x_{ijk}^-, y_{ijk}^-, z_{ijk}^-) \Delta V,
\]

dove la regione è suddivisa in riquadri di volume \(\Delta V\).

\subsubsection{Integrali iterati}\label{integrali-iterati-1}

Per il Teorema di Fubini, un integrale triplo può essere calcolato come
integrale iterato:

\[
\iiint_E f(x,y,z)\, dV = \int_a^b \int_c^d \int_e^f f(x,y,z)\, dz\, dy\, dx,
\]

per una scatola rettangolare \(E = [a,b]\times[c,d]\times[e,f]\).

L'ordine di integrazione può essere scelto per comodità.

\subsubsection{Esempi}\label{esempi-20}

\begin{enumerate}
\def\labelenumi{\arabic{enumi}.}
\tightlist
\item
  Scatola rettangolare
\end{enumerate}

\[
\iiint_E xyz\, dV, \quad E=[0,1]\times[0,2]\times[0,3].
\]

\[
= \int_0^1 \int_0^2 \int_0^3 xyz\,dz\,dy\,dx.
\]

Prima integra su \(z\):

\[
\int_0^3 xyz\,dz = xy \left[\tfrac{1}{2}z^2\right]_0^3 = \tfrac{9}{2}xy.
\]

Ora integra su \(y\):

\[
\int_0^2 \tfrac{9}{2}xy\,dy = \tfrac{9}{2}x \cdot \left[\tfrac{1}{2}y^2\right]_0^2 = 9x.
\]

Infine integra su \(x\):

\[
\int_0^1 9x\,dx = \tfrac{9}{2}.
\]

\begin{enumerate}
\def\labelenumi{\arabic{enumi}.}
\setcounter{enumi}{1}
\tightlist
\item
  Regione delimitata da aerei Lascia che
  \(E = \{(x,y,z) \mid 0 \leq x \leq 1, 0 \leq y \leq x, 0 \leq z \leq y\}\).
\end{enumerate}

\[
\iiint_E 1\,dV = \int_0^1 \int_0^x \int_0^y 1\,dz\,dy\,dx.
\]

Valutare:

\[
= \int_0^1 \int_0^x y\,dy\,dx = \int_0^1 \tfrac{1}{2}x^2\,dx = \tfrac{1}{6}.
\]Quindi il volume di questa regione triangolare è \(\tfrac{1}{6}\).

\subsubsection{Applicazioni}\label{applicazioni-2}

\begin{itemize}
\item
  Volume: \(V = \iiint_E 1 \, dV\).
\item
  Massa: se la densità è \(\rho(x,y,z)\), allora

  \[
  M = \iiint_E \rho(x,y,z)\, dV.
  \]
\item
  Valore medio:

  \[
  f_{\text{avg}} = \frac{1}{V(E)} \iiint_E f(x,y,z)\,dV.
  \]
\end{itemize}

\subsubsection{Perché è importante}\label{perchuxe9-uxe8-importante-17}

Gli integrali tripli generalizzano i calcoli di area e volume a solidi
arbitrari. Sono utilizzati in fisica (distribuzioni di massa, centro di
massa, campi gravitazionali), ingegneria e probabilità.

\subsubsection{Esercizi}\label{esercizi-33}

\begin{enumerate}
\def\labelenumi{\arabic{enumi}.}
\tightlist
\item
  Calcola \(\iiint_E (x+y+z)\,dV\) sul cubo
  \(E=[0,1]\times[0,1]\times[0,1]\).
\item
  Trova il volume del tetraedro delimitato da
  \(x=0, y=0, z=0, x+y+z=1\).
\item
  Valuta \(\iiint_E x^2 \, dV\) dove \(E=[0,2]\times[0,1]\times[0,1]\).
\item
  Mostra che \(\iiint_E 1\,dV\) è uguale al volume geometrico di \(E\).
\item
  Se la densità è \(\rho(x,y,z)=x+y+z\), calcola la massa del cubo
  unitario.
\end{enumerate}

\subsection{9.3 Applicazioni: volume, massa,
probabilità}\label{applicazioni-volume-massa-probabilituxe0}

Gli integrali tripli sono potenti perché ci consentono di calcolare
quantità in tre dimensioni accumulando valori su una regione solida.

\subsubsection{Volume}\label{volume}

L'applicazione più semplice è trovare il volume di una regione \(E\):

\[
V = \iiint_E 1 \, dV.
\]

Esempio: Trova il volume del solido delimitato dai piani delle
coordinate e dal piano \(x+y+z=1\).

\[
V = \iiint_E 1 \, dV = \int_0^1 \int_0^{1-x} \int_0^{1-x-y} 1 \, dz\, dy\, dx.
\]

La valutazione dà \(V = \tfrac{1}{6}\).

\subsubsection{Massa e densità}\label{massa-e-densituxe0}

Se un solido ha la funzione di densità \(\rho(x,y,z)\), la sua massa è

\[
M = \iiint_E \rho(x,y,z)\, dV.
\]

Il centro di massa è dato da

\[
\bar{x} = \frac{1}{M}\iiint_E x\rho(x,y,z)\,dV, \quad
\bar{y} = \frac{1}{M}\iiint_E y\rho(x,y,z)\,dV, \quad
\bar{z} = \frac{1}{M}\iiint_E z\rho(x,y,z)\,dV.
\]

Esempio:Per un cubo unitario con densità costante \(\rho=1\), il centro
di massa si trova in \((0.5,0.5,0.5)\).

\subsubsection{Probabilità}\label{probabilituxe0}

Se \(f(x,y,z)\) è una funzione di densità di probabilità in 3D, allora
la probabilità che la variabile casuale si trovi in una regione \(E\) è

\[
P(E) = \iiint_E f(x,y,z)\, dV,
\]

dove \(f(x,y,z) \geq 0\) e

\[
\iiint_{\mathbb{R}^3} f(x,y,z)\,dV = 1.
\]

Esempio: Se \(f(x,y,z) = \tfrac{3}{4}z^2\) per \(0 \leq z \leq 1\),
uniformemente in \(x,y\), allora

\[
P(0 \leq z \leq 0.5) = \int_0^{0.5} \tfrac{3}{4}z^2 \, dz = \tfrac{1}{32}.
\]

\subsubsection{Perché è importante}\label{perchuxe9-uxe8-importante-18}

\begin{itemize}
\tightlist
\item
  I volumi generalizzano la geometria a solidi irregolari.
\item
  Gli integrali di massa e densità collegano il calcolo infinitesimale
  alla fisica e all'ingegneria.
\item
  Le funzioni di densità di probabilità in dimensioni superiori sono
  ampiamente utilizzate in statistica e scienza dei dati.
\end{itemize}

\subsubsection{Esercizi}\label{esercizi-34}

\begin{enumerate}
\def\labelenumi{\arabic{enumi}.}
\tightlist
\item
  Trova il volume del solido delimitato da \(x^2+y^2+z^2 \leq 1\) (la
  sfera unitaria).
\item
  Calcola la massa di un cono con densità proporzionale a \(z\).
\item
  Trova il centro di massa di un tetraedro uniforme delimitato da
  \(x=0, y=0, z=0, x+y+z=1\).
\item
  Se \(f(x,y,z) = \frac{1}{8}\) sul cubo
  \([0,2]\times[0,2]\times[0,2]\), verificare che si tratti di una
  funzione di densità di probabilità.
\item
  Utilizzare un integrale triplo per calcolare la probabilità che un
  punto scelto a caso nella sfera unitaria abbia \(z > 0\).
\end{enumerate}

\subsection{9.4 Cambio di variabili: coordinate polari, cilindriche,
sferiche}\label{cambio-di-variabili-coordinate-polari-cilindriche-sferiche}

Molti integrali diventano più semplici se espressi in sistemi di
coordinate che corrispondono alla simmetria della regione. Invece delle
coordinate cartesiane \((x,y,z)\), possiamo utilizzare coordinate
polari, cilindriche o sferiche.

\subsubsection{Coordinate polari (2D)}\label{coordinate-polari-2d}

Per le funzioni di due variabili, possiamo passare alle coordinate
polari:

\[
x = r\cos\theta, \quad y = r\sin\theta, \quad r \geq 0, \; 0 \leq \theta < 2\pi.
\]

L'elemento area si trasforma come

\[
dA = r\,dr\,d\theta.
\]

Esempio:Trova l'area del cerchio unitario.

\[
A = \iint_{x^2+y^2\leq 1} 1\,dA = \int_0^{2\pi}\int_0^1 r\,dr\,d\theta = \pi.
\]

\subsubsection{Coordinate cilindriche
(3D)}\label{coordinate-cilindriche-3d}

In 3D, le coordinate cilindriche estendono le coordinate polari con
\(z\):

\[
x = r\cos\theta, \quad y = r\sin\theta, \quad z = z.
\]

L'elemento volume è

\[
dV = r\,dr\,d\theta\,dz.
\]

Esempio: Volume di un cilindro di raggio \(R\) e altezza \(h\):

\[
V = \int_0^h \int_0^{2\pi} \int_0^R r\,dr\,d\theta\,dz = \pi R^2 h.
\]

\subsubsection{Coordinate sferiche (3D)}\label{coordinate-sferiche-3d}

Per la simmetria sferica, utilizzare:

\[
x = \rho \sin\phi \cos\theta, \quad y = \rho \sin\phi \sin\theta, \quad z = \rho \cos\phi,
\]

dove

\begin{itemize}
\tightlist
\item
  \(\rho \geq 0\) è la distanza dall'origine,
\item
  \(0 \leq \phi \leq \pi\) è l'angolo dall'asse positivo \(z\),
\item
  \(0 \leq \theta < 2\pi\) è l'angolo nel piano \(xy\).
\end{itemize}

L'elemento volume è

\[
dV = \rho^2 \sin\phi \, d\rho\, d\phi\, d\theta.
\]

Esempio: Volume della sfera unitaria:

\[
V = \int_0^{2\pi} \int_0^\pi \int_0^1 \rho^2 \sin\phi \, d\rho\, d\phi\, d\theta.
\]

Valutazione:

\[
V = \left(\int_0^1 \rho^2 d\rho\right)\left(\int_0^\pi \sin\phi d\phi\right)\left(\int_0^{2\pi} d\theta\right) = \tfrac{1}{3}(2)(2\pi) = \tfrac{4\pi}{3}.
\]

\subsubsection{Perché è importante}\label{perchuxe9-uxe8-importante-19}

\begin{itemize}
\tightlist
\item
  Le coordinate polari semplificano le regioni circolari.
\item
  Le coordinate cilindriche gestiscono i cilindri e la simmetria
  rotazionale.
\item
  Le coordinate sferiche semplificano sfere, coni e problemi radiali.
\item
  Questi cambiamenti di variabili rendono gestibili integrali altrimenti
  impossibili.
\end{itemize}

\subsubsection{Esercizi}\label{esercizi-35}

\begin{enumerate}
\def\labelenumi{\arabic{enumi}.}
\tightlist
\item
  Calcola \(\iint_{x^2+y^2\leq 4} (x^2+y^2)\,dA\) utilizzando le
  coordinate polari.
\item
  Trova il volume di un cono di altezza \(h\) e raggio \(R\) utilizzando
  le coordinate cilindriche.
\item
  Utilizza le coordinate sferiche per valutare il volume di una palla di
  raggio \(R\).
\item
  Mostra che il fattore Jacobiano per le coordinate polari è \(r\).5.
  Trova la massa di una sfera solida di raggio \(R\) con densità
  proporzionale alla distanza dall'origine utilizzando le coordinate
  sferiche.
\end{enumerate}

\section{Capitolo 10. Calcolo
vettoriale}\label{capitolo-10.-calcolo-vettoriale}

\subsection{10.1 Campi vettoriali}\label{campi-vettoriali}

Un campo vettoriale assegna un vettore a ciascun punto nello spazio,
proprio come una funzione scalare assegna un numero. I campi vettoriali
vengono utilizzati per modellare flussi, forze e altre quantità
direzionali.

\subsubsection{Definizione}\label{definizione-10}

In due dimensioni, un campo vettoriale è una funzione

\[
\mathbf{F}(x,y) = \langle P(x,y), Q(x,y) \rangle,
\]

dove \(P\) e \(Q\) sono funzioni scalari.

In tre dimensioni,

\[
\mathbf{F}(x,y,z) = \langle P(x,y,z), Q(x,y,z), R(x,y,z) \rangle.
\]

\subsubsection{Esempi}\label{esempi-21}

\begin{enumerate}
\def\labelenumi{\arabic{enumi}.}
\tightlist
\item
  Campo radiale
\end{enumerate}

\[
\mathbf{F}(x,y) = \langle x, y \rangle.
\]

I vettori puntano verso l'esterno rispetto all'origine.

\begin{enumerate}
\def\labelenumi{\arabic{enumi}.}
\setcounter{enumi}{1}
\tightlist
\item
  Campo rotazionale
\end{enumerate}

\[
\mathbf{F}(x,y) = \langle -y, x \rangle.
\]

I vettori circolano attorno all'origine.

\begin{enumerate}
\def\labelenumi{\arabic{enumi}.}
\setcounter{enumi}{2}
\tightlist
\item
  Campo gravitazionale
\end{enumerate}

\[
\mathbf{F}(x,y,z) = -\frac{GM}{r^3}\langle x,y,z \rangle, \quad r=\sqrt{x^2+y^2+z^2}.
\]

\subsubsection{Visualizzazione dei campi
vettoriali}\label{visualizzazione-dei-campi-vettoriali}

\begin{itemize}
\tightlist
\item
  Disegna piccole frecce nei punti campione per indicare la direzione e
  la magnitudo.
\item
  Frecce più dense dove le magnitudini sono maggiori.
\item
  Utile per interpretare linee di flusso, traiettorie e forze.
\end{itemize}

\subsubsection{Linee di flusso}\label{linee-di-flusso}

Una linea di flusso (o curva integrale) di un campo vettoriale è una
curva \(\mathbf{r}(t)\) il cui vettore tangente in ogni punto
corrisponde al campo:

\[
\mathbf{r}'(t) = \mathbf{F}(\mathbf{r}(t)).
\]

Le linee di flusso descrivono i percorsi delle particelle in un campo di
velocità.

\subsubsection{Perché è importante}\label{perchuxe9-uxe8-importante-20}

\begin{itemize}
\tightlist
\item
  I campi vettoriali sono fondamentali in fisica (flusso dei fluidi,
  elettromagnetismo, gravitazione).
\item
  Costituiscono la base degli integrali di linea, degli integrali di
  superficie e dei grandi teoremi del calcolo vettoriale (Green, Stokes,
  Divergenza).
\item
  Fornire un modo geometrico per rappresentare le quantità direzionali.
\end{itemize}

\subsubsection{\texorpdfstring{Esercizi1. Disegna il campo vettoriale
\(\mathbf{F}(x,y) = \langle y, -x \rangle\).}{Esercizi1. Disegna il campo vettoriale \textbackslash mathbf\{F\}(x,y) = \textbackslash langle y, -x \textbackslash rangle.}}\label{esercizi1.-disegna-il-campo-vettoriale-mathbffxy-langle-y--x-rangle.}

\begin{enumerate}
\def\labelenumi{\arabic{enumi}.}
\setcounter{enumi}{1}
\tightlist
\item
  Determina se i vettori di \(\mathbf{F}(x,y) = \langle x,y \rangle\)
  puntano verso o lontano dall'origine.
\item
  Per \(\mathbf{F}(x,y,z) = \langle y, z, x \rangle\), calcolare
  \(\mathbf{F}(1,2,3)\).
\item
  Descrivi le linee di flusso di
  \(\mathbf{F}(x,y) = \langle -y, x \rangle\).
\item
  Spiega perché i campi gravitazionali ed elettrici sono esempi di campi
  vettoriali radiali.
\end{enumerate}

\subsection{10.2 Integrali di linea}\label{integrali-di-linea}

Un integrale di linea estende l'idea di integrale alle funzioni valutate
lungo una curva. Invece di integrare su un intervallo o regione,
integriamo su un percorso nello spazio.

\subsubsection{Definizione: integrale
scalare}\label{definizione-integrale-scalare}

Se \(f(x,y)\) è una funzione scalare e \(C\) è una curva parametrizzata
da \(\mathbf{r}(t) = \langle x(t), y(t) \rangle, \; a \leq t \leq b\),
l'integrale di linea è

\[
\int_C f(x,y)\, ds = \int_a^b f(x(t),y(t)) \, |\mathbf{r}'(t)|\, dt,
\]

dove \(ds\) è la lunghezza dell'arco.

Questo misura l'accumulo di \(f\) lungo la curva.

\subsubsection{Definizione: integrale di linea
vettoriale}\label{definizione-integrale-di-linea-vettoriale}

Per un campo vettoriale
\(\mathbf{F}(x,y) = \langle P(x,y), Q(x,y) \rangle\), la linea integrale
lungo \(C\) è

\[
\int_C \mathbf{F} \cdot d\mathbf{r} = \int_a^b \mathbf{F}(\mathbf{r}(t)) \cdot \mathbf{r}'(t)\, dt.
\]

Questo misura il lavoro svolto dal campo lungo la curva.

\subsubsection{Esempi}\label{esempi-22}

\begin{enumerate}
\def\labelenumi{\arabic{enumi}.}
\tightlist
\item
  Integrale di linea scalare
\end{enumerate}

\[
f(x,y) = x+y, \quad C: \mathbf{r}(t) = \langle t, t^2 \rangle, \; 0 \leq t \leq 1.
\]

Poi

\[
\int_C f(x,y)\, ds = \int_0^1 (t+t^2)\sqrt{(1)^2+(2t)^2}\, dt.
\]

\begin{enumerate}
\def\labelenumi{\arabic{enumi}.}
\setcounter{enumi}{1}
\tightlist
\item
  Lavoro compiuto da una forza
\end{enumerate}

\[
\mathbf{F}(x,y) = \langle y, x \rangle, \quad C: \mathbf{r}(t) = \langle t, t^2 \rangle, \; 0 \leq t \leq 1.
\]

\[
\int_C \mathbf{F} \cdot d\mathbf{r} = \int_0^1 \langle t^2, t \rangle \cdot \langle 1, 2t \rangle\, dt = \int_0^1 (t^2 + 2t^2)\, dt = \int_0^1 3t^2\, dt = 1.\]

\subsubsection{Physical Interpretation}\label{physical-interpretation}

\begin{itemize}
\tightlist
\item
  Scalar line integral: accumulation of density along a wire.
\item
  Vector line integral: work done by a force moving an object along a
  path.
\end{itemize}

\subsubsection{Why This Matters}\label{why-this-matters}

\begin{itemize}
\tightlist
\item
  Line integrals connect vector fields with physical quantities like
  work and circulation.
\item
  They are building blocks for Green's Theorem and Stokes' Theorem.
\item
  Appear in physics (electric potential, fluid flow, mechanics).
\end{itemize}

\subsubsection{Exercises}\label{exercises-3}

\begin{enumerate}
\def\labelenumi{\arabic{enumi}.}
\tightlist
\item
  Compute \(\int_C (x^2+y^2)\, ds\) where \(C\) is the line segment from
  (0,0) to (1,1).
\item
  Evaluate \(\int_C \mathbf{F}\cdot d\mathbf{r}\) for
  \(\mathbf{F}(x,y) = \langle -y, x \rangle\) along the unit circle
  \(x^2+y^2=1\).
\item
  Interpret the meaning of \(\int_C 1\,ds\).
\item
  For \(\mathbf{F}(x,y,z) = \langle z,0,x \rangle\), compute the line
  integral along
  \(\mathbf{r}(t) = \langle t,t,1 \rangle, 0 \leq t \leq 1\).
\item
  Explain the difference between scalar and vector line integrals.
\end{enumerate}

\subsection{10.3 Surface Integrals}\label{surface-integrals}

A surface integral generalizes line integrals to two-dimensional
surfaces in three-dimensional space. They allow us to compute flux
through surfaces and accumulation of scalar fields over curved surfaces.

\subsubsection{Scalar Surface Integral}\label{scalar-surface-integral}

If a surface \(S\) is parameterized by

\[
\mathbf{r}(u,v) = \langle x(u,v), y(u,v), z(u,v) \rangle,
\]

then the surface integral of a scalar function \(f(x,y,z)\) is

\[
\iint_S f(x,y,z)\, dS = \iint_D f(\mathbf{r}(u,v)) \, |\mathbf{r}_u \times \mathbf{r}_v| \, du\,dv,
\]

where \(\mathbf{r}_u\) and \(\mathbf{r}_v\) are partial derivatives of
\(\mathbf{r}(u,v)\), and \(D\) is the parameter domain.

\subsubsection{Vector Surface Integral
(Flux)}\label{vector-surface-integral-flux}

For a vector field \(\mathbf{F}(x,y,z)\), the flux through a surface
\(S\) is

\[
\iint_S \mathbf{F}\cdot d\mathbf{S} = \iint_S \mathbf{F}\cdot \mathbf{n}\, dS,
\]dove \(\mathbf{n}\) è il vettore normale all'unità. Utilizzando la
parametrizzazione,

\[
\iint_S \mathbf{F}\cdot d\mathbf{S} = \iint_D \mathbf{F}(\mathbf{r}(u,v)) \cdot (\mathbf{r}_u \times \mathbf{r}_v)\,du\,dv.
\]

\subsubsection{Esempi}\label{esempi-23}

\begin{enumerate}
\def\labelenumi{\arabic{enumi}.}
\tightlist
\item
  Integrale di superficie scalare Superficie: piano \(z=1\) sopra il
  disco dell'unità \(x^2+y^2 \leq 1\).
\end{enumerate}

\[
\iint_S 1\, dS = \text{area of the disk} = \pi.
\]

\begin{enumerate}
\def\labelenumi{\arabic{enumi}.}
\setcounter{enumi}{1}
\tightlist
\item
  Flusso attraverso una sfera Sia
  \(\mathbf{F}(x,y,z) = \langle x,y,z \rangle\) e \(S\) = sfera di
  raggio \(R\). Il vettore normale è
  \(\mathbf{n} = \frac{1}{R}\langle x,y,z \rangle\).
\end{enumerate}

\[
\mathbf{F}\cdot \mathbf{n} = \frac{x^2+y^2+z^2}{R} = R.
\]

Quindi

\[
\iint_S \mathbf{F}\cdot d\mathbf{S} = \iint_S R\, dS = R \cdot 4\pi R^2 = 4\pi R^3.
\]

\subsubsection{Perché è importante}\label{perchuxe9-uxe8-importante-21}

\begin{itemize}
\tightlist
\item
  Gli integrali scalari di superficie misurano l'area e le distribuzioni
  superficiali.
\item
  Gli integrali di superficie vettoriali misurano il flusso: la quantità
  di un campo che passa attraverso una superficie.
\item
  Applicazioni: elettromagnetismo, flusso di fluidi, trasferimento di
  calore e altro.
\end{itemize}

\subsubsection{Esercizi}\label{esercizi-36}

\begin{enumerate}
\def\labelenumi{\arabic{enumi}.}
\tightlist
\item
  Calcola \(\iint_S 1\, dS\) per la superficie di un cubo di lato lungo
  2.
\item
  Trova il flusso di \(\mathbf{F}(x,y,z) = \langle x,y,z \rangle\)
  attraverso la sfera unitaria.
\item
  Valutare \(\iint_S z\, dS\) per il paraboloide
  \(z = x^2+y^2, \, z \leq 1\).
\item
  Per \(\mathbf{F}(x,y,z) = \langle y,0,0 \rangle\), calcolare il flusso
  attraverso il piano \(x=1\), \(0 \leq y,z \leq 1\).
\item
  Spiega fisicamente cosa significa se il flusso di un campo vettoriale
  attraverso una superficie chiusa è zero.
\end{enumerate}

\subsection{10.4 Teorema di Green}\label{teorema-di-green}

Il Teorema di Green è un risultato fondamentale nel calcolo vettoriale
che collega un integrale lineare attorno a una curva chiusa a un
integrale doppio sulla regione che racchiude. È una versione
bidimensionale del teorema di Stokes.

\subsubsection{\texorpdfstring{Enunciato del Teorema di GreenSia \(C\)
una curva semplice, chiusa e orientata positivamente nel piano e sia
\(R\) la regione che racchiude. Se
\(\mathbf{F}(x,y) = \langle P(x,y), Q(x,y) \rangle\) ha derivate
parziali continue su una regione aperta contenente \(R\),
allora}{Enunciato del Teorema di GreenSia C una curva semplice, chiusa e orientata positivamente nel piano e sia R la regione che racchiude. Se \textbackslash mathbf\{F\}(x,y) = \textbackslash langle P(x,y), Q(x,y) \textbackslash rangle ha derivate parziali continue su una regione aperta contenente R, allora}}\label{enunciato-del-teorema-di-greensia-c-una-curva-semplice-chiusa-e-orientata-positivamente-nel-piano-e-sia-r-la-regione-che-racchiude.-se-mathbffxy-langle-pxy-qxy-rangle-ha-derivate-parziali-continue-su-una-regione-aperta-contenente-r-allora}

\[
\oint_C \mathbf{F} \cdot d\mathbf{r} = \oint_C P\,dx + Q\,dy = \iint_R \left( \frac{\partial Q}{\partial x} - \frac{\partial P}{\partial y} \right)\, dA.
\]

\subsubsection{Interpretazione}\label{interpretazione-2}

\begin{itemize}
\tightlist
\item
  L'integrale di linea attorno a \(C\) misura la circolazione del campo
  vettoriale lungo il confine.
\item
  Il doppio integrale su \(R\) misura l'arricciatura totale (rotazione)
  del campo all'interno della regione.
\end{itemize}

\subsubsection{Esempio 1: formula
dell'area}\label{esempio-1-formula-dellarea}

Se \(\mathbf{F} = \langle -y/2, x/2 \rangle\), allora

\[
\frac{\partial Q}{\partial x} - \frac{\partial P}{\partial y} = 1.
\]

Quindi il Teorema di Green dà

\[
\text{Area}(R) = \iint_R 1\,dA = \oint_C \left(-\tfrac{y}{2}\,dx + \tfrac{x}{2}\,dy\right).
\]

Ciò fornisce un modo per calcolare l'area utilizzando un integrale di
linea.

\subsubsection{Esempio 2: Circolazione}\label{esempio-2-circolazione}

Sia \(\mathbf{F}(x,y) = \langle -y, x \rangle\) e \(C\) il cerchio
unitario.

\begin{itemize}
\tightlist
\item
  \(P=-y, Q=x\).
\item
  \(Q_x - P_y = 1 - (-1) = 2\).
\item
  Integrale doppio sul disco unitario:
\end{itemize}

\[
\iint_R 2\,dA = 2\pi (1^2) = 2\pi.
\]

Quindi la circolazione attorno al cerchio è \(2\pi\).

\subsubsection{Perché è importante}\label{perchuxe9-uxe8-importante-22}

\begin{itemize}
\tightlist
\item
  Converte integrali di linea difficili in integrali doppi o viceversa.
\item
  Fornisce un ponte tra proprietà locali (curl) e proprietà globali
  (circolazione).
\item
  Ampiamente usato in fisica per il flusso dei fluidi,
  l'elettromagnetismo e i campi vettoriali planari.
\end{itemize}

\subsubsection{Esercizi}\label{esercizi-37}

\begin{enumerate}
\def\labelenumi{\arabic{enumi}.}
\tightlist
\item
  Utilizza il Teorema di Green per calcolare l'area all'interno
  dell'ellisse \(\frac{x^2}{a^2} + \frac{y^2}{b^2} = 1\).
\item
  Verifica il Teorema di Green per
  \(\mathbf{F}(x,y) = \langle -y, x \rangle\) lungo il quadrato con
  vertici (0,0), (1,0), (1,1), (0,1).3. Calcola la circolazione di
  \(\mathbf{F}(x,y) = \langle -y, x \rangle\) attorno al cerchio
  unitario.
\item
  Mostra che se \(\nabla \times \mathbf{F} = 0\), allora l'integrale di
  linea di \(\mathbf{F}\) attorno a qualsiasi curva chiusa è zero.
\item
  Usa il Teorema di Green per dimostrarlo
\end{enumerate}

\[
\oint_C x\,dy = -\oint_C y\,dx
\]

per qualsiasi curva chiusa \(C\).

\subsection{10.5 Teorema di Stokes}\label{teorema-di-stokes}

Il Teorema di Stokes generalizza il Teorema di Green a tre dimensioni.
Mette in relazione un integrale di superficie del ricciolo di un campo
vettoriale su una superficie con una linea integrale del campo
vettoriale attorno al confine di quella superficie.

\subsubsection{Enunciato del Teorema di
Stokes}\label{enunciato-del-teorema-di-stokes}

Sia \(S\) una superficie orientata e liscia con curva di confine \(C\)
(orientata positivamente). Se \(\mathbf{F}(x,y,z)\) è un campo
vettoriale con derivate parziali continue, allora

\[
\iint_S (\nabla \times \mathbf{F}) \cdot d\mathbf{S} = \oint_C \mathbf{F} \cdot d\mathbf{r}.
\]

\begin{itemize}
\tightlist
\item
  Lato sinistro: flusso del ricciolo di \(\mathbf{F}\) attraverso la
  superficie.
\item
  Lato destro: circolazione di \(\mathbf{F}\) lungo la curva di confine.
\end{itemize}

\subsubsection{Interpretazione}\label{interpretazione-3}

\begin{itemize}
\tightlist
\item
  La linea integrale attorno al confine equivale alla ``rotazione''
  totale all'interno della superficie.
\item
  Estende il Teorema di Green (un caso speciale quando la superficie
  giace nel piano).
\end{itemize}

\subsubsection{Esempio 1: il teorema di Green come caso
speciale}\label{esempio-1-il-teorema-di-green-come-caso-speciale}

Se \(S\) è una regione piatta nel piano \(xy\), il Teorema di Stokes si
riduce al Teorema di Green.

\subsubsection{Esempio 2: Circolazione su un
emisfero}\label{esempio-2-circolazione-su-un-emisfero}

Sia \(\mathbf{F}(x,y,z) = \langle -y, x, 0 \rangle\) e \(S\) l'emisfero
superiore di raggio 1.

\begin{itemize}
\tightlist
\item
  Confine \(C\): circonferenza unitaria nel piano \(xy\).
\item
  Per il Teorema di Stokes:
\end{itemize}

\[
\oint_C \mathbf{F}\cdot d\mathbf{r} = \iint_S (\nabla \times \mathbf{F})\cdot d\mathbf{S}.
\]

\begin{itemize}
\tightlist
\item
  Ricciolo: \(\nabla \times \mathbf{F} = \langle 0,0,2 \rangle\).
\item
  Normale all'emisfero punta verso l'esterno:
  \(\mathbf{n} = \langle 0,0,1 \rangle\).
\item
  Quindi integrando = 2.- Area dell'emisfero = \(2\pi (1^2)\).
\end{itemize}

\[
\iint_S 2\, dS = 2 \cdot 2\pi = 4\pi.
\]

Pertanto, la circolazione attorno all'equatore è \(4\pi\).

\subsubsection{Perché è importante}\label{perchuxe9-uxe8-importante-23}

\begin{itemize}
\tightlist
\item
  Fornisce una connessione profonda tra integrali di superficie e
  integrali di linea.
\item
  Semplifica i calcoli consentendo la scelta di superfici convenienti.
\item
  Ampiamente usato nell'elettromagnetismo (legge di Faraday) e nella
  fluidodinamica.
\end{itemize}

\subsubsection{Esercizi}\label{esercizi-38}

\begin{enumerate}
\def\labelenumi{\arabic{enumi}.}
\tightlist
\item
  Verifica il teorema di Stokes per
  \(\mathbf{F}(x,y,z) = \langle y, -x, 0 \rangle\) sul disco unitario
  nel piano \(xy\).
\item
  Calcola \(\oint_C \mathbf{F}\cdot d\mathbf{r}\) dove
  \(\mathbf{F}(x,y,z) = \langle z, 0, x \rangle\) e \(C\) è il confine
  del triangolo con vertici (0,0,0), (1,0,0), (0,1,0).
\item
  Mostra che se \(\nabla \times \mathbf{F} = 0\), allora la circolazione
  attorno a qualsiasi curva chiusa è zero.
\item
  Applica il Teorema di Stokes per calcolare la circolazione di
  \(\mathbf{F}(x,y,z) = \langle -y, x, z \rangle\) attorno al confine
  del quadrato unitario nel piano \(z=0\).
\item
  Spiega come il Teorema di Stokes generalizza il Teorema di Green.
\end{enumerate}

\subsection{10.6 Teorema della
divergenza}\label{teorema-della-divergenza}

Il Teorema della Divergenza (chiamato anche Teorema di Gauss) mette in
relazione il flusso di un campo vettoriale attraverso una superficie
chiusa con l'integrale triplo della divergenza del campo all'interno
della superficie.

\subsubsection{Enunciato del Teorema della
Divergenza}\label{enunciato-del-teorema-della-divergenza}

Sia \(E\) una regione solida in \(\mathbb{R}^3\) con la superficie di
confine \(S\) (orientata verso l'esterno). Se \(\mathbf{F}(x,y,z)\) è un
campo vettoriale con derivate parziali continue su \(E\), allora

\[
\iint_S \mathbf{F} \cdot d\mathbf{S} = \iiint_E (\nabla \cdot \mathbf{F}) \, dV.
\]

\begin{itemize}
\tightlist
\item
  Lato sinistro: flusso di \(\mathbf{F}\) attraverso la superficie
  chiusa \(S\).
\item
  Lato destro: integrale triplo della divergenza all'interno della
  regione.
\end{itemize}

\subsubsection{Divergenza}\label{divergenza}

La divergenza di un campo vettoriale
\(\mathbf{F}(x,y,z) = \langle P, Q, R \rangle\) è

\[\nabla \cdot \mathbf{F} = \frac{\partial P}{\partial x} + \frac{\partial Q}{\partial y} + \frac{\partial R}{\partial z}.
\]

It measures the ``net outflow'' per unit volume at each point.

\subsubsection{Example 1: Flux of a Radial
Field}\label{example-1-flux-of-a-radial-field}

Let \(\mathbf{F}(x,y,z) = \langle x, y, z \rangle\), and let \(E\) be
the unit ball \(x^2+y^2+z^2 \leq 1\).

\begin{itemize}
\tightlist
\item
  Divergence: \(\nabla \cdot \mathbf{F} = 1+1+1 = 3\).
\item
  Volume of unit ball: \(\tfrac{4}{3}\pi\). So
\end{itemize}

\[
\iiint_E (\nabla \cdot \mathbf{F})\, dV = 3 \cdot \tfrac{4}{3}\pi = 4\pi.
\]

Pertanto, il flusso attraverso la sfera unitaria è \(4\pi\).

\subsubsection{Esempio 2: Campo
costante}\label{esempio-2-campo-costante}

Lascia che \(\mathbf{F}(x,y,z) = \langle 1, 0, 0 \rangle\).

\begin{itemize}
\tightlist
\item
  Divergenza: \(\nabla \cdot \mathbf{F} = 0\).
\item
  Quindi il flusso attraverso qualsiasi superficie chiusa è zero,
  coerentemente con l'intuizione (nessun deflusso netto).
\end{itemize}

\subsubsection{Perché è importante}\label{perchuxe9-uxe8-importante-24}

\begin{itemize}
\item
  Converte gli integrali di superficie in integrali di volume più
  semplici.
\item
  Utilizzato in fisica: legge di Gauss in elettromagnetismo, flusso di
  fluidi e trasferimento di calore.
\item
  Completa il quadro unificante:

  \begin{itemize}
  \tightlist
  \item
    Teorema di Green (arricciatura 2D ↔ circolazione)
  \item
    Teorema di Stokes (arricciatura 3D ↔ circolazione sulle superfici)
  \item
    Teorema della divergenza (divergenza 3D ↔ flusso su superfici
    chiuse)
  \end{itemize}
\end{itemize}

\subsubsection{Esercizi}\label{esercizi-39}

\begin{enumerate}
\def\labelenumi{\arabic{enumi}.}
\tightlist
\item
  Utilizza il Teorema della Divergenza per calcolare il flusso di
  \(\mathbf{F}(x,y,z) = \langle x,y,z \rangle\) attraverso la superficie
  di una sfera di raggio \(R\).
\item
  Verifica il teorema della divergenza per
  \(\mathbf{F}(x,y,z) = \langle y, z, x \rangle\) sul cubo unitario
  \([0,1]^3\).
\item
  Mostra che se \(\nabla \cdot \mathbf{F} = 0\), allora il flusso totale
  attraverso qualsiasi superficie chiusa è zero.
\item
  Calcola il flusso di
  \(\mathbf{F}(x,y,z) = \langle x^2, y^2, z^2 \rangle\) attraverso la
  sfera unitaria.
\item
  Spiegare come il Teorema della Divergenza generalizza il Teorema
  Fondamentale unidimensionale del Calcolo infinitesimale.
\end{enumerate}

\section{Parte IV. Processi infiniti}\label{parte-iv.-processi-infiniti}

\section{Capitolo 11. Successioni e convergenza\#\# 11.1 Definizioni ed
esempi}\label{capitolo-11.-successioni-e-convergenza-11.1-definizioni-ed-esempi}

Una sequenza è un elenco ordinato di numeri, solitamente scritto come

\[
a_1, a_2, a_3, \dots
\]

o più in generale \((a_n)_{n=1}^\infty\). Ciascun \(a_n\) è chiamato
l'ennesimo termine della sequenza.

\subsubsection{Definizione di una
sequenza}\label{definizione-di-una-sequenza}

Una sequenza può essere definita in due modi:

\begin{enumerate}
\def\labelenumi{\arabic{enumi}.}
\item
  Formula esplicita: fornisce una regola diretta per l'ennesimo termine.

  \begin{itemize}
  \item
    Esempio: \(a_n = \frac{1}{n}\) definisce la sequenza

    \[
    1, \tfrac{1}{2}, \tfrac{1}{3}, \tfrac{1}{4}, \dots
    \]
  \end{itemize}
\item
  Definizione ricorsiva: definisce i termini utilizzando termini
  precedenti.

  \begin{itemize}
  \item
    Esempio: sequenza di Fibonacci:

    \[
    a_1 = 1, \quad a_2 = 1, \quad a_{n} = a_{n-1} + a_{n-2} \quad (n \geq 3).
    \]
  \end{itemize}
\end{enumerate}

\subsubsection{Esempi di sequenze}\label{esempi-di-sequenze}

\begin{enumerate}
\def\labelenumi{\arabic{enumi}.}
\item
  Sequenza aritmetica:

  \[
  a_n = a_1 + (n-1)d.
  \]

  Esempio: \(a_n = 2n+1\) → sequenza di numeri dispari.
\item
  Sequenza geometrica:

  \[
  a_n = a_1 r^{n-1}.
  \]

  Esempio: \(a_n = 2^n\) → potenze di 2.
\item
  Sequenza armonica:

  \[
  a_n = \frac{1}{n}.
  \]
\item
  Sequenza alternata:

  \[
  a_n = (-1)^n.
  \]
\end{enumerate}

\subsubsection{Successioni nel calcolo
infinitesimale}\label{successioni-nel-calcolo-infinitesimale}

Le sequenze sono la base per processi infiniti:

\begin{itemize}
\tightlist
\item
  Limiti di successioni → definire la convergenza.
\item
  Serie → somme infinite costruite da successioni.
\item
  Funzioni approssimate da sequenze e serie.
\end{itemize}

\subsubsection{Perché è importante}\label{perchuxe9-uxe8-importante-25}

\begin{itemize}
\tightlist
\item
  Le successioni forniscono gli elementi costitutivi per serie infinite
  e approssimazioni.
\item
  Permettono di definire rigorosamente l'\,``avvicinamento
  all'infinito'' e la convergenza.
\item
  Molte funzioni importanti (esponenziale, trigonometrica) possono
  essere espresse attraverso sequenze e serie.
\end{itemize}

\subsubsection{Esercizi}\label{esercizi-40}

\begin{enumerate}
\def\labelenumi{\arabic{enumi}.}
\tightlist
\item
  Scrivi i primi cinque termini della sequenza \(a_n = \frac{n}{n+1}\).
\item
  Determina se \(a_n = (-1)^n n\) è limitato.
\item
  Fornire una definizione ricorsiva per la sequenza \(2,4,8,16,\dots\).
\item
  Trova il decimo termine della sequenza aritmetica con \(a_1=3\) e
  \(d=5\).5. Scrivere una formula esplicita per la sequenza definita da
  \(a_1=1\), \(a_{n+1}=2a_n\).
\end{enumerate}

\subsection{11.2 Sequenze monotone e
limitate}\label{sequenze-monotone-e-limitate}

Per capire se una sequenza converge, dobbiamo studiarne il
comportamento: aumenta, diminuisce, resta entro dei limiti o cresce
senza limiti? Due concetti importanti sono monotonia e limitatezza.

\subsubsection{Sequenze monotone}\label{sequenze-monotone}

Una sequenza \((a_n)\) si dice monotona se è sempre crescente o sempre
decrescente.

\begin{itemize}
\item
  Aumento monotono:

  \[
  a_{n+1} \geq a_n \quad \text{for all } n.
  \]
\item
  Decrescente monotono:

  \[
  a_{n+1} \leq a_n \quad \text{for all } n.
  \]
\end{itemize}

Esempi:

\begin{enumerate}
\def\labelenumi{\arabic{enumi}.}
\tightlist
\item
  \(a_n = n\) è monotono crescente.
\item
  \(a_n = \frac{1}{n}\) è monotono decrescente.
\end{enumerate}

\subsubsection{Successioni limitate}\label{successioni-limitate}

Una sequenza è limitata sopra se esiste un numero \(M\) tale che
\(a_n \leq M\) per tutti \(n\). È delimitato di seguito se esiste \(m\)
tale che \(a_n \geq m\) per tutti \(n\).

Se valgono entrambe le condizioni la successione è limitata.

Esempi:

\begin{enumerate}
\def\labelenumi{\arabic{enumi}.}
\tightlist
\item
  \(a_n = \frac{1}{n}\) è compreso tra 0 e 1.
\item
  \(a_n = (-1)^n\) è compreso tra -1 e 1.
\item
  \(a_n = n\) non è limitato.
\end{enumerate}

\subsubsection{Teorema della convergenza
monotona}\label{teorema-della-convergenza-monotona}

Un risultato fondamentale in analisi:

\begin{itemize}
\tightlist
\item
  Ogni successione monotona crescente limitata superiormente converge.
\item
  Ogni successione monotona decrescente limitata inferiormente converge.
\end{itemize}

Questo teorema garantisce la convergenza senza trovare esplicitamente il
limite.

\subsubsection{Esempio}\label{esempio-1}

\begin{enumerate}
\def\labelenumi{\arabic{enumi}.}
\item
  Sequenza: \(a_n = 1 - \frac{1}{n}\).

  \begin{itemize}
  \tightlist
  \item
    In aumento: dal \(a_{n+1} - a_n = \frac{1}{n} - \frac{1}{n+1} > 0\).
  \item
    Limitato sopra da 1.
  \item
    Quindi converge.
  \item
    Limite: \(\lim_{n\to\infty} a_n = 1\).
  \end{itemize}
\end{enumerate}

\subsubsection{Perché è importante}\label{perchuxe9-uxe8-importante-26}

\begin{itemize}
\tightlist
\item
  Monotonia e limitatezza forniscono test rapidi per la convergenza.
\item
  Sono essenziali nelle dimostrazioni e nella costruzione rigorosa dei
  limiti.
\item
  Queste idee si estendono naturalmente alle funzioni e alle
  serie.\#\#\# Esercizi
\end{itemize}

\begin{enumerate}
\def\labelenumi{\arabic{enumi}.}
\tightlist
\item
  Determina se \(a_n = \frac{n}{n+1}\) è monotono e limitato.
\item
  Mostra che \(a_n = \sqrt{n}\) è monotono crescente ma non limitato.
\item
  Dimostra che \(a_n = 2 - \frac{1}{n}\) converge e trova il suo limite.
\item
  Fornisci un esempio di una successione limitata che non sia monotona.
\item
  Applica il teorema della convergenza monotona a
  \(a_n = \ln\!\big(1+\frac{1}{n}\big)\).
\end{enumerate}

\subsection{11.3 Limiti delle sequenze}\label{limiti-delle-sequenze}

La domanda centrale su una sequenza è se i suoi termini si avvicinano a
un singolo valore man mano che \(n\) cresce. Ciò porta al concetto di
limite di una successione.

\subsubsection{Definizione}\label{definizione-11}

Una sequenza \((a_n)\) ha un limite \(L\) se per ogni
\(\varepsilon > 0\) esiste un intero \(N\) tale che

\[
|a_n - L| < \varepsilon \quad \text{whenever } n > N.
\]

Scriviamo allora

\[
\lim_{n\to\infty} a_n = L.
\]

Se non esiste alcun \(L\), la sequenza diverge.

\subsubsection{Intuizione}\label{intuizione}

\begin{itemize}
\tightlist
\item
  I termini della sequenza si avvicinano arbitrariamente a \(L\) man
  mano che \(n\) diventa grande.
\item
  Al di là dell'indice \(N\), tutti i termini rimangono all'interno di
  una piccola fascia attorno a \(L\).
\end{itemize}

\subsubsection{Esempi}\label{esempi-24}

\begin{enumerate}
\def\labelenumi{\arabic{enumi}.}
\item
  \(a_n = \frac{1}{n}\). Man mano che \(n\) cresce, i termini si
  riducono verso 0.

  \[
  \lim_{n\to\infty} \frac{1}{n} = 0.
  \]
\item
  \(a_n = (-1)^n\). I termini si alternano tra -1 e 1, quindi non esiste
  un limite unico. La sequenza diverge.
\item
  \(a_n = \frac{n}{n+1}\). Poiché \(n \to \infty\), numeratore e
  denominatore sono quasi uguali, quindi

  \[
  \lim_{n\to\infty} \frac{n}{n+1} = 1.
  \]
\end{enumerate}

\subsubsection{Proprietà dei limiti}\label{proprietuxe0-dei-limiti}

Se \(\lim a_n = A\) e \(\lim b_n = B\):

\begin{itemize}
\item
  \(\lim (a_n+b_n) = A+B\).
\item
  \(\lim (a_n b_n) = AB\).
\item
  \(\lim (c a_n) = cA\) per la costante \(c\).
\item
  Se \(b_n \neq 0\) e \(B \neq 0\), allora

  \[
  \lim \frac{a_n}{b_n} = \frac{A}{B}.
  \]
\end{itemize}

\subsubsection{Teorema: Principio di
compressione}\label{teorema-principio-di-compressione}

Se \(a_n \leq b_n \leq c_n\) per tutti i grandi \(n\) e

\[
\lim_{n\to\infty} a_n = \lim_{n\to\infty} c_n = L,
\]

allora

\[\lim_{n\to\infty} b_n = L.
\]

Example:

\[
a_n = -\tfrac{1}{n}, \quad b_n = \tfrac{\sin n}{n}, \quad c_n = \tfrac{1}{n}.
\]

Since \(-\tfrac{1}{n} \leq \tfrac{\sin n}{n} \leq \tfrac{1}{n}\) and
both bounding sequences go to 0,

\[
\lim_{n\to\infty} \frac{\sin n}{n} = 0.
\]

\subsubsection{Why This Matters}\label{why-this-matters-1}

\begin{itemize}
\tightlist
\item
  Limits make rigorous the idea of sequences ``approaching'' a value.
\item
  Convergence of sequences underpins infinite series and continuity.
\item
  These concepts are essential in defining real numbers via limits.
\end{itemize}

\subsubsection{Exercises}\label{exercises-4}

\begin{enumerate}
\def\labelenumi{\arabic{enumi}.}
\tightlist
\item
  Find \(\lim_{n\to\infty} \frac{2n+1}{3n+4}\).
\item
  Determine if \(a_n = \sqrt{n+1} - \sqrt{n}\) converges.
\item
  Does \(a_n = \cos n\) converge? Why or why not?
\item
  Use the Squeeze Principle to show
  \(\lim_{n\to\infty} \frac{\sin n}{n} = 0\).
\item
  Prove that if \(\lim a_n = L\), then \(\lim |a_n| = |L|\).
\end{enumerate}

\section{Chapter 12. Infinite series}\label{chapter-12.-infinite-series}

\subsection{12.1 Series and Convergence}\label{series-and-convergence}

A series is the sum of the terms of a sequence. Instead of just listing
numbers, we add them together and study whether the infinite sum
approaches a finite value.

\subsubsection{Definition}\label{definition}

Given a sequence \((a_n)\), the corresponding series is

\[
\sum_{n=1}^\infty a_n = a_1 + a_2 + a_3 + \dots
\]

We define the nth partial sum as

\[
S_n = \sum_{k=1}^n a_k.
\]

If the sequence \((S_n)\) converges to a finite limit \(S\), then the
series converges and

\[
\sum_{n=1}^\infty a_n = S.
\]

If \((S_n)\) diverges, then the series diverges.

\subsubsection{Examples}\label{examples-2}

\begin{enumerate}
\def\labelenumi{\arabic{enumi}.}
\tightlist
\item
  Geometric series
\end{enumerate}

\[
\sum_{n=0}^\infty ar^n = \frac{a}{1-r}, \quad |r| <1.
\]

Example:

\[
1 + \tfrac{1}{2} + \tfrac{1}{4} + \tfrac{1}{8} + \dots = 2.
\]

\begin{enumerate}
\def\labelenumi{\arabic{enumi}.}
\setcounter{enumi}{1}
\tightlist
\item
  Harmonic series
\end{enumerate}

\[
\sum_{n=1}^\infty \frac{1}{n}.
\]

This series diverges, even though the terms go to 0.

\begin{enumerate}
\def\labelenumi{\arabic{enumi}.}
\setcounter{enumi}{2}
\tightlist
\item
  p-series
\end{enumerate}

\[
\sum_{n=1}^\infty \frac{1}{n^p}.
\]

\begin{itemize}
\tightlist
\item
  Converge se \(p > 1\).
\item
  Diverge se \(p \leq 1\).\#\#\# Condizione necessaria per la
  convergenza
\end{itemize}

Se \(\sum a_n\) converge, allora necessariamente

\[
\lim_{n\to\infty} a_n = 0.
\]

Se \(\lim a_n \neq 0\), la serie diverge. Ma non è vero il contrario:
\(\lim a_n = 0\) non garantisce la convergenza (ad esempio, la serie
armonica).

\subsubsection{Perché è importante}\label{perchuxe9-uxe8-importante-27}

\begin{itemize}
\tightlist
\item
  Le serie estendono l'addizione finita a processi infiniti.
\item
  Le serie convergenti vengono utilizzate per approssimare funzioni,
  calcolare aree e modellare processi fisici.
\item
  Lo studio delle serie porta a potenti test di convergenza.
\end{itemize}

\subsubsection{Esercizi}\label{esercizi-41}

\begin{enumerate}
\def\labelenumi{\arabic{enumi}.}
\tightlist
\item
  Determina se \(\sum_{n=1}^\infty \frac{2}{3^n}\) converge e trova la
  sua somma.
\item
  Mostra che \(\sum_{n=1}^\infty \frac{1}{n^2}\) converge.
\item
  \(\sum_{n=1}^\infty \frac{1}{\sqrt{n}}\) converge?
\item
  Scrivi le prime quattro somme parziali della serie
  \(\sum_{n=1}^\infty \frac{1}{2^n}\).
\item
  Spiega perché \(\lim a_n = 0\) è necessario ma non sufficiente per la
  convergenza.
\end{enumerate}

\subsection{12.2 Test di convergenza}\label{test-di-convergenza}

Poiché molte serie non possono essere sommate direttamente, i matematici
hanno sviluppato dei test per decidere se una serie converge o diverge.
Questi test sono strumenti per analizzare somme infinite.

\subsubsection{1. Il test all'ennesimo termine per la
divergenza}\label{il-test-allennesimo-termine-per-la-divergenza}

Se

\[
\lim_{n\to\infty} a_n \neq 0 \quad \text{or does not exist},
\]

allora

\[
\sum a_n
\]

diverge.

Se \(\lim a_n = 0\), il test non è conclusivo.

\subsubsection{2. Test comparativo}\label{test-comparativo}

Supponiamo che \(0 \leq a_n \leq b_n\) per tutti i \(n\).

\begin{itemize}
\tightlist
\item
  Se \(\sum b_n\) converge, allora converge anche \(\sum a_n\).
\item
  Se \(\sum a_n\) diverge, allora anche \(\sum b_n\) diverge.
\end{itemize}

\subsubsection{3. Test di confronto dei
limiti}\label{test-di-confronto-dei-limiti}

Se \(a_n, b_n > 0\) e

\[
\lim_{n\to\infty} \frac{a_n}{b_n} = c,
\]

dove \(0 < c < \infty\), quindi \(\sum a_n\) e \(\sum b_n\) convergono
entrambi o divergono entrambi.

\subsubsection{4. Test del rapporto}\label{test-del-rapporto}

Per \(\sum a_n\), calcola

\[
L = \lim_{n\to\infty} \left| \frac{a_{n+1}}{a_n} \right|.
\]- Se \(L < 1\), la serie converge assolutamente. - Se \(L > 1\) o
\(L = \infty\), la serie diverge. - Se \(L = 1\), il test non è
conclusivo.

\subsubsection{5. Test della radice}\label{test-della-radice}

Per \(\sum a_n\), calcola

\[
L = \lim_{n\to\infty} \sqrt[n]{|a_n|}.
\]

\begin{itemize}
\tightlist
\item
  Se \(L < 1\), la serie converge assolutamente.
\item
  Se \(L > 1\), la serie diverge.
\item
  Se \(L = 1\), il test non è conclusivo.
\end{itemize}

\subsubsection{6. Test delle serie alternate (test di
Leibniz)}\label{test-delle-serie-alternate-test-di-leibniz}

Per le serie del modulo

\[
\sum (-1)^n b_n \quad \text{or} \quad \sum (-1)^{n+1} b_n,
\]

se

\begin{enumerate}
\def\labelenumi{\arabic{enumi}.}
\tightlist
\item
  \(b_{n+1} \leq b_n\) (diminuente) e
\item
  \(\lim_{n\to\infty} b_n = 0\),
\end{enumerate}

allora la serie converge.

\subsubsection{Esempi}\label{esempi-25}

\begin{enumerate}
\def\labelenumi{\arabic{enumi}.}
\tightlist
\item
  \(\sum \frac{1}{n^2}\): Test comparativo → converge.
\item
  \(\sum \frac{1}{n}\): Serie armonica → diverge.
\item
  \(\sum \frac{(-1)^n}{n}\): test in serie alternata → converge.
\item
  \(\sum \frac{n!}{n^n}\): Test del rapporto → converge.
\item
  \(\sum \frac{2^n}{n}\): Test radice → diverge.
\end{enumerate}

\subsubsection{Perché è importante}\label{perchuxe9-uxe8-importante-28}

\begin{itemize}
\tightlist
\item
  I test di convergenza permettono di classificare le serie senza
  bisogno di somme esplicite.
\item
  Forniscono modi sistematici per gestire infiniti processi nel calcolo.
\item
  Sono fondamentali per argomenti successivi come le serie di potenze e
  le serie di Fourier.
\end{itemize}

\subsubsection{Esercizi}\label{esercizi-42}

\begin{enumerate}
\def\labelenumi{\arabic{enumi}.}
\tightlist
\item
  Testare la convergenza di \(\sum \frac{1}{n^3}\).
\item
  Utilizzare il test del rapporto per \(\sum \frac{3^n}{n!}\).
\item
  Applica il test root a \(\sum \left(\frac{1}{2}\right)^n\).
\item
  Determinare la convergenza di \(\sum \frac{(-1)^n}{\sqrt{n}}\).
\item
  Utilizzare il test di confronto dei limiti con \(\frac{1}{n^2}\) per
  testare \(\sum \frac{1}{n^2+1}\).
\end{enumerate}

\subsection{12.3 Convergenza assoluta vs
condizionale}\label{convergenza-assoluta-vs-condizionale}

Non tutte le serie si comportano allo stesso modo quando i segni si
alternano. Per gestire questo, distinguiamo tra convergenza assoluta e
convergenza condizionale.

\subsubsection{Convergenza assoluta}\label{convergenza-assoluta}

Una serie \(\sum a_n\) è assolutamente convergente se

\[
\sum |a_n|
\]

converge.Teorema: Se una serie converge assolutamente, allora converge
anche.

Esempio:

\[
\sum \frac{(-1)^n}{n^2}.
\]

Qui \(\sum \left|\frac{(-1)^n}{n^2}\right| = \sum \frac{1}{n^2}\)
converge (serie p, \(p=2\)). Quindi la serie è assolutamente
convergente.

\subsubsection{Convergenza condizionale}\label{convergenza-condizionale}

Una serie \(\sum a_n\) è condizionatamente convergente se converge, ma
non assolutamente.

Esempio:

\[
\sum \frac{(-1)^n}{n}.
\]

\begin{itemize}
\tightlist
\item
  Test delle serie alternate → converge.
\item
  Ma \(\sum \left|\frac{(-1)^n}{n}\right| = \sum \frac{1}{n}\) diverge
  (serie armonica). Quindi la serie è condizionatamente convergente.
\end{itemize}

\subsubsection{Teorema del
riarrangiamento}\label{teorema-del-riarrangiamento}

Per le serie condizionatamente convergenti, la riorganizzazione dei
termini può modificare la somma, persino farla divergere o convergere a
un valore diverso.

Questo risultato sorprendente mostra la natura delicata della
convergenza condizionale.

\subsubsection{Perché è importante}\label{perchuxe9-uxe8-importante-29}

\begin{itemize}
\tightlist
\item
  La convergenza assoluta è più forte e garantisce stabilità.
\item
  La convergenza condizionale evidenzia l'importanza dell'ordine nelle
  somme infinite.
\item
  Molte serie alterne incontrate nella pratica sono convergenti solo
  condizionatamente.
\end{itemize}

\subsubsection{Esercizi}\label{esercizi-43}

\begin{enumerate}
\def\labelenumi{\arabic{enumi}.}
\tightlist
\item
  Mostra che \(\sum \frac{(-1)^n}{n^3}\) converge assolutamente.
\item
  Mostra che \(\sum \frac{(-1)^n}{n}\) è condizionatamente convergente.
\item
  Testare \(\sum \frac{(-1)^n}{\sqrt{n}}\) per la convergenza assoluta e
  condizionale.
\item
  Spiega perché la convergenza assoluta implica convergenza, ma non è
  vero il contrario.
\item
  Ricerca e riassumi il teorema del riarrangiamento di Riemann con
  parole tue.
\end{enumerate}

\section{Capitolo 13. Serie di potenze ed
espansioni}\label{capitolo-13.-serie-di-potenze-ed-espansioni}

\subsection{13.1 Serie di potenze}\label{serie-di-potenze}

Una serie di potenze è una serie infinita in cui ogni termine implica
una potenza della variabile. Le serie di potenze sono centrali nel
calcolo perché ci permettono di rappresentare le funzioni come polinomi
infiniti.

\subsubsection{Modulo generale}\label{modulo-generale}

Una serie di potenze centrata su \(a\) ha la forma

\[\sum_{n=0}^\infty c_n (x-a)^n,
\]

where \(c_n\) are constants called the coefficients.

\begin{itemize}
\item
  If \(a=0\), the series is centered at the origin:

  \[
  \sum_{n=0}^\infty c_n x^n.
  \]
\end{itemize}

\subsubsection{Examples}\label{examples-3}

\begin{enumerate}
\def\labelenumi{\arabic{enumi}.}
\tightlist
\item
  Geometric series
\end{enumerate}

\[
\sum_{n=0}^\infty x^n = \frac{1}{1-x}, \quad |x|<1.
\]

\begin{enumerate}
\def\labelenumi{\arabic{enumi}.}
\setcounter{enumi}{1}
\tightlist
\item
  Exponential function
\end{enumerate}

\[
e^x = \sum_{n=0}^\infty \frac{x^n}{n!}.
\]

\begin{enumerate}
\def\labelenumi{\arabic{enumi}.}
\setcounter{enumi}{2}
\tightlist
\item
  Sine and cosine
\end{enumerate}

\[
\sin x = \sum_{n=0}^\infty (-1)^n \frac{x^{2n+1}}{(2n+1)!}, \quad  
\cos x = \sum_{n=0}^\infty (-1)^n \frac{x^{2n}}{(2n)!}.
\]

\subsubsection{Interval of Convergence}\label{interval-of-convergence}

For each power series, there exists a radius of convergence \(R\) such
that:

\begin{itemize}
\tightlist
\item
  The series converges if \(|x-a| < R\).
\item
  The series diverges if \(|x-a| > R\).
\item
  At \(|x-a| = R\), convergence must be checked separately.
\end{itemize}

\subsubsection{Why This Matters}\label{why-this-matters-2}

\begin{itemize}
\tightlist
\item
  Power series allow us to approximate functions by polynomials.
\item
  They connect calculus with analysis and differential equations.
\item
  Many special functions in mathematics and physics are defined by their
  power series.
\end{itemize}

\subsubsection{Exercises}\label{exercises-5}

\begin{enumerate}
\def\labelenumi{\arabic{enumi}.}
\tightlist
\item
  Write the power series for \(\sum_{n=0}^\infty \frac{(x-2)^n}{n!}\).
\item
  Find the first four terms of the power series for \(e^x\).
\item
  Express \(\frac{1}{1+x}\) as a power series centered at 0.
\item
  Determine whether the series \(\sum_{n=0}^\infty n! x^n\) converges at
  \(x=0.1\).
\item
  Explain why power series are sometimes called ``infinite
  polynomials.''
\end{enumerate}

\subsection{13.2 Radius of Convergence}\label{radius-of-convergence}

Every power series converges for some values of \(x\) and diverges for
others. The boundary between these two behaviors is described by the
radius of convergence.

\subsubsection{Definition}\label{definition-1}

For a power series

\[
\sum_{n=0}^\infty c_n (x-a)^n,
\]

esiste un numero \(R \geq 0\) (possibilmente infinito) tale che:

\begin{itemize}
\tightlist
\item
  La serie converge assolutamente se \(|x-a| < R\).
\item
  La serie diverge se \(|x-a| > R\).- A \(|x-a| = R\), la convergenza
  deve essere verificata separatamente.
\end{itemize}

Questo numero \(R\) è chiamato raggio di convergenza.

\subsubsection{Trovare il raggio di
convergenza}\label{trovare-il-raggio-di-convergenza}

Due metodi comuni:

\begin{enumerate}
\def\labelenumi{\arabic{enumi}.}
\tightlist
\item
  Prova del rapporto
\end{enumerate}

\[
R = \lim_{n\to\infty} \left| \frac{c_n}{c_{n+1}} \right|.
\]

\begin{enumerate}
\def\labelenumi{\arabic{enumi}.}
\setcounter{enumi}{1}
\tightlist
\item
  Prova della radice
\end{enumerate}

\[
R = \frac{1}{\limsup_{n\to\infty} \sqrt[n]{|c_n|}}.
\]

\subsubsection{Esempi}\label{esempi-26}

\begin{enumerate}
\def\labelenumi{\arabic{enumi}.}
\tightlist
\item
  Serie:
\end{enumerate}

\[
\sum_{n=0}^\infty \frac{x^n}{n!}.
\]

Utilizzando il test del rapporto:

\[
\lim_{n\to\infty} \frac{1/(n!)}{1/((n+1)!)} = \infty.
\]

Quindi \(R = \infty\) (converge per tutti i reali \(x\)).

\begin{enumerate}
\def\labelenumi{\arabic{enumi}.}
\setcounter{enumi}{1}
\tightlist
\item
  Serie:
\end{enumerate}

\[
\sum_{n=0}^\infty x^n.
\]

Qui \(c_n = 1\).

\[
R = 1.
\]

Converge per \(|x| < 1\).

\begin{enumerate}
\def\labelenumi{\arabic{enumi}.}
\setcounter{enumi}{2}
\tightlist
\item
  Serie:
\end{enumerate}

\[
\sum_{n=1}^\infty \frac{x^n}{n}.
\]

Prova del rapporto:

\[
\lim_{n\to\infty} \left|\frac{(x^{n+1}/(n+1))}{(x^n/n)}\right| = |x|.
\]

Quindi \(R = 1\). Converge per \(|x| < 1\), diverge per \(|x| > 1\). A
\(x=\pm 1\), prova separatamente.

\subsubsection{Intervallo di
convergenza}\label{intervallo-di-convergenza}

L'insieme dei valori \(x\) in cui la serie converge è chiamato
intervallo di convergenza.

\begin{itemize}
\tightlist
\item
  Sempre centrato su \(a\).
\item
  Estende le unità \(R\) in entrambe le direzioni.
\item
  Gli endpoint \(x=a\pm R\) devono essere controllati individualmente.
\end{itemize}

\subsubsection{Perché è importante}\label{perchuxe9-uxe8-importante-30}

\begin{itemize}
\tightlist
\item
  Il raggio di convergenza ci dice dove le serie di potenze si
  comportano come funzioni.
\item
  Essenziale per utilizzare nella pratica le espansioni della serie
  Taylor.
\item
  Determina il dominio di validità delle soluzioni in serie in fisica e
  ingegneria.
\end{itemize}

\subsubsection{Esercizi}\label{esercizi-44}

\begin{enumerate}
\def\labelenumi{\arabic{enumi}.}
\tightlist
\item
  Trova il raggio di convergenza di
  \(\sum_{n=0}^\infty \frac{(x-3)^n}{n!}\).
\item
  Calcola il raggio di convergenza di
  \(\sum_{n=1}^\infty \frac{x^n}{n^2}\).
\item
  Utilizzare il test del rapporto per trovare \(R\) per
  \(\sum_{n=0}^\infty n!x^n\).
\item
  Determinare l'intervallo di convergenza per
  \(\sum_{n=1}^\infty \frac{(x+1)^n}{n}\).
\item
  Spiega perché la serie esponenziale converge per tutti i \(x\), mentre
  la serie geometrica converge solo per \(|x|<1\).\#\# 13.3 Serie di
  Taylor e Maclaurin
\end{enumerate}

Le serie di potenze diventano particolarmente potenti quando vengono
utilizzate per rappresentare funzioni familiari. Questo viene fatto
attraverso la serie di Taylor, e il caso speciale centrato su 0 è
chiamato serie di Maclaurin.

\subsubsection{Serie Taylor}\label{serie-taylor}

Se una funzione \(f(x)\) è infinitamente differenziabile in \(x=a\), la
sua serie di Taylor su \(a\) è

\[
f(x) = \sum_{n=0}^\infty \frac{f^{(n)}(a)}{n!}(x-a)^n.
\]

Qui \(f^{(n)}(a)\) denota il \(n\)-esimo derivato di \(f\) in \(a\).

\subsubsection{Serie Maclaurin}\label{serie-maclaurin}

Una serie Taylor incentrata su \(a=0\):

\[
f(x) = \sum_{n=0}^\infty \frac{f^{(n)}(0)}{n!} x^n.
\]

\subsubsection{Esempi}\label{esempi-27}

\begin{enumerate}
\def\labelenumi{\arabic{enumi}.}
\tightlist
\item
  Funzione esponenziale
\end{enumerate}

\[
e^x = 1 + x + \frac{x^2}{2!} + \frac{x^3}{3!} + \cdots
\]

\begin{enumerate}
\def\labelenumi{\arabic{enumi}.}
\setcounter{enumi}{1}
\tightlist
\item
  Seno e coseno
\end{enumerate}

\[
\sin x = x - \frac{x^3}{3!} + \frac{x^5}{5!} - \cdots
\]

\[
\cos x = 1 - \frac{x^2}{2!} + \frac{x^4}{4!} - \cdots
\]

\begin{enumerate}
\def\labelenumi{\arabic{enumi}.}
\setcounter{enumi}{2}
\tightlist
\item
  Logaritmo naturale (per \(|x|<1\))
\end{enumerate}

\[
\ln(1+x) = x - \frac{x^2}{2} + \frac{x^3}{3} - \frac{x^4}{4} + \cdots
\]

\subsubsection{Approssimazione polinomiale di
Taylor}\label{approssimazione-polinomiale-di-taylor}

La somma finita dei primi termini \(n\) è il polinomio di Taylor di
grado \(n\):

\[
P_n(x) = \sum_{k=0}^n \frac{f^{(k)}(a)}{k!}(x-a)^k.
\]

Questo polinomio si avvicina a \(f(x)\) vicino a \(x=a\).

\subsubsection{Resto (termine di errore)}\label{resto-termine-di-errore}

La differenza tra la funzione e il suo polinomio di Taylor è il resto:

\[
R_n(x) = f(x) - P_n(x).
\]

Una forma (la forma di Lagrange) è

\[
R_n(x) = \frac{f^{(n+1)}(c)}{(n+1)!}(x-a)^{n+1},
\]

per alcuni \(c\) compresi tra \(a\) e \(x\).

\subsubsection{Perché è importante}\label{perchuxe9-uxe8-importante-31}

\begin{itemize}
\tightlist
\item
  Le serie di Taylor forniscono approssimazioni polinomiali a funzioni
  complicate.
\item
  Sono essenziali nell'analisi numerica, nella fisica e nell'ingegneria.
\item
  Gli espansioni in serie di Maclaurin forniscono formule semplici per
  funzioni esponenziali, trigonometriche e logaritmiche.
\end{itemize}

\subsubsection{Esercizi}\label{esercizi-45}

\begin{enumerate}
\def\labelenumi{\arabic{enumi}.}
\tightlist
\item
  Trova la serie Maclaurin per
  \(f(x)=\cosh x = \tfrac{e^x+e^{-x}}{2}\).2. Scrivi la serie di Taylor
  per \(f(x)=e^x\) centrata su \(a=2\).
\item
  Calcola il polinomio di Taylor di grado 3 per \(f(x)=\ln(1+x)\) in
  \(a=0\).
\item
  Utilizzare la serie Maclaurin per \(\sin x\) per approssimare
  \(\sin(0.1)\).
\item
  Spiegare perché le serie di Taylor spesso forniscono buone
  approssimazioni locali ma possono divergere per \(|x|\) grandi.
\end{enumerate}

\subsection{13.4 Applicazioni della serie di
Taylor}\label{applicazioni-della-serie-di-taylor}

Le serie di Taylor non sono solo strumenti teorici: vengono utilizzate
per approssimare funzioni, risolvere equazioni e analizzare sistemi
fisici. Le loro applicazioni spaziano dalla matematica, alla scienza e
all'ingegneria.

\subsubsection{Approssimazione di
funzioni}\label{approssimazione-di-funzioni}

Le funzioni complicate possono essere approssimate mediante polinomi
vicino a un punto.

Esempio: approssimare \(e^x\) vicino a \(x=0\) utilizzando il polinomio
di Maclaurin di grado 3:

\[
P_3(x) = 1 + x + \tfrac{x^2}{2} + \tfrac{x^3}{6}.
\]

Per \(x\) piccolo, fornisce stime accurate di \(e^x\).

\subsubsection{Metodi numerici}\label{metodi-numerici}

Le serie di Taylor forniscono la base per gli algoritmi numerici:

\begin{itemize}
\tightlist
\item
  Approssimazione di radici quadrate, logaritmi e valori trigonometrici.
\item
  Stima dell'errore tramite il termine residuo.
\item
  Utilizzato in metodi iterativi come il metodo di Newton (dove la
  linearizzazione locale deriva dalla serie di Taylor).
\end{itemize}

\subsubsection{Risoluzione di equazioni
differenziali}\label{risoluzione-di-equazioni-differenziali}

Molte equazioni differenziali hanno soluzioni espresse come serie di
Taylor (o di potenze).

Esempio: La soluzione di \(y'' + y = 0\) con \(y(0)=0, y'(0)=1\) è
\(\sin x\), che deriva naturalmente dalla sua serie Maclaurin.

\subsubsection{Fisica e Ingegneria}\label{fisica-e-ingegneria}

\begin{itemize}
\item
  Approssimazione a piccolo angolo:

  \[
  \sin x \approx x, \quad \cos x \approx 1 - \tfrac{x^2}{2}, \quad |x| \ll 1.
  \]

  Utilizzato nel movimento del pendolo, nell'ottica e nella meccanica
  delle onde.
\item
  Relatività e meccanica quantistica: gli sviluppi di Taylor
  semplificano le espressioni non lineari per l'uso pratico.-
  Approssimazione delle funzioni energetiche: in meccanica, le funzioni
  energetiche potenziali vengono espanse vicino ai punti di equilibrio.
\end{itemize}

\subsubsection{Probabilità e
Statistica}\label{probabilituxe0-e-statistica}

\begin{itemize}
\tightlist
\item
  Le funzioni generatrici dei momenti e le funzioni caratteristiche
  utilizzano le serie di potenze.
\item
  Le approssimazioni delle distribuzioni di probabilità (ad esempio,
  l'approssimazione normale alla binomiale) utilizzano le espansioni di
  Taylor.
\end{itemize}

\subsubsection{Perché è importante}\label{perchuxe9-uxe8-importante-32}

\begin{itemize}
\tightlist
\item
  Le serie di Taylor forniscono un ponte tra le formule esatte e il
  calcolo pratico.
\item
  Permettono di ridurre problemi complessi ad approssimazioni
  polinomiali gestibili.
\item
  Le applicazioni li rendono uno degli strumenti più importanti nella
  matematica applicata.
\end{itemize}

\subsubsection{Esercizi}\label{esercizi-46}

\begin{enumerate}
\def\labelenumi{\arabic{enumi}.}
\tightlist
\item
  Utilizzare la serie Maclaurin per \(e^x\) per approssimare \(e^{0.1}\)
  fino a quattro cifre decimali.
\item
  Applicare l'approssimazione del piccolo angolo per stimare
  \(\sin(5^\circ)\).
\item
  Risolvi l'equazione differenziale \(y'' = -y\) utilizzando un
  approccio in serie di potenze.
\item
  Espandi \(\ln(1+x)\) fino al 4° grado e usalo per approssimare
  \(\ln(1.1)\).
\item
  Spiegare perché le approssimazioni polinomiali sono particolarmente
  utili per computer e calcolatrici.
\end{enumerate}

\section{Appendici}\label{appendici}

\subsection{Appendice A. Elementi essenziali del
precalcolo}\label{appendice-a.-elementi-essenziali-del-precalcolo}

\subsubsection{A.1 Ripasso di algebra}\label{a.1-ripasso-di-algebra}

Prima di immergersi nel calcolo, è utile rivedere alcune abilità di
algebra che appariranno ancora e ancora. Questi sono gli ``strumenti''
di cui avrai bisogno per manipolare espressioni, risolvere equazioni e
semplificare i risultati.

\paragraph{Esponenti e potenze}\label{esponenti-e-potenze}

\begin{itemize}
\item
  Regole di base:

  \[
  a^m \cdot a^n = a^{m+n}, \quad \frac{a^m}{a^n} = a^{m-n}, \quad (a^m)^n = a^{mn}.
  \]
\item
  Esponenti negativi:

  \[
  a^{-n} = \frac{1}{a^n}, \quad a \neq 0.
  \]
\item
  Esponenti frazionari:

  \[
  a^{1/n} = \sqrt[n]{a}, \quad a^{m/n} = \sqrt[n]{a^m}.
  \]
\end{itemize}

\paragraph{Fattorizzazione}\label{fattorizzazione}

La fattorizzazione semplifica le espressioni e aiuta a risolvere le
equazioni.

\begin{enumerate}
\def\labelenumi{\arabic{enumi}.}
\item
  Fattore comune:

  \[
  6x^2+9x = 3x(2x+3).
  \]
\item
  Differenza di quadrati:

  \[a^2-b^2 = (a-b)(a+b).
  \]
\item
  Quadratic trinomials:

  \[
  x^2+5x+6 = (x+2)(x+3).
  \]
\end{enumerate}

\paragraph{Polynomials}\label{polynomials}

\begin{itemize}
\tightlist
\item
  Standard form: \(P(x) = a_nx^n + a_{n-1}x^{n-1} + \cdots + a_0\).
\item
  Degree: the largest power of \(x\).
\item
  Long division and synthetic division are useful for simplifying
  rational functions.
\end{itemize}

\paragraph{Rational Expressions}\label{rational-expressions}

Simplify by factoring numerator and denominator:

\[
\frac{x^2-1}{x^2-2x+1} = \frac{(x-1)(x+1)}{(x-1)^2} = \frac{x+1}{x-1}, \quad x \neq 1.
\]

\paragraph{Logarithms}\label{logarithms}

\begin{itemize}
\item
  Definition: \(\log_a b = c\) means \(a^c = b\).
\item
  Common bases: natural log (\(\ln x = \log_e x\)) and base 10
  (\(\log x\)).
\item
  Rules:

  \[
  \log(ab) = \log a + \log b, \quad \log\left(\frac{a}{b}\right) = \log a - \log b, \quad \log(a^n) = n\log a.
  \]
\end{itemize}

\paragraph{Equations}\label{equations}

\begin{itemize}
\item
  Linear: solve \(ax+b=0\) → \(x=-b/a\).
\item
  Quadratic: \(ax^2+bx+c=0\) has solutions

  \[
  x=\frac{-b\pm \sqrt{b^2-4ac}}{2a}.
  \]
\item
  Exponential: \(e^x = k\) → \(x = \ln k\).
\end{itemize}

\subsubsection{A.2 Trigonometry Basics}\label{a.2-trigonometry-basics}

Trigonometry provides the language of angles and periodic phenomena.
Since calculus often deals with oscillations, motion, and waves, a solid
grasp of trigonometric functions and their properties is essential.

\paragraph{The Unit Circle}\label{the-unit-circle}

\begin{itemize}
\item
  Defined as the circle of radius 1 centered at the origin in the
  coordinate plane.
\item
  For an angle \(\theta\) measured from the positive \(x\)-axis:

  \[
  (\cos \theta, \sin \theta)
  \]

  dà le coordinate del punto sulla circonferenza.
\end{itemize}

Valori speciali:

\begin{longtable}[]{@{}
  >{\raggedright\arraybackslash}p{(\linewidth - 6\tabcolsep) * \real{0.3333}}
  >{\raggedright\arraybackslash}p{(\linewidth - 6\tabcolsep) * \real{0.1667}}
  >{\raggedright\arraybackslash}p{(\linewidth - 6\tabcolsep) * \real{0.1667}}
  >{\raggedright\arraybackslash}p{(\linewidth - 6\tabcolsep) * \real{0.3333}}@{}}
\toprule\noalign{}
\begin{minipage}[b]{\linewidth}\raggedright
\(\theta\)
\end{minipage} & \begin{minipage}[b]{\linewidth}\raggedright
\(\sin \theta\)
\end{minipage} & \begin{minipage}[b]{\linewidth}\raggedright
\(\cos \theta\)
\end{minipage} & \begin{minipage}[b]{\linewidth}\raggedright
\(\tan \theta = \frac{\sin \theta}{\cos \theta}\)
\end{minipage} \\
\midrule\noalign{}
\endhead
\bottomrule\noalign{}
\endlastfoot
\(0\) & 0 & 1 & 0 \\
\(\pi/6\) & 1/2 & \(\sqrt{3}/2\) & \(1/\sqrt{3}\) \\
\(\pi/3\) & \(\sqrt{3}/2\) & 1/2 & \(\sqrt{3}\) \\
\(\pi/2\) & 1 & 0 & indefinito \\
\end{longtable}

\paragraph{Identità Fondamentali}\label{identituxe0-fondamentali}

\begin{enumerate}
\def\labelenumi{\arabic{enumi}.}
\tightlist
\item
  Identità pitagorica
\end{enumerate}

\[
\sin^2\theta + \cos^2\theta = 1.
\]

\begin{enumerate}
\def\labelenumi{\arabic{enumi}.}
\setcounter{enumi}{1}
\tightlist
\item
  Identità quoziente
\end{enumerate}

\[
\tan\theta = \frac{\sin\theta}{\cos\theta}, \quad \cot\theta = \frac{\cos\theta}{\sin\theta}.
\]

\begin{enumerate}
\def\labelenumi{\arabic{enumi}.}
\setcounter{enumi}{2}
\tightlist
\item
  Identità reciproche
\end{enumerate}

\[
\sec\theta = \frac{1}{\cos\theta}, \quad \csc\theta = \frac{1}{\sin\theta}.
\]

\paragraph{Formule di addizione degli
angoli}\label{formule-di-addizione-degli-angoli}

\[
\sin(\alpha+\beta) = \sin\alpha\cos\beta + \cos\alpha\sin\beta,
\]

\[
\cos(\alpha+\beta) = \cos\alpha\cos\beta - \sin\alpha\sin\beta.
\]

Casi particolari:

\begin{itemize}
\item
  Doppio angolo:

  \[
  \sin(2\theta) = 2\sin\theta\cos\theta, \quad
  \cos(2\theta) = \cos^2\theta - \sin^2\theta.
  \]
\end{itemize}

\paragraph{Grafici}\label{grafici}

\begin{itemize}
\tightlist
\item
  \(\sin x\): onda che inizia da 0, ampiezza 1, periodo \(2\pi\).
\item
  \(\cos x\): onda che inizia da 1, ampiezza 1, periodo \(2\pi\).
\item
  \(\tan x\): ripete ogni \(\pi\), indefinito a multipli dispari di
  \(\pi/2\).
\end{itemize}

\subsubsection{A.3 Geometria delle
coordinate}\label{a.3-geometria-delle-coordinate}

La geometria delle coordinate collega l'algebra e la geometria
descrivendo oggetti geometrici (linee, cerchi, curve) utilizzando
equazioni. Il calcolo fa molto affidamento su questa struttura per
rappresentare graficamente le funzioni, trovare pendenze e analizzare le
curve.

\paragraph{Il piano cartesiano}\label{il-piano-cartesiano}

\begin{itemize}
\item
  Un punto è rappresentato dalle coordinate \((x,y)\).
\item
  Distanza tra due punti \((x_1,y_1)\) e \((x_2,y_2)\):

  \[
  d = \sqrt{(x_2-x_1)^2 + (y_2-y_1)^2}.
  \]
\item
  Punto medio di un segmento di linea:

  \[
  M = \left(\frac{x_1+x_2}{2}, \frac{y_1+y_2}{2}\right).
  \]
\end{itemize}

\paragraph{Righe}\label{righe}

\begin{enumerate}
\def\labelenumi{\arabic{enumi}.}
\item
  Formula della pendenza

  \[
  m = \frac{y_2-y_1}{x_2-x_1}.
  \]
\item
  Equazione di una linea

  \begin{itemize}
  \item
    Forma punto-pendenza:

    \[y-y_1 = m(x-x_1).
    \]
  \item
    Slope-intercept form:

    \[
    y = mx+b.
    \]
  \end{itemize}
\item
  Parallel and perpendicular lines

  \begin{itemize}
  \tightlist
  \item
    Parallel lines: same slope.
  \item
    Perpendicular lines: slopes satisfy \(m_1m_2 = -1\).
  \end{itemize}
\end{enumerate}

\paragraph{Circles}\label{circles}

Equation of a circle with center \((h,k)\) and radius \(r\):

\[
(xh)^2+(yk)^2 = r^2.
\]

Special case: unit circle centered at origin:

\[
x^2+y^2=1.
\]

\paragraph{Conic Sections}\label{conic-sections}

\begin{enumerate}
\def\labelenumi{\arabic{enumi}.}
\item
  Parabola:

  \begin{itemize}
  \item
    Standard form (opening up/down):

    \[
    y = ascia^2+bx+c.
    \]
  \end{itemize}
\item
  Ellipse (centered at origin):

  \[
  \frac{x^2}{a^2}+\frac{y^2}{b^2}=1.
  \]
\item
  Hyperbola (centered at origin):

  \[
  \frac{x^2}{a^2}-\frac{y^2}{b^2}=1.
  \]
\end{enumerate}

\subsection{Appendix B. Key Formulas and
Tables}\label{appendix-b.-key-formulas-and-tables}

\subsubsection{B.1 Derivative Table}\label{b.1-derivative-table}

Derivatives measure rates of change and slopes of functions. Having a
quick-reference table helps learners avoid re-deriving formulas each
time.

\paragraph{Basic Rules}\label{basic-rules}

\begin{enumerate}
\def\labelenumi{\arabic{enumi}.}
\tightlist
\item
  Constant rule
\end{enumerate}

\[
\frac{d}{dx}[c] = 0
\]

\begin{enumerate}
\def\labelenumi{\arabic{enumi}.}
\setcounter{enumi}{1}
\tightlist
\item
  Power rule
\end{enumerate}

\[
\frac{d}{dx}[x^n] = nx^{n-1}, \quad (n \in \mathbb{R})
\]

\begin{enumerate}
\def\labelenumi{\arabic{enumi}.}
\setcounter{enumi}{2}
\tightlist
\item
  Constant multiple rule
\end{enumerate}

\[
\frac{d}{dx}[c f(x)] = c f'(x)
\]

\begin{enumerate}
\def\labelenumi{\arabic{enumi}.}
\setcounter{enumi}{3}
\tightlist
\item
  Sum and difference rule
\end{enumerate}

\[
\frac{d}{dx}[f(x)\pm g(x)] = f'(x)\pm g'(x)
\]

\paragraph{Trigonometric Functions}\label{trigonometric-functions}

\[
\frac{d}{dx}[\sin x] = \cos x
\]

\[
\frac{d}{dx}[\cos x] = -\sin x
\]

\[
\frac{d}{dx}[\tan x] = \sec^2 x, \quad x \neq \tfrac{\pi}{2}+k\pi
\]

\[
\frac{d}{dx}[\cot x] = -\csc^2 x
\]

\[
\frac{d}{dx}[\sec x] = \sec x \tan x
\]

\[
\frac{d}{dx}[\csc x] = -\csc x \cot x
\]

\paragraph{Exponential and Logarithmic
Functions}\label{exponential-and-logarithmic-functions}

\[
\frac{d}{dx}[e^x] = e^x
\]

\[
\frac{d}{dx}[a^x] = a^x \ln a, \quad a>0, a\neq 1
\]

\[
\frac{d}{dx}[\ln x] = \frac{1}{x}, \quad x>0
\]

\[
\frac{d}{dx}[\log_a x] = \frac{1}{x\ln a}, \quad a>0, a\neq 1
\]

\paragraph{Inverse Trigonometric
Functions}\label{inverse-trigonometric-functions}

\[\frac{d}{dx}[\arcsin x] = \frac{1}{\sqrt{1-x^2}}, \quad |x|<1
\]

\[
\frac{d}{dx}[\arccos x] = -\frac{1}{\sqrt{1-x^2}}, \quad |x|<1
\]

\[
\frac{d}{dx}[\arctan x] = \frac{1}{1+x^2}, \quad x \in \mathbb{R}
\]

\paragraph{Product, Quotient, and Chain
Rules}\label{product-quotient-and-chain-rules}

\begin{enumerate}
\def\labelenumi{\arabic{enumi}.}
\tightlist
\item
  Product Rule
\end{enumerate}

\[
\frac{d}{dx}[f(x)g(x)] = f'(x)g(x)+f(x)g'(x)
\]

\begin{enumerate}
\def\labelenumi{\arabic{enumi}.}
\setcounter{enumi}{1}
\tightlist
\item
  Quotient Rule
\end{enumerate}

\[
\frac{d}{dx}\left[\frac{f(x)}{g(x)}\right] = \frac{f'(x)g(x)-f(x)g'(x)}{[g(x)]^2}, \quad g(x)\neq 0
\]

\begin{enumerate}
\def\labelenumi{\arabic{enumi}.}
\setcounter{enumi}{2}
\tightlist
\item
  Chain Rule
\end{enumerate}

\[
\frac{d}{dx}[f(g(x))] = f'(g(x))\cdot g'(x)
\]

\subsubsection{B.3 Common Series
Expansions}\label{b.3-common-series-expansions}

Power series let us express functions as infinite polynomials. These
expansions are essential for approximations, solving differential
equations, and building intuition about functions in calculus.

\paragraph{Geometric Series}\label{geometric-series}

\[
\frac{1}{1-x} = \sum_{n=0}^\infty x^n, \quad |x| <1
\]

\paragraph{Exponential Function}\label{exponential-function}

\[
e^x = \sum_{n=0}^\infty \frac{x^n}{n!}
= 1 + x + \frac{x^2}{2!} + \frac{x^3}{3!} + \cdots
\]

Valid for all \(x\).

\paragraph{Trigonometric Functions}\label{trigonometric-functions-1}

\[
\sin x = \sum_{n=0}^\infty (-1)^n \frac{x^{2n+1}}{(2n+1)!}
= x - \frac{x^3}{3!} + \frac{x^5}{5!} - \cdots
\]

\[
\cos x = \sum_{n=0}^\infty (-1)^n \frac{x^{2n}}{(2n)!}
= 1 - \frac{x^2}{2!} + \frac{x^4}{4!} - \cdots
\]

\[
\tan^{-1} x = \sum_{n=0}^\infty (-1)^n \frac{x^{2n+1}}{2n+1}, \quad |x|\leq 1
\]

\paragraph{Logarithm}\label{logarithm}

\[
\ln(1+x) = \sum_{n=1}^\infty (-1)^{n+1} \frac{x^n}{n}, \quad -1 < x \leq 1
\]

\paragraph{Binomial Expansion
(Generalized)}\label{binomial-expansion-generalized}

\[
(1+x)^r = \sum_{n=0}^\infty \binom{r}{n} x^n, \quad |x|<1
\]

where

\[
\binom{r}{n} = \frac{r(r-1)(r-2)\cdots(r-n+1)}{n!}.
\]

\subsection{Appendice C. Schizzi di
prova}\label{appendice-c.-schizzi-di-prova}

\subsubsection{\texorpdfstring{C.1 Leggi sui limiti e
\(\varepsilon\)--\(\delta\) DefinizioneIl calcolo si basa sul
significato preciso di un limite. Mentre l'intuizione (``i valori si
avvicinano sempre di più'') è utile, una definizione formale garantisce
rigore ed evita
paradossi.}{C.1 Leggi sui limiti e \textbackslash varepsilon--\textbackslash delta DefinizioneIl calcolo si basa sul significato preciso di un limite. Mentre l'intuizione (``i valori si avvicinano sempre di più'') è utile, una definizione formale garantisce rigore ed evita paradossi.}}\label{c.1-leggi-sui-limiti-e-varepsilondelta-definizioneil-calcolo-si-basa-sul-significato-preciso-di-un-limite.-mentre-lintuizione-i-valori-si-avvicinano-sempre-di-piuxf9-uxe8-utile-una-definizione-formale-garantisce-rigore-ed-evita-paradossi.}

\paragraph{Idea intuitiva}\label{idea-intuitiva}

Scriviamo

\[
\lim_{x \to a} f(x) = L
\]

ciò significa che poiché \(x\) si avvicina arbitrariamente a \(a\), i
valori di \(f(x)\) si avvicinano arbitrariamente a \(L\).

\paragraph{\texorpdfstring{Definizione formale
(\(\varepsilon\)--\(\delta\))}{Definizione formale (\textbackslash varepsilon--\textbackslash delta)}}\label{definizione-formale-varepsilondelta}

Lo diciamo

\[
\lim_{x \to a} f(x) = L
\]

se per ogni \(\varepsilon > 0\) esiste un \(\delta > 0\) tale che ogni
volta

\[
0 < |x-a| < \delta,
\]

abbiamo

\[
|f(x) - L| < \varepsilon.
\]

\begin{itemize}
\tightlist
\item
  \(\varepsilon\): quanto vicino vogliamo che \(f(x)\) sia a \(L\).
\item
  \(\delta\): quanto deve essere vicino \(x\) a \(a\) per raggiungere
  questo obiettivo.
\end{itemize}

\paragraph{Esempio}\label{esempio-2}

Mostralo

\[
\lim_{x \to 2} (3x+1) = 7.
\]

\begin{itemize}
\tightlist
\item
  Lascia che \(\varepsilon > 0\).
\item
  Vogliamo \(|(3x+1)-7| < \varepsilon\).
\item
  Semplifica: \(|3x-6| = 3|x-2| < \varepsilon\).
\item
  Questo vale se scegliamo \(\delta = \varepsilon/3\).
\end{itemize}

Quindi, per definizione, il limite è 7.

\paragraph{Leggi sui limiti}\label{leggi-sui-limiti}

Se \(\lim_{x \to a} f(x) = L\) e \(\lim_{x \to a} g(x) = M\), allora:

\begin{enumerate}
\def\labelenumi{\arabic{enumi}.}
\tightlist
\item
  Somma/Differenza
\end{enumerate}

\[
\lim_{x \to a} [f(x) \pm g(x)] = L \pm M
\]

\begin{enumerate}
\def\labelenumi{\arabic{enumi}.}
\setcounter{enumi}{1}
\tightlist
\item
  Multiplo costante
\end{enumerate}

\[
\lim_{x \to a} [c f(x)] = cL
\]

\begin{enumerate}
\def\labelenumi{\arabic{enumi}.}
\setcounter{enumi}{2}
\tightlist
\item
  Prodotto
\end{enumerate}

\[
\lim_{x \to a} [f(x)g(x)] = LM
\]

\begin{enumerate}
\def\labelenumi{\arabic{enumi}.}
\setcounter{enumi}{3}
\tightlist
\item
  Quoziente (se \(M \neq 0\))
\end{enumerate}

\[
\lim_{x \to a} \frac{f(x)}{g(x)} = \frac{L}{M}
\]

\begin{enumerate}
\def\labelenumi{\arabic{enumi}.}
\setcounter{enumi}{4}
\tightlist
\item
  Poteri e radici
\end{enumerate}

\[
\lim_{x \to a} [f(x)]^n = L^n, \quad \lim_{x \to a} \sqrt[n]{f(x)} = \sqrt[n]{L} \ (\text{if defined}).
\]

\subsubsection{C.2 Schizzo di dimostrazione: il teorema fondamentale del
calcolo
infinitesimale}\label{c.2-schizzo-di-dimostrazione-il-teorema-fondamentale-del-calcolo-infinitesimale}

Il Teorema Fondamentale del Calcolo (FTC) collega le due operazioni
centrali del calcolo: differenziazione e integrazione. Ciò dimostra che
si tratta, in effetti, di processi inversi.

\paragraph{Enunciato del Teorema}\label{enunciato-del-teorema}

Parte I (Differenziazione di un integrale): Se \(f\) è continuo su
\([a,b]\) e definiamo

\[F(x) = \int_a^x f(t)\,dt,
\]

then \(F\) is differentiable on \((a,b)\) and

\[
F'(x) = f(x).
\]

Part II (Evaluation of a Definite Integral): If \(F\) is any
antiderivative of \(f\) on \([a,b]\), then

\[
\int_a^b f(x)\,dx = F(b)-F(a).
\]

\paragraph{Proof Sketch of Part I}\label{proof-sketch-of-part-i}

\begin{enumerate}
\def\labelenumi{\arabic{enumi}.}
\item
  Start with the definition of the derivative:

  \[
  F'(x) = \lim_{h\to 0} \frac{F(x+h)-F(x)}{h}.
  \]
\item
  Substituting \(F(x) = \int_a^x f(t)\,dt\):

  \[
  F(x+h)-F(x) = \int_a^{x+h} f(t)\,dt - \int_a^x f(t)\,dt.
  \]
\item
  By the additivity of integrals:

  \[
  F(x+h)-F(x) = \int_x^{x+h} f(t)\,dt.
  \]
\item
  Therefore:

  \[
  \frac{F(x+h)-F(x)}{h} = \frac{1}{h}\int_x^{x+h} f(t)\,dt.
  \]
\item
  By the Mean Value Theorem for integrals, there exists
  \(c \in [x,x+h]\) such that

  \[
  \frac{1}{h}\int_x^{x+h} f(t)\,dt = f(c).
  \]
\item
  As \(h \to 0\), \(c \to x\), and since \(f\) is continuous:

  \[
  \lim_{h\to 0} f(c) = f(x).
  \]
\end{enumerate}

Thus, \(F'(x) = f(x)\).

\paragraph{Proof Sketch of Part II}\label{proof-sketch-of-part-ii}

\begin{enumerate}
\def\labelenumi{\arabic{enumi}.}
\item
  Let \(F\) be an antiderivative of \(f\), so \(F'(x) = f(x)\).
\item
  By Part I, the function

  \[
  G(x) = \int_a^x f(t)\,dt
  \]

  is also an antiderivative of \(f\).
\item
  Since \(F\) and \(G\) differ only by a constant,

  \[
  F(x) = G(x) + C.
  \]
\item
  Evaluating at the endpoints:

  \[
  \int_a^b f(x)\,dx = Sol(b)-Sol(a) = (F(b)+C)-(F(a)+C) = F(b)-F(a).
  \]
\end{enumerate}

\subsubsection{C.3 Proof Sketch: Convergence of the Geometric
Series}\label{c.3-proof-sketch-convergence-of-the-geometric-series}

The geometric series is one of the simplest and most important infinite
series. It serves as a model for understanding convergence and is the
foundation for many later results in calculus.

\paragraph{The Series}\label{the-series}

\[
\sum_{n=0}^\infty ar^n = a + ar + ar^2 + ar^3 + \cdots
\]

where \(a\) is the first term and \(r\) is the common ratio.

\paragraph{Partial Sum Formula}\label{partial-sum-formula}

The \(n\)-th partial sum is

\[S_n = a + ar + ar^2 + \cdots + ar^n.
\]

Multiply both sides by \(r\):

\[
rS_n = ar + ar^2 + \cdots + ar^{n+1}.
\]

Subtract the two equations:

\[
S_n - rS_n = a - ar^{n+1}.
\]

\[
S_n(1-r) = a(1-r^{n+1}).
\]

So

\[
S_n = \frac{a(1-r^{n+1})}{1-r}, \quad r \neq 1.
\]

\paragraph{Convergence}\label{convergence}

Take the limit as \(n \to \infty\):

\begin{itemize}
\item
  If \(|r| < 1\), then \(r^{n+1} \to 0\).

  \[
  \lim_{n\to\infty} S_n = \frac{a}{1-r}.
  \]
\item
  If \(|r| \geq 1\), then \(r^{n+1}\) does not go to 0. The series
  diverges.
\end{itemize}

\paragraph{Result}\label{result}

\[
\sum_{n=0}^\infty ar^n =
\begin{casi}
\dfrac{a}{1-r}, & |r|<1, \\[6pt]
\text{diverge}, & |r|\geq 1.
\end{casi}
\]

\subsection{Appendix D. Applications and
Connections}\label{appendix-d.-applications-and-connections}

\subsubsection{D.1 Physics Connections: Velocity, Acceleration, and
Work}\label{d.1-physics-connections-velocity-acceleration-and-work}

Calculus was originally developed to solve problems in physics -
especially motion and change. Here are some of the most important
connections.

\paragraph{Position, Velocity, and
Acceleration}\label{position-velocity-and-acceleration}

\begin{itemize}
\item
  Position function: \(s(t)\) gives the location of an object at time
  \(t\).
\item
  Velocity: the derivative of position.

  \[
  v(t) = s'(t) = \frac{ds}{dt}
  \]
\item
  Acceleration: the derivative of velocity (or second derivative of
  position).

  \[
  a(t) = v'(t) = s''(t) = \frac{d^2s}{dt^2}
  \]
\end{itemize}

Example: If \(s(t) = 4t^2\) meters, then:

\[
v(t) = 8t, \quad a(t) = 8.
\]

So the object moves faster linearly with time, under constant
acceleration.

\paragraph{Work and Force}\label{work-and-force}

In physics, work is the product of force and distance. If force varies
with position, calculus gives:

\[
W = \int_a^b F(x)\, dx
\]

where \(F(x)\) is the force at position \(x\), and the object moves from
\(x=a\) to \(x=b\).

Example: A spring with Hooke's law force \(F(x) = kx\) requires work

\[
W = \int_0^d kx\, dx = \frac{1}{2}kd^2
\]

per allungare la molla di una distanza \(d\).

\paragraph{\texorpdfstring{Energia e aree sotto curve- Energia cinetica:
\(E_k = \tfrac{1}{2}mv^2\).}{Energia e aree sotto curve- Energia cinetica: E\_k = \textbackslash tfrac\{1\}\{2\}mv\^{}2.}}\label{energia-e-aree-sotto-curve--energia-cinetica-e_k-tfrac12mv2.}

\begin{itemize}
\tightlist
\item
  L'energia potenziale spesso coinvolge integrali (ad esempio, l'energia
  potenziale gravitazionale derivante dalla forza di gravità).
\item
  In generale, l'integrazione di una funzione di forza fornisce
  l'energia immagazzinata o il lavoro svolto.
\end{itemize}

\paragraph{Pratica veloce}\label{pratica-veloce}

\begin{enumerate}
\def\labelenumi{\arabic{enumi}.}
\tightlist
\item
  Se \(s(t) = t^3 - 3t\), trova \(v(t)\) e \(a(t)\).
\item
  Calcola il lavoro compiuto da una forza costante di 10 N che sposta un
  oggetto di 5 m.
\item
  Una molla ha costante \(k=200\). Quanto lavoro è necessario per
  allungarlo di 0,1 m?
\item
  Mostra che l'accelerazione è la derivata seconda della posizione.
\item
  Spiegare come l'integrale \(\int v(t)\, dt\) si relaziona allo
  spostamento.
\end{enumerate}

\subsubsection{D.2 Connessioni tra probabilità e
statistica}\label{d.2-connessioni-tra-probabilituxe0-e-statistica}

Il calcolo infinitesimale è profondamente connesso con la probabilità e
la statistica, soprattutto quando si ha a che fare con variabili casuali
continue. Gli integrali diventano essenziali per definire probabilità,
medie e aspettative.

\paragraph{Funzioni di densità di probabilità
(PDF)}\label{funzioni-di-densituxe0-di-probabilituxe0-pdf}

Per una variabile casuale continua \(X\), le probabilità sono descritte
da una funzione di densità di probabilità \(f(x)\):

\begin{enumerate}
\def\labelenumi{\arabic{enumi}.}
\item
  \(f(x) \geq 0\) per tutti i \(x\).
\item
  La probabilità totale è uguale a 1:

  \[
  \int_{-\infty}^{\infty} f(x)\, dx = 1.
  \]
\end{enumerate}

La probabilità che \(X\) si trovi in un intervallo \([a,b]\) è

\[
P(a \leq X \leq b) = \int_a^b f(x)\, dx.
\]

\paragraph{Valore atteso (media)}\label{valore-atteso-media-1}

Il valore atteso (risultato medio) è

\[
E[X] = \int_{-\infty}^{\infty} x f(x)\, dx.
\]

Questa è la versione di calcolo di una media ponderata.

\paragraph{Varianza}\label{varianza}

Le misure di varianza si diffondono:

\[
\text{Var}(X) = E[(X-\mu)^2] = \int_{-\infty}^{\infty} (x-\mu)^2 f(x)\, dx,
\]

dove \(\mu = E[X]\).

\paragraph{Distribuzioni comuni}\label{distribuzioni-comuni}

\begin{enumerate}
\def\labelenumi{\arabic{enumi}.}
\item
  Distribuzione uniforme su \([a,b]\):

  \[
  f(x) = \frac{1}{b-a}, \quad a \leq x \leq b.
  \]

  Significa: \(\frac{a+b}{2}\).
\item
  Distribuzione esponenziale con parametro \(\lambda > 0\):

  \[
  f(x) = \lambda e^{-\lambda x}, \quad x \geq 0.\]

  Mean: \(1/\lambda\).
\item
  Normal (Gaussian) distribution:

  \[
  f(x) = \frac{1}{\sqrt{2\pi\sigma^2}} e^{-(x-\mu)^2/(2\sigma^2)}.
  \]

  Integrals of this distribution connect to the error function.
\end{enumerate}

\paragraph{Why This Matters}\label{why-this-matters-3}

\begin{itemize}
\tightlist
\item
  Integrals turn probabilities into areas under curves.
\item
  Expectation and variance link calculus to averages and variability.
\item
  Most real-world data models (finance, physics, biology, AI) use these
  continuous probability distributions.
\end{itemize}

\paragraph{Quick Practice}\label{quick-practice}

\begin{enumerate}
\def\labelenumi{\arabic{enumi}.}
\tightlist
\item
  For \(f(x) = \tfrac{1}{2}\) on \([0,2]\), compute
  \(P(0.5 \leq X \leq 1.5)\).
\item
  For exponential distribution with \(\lambda = 2\), compute \(E[X]\).
\item
  Show that the total area under the standard normal curve equals 1.
\item
  Find the mean of a uniform distribution on \([3,7]\).
\item
  Explain why probabilities are computed with integrals, not sums, for
  continuous variables.
\end{enumerate}

\subsubsection{D.3 Computer Science Connections: Taylor Approximations
in
Algorithms}\label{d.3-computer-science-connections-taylor-approximations-in-algorithms}

Calculus is not only for physics - it also underpins many tools and
techniques in computer science. One of the clearest bridges is through
Taylor series, which provide efficient ways to approximate functions in
numerical computing and algorithms.

\paragraph{Function Approximation for
Computing}\label{function-approximation-for-computing}

Computers cannot directly store or calculate most functions exactly
(like \(e^x\), \(\sin x\), or \(\ln x\)). Instead, they use polynomial
approximations derived from Taylor expansions.

Example: To approximate \(e^x\), truncate the Maclaurin series:

\[
e^x \circa 1 + x + \frac{x^2}{2!} + \frac{x^3}{3!}.
\]

Per \(x\) piccolo, questo polinomio fornisce risultati accurati con solo
pochi termini.

\paragraph{Efficienza negli algoritmi}\label{efficienza-negli-algoritmi}

\begin{itemize}
\tightlist
\item
  Funzioni trigonometriche: gli algoritmi per calcolatrici e CPU spesso
  utilizzano espansioni in serie (o variazioni come i polinomi di
  Chebyshev).- Esponenziale/logaritmo: gli sviluppi di Taylor sono il
  fondamento delle approssimazioni veloci nelle librerie numeriche.
\item
  Ricerca della radice: il metodo di Newton si basa sull'approssimazione
  lineare, un'applicazione diretta della serie di Taylor (derivata
  prima).
\end{itemize}

\paragraph{Analisi numerica}\label{analisi-numerica}

Le espansioni di Taylor sono centrali nell'analisi degli errori:

\begin{itemize}
\item
  Approssimazione del termine di errore utilizzando la formula del
  resto:

  \[
  R_n(x) = \frac{f^{(n+1)}(c)}{(n+1)!}(x-a)^{n+1}.
  \]
\item
  Questo ci dice quanti termini sono necessari per una determinata
  precisione.
\end{itemize}

\paragraph{Connessioni per l'apprendimento
automatico}\label{connessioni-per-lapprendimento-automatico}

\begin{itemize}
\tightlist
\item
  L'ottimizzazione basata sul gradiente (come la discesa del gradiente)
  utilizza i derivati per aggiornare i parametri in modo efficiente.
\item
  Le funzioni di attivazione (come \(\tanh x\) o
  \(\sigma(x)=1/(1+e^{-x})\)) sono spesso approssimate da polinomi o
  funzioni a tratti per la velocità.
\item
  Le approssimazioni in serie possono accelerare l'addestramento e
  l'inferenza in ambienti limitati.
\end{itemize}

\paragraph{Perché è importante}\label{perchuxe9-uxe8-importante-33}

\begin{itemize}
\tightlist
\item
  Le approssimazioni di Taylor collegano la matematica continua con il
  calcolo discreto.
\item
  Mostrano come i concetti di calcolo vengono utilizzati negli
  algoritmi, nei metodi numerici e nell'apprendimento automatico.
\item
  Comprendere le approssimazioni aiuta a evitare le trappole quando ci
  si affida ai computer per i calcoli.
\end{itemize}

\paragraph{Pratica veloce}\label{pratica-veloce-1}

\begin{enumerate}
\def\labelenumi{\arabic{enumi}.}
\tightlist
\item
  Approssimativo \(\sin(0.1)\) utilizzando i primi tre termini della
  serie Maclaurin.
\item
  Utilizzare il termine residuo per stimare l'errore
  nell'approssimazione di \(e^1\) con un polinomio di grado 3.
\item
  Spiega come il metodo di Newton utilizza il teorema di Taylor.
\item
  Perché i computer potrebbero preferire le approssimazioni polinomiali
  alle formule esatte per le funzioni?
\item
  Nell'apprendimento automatico, perché la derivata (gradiente) è così
  critica per l'ottimizzazione?
\end{enumerate}




\end{document}
