% Options for packages loaded elsewhere
\PassOptionsToPackage{unicode}{hyperref}
\PassOptionsToPackage{hyphens}{url}
\PassOptionsToPackage{dvipsnames,svgnames,x11names}{xcolor}
%
\documentclass[
  letterpaper,
  DIV=11,
  numbers=noendperiod]{scrartcl}

\usepackage{amsmath,amssymb}
\usepackage{iftex}
\ifPDFTeX
  \usepackage[T1]{fontenc}
  \usepackage[utf8]{inputenc}
  \usepackage{textcomp} % provide euro and other symbols
\else % if luatex or xetex
  \usepackage{unicode-math}
  \defaultfontfeatures{Scale=MatchLowercase}
  \defaultfontfeatures[\rmfamily]{Ligatures=TeX,Scale=1}
\fi
\usepackage{lmodern}
\ifPDFTeX\else  
    % xetex/luatex font selection
\fi
% Use upquote if available, for straight quotes in verbatim environments
\IfFileExists{upquote.sty}{\usepackage{upquote}}{}
\IfFileExists{microtype.sty}{% use microtype if available
  \usepackage[]{microtype}
  \UseMicrotypeSet[protrusion]{basicmath} % disable protrusion for tt fonts
}{}
\makeatletter
\@ifundefined{KOMAClassName}{% if non-KOMA class
  \IfFileExists{parskip.sty}{%
    \usepackage{parskip}
  }{% else
    \setlength{\parindent}{0pt}
    \setlength{\parskip}{6pt plus 2pt minus 1pt}}
}{% if KOMA class
  \KOMAoptions{parskip=half}}
\makeatother
\usepackage{xcolor}
\setlength{\emergencystretch}{3em} % prevent overfull lines
\setcounter{secnumdepth}{-\maxdimen} % remove section numbering
% Make \paragraph and \subparagraph free-standing
\makeatletter
\ifx\paragraph\undefined\else
  \let\oldparagraph\paragraph
  \renewcommand{\paragraph}{
    \@ifstar
      \xxxParagraphStar
      \xxxParagraphNoStar
  }
  \newcommand{\xxxParagraphStar}[1]{\oldparagraph*{#1}\mbox{}}
  \newcommand{\xxxParagraphNoStar}[1]{\oldparagraph{#1}\mbox{}}
\fi
\ifx\subparagraph\undefined\else
  \let\oldsubparagraph\subparagraph
  \renewcommand{\subparagraph}{
    \@ifstar
      \xxxSubParagraphStar
      \xxxSubParagraphNoStar
  }
  \newcommand{\xxxSubParagraphStar}[1]{\oldsubparagraph*{#1}\mbox{}}
  \newcommand{\xxxSubParagraphNoStar}[1]{\oldsubparagraph{#1}\mbox{}}
\fi
\makeatother


\providecommand{\tightlist}{%
  \setlength{\itemsep}{0pt}\setlength{\parskip}{0pt}}\usepackage{longtable,booktabs,array}
\usepackage{calc} % for calculating minipage widths
% Correct order of tables after \paragraph or \subparagraph
\usepackage{etoolbox}
\makeatletter
\patchcmd\longtable{\par}{\if@noskipsec\mbox{}\fi\par}{}{}
\makeatother
% Allow footnotes in longtable head/foot
\IfFileExists{footnotehyper.sty}{\usepackage{footnotehyper}}{\usepackage{footnote}}
\makesavenoteenv{longtable}
\usepackage{graphicx}
\makeatletter
\newsavebox\pandoc@box
\newcommand*\pandocbounded[1]{% scales image to fit in text height/width
  \sbox\pandoc@box{#1}%
  \Gscale@div\@tempa{\textheight}{\dimexpr\ht\pandoc@box+\dp\pandoc@box\relax}%
  \Gscale@div\@tempb{\linewidth}{\wd\pandoc@box}%
  \ifdim\@tempb\p@<\@tempa\p@\let\@tempa\@tempb\fi% select the smaller of both
  \ifdim\@tempa\p@<\p@\scalebox{\@tempa}{\usebox\pandoc@box}%
  \else\usebox{\pandoc@box}%
  \fi%
}
% Set default figure placement to htbp
\def\fps@figure{htbp}
\makeatother

\KOMAoption{captions}{tableheading}
\makeatletter
\@ifpackageloaded{caption}{}{\usepackage{caption}}
\AtBeginDocument{%
\ifdefined\contentsname
  \renewcommand*\contentsname{Содержание}
\else
  \newcommand\contentsname{Содержание}
\fi
\ifdefined\listfigurename
  \renewcommand*\listfigurename{Список Иллюстраций}
\else
  \newcommand\listfigurename{Список Иллюстраций}
\fi
\ifdefined\listtablename
  \renewcommand*\listtablename{Список Таблиц}
\else
  \newcommand\listtablename{Список Таблиц}
\fi
\ifdefined\figurename
  \renewcommand*\figurename{Рисунок}
\else
  \newcommand\figurename{Рисунок}
\fi
\ifdefined\tablename
  \renewcommand*\tablename{Таблица}
\else
  \newcommand\tablename{Таблица}
\fi
}
\@ifpackageloaded{float}{}{\usepackage{float}}
\floatstyle{ruled}
\@ifundefined{c@chapter}{\newfloat{codelisting}{h}{lop}}{\newfloat{codelisting}{h}{lop}[chapter]}
\floatname{codelisting}{Список}
\newcommand*\listoflistings{\listof{codelisting}{Список Каталогов}}
\makeatother
\makeatletter
\makeatother
\makeatletter
\@ifpackageloaded{caption}{}{\usepackage{caption}}
\@ifpackageloaded{subcaption}{}{\usepackage{subcaption}}
\makeatother

\ifLuaTeX
\usepackage[bidi=basic]{babel}
\else
\usepackage[bidi=default]{babel}
\fi
\babelprovide[main,import]{russian}
% get rid of language-specific shorthands (see #6817):
\let\LanguageShortHands\languageshorthands
\def\languageshorthands#1{}
\usepackage{bookmark}

\IfFileExists{xurl.sty}{\usepackage{xurl}}{} % add URL line breaks if available
\urlstyle{same} % disable monospaced font for URLs
\hypersetup{
  pdftitle={Маленькая книга по исчислению},
  pdflang={ru},
  colorlinks=true,
  linkcolor={blue},
  filecolor={Maroon},
  citecolor={Blue},
  urlcolor={Blue},
  pdfcreator={LaTeX via pandoc}}


\title{Маленькая книга по исчислению}
\author{}
\date{}

\begin{document}
\maketitle


\section{Маленькая книга по
исчислению}\label{ux43cux430ux43bux435ux43dux44cux43aux430ux44f-ux43aux43dux438ux433ux430-ux43fux43e-ux438ux441ux447ux438ux441ux43bux435ux43dux438ux44e}

Краткое, удобное для начинающих введение в основные идеи исчисления.

\subsection{Форматы}\label{ux444ux43eux440ux43cux430ux442ux44b}

\begin{itemize}
\tightlist
\item
  \href{../artifacts/ru/book.pdf}{Скачать PDF} -- версия для печати
\item
  \href{../artifacts/ru/book.epub}{Скачать EPUB} -- подходит для
  электронных книг
\item
  \href{../artifacts/ru/book.tex}{Исходник LaTeX} -- источник латекса
\end{itemize}

\section{Часть 1. Пределы и
производные}\label{ux447ux430ux441ux442ux44c-1.-ux43fux440ux435ux434ux435ux43bux44b-ux438-ux43fux440ux43eux438ux437ux432ux43eux434ux43dux44bux435}

\section{Глава 1. Функции и
ограничения}\label{ux433ux43bux430ux432ux430-1.-ux444ux443ux43dux43aux446ux438ux438-ux438-ux43eux433ux440ux430ux43dux438ux447ux435ux43dux438ux44f}

\subsection{1.1 Функции}\label{ux444ux443ux43dux43aux446ux438ux438}

Функция -- один из самых основных объектов математики. По своей сути
функция --- это правило, которое принимает входные данные и выдает ровно
один выходной результат. Функции позволяют нам описывать отношения,
моделировать явления реального мира и строить весь механизм исчисления.

\subsubsection{Определение}\label{ux43eux43fux440ux435ux434ux435ux43bux435ux43dux438ux435}

Формально записывается функция \(f\) из набора \(X\) (называемого
доменом) в набор \(Y\) (называемого кодоменом).

\[
f : X \to Y.
\]

Для каждого элемента \(x \in X\) существует уникальный элемент
\(f(x) \in Y\). Значение \(f(x)\) называется образом \(x\) в \(f\).

Если \(y = f(x)\), то \(y\) --- это выход, соответствующий входу \(x\).
Набор всех выходных данных, которые фактически появляются, называется
диапазоном (подмножество кодомена).

\subsubsection{Примеры}\label{ux43fux440ux438ux43cux435ux440ux44b}

\begin{enumerate}
\def\labelenumi{\arabic{enumi}.}
\item
  Функция \(f(x) = x^2\) сопоставляет каждое действительное число \(x\)
  с его квадратом.

  \begin{itemize}
  \tightlist
  \item
    Домен: все действительные числа \(\mathbb{R}\).
  \item
    Кодомен: все действительные числа \(\mathbb{R}\).
  \item
    Диапазон: все неотрицательные действительные числа \([0, \infty)\).
  \end{itemize}
\item
  Функция \(g(x) = \dfrac{1}{x}\) присваивает каждому ненулевому
  действительному числу его обратное значение.

  \begin{itemize}
  \tightlist
  \item
    Домен: \(\mathbb{R} \setminus \{0\}\).
  \item
    Диапазон: \(\mathbb{R} \setminus \{0\}\).
  \end{itemize}
\item
  Реальный пример: пусть \(T(t)\) --- это наружная температура (в °C) в
  момент времени \(t\) (в часах). Это функция от «времени суток» до
  «температуры».
\end{enumerate}

\subsubsection{Способы представления
функций}\label{ux441ux43fux43eux441ux43eux431ux44b-ux43fux440ux435ux434ux441ux442ux430ux432ux43bux435ux43dux438ux44f-ux444ux443ux43dux43aux446ux438ux439}

Функции могут быть представлены несколькими полезными способами:

-- Формулы: например, \(f(x) = \sin x + x^2\). - Графики: отображение
всех точек \((x, f(x))\) в координатной плоскости. - Таблицы: сопряжение
входов и выходов для дискретных наборов данных. - Словесные описания:
«Поставьте каждому ученику оценку».

Каждое представление подчеркивает различные аспекты одной и той же
функции.

\subsubsection{Терминология}\label{ux442ux435ux440ux43cux438ux43dux43eux43bux43eux433ux438ux44f}

\begin{itemize}
\tightlist
\item
  Независимая переменная: входная (обычно обозначается \(x\)).
\item
  Зависимая переменная: вывод (обычно пишется \(y\), где \(y = f(x)\)).
\item
  Обозначение функции: \(f(x)\) читается как «\(f\) из \(x\)».
\end{itemize}

\subsubsection{Почему функции важны в
исчислении}\label{ux43fux43eux447ux435ux43cux443-ux444ux443ux43dux43aux446ux438ux438-ux432ux430ux436ux43dux44b-ux432-ux438ux441ux447ux438ux441ux43bux435ux43dux438ux438}

Исчисление -- это изучение того, как изменяются функции. Производные
измеряют мгновенные темпы изменений, а интегралы измеряют накопленные
эффекты. Чтобы освоить эти идеи, нам сначала нужно четкое понимание
того, что такое функции и как они ведут себя.

\subsubsection{Упражнения}\label{ux443ux43fux440ux430ux436ux43dux435ux43dux438ux44f}

\begin{enumerate}
\def\labelenumi{\arabic{enumi}.}
\item
  Для функции \(f(x) = 3x - 2\):- Найдите домен, кодомен и диапазон.
\item
  Для каких входов определена функция \(h(x) = \sqrt{x-1}\)? Каков его
  диапазон?
\item
  Приведите реальный пример функции из вашей повседневной жизни. Четко
  укажите домен и кодомен.
\item
  Нарисуйте график \(f(x) = |x|\). Каков диапазон?
\item
  Предположим, \(g(x) = \dfrac{1}{x^2+1}\). Объясните, почему его
  диапазон --- это интервал \((0, 1]\).
\end{enumerate}

\subsection{1.2 Графы и
преобразования}\label{ux433ux440ux430ux444ux44b-ux438-ux43fux440ux435ux43eux431ux440ux430ux437ux43eux432ux430ux43dux438ux44f}

Функцию можно понять не только по формулам, но и по ее графику. График
функции \(f\) представляет собой набор всех упорядоченных пар
\((x, f(x))\), где \(x\) принадлежит домену \(f\). Отображение этих пар
на координатной плоскости дает представление о том, как ведет себя
функция.

\subsubsection{Базовые
графики}\label{ux431ux430ux437ux43eux432ux44bux435-ux433ux440ux430ux444ux438ux43aux438}

Некоторые графики настолько фундаментальны, что их следует запомнить:

\begin{itemize}
\tightlist
\item
  \(f(x) = x\): прямая линия, проходящая через начало координат.
\item
  \(f(x) = x^2\): парабола, открывающаяся вверх.
\item
  \(f(x) = |x|\): график в форме буквы «V».
\item
  \(f(x) = \frac{1}{x}\): гипербола с двумя ветвями.
\item
  \(f(x) = \sin x\): волнообразная периодическая кривая.
\end{itemize}

Они служат строительными блоками для более сложных функций.

\subsubsection{Преобразования}\label{ux43fux440ux435ux43eux431ux440ux430ux437ux43eux432ux430ux43dux438ux44f}

Графики можно сдвигать, растягивать или отражать, используя простые
правила:

\begin{enumerate}
\def\labelenumi{\arabic{enumi}.}
\item
  Вертикальные сдвиги. Добавление константы перемещает график вверх или
  вниз.

  \[
  y = f(x) + c \quad \text{is } f(x) \text{ shifted upward by } c.
  \]
\item
  Горизонтальные сдвиги: добавление внутри аргумента перемещает график
  влево или вправо.

  \[
  y = f(x - c) \quad \text{is } f(x) \text{ shifted right by } c.
  \]
\item
  Вертикальное масштабирование. Умножение на константу растягивает или
  сжимает график по вертикали.

  \[
  y = a f(x), \quad a > 1 \text{ stretches; } 0 < a < 1 \text{ compresses.}
  \]
\item
  Горизонтальное масштабирование. Умножение внутри аргумента растягивает
  или сжимает график по горизонтали.

  \[
  y = f(bx), \quad b > 1 \text{ compresses toward the } y\text{-axis}.
  \]
\item
  Размышления:

  \begin{itemize}
  \tightlist
  \item
    \(y = -f(x)\): отражение по оси \(x\).
  \item
    \(y = f(-x)\): отражение по оси \(y\).
  \end{itemize}
\end{enumerate}

\subsubsection{Объединение
преобразований}\label{ux43eux431ux44aux435ux434ux438ux43dux435ux43dux438ux435-ux43fux440ux435ux43eux431ux440ux430ux437ux43eux432ux430ux43dux438ux439}

Сложные графы часто возникают в результате последовательного объединения
нескольких преобразований. Например:

\[
y = 2(x-1)^2 + 3
\]

получается, если взять параболу \(y = x^2\), сдвинуть ее вправо на 1,
растянуть по вертикали на 2 и сдвинуть вверх на 3.

\subsubsection{Упражнения}\label{ux443ux43fux440ux430ux436ux43dux435ux43dux438ux44f-1}

\begin{enumerate}
\def\labelenumi{\arabic{enumi}.}
\tightlist
\item
  Нарисуйте график \(y = (x+2)^2 - 1\). Определите последовательность
  преобразований из \(y = x^2\).
\item
  Что произойдет с графиком \(y = f(x)\), если заменить \(x\) на \(-x\)?
  Попробуйте это с \(f(x) = \sqrt{x}\).
\item
  Опишите преобразования, которые превращают \(y = \sin x\) в
  \(y = 3\sin(x - \pi/4)\).4. Нарисуйте график \(y = |x-1| + 2\).
  Укажите ее вершину и наклон каждой ветви.
\item
  Для \(y = \frac{1}{x-2}\) объясните, как был преобразован график
  \(y = \frac{1}{x}\).
\end{enumerate}

\subsection{1.3 Интуитивное представление о
пределах}\label{ux438ux43dux442ux443ux438ux442ux438ux432ux43dux43eux435-ux43fux440ux435ux434ux441ux442ux430ux432ux43bux435ux43dux438ux435-ux43e-ux43fux440ux435ux434ux435ux43bux430ux445}

Во многих ситуациях значение функции в определенной точке менее важно,
чем значения, которые она принимает вблизи этой точки. Концепция предела
отражает эту идею.

\subsubsection{Приближение к
значению}\label{ux43fux440ux438ux431ux43bux438ux436ux435ux43dux438ux435-ux43a-ux437ux43dux430ux447ux435ux43dux438ux44e}

Представьте, что вы идете к стене. Еще до того, как вы прикоснетесь к
нему, вы подходите все ближе и ближе. Точно так же, когда \(x\)
приближается к числу \(a\), значения \(f(x)\) могут приближаться к
некоторому числу \(L\). Затем мы говорим:

\[
\lim_{x \to a} f(x) = L.
\]

Это выражает идею о том, что \(f(x)\) можно сделать настолько близким к
\(L\), насколько мы хотим, просто приняв \(x\) достаточно близко к
\(a\).

\subsubsection{Примеры}\label{ux43fux440ux438ux43cux435ux440ux44b-1}

\begin{enumerate}
\def\labelenumi{\arabic{enumi}.}
\item
  Для \(f(x) = 2x + 3\): Как \(x \to 1\), \(f(x) \to 5\).
\item
  Для \(f(x) = \dfrac{\sin x}{x}\): Как \(x \to 0\), функция
  приближается к 1, хотя \(f(0)\) не определена.
\item
  Для \(f(x) = \dfrac{1}{x}\): Как \(x \to 0^+\) (приближаясь справа),
  \(f(x) \to +\infty\). Как \(x \to 0^-\) (приближаясь слева),
  \(f(x) \to -\infty\). Поскольку левое и правое поведение различаются,
  предела в 0 не существует.
\end{enumerate}

\subsubsection{Важность
ограничений}\label{ux432ux430ux436ux43dux43eux441ux442ux44c-ux43eux433ux440ux430ux43dux438ux447ux435ux43dux438ux439}

\begin{itemize}
\tightlist
\item
  Они позволяют нам определять функции в тех точках, где они изначально
  не определены.
\item
  Они фиксируют поведение вблизи разрывов и сингулярностей.
\item
  Они образуют основу для производных (мгновенных скоростей изменения) и
  интегралов (площадей как пределов сумм).
\end{itemize}

\subsubsection{Односторонние
ограничения}\label{ux43eux434ux43dux43eux441ux442ux43eux440ux43eux43dux43dux438ux435-ux43eux433ux440ux430ux43dux438ux447ux435ux43dux438ux44f}

Иногда поведение слева и справа необходимо изучать отдельно:

\[
\lim_{x \to a^-} f(x), \quad \lim_{x \to a^+} f(x).
\]

Если оба согласны, то существует двусторонний предел.

\subsubsection{Упражнения}\label{ux443ux43fux440ux430ux436ux43dux435ux43dux438ux44f-2}

\begin{enumerate}
\def\labelenumi{\arabic{enumi}.}
\tightlist
\item
  Вычислите \(\lim_{x \to 2} (3x^2 - x)\).
\item
  Что такое \(\lim_{x \to 0} \frac{\sin x}{x}\)? Используйте интуицию из
  графика \(\sin x\).
\item
  Оцените \(\lim_{x \to 0} |x|/x\). Существует ли двусторонний предел?
\item
  Найдите \(\lim_{x \to \infty} \frac{1}{x}\). Проинтерпретируйте этот
  результат словами.
\item
  Что такое \(f(x) = \frac{x^2-1}{x-1}\) для \(\lim_{x \to 1} f(x)\)?
  Сравните со значением \(f(1)\).
\end{enumerate}

\subsection{1.4 Формальное определение
пределов}\label{ux444ux43eux440ux43cux430ux43bux44cux43dux43eux435-ux43eux43fux440ux435ux434ux435ux43bux435ux43dux438ux435-ux43fux440ux435ux434ux435ux43bux43eux432}

Интуитивное представление о пределе можно уточнить, используя
определение эпсилон-дельта. Это дает нам строгий способ сказать, что
\(f(x)\) приближается к значению \(L\), поскольку \(x\) приближается к
\(a\).

\subsubsection{Определение}\label{ux43eux43fux440ux435ux434ux435ux43bux435ux43dux438ux435-1}

Мы пишем

\[
\lim_{x \to a} f(x) = L
\]

если выполняется следующее условие:

Для каждого \(\varepsilon > 0\) (независимо от его размера) существует
\(\delta > 0\) такой, что всякий раз, когда

\[
0 < |x - a| < \delta,
\]

отсюда следует, что

\[
|f(x) - L| < \varepsilon.
\]Другими словами: мы можем сделать \(f(x)\) настолько близким к \(L\),
при условии, что \(x\) достаточно близок к \(a\) (но не равен \(a\)).

\subsubsection{Пример 1: линейная
функция}\label{ux43fux440ux438ux43cux435ux440-1-ux43bux438ux43dux435ux439ux43dux430ux44f-ux444ux443ux43dux43aux446ux438ux44f}

Для \(f(x) = 2x + 1\) покажите, что \(\lim_{x \to 3} f(x) = 7\).

\begin{itemize}
\tightlist
\item
  Нам нужен \(|f(x) - 7| < \varepsilon\).
\item
  Но \(f(x) - 7 = 2x + 1 - 7 = 2(x - 3)\).
\item
  Итак \(|f(x) - 7| = 2|x - 3|\).
\item
  Если мы выберем \(\delta = \varepsilon / 2\), то всякий раз, когда
  \(|x - 3| < \delta\), у нас будет \(|f(x) - 7| < \varepsilon\). Это
  доказывает предел.
\end{itemize}

\subsubsection{Пример 2: обратная
функция}\label{ux43fux440ux438ux43cux435ux440-2-ux43eux431ux440ux430ux442ux43dux430ux44f-ux444ux443ux43dux43aux446ux438ux44f}

Для \(f(x) = \frac{1}{x}\) рассмотрите
\(\lim_{x \to 2} f(x) = \tfrac{1}{2}\).

\begin{itemize}
\tightlist
\item
  Нам нужен \(\left|\frac{1}{x} - \frac{1}{2}\right| < \varepsilon\).
\item
  Это неравенство требует алгебраических манипуляций, но его можно
  удовлетворить, выбрав \(\delta\) в зависимости от \(\varepsilon\).
  Процесс сложнее, но принцип тот же.
\end{itemize}

\subsubsection{Почему это
важно}\label{ux43fux43eux447ux435ux43cux443-ux44dux442ux43e-ux432ux430ux436ux43dux43e}

\begin{itemize}
\tightlist
\item
  Определение эпсилон-дельта гарантирует, что пределы не являются
  расплывчатыми или основаны только на интуиции.
\item
  Это основа непрерывности, производных и интегралов.
\item
  Хотя новичкам это может показаться абстрактным, работа с простыми
  примерами способствует знакомству.
\end{itemize}

\subsubsection{Упражнения}\label{ux443ux43fux440ux430ux436ux43dux435ux43dux438ux44f-3}

\begin{enumerate}
\def\labelenumi{\arabic{enumi}.}
\tightlist
\item
  Используя определение эпсилон-дельта, докажите, что
  \(\lim_{x \to 4} (x+1) = 5\).
\item
  Покажите, что \(\lim_{x \to 0} 5x = 0\), используя формальное
  определение.
\item
  Объясните, почему \(\lim_{x \to 0} \frac{1}{x}\) не существует.
\item
  Для \(f(x) = x^2\) покажите, что \(\lim_{x \to 2} f(x) = 4\).
\item
  Своими словами объясните роль \(\varepsilon\) и \(\delta\) в
  определении лимита.
\end{enumerate}

\subsection{1.5
Непрерывность}\label{ux43dux435ux43fux440ux435ux440ux44bux432ux43dux43eux441ux442ux44c}

Функция называется непрерывной, если ее график можно построить, не
отрывая карандаша от бумаги. Точнее, непрерывность гарантирует, что
небольшие изменения на входе приводят к небольшим изменениям на выходе.

\subsubsection{Определение}\label{ux43eux43fux440ux435ux434ux435ux43bux435ux43dux438ux435-2}

Функция \(f\) непрерывна в точке \(a\), если выполняются три условия:

\begin{enumerate}
\def\labelenumi{\arabic{enumi}.}
\tightlist
\item
  \(f(a)\) определен.
\item
  \(\lim_{x \to a} f(x)\) существует.
\item
  \(\lim_{x \to a} f(x) = f(a)\).
\end{enumerate}

Если функция непрерывна в каждой точке интервала, мы говорим, что она
непрерывна на этом интервале.

\subsubsection{Примеры}\label{ux43fux440ux438ux43cux435ux440ux44b-2}

\begin{enumerate}
\def\labelenumi{\arabic{enumi}.}
\item
  Полиномиальные функции. Такие функции, как \(f(x) = x^2 + 3x - 5\),
  непрерывны всюду на \(\mathbb{R}\).
\item
  Рациональные функции: \(f(x) = \frac{1}{x-1}\) непрерывен везде, кроме
  \(x = 1\), где он не определен.
\item
  Кусочные функции:

  \[
  f(x) =
  \begin{cases}
  x^2 & x < 1, \\
  2 & x = 1, \\
  x+1 & x > 1,
  \end{cases}
  \]

  Эта функция имеет «переход» в \(x = 1\), поэтому здесь она не
  непрерывна.
\end{enumerate}

\subsubsection{Типы
разрывов}\label{ux442ux438ux43fux44b-ux440ux430ux437ux440ux44bux432ux43eux432}

\begin{enumerate}
\def\labelenumi{\arabic{enumi}.}
\tightlist
\item
  Устранимый разрыв: «дырка» в графике. Пример:
  \(f(x) = \frac{x^2-1}{x-1}\) в \(x=1\).2. Разрыв прыжка. Пределы для
  левой и правой руки различны.
\item
  Бесконечный разрыв: функция переходит к \(\pm\infty\) рядом с точкой,
  как и к \(f(x) = 1/x\) рядом с \(x = 0\).
\end{enumerate}

\subsubsection{Теорема о промежуточном
значении}\label{ux442ux435ux43eux440ux435ux43cux430-ux43e-ux43fux440ux43eux43cux435ux436ux443ux442ux43eux447ux43dux43eux43c-ux437ux43dux430ux447ux435ux43dux438ux438}

Если функция непрерывна на интервале \([a, b]\), то для любого числа
\(N\) между \(f(a)\) и \(f(b)\) существует некоторый \(c \in [a, b]\)
такой, что \(f(c) = N\).

Это свойство имеет решающее значение для доказательства существования
корней и решений уравнений.

\subsubsection{Упражнения}\label{ux443ux43fux440ux430ux436ux43dux435ux43dux438ux44f-4}

\begin{enumerate}
\def\labelenumi{\arabic{enumi}.}
\tightlist
\item
  Определите, является ли функция \(f(x) = |x|\) непрерывной в
  \(x = 0\).
\item
  Определите точки разрыва для \(f(x) = \frac{x+2}{x^2-1}\).
\item
  Объясните, почему каждая полиномиальная функция всюду непрерывна.
\item
  Приведите пример функции со скачком разрыва. Нарисуйте его график.
\item
  Используйте теорему о промежуточном значении, чтобы показать, что
  уравнение \(x^3 + x - 1 = 0\) имеет решение в диапазоне от 0 до 1.
\end{enumerate}

\section{Глава 2.
Производные}\label{ux433ux43bux430ux432ux430-2.-ux43fux440ux43eux438ux437ux432ux43eux434ux43dux44bux435}

\subsection{2.1 Производная как скорость
изменения}\label{ux43fux440ux43eux438ux437ux432ux43eux434ux43dux430ux44f-ux43aux430ux43a-ux441ux43aux43eux440ux43eux441ux442ux44c-ux438ux437ux43cux435ux43dux435ux43dux438ux44f}

Производная --- одна из центральных идей исчисления. Он измеряет, как
изменяется функция по мере изменения ее входных данных --- другими
словами, скорость изменения выходных данных по отношению к входным
данным.

\subsubsection{Средняя скорость
изменения}\label{ux441ux440ux435ux434ux43dux44fux44f-ux441ux43aux43eux440ux43eux441ux442ux44c-ux438ux437ux43cux435ux43dux435ux43dux438ux44f}

Для функции \(f(x)\) средняя скорость изменения между двумя точками
\(x = a\) и \(x = b\) равна

\[
\frac{f(b) - f(a)}{b - a}.
\]

Это наклон секущей линии, проходящей через точки \((a, f(a))\) и
\((b, f(b))\).

\subsubsection{Мгновенная скорость
изменения}\label{ux43cux433ux43dux43eux432ux435ux43dux43dux430ux44f-ux441ux43aux43eux440ux43eux441ux442ux44c-ux438ux437ux43cux435ux43dux435ux43dux438ux44f}

Чтобы измерить, насколько быстро \(f(x)\) меняется в одной точке, мы
позволяем интервалу сокращаться:

\[
f'(a) = \lim_{h \to 0} \frac{f(a+h) - f(a)}{h}.
\]

Этот предел, если он существует, называется производным от \(f\) по
адресу \(a\). Геометрически это наклон касательной к графику \(f\) в
точке \((a, f(a))\).

\subsubsection{Обозначение}\label{ux43eux431ux43eux437ux43dux430ux447ux435ux43dux438ux435}

\begin{itemize}
\tightlist
\item
  \(f'(x)\): простое обозначение.
\item
  \(\dfrac{dy}{dx}\): обозначение Лейбница, используется, когда
  \(y = f(x)\).
\item
  \(Df(x)\): обозначение оператора.
\end{itemize}

Все эти символы относятся к одному и тому же понятию.

\subsubsection{Примеры}\label{ux43fux440ux438ux43cux435ux440ux44b-3}

\begin{enumerate}
\def\labelenumi{\arabic{enumi}.}
\item
  Для \(f(x) = x^2\):

  \[
  f'(x) = \lim_{h \to 0} \frac{(x+h)^2 - x^2}{h} = \lim_{h \to 0} \frac{2xh + h^2}{h} = 2x.
  \]

  Наклон параболы в точке \(x\) равен \(2x\).
\item
  Для \(f(x) = \sin x\):

  \[
  f'(x) = \cos x.
  \]
\item
  Для \(f(x) = c\) (константа):

  \[
  f'(x) = 0.
  \]

  Постоянная функция никогда не меняется.
\end{enumerate}

\subsubsection{Интерпретация}\label{ux438ux43dux442ux435ux440ux43fux440ux435ux442ux430ux446ux438ux44f}

\begin{itemize}
\tightlist
\item
  В физике: если \(s(t)\) --- это позиция, то \(s'(t)\) --- это
  скорость.
\item
  В экономике: если \(C(x)\) --- это затраты, то \(C'(x)\) --- это
  предельные издержки.
\item
  В биологии: если \(P(t)\) --- это численность населения, то \(P'(t)\)
  --- это темпы роста.
\end{itemize}

Производная делает «изменение» точным во многих контекстах.

\subsubsection{Упражнения}\label{ux443ux43fux440ux430ux436ux43dux435ux43dux438ux44f-5}

\begin{enumerate}
\def\labelenumi{\arabic{enumi}.}
\tightlist
\item
  Вычислите \(f'(x)\) для \(f(x) = 3x^2 - 2x + 1\).2. Найдите наклон
  касательной к \(f(x) = x^3\) в точке \(x = 2\).
\item
  Если \(s(t) = t^2 + 2t\) представляет расстояние в метрах, какова
  скорость в \(t = 5\)?
\item
  Используйте определение предела для вычисления производной
  \(f(x) = \frac{1}{x}\).
\item
  Нарисуйте график \(y = x^2\) и проведите касательную линию в
  \(x = 1\).
\end{enumerate}

\subsection{2.2 Правила
дифференциации}\label{ux43fux440ux430ux432ux438ux43bux430-ux434ux438ux444ux444ux435ux440ux435ux43dux446ux438ux430ux446ux438ux438}

После того как производная определена, нам нужны эффективные способы ее
вычисления. Правила дифференциации --- это ярлыки, которые избавляют нас
от многократного применения определения предела.

\subsubsection{Постоянное
правило}\label{ux43fux43eux441ux442ux43eux44fux43dux43dux43eux435-ux43fux440ux430ux432ux438ux43bux43e}

Если \(f(x) = c\) где \(c\) --- константа, то

\[
f'(x) = 0.
\]

\subsubsection{Правило
силы}\label{ux43fux440ux430ux432ux438ux43bux43e-ux441ux438ux43bux44b}

Для \(f(x) = x^n\), где \(n\) --- действительное число,

\[
\frac{d}{dx} \big( x^n \big) = n x^{n-1}.
\]

Примеры:

\begin{itemize}
\tightlist
\item
  \(\frac{d}{dx}(x^2) = 2x\).
\item
  \(\frac{d}{dx}(x^5) = 5x^4\).
\item
  \(\frac{d}{dx}(\sqrt{x}) = \frac{1}{2\sqrt{x}}\).
\end{itemize}

\subsubsection{Правило постоянного множественного
числа}\label{ux43fux440ux430ux432ux438ux43bux43e-ux43fux43eux441ux442ux43eux44fux43dux43dux43eux433ux43e-ux43cux43dux43eux436ux435ux441ux442ux432ux435ux43dux43dux43eux433ux43e-ux447ux438ux441ux43bux430}

Если \(f(x) = c \cdot g(x)\), то

\[
f'(x) = c \cdot g'(x).
\]

\subsubsection{Правила суммы и
разности}\label{ux43fux440ux430ux432ux438ux43bux430-ux441ux443ux43cux43cux44b-ux438-ux440ux430ux437ux43dux43eux441ux442ux438}

\begin{itemize}
\tightlist
\item
  \((f + g)' = f' + g'\).
\item
  \((f - g)' = f' - g'\).
\end{itemize}

\subsubsection{Правило
продукта}\label{ux43fux440ux430ux432ux438ux43bux43e-ux43fux440ux43eux434ux443ux43aux442ux430}

Для \(f(x)\) и \(g(x)\):

\[
(fg)' = f'g + fg'.
\]

Пример: если \(f(x) = x^2\), \(g(x) = \sin x\):

\[
(fg)' = (2x)(\sin x) + (x^2)(\cos x).
\]

\subsubsection{Правило
частного}\label{ux43fux440ux430ux432ux438ux43bux43e-ux447ux430ux441ux442ux43dux43eux433ux43e}

Для \(f(x)\) и \(g(x)\):

\[
\left(\frac{f}{g}\right)' = \frac{f'g - fg'}{g^2}, \quad g(x) \neq 0.
\]

Пример: если \(f(x) = x^2\), \(g(x) = x+1\):

\[
\left(\frac{x^2}{x+1}\right)' = \frac{(2x)(x+1) - (x^2)(1)}{(x+1)^2}.
\]

\subsubsection{Производные общих
функций}\label{ux43fux440ux43eux438ux437ux432ux43eux434ux43dux44bux435-ux43eux431ux449ux438ux445-ux444ux443ux43dux43aux446ux438ux439}

\begin{itemize}
\tightlist
\item
  \(\frac{d}{dx}(\sin x) = \cos x\).
\item
  \(\frac{d}{dx}(\cos x) = -\sin x\).
\item
  \(\frac{d}{dx}(e^x) = e^x\).
\item
  \(\frac{d}{dx}(\ln x) = \frac{1}{x}, \quad x > 0\).
\end{itemize}

\subsubsection{Упражнения}\label{ux443ux43fux440ux430ux436ux43dux435ux43dux438ux44f-6}

\begin{enumerate}
\def\labelenumi{\arabic{enumi}.}
\tightlist
\item
  Дифференцируйте \(f(x) = 7x^3 - 4x + 9\).
\item
  Используйте правило произведения, чтобы найти производную от
  \(f(x) = x^2 e^x\).
\item
  Примените правило частного к \(f(x) = \frac{\sin x}{x}\).
\item
  Вычислите \(\frac{d}{dx}(\ln(x^2))\), используя цепочку правил.
\item
  Докажите, что производная от \(f(x) = \frac{1}{x}\) равна
  \(-\frac{1}{x^2}\).
\end{enumerate}

\subsection{2.3 Правило
цепочки}\label{ux43fux440ux430ux432ux438ux43bux43e-ux446ux435ux43fux43eux447ux43aux438}

Часто функции создаются путем объединения более простых функций. Для
дифференциации таких сложных функций воспользуемся правилом цепочки.

\subsubsection{Правило}\label{ux43fux440ux430ux432ux438ux43bux43e}

Если \(y = f(g(x))\), то

\[
\frac{dy}{dx} = f'(g(x)) \cdot g'(x).
\]

Другими словами: дифференцируйте внешнюю функцию, оставьте внутреннюю
неизменной, затем умножьте на производную внутренней.

\subsubsection{Примеры}\label{ux43fux440ux438ux43cux435ux440ux44b-4}

\begin{enumerate}
\def\labelenumi{\arabic{enumi}.}
\item
  Квадрат линейной функции

  \[
  y = (3x+2)^2
  \]

  Внешняя функция: \(f(u) = u^2\), внутренняя функция: \(g(x) = 3x+2\).

  \[
  y' = 2(3x+2) \cdot 3 = 6(3x+2).
  \]
\item
  Экспонента с квадратичным внутри

  \[
  y = e^{x^2}
  \]

  Внешняя функция: \(f(u) = e^u\), внутренняя функция: \(g(x) = x^2\).

  \[y' = e^{x^2} \cdot 2x = 2x e^{x^2}.
  \]
\item
  Logarithm with root inside

  \[
  у = \ln(\sqrt{x})
  \]

  Outer: \(f(u) = \ln u\), inner: \(g(x) = \sqrt{x}\).

  \[
  y' = \frac{1}{\sqrt{x}} \cdot \frac{1}{2\sqrt{x}} = \frac{1}{2x}.
  \]
\end{enumerate}

\subsubsection{Generalized Chain Rule}\label{generalized-chain-rule}

For multiple nested functions \(y = f(g(h(x)))\):

\[
\frac{dy}{dx} = f'(g(h(x))) \cdot g'(h(x)) \cdot h'(x).
\]

This extends naturally to deeper compositions.

\subsubsection{Why the Chain Rule
Matters}\label{why-the-chain-rule-matters}

\begin{itemize}
\tightlist
\item
  It handles nearly all real-world models where one quantity depends on
  another indirectly.
\item
  It connects calculus with physics (e.g., velocity depending on time
  through position).
\item
  It is essential in implicit differentiation and advanced topics.
\end{itemize}

\subsubsection{Exercises}\label{exercises}

\begin{enumerate}
\def\labelenumi{\arabic{enumi}.}
\tightlist
\item
  Differentiate \(y = (5x^2 + 1)^3\).
\item
  Find \(\frac{d}{dx}(\sin(3x))\).
\item
  Compute \(\frac{d}{dx}(\ln(1+x^2))\).
\item
  Differentiate \(y = \cos^2(x)\).
\item
  Apply the generalized chain rule to \(y = e^{\sin(x^2)}\).
\end{enumerate}

\subsection{2.4 Implicit
Differentiation}\label{implicit-differentiation}

Not all functions are given in the form \(y = f(x)\). Sometimes \(x\)
and \(y\) are related by an equation, and solving explicitly for \(y\)
is difficult or impossible. In such cases, we use implicit
differentiation.

\subsubsection{The Idea}\label{the-idea}

If an equation involves both \(x\) and \(y\), we can differentiate both
sides with respect to \(x\), treating \(y\) as a function of \(x\). Each
time we differentiate a term involving \(y\), we multiply by
\(\frac{dy}{dx}\).

\subsubsection{Example 1: A Circle}\label{example-1-a-circle}

Equation:

\[
х^2 + у^2 = 25
\]

Differentiate with respect to \(x\):

\[
2x + 2y \frac{dy}{dx} = 0.
\]

Solve for \(\frac{dy}{dx}\):

\[
\frac{dy}{dx} = -\frac{x}{y}.
\]

This gives the slope of the tangent to the circle at any point.

\subsubsection{Example 2: A Product of
Variables}\label{example-2-a-product-of-variables}

Equation:

\[
ху = 1
\]

Differentiate:

\[
х \frac{dy}{dx} + y = 0.
\]

So,

\[
\frac{dy}{dx} = -\frac{y}{x}.
\]

\subsubsection{Example 3: Trigonometric
Relation}\label{example-3-trigonometric-relation}

Equation:

\[
\sin(xy) = х
\]

Differentiate:

\[
\cos(xy) \cdot \Big(y + x\frac{dy}{dx}\Big) = 1.
\]

Solve for \(\frac{dy}{dx}\):

\[
\frac{dy}{dx} = \frac{1 - y\cos(xy)}{x\cos(xy)}.
\]

\subsubsection{Чем полезно неявное
дифференцирование}\label{ux447ux435ux43c-ux43fux43eux43bux435ux437ux43dux43e-ux43dux435ux44fux432ux43dux43eux435-ux434ux438ux444ux444ux435ux440ux435ux43dux446ux438ux440ux43eux432ux430ux43dux438ux435}

\begin{itemize}
\tightlist
\item
  Многие важные кривые (круги, эллипсы, гиперболы) естественным образом
  определяются неявно.
\item
  Это позволяет нам дифференцировать уравнения без предварительного
  решения \(y\).
\item
  Это ключевой шаг в более сложных темах, таких как связанные скорости и
  дифференциальные уравнения.
\end{itemize}

\subsubsection{Упражнения}\label{ux443ux43fux440ux430ux436ux43dux435ux43dux438ux44f-7}

\begin{enumerate}
\def\labelenumi{\arabic{enumi}.}
\tightlist
\item
  Для кривой \(x^2 + xy + y^2 = 7\) найдите \(\frac{dy}{dx}\).
\item
  Неявно дифференцировать \(\cos(x) + \cos(y) = 1\).
\item
  Найдите наклон касательной к \(x^3 + y^3 = 9\) в точке \((1, 2)\).4.
  Учитывая \(x^2 + y^2 = 10\), вычислите \(\frac{dy}{dx}\), когда
  \((x, y) = (1, 3)\).
\item
  Продифференцируйте \(e^{xy} = x + y\), чтобы найти \(\frac{dy}{dx}\).
\end{enumerate}

\subsection{2.5 Производные высшего
порядка}\label{ux43fux440ux43eux438ux437ux432ux43eux434ux43dux44bux435-ux432ux44bux441ux448ux435ux433ux43e-ux43fux43eux440ux44fux434ux43aux430}

До сих пор мы изучали первую производную, которая измеряет скорость
изменения функции. Но сами деривативы также могут быть дифференцированы,
что приводит к появлению деривативов более высокого порядка.

\subsubsection{Определение}\label{ux43eux43fux440ux435ux434ux435ux43bux435ux43dux438ux435-3}

\begin{itemize}
\item
  Вторая производная \(f\) является производной производной:

  \[
  f''(x) = \frac{d}{dx}\left(f'(x)\right).
  \]
\item
  В более общем смысле производная \(n\) записывается как

  \[
  f^{(n)}(x) = \frac{d^n}{dx^n} f(x).
  \]
\end{itemize}

\subsubsection{Примеры}\label{ux43fux440ux438ux43cux435ux440ux44b-5}

\begin{enumerate}
\def\labelenumi{\arabic{enumi}.}
\item
  \(f(x) = x^3\)

  \begin{itemize}
  \tightlist
  \item
    Первая производная: \(f'(x) = 3x^2\).
  \item
    Вторая производная: \(f''(x) = 6x\).
  \item
    Третья производная: \(f^{(3)}(x) = 6\).
  \item
    Четвертая производная: \(f^{(4)}(x) = 0\).
  \end{itemize}
\item
  \(f(x) = \sin x\)

  \begin{itemize}
  \tightlist
  \item
    \(f'(x) = \cos x\).
  \item
    \(f''(x) = -\sin x\).
  \item
    \(f^{(3)}(x) = -\cos x\).
  \item
    \(f^{(4)}(x) = \sin x\). Производные повторяются в цикле длиной 4.
  \end{itemize}
\item
  \(f(x) = e^x\)

  \begin{itemize}
  \tightlist
  \item
    Каждая производная \(e^x\).
  \end{itemize}
\end{enumerate}

\subsubsection{Приложения}\label{ux43fux440ux438ux43bux43eux436ux435ux43dux438ux44f}

\begin{itemize}
\item
  Вогнутость: знак \(f''(x)\) указывает, является ли график \(f\)
  вогнутым вверх (\(f'' > 0\)) или вогнутым вниз (\(f'' < 0\)).
\item
  Точки перегиба: точки, где \(f''(x) = 0\) и вогнутость меняются.
\item
  Движение: в физике, если \(s(t)\) --- это позиция:

  \begin{itemize}
  \tightlist
  \item
    \(s'(t)\) = скорость,
  \item
    \(s''(t)\) = ускорение,
  \item
    \(s^{(3)}(t)\) = рывок (скорость изменения ускорения).
  \end{itemize}
\item
  Приближения: производные более высокого порядка появляются в рядах
  Тейлора и используются для аппроксимации функций.
\end{itemize}

\subsubsection{Упражнения}\label{ux443ux43fux440ux430ux436ux43dux435ux43dux438ux44f-8}

\begin{enumerate}
\def\labelenumi{\arabic{enumi}.}
\tightlist
\item
  Вычислите первые четыре производные \(f(x) = \cos x\).
\item
  Найдите \(f''(x)\) для \(f(x) = x^4 - 2x^2 + 3\).
\item
  Для \(f(x) = e^{2x}\) покажите, что \(f^{(n)}(x) = 2^n e^{2x}\).
\item
  Определите интервалы, в которых \(f(x) = x^3 - 3x\) вогнут вверх и
  вогнут вниз.
\item
  Если \(s(t) = t^3 - 6t^2 + 9t\), найдите скорость и ускорение в
  \(t = 2\).
\end{enumerate}

\section{Глава 3. Применение
деривативов}\label{ux433ux43bux430ux432ux430-3.-ux43fux440ux438ux43cux435ux43dux435ux43dux438ux435-ux434ux435ux440ux438ux432ux430ux442ux438ux432ux43eux432}

\subsection{3.1 Касательные и
нормали}\label{ux43aux430ux441ux430ux442ux435ux43bux44cux43dux44bux435-ux438-ux43dux43eux440ux43cux430ux43bux438}

Одним из первых применений производных является нахождение уравнений
касательных и нормалей к кривой. Эти линии отражают локальную геометрию
функции в данной точке.

\subsubsection{Касательная
линия}\label{ux43aux430ux441ux430ux442ux435ux43bux44cux43dux430ux44f-ux43bux438ux43dux438ux44f}

Касательная линия к кривой \(y = f(x)\) в точке \((a, f(a))\) --- это
линия, которая просто «касается» графика в этой точке и имеет тот же
наклон, что и кривая.

Наклон касательной определяется производной:

\[
m_{\text{tangent}} = f'(a).
\]

Таким образом, уравнение касательной в точке \((a, f(a))\) имеет вид

\[
y - f(a) = f'(a)(x - a).
\]

\subsubsection{Обычная
линия}\label{ux43eux431ux44bux447ux43dux430ux44f-ux43bux438ux43dux438ux44f}

Нормальная линия перпендикулярна касательной в той же точке. Его наклон
является отрицательной величиной, обратной наклону касательной:

\[m_{\text{normal}} = -\frac{1}{f'(a)}.
\]

So the equation of the normal line is

\[
y - f(a) = -\frac{1}{f'(a)} (x - a), \quad f'(a) \neq 0.
\]

\subsubsection{Examples}\label{examples}

\begin{enumerate}
\def\labelenumi{\arabic{enumi}.}
\item
  \(f(x) = x^2\) at \(x = 1\).

  \begin{itemize}
  \tightlist
  \item
    \(f(1) = 1\), \(f'(x) = 2x\), so \(f'(1) = 2\).
  \item
    Tangent: \(y - 1 = 2(x - 1)\), or \(y = 2x - 1\).
  \item
    Normal: slope = \(-\tfrac{1}{2}\), so equation is
    \(y - 1 = -\tfrac{1}{2}(x - 1)\).
  \end{itemize}
\item
  \(f(x) = \sin x\) at \(x = \tfrac{\pi}{4}\).

  \begin{itemize}
  \tightlist
  \item
    \(f(\tfrac{\pi}{4}) = \tfrac{\sqrt{2}}{2}\),
    \(f'(\tfrac{\pi}{4}) = \cos(\tfrac{\pi}{4}) = \tfrac{\sqrt{2}}{2}\).
  \item
    Tangent:
    \(y - \tfrac{\sqrt{2}}{2} = \tfrac{\sqrt{2}}{2}(x - \tfrac{\pi}{4})\).
  \end{itemize}
\end{enumerate}

\subsubsection{Why Tangents and Normals
Matter}\label{why-tangents-and-normals-matter}

\begin{itemize}
\tightlist
\item
  Tangents approximate the curve locally (linear approximation).
\item
  Normals are useful in geometry, optics (reflection/refraction), and
  mechanics (force directions).
\item
  Both play a role in optimization and curvature studies.
\end{itemize}

\subsubsection{Exercises}\label{exercises-1}

\begin{enumerate}
\def\labelenumi{\arabic{enumi}.}
\tightlist
\item
  Find the tangent and normal lines to \(y = x^3\) at \(x = 2\).
\item
  Determine the tangent and normal lines to \(y = e^x\) at \(x = 0\).
\item
  For \(y = \ln x\), compute the tangent line at \(x = 1\).
\item
  A circle is given by \(x^2 + y^2 = 9\). Use implicit differentiation
  to find the slope of the tangent at \((0,3)\).
\item
  Sketch the graph of \(y = \sqrt{x}\) and draw the tangent and normal
  lines at \(x = 4\).
\end{enumerate}

\subsection{3.2 Related Rates}\label{related-rates}

In many real-world problems, two or more quantities change with respect
to time, and their rates of change are connected. Related rates problems
use derivatives to describe these relationships.

\subsubsection{General Approach}\label{general-approach}

\begin{enumerate}
\def\labelenumi{\arabic{enumi}.}
\tightlist
\item
  Identify the variables that depend on time \(t\).
\item
  Write an equation relating the variables.
\item
  Differentiate both sides with respect to \(t\), applying the chain
  rule.
\item
  Substitute the known values at the given instant.
\item
  Solve for the unknown rate.
\end{enumerate}

\subsubsection{Example 1: Expanding
Circle}\label{example-1-expanding-circle}

A circle has radius \(r\), which increases at the rate of
\(\frac{dr}{dt} = 2 \,\text{cm/s}\). Find the rate at which the area
\(A = \pi r^2\) increases when \(r = 5\).

Differentiate:

\[
\frac{dA}{dt} = 2\pi r \frac{dr}{dt}.
\]

Substitute:

\[
\frac{dA}{dt} = 2\pi (5)(2) = 20\pi \,\text{см}^2/\text{s}.
\]

\subsubsection{Example 2: Sliding
Ladder}\label{example-2-sliding-ladder}

A 10 ft ladder leans against a wall. The bottom slides away at
\(\frac{dx}{dt} = 1 \,\text{ft/s}\). How fast is the top sliding down
when the bottom is 6 ft from the wall?

Equation: \(x^2 + y^2 = 100\), where \(y\) is the height.

Differentiate:

\[
2x \frac{dx}{dt} + 2y \frac{dy}{dt} = 0.
\]

At \(x = 6\), \(y = 8\). Substitute:

\[
2(6)(1) + 2(8)\frac{dy}{dt} = 0 \quad \Rightarrow \quad \frac{dy}{dt} = -\tfrac{6}{8} = -\tfrac{3}{4}.
\]Таким образом, верхняя часть сдвигается вниз в точке
\(0.75 \,\text{ft/s}\).

\subsubsection{Пример 3: Вода в
конусе}\label{ux43fux440ux438ux43cux435ux440-3-ux432ux43eux434ux430-ux432-ux43aux43eux43dux443ux441ux435}

Вода налита в конус высотой 12 см и радиусом 6 см. Когда вода достигает
глубины 4 см, уровень воды поднимается на \(2 \,\text{cm/s}\). С какой
скоростью увеличивается объем?

Уравнение: \(V = \tfrac{1}{3}\pi r^2 h\). Используя сходство,
\(r = \tfrac{h}{2}\). Замена:

\[
V = \tfrac{1}{12}\pi h^3.
\]

Дифференцировать:

\[
\frac{dV}{dt} = \tfrac{1}{4}\pi h^2 \frac{dh}{dt}.
\]

В \(h = 4\), \(\frac{dh}{dt} = 2\):

\[
\frac{dV}{dt} = \tfrac{1}{4}\pi (16)(2) = 8\pi \,\text{cm}^3/\text{s}.
\]

\subsubsection{Почему соответствующие тарифы имеют
значение}\label{ux43fux43eux447ux435ux43cux443-ux441ux43eux43eux442ux432ux435ux442ux441ux442ux432ux443ux44eux449ux438ux435-ux442ux430ux440ux438ux444ux44b-ux438ux43cux435ux44eux442-ux437ux43dux430ux447ux435ux43dux438ux435}

\begin{itemize}
\tightlist
\item
  Они описывают движение и изменения в физике, технике и биологии.
\item
  Они связывают геометрию с математическим анализом посредством
  процессов, зависящих от времени.
\item
  Они учат нас математически моделировать динамические системы.
\end{itemize}

\subsubsection{Упражнения}\label{ux443ux43fux440ux430ux436ux43dux435ux43dux438ux44f-9}

\begin{enumerate}
\def\labelenumi{\arabic{enumi}.}
\tightlist
\item
  Воздушный шар надувается так, что его радиус увеличивается на
  \(0.5 \,\text{cm/s}\). Найдите, с какой скоростью увеличится его
  объем, если радиус равен 10 см.
\item
  Автомобиль едет на север со скоростью 40 км/ч, а другой на восток со
  скоростью 30 км/ч. С какой скоростью увеличится расстояние между ними
  через 2 часа?
\item
  Прожектор с высоты 20 м от стены освещает человека ростом 2 м,
  уходящего со скоростью 1,5 м/с. Как быстро изменится длина его тени на
  стене, когда он окажется на расстоянии 5 м от света?
\item
  Длина стороны куба растет со скоростью 2 см/с. С какой скоростью
  увеличится площадь поверхности, если сторона равна 3 см?
\item
  Песок насыпают в кучу, образуя конус радиусом, всегда равным высоте.
  Если высота увеличивается со скоростью 5 см/с, с какой скоростью
  увеличится объем при высоте 10 см?
\end{enumerate}

\subsection{3.3 Проблемы
оптимизации}\label{ux43fux440ux43eux431ux43bux435ux43cux44b-ux43eux43fux442ux438ux43cux438ux437ux430ux446ux438ux438}

В задачах оптимизации производные используются для нахождения
максимальных или минимальных значений функции, часто при определенных
ограничениях. Эти задачи моделируют ситуации, когда мы хотим
максимизировать эффективность, прибыль или площадь или минимизировать
затраты, расстояние или время.

\subsubsection{Общие
шаги}\label{ux43eux431ux449ux438ux435-ux448ux430ux433ux438}

\begin{enumerate}
\def\labelenumi{\arabic{enumi}.}
\tightlist
\item
  Поймите проблему: определите количество, которое необходимо
  оптимизировать.
\item
  Модель с функцией: запишите целевую функцию через одну переменную.
\item
  Примените ограничения. Используйте заданные условия для уменьшения
  переменных.
\item
  Дифференцировать: вычислить производную целевой функции.
\item
  Найдите критические точки: решите \(f'(x) = 0\) или где \(f'(x)\) не
  определен.
\item
  Проверка максимумов/минимумов. Используйте тест второй производной или
  проверьте конечные точки.
\item
  Интерпретируйте результат: Изложите ответ в исходном контексте.
\end{enumerate}

\subsubsection{Пример 1: максимальная площадь
прямоугольника}\label{ux43fux440ux438ux43cux435ux440-1-ux43cux430ux43aux441ux438ux43cux430ux43bux44cux43dux430ux44f-ux43fux43bux43eux449ux430ux434ux44c-ux43fux440ux44fux43cux43eux443ux433ux43eux43bux44cux43dux438ux43aux430}

Периметр прямоугольника равен 40. Какие размеры максимизируют его
площадь?

\begin{itemize}
\tightlist
\item
  Пусть длина \(x\), ширина \(y\). Ограничение:
  \(2x + 2y = 40 \Rightarrow y = 20 - x\).
\item
  Область: \(A = xy = x(20 - x) = 20x - x^2\).- Производное:
  \(A'(x) = 20 - 2x\). Установите равным 0: \(x = 10\).
\item
  Тогда \(y = 10\).
\item
  Максимальная площадь: \(100\). Прямоугольник представляет собой
  квадрат.
\end{itemize}

\subsubsection{Пример 2: Минимизация
расстояния}\label{ux43fux440ux438ux43cux435ux440-2-ux43cux438ux43dux438ux43cux438ux437ux430ux446ux438ux44f-ux440ux430ux441ux441ux442ux43eux44fux43dux438ux44f}

Найдите точку на параболе \(y = x^2\), ближайшую к \((0,3)\).

\begin{itemize}
\tightlist
\item
  Расстояние в квадрате: \(D(x) = (x-0)^2 + (x^2 - 3)^2\).
\item
  Разверните:
  \(D(x) = x^2 + (x^2 - 3)^2 = x^2 + x^4 - 6x^2 + 9 = x^4 - 5x^2 + 9\).
\item
  Производное: \(D'(x) = 4x^3 - 10x\). Решите: \(x(4x^2 - 10) = 0\).
\item
  Решения: \(x = 0\), \(x = \pm \sqrt{2.5}\).
\item
  Проверка дает минимальное расстояние по адресу \(x = \pm \sqrt{2.5}\).
\end{itemize}

\subsubsection{Пример 3: Коробка максимального
объема}\label{ux43fux440ux438ux43cux435ux440-3-ux43aux43eux440ux43eux431ux43aux430-ux43cux430ux43aux441ux438ux43cux430ux43bux44cux43dux43eux433ux43e-ux43eux431ux44aux435ux43cux430}

Коробку без верха сделать из квадратного куска картона со стороной 20
см, вырезав из углов одинаковые квадраты и загнув боковые стороны.
Найдите размер разреза, который максимизирует объем.

\begin{itemize}
\tightlist
\item
  Пусть размер выреза = \(x\). Затем размеры:
  \((20 - 2x) \times (20 - 2x) \times x\).
\item
  Том: \(V(x) = x(20 - 2x)^2\).
\item
  Производное: \(V'(x) = (20 - 2x)(20 - 6x)\).
\item
  Критические точки: \(x = 10\) (даёт нулевой объём) или
  \(x = \tfrac{20}{6} \approx 3.33\).
\item
  В \(x \approx 3.33\) громкость максимальная.
\end{itemize}

\subsubsection{Почему оптимизация
важна}\label{ux43fux43eux447ux435ux43cux443-ux43eux43fux442ux438ux43cux438ux437ux430ux446ux438ux44f-ux432ux430ux436ux43dux430}

\begin{itemize}
\tightlist
\item
  Инженеры используют его для проектирования эффективных конструкций.
\item
  Предприятия используют его для максимизации прибыли или минимизации
  затрат.
\item
  Ученые используют его для моделирования природных систем, стремящихся
  к равновесию.
\end{itemize}

\subsubsection{Упражнения}\label{ux443ux43fux440ux430ux436ux43dux435ux43dux438ux44f-10}

\begin{enumerate}
\def\labelenumi{\arabic{enumi}.}
\tightlist
\item
  Фермер имеет 100-метровое ограждение для ограждения прямоугольного
  поля вдоль реки (поэтому ограждение необходимо только с трех сторон).
  Найдите размеры, увеличивающие площадь.
\item
  Найдите два положительных числа, сумма которых равна 20 и произведение
  как можно большего размера.
\item
  Изготовить цилиндр длиной 100 см\(^2\) материала. Найдите размеры
  максимального объема.
\item
  Проволоку длиной 10 м разрезают на две части, одну сгибают в квадрат,
  другую в круг. Как его следует обрезать, чтобы максимизировать общую
  площадь ограждения?
\item
  Необходимо построить закрытый ящик с квадратным основанием объемом 32
  м\(^3\). Найдите размеры, минимизирующие площадь поверхности.
\end{enumerate}

\subsection{3.4 Вогнутость и точки
перегиба}\label{ux432ux43eux433ux43dux443ux442ux43eux441ux442ux44c-ux438-ux442ux43eux447ux43aux438-ux43fux435ux440ux435ux433ux438ux431ux430}

Производные говорят нам не только о наклонах, но и о форме графика.
Вторая производная особенно полезна для понимания вогнутости и
определения точек перегиба.

\subsubsection{Вогнутость}\label{ux432ux43eux433ux43dux443ux442ux43eux441ux442ux44c}

\begin{itemize}
\item
  Функция \(f(x)\) вогнута вверх на интервале, если \(f''(x) > 0\).
  График изгибается вверх, как чашка.
\item
  Функция \(f(x)\) вогнута вниз на интервале, если \(f''(x) < 0\).
  График наклоняется вниз, словно нахмурившись.
\end{itemize}

Вогнутость описывает, как изменяется наклон функции: если наклон
увеличивается, график вогнут вверх; если наклоны уменьшаются, график
вогнут вниз.

\subsubsection{Точки
перегиба}\label{ux442ux43eux447ux43aux438-ux43fux435ux440ux435ux433ux438ux431ux430}

Точка перегиба --- это точка на графике, где изменяется вогнутость.-
Если \(f''(x) = 0\) или \(f''(x)\) не определен, точка является
кандидатом на роль точки перегиба. - Для подтверждения вогнутость должна
изменить знак по обе стороны от точки.

\subsubsection{Примеры}\label{ux43fux440ux438ux43cux435ux440ux44b-6}

\begin{enumerate}
\def\labelenumi{\arabic{enumi}.}
\item
  \(f(x) = x^3\)

  \begin{itemize}
  \tightlist
  \item
    \(f''(x) = 6x\).
  \item
    В \(x = 0\), \(f''(0) = 0\).
  \item
    Для \(x < 0\), \(f''(x) < 0\) → вогнутым вниз.
  \item
    Для \(x > 0\), \(f''(x) > 0\) → вогнутый вверх.
  \item
    Таким образом, \((0,0)\) является точкой перегиба.
  \end{itemize}
\item
  \(f(x) = x^4\)

  \begin{itemize}
  \tightlist
  \item
    \(f''(x) = 12x^2\).
  \item
    У \(x = 0\), \(f''(0) = 0\), но вогнутость не меняет знак (всегда ≥
    0).
  \item
    Нет точки перегиба.
  \end{itemize}
\end{enumerate}

\subsubsection{Рисование вогнутостей и
кривых}\label{ux440ux438ux441ux43eux432ux430ux43dux438ux435-ux432ux43eux433ux43dux443ux442ux43eux441ux442ux435ux439-ux438-ux43aux440ux438ux432ux44bux445}

\begin{itemize}
\tightlist
\item
  Если \(f'(x) = 0\) и \(f''(x) > 0\), то \(f\) имеет локальный минимум.
\item
  Если \(f'(x) = 0\) и \(f''(x) < 0\), то \(f\) имеет локальный
  максимум.
\item
  Это известно как тест второй производной.
\end{itemize}

\subsubsection{Почему это
важно}\label{ux43fux43eux447ux435ux43cux443-ux44dux442ux43e-ux432ux430ux436ux43dux43e-1}

Вогнутость и точки перегиба помогают нам понять «форму» графиков: где
они изгибаются, сглаживаются или поворачиваются. Эти идеи занимают
центральное место в построении кривых, физике (ускорение) и экономике
(убывающая отдача).

\subsubsection{Упражнения}\label{ux443ux43fux440ux430ux436ux43dux435ux43dux438ux44f-11}

\begin{enumerate}
\def\labelenumi{\arabic{enumi}.}
\tightlist
\item
  Определите интервалы вогнутости для \(f(x) = x^3 - 3x\). Найдите ее
  точки перегиба.
\item
  Для \(f(x) = \ln(x)\) определите вогнутость и возможные точки
  перегиба.
\item
  Примените второй тест производной к \(f(x) = x^2 e^{-x}\), чтобы
  классифицировать критические точки.
\item
  Нарисуйте \(f(x) = \sin x\), отмечая интервалы вогнутостей и точки
  перегиба.
\item
  Объясните, почему \(f(x) = e^x\) не имеет точек перегиба.
\end{enumerate}

\subsection{3.5 Создание эскиза
кривой}\label{ux441ux43eux437ux434ux430ux43dux438ux435-ux44dux441ux43aux438ux437ux430-ux43aux440ux438ux432ux43eux439}

Построение кривой --- это процесс построения графика функции с
использованием информации о ее производных. Вместо того, чтобы строить
множество точек, мы анализируем ключевые особенности: точки пересечения,
асимптоты, интервалы увеличения/уменьшения и вогнутость.

\subsubsection{Шаги по созданию эскиза
кривой}\label{ux448ux430ux433ux438-ux43fux43e-ux441ux43eux437ux434ux430ux43dux438ux44e-ux44dux441ux43aux438ux437ux430-ux43aux440ux438ux432ux43eux439}

\begin{enumerate}
\def\labelenumi{\arabic{enumi}.}
\item
  Домен: укажите, где определена функция.
\item
  Перехваты: найдите, где график пересекает оси.
\item
  Асимптоты:

  \begin{itemize}
  \tightlist
  \item
    Вертикальные асимптоты возникают там, где функция не определена и
    стремится к бесконечности.
  \item
    Горизонтальные или наклонные асимптоты описывают конечное поведение
    как \(x \to \pm\infty\).
  \end{itemize}
\item
  Первая производная \(f'(x)\):

  \begin{itemize}
  \tightlist
  \item
    Позитив → функция возрастает.
  \item
    Отрицательный → функция уменьшается.
  \item
    Нули \(f'(x)\) → критические точки (возможные максимумы/минимумы).
  \end{itemize}
\item
  Вторая производная \(f''(x)\):

  \begin{itemize}
  \tightlist
  \item
    Положительный → вогнутый вверх.
  \item
    Отрицательный → вогнутый вниз.
  \item
    Нули или неопределенное значение → возможные точки перегиба.
  \end{itemize}
\item
  Объедините информацию. Используйте все результаты, чтобы построить
  четкий и точный график.
\end{enumerate}

\subsubsection{\texorpdfstring{Пример 1:
\(f(x) = x^3 - 3x\)}{Пример 1: f(x) = x\^{}3 - 3x}}\label{ux43fux440ux438ux43cux435ux440-1-fx-x3---3x}

\begin{itemize}
\item
  Домен: все действительные числа.
\item
  Перехваты: по адресу \((0,0)\).
\item
  Производное: \(f'(x) = 3x^2 - 3 = 3(x-1)(x+1)\).

  \begin{itemize}
  \item
    Увеличение: \((-\infty, -1) \cup (1, \infty)\).
  \item
    Уменьшение: \((-1, 1)\).- Вторая производная: \(f''(x) = 6x\).
  \item
    Вогнутый вниз для \(x < 0\), вогнутый вверх для \(x > 0\).
  \item
    Точка перегиба в \((0,0)\).
  \end{itemize}
\item
  Форма: S-образная кривая с локальным максимумом в \((-1, 2)\) и
  локальным минимумом в \((1, -2)\).
\end{itemize}

\subsubsection{\texorpdfstring{Пример 2:
\(f(x) = \frac{1}{x}\)}{Пример 2: f(x) = \textbackslash frac\{1\}\{x\}}}\label{ux43fux440ux438ux43cux435ux440-2-fx-frac1x}

\begin{itemize}
\item
  Домен: \(x \neq 0\).
\item
  Вертикальная асимптота: \(x = 0\).
\item
  Горизонтальная асимптота: \(y = 0\).
\item
  Производная: \(f'(x) = -\frac{1}{x^2}\) (всегда отрицательная).
  Функция всегда убывает.
\item
  Вторая производная: \(f''(x) = \frac{2}{x^3}\).

  \begin{itemize}
  \tightlist
  \item
    Вогнутый вверх для \(x > 0\).
  \item
    Вогнутый вниз для \(x < 0\).
  \end{itemize}
\item
  График: гипербола с двумя ветвями.
\end{itemize}

\subsubsection{Почему полезно рисовать
кривые}\label{ux43fux43eux447ux435ux43cux443-ux43fux43eux43bux435ux437ux43dux43e-ux440ux438ux441ux43eux432ux430ux442ux44c-ux43aux440ux438ux432ux44bux435}

\begin{itemize}
\tightlist
\item
  Обеспечивает понимание общего поведения функций без исчерпывающих
  вычислений.
\item
  Необходим при сдаче экзаменов по математическому анализу и прикладных
  задачах.
\item
  Соединяет алгебраический анализ и геометрическое понимание.
\end{itemize}

\subsubsection{Упражнения}\label{ux443ux43fux440ux430ux436ux43dux435ux43dux438ux44f-12}

\begin{enumerate}
\def\labelenumi{\arabic{enumi}.}
\tightlist
\item
  Нарисуйте кривую \(f(x) = x^4 - 2x^2\). Определите максимумы, минимумы
  и точки перегиба.
\item
  Проанализируйте и зарисуйте \(f(x) = \ln(x)\). Покажите точки
  пересечения, асимптоты и вогнутость.
\item
  Для \(f(x) = e^{-x}\) опишите рост/распад, асимптоты и вогнутость.
\item
  Нарисуйте график \(f(x) = \tan x\) на интервале \((- \pi, \pi)\).
  Отметьте асимптоты.
\item
  Используйте тесты первой и второй производных для классификации
  критических точек \(f(x) = x^3 - 6x^2 + 9x\).
\end{enumerate}

\section{Часть II.
Интегралы}\label{ux447ux430ux441ux442ux44c-ii.-ux438ux43dux442ux435ux433ux440ux430ux43bux44b}

\section{Глава 4. Первообразные и определенные
интегралы}\label{ux433ux43bux430ux432ux430-4.-ux43fux435ux440ux432ux43eux43eux431ux440ux430ux437ux43dux44bux435-ux438-ux43eux43fux440ux435ux434ux435ux43bux435ux43dux43dux44bux435-ux438ux43dux442ux435ux433ux440ux430ux43bux44b}

\subsection{4.1 Неопределенные
интегралы}\label{ux43dux435ux43eux43fux440ux435ux434ux435ux43bux435ux43dux43dux44bux435-ux438ux43dux442ux435ux433ux440ux430ux43bux44b}

Неопределенный интеграл --- это обратный процесс дифференцирования. Если
производная мера изменяется, то интеграл восстанавливает исходную
функцию по скорости ее изменения.

\subsubsection{Определение}\label{ux43eux43fux440ux435ux434ux435ux43bux435ux43dux438ux435-4}

Если \(F'(x) = f(x)\), то

\[
\int f(x)\,dx = F(x) + C,
\]

где \(C\) --- константа интегрирования.

Каждый неопределенный интеграл представляет собой семейство функций,
отличающихся только константой, поскольку дифференцирование устраняет
константы.

\subsubsection{Основные
правила}\label{ux43eux441ux43dux43eux432ux43dux44bux435-ux43fux440ux430ux432ux438ux43bux430}

\begin{enumerate}
\def\labelenumi{\arabic{enumi}.}
\tightlist
\item
  Постоянное правило
\end{enumerate}

\[
\int c\,dx = cx + C.
\]

\begin{enumerate}
\def\labelenumi{\arabic{enumi}.}
\setcounter{enumi}{1}
\tightlist
\item
  Правило власти
\end{enumerate}

\[
\int x^n\,dx = \frac{x^{n+1}}{n+1} + C, \quad n \neq -1.
\]

\begin{enumerate}
\def\labelenumi{\arabic{enumi}.}
\setcounter{enumi}{2}
\tightlist
\item
  Правило сумм
\end{enumerate}

\[
\int \big(f(x) + g(x)\big)\,dx = \int f(x)\,dx + \int g(x)\,dx.
\]

\begin{enumerate}
\def\labelenumi{\arabic{enumi}.}
\setcounter{enumi}{3}
\tightlist
\item
  Постоянное множественное правило
\end{enumerate}

\[
\int c f(x)\,dx = c \int f(x)\,dx.
\]

\subsubsection{Общие
интегралы}\label{ux43eux431ux449ux438ux435-ux438ux43dux442ux435ux433ux440ux430ux43bux44b}

\begin{itemize}
\tightlist
\item
  \(\int e^x dx = e^x + C\)
\item
  \(\int \sin x dx = -\cos x + C\)
\item
  \(\int \cos x dx = \sin x + C\)
\item
  \(\int \frac{1}{x} dx = \ln|x| + C\)
\end{itemize}

\subsubsection{Примеры}\label{ux43fux440ux438ux43cux435ux440ux44b-7}

\begin{enumerate}
\def\labelenumi{\arabic{enumi}.}
\item
  \(\int (3x^2 - 4)\,dx = x^3 - 4x + C\).
\item
  \(\int \cos(2x)\,dx = \tfrac{1}{2}\sin(2x) + C\).
\item
  \(\int \frac{1}{x}\,dx = \ln|x| + C\).
\end{enumerate}

\subsubsection{Интерпретация}\label{ux438ux43dux442ux435ux440ux43fux440ux435ux442ux430ux446ux438ux44f-1}

\begin{itemize}
\tightlist
\item
  Неопределенные интегралы являются первообразными.
\item
  Они являются основой для определенных интегралов, которые измеряют
  накопленные величины, такие как площадь, расстояние и масса.- В
  прикладном контексте интеграция позволяет нам перейти от ставок
  обратно к итоговым показателям.
\end{itemize}

\subsubsection{Упражнения}\label{ux443ux43fux440ux430ux436ux43dux435ux43dux438ux44f-13}

\begin{enumerate}
\def\labelenumi{\arabic{enumi}.}
\tightlist
\item
  Найдите \(\int (5x^4 + 2x)\,dx\).
\item
  Вычислите \(\int (e^x + 3)\,dx\).
\item
  Найдите общее решение \(f'(x) = 6x\) с помощью интеграции.
\item
  Оцените \(\int \frac{2}{x}\,dx\).
\item
  Если скорость равна \(v(t) = 4t\), найдите функцию положения \(s(t)\).
\end{enumerate}

\subsection{4.2 Определенный интеграл как
площадь}\label{ux43eux43fux440ux435ux434ux435ux43bux435ux43dux43dux44bux439-ux438ux43dux442ux435ux433ux440ux430ux43b-ux43aux430ux43a-ux43fux43bux43eux449ux430ux434ux44c}

В то время как неопределенные интегралы представляют собой семейства
первообразных, определенный интеграл дает числовое значение: накопленную
площадь под кривой между двумя точками.

\subsubsection{Определение}\label{ux43eux43fux440ux435ux434ux435ux43bux435ux43dux438ux435-5}

Для функции \(f(x)\), определенной в \([a, b]\), определенный интеграл
равен

\[
\int_a^b f(x)\,dx = \lim_{n \to \infty} \sum_{i=1}^n f(x_i^-) \,\Delta x,
\]

где интервал \([a, b]\) разделен на \(n\) подинтервалов шириной
\(\Delta x\), а \(x_i^-\) --- это точка выборки в каждом подинтервале.

Это предел сумм Римана.

\subsubsection{Геометрическая
интерпретация}\label{ux433ux435ux43eux43cux435ux442ux440ux438ux447ux435ux441ux43aux430ux44f-ux438ux43dux442ux435ux440ux43fux440ux435ux442ux430ux446ux438ux44f}

\begin{itemize}
\tightlist
\item
  Если \(f(x) \geq 0\) на \([a, b]\), то \(\int_a^b f(x)\,dx\) равен
  площади под кривой \(y = f(x)\) от \(x=a\) до \(x=b\).
\item
  Если \(f(x)\) опускается ниже оси \(x\), интеграл вычисляет площадь со
  знаком: области ниже оси считаются отрицательными.
\end{itemize}

\subsubsection{Свойства определенного
интеграла}\label{ux441ux432ux43eux439ux441ux442ux432ux430-ux43eux43fux440ux435ux434ux435ux43bux435ux43dux43dux43eux433ux43e-ux438ux43dux442ux435ux433ux440ux430ux43bux430}

\begin{enumerate}
\def\labelenumi{\arabic{enumi}.}
\tightlist
\item
  Аддитивность по интервалам
\end{enumerate}

\[
\int_a^c f(x)\,dx = \int_a^b f(x)\,dx + \int_b^c f(x)\,dx.
\]

\begin{enumerate}
\def\labelenumi{\arabic{enumi}.}
\setcounter{enumi}{1}
\tightlist
\item
  Изменение ограничений
\end{enumerate}

\[
\int_a^b f(x)\,dx = -\int_b^a f(x)\,dx.
\]

\begin{enumerate}
\def\labelenumi{\arabic{enumi}.}
\setcounter{enumi}{2}
\tightlist
\item
  Интервал нулевой ширины
\end{enumerate}

\[
\int_a^a f(x)\,dx = 0.
\]

\begin{enumerate}
\def\labelenumi{\arabic{enumi}.}
\setcounter{enumi}{3}
\tightlist
\item
  Линейность
\end{enumerate}

\[
\int_a^b \big( cf(x) + g(x)\big)\,dx = c\int_a^b f(x)\,dx + \int_a^b g(x)\,dx.
\]

\subsubsection{Примеры}\label{ux43fux440ux438ux43cux435ux440ux44b-8}

\begin{enumerate}
\def\labelenumi{\arabic{enumi}.}
\item
  \(\int_0^2 x\,dx = \left[\tfrac{1}{2}x^2\right]_0^2 = 2.\) Это площадь
  прямоугольного треугольника под линией \(y=x\).
\item
  \(\int_{-1}^1 x^3\,dx = 0.\) Нечетная функция \(x^3\) имеет
  симметричные области, которые отменяются.
\item
  \(\int_0^\pi \sin x\,dx = 2.\) Это соответствует площади под одной
  дугой синусоиды.
\end{enumerate}

\subsubsection{Почему это
важно}\label{ux43fux43eux447ux435ux43cux443-ux44dux442ux43e-ux432ux430ux436ux43dux43e-2}

\begin{itemize}
\tightlist
\item
  Определенные интегралы измеряют накопленные величины: расстояние,
  массу, энергию, вероятность.
\item
  Они соединяют алгебраические вычисления с геометрической интуицией.
\item
  Следующим шагом является Основная теорема исчисления, которая
  связывает определенные интегралы с первообразными.
\end{itemize}

\subsubsection{Упражнения}\label{ux443ux43fux440ux430ux436ux43dux435ux43dux438ux44f-14}

\begin{enumerate}
\def\labelenumi{\arabic{enumi}.}
\tightlist
\item
  Вычислите \(\int_0^3 (2x+1)\,dx\).
\item
  Найдите область между \(y = x^2\) и осью \(x\) от \(x = 0\) до
  \(x = 2\).
\item
  Оцените \(\int_{-2}^2 (x^2 - 1)\,dx\).
\item
  Покажите, что \(\int_{-a}^a f(x)\,dx = 0\), если \(f(x)\) нечетный.
\item
  Аппроксимируйте \(\int_0^1 e^x\,dx\), используя сумму Римана с \(n=4\)
  подинтервалами и правыми конечными точками.
\end{enumerate}

\subsection{4.3 Основная теорема исчисленияФундаментальная теорема
исчисления (ФТК) объединяет две основные идеи исчисления:
дифференцирование и интегрирование. Это показывает, что поиск площадей и
поиск темпов изменений --- это две стороны одной
медали.}\label{ux43eux441ux43dux43eux432ux43dux430ux44f-ux442ux435ux43eux440ux435ux43cux430-ux438ux441ux447ux438ux441ux43bux435ux43dux438ux44fux444ux443ux43dux434ux430ux43cux435ux43dux442ux430ux43bux44cux43dux430ux44f-ux442ux435ux43eux440ux435ux43cux430-ux438ux441ux447ux438ux441ux43bux435ux43dux438ux44f-ux444ux442ux43a-ux43eux431ux44aux435ux434ux438ux43dux44fux435ux442-ux434ux432ux435-ux43eux441ux43dux43eux432ux43dux44bux435-ux438ux434ux435ux438-ux438ux441ux447ux438ux441ux43bux435ux43dux438ux44f-ux434ux438ux444ux444ux435ux440ux435ux43dux446ux438ux440ux43eux432ux430ux43dux438ux435-ux438-ux438ux43dux442ux435ux433ux440ux438ux440ux43eux432ux430ux43dux438ux435.-ux44dux442ux43e-ux43fux43eux43aux430ux437ux44bux432ux430ux435ux442-ux447ux442ux43e-ux43fux43eux438ux441ux43a-ux43fux43bux43eux449ux430ux434ux435ux439-ux438-ux43fux43eux438ux441ux43a-ux442ux435ux43cux43fux43eux432-ux438ux437ux43cux435ux43dux435ux43dux438ux439-ux44dux442ux43e-ux434ux432ux435-ux441ux442ux43eux440ux43eux43dux44b-ux43eux434ux43dux43eux439-ux43cux435ux434ux430ux43bux438.}

\subsubsection{Часть 1: Дифференцирование
интеграла}\label{ux447ux430ux441ux442ux44c-1-ux434ux438ux444ux444ux435ux440ux435ux43dux446ux438ux440ux43eux432ux430ux43dux438ux435-ux438ux43dux442ux435ux433ux440ux430ux43bux430}

Если \(f\) является непрерывным на \([a, b]\), определите

\[
F(x) = \int_a^x f(t)\,dt.
\]

Тогда \(F\) дифференцируем и

\[
F'(x) = f(x).
\]

Другими словами: производная функции накопленной площади сама является
исходной функцией.

\subsubsection{Часть 2: Вычисление определенных
интегралов}\label{ux447ux430ux441ux442ux44c-2-ux432ux44bux447ux438ux441ux43bux435ux43dux438ux435-ux43eux43fux440ux435ux434ux435ux43bux435ux43dux43dux44bux445-ux438ux43dux442ux435ux433ux440ux430ux43bux43eux432}

Если \(f\) является продолжением \([a, b]\) и \(F\) является любой
производной \(f\), то

\[
\int_a^b f(x)\,dx = F(b) - F(a).
\]

Это говорит нам о том, что мы можем вычислять определенные интегралы,
просто находя первообразную, а не вычисляя пределы сумм Римана.

\subsubsection{Примеры}\label{ux43fux440ux438ux43cux435ux440ux44b-9}

\begin{enumerate}
\def\labelenumi{\arabic{enumi}.}
\item
  \(\int_0^2 x^2\,dx\).

  \begin{itemize}
  \tightlist
  \item
    Производная: \(F(x) = \tfrac{1}{3}x^3\).
  \item
    Применить FTC: \(F(2) - F(0) = \tfrac{8}{3} - 0 = \tfrac{8}{3}.\)
  \end{itemize}
\item
  Если \(F(x) = \int_1^x \cos t \, dt\), то \(F'(x) = \cos x\).
\item
  \(\int_1^4 \frac{1}{x}\,dx\).

  \begin{itemize}
  \tightlist
  \item
    Производная: \(\ln|x|\).
  \item
    Применить FTC: \(\ln 4 - \ln 1 = \ln 4.\)
  \end{itemize}
\end{enumerate}

\subsubsection{Почему Федеральная торговая комиссия имеет
значение}\label{ux43fux43eux447ux435ux43cux443-ux444ux435ux434ux435ux440ux430ux43bux44cux43dux430ux44f-ux442ux43eux440ux433ux43eux432ux430ux44f-ux43aux43eux43cux438ux441ux441ux438ux44f-ux438ux43cux435ux435ux442-ux437ux43dux430ux447ux435ux43dux438ux435}

\begin{itemize}
\tightlist
\item
  Он превращает интеграцию из предельного процесса в практическое
  вычисление.
\item
  Это подтверждает, что дифференцирование и интегрирование являются
  обратными операциями.
\item
  Это центральная теорема, которая делает исчисление полезным в
  математике, естественных науках и технике.
\end{itemize}

\subsubsection{Упражнения}\label{ux443ux43fux440ux430ux436ux43dux435ux43dux438ux44f-15}

\begin{enumerate}
\def\labelenumi{\arabic{enumi}.}
\tightlist
\item
  Оцените \(\int_0^3 (2x+1)\,dx\) с помощью FTC.
\item
  Если \(F(x) = \int_0^x e^t\,dt\), найдите \(F'(x)\).
\item
  Вычислите \(\int_0^\pi \sin x \, dx\).
\item
  Докажите, что если \(f'(x) = g(x)\), то
  \(\int_a^b g(x)\,dx = f(b) - f(a)\).
\item
  С помощью FTC объясните, почему площадь под \(y = \cos x\) от \(0\) до
  \(\pi/2\) равна 1.
\end{enumerate}

\subsection{4.4 Свойства
интегралов}\label{ux441ux432ux43eux439ux441ux442ux432ux430-ux438ux43dux442ux435ux433ux440ux430ux43bux43eux432}

Определенный интеграл обладает несколькими важными свойствами, которые
делают его гибким и мощным в приложениях. Эти свойства следуют из
определения предела сумм и из основной теоремы исчисления.

\subsubsection{Линейность}\label{ux43bux438ux43dux435ux439ux43dux43eux441ux442ux44c}

Для функций \(f(x)\) и \(g(x)\) и констант \(c, d\):

\[
\int_a^b \big(c f(x) + d g(x)\big)\,dx = c \int_a^b f(x)\,dx + d \int_a^b g(x)\,dx.
\]

Это позволяет разбивать сложные интегралы на более простые части.

\subsubsection{Аддитивность по
интервалам}\label{ux430ux434ux434ux438ux442ux438ux432ux43dux43eux441ux442ux44c-ux43fux43e-ux438ux43dux442ux435ux440ux432ux430ux43bux430ux43c}

Если \(a < c < b\), то

\[
\int_a^b f(x)\,dx = \int_a^c f(x)\,dx + \int_c^b f(x)\,dx.
\]

Мы можем вычислять интегралы по частям.

\subsubsection{Отмена
лимитов}\label{ux43eux442ux43cux435ux43dux430-ux43bux438ux43cux438ux442ux43eux432}

\[
\int_a^b f(x)\,dx = -\int_b^a f(x)\,dx.
\]

Перестановка границ меняет знак интеграла.

\subsubsection{Свойство
сравнения}\label{ux441ux432ux43eux439ux441ux442ux432ux43e-ux441ux440ux430ux432ux43dux435ux43dux438ux44f}

Если \(f(x) \leq g(x)\) для всех \(x\) в \([a, b]\), то

\[
\int_a^b f(x)\,dx \leq \int_a^b g(x)\,dx.
\]Это позволяет нам сравнивать площади без прямых вычислений.

\subsubsection{Абсолютное неравенство
ценностей}\label{ux430ux431ux441ux43eux43bux44eux442ux43dux43eux435-ux43dux435ux440ux430ux432ux435ux43dux441ux442ux432ux43e-ux446ux435ux43dux43dux43eux441ux442ux435ux439}

\[
\left| \int_a^b f(x)\,dx \right| \leq \int_a^b |f(x)|\,dx.
\]

Это свойство важно при анализе и тестах на сходимость.

\subsubsection{Симметрия}\label{ux441ux438ux43cux43cux435ux442ux440ux438ux44f}

\begin{itemize}
\item
  Если \(f(x)\) четный (симметричен относительно оси \(y\)):

  \[
  \int_{-a}^a f(x)\,dx = 2\int_0^a f(x)\,dx.
  \]
\item
  Если \(f(x)\) нечетное число (симметрично относительно источника):

  \[
  \int_{-a}^a f(x)\,dx = 0.
  \]
\end{itemize}

\subsubsection{Примеры}\label{ux43fux440ux438ux43cux435ux440ux44b-10}

\begin{enumerate}
\def\labelenumi{\arabic{enumi}.}
\item
  \(\int_0^2 (3x^2 + 4)\,dx = \int_0^2 3x^2\,dx + \int_0^2 4\,dx = 8 + 8 = 16.\)
\item
  Поскольку \(f(x) = x^3\) нечетный, \(\int_{-1}^1 x^3\,dx = 0.\)
\item
  Поскольку \(f(x) = x^2\) четный,
  \(\int_{-2}^2 x^2\,dx = 2\int_0^2 x^2\,dx = 2\cdot \tfrac{8}{3} = \tfrac{16}{3}.\)
\end{enumerate}

\subsubsection{Почему эти свойства
важны}\label{ux43fux43eux447ux435ux43cux443-ux44dux442ux438-ux441ux432ux43eux439ux441ux442ux432ux430-ux432ux430ux436ux43dux44b}

\begin{itemize}
\tightlist
\item
  Они упрощают расчеты.
\item
  Выявляют геометрические и симметричные особенности функций.
\item
  Они предоставляют теоретические инструменты для более углубленного
  анализа.
\end{itemize}

\subsubsection{Упражнения}\label{ux443ux43fux440ux430ux436ux43dux435ux43dux438ux44f-16}

\begin{enumerate}
\def\labelenumi{\arabic{enumi}.}
\tightlist
\item
  Используйте симметрию для оценки \(\int_{-5}^5 (x^4 - x^3)\,dx\).
\item
  Покажите, что
  \(\int_1^4 (2x+3)\,dx = \int_1^2 (2x+3)\,dx + \int_2^4 (2x+3)\,dx\).
\item
  Оцените \(\int_0^\pi \sin(x)\,dx\) и сравните с
  \(\int_{-\pi}^\pi \sin(x)\,dx\).
\item
  Докажите, что если \(f(x) \geq 0\) на \([a, b]\), то
  \(\int_a^b f(x)\,dx \geq 0\).
\item
  Вычислите \(\int_{-3}^3 (x^2 + 1)\,dx\), используя четные/нечетные
  свойства.
\end{enumerate}

\section{Глава 5. Техники
интеграции}\label{ux433ux43bux430ux432ux430-5.-ux442ux435ux445ux43dux438ux43aux438-ux438ux43dux442ux435ux433ux440ux430ux446ux438ux438}

\subsection{5.1 Замена}\label{ux437ux430ux43cux435ux43dux430}

Одним из наиболее полезных методов интегрирования является метод замены,
также называемый -u-замещение-. Это процесс, обратный цепному правилу
для деривативов.

\subsubsection{Идея}\label{ux438ux434ux435ux44f}

Если интеграл содержит составную функцию, мы можем упростить его,
заменяя переменные.

Формально, если \(u = g(x)\) --- дифференцируемая функция, то

\[
\int f(g(x)) g'(x)\,dx = \int f(u)\,du.
\]

Эта замена упрощает вычисление интеграла.

\subsubsection{Шаги по
замене}\label{ux448ux430ux433ux438-ux43fux43e-ux437ux430ux43cux435ux43dux435}

\begin{enumerate}
\def\labelenumi{\arabic{enumi}.}
\tightlist
\item
  Определите внутреннюю функцию \(u = g(x)\), производная которой также
  присутствует в подынтегральной функции.
\item
  Вычислите \(du = g'(x)\,dx\).
\item
  Перепишите интеграл через \(u\).
\item
  Интегрируйте по \(u\).
\item
  Замените обратно \(u = g(x)\).
\end{enumerate}

\subsubsection{Примеры}\label{ux43fux440ux438ux43cux435ux440ux44b-11}

\begin{enumerate}
\def\labelenumi{\arabic{enumi}.}
\item
  Простая замена

  \[
  \int 2x \cos(x^2)\,dx
  \]

  Пусть \(u = x^2\), значит \(du = 2x\,dx\). Тогда интеграл станет
  \(\int \cos u \,du = \sin u + C = \sin(x^2) + C\).
\item
  Логарифмический случай

  \[
  \int \frac{2x}{x^2+1}\,dx
  \]

  Пусть \(u = x^2 + 1\), значит \(du = 2x\,dx\). Тогда интеграл станет
  \(\int \frac{1}{u}\,du = \ln|u| + C = \ln(x^2+1) + C\).
\item
  Тригонометрическая замена.

  \[
  \int \sin(3x)\,dx
  \]

  Пусть \(u = 3x\), поэтому \(du = 3\,dx\), следовательно,
  \(dx = \frac{du}{3}\).Интеграл становится
  \(\tfrac{1}{3}\int \sin u\,du = -\tfrac{1}{3}\cos u + C = -\tfrac{1}{3}\cos(3x) + C\).
\end{enumerate}

\subsubsection{Определенные интегралы с
заменой}\label{ux43eux43fux440ux435ux434ux435ux43bux435ux43dux43dux44bux435-ux438ux43dux442ux435ux433ux440ux430ux43bux44b-ux441-ux437ux430ux43cux435ux43dux43eux439}

При вычислении определенных интегралов необходимо также изменить
пределы:

\[
\int_a^b f(g(x)) g'(x)\,dx = \int_{g(a)}^{g(b)} f(u)\,du.
\]

Пример:

\[
\int_0^1 2x e^{x^2}\,dx.
\]

Пусть \(u = x^2\), \(du = 2x\,dx\). Ограничения: когда \(x=0, u=0\);
когда \(x=1, u=1\). Таким образом, интеграл становится

\[
\int_0^1 e^u\,du = e - 1.
\]

\subsubsection{Упражнения}\label{ux443ux43fux440ux430ux436ux43dux435ux43dux438ux44f-17}

\begin{enumerate}
\def\labelenumi{\arabic{enumi}.}
\tightlist
\item
  Оцените \(\int (x^2+1)^5 (2x)\,dx\).
\item
  Вычислите \(\int \frac{\cos x}{\sin x}\,dx\).
\item
  Оцените \(\int_0^\pi \sin(2x)\,dx\) с помощью подстановки.
\item
  Найдите \(\int e^{3x}\,dx\).
\item
  Вычислите \(\int \frac{1}{\sqrt{1+x^2}}\,dx\), используя
  \(u = 1+x^2\).
\end{enumerate}

\subsection{5.2 Интегрирование по
частям}\label{ux438ux43dux442ux435ux433ux440ux438ux440ux43eux432ux430ux43dux438ux435-ux43fux43e-ux447ux430ux441ux442ux44fux43c}

Интегрирование по частям --- это метод, основанный на правиле
произведения для деривативов. Это помогает вычислять интегралы,
включающие произведения функций, с которыми нелегко справиться одной
заменой.

\subsubsection{Формула}\label{ux444ux43eux440ux43cux443ux43bux430}

Из правила продукта:

\[
\frac{d}{dx}[u(x)v(x)] = u'(x)v(x) + u(x)v'(x).
\]

Интегрирование обеих сторон дает формулу интегрирования по частям:

\[
\int u\,dv = uv - \int v\,du.
\]

Здесь:

\begin{itemize}
\tightlist
\item
  \(u\) = функция, выбранная для дифференциации,
\item
  \(dv\) = оставшаяся часть подынтегральной функции, подлежащей
  интегрированию.
\end{itemize}

\subsubsection{\texorpdfstring{Выбор \(u\) и
\(dv\)}{Выбор u и dv}}\label{ux432ux44bux431ux43eux440-u-ux438-dv}

Распространенным принципом является LIATE (логарифмический, обратный
тригонометрический, алгебраический, тригонометрический,
экспоненциальный).

\begin{itemize}
\tightlist
\item
  Выберите \(u\) из самой ранней представленной категории.
\item
  В качестве остальных выберите \(dv\).
\end{itemize}

\subsubsection{Примеры}\label{ux43fux440ux438ux43cux435ux440ux44b-12}

\begin{enumerate}
\def\labelenumi{\arabic{enumi}.}
\tightlist
\item
  Полином × Экспонента
\end{enumerate}

\[
\int x e^x\,dx
\]

Пусть \(u = x\), \(dv = e^x dx\). Затем \(du = dx\), \(v = e^x\).

\[
\int x e^x\,dx = x e^x - \int e^x dx = x e^x - e^x + C.
\]

\begin{enumerate}
\def\labelenumi{\arabic{enumi}.}
\setcounter{enumi}{1}
\tightlist
\item
  Полином × Триг
\end{enumerate}

\[
\int x \cos x\,dx
\]

Пусть \(u = x\), \(dv = \cos x dx\). Затем \(du = dx\), \(v = \sin x\).

\[
\int x \cos x\,dx = x \sin x - \int \sin x dx = x \sin x + \cos x + C.
\]

\begin{enumerate}
\def\labelenumi{\arabic{enumi}.}
\setcounter{enumi}{2}
\tightlist
\item
  Логарифм
\end{enumerate}

\[
\int \ln x\,dx
\]

Пусть \(u = \ln x\), \(dv = dx\). Затем \(du = \frac{1}{x}dx\),
\(v = x\).

\[
\int \ln x\,dx = x \ln x - \int 1 dx = x \ln x - x + C.
\]

\subsubsection{Определенный интегральный
пример}\label{ux43eux43fux440ux435ux434ux435ux43bux435ux43dux43dux44bux439-ux438ux43dux442ux435ux433ux440ux430ux43bux44cux43dux44bux439-ux43fux440ux438ux43cux435ux440}

\[
\int_0^1 x e^x\,dx
\]

Используя предыдущий результат: \(\int x e^x dx = (x-1)e^x\). Оценить:

\[
\big[(x-1)e^x\big]_0^1 = (0)e^1 - (-1)e^0 = 0 + 1 = 1.
\]

\subsubsection{Почему это
важно}\label{ux43fux43eux447ux435ux43cux443-ux44dux442ux43e-ux432ux430ux436ux43dux43e-3}

Интегрирование по частям имеет решающее значение, когда замена не
удалась, особенно с логарифмами, обратными тригонометрическими функциями
и произведениями, включающими полиномы с экспонентами или
тригонометрическими функциями.

\subsubsection{Упражнения}\label{ux443ux43fux440ux430ux436ux43dux435ux43dux438ux44f-18}

\begin{enumerate}
\def\labelenumi{\arabic{enumi}.}
\tightlist
\item
  Оцените \(\int x \sin x\,dx\).
\item
  Найдите \(\int e^x \cos x\,dx\).
\item
  Вычислите \(\int_1^2 \ln x\,dx\).
\item
  Оцените \(\int x^2 e^x\,dx\).5. Используйте интеграцию по частям,
  чтобы отобразить
  \(\int \arctan x\,dx = x\arctan x - \tfrac{1}{2}\ln(1+x^2) + C\).
\end{enumerate}

\subsection{5.3 Тригонометрические интегралы и
замены}\label{ux442ux440ux438ux433ux43eux43dux43eux43cux435ux442ux440ux438ux447ux435ux441ux43aux438ux435-ux438ux43dux442ux435ux433ux440ux430ux43bux44b-ux438-ux437ux430ux43cux435ux43dux44b}

Многие интегралы включают тригонометрические функции. Их часто можно
упростить, используя тождества или делая специальные замены.

\subsubsection{Тригонометрические
интегралы}\label{ux442ux440ux438ux433ux43eux43dux43eux43cux435ux442ux440ux438ux447ux435ux441ux43aux438ux435-ux438ux43dux442ux435ux433ux440ux430ux43bux44b}

\begin{enumerate}
\def\labelenumi{\arabic{enumi}.}
\tightlist
\item
  Степени синуса и косинуса.
\end{enumerate}

\begin{itemize}
\tightlist
\item
  Если степень синуса нечетная: сохраните один \(\sin x\), преобразуйте
  остальные с помощью \(\sin^2x = 1 - \cos^2x\) и замените
  \(u = \cos x\).
\item
  Если степень косинуса нечетная: сохраните один \(\cos x\),
  преобразуйте остальные с помощью \(\cos^2x = 1 - \sin^2x\) и замените
  \(u = \sin x\).
\item
  Если оба четные: используйте тождества половинного угла.
\end{itemize}

Пример:

\[
\int \sin^3x \cos x \, dx
\]

Пусть \(u = \sin x\), \(du = \cos x\,dx\):

\[
\int u^3\,du = \tfrac{u^4}{4} + C = \tfrac{\sin^4x}{4} + C.
\]

\begin{enumerate}
\def\labelenumi{\arabic{enumi}.}
\setcounter{enumi}{1}
\tightlist
\item
  Произведения синуса и косинуса на разные углы Используйте формулы
  произведения к сумме:
\end{enumerate}

\[
\sin A \cos B = \tfrac{1}{2}[\sin(A+B) + \sin(A-B)].
\]

Пример:

\[
\int \sin(2x)\cos(3x)\,dx = \tfrac{1}{2}\int [\sin(5x) - \sin(x)]\,dx.
\]

\begin{enumerate}
\def\labelenumi{\arabic{enumi}.}
\setcounter{enumi}{2}
\tightlist
\item
  Степени секущей и тангенса.
\end{enumerate}

\begin{itemize}
\tightlist
\item
  Если степень секущего четная: сохраните \(\sec^2x\), преобразуйте
  остаток с помощью \(\sec^2x = 1 + \tan^2x\) и замените \(u = \tan x\).
\item
  Если степень тангенса нечетная: сохраните \(\sec^2x\), преобразуйте
  остаток с помощью \(\tan^2x = \sec^2x - 1\) и замените \(u = \tan x\).
\end{itemize}

Пример:

\[
\int \tan^3x \sec^2x \, dx
\]

Пусть \(u = \tan x\), \(du = \sec^2x\,dx\):

\[
\int u^3\,du = \tfrac{u^4}{4} + C = \tfrac{\tan^4x}{4} + C.
\]

\subsubsection{Тригонометрические
замены}\label{ux442ux440ux438ux433ux43eux43dux43eux43cux435ux442ux440ux438ux447ux435ux441ux43aux438ux435-ux437ux430ux43cux435ux43dux44b}

Для интегралов, включающих \(\sqrt{a^2 - x^2}\), \(\sqrt{a^2 + x^2}\)
или \(\sqrt{x^2 - a^2}\), используйте специальные замены:

\begin{enumerate}
\def\labelenumi{\arabic{enumi}.}
\tightlist
\item
  \(x = a \sin \theta\) для \(\sqrt{a^2 - x^2}\).
\item
  \(x = a \tan \theta\) для \(\sqrt{a^2 + x^2}\).
\item
  \(x = a \sec \theta\) для \(\sqrt{x^2 - a^2}\).
\end{enumerate}

Пример:

\[
\int \sqrt{a^2 - x^2}\,dx
\]

Пусть \(x = a\sin\theta\), поэтому \(dx = a\cos\theta\,d\theta\):

\[
\int \sqrt{a^2 - a^2\sin^2\theta}(a\cos\theta\,d\theta) = \int a^2 \cos^2\theta \, d\theta.
\]

Упростите, используя тождества половинного угла.

\subsubsection{Почему эти методы
важны}\label{ux43fux43eux447ux435ux43cux443-ux44dux442ux438-ux43cux435ux442ux43eux434ux44b-ux432ux430ux436ux43dux44b}

\begin{itemize}
\tightlist
\item
  Они преобразуют сложные алгебраические формы в управляемые
  тригонометрические.
\item
  Они особенно полезны при решении задач, связанных с площадями,
  объемами и длинами дуг.
\item
  Они закладывают основу для передовых методов интеграции.
\end{itemize}

\subsubsection{Упражнения}\label{ux443ux43fux440ux430ux436ux43dux435ux43dux438ux44f-19}

\begin{enumerate}
\def\labelenumi{\arabic{enumi}.}
\tightlist
\item
  Оцените \(\int \sin^4x \cos^2x \, dx\).
\item
  Вычислите \(\int \sin(5x)\cos(2x)\,dx\).
\item
  Оцените \(\int \tan^2x \sec^2x \, dx\).
\item
  Найдите \(\int \sqrt{9 - x^2}\,dx\) с помощью подстановки.
\item
  Покажите, что
  \(\int \frac{dx}{\sqrt{x^2 + a^2}} = \ln|x + \sqrt{x^2 + a^2}| + C\),
  используя \(x = a\tan\theta\).
\end{enumerate}

\subsection{5.4 Частные дробиПри интегрировании рациональных функций
(отношений многочленов) одним из эффективных методов является разложение
на частичные дроби. Этот метод выражает сложную дробь как сумму более
простых дробей, которые легче
интегрировать.}\label{ux447ux430ux441ux442ux43dux44bux435-ux434ux440ux43eux431ux438ux43fux440ux438-ux438ux43dux442ux435ux433ux440ux438ux440ux43eux432ux430ux43dux438ux438-ux440ux430ux446ux438ux43eux43dux430ux43bux44cux43dux44bux445-ux444ux443ux43dux43aux446ux438ux439-ux43eux442ux43dux43eux448ux435ux43dux438ux439-ux43cux43dux43eux433ux43eux447ux43bux435ux43dux43eux432-ux43eux434ux43dux438ux43c-ux438ux437-ux44dux444ux444ux435ux43aux442ux438ux432ux43dux44bux445-ux43cux435ux442ux43eux434ux43eux432-ux44fux432ux43bux44fux435ux442ux441ux44f-ux440ux430ux437ux43bux43eux436ux435ux43dux438ux435-ux43dux430-ux447ux430ux441ux442ux438ux447ux43dux44bux435-ux434ux440ux43eux431ux438.-ux44dux442ux43eux442-ux43cux435ux442ux43eux434-ux432ux44bux440ux430ux436ux430ux435ux442-ux441ux43bux43eux436ux43dux443ux44e-ux434ux440ux43eux431ux44c-ux43aux430ux43a-ux441ux443ux43cux43cux443-ux431ux43eux43bux435ux435-ux43fux440ux43eux441ux442ux44bux445-ux434ux440ux43eux431ux435ux439-ux43aux43eux442ux43eux440ux44bux435-ux43bux435ux433ux447ux435-ux438ux43dux442ux435ux433ux440ux438ux440ux43eux432ux430ux442ux44c.}

\subsubsection{Идея}\label{ux438ux434ux435ux44f-1}

Если \(R(x) = \frac{P(x)}{Q(x)}\) --- рациональная функция, у которой
степень \(P(x)\) меньше степени \(Q(x)\), мы можем разложить \(R(x)\) на
более простые дроби.

Эти более простые части соответствуют множителям знаменателя \(Q(x)\).

\subsubsection{Общие
формы}\label{ux43eux431ux449ux438ux435-ux444ux43eux440ux43cux44b}

\begin{enumerate}
\def\labelenumi{\arabic{enumi}.}
\tightlist
\item
  Различные линейные факторы Если
\end{enumerate}

\[
\frac{1}{(x-a)(x-b)},
\]

затем разложить как

\[
\frac{A}{x-a} + \frac{B}{x-b}.
\]

\begin{enumerate}
\def\labelenumi{\arabic{enumi}.}
\setcounter{enumi}{1}
\tightlist
\item
  Повторяющиеся линейные факторы Если знаменатель имеет \((x-a)^n\), то
  члены
\end{enumerate}

\[
\frac{A_1}{x-a} + \frac{A_2}{(x-a)^2} + \dots + \frac{A_n}{(x-a)^n}.
\]

\begin{enumerate}
\def\labelenumi{\arabic{enumi}.}
\setcounter{enumi}{2}
\tightlist
\item
  Неприводимые квадратичные множители Если знаменатель имеет
  \((x^2+bx+c)\), то числитель линейный:
\end{enumerate}

\[
\frac{Ax+B}{x^2+bx+c}.
\]

\subsubsection{Пример 1: различные линейные
факторы}\label{ux43fux440ux438ux43cux435ux440-1-ux440ux430ux437ux43bux438ux447ux43dux44bux435-ux43bux438ux43dux435ux439ux43dux44bux435-ux444ux430ux43aux442ux43eux440ux44b}

\[
\int \frac{1}{x^2 - 1}\,dx
\]

Знаменатель коэффициента: \((x-1)(x+1)\). Разложить:

\[
\frac{1}{x^2-1} = \frac{1}{2}\left(\frac{1}{x-1} - \frac{1}{x+1}\right).
\]

Интегрировать:

\[
\int \frac{1}{x^2 - 1}\,dx = \tfrac{1}{2}\ln\left|\frac{x-1}{x+1}\right| + C.
\]

\subsubsection{Пример 2: повторяющийся линейный
коэффициент}\label{ux43fux440ux438ux43cux435ux440-2-ux43fux43eux432ux442ux43eux440ux44fux44eux449ux438ux439ux441ux44f-ux43bux438ux43dux435ux439ux43dux44bux439-ux43aux43eux44dux444ux444ux438ux446ux438ux435ux43dux442}

\[
\int \frac{1}{(x-1)^2}\,dx
\]

Это уже просто:

\[
\int (x-1)^{-2}\,dx = -\frac{1}{x-1} + C.
\]

\subsubsection{Пример 3: неприводимый квадратичный
коэффициент}\label{ux43fux440ux438ux43cux435ux440-3-ux43dux435ux43fux440ux438ux432ux43eux434ux438ux43cux44bux439-ux43aux432ux430ux434ux440ux430ux442ux438ux447ux43dux44bux439-ux43aux43eux44dux444ux444ux438ux446ux438ux435ux43dux442}

\[
\int \frac{x}{x^2+1}\,dx
\]

Замените \(u = x^2+1\) или учтите, что числитель является производной от
знаменателя.

\[
\int \frac{x}{x^2+1}\,dx = \tfrac{1}{2}\ln(x^2+1) + C.
\]

\subsubsection{Этапы разложения на частичные
дроби}\label{ux44dux442ux430ux43fux44b-ux440ux430ux437ux43bux43eux436ux435ux43dux438ux44f-ux43dux430-ux447ux430ux441ux442ux438ux447ux43dux44bux435-ux434ux440ux43eux431ux438}

\begin{enumerate}
\def\labelenumi{\arabic{enumi}.}
\tightlist
\item
  Фактор знаменатель.
\item
  Напишите общую форму простейшей дроби.
\item
  Умножьте на знаменатель, чтобы очистить дроби.
\item
  Найдите неизвестные константы.
\item
  Интегрируйте каждый термин.
\end{enumerate}

\subsubsection{Почему это
важно}\label{ux43fux43eux447ux435ux43cux443-ux44dux442ux43e-ux432ux430ux436ux43dux43e-4}

\begin{itemize}
\tightlist
\item
  Преобразует сложные рациональные функции в простые логарифмические или
  арктангенсальные формы.
\item
  Особенно полезно в дифференциальных уравнениях и преобразованиях
  Лапласа.
\item
  Основы продвинутого исчисления и инженерии.
\end{itemize}

\subsubsection{Упражнения}\label{ux443ux43fux440ux430ux436ux43dux435ux43dux438ux44f-20}

\begin{enumerate}
\def\labelenumi{\arabic{enumi}.}
\tightlist
\item
  Разложить и интегрировать \(\int \frac{3x+5}{x^2-1}\,dx\).
\item
  Оцените \(\int \frac{1}{x^2(x+1)}\,dx\).
\item
  Вычислите \(\int \frac{2x+1}{x^2+2x+2}\,dx\).
\item
  Найдите \(\int \frac{1}{x^3 - x}\,dx\).
\item
  Докажите, что \(\int \frac{dx}{x^2+1} = \arctan x + C\), используя
  простейшие дроби или замену.
\end{enumerate}

\subsection{5.5 Несобственные
интегралы}\label{ux43dux435ux441ux43eux431ux441ux442ux432ux435ux43dux43dux44bux435-ux438ux43dux442ux435ux433ux440ux430ux43bux44b}

Некоторые интегралы невозможно вычислить напрямую, поскольку интервал
бесконечен или подынтегральная функция становится неограниченной. Такие
интегралы называются несобственными. Они определяются с помощью лимитов.

\subsubsection{Определение}\label{ux43eux43fux440ux435ux434ux435ux43bux435ux43dux438ux435-6}

\begin{enumerate}
\def\labelenumi{\arabic{enumi}.}
\tightlist
\item
  Бесконечный интервал
\end{enumerate}

\[\int_a^\infty f(x)\,dx = \lim_{b \to \infty} \int_a^b f(x)\,dx.
\]

\[
\int_{-\infty}^a f(x)\,dx = \lim_{b \to -\infty} \int_b^a f(x)\,dx.
\]

\begin{enumerate}
\def\labelenumi{\arabic{enumi}.}
\setcounter{enumi}{1}
\tightlist
\item
  Unbounded integrand If \(f(x)\) has a vertical asymptote at \(c\),
  then
\end{enumerate}

\[
\int_a^c f(x)\,dx = \lim_{t \to c^-} \int_a^t f(x)\,dx,
\]

\[
\int_c^b f(x)\,dx = \lim_{t \to c^+} \int_t^b f(x)\,dx.
\]

\subsubsection{Convergence and
Divergence}\label{convergence-and-divergence}

\begin{itemize}
\tightlist
\item
  If the limit exists and is finite, the improper integral converges.
\item
  If the limit does not exist or is infinite, the improper integral
  diverges.
\end{itemize}

\subsubsection{Examples}\label{examples-1}

\begin{enumerate}
\def\labelenumi{\arabic{enumi}.}
\tightlist
\item
  Exponential decay
\end{enumerate}

\[
\int_1^\infty \frac{1}{x^2}\,dx = \lim_{b \to \infty} \Big[-\tfrac{1}{x}\Big]_1^b = 1.
\]

This converges.

\begin{enumerate}
\def\labelenumi{\arabic{enumi}.}
\setcounter{enumi}{1}
\tightlist
\item
  Harmonic function
\end{enumerate}

\[
\int_1^\infty \frac{1}{x}\,dx = \lim_{b \to \infty} \ln b.
\]

This diverges to infinity.

\begin{enumerate}
\def\labelenumi{\arabic{enumi}.}
\setcounter{enumi}{2}
\tightlist
\item
  Asymptote at 0
\end{enumerate}

\[
\int_0^1 \frac{1}{\sqrt{x}}\,dx = \lim_{t \to 0^+} \int_t^1 x^{-1/2}\,dx.
\]

\[
= \lim_{t \to 0^+} [2\sqrt{x}]_t^1 = 2.
\]

This converges.

\begin{enumerate}
\def\labelenumi{\arabic{enumi}.}
\setcounter{enumi}{3}
\tightlist
\item
  Asymptote at 0 (divergent)
\end{enumerate}

\[
\int_0^1 \frac{1}{x}\,dx = \lim_{t \to 0^+} \ln(1) - \ln(t).
\]

This diverges since \(\ln(t) \to -\infty\).

\subsubsection{Comparison Test for Improper
Integrals}\label{comparison-test-for-improper-integrals}

\begin{itemize}
\tightlist
\item
  If \(0 \leq f(x) \leq g(x)\) for large \(x\), and \(\int g(x)\,dx\)
  converges, then \(\int f(x)\,dx\) also converges.
\item
  If \(\int f(x)\,dx\) diverges and \(f(x) \geq g(x) \geq 0\), then
  \(\int g(x)\,dx\) also diverges.
\end{itemize}

\subsubsection{Why Improper Integrals
Matter}\label{why-improper-integrals-matter}

\begin{itemize}
\tightlist
\item
  They extend integration to infinite domains and unbounded functions.
\item
  They are essential in probability (continuous distributions), physics
  (gravitational/electric fields), and Fourier analysis.
\end{itemize}

\subsubsection{Exercises}\label{exercises-2}

\begin{enumerate}
\def\labelenumi{\arabic{enumi}.}
\tightlist
\item
  Determine whether \(\int_1^\infty \frac{1}{x^p}\,dx\) converges for
  various values of \(p\).
\item
  Evaluate \(\int_0^\infty e^{-x}\,dx\).
\item
  Test convergence of \(\int_0^1 \frac{1}{x^p}\,dx\) depending on \(p\).
\item
  Compute \(\int_{-\infty}^\infty \frac{1}{1+x^2}\,dx\).
\item
  Use the comparison test to show that
  \(\int_1^\infty \frac{1}{x^2+1}\,dx\) converges.
\end{enumerate}

\section{Chapter 6. Applications of
Integration}\label{chapter-6.-applications-of-integration}

\subsection{6.1 Areas and Volumes}\label{areas-and-volumes}

One of the most important applications of integration is finding areas
under curves and volumes of solids.

\subsubsection{Area Between Curves}\label{area-between-curves}

If \(f(x) \geq g(x)\) on \([a, b]\), then the area between the curves
\(y=f(x)\) and \(y=g(x)\) is

\[
A = \int_a^b \big(f(x) - g(x)\big)\,dx.
\]

Example: Find the area between \(y=x^2\) and \(y=x\) on \([0,1]\).

\[
A = \int_0^1 (x - x^2)\,dx = \left[\tfrac{1}{2}x^2 - \tfrac{1}{3}x^3\right]_0^1 = \tfrac{1}{6}.
\]

\subsubsection{Volumes by Slicing}\label{volumes-by-slicing}

If a solid has cross-sectional area \(A(x)\) at position \(x\), then the
volume is

\[
V = \int_a^b A(x)\,dx.
\]\#\#\# Тома революции

Когда область вращается вокруг оси, объем полученного твердого тела
можно найти с помощью интегрирования.

\begin{enumerate}
\def\labelenumi{\arabic{enumi}.}
\tightlist
\item
  Дисковый метод Если регион под \(y=f(x)\), \(x\in[a,b]\) вращается
  вокруг оси \(x\):
\end{enumerate}

\[
V = \pi \int_a^b [f(x)]^2\,dx.
\]

\begin{enumerate}
\def\labelenumi{\arabic{enumi}.}
\setcounter{enumi}{1}
\tightlist
\item
  Метод шайбы Если область между \(y=f(x)\) и \(y=g(x)\) вращается
  вокруг оси \(x\):
\end{enumerate}

\[
V = \pi \int_a^b \Big([f(x)]^2 - [g(x)]^2\Big)\,dx.
\]

\begin{enumerate}
\def\labelenumi{\arabic{enumi}.}
\setcounter{enumi}{2}
\tightlist
\item
  Метод оболочки Если область под \(y=f(x)\) вращается вокруг оси \(y\):
\end{enumerate}

\[
V = 2\pi \int_a^b x f(x)\,dx.
\]

\subsubsection{Примеры}\label{ux43fux440ux438ux43cux435ux440ux44b-13}

\begin{enumerate}
\def\labelenumi{\arabic{enumi}.}
\tightlist
\item
  Дисковый метод Вращайте \(y=\sqrt{x}\), \(0 \leq x \leq 4\) вокруг оси
  \(x\):
\end{enumerate}

\[
V = \pi \int_0^4 (\sqrt{x})^2\,dx = \pi \int_0^4 x\,dx = \pi \left[\tfrac{1}{2}x^2\right]_0^4 = 8\pi.
\]

\begin{enumerate}
\def\labelenumi{\arabic{enumi}.}
\setcounter{enumi}{1}
\tightlist
\item
  Метод шайбы Вращение области между \(y=\sqrt{x}\) и \(y=1\),
  \(0 \leq x \leq 1\) вокруг оси \(x\):
\end{enumerate}

\[
V = \pi \int_0^1 \big((\sqrt{x})^2 - (1)^2\big)\,dx = \pi \int_0^1 (x-1)\,dx = -\tfrac{\pi}{2}.
\]

(Возьмите абсолютное значение для тома: \(V = \tfrac{\pi}{2}\)).

\begin{enumerate}
\def\labelenumi{\arabic{enumi}.}
\setcounter{enumi}{2}
\tightlist
\item
  Метод оболочки Повернуть область под \(y=x\), \(0 \leq x \leq 1\)
  вокруг оси \(y\):
\end{enumerate}

\[
V = 2\pi \int_0^1 x(x)\,dx = 2\pi \int_0^1 x^2\,dx = 2\pi \cdot \tfrac{1}{3} = \tfrac{2\pi}{3}.
\]

\subsubsection{Почему это
важно}\label{ux43fux43eux447ux435ux43cux443-ux44dux442ux43e-ux432ux430ux436ux43dux43e-5}

\begin{itemize}
\tightlist
\item
  Обеспечивает точные способы вычисления площадей и объемов в геометрии.
\item
  Необходим в физике, технике и теории вероятностей.
\item
  Знакомит с геометрическим мышлением с интеграцией.
\end{itemize}

\subsubsection{Упражнения}\label{ux443ux43fux440ux430ux436ux43dux435ux43dux438ux44f-21}

\begin{enumerate}
\def\labelenumi{\arabic{enumi}.}
\tightlist
\item
  Найдите область между \(y=\cos x\) и \(y=\sin x\) на \([0, \pi/2]\).
\item
  Вычислите объем твердого тела, образованного вращением \(y=x^2\),
  \(0 \leq x \leq 1\) вокруг оси \(x\).
\item
  Найдите объем твердого тела, образовавшегося в результате вращения
  области между \(y=x\) и \(y=\sqrt{x}\) на \([0,1]\) вокруг оси \(y\).
\item
  Используйте метод шайбы, чтобы вычислить объем твердого тела,
  образованного вращением \(y=\sqrt{1-x^2}\) (полукруг) вокруг оси
  \(x\).
\item
  Найдите область, заключенную между \(y=x^2+1\) и \(y=3x\).
\end{enumerate}

\subsection{6.2 Длина дуги и площадь
поверхности}\label{ux434ux43bux438ux43dux430-ux434ux443ux433ux438-ux438-ux43fux43bux43eux449ux430ux434ux44c-ux43fux43eux432ux435ux440ux445ux43dux43eux441ux442ux438}

Интегрирование также можно использовать для измерения длины кривых и
площади поверхности твердых тел, образованных вращающимися кривыми.

\subsubsection{Длина
дуги}\label{ux434ux43bux438ux43dux430-ux434ux443ux433ux438}

Для плавной кривой \(y=f(x)\) на интервале \([a,b]\) длина кривой равна

\[
L = \int_a^b \sqrt{1 + \big(f'(x)\big)^2}\,dx.
\]

Это происходит в результате аппроксимации кривой отрезками линий и
принятия предела.

Пример: Найдите длину \(y=\tfrac{1}{2}x^{3/2}\) от \(x=0\) до \(x=4\).

\begin{itemize}
\tightlist
\item
  Производное: \(f'(x) = \tfrac{3}{4}\sqrt{x}\).
\item
  Формула:
\end{itemize}

\[
L = \int_0^4 \sqrt{1 + \Big(\tfrac{3}{4}\sqrt{x}\Big)^2}\,dx
= \int_0^4 \sqrt{1 + \tfrac{9}{16}x}\,dx.
\]

Этот интеграл можно вычислить с помощью подстановки.\#\#\# Площадь
вращения

Если кривая \(y=f(x)\), \(a \leq x \leq b\) вращается вокруг оси \(x\),
площадь поверхности полученного твердого тела равна

\[
S = 2\pi \int_a^b f(x)\sqrt{1 + \big(f'(x)\big)^2}\,dx.
\]

Если вращаться вокруг оси \(y\):

\[
S = 2\pi \int_a^b x \sqrt{1 + \big(f'(x)\big)^2}\,dx.
\]

\subsubsection{Примеры}\label{ux43fux440ux438ux43cux435ux440ux44b-14}

\begin{enumerate}
\def\labelenumi{\arabic{enumi}.}
\tightlist
\item
  Длина дуги линии Для \(y=x\), \(0 \leq x \leq 3\):
\end{enumerate}

\[
L = \int_0^3 \sqrt{1+(1)^2}\,dx = \int_0^3 \sqrt{2}\,dx = 3\sqrt{2}.
\]

\begin{enumerate}
\def\labelenumi{\arabic{enumi}.}
\setcounter{enumi}{1}
\tightlist
\item
  Площадь поверхности сферы. Возьмите \(y = \sqrt{r^2 - x^2}\),
  \(-r \leq x \leq r\) и вращайтесь вокруг оси \(x\).
\end{enumerate}

\[
S = 2\pi \int_{-r}^r \sqrt{r^2 - x^2}\sqrt{1+\left(\frac{-x}{\sqrt{r^2-x^2}}\right)^2}\,dx.
\]

Упрощение дает \(S = 4\pi r^2\), знакомую формулу площади поверхности
сферы.

\subsubsection{Почему это
важно}\label{ux43fux43eux447ux435ux43cux443-ux44dux442ux43e-ux432ux430ux436ux43dux43e-6}

\begin{itemize}
\tightlist
\item
  Длина дуги расширяет представление о расстоянии до изогнутых путей.
\item
  Площадь поверхности вращения находит применение в физике, технике и
  дизайне.
\item
  Обеспечивает мост между математическим анализом и геометрией.
\end{itemize}

\subsubsection{Упражнения}\label{ux443ux43fux440ux430ux436ux43dux435ux43dux438ux44f-22}

\begin{enumerate}
\def\labelenumi{\arabic{enumi}.}
\tightlist
\item
  Найдите длину дуги \(y=\sqrt{x}\) от \(x=0\) до \(x=4\).
\item
  Вычислите площадь поверхности твердого тела, полученную вращением
  \(y=x^2\), \(0 \leq x \leq 1\) вокруг оси \(x\).
\item
  Найдите длину дуги \(y=\ln(\cosh x)\) от \(x=0\) до \(x=1\).
\item
  Покажите, что вращение \(y=\sqrt{r^2 - x^2}\) от \(0\) до \(r\) вокруг
  оси \(x\) дает половину площади поверхности сферы.
\item
  Выведите формулу площади поверхности конуса, вращая линию.
\end{enumerate}

\subsection{6.3 Работа и средние
значения}\label{ux440ux430ux431ux43eux442ux430-ux438-ux441ux440ux435ux434ux43dux438ux435-ux437ux43dux430ux447ux435ux43dux438ux44f}

Интеграция не ограничивается геометрией. Это также помогает вычислить
работу, совершенную силой, и среднее значение функции за интервал.

\subsubsection{Работа}\label{ux440ux430ux431ux43eux442ux430}

Если переменная сила \(F(x)\) перемещает объект по прямой от \(x=a\) до
\(x=b\), то общая работа равна

\[
W = \int_a^b F(x)\,dx.
\]

Эта формула обобщает простой случай \(W = F \cdot d\) для постоянной
силы.

Пример 1: Пружинная сила (Закон Гука) Для пружины, растянутой с длины
\(a\) до \(b\) с силой \(F(x) = kx\):

\[
W = \int_a^b kx\,dx = \tfrac{1}{2}k(b^2-a^2).
\]

Пример 2: Перекачивание воды Если из бака откачать воду, то требуемая
работа равна

\[
W = \int_a^b \text{(weight density)} \times \text{(cross-sectional area)} \times \text{(distance lifted)} \, dx.
\]

\subsubsection{Среднее значение
функции}\label{ux441ux440ux435ux434ux43dux435ux435-ux437ux43dux430ux447ux435ux43dux438ux435-ux444ux443ux43dux43aux446ux438ux438}

Среднее значение непрерывной функции \(f(x)\) на \([a,b]\) равно

\[
f_{\text{avg}} = \frac{1}{b-a} \int_a^b f(x)\,dx.
\]

Это непрерывный аналог усреднения списка чисел.

Пример 1: Для \(f(x)=x^2\) на \([0,2]\):

\[
f_{\text{avg}} = \tfrac{1}{2-0}\int_0^2 x^2 dx = \tfrac{1}{2}\cdot \tfrac{8}{3} = \tfrac{4}{3}.
\]

Пример 2:Если скорость частицы равна \(v(t)\), то средняя скорость по
\([a,b]\) равна

\[
v_{\text{avg}} = \frac{1}{b-a}\int_a^b v(t)\,dt.
\]

\subsubsection{Почему это
важно}\label{ux43fux43eux447ux435ux43cux443-ux44dux442ux43e-ux432ux430ux436ux43dux43e-7}

\begin{itemize}
\tightlist
\item
  Интегралы работы появляются в физических, инженерных и энергетических
  расчетах.
\item
  Среднее значение дает одно репрезентативное число для различных
  количеств.
\item
  Оба связывают исчисление с реальными проблемами движения, силы и
  эффективности.
\end{itemize}

\subsubsection{Упражнения}\label{ux443ux43fux440ux430ux436ux43dux435ux43dux438ux44f-23}

\begin{enumerate}
\def\labelenumi{\arabic{enumi}.}
\tightlist
\item
  Вычислите работу, необходимую для растяжения пружины с 2 м до 5 м,
  если \(k=10\).
\item
  Объект массой 100 кг поднимается вертикально на высоту 5 м в
  гравитационном поле (\(g=9.8 \,\text{m/s}^2\)). Выразите работу в виде
  интеграла и оцените.
\item
  Найдите среднее значение \(f(x)=\sin x\) на \([0,\pi]\).
\item
  Вычислите среднюю температуру, если
  \(T(t)=20+5\cos(\tfrac{\pi t}{12})\), за 24 часа в сутки.
\item
  Резервуар глубиной 10 м наполнен водой. Вычислите работу, необходимую
  для перекачки всей воды наверх, учитывая, что вода весит
  \(9800 \,\text{N/m}^3\).
\end{enumerate}

\subsection{6.4 Плотности вероятности и непрерывные
распределения}\label{ux43fux43bux43eux442ux43dux43eux441ux442ux438-ux432ux435ux440ux43eux44fux442ux43dux43eux441ux442ux438-ux438-ux43dux435ux43fux440ux435ux440ux44bux432ux43dux44bux435-ux440ux430ux441ux43fux440ux435ux434ux435ux43bux435ux43dux438ux44f}

Интегрирование также играет центральную роль в теории вероятностей,
особенно для непрерывных случайных величин. Вместо дискретных
результатов мы описываем вероятности с помощью функций, называемых
функциями плотности вероятности (pdf).

\subsubsection{Функции плотности
вероятности}\label{ux444ux443ux43dux43aux446ux438ux438-ux43fux43bux43eux442ux43dux43eux441ux442ux438-ux432ux435ux440ux43eux44fux442ux43dux43eux441ux442ux438}

Функция плотности вероятности \(f(x)\) должна удовлетворять двум
условиям:

\begin{enumerate}
\def\labelenumi{\arabic{enumi}.}
\item
  \(f(x) \geq 0\) для всех \(x\).
\item
  Общая площадь под кривой равна 1:

  \[
  \int_{-\infty}^\infty f(x)\,dx = 1.
  \]
\end{enumerate}

Если \(X\) является непрерывной случайной величиной с pdf \(f(x)\), то
вероятность того, что \(X\) находится между \(a\) и \(b\), равна

\[
P(a \leq X \leq b) = \int_a^b f(x)\,dx.
\]

\subsubsection{Кумулятивная функция
распределения}\label{ux43aux443ux43cux443ux43bux44fux442ux438ux432ux43dux430ux44f-ux444ux443ux43dux43aux446ux438ux44f-ux440ux430ux441ux43fux440ux435ux434ux435ux43bux435ux43dux438ux44f}

Кумулятивная функция распределения (cdf) определяется как

\[
F(x) = \int_{-\infty}^x f(t)\,dt.
\]

Он дает вероятность того, что случайная величина меньше или равна \(x\).

\subsubsection{Ожидаемое значение
(среднее)}\label{ux43eux436ux438ux434ux430ux435ux43cux43eux435-ux437ux43dux430ux447ux435ux43dux438ux435-ux441ux440ux435ux434ux43dux435ux435}

Ожидаемое значение непрерывной случайной величины представляет собой
средневзвешенное значение:

\[
E[X] = \int_{-\infty}^\infty x f(x)\,dx.
\]

\subsubsection{Примеры}\label{ux43fux440ux438ux43cux435ux440ux44b-15}

\begin{enumerate}
\def\labelenumi{\arabic{enumi}.}
\tightlist
\item
  Равномерное распределение Для \(f(x) = \tfrac{1}{b-a}\) на \([a,b]\):
\end{enumerate}

\begin{itemize}
\item
  Вероятность интервала \([c,d]\):

  \[
  P(c \leq X \leq d) = \frac{d-c}{b-a}.
  \]
\item
  Ожидаемое значение: \(E[X] = \tfrac{a+b}{2}\).
\end{itemize}

\begin{enumerate}
\def\labelenumi{\arabic{enumi}.}
\setcounter{enumi}{1}
\tightlist
\item
  Экспоненциальное распределение Для \(f(x) = \lambda e^{-\lambda x}\),
  \(x \geq 0\):
\end{enumerate}

\begin{itemize}
\tightlist
\item
  \(\int_0^\infty \lambda e^{-\lambda x}\,dx = 1\).
\item
  Среднее значение: \(E[X] = \tfrac{1}{\lambda}\).
\end{itemize}

\begin{enumerate}
\def\labelenumi{\arabic{enumi}.}
\setcounter{enumi}{2}
\tightlist
\item
  Нормальное распределение Колоколообразная кривая:
\end{enumerate}

\[
f(x) = \frac{1}{\sqrt{2\pi\sigma^2}} e^{-\frac{(x-\mu)^2}{2\sigma^2}}.
\]

Он интегрируется до 1, но требует передовых методов.

\subsubsection{Почему это важно- Плотности вероятности описывают
неопределенность в науке, технике и
статистике.}\label{ux43fux43eux447ux435ux43cux443-ux44dux442ux43e-ux432ux430ux436ux43dux43e--ux43fux43bux43eux442ux43dux43eux441ux442ux438-ux432ux435ux440ux43eux44fux442ux43dux43eux441ux442ux438-ux43eux43fux438ux441ux44bux432ux430ux44eux442-ux43dux435ux43eux43fux440ux435ux434ux435ux43bux435ux43dux43dux43eux441ux442ux44c-ux432-ux43dux430ux443ux43aux435-ux442ux435ux445ux43dux438ux43aux435-ux438-ux441ux442ux430ux442ux438ux441ux442ux438ux43aux435.}

\begin{itemize}
\tightlist
\item
  Интегралы связывают площади под кривыми с вероятностями.
\item
  Непрерывные распределения обобщают идею подсчета результатов для
  измерения вероятности на протяжении интервалов.
\end{itemize}

\subsubsection{Упражнения}\label{ux443ux43fux440ux430ux436ux43dux435ux43dux438ux44f-24}

\begin{enumerate}
\def\labelenumi{\arabic{enumi}.}
\tightlist
\item
  Покажите, что равномерная плотность \(f(x) = \tfrac{1}{b-a}\) на
  \([a,b]\) интегрируется до 1.
\item
  Для экспоненциального распределения с \(\lambda = 2\) вычислите
  \(P(0 \leq X \leq 1)\).
\item
  Найдите ожидаемое значение \(X\), если \(f(x) = 3x^2\) на \([0,1]\).
\item
  Убедитесь, что нормальное распределение со средним значением 0 и
  дисперсией 1 имеет общую вероятность 1 (нет необходимости в полном
  доказательстве, но объясните, почему это справедливо).
\item
  Вычислите cdf однородного дистрибутива на \([0,1]\).
\end{enumerate}

\section{Часть III. Многомерное
исчисление}\label{ux447ux430ux441ux442ux44c-iii.-ux43cux43dux43eux433ux43eux43cux435ux440ux43dux43eux435-ux438ux441ux447ux438ux441ux43bux435ux43dux438ux435}

\section{Глава 7. Векторные функции и
кривые}\label{ux433ux43bux430ux432ux430-7.-ux432ux435ux43aux442ux43eux440ux43dux44bux435-ux444ux443ux43dux43aux446ux438ux438-ux438-ux43aux440ux438ux432ux44bux435}

\subsection{7.1 Векторные функции и пространственные
кривые}\label{ux432ux435ux43aux442ux43eux440ux43dux44bux435-ux444ux443ux43dux43aux446ux438ux438-ux438-ux43fux440ux43eux441ux442ux440ux430ux43dux441ux442ux432ux435ux43dux43dux44bux435-ux43aux440ux438ux432ux44bux435}

В исчислении с несколькими переменными функции могут выводить векторы
вместо чисел. Их называют векторными функциями, и они необходимы для
описания кривых в пространстве.

\subsubsection{Определение}\label{ux43eux43fux440ux435ux434ux435ux43bux435ux43dux438ux435-7}

Вектор-функция -- это функция вида

\[
\mathbf{r}(t) = \langle x(t), y(t), z(t) \rangle,
\]

где \(x(t), y(t), z(t)\) --- функции с действительным знаком.

\begin{itemize}
\tightlist
\item
  Вход \(t\) часто называют параметром.
\item
  Выходные данные представляют собой вектор в 2D или 3D-пространстве.
\item
  График векторной функции в 3D представляет собой пространственную
  кривую.
\end{itemize}

\subsubsection{Примеры}\label{ux43fux440ux438ux43cux435ux440ux44b-16}

\begin{enumerate}
\def\labelenumi{\arabic{enumi}.}
\tightlist
\item
  Линия
\end{enumerate}

\[
\mathbf{r}(t) = \langle 1+2t, \; 3-t, \; 4+5t \rangle.
\]

Это описывает прямую линию, проходящую через точку \((1,3,4)\) с
вектором направления \(\langle 2,-1,5 \rangle\).

\begin{enumerate}
\def\labelenumi{\arabic{enumi}.}
\setcounter{enumi}{1}
\tightlist
\item
  Круг в плоскости
\end{enumerate}

\[
\mathbf{r}(t) = \langle \cos t, \; \sin t, \; 0 \rangle, \quad 0 \leq t < 2\pi.
\]

\begin{enumerate}
\def\labelenumi{\arabic{enumi}.}
\setcounter{enumi}{2}
\tightlist
\item
  Спираль
\end{enumerate}

\[
\mathbf{r}(t) = \langle \cos t, \; \sin t, \; t \rangle.
\]

Это спираль, восходящая вокруг оси \(z\).

\subsubsection{Ограничения и
непрерывность}\label{ux43eux433ux440ux430ux43dux438ux447ux435ux43dux438ux44f-ux438-ux43dux435ux43fux440ux435ux440ux44bux432ux43dux43eux441ux442ux44c}

Векторная функция является непрерывной в \(t=a\), если каждый компонент
\(x(t), y(t), z(t)\) непрерывен в \(t=a\).

\[
\lim_{t \to a} \mathbf{r}(t) = \langle \lim_{t \to a} x(t), \; \lim_{t \to a} y(t), \; \lim_{t \to a} z(t) \rangle.
\]

\subsubsection{Геометрия пространственных
кривых}\label{ux433ux435ux43eux43cux435ux442ux440ux438ux44f-ux43fux440ux43eux441ux442ux440ux430ux43dux441ux442ux432ux435ux43dux43dux44bux445-ux43aux440ux438ux432ux44bux445}

\begin{itemize}
\tightlist
\item
  Каждая кривая имеет касательное направление, заданное производной.
\item
  Пространственные кривые могут моделировать пути движения, траектории
  частиц и геометрические фигуры.
\end{itemize}

\subsubsection{Почему это
важно}\label{ux43fux43eux447ux435ux43cux443-ux44dux442ux43e-ux432ux430ux436ux43dux43e-8}

Векторные функции являются основой исчисления многих переменных,
позволяя нам распространить идеи производных и интегралов на более
высокие измерения. Они также естественным образом появляются в физике
(движение в 3D, электромагнетизм, гидродинамика).

\subsubsection{Упражнения}\label{ux443ux43fux440ux430ux436ux43dux435ux43dux438ux44f-25}

\begin{enumerate}
\def\labelenumi{\arabic{enumi}.}
\tightlist
\item
  Напишите векторную функцию для линии, проходящей через \((0,1,2)\),
  параллельной вектору \(\langle 3,-2,1 \rangle\).2. Опишите кривую,
  заданную
  \(\mathbf{r}(t) = \langle 2\cos t, \; 2\sin t, \; 3 \rangle\).
\item
  Определите, является ли
  \(\mathbf{r}(t) = \langle e^t, \; \ln t, \; t^2 \rangle\) непрерывным
  на уровне \(t=1\).
\item
  Нарисуйте спираль
  \(\mathbf{r}(t) = \langle \cos t, \; \sin t, \; 2t \rangle\).
\item
  Найдите точку на кривой
  \(\mathbf{r}(t) = \langle t, \; t^2, \; t^3 \rangle\), когда \(t=2\).
\end{enumerate}

\subsection{7.2 Производные и интегралы векторных
функций}\label{ux43fux440ux43eux438ux437ux432ux43eux434ux43dux44bux435-ux438-ux438ux43dux442ux435ux433ux440ux430ux43bux44b-ux432ux435ux43aux442ux43eux440ux43dux44bux445-ux444ux443ux43dux43aux446ux438ux439}

Векторные функции можно дифференцировать и интегрировать так же, как и
обычные функции --- мы просто применяем операцию к каждому компоненту.
Это позволяет нам изучать движение, скорость, ускорение и накопление в
более высоких измерениях.

\subsubsection{Производная векторной
функции}\label{ux43fux440ux43eux438ux437ux432ux43eux434ux43dux430ux44f-ux432ux435ux43aux442ux43eux440ux43dux43eux439-ux444ux443ux43dux43aux446ux438ux438}

Если

\[
\mathbf{r}(t) = \langle x(t), y(t), z(t) \rangle,
\]

тогда

\[
\mathbf{r}'(t) = \langle x'(t), y'(t), z'(t) \rangle.
\]

Этот производный вектор указывает в касательном направлении к кривой в
параметре \(t\).

\begin{itemize}
\tightlist
\item
  Скорость: если \(\mathbf{r}(t)\) указывает положение частицы в момент
  времени \(t\), то \(\mathbf{v}(t) = \mathbf{r}'(t)\) --- это вектор ее
  скорости.
\item
  Скорость: величина \(|\mathbf{v}(t)|\) --- это скорость частицы.
\item
  Ускорение: \(\mathbf{a}(t) = \mathbf{v}'(t) = \mathbf{r}''(t)\).
\end{itemize}

\subsubsection{Примеры}\label{ux43fux440ux438ux43cux435ux440ux44b-17}

\begin{enumerate}
\def\labelenumi{\arabic{enumi}.}
\tightlist
\item
  Спираль
\end{enumerate}

\[
\mathbf{r}(t) = \langle \cos t, \sin t, t \rangle.
\]

\begin{itemize}
\tightlist
\item
  Скорость: \(\mathbf{v}(t) = \langle -\sin t, \cos t, 1 \rangle\).
\item
  Скорость:
  \(|\mathbf{v}(t)| = \sqrt{(-\sin t)^2 + (\cos t)^2 + 1^2} = \sqrt{2}\).
\item
  Ускорение: \(\mathbf{a}(t) = \langle -\cos t, -\sin t, 0 \rangle\).
\end{itemize}

\begin{enumerate}
\def\labelenumi{\arabic{enumi}.}
\setcounter{enumi}{1}
\tightlist
\item
  Движение снаряда
\end{enumerate}

\[
\mathbf{r}(t) = \langle v_0 \cos\theta \cdot t, \; v_0 \sin\theta \cdot t - \tfrac{1}{2}gt^2 \rangle.
\]

Это моделирует параболическую траекторию снаряда под действием силы
тяжести.

\subsubsection{Интеграл векторной
функции}\label{ux438ux43dux442ux435ux433ux440ux430ux43b-ux432ux435ux43aux442ux43eux440ux43dux43eux439-ux444ux443ux43dux43aux446ux438ux438}

Если

\[
\mathbf{r}(t) = \langle x(t), y(t), z(t) \rangle,
\]

тогда

\[
\int \mathbf{r}(t)\,dt = \left\langle \int x(t)\,dt, \; \int y(t)\,dt, \; \int z(t)\,dt \right\rangle + \mathbf{C},
\]

где \(\mathbf{C}\) --- постоянный вектор.

\subsubsection{Пример}\label{ux43fux440ux438ux43cux435ux440}

\[
\mathbf{r}(t) = \langle t, t^2, t^3 \rangle.
\]

\begin{itemize}
\tightlist
\item
  Производное: \(\mathbf{r}'(t) = \langle 1, 2t, 3t^2 \rangle\).
\item
  Интеграл:
\end{itemize}

\[
\int \mathbf{r}(t)\,dt = \langle \tfrac{1}{2}t^2, \tfrac{1}{3}t^3, \tfrac{1}{4}t^4 \rangle + \mathbf{C}.
\]

\subsubsection{Почему это
важно}\label{ux43fux43eux447ux435ux43cux443-ux44dux442ux43e-ux432ux430ux436ux43dux43e-9}

\begin{itemize}
\tightlist
\item
  Производные векторных функций описывают движение и силы в
  пространстве.
\item
  Интегралы дают перемещение, работу и накопленные величины.
\item
  Эти инструменты напрямую соединяют математический анализ с физикой и
  инженерией.
\end{itemize}

\subsubsection{Упражнения}\label{ux443ux43fux440ux430ux436ux43dux435ux43dux438ux44f-26}

\begin{enumerate}
\def\labelenumi{\arabic{enumi}.}
\tightlist
\item
  Для \(\mathbf{r}(t) = \langle t, \cos t, \sin t \rangle\) найдите
  скорость, скорость и ускорение.2. Вычислите \(\mathbf{r}'(t)\) для
  \(\mathbf{r}(t) = \langle e^t, \ln t, t^2 \rangle\).
\item
  Интегрируйте \(\mathbf{r}(t) = \langle 1, t, t^2 \rangle\).
\item
  Частица имеет скорость \(\mathbf{v}(t) = \langle t, 2, 0 \rangle\).
  Найдите его вектор положения, если
  \(\mathbf{r}(0) = \langle 1, 0, 0 \rangle\).
\item
  Докажите, что скорость
  \(\mathbf{r}(t) = \langle \cos t, \sin t, 0 \rangle\) постоянна.
\end{enumerate}

\subsection{7.3 Длина и кривизна
дуги}\label{ux434ux43bux438ux43dux430-ux438-ux43aux440ux438ux432ux438ux437ux43dux430-ux434ux443ux433ux438}

Векторное исчисление предоставляет инструменты для измерения не только
пути, проложенного кривой, но и того, насколько резко она изгибается.
Они выражаются через длину и кривизну дуги.

\subsubsection{Длина дуги пространственной
кривой}\label{ux434ux43bux438ux43dux430-ux434ux443ux433ux438-ux43fux440ux43eux441ux442ux440ux430ux43dux441ux442ux432ux435ux43dux43dux43eux439-ux43aux440ux438ux432ux43eux439}

Если кривая задана формулой

\[
\mathbf{r}(t) = \langle x(t), y(t), z(t) \rangle, \quad a \leq t \leq b,
\]

тогда длина дуги

\[
L = \int_a^b |\mathbf{r}'(t)|\,dt,
\]

где

\[
|\mathbf{r}'(t)| = \sqrt{(x'(t))^2 + (y'(t))^2 + (z'(t))^2}.
\]

Пример: Для спирали
\(\mathbf{r}(t) = \langle \cos t, \sin t, t \rangle, \, 0 \leq t \leq 2\pi\):

\begin{itemize}
\tightlist
\item
  Скорость: \(\mathbf{r}'(t) = \langle -\sin t, \cos t, 1 \rangle\).
\item
  Скорость:
  \(|\mathbf{r}'(t)| = \sqrt{(-\sin t)^2 + (\cos t)^2 + 1^2} = \sqrt{2}\).
\item
  Длина дуги:
\end{itemize}

\[
L = \int_0^{2\pi} \sqrt{2}\,dt = 2\pi\sqrt{2}.
\]

\subsubsection{Кривизна}\label{ux43aux440ux438ux432ux438ux437ux43dux430}

Кривизна измеряет, насколько быстро кривая меняет направление.

Для плавной кривой \(\mathbf{r}(t)\):

\[
\kappa(t) = \frac{|\mathbf{r}'(t) \times \mathbf{r}''(t)|}{|\mathbf{r}'(t)|^3}.
\]

\begin{itemize}
\tightlist
\item
  \(\kappa = 0\): прямая линия.
\item
  Больший \(\kappa\): кривая изгибается более резко.
\end{itemize}

Пример: Для круга радиусом \(r\):

\[
\mathbf{r}(t) = \langle r\cos t, r\sin t \rangle.
\]

Затем \(\kappa = \tfrac{1}{r}\). Таким образом, кривизна постоянна и
обратно пропорциональна радиусу.

\subsubsection{Единичные касательные и нормальные
векторы}\label{ux435ux434ux438ux43dux438ux447ux43dux44bux435-ux43aux430ux441ux430ux442ux435ux43bux44cux43dux44bux435-ux438-ux43dux43eux440ux43cux430ux43bux44cux43dux44bux435-ux432ux435ux43aux442ux43eux440ux44b}

\begin{itemize}
\tightlist
\item
  Касательный вектор:
\end{itemize}

\[
\mathbf{T}(t) = \frac{\mathbf{r}'(t)}{|\mathbf{r}'(t)|}.
\]

\begin{itemize}
\tightlist
\item
  Вектор нормали: указывает на центр кривизны, определяемый как
\end{itemize}

\[
\mathbf{N}(t) = \frac{\mathbf{T}'(t)}{|\mathbf{T}'(t)|}.
\]

Эти векторы описывают геометрию движения: направление движения и
направление поворота.

\subsubsection{Почему это
важно}\label{ux43fux43eux447ux435ux43cux443-ux44dux442ux43e-ux432ux430ux436ux43dux43e-10}

\begin{itemize}
\tightlist
\item
  Длина дуги обобщает понятие расстояния до кривых в пространстве.
\item
  Кривизна описывает изгиб, имеющий решающее значение в физике
  (центростремительное ускорение), технике (дороги, американские горки)
  и компьютерной графике.
\end{itemize}

\subsubsection{Упражнения}\label{ux443ux43fux440ux430ux436ux43dux435ux43dux438ux44f-27}

\begin{enumerate}
\def\labelenumi{\arabic{enumi}.}
\tightlist
\item
  Найдите длину дуги \(\mathbf{r}(t) = \langle t, t^2, 0 \rangle\) от
  \(t=0\) до \(t=1\).
\item
  Вычислите кривизну круга
  \(\mathbf{r}(t) = \langle \cos t, \sin t \rangle\).
\item
  Для \(\mathbf{r}(t) = \langle t, \cos t, \sin t \rangle\) вычислите
  \(|\mathbf{r}'(t)|\).
\item
  Покажите, что прямая линия имеет кривизну \(\kappa = 0\).5. Найдите
  касательный вектор к
  \(\mathbf{r}(t) = \langle e^t, e^{-t}, t \rangle\) в точке \(t=0\).
\end{enumerate}

\subsection{7.4 Движение в
пространстве}\label{ux434ux432ux438ux436ux435ux43dux438ux435-ux432-ux43fux440ux43eux441ux442ux440ux430ux43dux441ux442ux432ux435}

Векторные функции особенно эффективны при описании движения в двух или
трех измерениях. Положение, скорость и ускорение естественным образом
выражаются с помощью производных и интегралов векторных функций.

\subsubsection{Положение, скорость и
ускорение}\label{ux43fux43eux43bux43eux436ux435ux43dux438ux435-ux441ux43aux43eux440ux43eux441ux442ux44c-ux438-ux443ux441ux43aux43eux440ux435ux43dux438ux435}

\begin{itemize}
\tightlist
\item
  Вектор положения:
\end{itemize}

\[
\mathbf{r}(t) = \langle x(t), y(t), z(t) \rangle
\]

\begin{itemize}
\tightlist
\item
  Вектор скорости (производная положения):
\end{itemize}

\[
\mathbf{v}(t) = \mathbf{r}'(t) = \langle x'(t), y'(t), z'(t) \rangle
\]

\begin{itemize}
\tightlist
\item
  Скорость (величина скорости):
\end{itemize}

\[
|\mathbf{v}(t)| = \sqrt{(x'(t))^2 + (y'(t))^2 + (z'(t))^2}
\]

\begin{itemize}
\tightlist
\item
  Вектор ускорения (производная скорости):
\end{itemize}

\[
\mathbf{a}(t) = \mathbf{v}'(t) = \mathbf{r}''(t).
\]

\subsubsection{Тангенциальные и нормальные
компоненты}\label{ux442ux430ux43dux433ux435ux43dux446ux438ux430ux43bux44cux43dux44bux435-ux438-ux43dux43eux440ux43cux430ux43bux44cux43dux44bux435-ux43aux43eux43cux43fux43eux43dux435ux43dux442ux44b}

Ускорение можно разложить на две составляющие:

\[
\mathbf{a}(t) = a_T \mathbf{T}(t) + a_N \mathbf{N}(t),
\]

где:

\begin{itemize}
\tightlist
\item
  \(\mathbf{T}(t)\) = единичный касательный вектор,
\item
  \(\mathbf{N}(t)\) = главный вектор нормали,
\item
  \(a_T = \frac{d}{dt}|\mathbf{v}(t)|\) = тангенциальное ускорение
  (изменение скорости),
\item
  \(a_N = \kappa |\mathbf{v}(t)|^2\) = нормальное ускорение (изменение
  направления).
\end{itemize}

\subsubsection{Движение снаряда в
3D}\label{ux434ux432ux438ux436ux435ux43dux438ux435-ux441ux43dux430ux440ux44fux434ux430-ux432-3d}

При силе тяжести, действующей в направлении \(-z\):

\[
\mathbf{r}(t) = \langle v_0 \cos\theta \cos\phi \cdot t,\; v_0 \cos\theta \sin\phi \cdot t,\; v_0 \sin\theta \cdot t - \tfrac{1}{2}gt^2 \rangle,
\]

где \(v_0\) --- начальная скорость, \(\theta\) угол запуска и \(\phi\)
азимутальное направление.

\subsubsection{Пример: винтовое
движение}\label{ux43fux440ux438ux43cux435ux440-ux432ux438ux43dux442ux43eux432ux43eux435-ux434ux432ux438ux436ux435ux43dux438ux435}

\[
\mathbf{r}(t) = \langle \cos t, \sin t, t \rangle
\]

\begin{itemize}
\tightlist
\item
  Скорость: \(\mathbf{v}(t) = \langle -\sin t, \cos t, 1 \rangle\).
\item
  Скорость: \(|\mathbf{v}(t)| = \sqrt{2}\).
\item
  Ускорение: \(\mathbf{a}(t) = \langle -\cos t, -\sin t, 0 \rangle\).
\item
  Движение равномерное по скорости, по спирали вверх.
\end{itemize}

\subsubsection{Почему это
важно}\label{ux43fux43eux447ux435ux43cux443-ux44dux442ux43e-ux432ux430ux436ux43dux43e-11}

\begin{itemize}
\tightlist
\item
  Обеспечивает математический язык для реального движения.
\item
  Необходим в физике (силы, траектории, круговое движение).
\item
  Фонд передовой механики и инженерных моделей.
\end{itemize}

\subsubsection{Упражнения}\label{ux443ux43fux440ux430ux436ux43dux435ux43dux438ux44f-28}

\begin{enumerate}
\def\labelenumi{\arabic{enumi}.}
\tightlist
\item
  Частица движется по \(\mathbf{r}(t) = \langle t, t^2, t^3 \rangle\).
  Найдите скорость и ускорение в \(t=1\).
\item
  Докажите, что скорость спирали
  \(\mathbf{r}(t) = \langle \cos t, \sin t, t \rangle\) постоянна.
\item
  Снаряд запускается с \(v_0 = 20 \,\text{m/s}\) под углом \(45^\circ\).
  Запишите его вектор положения, предполагая движение в вертикальной
  плоскости.
\item
  Для \(\mathbf{r}(t) = \langle e^t, e^{-t}, t \rangle\) найдите
  \(\mathbf{v}(t)\) и \(\mathbf{a}(t)\).
\item
  Разложим вектор ускорения на тангенциальную и нормальную составляющие
  для движения по окружности радиуса \(r\).\# Глава 8. Функции
  нескольких переменных
\end{enumerate}

\subsection{8.1 Пределы и непрерывность нескольких
переменных}\label{ux43fux440ux435ux434ux435ux43bux44b-ux438-ux43dux435ux43fux440ux435ux440ux44bux432ux43dux43eux441ux442ux44c-ux43dux435ux441ux43aux43eux43bux44cux43aux438ux445-ux43fux435ux440ux435ux43cux435ux43dux43dux44bux445}

В исчислении с несколькими переменными функции могут зависеть от двух
или более переменных, например \(f(x,y)\) или \(f(x,y,z)\). Понятия
пределов и непрерывности естественным образом вытекают из исчисления с
одной переменной, но они более тонкие, поскольку мы должны учитывать все
возможные пути подхода.

\subsubsection{Ограничения в двух
переменных}\label{ux43eux433ux440ux430ux43dux438ux447ux435ux43dux438ux44f-ux432-ux434ux432ux443ux445-ux43fux435ux440ux435ux43cux435ux43dux43dux44bux445}

Для функции \(f(x,y)\) мы говорим

\[
\lim_{(x,y) \to (a,b)} f(x,y) = L
\]

если \(f(x,y)\) приближается к \(L\) произвольно, когда \((x,y)\)
приближается к \((a,b)\) по любому пути.

Если разные пути дают разные предельные значения, то предела не
существует.

Пример 1 (существует ограничение):

\[
f(x,y) = x^2 + y^2, \quad \lim_{(x,y) \to (0,0)} f(x,y) = 0.
\]

Пример 2 (лимит не существует):

\[
f(x,y) = \frac{xy}{x^2+y^2}, \quad (x,y) \to (0,0).
\]

\begin{itemize}
\tightlist
\item
  Вдоль \(y=0\) функция равна 0.
\item
  Рядом с \(y=x\) используется функция \(\tfrac{1}{2}\). Разные
  результаты → лимита не существует.
\end{itemize}

\subsubsection{Непрерывность}\label{ux43dux435ux43fux440ux435ux440ux44bux432ux43dux43eux441ux442ux44c-1}

Функция \(f(x,y)\) является непрерывной в \((a,b)\), если

\[
\lim_{(x,y)\to(a,b)} f(x,y) = f(a,b).
\]

Полиномы и рациональные функции (где знаменатель ≠ 0) непрерывны всюду в
своих областях определения.

\subsubsection{Расширение до трех и более
переменных}\label{ux440ux430ux441ux448ux438ux440ux435ux43dux438ux435-ux434ux43e-ux442ux440ux435ux445-ux438-ux431ux43eux43bux435ux435-ux43fux435ux440ux435ux43cux435ux43dux43dux44bux445}

Для \(f(x,y,z)\) пределы и непрерывность определяются одинаково, но к
точке \((a,b,c)\) необходимо приближаться с бесконечного множества
направлений в пространстве.

\subsubsection{Почему это
важно}\label{ux43fux43eux447ux435ux43cux443-ux44dux442ux43e-ux432ux430ux436ux43dux43e-12}

\begin{itemize}
\tightlist
\item
  Непрерывность гарантирует отсутствие скачков, дыр и асимптот в
  функциях многих переменных.
\item
  Пределы имеют основополагающее значение для определения частных
  производных и кратных интегралов.
\item
  Эти концепции являются строительными блоками для многомерного
  исчисления.
\end{itemize}

\subsubsection{Упражнения}\label{ux443ux43fux440ux430ux436ux43dux435ux43dux438ux44f-29}

\begin{enumerate}
\def\labelenumi{\arabic{enumi}.}
\tightlist
\item
  Определите, существует ли \(\lim_{(x,y)\to(0,0)} (x^2+y^2)\).
\item
  Покажите, что \(\lim_{(x,y)\to(0,0)} \frac{x^2y}{x^2+y^2} = 0\) вдоль
  всех прямых путей \(y=mx\).
\item
  Существует ли ограничение для \(f(x,y) = \frac{x^2-y^2}{x^2+y^2}\) как
  \((x,y)\to(0,0)\)?
\item
  Объясните, почему многочлены от двух переменных непрерывны всюду.
\item
  Приведите пример функции двух переменных, разрывной в точке, и
  объясните, почему.
\end{enumerate}

\subsection{8.2 Частные
производные}\label{ux447ux430ux441ux442ux43dux44bux435-ux43fux440ux43eux438ux437ux432ux43eux434ux43dux44bux435}

В функциях нескольких переменных мы часто хотим измерить, как изменяется
функция, когда изменяется только одна переменная, а остальные остаются
постоянными. Это приводит к идее частных производных.

\subsubsection{Определение}\label{ux43eux43fux440ux435ux434ux435ux43bux435ux43dux438ux435-8}

Для функции \(f(x,y)\) частная производная по \(x\) в точке \((a,b)\)
равна

\[
\frac{\partial f}{\partial x}(a,b) = \lim_{h \to 0} \frac{f(a+h, b) - f(a,b)}{h}.
\]

Аналогично, частная производная по \(y\) равна

\[\frac{\partial f}{\partial y}(a,b) = \lim_{h \to 0} \frac{f(a, b+h) - f(a,b)}{h}.
\]

We treat all other variables as constants when differentiating.

\subsubsection{Notation}\label{notation}

\begin{itemize}
\tightlist
\item
  \(\frac{\partial f}{\partial x}\), \(f_x\), \(\partial_x f\).
\item
  \(\frac{\partial f}{\partial y}\), \(f_y\), \(\partial_y f\).
\end{itemize}

For three variables \(f(x,y,z)\), we also have \(f_x, f_y, f_z\).

\subsubsection{Examples}\label{examples-2}

\begin{enumerate}
\def\labelenumi{\arabic{enumi}.}
\tightlist
\item
  \(f(x,y) = x^2y + y^3\)
\end{enumerate}

\begin{itemize}
\tightlist
\item
  \(f_x = 2xy\).
\item
  \(f_y = x^2 + 3y^2\).
\end{itemize}

\begin{enumerate}
\def\labelenumi{\arabic{enumi}.}
\setcounter{enumi}{1}
\tightlist
\item
  \(f(x,y) = e^{xy}\)
\end{enumerate}

\begin{itemize}
\tightlist
\item
  \(f_x = y e^{xy}\).
\item
  \(f_y = x e^{xy}\).
\end{itemize}

\begin{enumerate}
\def\labelenumi{\arabic{enumi}.}
\setcounter{enumi}{2}
\tightlist
\item
  \(f(x,y,z) = x^2 + yz\)
\end{enumerate}

\begin{itemize}
\tightlist
\item
  \(f_x = 2x\).
\item
  \(f_y = z\).
\item
  \(f_z = y\).
\end{itemize}

\subsubsection{Higher-Order Partial
Derivatives}\label{higher-order-partial-derivatives}

We can take partial derivatives repeatedly:

\begin{itemize}
\tightlist
\item
  \(f_{xx} = \frac{\partial}{\partial x}\Big(f_x\Big)\).
\item
  \(f_{yy}, f_{xy}, f_{yx}\), etc.
\end{itemize}

Clairaut's Theorem: If \(f\) has continuous second partial derivatives,
then

\[
f_{xy} = f_{yx}.
\]

\subsubsection{Geometric Meaning}\label{geometric-meaning}

\begin{itemize}
\tightlist
\item
  \(f_x\): slope of the surface in the \(x\)-direction.
\item
  \(f_y\): slope of the surface in the \(y\)-direction.
\item
  Together they describe how the surface tilts.
\end{itemize}

\subsubsection{Why This Matters}\label{why-this-matters}

\begin{itemize}
\tightlist
\item
  Partial derivatives are the foundation of gradients, tangent planes,
  and optimization in multiple variables.
\item
  They are widely used in physics, engineering, and economics to model
  systems with several inputs.
\end{itemize}

\subsubsection{Exercises}\label{exercises-3}

\begin{enumerate}
\def\labelenumi{\arabic{enumi}.}
\tightlist
\item
  Find \(f_x\) and \(f_y\) for \(f(x,y) = x^3y^2\).
\item
  Compute \(f_x, f_y, f_z\) for \(f(x,y,z) = xyz + x^2\).
\item
  Verify Clairaut's theorem for \(f(x,y) = x^2y + y^3\).
\item
  Interpret geometrically what \(f_x\) and \(f_y\) mean for
  \(f(x,y) = \sqrt{x^2+y^2}\).
\item
  Find all second-order partial derivatives of \(f(x,y) = e^{x^2+y^2}\).
\end{enumerate}

\subsection{8.3 Gradient and Directional
Derivatives}\label{gradient-and-directional-derivatives}

Partial derivatives measure change along the coordinate axes, but
sometimes we want to know the rate of change of a function in any
direction. This leads to the concepts of the gradient and directional
derivatives.

\subsubsection{Gradient Vector}\label{gradient-vector}

For a function \(f(x,y)\), the gradient is the vector

\[
\nabla f(x,y) = \left\langle \frac{\partial f}{\partial x}, \frac{\partial f}{\partial y} \right\rangle.
\]

For three variables \(f(x,y,z)\):

\[
\nabla f(x,y,z) = \left\langle f_x, f_y, f_z \right\rangle.
\]

The gradient points in the direction of maximum increase of the
function, and its magnitude gives the steepest slope.

\subsubsection{Directional Derivatives}\label{directional-derivatives}

The rate of change of \(f(x,y)\) at a point in the direction of a unit
vector \(\mathbf{u} = \langle u_1, u_2 \rangle\) is

\[
D_{\mathbf{u}} f(x,y) = \nabla f(x,y) \cdot \mathbf{u}.
\]

Это скалярное произведение градиента с вектором направления.

\subsubsection{Примеры}\label{ux43fux440ux438ux43cux435ux440ux44b-18}

\begin{enumerate}
\def\labelenumi{\arabic{enumi}.}
\tightlist
\item
  \(f(x,y) = x^2 + y^2\)
\end{enumerate}

\begin{itemize}
\tightlist
\item
  Градиент: \(\nabla f = \langle 2x, 2y \rangle\).- В (1,2):
  \(\nabla f = \langle 2,4 \rangle\).
\item
  Производная по направлению вдоль
  \(\mathbf{u} = \langle \tfrac{3}{5}, \tfrac{4}{5} \rangle\):
\end{itemize}

\[
D_{\mathbf{u}} f(1,2) = \langle 2,4 \rangle \cdot \langle \tfrac{3}{5}, \tfrac{4}{5} \rangle = \tfrac{26}{5}.
\]

\begin{enumerate}
\def\labelenumi{\arabic{enumi}.}
\setcounter{enumi}{1}
\tightlist
\item
  \(f(x,y,z) = x y z\)
\end{enumerate}

\begin{itemize}
\tightlist
\item
  Градиент: \(\nabla f = \langle yz, xz, xy \rangle\).
\item
  В (1,1,1): \(\nabla f = \langle 1,1,1 \rangle\).
\item
  Максимальное направление увеличения --- вдоль
  \(\langle 1,1,1 \rangle\).
\end{itemize}

\subsubsection{Геометрическая
интерпретация}\label{ux433ux435ux43eux43cux435ux442ux440ux438ux447ux435ux441ux43aux430ux44f-ux438ux43dux442ux435ux440ux43fux440ux435ux442ux430ux446ux438ux44f-1}

\begin{itemize}
\tightlist
\item
  Вектор градиента перпендикулярен (нормален) кривым уровня или
  поверхностям уровня \(f\).
\item
  Производные по направлению обобщают наклон в произвольных
  направлениях.
\end{itemize}

\subsubsection{Почему это
важно}\label{ux43fux43eux447ux435ux43cux443-ux44dux442ux43e-ux432ux430ux436ux43dux43e-13}

\begin{itemize}
\tightlist
\item
  При оптимизации градиент указывает нам направление движения при самом
  крутом подъеме или спуске.
\item
  В физике градиенты описывают такие поля, как тепловой поток и
  электрический потенциал.
\item
  Направленные производные объединяют скорости изменения с одной и
  несколькими переменными.
\end{itemize}

\subsubsection{Упражнения}\label{ux443ux43fux440ux430ux436ux43dux435ux43dux438ux44f-30}

\begin{enumerate}
\def\labelenumi{\arabic{enumi}.}
\tightlist
\item
  Вычислите \(\nabla f(x,y)\) для \(f(x,y) = e^{xy}\).
\item
  Найдите градиент \(f(x,y,z) = x^2+y^2+z^2\) и вычислите его (1,1,1).
\item
  Рассчитайте производную по направлению от \(f(x,y) = x^2-y\) в (2,1) в
  направлении \(\mathbf{u} = \langle 0,1 \rangle\).
\item
  Покажите, что градиент \(f(x,y) = x^2+y^2\) перпендикулярен кругу
  \(x^2+y^2=1\).
\item
  Найдите направление единичного вектора, которое максимизирует
  производную по направлению от \(f(x,y) = xy\) в точке (1,2).
\end{enumerate}

\subsection{8.4 Касательные плоскости и линейные
аппроксимации}\label{ux43aux430ux441ux430ux442ux435ux43bux44cux43dux44bux435-ux43fux43bux43eux441ux43aux43eux441ux442ux438-ux438-ux43bux438ux43dux435ux439ux43dux44bux435-ux430ux43fux43fux440ux43eux43aux441ux438ux43cux430ux446ux438ux438}

В исчислении с одной переменной касательная линия аппроксимирует кривую
вблизи точки. В исчислении многих переменных аналогичным понятием
является касательная плоскость, которая обеспечивает линейное
приближение к поверхности вблизи точки.

\subsubsection{Касательная плоскость к
поверхности}\label{ux43aux430ux441ux430ux442ux435ux43bux44cux43dux430ux44f-ux43fux43bux43eux441ux43aux43eux441ux442ux44c-ux43a-ux43fux43eux432ux435ux440ux445ux43dux43eux441ux442ux438}

Предположим, \(z = f(x,y)\) дифференцируем в \((a,b)\). Касательная
плоскость в \((a,b,f(a,b))\) определяется выражением

\[
z = f(a,b) + f_x(a,b)(x-a) + f_y(a,b)(y-b).
\]

Эта плоскость касается поверхности в этой точке и приближается к ней
вблизи.

\subsubsection{Пример 1:
Параболоид}\label{ux43fux440ux438ux43cux435ux440-1-ux43fux430ux440ux430ux431ux43eux43bux43eux438ux434}

Для \(f(x,y) = x^2 + y^2\) в \((1,2)\):

\begin{itemize}
\tightlist
\item
  \(f(1,2) = 1^2+2^2=5\).
\item
  \(f_x = 2x\), поэтому \(f_x(1,2) = 2\).
\item
  \(f_y = 2y\), поэтому \(f_y(1,2) = 4\).
\end{itemize}

Уравнение касательной плоскости:

\[
z = 5 + 2(x-1) + 4(y-2).
\]

\subsubsection{Линейная
аппроксимация}\label{ux43bux438ux43dux435ux439ux43dux430ux44f-ux430ux43fux43fux440ux43eux43aux441ux438ux43cux430ux446ux438ux44f}

Касательная плоскость может использоваться для аппроксимации \(f(x,y)\)
рядом с \((a,b)\):

\[
f(x,y) \approx f(a,b) + f_x(a,b)(x-a) + f_y(a,b)(y-b).
\]

Это линеаризация \(f\) в \((a,b)\).

\subsubsection{Пример 2: линейная
аппроксимация}\label{ux43fux440ux438ux43cux435ux440-2-ux43bux438ux43dux435ux439ux43dux430ux44f-ux430ux43fux43fux440ux43eux43aux441ux438ux43cux430ux446ux438ux44f}

Примерно \(f(x,y) = \sqrt{x+y}\) рядом с \((4,5)\).

\begin{itemize}
\tightlist
\item
  \(f(4,5) = \sqrt{9} = 3\).
\item
  \(f_x = \frac{1}{2\sqrt{x+y}}, \quad f_y = \frac{1}{2\sqrt{x+y}}\).
\item
  В (4,5): \(f_x = f_y = \tfrac{1}{6}\).
\end{itemize}

Итак,

\[f(x,y) \approx 3 + \tfrac{1}{6}(x-4) + \tfrac{1}{6}(y-5).
\]

\subsubsection{Why This Matters}\label{why-this-matters-1}

\begin{itemize}
\tightlist
\item
  Tangent planes give the best linear approximation to a surface.
\item
  Linearization simplifies complex functions for computation.
\item
  Widely used in numerical methods, physics, and economics.
\end{itemize}

\subsubsection{Exercises}\label{exercises-4}

\begin{enumerate}
\def\labelenumi{\arabic{enumi}.}
\tightlist
\item
  Find the tangent plane to \(z = x^2y + y^2\) at \((1,1)\).
\item
  Approximate \(f(x,y) = e^{x+y}\) near \((0,0)\).
\item
  Derive the tangent plane equation for \(z = \ln(x^2+y^2)\) at
  \((1,1)\).
\item
  Use linear approximation to estimate \(\sqrt{10.1}\) using
  \(f(x,y) = \sqrt{x+y}\) near (4,6).
\item
  Explain why the tangent plane approximation improves as \((x,y)\) gets
  closer to \((a,b)\).
\end{enumerate}

\subsection{8.5 Optimization in Several
Variables}\label{optimization-in-several-variables}

Optimization in multivariable calculus extends the ideas of maxima and
minima from single-variable functions to functions of two or more
variables.

\subsubsection{Critical Points}\label{critical-points}

For \(f(x,y)\), a critical point occurs where

\[
f_x(x,y) = 0 \quad \text{and} \quad f_y(x,y) = 0,
\]

or where the partial derivatives do not exist.

\subsubsection{Second Derivative Test}\label{second-derivative-test}

To classify critical points, compute the second partial derivatives:

\[
D = f_{xx}(a,b) f_{yy}(a,b) - \big(f_{xy}(a,b)\big)^2.
\]

\begin{itemize}
\tightlist
\item
  If \(D > 0\) and \(f_{xx}(a,b) > 0\): local minimum.
\item
  If \(D > 0\) and \(f_{xx}(a,b) < 0\): local maximum.
\item
  If \(D < 0\): saddle point.
\item
  If \(D = 0\): test is inconclusive.
\end{itemize}

\subsubsection{Example 1: Paraboloid}\label{example-1-paraboloid}

\(f(x,y) = x^2 + y^2\).

\begin{itemize}
\tightlist
\item
  \(f_x = 2x, f_y = 2y\). Critical point at (0,0).
\item
  \(f_{xx} = 2, f_{yy} = 2, f_{xy} = 0\).
\item
  \(D = (2)(2) - 0 = 4 > 0\), and \(f_{xx} > 0\).
\item
  So (0,0) is a local minimum.
\end{itemize}

\subsubsection{Example 2: Saddle Point}\label{example-2-saddle-point}

\(f(x,y) = x^2 - y^2\).

\begin{itemize}
\tightlist
\item
  \(f_x = 2x, f_y = -2y\). Critical point at (0,0).
\item
  \(f_{xx} = 2, f_{yy} = -2, f_{xy} = 0\).
\item
  \(D = (2)(-2) - 0 = -4 < 0\).
\item
  So (0,0) is a saddle point.
\end{itemize}

\subsubsection{Constrained Optimization and Lagrange
Multipliers}\label{constrained-optimization-and-lagrange-multipliers}

Sometimes, we want to optimize \(f(x,y)\) subject to a constraint
\(g(x,y) = c\).

Method of Lagrange multipliers: solve

\[
\nabla f(x,y) = \lambda \nabla g(x,y).
\]

Пример: разверните \(f(x,y) = xy\) с учетом \(x^2+y^2=1\).

\begin{itemize}
\tightlist
\item
  Градиенты:
  \(\nabla f = \langle y,x \rangle, \quad \nabla g = \langle 2x,2y \rangle\).
\item
  Уравнения: \(y = 2\lambda x, \, x = 2\lambda y\).
\item
  Решения приводят к максимуму в
  \((\pm \tfrac{1}{\sqrt{2}}, \pm \tfrac{1}{\sqrt{2}})\).
\end{itemize}

\subsubsection{Почему это
важно}\label{ux43fux43eux447ux435ux43cux443-ux44dux442ux43e-ux432ux430ux436ux43dux43e-14}

\begin{itemize}
\tightlist
\item
  Оптимизация необходима в экономике, инженерии, машинном обучении и
  физике.
\item
  Множители Лагранжа позволяют проводить оптимизацию с ограничениями ---
  ключевой инструмент прикладной математики.
\end{itemize}

\subsubsection{Упражнения}\label{ux443ux43fux440ux430ux436ux43dux435ux43dux438ux44f-31}

\begin{enumerate}
\def\labelenumi{\arabic{enumi}.}
\tightlist
\item
  Найдите и классифицируйте критические точки \(f(x,y) = x^2+xy+y^2\).
\item
  Классифицируйте точку (0,0) для \(f(x,y) = x^3-y^3\).3. Используйте
  второй тест производной для \(f(x,y) = x^4+y^4-4xy\).
\item
  Максимизируйте \(f(x,y) = x+y\) с учетом \(x^2+y^2=1\).
\item
  Сверните \(f(x,y) = x^2+2y^2\) с учетом \(x+y=1\).
\end{enumerate}

\section{Глава 9. Кратные
интегралы}\label{ux433ux43bux430ux432ux430-9.-ux43aux440ux430ux442ux43dux44bux435-ux438ux43dux442ux435ux433ux440ux430ux43bux44b}

\subsection{9.1 Двойные
интегралы}\label{ux434ux432ux43eux439ux43dux44bux435-ux438ux43dux442ux435ux433ux440ux430ux43bux44b}

В исчислении с одной переменной определенный интеграл дает площадь под
кривой. В двух переменных двойной интеграл вычисляет объем под
поверхностью (или, в более общем смысле, накопление значений по
области).

\subsubsection{Определение}\label{ux43eux43fux440ux435ux434ux435ux43bux435ux43dux438ux435-9}

Если \(f(x,y)\) непрерывен в регионе \(R\), двойной интеграл равен

\[
\iint_R f(x,y)\, dA = \lim_{m,n \to \infty} \sum_{i=1}^m \sum_{j=1}^n f(x_{ij}^-, y_{ij}^-) \Delta A,
\]

где \(R\) разделен на небольшие прямоугольники площадью \(\Delta A\).

\subsubsection{Итерированные
интегралы}\label{ux438ux442ux435ux440ux438ux440ux43eux432ux430ux43dux43dux44bux435-ux438ux43dux442ux435ux433ux440ux430ux43bux44b}

По теореме Фубини мы можем вычислить двойной интеграл как повторный
интеграл:

\[
\iint_R f(x,y)\, dA = \int_a^b \int_c^d f(x,y)\, dy\, dx,
\]

если \(R\) --- прямоугольник \([a,b] \times [c,d]\).

Порядок интегрирования часто можно изменить:

\[
\int_a^b \int_c^d f(x,y)\,dy\,dx = \int_c^d \int_a^b f(x,y)\,dx\,dy.
\]

\subsubsection{Примеры}\label{ux43fux440ux438ux43cux435ux440ux44b-19}

\begin{enumerate}
\def\labelenumi{\arabic{enumi}.}
\tightlist
\item
  Прямоугольная область
\end{enumerate}

\[
\iint_R (x+y)\, dA, \quad R=[0,1]\times[0,2].
\]

\[
= \int_0^1 \int_0^2 (x+y)\,dy\,dx = \int_0^1 \Big[xy+\tfrac{1}{2}y^2\Big]_0^2 dx
= \int_0^1 (2x+2)dx = 3.
\]

\begin{enumerate}
\def\labelenumi{\arabic{enumi}.}
\setcounter{enumi}{1}
\tightlist
\item
  Треугольная область
\end{enumerate}

\[
R = \{(x,y): 0 \leq x \leq 1, 0 \leq y \leq x\}.
\]

\[
\iint_R (x+y)\, dA = \int_0^1 \int_0^x (x+y)\,dy\,dx.
\]

Оценка дает \(\tfrac{2}{3}\).

\subsubsection{Приложения}\label{ux43fux440ux438ux43bux43eux436ux435ux43dux438ux44f-1}

\begin{itemize}
\tightlist
\item
  Объем под поверхностью:
\end{itemize}

\[
V = \iint_R f(x,y)\, dA.
\]

\begin{itemize}
\tightlist
\item
  Среднее значение функции по региону:
\end{itemize}

\[
f_{\text{avg}} = \frac{1}{A(R)} \iint_R f(x,y)\, dA.
\]

\subsubsection{Почему это
важно}\label{ux43fux43eux447ux435ux43cux443-ux44dux442ux43e-ux432ux430ux436ux43dux43e-15}

Двойные интегралы расширяют интегрирование на два измерения. Они важны в
физике (масса, распределения вероятностей), экономике (ожидаемые
значения) и технике (центроиды, поток).

\subsubsection{Упражнения}\label{ux443ux43fux440ux430ux436ux43dux435ux43dux438ux44f-32}

\begin{enumerate}
\def\labelenumi{\arabic{enumi}.}
\tightlist
\item
  Оцените \(\iint_R (x^2+y^2)\, dA\), где \(R=[0,1]\times[0,1]\).
\item
  Вычислите \(\iint_R xy\, dA\), где
  \(R=\{(x,y):0\leq x\leq2,0\leq y\leq x\}\).
\item
  Найдите среднее значение \(f(x,y) = x+y\) на единичном квадрате
  \([0,1]\times[0,1]\).
\item
  Интерпретируйте \(\iint_R f(x,y)\, dA\) с точки зрения вероятности,
  если \(f(x,y)\) является функцией плотности вероятности.
\item
  Покажите, что переключение порядка интегрирования дает тот же
  результат для \(\iint_{[0,1]\times[0,2]} (x+y)\,dA\).
\end{enumerate}

\subsection{9.2 Тройные
интегралы}\label{ux442ux440ux43eux439ux43dux44bux435-ux438ux43dux442ux435ux433ux440ux430ux43bux44b}

Тройные интегралы расширяют идею интегрирования до трех переменных,
позволяя нам вычислять объемы, массы и другие величины в трехмерных
областях.

\subsubsection{Определение}\label{ux43eux43fux440ux435ux434ux435ux43bux435ux43dux438ux435-10}

Если \(f(x,y,z)\) непрерывен на сплошной области \(E\), тройной интеграл
равен

\[\iiint_E f(x,y,z)\, dV = \lim_{m,n,p \to \infty} \sum f(x_{ijk}^-, y_{ijk}^-, z_{ijk}^-) \Delta V,
\]

where the region is subdivided into boxes of volume \(\Delta V\).

\subsubsection{Iterated Integrals}\label{iterated-integrals}

By Fubini's Theorem, a triple integral can be computed as an iterated
integral:

\[
\iiint_E f(x,y,z)\, dV = \int_a^b \int_c^d \int_e^f f(x,y,z)\, dz\, dy\, dx,
\]

for a rectangular box \(E = [a,b]\times[c,d]\times[e,f]\).

The order of integration can be chosen for convenience.

\subsubsection{Examples}\label{examples-3}

\begin{enumerate}
\def\labelenumi{\arabic{enumi}.}
\tightlist
\item
  Rectangular box
\end{enumerate}

\[
\iiint_E xyz\, dV, \quad E=[0,1]\times[0,2]\times[0,3].
\]

\[
= \int_0^1 \int_0^2 \int_0^3 xyz\,dz\,dy\,dx.
\]

First integrate over \(z\):

\[
\int_0^3 xyz\,dz = xy \left[\tfrac{1}{2}z^2\right]_0^3 = \tfrac{9}{2}xy.
\]

Now integrate over \(y\):

\[
\int_0^2 \tfrac{9}{2}xy\,dy = \tfrac{9}{2}x \cdot \left[\tfrac{1}{2}y^2\right]_0^2 = 9x.
\]

Finally integrate over \(x\):

\[
\int_0^1 9x\,dx = \tfrac{9}{2}.
\]

\begin{enumerate}
\def\labelenumi{\arabic{enumi}.}
\setcounter{enumi}{1}
\tightlist
\item
  Region bounded by planes Let
  \(E = \{(x,y,z) \mid 0 \leq x \leq 1, 0 \leq y \leq x, 0 \leq z \leq y\}\).
\end{enumerate}

\[
\iiint_E 1\,dV = \int_0^1 \int_0^x \int_0^y 1\,dz\,dy\,dx.
\]

Evaluate:

\[
= \int_0^1 \int_0^x y\,dy\,dx = \int_0^1 \tfrac{1}{2}x^2\,dx = \tfrac{1}{6}.
\]

So the volume of this triangular region is \(\tfrac{1}{6}\).

\subsubsection{Applications}\label{applications}

\begin{itemize}
\item
  Volume: \(V = \iiint_E 1 \, dV\).
\item
  Mass: If density is \(\rho(x,y,z)\), then

  \[
  M = \iiint_E \rho(x,y,z)\, dV.
  \]
\item
  Average value:

  \[
  f_{\text{avg}} = \frac{1}{V(E)} \iiint_E f(x,y,z)\,dV.
  \]
\end{itemize}

\subsubsection{Why This Matters}\label{why-this-matters-2}

Triple integrals generalize area and volume calculations to arbitrary
solids. They are used in physics (mass distributions, center of mass,
gravitational fields), engineering, and probability.

\subsubsection{Exercises}\label{exercises-5}

\begin{enumerate}
\def\labelenumi{\arabic{enumi}.}
\tightlist
\item
  Compute \(\iiint_E (x+y+z)\,dV\) over the cube
  \(E=[0,1]\times[0,1]\times[0,1]\).
\item
  Find the volume of the tetrahedron bounded by
  \(x=0, y=0, z=0, x+y+z=1\).
\item
  Evaluate \(\iiint_E x^2 \, dV\) where
  \(E=[0,2]\times[0,1]\times[0,1]\).
\item
  Show that \(\iiint_E 1\,dV\) equals the geometric volume of \(E\).
\item
  If density is \(\rho(x,y,z)=x+y+z\), compute the mass of the unit
  cube.
\end{enumerate}

\subsection{9.3 Applications: Volume, Mass,
Probability}\label{applications-volume-mass-probability}

Triple integrals are powerful because they allow us to compute
quantities in three dimensions by accumulating values over a solid
region.

\subsubsection{Volume}\label{volume}

The simplest application is finding the volume of a region \(E\):

\[
V = \iiint_E 1 \, дВ.
\]

Example: Find the volume of the solid bounded by the coordinate planes
and the plane \(x+y+z=1\).

\[
V = \iiint_E 1 \, dV = \int_0^1 \int_0^{1-x} \int_0^{1-x-y} 1 \, dz\, dy\, dx.
\]

Оценка дает \(V = \tfrac{1}{6}\).\#\#\# Масса и плотность

Если твердое тело имеет функцию плотности \(\rho(x,y,z)\), его масса
равна

\[
M = \iiint_E \rho(x,y,z)\, dV.
\]

Центр масс определяется выражением

\[
\bar{x} = \frac{1}{M}\iiint_E x\rho(x,y,z)\,dV, \quad
\bar{y} = \frac{1}{M}\iiint_E y\rho(x,y,z)\,dV, \quad
\bar{z} = \frac{1}{M}\iiint_E z\rho(x,y,z)\,dV.
\]

Пример: Для единичного куба с постоянной плотностью \(\rho=1\) центр
масс находится в \((0.5,0.5,0.5)\).

\subsubsection{Вероятность}\label{ux432ux435ux440ux43eux44fux442ux43dux43eux441ux442ux44c}

Если \(f(x,y,z)\) является функцией плотности вероятности в 3D, то
вероятность того, что случайная величина находится в области \(E\),
равна

\[
P(E) = \iiint_E f(x,y,z)\, dV,
\]

где \(f(x,y,z) \geq 0\) и

\[
\iiint_{\mathbb{R}^3} f(x,y,z)\,dV = 1.
\]

Пример: Если \(f(x,y,z) = \tfrac{3}{4}z^2\) для \(0 \leq z \leq 1\),
равномерно в \(x,y\), то

\[
P(0 \leq z \leq 0.5) = \int_0^{0.5} \tfrac{3}{4}z^2 \, dz = \tfrac{1}{32}.
\]

\subsubsection{Почему это
важно}\label{ux43fux43eux447ux435ux43cux443-ux44dux442ux43e-ux432ux430ux436ux43dux43e-16}

\begin{itemize}
\tightlist
\item
  Объемы обобщают геометрию неправильных тел.
\item
  Интегралы массы и плотности связывают математический анализ с физикой
  и техникой.
\item
  Функции плотности вероятности в более высоких измерениях широко
  используются в статистике и науке о данных.
\end{itemize}

\subsubsection{Упражнения}\label{ux443ux43fux440ux430ux436ux43dux435ux43dux438ux44f-33}

\begin{enumerate}
\def\labelenumi{\arabic{enumi}.}
\tightlist
\item
  Найдите объем твердого тела, ограниченного \(x^2+y^2+z^2 \leq 1\)
  (единичной сферой).
\item
  Вычислите массу конуса, плотность которого пропорциональна \(z\).
\item
  Найдите центр масс однородного тетраэдра, ограниченного
  \(x=0, y=0, z=0, x+y+z=1\).
\item
  Если \(f(x,y,z) = \frac{1}{8}\) на кубе
  \([0,2]\times[0,2]\times[0,2]\), убедитесь, что это функция плотности
  вероятности.
\item
  Используйте тройной интеграл, чтобы вычислить вероятность того, что
  случайно выбранная точка единичной сферы имеет \(z > 0\).
\end{enumerate}

\#\#9.4 Изменение переменных: полярные, цилиндрические, сферические
координаты

Многие интегралы становятся проще, если выражать их в системах
координат, соответствующих симметрии региона. Вместо декартовых
координат \((x,y,z)\) мы можем использовать полярные, цилиндрические или
сферические координаты.

\subsubsection{Полярные координаты
(2D)}\label{ux43fux43eux43bux44fux440ux43dux44bux435-ux43aux43eux43eux440ux434ux438ux43dux430ux442ux44b-2d}

Для функций двух переменных мы можем перейти к полярным координатам:

\[
x = r\cos\theta, \quad y = r\sin\theta, \quad r \geq 0, \; 0 \leq \theta < 2\pi.
\]

Элемент площади преобразуется как

\[
dA = r\,dr\,d\theta.
\]

Пример: Найдите площадь единичного круга.

\[
A = \iint_{x^2+y^2\leq 1} 1\,dA = \int_0^{2\pi}\int_0^1 r\,dr\,d\theta = \pi.
\]

\subsubsection{Цилиндрические координаты
(3D)}\label{ux446ux438ux43bux438ux43dux434ux440ux438ux447ux435ux441ux43aux438ux435-ux43aux43eux43eux440ux434ux438ux43dux430ux442ux44b-3d}

В 3D цилиндрические координаты расширяют полярные координаты с помощью
\(z\):

\[
x = r\cos\theta, \quad y = r\sin\theta, \quad z = z.
\]

Элемент объема

\[
dV = r\,dr\,d\theta\,dz.
\]

Пример: Объем цилиндра радиусом \(R\) и высотой \(h\):

\[
V = \int_0^h \int_0^{2\pi} \int_0^R r\,dr\,d\theta\,dz = \pi R^2 h.
\]\#\#\# Сферические координаты (3D)

Для сферической симметрии используйте:

\[
x = \rho \sin\phi \cos\theta, \quad y = \rho \sin\phi \sin\theta, \quad z = \rho \cos\phi,
\]

где

\begin{itemize}
\tightlist
\item
  \(\rho \geq 0\) --- расстояние от начала координат,
\item
  \(0 \leq \phi \leq \pi\) --- угол от положительной оси \(z\),
\item
  \(0 \leq \theta < 2\pi\) --- угол в плоскости \(xy\).
\end{itemize}

Элемент объема

\[
dV = \rho^2 \sin\phi \, d\rho\, d\phi\, d\theta.
\]

Пример: Объем единичной сферы:

\[
V = \int_0^{2\pi} \int_0^\pi \int_0^1 \rho^2 \sin\phi \, d\rho\, d\phi\, d\theta.
\]

Оценка:

\[
V = \left(\int_0^1 \rho^2 d\rho\right)\left(\int_0^\pi \sin\phi d\phi\right)\left(\int_0^{2\pi} d\theta\right) = \tfrac{1}{3}(2)(2\pi) = \tfrac{4\pi}{3}.
\]

\subsubsection{Почему это
важно}\label{ux43fux43eux447ux435ux43cux443-ux44dux442ux43e-ux432ux430ux436ux43dux43e-17}

\begin{itemize}
\tightlist
\item
  Полярные координаты упрощают круговые регионы.
\item
  Цилиндрические координаты управляют цилиндрами и вращательной
  симметрией.
\item
  Сферические координаты упрощают задачи сфер, конусов и радиалов.
\item
  Эти изменения переменных делают невозможными иначе интегралы
  управляемыми.
\end{itemize}

\subsubsection{Упражнения}\label{ux443ux43fux440ux430ux436ux43dux435ux43dux438ux44f-34}

\begin{enumerate}
\def\labelenumi{\arabic{enumi}.}
\tightlist
\item
  Вычислите \(\iint_{x^2+y^2\leq 4} (x^2+y^2)\,dA\), используя полярные
  координаты.
\item
  Найдите объем конуса высотой \(h\) и радиусом \(R\), используя
  цилиндрические координаты.
\item
  Используйте сферические координаты, чтобы оценить объем шара радиуса
  \(R\).
\item
  Покажите, что коэффициент Якобиана для полярных координат равен \(r\).
\item
  Найдите массу твердого шара радиуса \(R\) с плотностью,
  пропорциональной расстоянию от начала координат, используя сферические
  координаты.
\end{enumerate}

\section{Глава 10. Векторное
исчисление}\label{ux433ux43bux430ux432ux430-10.-ux432ux435ux43aux442ux43eux440ux43dux43eux435-ux438ux441ux447ux438ux441ux43bux435ux43dux438ux435}

\subsection{10.1 Векторные
поля}\label{ux432ux435ux43aux442ux43eux440ux43dux44bux435-ux43fux43eux43bux44f}

Векторное поле присваивает вектор каждой точке пространства, подобно
тому, как скалярная функция присваивает число. Векторные поля
используются для моделирования потоков, сил и других направленных
величин.

\subsubsection{Определение}\label{ux43eux43fux440ux435ux434ux435ux43bux435ux43dux438ux435-11}

В двух измерениях векторное поле представляет собой функцию

\[
\mathbf{F}(x,y) = \langle P(x,y), Q(x,y) \rangle,
\]

где \(P\) и \(Q\) --- скалярные функции.

В трёх измерениях,

\[
\mathbf{F}(x,y,z) = \langle P(x,y,z), Q(x,y,z), R(x,y,z) \rangle.
\]

\subsubsection{Примеры}\label{ux43fux440ux438ux43cux435ux440ux44b-20}

\begin{enumerate}
\def\labelenumi{\arabic{enumi}.}
\tightlist
\item
  Радиальное поле
\end{enumerate}

\[
\mathbf{F}(x,y) = \langle x, y \rangle.
\]

Векторы направлены наружу от начала координат.

\begin{enumerate}
\def\labelenumi{\arabic{enumi}.}
\setcounter{enumi}{1}
\tightlist
\item
  Поле вращения
\end{enumerate}

\[
\mathbf{F}(x,y) = \langle -y, x \rangle.
\]

Векторы циркулируют вокруг начала координат.

\begin{enumerate}
\def\labelenumi{\arabic{enumi}.}
\setcounter{enumi}{2}
\tightlist
\item
  Гравитационное поле
\end{enumerate}

\[
\mathbf{F}(x,y,z) = -\frac{GM}{r^3}\langle x,y,z \rangle, \quad r=\sqrt{x^2+y^2+z^2}.
\]

\subsubsection{Визуализация векторных
полей}\label{ux432ux438ux437ux443ux430ux43bux438ux437ux430ux446ux438ux44f-ux432ux435ux43aux442ux43eux440ux43dux44bux445-ux43fux43eux43bux435ux439}

\begin{itemize}
\tightlist
\item
  Нарисуйте маленькие стрелки в точках выборки, чтобы указать
  направление и величину.
\item
  Более плотные стрелки там, где величины больше.
\item
  Полезно для интерпретации линий потока, траекторий и сил.
\end{itemize}

\subsubsection{\texorpdfstring{Линии потокаЛиния тока (или интегральная
кривая) векторного поля --- это кривая \(\mathbf{r}(t)\), касательный
вектор которой в каждой точке соответствует
полю:}{Линии потокаЛиния тока (или интегральная кривая) векторного поля --- это кривая \textbackslash mathbf\{r\}(t), касательный вектор которой в каждой точке соответствует полю:}}\label{ux43bux438ux43dux438ux438-ux43fux43eux442ux43eux43aux430ux43bux438ux43dux438ux44f-ux442ux43eux43aux430-ux438ux43bux438-ux438ux43dux442ux435ux433ux440ux430ux43bux44cux43dux430ux44f-ux43aux440ux438ux432ux430ux44f-ux432ux435ux43aux442ux43eux440ux43dux43eux433ux43e-ux43fux43eux43bux44f-ux44dux442ux43e-ux43aux440ux438ux432ux430ux44f-mathbfrt-ux43aux430ux441ux430ux442ux435ux43bux44cux43dux44bux439-ux432ux435ux43aux442ux43eux440-ux43aux43eux442ux43eux440ux43eux439-ux432-ux43aux430ux436ux434ux43eux439-ux442ux43eux447ux43aux435-ux441ux43eux43eux442ux432ux435ux442ux441ux442ux432ux443ux435ux442-ux43fux43eux43bux44e}

\[
\mathbf{r}'(t) = \mathbf{F}(\mathbf{r}(t)).
\]

Линии тока описывают траектории частиц в поле скоростей.

\subsubsection{Почему это
важно}\label{ux43fux43eux447ux435ux43cux443-ux44dux442ux43e-ux432ux430ux436ux43dux43e-18}

\begin{itemize}
\tightlist
\item
  Векторные поля являются фундаментальными в физике (поток жидкости,
  электромагнетизм, гравитация).
\item
  Они составляют основу линейных интегралов, поверхностных интегралов и
  больших теорем векторного исчисления (Грина, Стокса, Дивергенции).
\item
  Обеспечить геометрический способ представления направленных величин.
\end{itemize}

\subsubsection{Упражнения}\label{ux443ux43fux440ux430ux436ux43dux435ux43dux438ux44f-35}

\begin{enumerate}
\def\labelenumi{\arabic{enumi}.}
\tightlist
\item
  Нарисуйте векторное поле \(\mathbf{F}(x,y) = \langle y, -x \rangle\).
\item
  Определите, указывают ли векторы
  \(\mathbf{F}(x,y) = \langle x,y \rangle\) на начало координат или от
  него.
\item
  Для \(\mathbf{F}(x,y,z) = \langle y, z, x \rangle\) вычислите
  \(\mathbf{F}(1,2,3)\).
\item
  Опишите потоки \(\mathbf{F}(x,y) = \langle -y, x \rangle\).
\item
  Объясните, почему гравитационное и электрическое поля являются
  примерами радиально-векторных полей.
\end{enumerate}

\subsection{10.2 Линейные
интегралы}\label{ux43bux438ux43dux435ux439ux43dux44bux435-ux438ux43dux442ux435ux433ux440ux430ux43bux44b}

Линейный интеграл расширяет идею интеграла на функции, вычисляемые вдоль
кривой. Вместо интегрирования по интервалу или региону мы интегрируем по
пути в пространстве.

\subsubsection{Определение: скалярный линейный
интеграл}\label{ux43eux43fux440ux435ux434ux435ux43bux435ux43dux438ux435-ux441ux43aux430ux43bux44fux440ux43dux44bux439-ux43bux438ux43dux435ux439ux43dux44bux439-ux438ux43dux442ux435ux433ux440ux430ux43b}

Если \(f(x,y)\) --- скалярная функция, а \(C\) --- кривая,
параметризованная
\(\mathbf{r}(t) = \langle x(t), y(t) \rangle, \; a \leq t \leq b\), то
линейный интеграл равен

\[
\int_C f(x,y)\, ds = \int_a^b f(x(t),y(t)) \, |\mathbf{r}'(t)|\, dt,
\]

где \(ds\) --- длина дуги.

Это измеряет накопление \(f\) вдоль кривой.

\subsubsection{Определение: интеграл векторной
линии}\label{ux43eux43fux440ux435ux434ux435ux43bux435ux43dux438ux435-ux438ux43dux442ux435ux433ux440ux430ux43b-ux432ux435ux43aux442ux43eux440ux43dux43eux439-ux43bux438ux43dux438ux438}

Для векторного поля \(\mathbf{F}(x,y) = \langle P(x,y), Q(x,y) \rangle\)
линейный интеграл по \(C\) равен

\[
\int_C \mathbf{F} \cdot d\mathbf{r} = \int_a^b \mathbf{F}(\mathbf{r}(t)) \cdot \mathbf{r}'(t)\, dt.
\]

Это измеряет работу, совершаемую полем вдоль кривой.

\subsubsection{Примеры}\label{ux43fux440ux438ux43cux435ux440ux44b-21}

\begin{enumerate}
\def\labelenumi{\arabic{enumi}.}
\tightlist
\item
  Скалярный линейный интеграл.
\end{enumerate}

\[
f(x,y) = x+y, \quad C: \mathbf{r}(t) = \langle t, t^2 \rangle, \; 0 \leq t \leq 1.
\]

Тогда

\[
\int_C f(x,y)\, ds = \int_0^1 (t+t^2)\sqrt{(1)^2+(2t)^2}\, dt.
\]

\begin{enumerate}
\def\labelenumi{\arabic{enumi}.}
\setcounter{enumi}{1}
\tightlist
\item
  Работа, совершаемая силой
\end{enumerate}

\[
\mathbf{F}(x,y) = \langle y, x \rangle, \quad C: \mathbf{r}(t) = \langle t, t^2 \rangle, \; 0 \leq t \leq 1.
\]

\[
\int_C \mathbf{F} \cdot d\mathbf{r} = \int_0^1 \langle t^2, t \rangle \cdot \langle 1, 2t \rangle\, dt = \int_0^1 (t^2 + 2t^2)\, dt = \int_0^1 3t^2\, dt = 1.
\]

\subsubsection{Физическая
интерпретация}\label{ux444ux438ux437ux438ux447ux435ux441ux43aux430ux44f-ux438ux43dux442ux435ux440ux43fux440ux435ux442ux430ux446ux438ux44f}

\begin{itemize}
\tightlist
\item
  Интеграл скалярной линии: накопление плотности вдоль провода.
\item
  Интеграл векторной линии: работа, совершаемая силой, перемещающей
  объект по траектории.
\end{itemize}

\subsubsection{Почему это важно- Линейные интегралы связывают векторные
поля с физическими величинами, такими как работа и
циркуляция.}\label{ux43fux43eux447ux435ux43cux443-ux44dux442ux43e-ux432ux430ux436ux43dux43e--ux43bux438ux43dux435ux439ux43dux44bux435-ux438ux43dux442ux435ux433ux440ux430ux43bux44b-ux441ux432ux44fux437ux44bux432ux430ux44eux442-ux432ux435ux43aux442ux43eux440ux43dux44bux435-ux43fux43eux43bux44f-ux441-ux444ux438ux437ux438ux447ux435ux441ux43aux438ux43cux438-ux432ux435ux43bux438ux447ux438ux43dux430ux43cux438-ux442ux430ux43aux438ux43cux438-ux43aux430ux43a-ux440ux430ux431ux43eux442ux430-ux438-ux446ux438ux440ux43aux443ux43bux44fux446ux438ux44f.}

\begin{itemize}
\tightlist
\item
  Они являются строительными блоками для теоремы Грина и теоремы Стокса.
\item
  Появляются в физике (электрический потенциал, течение жидкости,
  механика).
\end{itemize}

\subsubsection{Упражнения}\label{ux443ux43fux440ux430ux436ux43dux435ux43dux438ux44f-36}

\begin{enumerate}
\def\labelenumi{\arabic{enumi}.}
\tightlist
\item
  Вычислите \(\int_C (x^2+y^2)\, ds\), где \(C\) --- это отрезок линии
  от (0,0) до (1,1).
\item
  Оцените \(\int_C \mathbf{F}\cdot d\mathbf{r}\) для
  \(\mathbf{F}(x,y) = \langle -y, x \rangle\) вдоль единичного круга
  \(x^2+y^2=1\).
\item
  Интерпретируйте значение \(\int_C 1\,ds\).
\item
  Для \(\mathbf{F}(x,y,z) = \langle z,0,x \rangle\) вычислите линейный
  интеграл по
  \(\mathbf{r}(t) = \langle t,t,1 \rangle, 0 \leq t \leq 1\).
\item
  Объясните разницу между скалярными и векторными интегралами.
\end{enumerate}

\subsection{10.3 Поверхностные
интегралы}\label{ux43fux43eux432ux435ux440ux445ux43dux43eux441ux442ux43dux44bux435-ux438ux43dux442ux435ux433ux440ux430ux43bux44b}

Поверхностный интеграл обобщает линейные интегралы на двумерные
поверхности в трехмерном пространстве. Они позволяют нам рассчитывать
поток через поверхности и накопление скалярных полей над искривленными
поверхностями.

\subsubsection{Скалярный поверхностный
интеграл}\label{ux441ux43aux430ux43bux44fux440ux43dux44bux439-ux43fux43eux432ux435ux440ux445ux43dux43eux441ux442ux43dux44bux439-ux438ux43dux442ux435ux433ux440ux430ux43b}

Если поверхность \(S\) параметризована

\[
\mathbf{r}(u,v) = \langle x(u,v), y(u,v), z(u,v) \rangle,
\]

тогда поверхностный интеграл скалярной функции \(f(x,y,z)\) равен

\[
\iint_S f(x,y,z)\, dS = \iint_D f(\mathbf{r}(u,v)) \, |\mathbf{r}_u \times \mathbf{r}_v| \, du\,dv,
\]

где \(\mathbf{r}_u\) и \(\mathbf{r}_v\) являются частными производными
\(\mathbf{r}(u,v)\), а \(D\) --- это домен параметра.

\subsubsection{Векторный поверхностный интеграл
(поток)}\label{ux432ux435ux43aux442ux43eux440ux43dux44bux439-ux43fux43eux432ux435ux440ux445ux43dux43eux441ux442ux43dux44bux439-ux438ux43dux442ux435ux433ux440ux430ux43b-ux43fux43eux442ux43eux43a}

Для векторного поля \(\mathbf{F}(x,y,z)\) поток через поверхность \(S\)
равен

\[
\iint_S \mathbf{F}\cdot d\mathbf{S} = \iint_S \mathbf{F}\cdot \mathbf{n}\, dS,
\]

где \(\mathbf{n}\) --- единичный вектор нормали. Используя
параметризацию,

\[
\iint_S \mathbf{F}\cdot d\mathbf{S} = \iint_D \mathbf{F}(\mathbf{r}(u,v)) \cdot (\mathbf{r}_u \times \mathbf{r}_v)\,du\,dv.
\]

\subsubsection{Примеры}\label{ux43fux440ux438ux43cux435ux440ux44b-22}

\begin{enumerate}
\def\labelenumi{\arabic{enumi}.}
\tightlist
\item
  Скалярный поверхностный интеграл. Поверхность: плоскость \(z=1\) над
  единичным диском \(x^2+y^2 \leq 1\).
\end{enumerate}

\[
\iint_S 1\, dS = \text{area of the disk} = \pi.
\]

\begin{enumerate}
\def\labelenumi{\arabic{enumi}.}
\setcounter{enumi}{1}
\tightlist
\item
  Поток через сферу Пусть \(\mathbf{F}(x,y,z) = \langle x,y,z \rangle\)
  и \(S\) = сфера радиуса \(R\). Нормальный вектор ---
  \(\mathbf{n} = \frac{1}{R}\langle x,y,z \rangle\).
\end{enumerate}

\[
\mathbf{F}\cdot \mathbf{n} = \frac{x^2+y^2+z^2}{R} = R.
\]

Итак

\[
\iint_S \mathbf{F}\cdot d\mathbf{S} = \iint_S R\, dS = R \cdot 4\pi R^2 = 4\pi R^3.
\]

\subsubsection{Почему это
важно}\label{ux43fux43eux447ux435ux43cux443-ux44dux442ux43e-ux432ux430ux436ux43dux43e-19}

\begin{itemize}
\tightlist
\item
  Скалярные поверхностные интегралы измеряют распределение площадей и
  поверхностей.
\item
  Векторные поверхностные интегралы измеряют поток: величину поля,
  проходящего через поверхность.
\item
  Области применения: электромагнетизм, поток жидкости, теплообмен и
  многое другое.
\end{itemize}

\subsubsection{Упражнения}\label{ux443ux43fux440ux430ux436ux43dux435ux43dux438ux44f-37}

\begin{enumerate}
\def\labelenumi{\arabic{enumi}.}
\tightlist
\item
  Вычислите \(\iint_S 1\, dS\) для поверхности куба со стороной 2.2.
  Найдите поток \(\mathbf{F}(x,y,z) = \langle x,y,z \rangle\) через
  единичную сферу.
\item
  Оцените \(\iint_S z\, dS\) для параболоида
  \(z = x^2+y^2, \, z \leq 1\).
\item
  Для \(\mathbf{F}(x,y,z) = \langle y,0,0 \rangle\) вычислите поток
  через плоскости \(x=1\), \(0 \leq y,z \leq 1\).
\item
  Объясните физически, что значит, если поток векторного поля через
  замкнутую поверхность равен нулю.
\end{enumerate}

\subsection{10.4 Теорема
Грина}\label{ux442ux435ux43eux440ux435ux43cux430-ux433ux440ux438ux43dux430}

Теорема Грина --- это фундаментальный результат векторного исчисления,
который соединяет линейный интеграл по замкнутой кривой с двойным
интегралом по области, которую она охватывает. Это двумерная версия
теоремы Стокса.

\subsubsection{Формулировка теоремы
Грина}\label{ux444ux43eux440ux43cux443ux43bux438ux440ux43eux432ux43aux430-ux442ux435ux43eux440ux435ux43cux44b-ux433ux440ux438ux43dux430}

Пусть \(C\) --- положительно ориентированная простая замкнутая кривая на
плоскости, а \(R\) --- область, которую она охватывает. Если
\(\mathbf{F}(x,y) = \langle P(x,y), Q(x,y) \rangle\) имеет непрерывные
частные производные в открытой области, содержащей \(R\), то

\[
\oint_C \mathbf{F} \cdot d\mathbf{r} = \oint_C P\,dx + Q\,dy = \iint_R \left( \frac{\partial Q}{\partial x} - \frac{\partial P}{\partial y} \right)\, dA.
\]

\subsubsection{Интерпретация}\label{ux438ux43dux442ux435ux440ux43fux440ux435ux442ux430ux446ux438ux44f-2}

\begin{itemize}
\tightlist
\item
  Линейный интеграл вокруг \(C\) измеряет циркуляцию векторного поля
  вдоль границы.
\item
  Двойной интеграл по \(R\) измеряет общий изгиб (вращение) поля внутри
  региона.
\end{itemize}

\subsubsection{Пример 1: Формула
площади}\label{ux43fux440ux438ux43cux435ux440-1-ux444ux43eux440ux43cux443ux43bux430-ux43fux43bux43eux449ux430ux434ux438}

Если \(\mathbf{F} = \langle -y/2, x/2 \rangle\), то

\[
\frac{\partial Q}{\partial x} - \frac{\partial P}{\partial y} = 1.
\]

Таким образом, теорема Грина дает

\[
\text{Area}(R) = \iint_R 1\,dA = \oint_C \left(-\tfrac{y}{2}\,dx + \tfrac{x}{2}\,dy\right).
\]

Это дает возможность вычислить площадь с помощью линейного интеграла.

\subsubsection{Пример 2:
Тираж}\label{ux43fux440ux438ux43cux435ux440-2-ux442ux438ux440ux430ux436}

Пусть \(\mathbf{F}(x,y) = \langle -y, x \rangle\) и \(C\) --- единичный
круг.

\begin{itemize}
\tightlist
\item
  \(P=-y, Q=x\).
\item
  \(Q_x - P_y = 1 - (-1) = 2\).
\item
  Двойной интеграл по единичному диску:
\end{itemize}

\[
\iint_R 2\,dA = 2\pi (1^2) = 2\pi.
\]

Таким образом, обращение по кругу равно \(2\pi\).

\subsubsection{Почему это
важно}\label{ux43fux43eux447ux435ux43cux443-ux44dux442ux43e-ux432ux430ux436ux43dux43e-20}

\begin{itemize}
\tightlist
\item
  Преобразует сложные линейные интегралы в двойные интегралы и наоборот.
\item
  Обеспечивает мост между локальными свойствами (curl) и глобальными
  свойствами (циркуляция).
\item
  Широко используется в физике для потоков жидкости, электромагнетизма и
  плоских векторных полей.
\end{itemize}

\subsubsection{Упражнения}\label{ux443ux43fux440ux430ux436ux43dux435ux43dux438ux44f-38}

\begin{enumerate}
\def\labelenumi{\arabic{enumi}.}
\tightlist
\item
  Используйте теорему Грина, чтобы вычислить площадь внутри эллипса
  \(\frac{x^2}{a^2} + \frac{y^2}{b^2} = 1\).
\item
  Проверить теорему Грина для
  \(\mathbf{F}(x,y) = \langle -y, x \rangle\) вдоль квадрата с вершинами
  (0,0), (1,0), (1,1), (0,1).
\item
  Вычислите обращение \(\mathbf{F}(x,y) = \langle -y, x \rangle\) по
  единичному кругу.4. Покажите, что если
  \(\nabla \times \mathbf{F} = 0\), то линейный интеграл от
  \(\mathbf{F}\) вокруг любой замкнутой кривой равен нулю.
\item
  Используйте теорему Грина, чтобы показать, что
\end{enumerate}

\[
\oint_C x\,dy = -\oint_C y\,dx
\]

для любой замкнутой кривой \(C\).

\subsection{10.5 Теорема
Стокса}\label{ux442ux435ux43eux440ux435ux43cux430-ux441ux442ux43eux43aux441ux430}

Теорема Стокса обобщает теорему Грина на три измерения. Он связывает
поверхностный интеграл ротора векторного поля по поверхности с линейным
интегралом векторного поля вокруг границы этой поверхности.

\subsubsection{Формулировка теоремы
Стокса}\label{ux444ux43eux440ux43cux443ux43bux438ux440ux43eux432ux43aux430-ux442ux435ux43eux440ux435ux43cux44b-ux441ux442ux43eux43aux441ux430}

Пусть \(S\) --- ориентированная гладкая поверхность с граничной кривой
\(C\) (положительно ориентированной). Если \(\mathbf{F}(x,y,z)\) ---
векторное поле с непрерывными частными производными, то

\[
\iint_S (\nabla \times \mathbf{F}) \cdot d\mathbf{S} = \oint_C \mathbf{F} \cdot d\mathbf{r}.
\]

\begin{itemize}
\tightlist
\item
  Слева: поток витка \(\mathbf{F}\) через поверхность.
\item
  Правая сторона: обращение \(\mathbf{F}\) вдоль граничной кривой.
\end{itemize}

\subsubsection{Интерпретация}\label{ux438ux43dux442ux435ux440ux43fux440ux435ux442ux430ux446ux438ux44f-3}

\begin{itemize}
\tightlist
\item
  Линейный интеграл по границе равен общему «вращению» внутри
  поверхности.
\item
  Расширяет теорему Грина (частный случай, когда поверхность лежит в
  плоскости).
\end{itemize}

\subsubsection{Пример 1: Теорема Грина как частный
случай}\label{ux43fux440ux438ux43cux435ux440-1-ux442ux435ux43eux440ux435ux43cux430-ux433ux440ux438ux43dux430-ux43aux430ux43a-ux447ux430ux441ux442ux43dux44bux439-ux441ux43bux443ux447ux430ux439}

Если \(S\) является плоской областью в плоскости \(xy\), теорема Стокса
сводится к теореме Грина.

\subsubsection{Пример 2: кровообращение в
полушарии}\label{ux43fux440ux438ux43cux435ux440-2-ux43aux440ux43eux432ux43eux43eux431ux440ux430ux449ux435ux43dux438ux435-ux432-ux43fux43eux43bux443ux448ux430ux440ux438ux438}

Пусть \(\mathbf{F}(x,y,z) = \langle -y, x, 0 \rangle\) и \(S\) ---
верхняя полусфера радиуса 1.

\begin{itemize}
\tightlist
\item
  Граница \(C\): единичный круг в плоскости \(xy\).
\item
  По теореме Стокса:
\end{itemize}

\[
\oint_C \mathbf{F}\cdot d\mathbf{r} = \iint_S (\nabla \times \mathbf{F})\cdot d\mathbf{S}.
\]

\begin{itemize}
\tightlist
\item
  Завиток: \(\nabla \times \mathbf{F} = \langle 0,0,2 \rangle\).
\item
  Нормаль к полушарию направлена \hspace{0pt}\hspace{0pt}наружу:
  \(\mathbf{n} = \langle 0,0,1 \rangle\).
\item
  Итак, подынтегр = 2.
\item
  Площадь полушария = \(2\pi (1^2)\).
\end{itemize}

\[
\iint_S 2\, dS = 2 \cdot 2\pi = 4\pi.
\]

Таким образом, циркуляция вокруг экватора равна \(4\pi\).

\subsubsection{Почему это
важно}\label{ux43fux43eux447ux435ux43cux443-ux44dux442ux43e-ux432ux430ux436ux43dux43e-21}

\begin{itemize}
\tightlist
\item
  Обеспечивает глубокую связь между поверхностными и линейными
  интегралами.
\item
  Упрощает расчеты, позволяя выбирать удобные поверхности.
\item
  Широко используется в электромагнетизме (закон Фарадея) и
  гидродинамике.
\end{itemize}

\subsubsection{Упражнения}\label{ux443ux43fux440ux430ux436ux43dux435ux43dux438ux44f-39}

\begin{enumerate}
\def\labelenumi{\arabic{enumi}.}
\tightlist
\item
  Проверьте теорему Стокса для
  \(\mathbf{F}(x,y,z) = \langle y, -x, 0 \rangle\) над единичным диском
  в плоскости \(xy\).
\item
  Вычислите \(\oint_C \mathbf{F}\cdot d\mathbf{r}\), где
  \(\mathbf{F}(x,y,z) = \langle z, 0, x \rangle\), а \(C\) --- граница
  треугольника с вершинами (0,0,0), (1,0,0), (0,1,0).
\item
  Покажите, что если \(\nabla \times \mathbf{F} = 0\), то обращение
  вокруг любой замкнутой кривой равно нулю.4. Примените теорему Стокса,
  чтобы вычислить обращение
  \(\mathbf{F}(x,y,z) = \langle -y, x, z \rangle\) вокруг границы
  единичного квадрата в плоскости \(z=0\).
\item
  Объясните, как теорема Стокса обобщает теорему Грина.
\end{enumerate}

\subsection{10.6 Теорема о
дивергенции}\label{ux442ux435ux43eux440ux435ux43cux430-ux43e-ux434ux438ux432ux435ux440ux433ux435ux43dux446ux438ux438}

Теорема о дивергенции (также называемая теоремой Гаусса) связывает поток
векторного поля через замкнутую поверхность с тройным интегралом
дивергенции поля внутри поверхности.

\subsubsection{Формулировка теоремы о
дивергенции}\label{ux444ux43eux440ux43cux443ux43bux438ux440ux43eux432ux43aux430-ux442ux435ux43eux440ux435ux43cux44b-ux43e-ux434ux438ux432ux435ux440ux433ux435ux43dux446ux438ux438}

Пусть \(E\) --- сплошная область в \(\mathbb{R}^3\) с граничной
поверхностью \(S\) (ориентированной наружу). Если \(\mathbf{F}(x,y,z)\)
--- векторное поле с непрерывными частными производными по \(E\), то

\[
\iint_S \mathbf{F} \cdot d\mathbf{S} = \iiint_E (\nabla \cdot \mathbf{F}) \, dV.
\]

\begin{itemize}
\tightlist
\item
  Слева: поток \(\mathbf{F}\) через замкнутую поверхность \(S\).
\item
  Правая часть: тройной интеграл от расхождения внутри региона.
\end{itemize}

\subsubsection{Дивергенция}\label{ux434ux438ux432ux435ux440ux433ux435ux43dux446ux438ux44f}

Дивергенция векторного поля
\(\mathbf{F}(x,y,z) = \langle P, Q, R \rangle\) равна

\[
\nabla \cdot \mathbf{F} = \frac{\partial P}{\partial x} + \frac{\partial Q}{\partial y} + \frac{\partial R}{\partial z}.
\]

Он измеряет «чистый отток» на единицу объема в каждой точке.

\subsubsection{Пример 1: Поток радиального
поля}\label{ux43fux440ux438ux43cux435ux440-1-ux43fux43eux442ux43eux43a-ux440ux430ux434ux438ux430ux43bux44cux43dux43eux433ux43e-ux43fux43eux43bux44f}

Пусть \(\mathbf{F}(x,y,z) = \langle x, y, z \rangle\) и \(E\) ---
единичный шар \(x^2+y^2+z^2 \leq 1\).

\begin{itemize}
\tightlist
\item
  Расхождение: \(\nabla \cdot \mathbf{F} = 1+1+1 = 3\).
\item
  Объем единичного шара: \(\tfrac{4}{3}\pi\). Итак
\end{itemize}

\[
\iiint_E (\nabla \cdot \mathbf{F})\, dV = 3 \cdot \tfrac{4}{3}\pi = 4\pi.
\]

Таким образом, поток через единичную сферу равен \(4\pi\).

\subsubsection{Пример 2: константное
поле}\label{ux43fux440ux438ux43cux435ux440-2-ux43aux43eux43dux441ux442ux430ux43dux442ux43dux43eux435-ux43fux43eux43bux435}

Пусть \(\mathbf{F}(x,y,z) = \langle 1, 0, 0 \rangle\).

\begin{itemize}
\tightlist
\item
  Расхождение: \(\nabla \cdot \mathbf{F} = 0\).
\item
  Таким образом, поток через любую замкнутую поверхность равен нулю, что
  соответствует интуиции (нет чистого оттока).
\end{itemize}

\subsubsection{Почему это
важно}\label{ux43fux43eux447ux435ux43cux443-ux44dux442ux43e-ux432ux430ux436ux43dux43e-22}

\begin{itemize}
\item
  Преобразует поверхностные интегралы в более простые объемные
  интегралы.
\item
  Используется в физике: закон Гаусса в электромагнетизме, потоке
  жидкости и теплопередаче.
\item
  Завершает объединяющую структуру:

  \begin{itemize}
  \tightlist
  \item
    Теорема Грина (2D ротор ↔ циркуляция)
  \item
    Теорема Стокса (3D ротор ↔ циркуляция на поверхностях)
  \item
    Теорема о дивергенции (3D-дивергенция ↔ поток на замкнутых
    поверхностях)
  \end{itemize}
\end{itemize}

\subsubsection{Упражнения}\label{ux443ux43fux440ux430ux436ux43dux435ux43dux438ux44f-40}

\begin{enumerate}
\def\labelenumi{\arabic{enumi}.}
\tightlist
\item
  Используйте теорему о дивергенции, чтобы вычислить поток
  \(\mathbf{F}(x,y,z) = \langle x,y,z \rangle\) через поверхность сферы
  радиуса \(R\).
\item
  Проверьте теорему о дивергенции для
  \(\mathbf{F}(x,y,z) = \langle y, z, x \rangle\) на единичном кубе
  \([0,1]^3\).
\item
  Покажите, что если \(\nabla \cdot \mathbf{F} = 0\), то полный поток
  через любую замкнутую поверхность равен нулю.
\item
  Вычислите поток \(\mathbf{F}(x,y,z) = \langle x^2, y^2, z^2 \rangle\)
  через единичную сферу.5. Объясните, как теорема о дивергенции обобщает
  одномерную Фундаментальную теорему исчисления.
\end{enumerate}

\section{Часть IV. Бесконечные
процессы}\label{ux447ux430ux441ux442ux44c-iv.-ux431ux435ux441ux43aux43eux43dux435ux447ux43dux44bux435-ux43fux440ux43eux446ux435ux441ux441ux44b}

\section{Глава 11. Последовательности и
сходимость}\label{ux433ux43bux430ux432ux430-11.-ux43fux43eux441ux43bux435ux434ux43eux432ux430ux442ux435ux43bux44cux43dux43eux441ux442ux438-ux438-ux441ux445ux43eux434ux438ux43cux43eux441ux442ux44c}

\subsection{11.1 Определения и
примеры}\label{ux43eux43fux440ux435ux434ux435ux43bux435ux43dux438ux44f-ux438-ux43fux440ux438ux43cux435ux440ux44b}

Последовательность --- это упорядоченный список чисел, обычно
записываемый в виде

\[
a_1, a_2, a_3, \dots
\]

или, в более общем смысле, \((a_n)_{n=1}^\infty\). Каждый \(a_n\)
называется n-м членом последовательности.

\subsubsection{Определение
последовательности}\label{ux43eux43fux440ux435ux434ux435ux43bux435ux43dux438ux435-ux43fux43eux441ux43bux435ux434ux43eux432ux430ux442ux435ux43bux44cux43dux43eux441ux442ux438}

Последовательность можно определить двумя способами:

\begin{enumerate}
\def\labelenumi{\arabic{enumi}.}
\item
  Явная формула -- дает прямое правило для n-го члена.

  \begin{itemize}
  \item
    Пример: \(a_n = \frac{1}{n}\) определяет последовательность.

    \[
    1, \tfrac{1}{2}, \tfrac{1}{3}, \tfrac{1}{4}, \dots
    \]
  \end{itemize}
\item
  Рекурсивное определение -- определяет термины, используя более ранние
  термины.

  \begin{itemize}
  \item
    Пример: последовательность Фибоначчи:

    \[
    a_1 = 1, \quad a_2 = 1, \quad a_{n} = a_{n-1} + a_{n-2} \quad (n \geq 3).
    \]
  \end{itemize}
\end{enumerate}

\subsubsection{Примеры
последовательностей}\label{ux43fux440ux438ux43cux435ux440ux44b-ux43fux43eux441ux43bux435ux434ux43eux432ux430ux442ux435ux43bux44cux43dux43eux441ux442ux435ux439}

\begin{enumerate}
\def\labelenumi{\arabic{enumi}.}
\item
  Арифметическая последовательность:

  \[
  a_n = a_1 + (n-1)d.
  \]

  Пример: \(a_n = 2n+1\) → последовательность нечетных чисел.
\item
  Геометрическая последовательность:

  \[
  a_n = a_1 r^{n-1}.
  \]

  Пример: \(a_n = 2^n\) → степени 2.
\item
  Гармоническая последовательность:

  \[
  a_n = \frac{1}{n}.
  \]
\item
  Попеременная последовательность:

  \[
  a_n = (-1)^n.
  \]
\end{enumerate}

\subsubsection{Последовательности в
исчислении}\label{ux43fux43eux441ux43bux435ux434ux43eux432ux430ux442ux435ux43bux44cux43dux43eux441ux442ux438-ux432-ux438ux441ux447ux438ux441ux43bux435ux43dux438ux438}

Последовательности являются основой бесконечных процессов:

\begin{itemize}
\tightlist
\item
  Пределы последовательностей → определяют сходимость.
\item
  Ряды → бесконечные суммы, построенные из последовательностей.
\item
  Функции, аппроксимируемые последовательностями и рядами.
\end{itemize}

\subsubsection{Почему это
важно}\label{ux43fux43eux447ux435ux43cux443-ux44dux442ux43e-ux432ux430ux436ux43dux43e-23}

\begin{itemize}
\tightlist
\item
  Последовательности представляют собой строительные блоки для
  бесконечных рядов и аппроксимаций.
\item
  Они позволяют нам строго определить «приближение к бесконечности» и
  конвергенцию.
\item
  Многие важные функции (экспоненциальные, тригонометрические) можно
  выразить через последовательности и ряды.
\end{itemize}

\subsubsection{Упражнения}\label{ux443ux43fux440ux430ux436ux43dux435ux43dux438ux44f-41}

\begin{enumerate}
\def\labelenumi{\arabic{enumi}.}
\tightlist
\item
  Запишите первые пять членов последовательности
  \(a_n = \frac{n}{n+1}\).
\item
  Определите, ограничен ли \(a_n = (-1)^n n\).
\item
  Дайте рекурсивное определение последовательности \(2,4,8,16,\dots\).
\item
  Найдите 10-й член арифметической последовательности с помощью
  \(a_1=3\) и \(d=5\).
\item
  Напишите явную формулу для последовательности, определенной \(a_1=1\),
  \(a_{n+1}=2a_n\).
\end{enumerate}

\subsection{11.2 Монотонные и ограниченные
последовательности}\label{ux43cux43eux43dux43eux442ux43eux43dux43dux44bux435-ux438-ux43eux433ux440ux430ux43dux438ux447ux435ux43dux43dux44bux435-ux43fux43eux441ux43bux435ux434ux43eux432ux430ux442ux435ux43bux44cux43dux43eux441ux442ux438}

Чтобы понять, сходится ли последовательность, нам нужно изучить ее
поведение: увеличивается ли она, уменьшается, остается в пределах границ
или растет без ограничений? Двумя важными понятиями являются
монотонность и ограниченность.

\subsubsection{Монотонные
последовательности}\label{ux43cux43eux43dux43eux442ux43eux43dux43dux44bux435-ux43fux43eux441ux43bux435ux434ux43eux432ux430ux442ux435ux43bux44cux43dux43eux441ux442ux438}

Последовательность \((a_n)\) называется монотонной, если она всегда
возрастает или всегда убывает.

\begin{itemize}
\item
  Монотонный рост:

  \[
  a_{n+1} \geq a_n \quad \text{for all } n.
  \]
\item
  Монотонно убывающее:

  \[
  a_{n+1} \leq a_n \quad \text{for all } n.
  \]
\end{itemize}

Примеры:1. \(a_n = n\) монотонно увеличивается. 2. \(a_n = \frac{1}{n}\)
монотонно убывает.

\subsubsection{Ограниченные
последовательности}\label{ux43eux433ux440ux430ux43dux438ux447ux435ux43dux43dux44bux435-ux43fux43eux441ux43bux435ux434ux43eux432ux430ux442ux435ux43bux44cux43dux43eux441ux442ux438}

Последовательность ограничена сверху, если существует номер \(M\) такой,
что \(a_n \leq M\) для всех \(n\). Он ограничен снизу, если существует
\(m\) такой, что \(a_n \geq m\) для всех \(n\).

Если оба условия выполняются, последовательность ограничена.

Примеры:

\begin{enumerate}
\def\labelenumi{\arabic{enumi}.}
\tightlist
\item
  \(a_n = \frac{1}{n}\) ограничен диапазоном от 0 до 1.
\item
  \(a_n = (-1)^n\) ограничен диапазоном от -1 до 1.
\item
  \(a_n = n\) не ограничен.
\end{enumerate}

\subsubsection{Теорема о монотонной
сходимости}\label{ux442ux435ux43eux440ux435ux43cux430-ux43e-ux43cux43eux43dux43eux442ux43eux43dux43dux43eux439-ux441ux445ux43eux434ux438ux43cux43eux441ux442ux438}

Фундаментальный результат анализа:

\begin{itemize}
\tightlist
\item
  Любая монотонно возрастающая последовательность, ограниченная сверху,
  сходится.
\item
  Любая монотонно убывающая последовательность, ограниченная снизу,
  сходится.
\end{itemize}

Эта теорема гарантирует сходимость без явного нахождения предела.

\subsubsection{Пример}\label{ux43fux440ux438ux43cux435ux440-1}

\begin{enumerate}
\def\labelenumi{\arabic{enumi}.}
\item
  Последовательность: \(a_n = 1 - \frac{1}{n}\).

  \begin{itemize}
  \tightlist
  \item
    Увеличение: с \(a_{n+1} - a_n = \frac{1}{n} - \frac{1}{n+1} > 0\).
  \item
    Ограничено сверху единицей.
  \item
    Поэтому сходится.
  \item
    Ограничение: \(\lim_{n\to\infty} a_n = 1\).
  \end{itemize}
\end{enumerate}

\subsubsection{Почему это
важно}\label{ux43fux43eux447ux435ux43cux443-ux44dux442ux43e-ux432ux430ux436ux43dux43e-24}

\begin{itemize}
\tightlist
\item
  Монотонность и ограниченность позволяют быстро проверить сходимость.
  --- Они необходимы для доказательств и строгого построения пределов.
  --- Эти идеи естественным образом распространяются на функции и ряды.
\end{itemize}

\subsubsection{Упражнения}\label{ux443ux43fux440ux430ux436ux43dux435ux43dux438ux44f-42}

\begin{enumerate}
\def\labelenumi{\arabic{enumi}.}
\tightlist
\item
  Определите, является ли \(a_n = \frac{n}{n+1}\) монотонным и
  ограниченным.
\item
  Покажите, что \(a_n = \sqrt{n}\) монотонно возрастает, но не
  ограничен.
\item
  Докажите, что \(a_n = 2 - \frac{1}{n}\) сходится, и найдите его
  предел.
\item
  Приведите пример ограниченной последовательности, не являющейся
  монотонной.
\item
  Примените теорему монотонной сходимости к
  \(a_n = \ln\!\big(1+\frac{1}{n}\big)\).
\end{enumerate}

\subsection{11.3 Пределы
последовательностей}\label{ux43fux440ux435ux434ux435ux43bux44b-ux43fux43eux441ux43bux435ux434ux43eux432ux430ux442ux435ux43bux44cux43dux43eux441ux442ux435ux439}

Главный вопрос о последовательности заключается в том, приближаются ли
ее члены к одному значению по мере роста \(n\). Это приводит к понятию
предела последовательности.

\subsubsection{Определение}\label{ux43eux43fux440ux435ux434ux435ux43bux435ux43dux438ux435-12}

Последовательность \((a_n)\) имеет предел \(L\), если для каждого
\(\varepsilon > 0\) существует целое число \(N\) такое, что

\[
|a_n - L| < \varepsilon \quad \text{whenever } n > N.
\]

Затем мы пишем

\[
\lim_{n\to\infty} a_n = L.
\]

Если такого \(L\) не существует, последовательность расходится.

\subsubsection{Интуиция}\label{ux438ux43dux442ux443ux438ux446ux438ux44f}

\begin{itemize}
\tightlist
\item
  Члены последовательности становятся произвольно близкими к \(L\) по
  мере того, как \(n\) становится большим.
\item
  За пределами некоторого индекса \(N\) все термины остаются в пределах
  небольшого диапазона около \(L\).
\end{itemize}

\subsubsection{Примеры}\label{ux43fux440ux438ux43cux435ux440ux44b-23}

\begin{enumerate}
\def\labelenumi{\arabic{enumi}.}
\item
  \(a_n = \frac{1}{n}\). По мере роста \(n\) термины сжимаются до 0.

  \[
  \lim_{n\to\infty} \frac{1}{n} = 0.
  \]
\item
  \(a_n = (-1)^n\). Члены чередуются между -1 и 1, поэтому единого
  предела не существует. Последовательность расходится.
\item
  \(a_n = \frac{n}{n+1}\). Поскольку \(n \to \infty\), числитель и
  знаменатель почти равны, поэтому

  \[
  \lim_{n\to\infty} \frac{n}{n+1} = 1.
  \]
\end{enumerate}

\subsubsection{\texorpdfstring{Свойства пределовЕсли \(\lim a_n = A\) и
\(\lim b_n = B\):}{Свойства пределовЕсли \textbackslash lim a\_n = A и \textbackslash lim b\_n = B:}}\label{ux441ux432ux43eux439ux441ux442ux432ux430-ux43fux440ux435ux434ux435ux43bux43eux432ux435ux441ux43bux438-lim-a_n-a-ux438-lim-b_n-b}

\begin{itemize}
\item
  \(\lim (a_n+b_n) = A+B\).
\item
  \(\lim (a_n b_n) = AB\).
\item
  \(\lim (c a_n) = cA\) для константы \(c\).
\item
  Если \(b_n \neq 0\) и \(B \neq 0\), то

  \[
  \lim \frac{a_n}{b_n} = \frac{A}{B}.
  \]
\end{itemize}

\subsubsection{Теорема: принцип
сжатия}\label{ux442ux435ux43eux440ux435ux43cux430-ux43fux440ux438ux43dux446ux438ux43f-ux441ux436ux430ux442ux438ux44f}

Если \(a_n \leq b_n \leq c_n\) для всех больших \(n\) и

\[
\lim_{n\to\infty} a_n = \lim_{n\to\infty} c_n = L,
\]

тогда

\[
\lim_{n\to\infty} b_n = L.
\]

Пример:

\[
a_n = -\tfrac{1}{n}, \quad b_n = \tfrac{\sin n}{n}, \quad c_n = \tfrac{1}{n}.
\]

Поскольку \(-\tfrac{1}{n} \leq \tfrac{\sin n}{n} \leq \tfrac{1}{n}\) и
обе ограничивающие последовательности переходят в 0,

\[
\lim_{n\to\infty} \frac{\sin n}{n} = 0.
\]

\subsubsection{Почему это
важно}\label{ux43fux43eux447ux435ux43cux443-ux44dux442ux43e-ux432ux430ux436ux43dux43e-25}

\begin{itemize}
\tightlist
\item
  Ограничения делают строгой идею о том, что последовательности
  «приближаются» к значению.
\item
  Сходимость последовательностей лежит в основе бесконечных серий и
  непрерывности.
\item
  Эти понятия необходимы для определения действительных чисел через
  пределы.
\end{itemize}

\subsubsection{Упражнения}\label{ux443ux43fux440ux430ux436ux43dux435ux43dux438ux44f-43}

\begin{enumerate}
\def\labelenumi{\arabic{enumi}.}
\tightlist
\item
  Найдите \(\lim_{n\to\infty} \frac{2n+1}{3n+4}\).
\item
  Определите, сходится ли \(a_n = \sqrt{n+1} - \sqrt{n}\).
\item
  Сходится ли \(a_n = \cos n\)? Почему или почему нет?
\item
  Используйте принцип сжатия, чтобы отобразить
  \(\lim_{n\to\infty} \frac{\sin n}{n} = 0\).
\item
  Докажите, что если \(\lim a_n = L\), то \(\lim |a_n| = |L|\).
\end{enumerate}

\section{Глава 12. Бесконечная
серия}\label{ux433ux43bux430ux432ux430-12.-ux431ux435ux441ux43aux43eux43dux435ux447ux43dux430ux44f-ux441ux435ux440ux438ux44f}

\subsection{12.1 Ряд и
сходимость}\label{ux440ux44fux434-ux438-ux441ux445ux43eux434ux438ux43cux43eux441ux442ux44c}

Ряд -- это сумма членов последовательности. Вместо того, чтобы просто
перечислять числа, мы складываем их вместе и изучаем, приближается ли
бесконечная сумма к конечному значению.

\subsubsection{Определение}\label{ux43eux43fux440ux435ux434ux435ux43bux435ux43dux438ux435-13}

Учитывая последовательность \((a_n)\), соответствующая серия равна

\[
\sum_{n=1}^\infty a_n = a_1 + a_2 + a_3 + \dots
\]

Мы определяем n-ю частичную сумму как

\[
S_n = \sum_{k=1}^n a_k.
\]

Если последовательность \((S_n)\) сходится к конечному пределу \(S\), то
ряд сходится и

\[
\sum_{n=1}^\infty a_n = S.
\]

Если \((S_n)\) расходится, то и ряд расходится.

\subsubsection{Примеры}\label{ux43fux440ux438ux43cux435ux440ux44b-24}

\begin{enumerate}
\def\labelenumi{\arabic{enumi}.}
\tightlist
\item
  Геометрический ряд
\end{enumerate}

\[
\sum_{n=0}^\infty ar^n = \frac{a}{1-r}, \quad |r| < 1.
\]

Пример:

\[
1 + \tfrac{1}{2} + \tfrac{1}{4} + \tfrac{1}{8} + \dots = 2.
\]

\begin{enumerate}
\def\labelenumi{\arabic{enumi}.}
\setcounter{enumi}{1}
\tightlist
\item
  Гармонический ряд
\end{enumerate}

\[
\sum_{n=1}^\infty \frac{1}{n}.
\]

Этот ряд расходится, хотя члены обращаются к 0.

\begin{enumerate}
\def\labelenumi{\arabic{enumi}.}
\setcounter{enumi}{2}
\tightlist
\item
  р-серия
\end{enumerate}

\[
\sum_{n=1}^\infty \frac{1}{n^p}.
\]

\begin{itemize}
\tightlist
\item
  Сходится, если \(p > 1\).
\item
  Расходится, если \(p \leq 1\).
\end{itemize}

\subsubsection{Необходимое условие
сходимости}\label{ux43dux435ux43eux431ux445ux43eux434ux438ux43cux43eux435-ux443ux441ux43bux43eux432ux438ux435-ux441ux445ux43eux434ux438ux43cux43eux441ux442ux438}

Если \(\sum a_n\) сходится, то обязательно

\[
\lim_{n\to\infty} a_n = 0.
\]

Если \(\lim a_n \neq 0\), ряд расходится. Но обратное неверно:
\(\lim a_n = 0\) не гарантирует сходимости (например, гармонического
ряда).

\subsubsection{Почему это
важно}\label{ux43fux43eux447ux435ux43cux443-ux44dux442ux43e-ux432ux430ux436ux43dux43e-26}

\begin{itemize}
\tightlist
\item
  Ряды расширяют конечное сложение до бесконечных процессов.
\item
  Сходящиеся ряды используются для аппроксимации функций, вычисления
  площадей и моделирования физических процессов.- Изучение рядов
  приводит к мощным тестам сходимости.
\end{itemize}

\subsubsection{Упражнения}\label{ux443ux43fux440ux430ux436ux43dux435ux43dux438ux44f-44}

\begin{enumerate}
\def\labelenumi{\arabic{enumi}.}
\tightlist
\item
  Определите, сходится ли \(\sum_{n=1}^\infty \frac{2}{3^n}\), и найдите
  его сумму.
\item
  Покажите, что \(\sum_{n=1}^\infty \frac{1}{n^2}\) сходится.
\item
  Сходится ли \(\sum_{n=1}^\infty \frac{1}{\sqrt{n}}\)?
\item
  Запишите первые четыре частичные суммы ряда
  \(\sum_{n=1}^\infty \frac{1}{2^n}\).
\item
  Объясните, почему \(\lim a_n = 0\) необходим, но недостаточен для
  конвергенции.
\end{enumerate}

\subsection{12.2 Тесты
сходимости}\label{ux442ux435ux441ux442ux44b-ux441ux445ux43eux434ux438ux43cux43eux441ux442ux438}

Поскольку многие ряды невозможно суммировать напрямую, математики
разработали тесты, позволяющие определить, сходится ли ряд или
расходится. Эти тесты являются инструментами для анализа бесконечных
сумм.

\subsubsection{1. Проверка n-го члена на
дивергенцию}\label{ux43fux440ux43eux432ux435ux440ux43aux430-n-ux433ux43e-ux447ux43bux435ux43dux430-ux43dux430-ux434ux438ux432ux435ux440ux433ux435ux43dux446ux438ux44e}

Если

\[
\lim_{n\to\infty} a_n \neq 0 \quad \text{or does not exist},
\]

тогда

\[
\sum a_n
\]

расходится.

Если \(\lim a_n = 0\), проверка не дает результатов.

\subsubsection{2. Сравнительный
тест}\label{ux441ux440ux430ux432ux43dux438ux442ux435ux43bux44cux43dux44bux439-ux442ux435ux441ux442}

Предположим, \(0 \leq a_n \leq b_n\) для всех \(n\).

\begin{itemize}
\tightlist
\item
  Если \(\sum b_n\) сходится, то сходится и \(\sum a_n\).
\item
  Если \(\sum a_n\) расходится, то и \(\sum b_n\) тоже расходится.
\end{itemize}

\subsubsection{3. Тест сравнения
пределов}\label{ux442ux435ux441ux442-ux441ux440ux430ux432ux43dux435ux43dux438ux44f-ux43fux440ux435ux434ux435ux43bux43eux432}

Если \(a_n, b_n > 0\) и

\[
\lim_{n\to\infty} \frac{a_n}{b_n} = c,
\]

где \(0 < c < \infty\), затем \(\sum a_n\) и \(\sum b_n\) либо сходятся,
либо расходятся.

\subsubsection{4. Тест на
соотношение}\label{ux442ux435ux441ux442-ux43dux430-ux441ux43eux43eux442ux43dux43eux448ux435ux43dux438ux435}

Для \(\sum a_n\) вычислите

\[
L = \lim_{n\to\infty} \left| \frac{a_{n+1}}{a_n} \right|.
\]

\begin{itemize}
\tightlist
\item
  Если \(L < 1\), то ряд сходится абсолютно.
\item
  Если \(L > 1\) или \(L = \infty\), серия расходится.
\item
  Если \(L = 1\), проверка не дает результатов.
\end{itemize}

\subsubsection{5. Корневой
тест}\label{ux43aux43eux440ux43dux435ux432ux43eux439-ux442ux435ux441ux442}

Для \(\sum a_n\) вычислите

\[
L = \lim_{n\to\infty} \sqrt[n]{|a_n|}.
\]

\begin{itemize}
\tightlist
\item
  Если \(L < 1\), то ряд сходится абсолютно.
\item
  Если \(L > 1\), ряд расходится.
\item
  Если \(L = 1\), проверка не дает результатов.
\end{itemize}

\subsubsection{6. Тест чередующихся серий (тест
Лейбница)}\label{ux442ux435ux441ux442-ux447ux435ux440ux435ux434ux443ux44eux449ux438ux445ux441ux44f-ux441ux435ux440ux438ux439-ux442ux435ux441ux442-ux43bux435ux439ux431ux43dux438ux446ux430}

Для серии вида

\[
\sum (-1)^n b_n \quad \text{or} \quad \sum (-1)^{n+1} b_n,
\]

если

\begin{enumerate}
\def\labelenumi{\arabic{enumi}.}
\tightlist
\item
  \(b_{n+1} \leq b_n\) (уменьшается) и
\item
  \(\lim_{n\to\infty} b_n = 0\),
\end{enumerate}

то ряд сходится.

\subsubsection{Примеры}\label{ux43fux440ux438ux43cux435ux440ux44b-25}

\begin{enumerate}
\def\labelenumi{\arabic{enumi}.}
\tightlist
\item
  \(\sum \frac{1}{n^2}\): Сравнительный тест → сходится.
\item
  \(\sum \frac{1}{n}\): Гармонический ряд → расходится.
\item
  \(\sum \frac{(-1)^n}{n}\): Тест чередующихся серий → сходится.
\item
  \(\sum \frac{n!}{n^n}\): тест соотношения → сходится.
\item
  \(\sum \frac{2^n}{n}\): Корневой тест → расходится.
\end{enumerate}

\subsubsection{Почему это
важно}\label{ux43fux43eux447ux435ux43cux443-ux44dux442ux43e-ux432ux430ux436ux43dux43e-27}

\begin{itemize}
\tightlist
\item
  Тесты на сходимость позволяют классифицировать ряды без необходимости
  явного суммирования.
\item
  Они предоставляют систематические способы обработки бесконечных
  процессов в исчислении.
\item
  Они имеют решающее значение для последующих тем, таких как степенные
  ряды и ряды Фурье.
\end{itemize}

\subsubsection{Упражнения}\label{ux443ux43fux440ux430ux436ux43dux435ux43dux438ux44f-45}

\begin{enumerate}
\def\labelenumi{\arabic{enumi}.}
\tightlist
\item
  Проверка сходимости \(\sum \frac{1}{n^3}\).
\item
  Используйте тест соотношения для \(\sum \frac{3^n}{n!}\).3. Примените
  корневой тест к \(\sum \left(\frac{1}{2}\right)^n\).
\item
  Определите сходимость \(\sum \frac{(-1)^n}{\sqrt{n}}\).
\item
  Используйте тест сравнения пределов с \(\frac{1}{n^2}\) для проверки
  \(\sum \frac{1}{n^2+1}\).
\end{enumerate}

\subsection{12.3 Абсолютная и условная
сходимость}\label{ux430ux431ux441ux43eux43bux44eux442ux43dux430ux44f-ux438-ux443ux441ux43bux43eux432ux43dux430ux44f-ux441ux445ux43eux434ux438ux43cux43eux441ux442ux44c}

Не все серии ведут себя одинаково при чередовании знаков. Чтобы
справиться с этим, мы различаем абсолютную сходимость и условную
сходимость.

\subsubsection{Абсолютная
конвергенция}\label{ux430ux431ux441ux43eux43bux44eux442ux43dux430ux44f-ux43aux43eux43dux432ux435ux440ux433ux435ux43dux446ux438ux44f}

Ряд \(\sum a_n\) абсолютно сходящийся, если

\[
\sum |a_n|
\]

сходится.

Теорема: Если ряд сходится абсолютно, то и он сходится.

Пример:

\[
\sum \frac{(-1)^n}{n^2}.
\]

Здесь сходится
\(\sum \left|\frac{(-1)^n}{n^2}\right| = \sum \frac{1}{n^2}\) (p-серия,
\(p=2\)). Таким образом, ряд абсолютно сходится.

\subsubsection{Условная
сходимость}\label{ux443ux441ux43bux43eux432ux43dux430ux44f-ux441ux445ux43eux434ux438ux43cux43eux441ux442ux44c}

Ряд \(\sum a_n\) считается условно сходящимся, если он сходится, но не
абсолютно.

Пример:

\[
\sum \frac{(-1)^n}{n}.
\]

\begin{itemize}
\tightlist
\item
  Тест переменного ряда → сходится.
\item
  Но \(\sum \left|\frac{(-1)^n}{n}\right| = \sum \frac{1}{n}\)
  расходится (гармонический ряд). Значит, ряд условно сходится.
\end{itemize}

\subsubsection{Теорема о
перестановке}\label{ux442ux435ux43eux440ux435ux43cux430-ux43e-ux43fux435ux440ux435ux441ux442ux430ux43dux43eux432ux43aux435}

Для условно сходящегося ряда перестановка членов может изменить сумму
--- даже заставить ее расходиться или сходиться к другому значению.

Этот удивительный результат показывает деликатную природу условной
сходимости.

\subsubsection{Почему это
важно}\label{ux43fux43eux447ux435ux43cux443-ux44dux442ux43e-ux432ux430ux436ux43dux43e-28}

\begin{itemize}
\tightlist
\item
  Абсолютная конвергенция сильнее и гарантирует стабильность.
\item
  Условная сходимость подчеркивает важность порядка в бесконечных
  суммах. --- Многие встречающиеся на практике знакопеременные ряды
  сходятся лишь условно.
\end{itemize}

\subsubsection{Упражнения}\label{ux443ux43fux440ux430ux436ux43dux435ux43dux438ux44f-46}

\begin{enumerate}
\def\labelenumi{\arabic{enumi}.}
\tightlist
\item
  Покажите, что \(\sum \frac{(-1)^n}{n^3}\) абсолютно сходится.
\item
  Покажите, что \(\sum \frac{(-1)^n}{n}\) условно сходится.
\item
  Проверьте \(\sum \frac{(-1)^n}{\sqrt{n}}\) на абсолютную и условную
  сходимость.
\item
  Объясните, почему абсолютная сходимость влечет за собой сходимость, а
  обратное неверно.
\item
  Исследуйте и изложите своими словами теорему о перегруппировке Римана.
\end{enumerate}

\section{Глава 13. Степенные ряды и
разложения}\label{ux433ux43bux430ux432ux430-13.-ux441ux442ux435ux43fux435ux43dux43dux44bux435-ux440ux44fux434ux44b-ux438-ux440ux430ux437ux43bux43eux436ux435ux43dux438ux44f}

\subsection{13.1 Серия
мощности}\label{ux441ux435ux440ux438ux44f-ux43cux43eux449ux43dux43eux441ux442ux438}

Степенной ряд --- это бесконечный ряд, в котором каждый член содержит
степень переменной. Степенные ряды занимают центральное место в
исчислении, поскольку они позволяют нам представлять функции в виде
бесконечных многочленов.

\subsubsection{Общая
форма}\label{ux43eux431ux449ux430ux44f-ux444ux43eux440ux43cux430}

Степенной ряд с центром в \(a\) имеет вид

\[
\sum_{n=0}^\infty c_n (x-a)^n,
\]

где \(c_n\) --- константы, называемые коэффициентами.

\begin{itemize}
\item
  Если \(a=0\), серия центрируется в начале координат:

  \[
  \sum_{n=0}^\infty c_n x^n.
  \]
\end{itemize}

\subsubsection{Примеры}\label{ux43fux440ux438ux43cux435ux440ux44b-26}

\begin{enumerate}
\def\labelenumi{\arabic{enumi}.}
\tightlist
\item
  Геометрический ряд
\end{enumerate}

\[
\sum_{n=0}^\infty x^n = \frac{1}{1-x}, \quad |x|<1.
\]

\begin{enumerate}
\def\labelenumi{\arabic{enumi}.}
\setcounter{enumi}{1}
\tightlist
\item
  Показательная функция
\end{enumerate}

\[e^x = \sum_{n=0}^\infty \frac{x^n}{n!}.
\]

\begin{enumerate}
\def\labelenumi{\arabic{enumi}.}
\setcounter{enumi}{2}
\tightlist
\item
  Sine and cosine
\end{enumerate}

\[
\sin x = \sum_{n=0}^\infty (-1)^n \frac{x^{2n+1}}{(2n+1)!}, \quad  
\cos x = \sum_{n=0}^\infty (-1)^n \frac{x^{2n}}{(2n)!}.
\]

\subsubsection{Interval of Convergence}\label{interval-of-convergence}

For each power series, there exists a radius of convergence \(R\) such
that:

\begin{itemize}
\tightlist
\item
  The series converges if \(|x-a| < R\).
\item
  The series diverges if \(|x-a| > R\).
\item
  At \(|x-a| = R\), convergence must be checked separately.
\end{itemize}

\subsubsection{Why This Matters}\label{why-this-matters-3}

\begin{itemize}
\tightlist
\item
  Power series allow us to approximate functions by polynomials.
\item
  They connect calculus with analysis and differential equations.
\item
  Many special functions in mathematics and physics are defined by their
  power series.
\end{itemize}

\subsubsection{Exercises}\label{exercises-6}

\begin{enumerate}
\def\labelenumi{\arabic{enumi}.}
\tightlist
\item
  Write the power series for \(\sum_{n=0}^\infty \frac{(x-2)^n}{n!}\).
\item
  Find the first four terms of the power series for \(e^x\).
\item
  Express \(\frac{1}{1+x}\) as a power series centered at 0.
\item
  Determine whether the series \(\sum_{n=0}^\infty n! x^n\) converges at
  \(x=0.1\).
\item
  Explain why power series are sometimes called ``infinite
  polynomials.''
\end{enumerate}

\subsection{13.2 Radius of Convergence}\label{radius-of-convergence}

Every power series converges for some values of \(x\) and diverges for
others. The boundary between these two behaviors is described by the
radius of convergence.

\subsubsection{Definition}\label{definition}

For a power series

\[
\sum_{n=0}^\infty c_n (x-a)^n,
\]

there exists a number \(R \geq 0\) (possibly infinite) such that:

\begin{itemize}
\tightlist
\item
  The series converges absolutely if \(|x-a| < R\).
\item
  The series diverges if \(|x-a| > R\).
\item
  At \(|x-a| = R\), convergence must be checked separately.
\end{itemize}

This number \(R\) is called the radius of convergence.

\subsubsection{Finding the Radius of
Convergence}\label{finding-the-radius-of-convergence}

Two common methods:

\begin{enumerate}
\def\labelenumi{\arabic{enumi}.}
\tightlist
\item
  Ratio Test
\end{enumerate}

\[
R = \lim_{n\to\infty} \left| \frac{c_n}{c_{n+1}} \right|.
\]

\begin{enumerate}
\def\labelenumi{\arabic{enumi}.}
\setcounter{enumi}{1}
\tightlist
\item
  Root Test
\end{enumerate}

\[
R = \frac{1}{\limsup_{n\to\infty} \sqrt[n]{|c_n|}}.
\]

\subsubsection{Examples}\label{examples-4}

\begin{enumerate}
\def\labelenumi{\arabic{enumi}.}
\tightlist
\item
  Series:
\end{enumerate}

\[
\sum_{n=0}^\infty \frac{x^n}{n!}.
\]

Using ratio test:

\[
\lim_{n\to\infty} \frac{1/(n!)}{1/((n+1)!)} = \infty.
\]

So \(R = \infty\) (converges for all real \(x\)).

\begin{enumerate}
\def\labelenumi{\arabic{enumi}.}
\setcounter{enumi}{1}
\tightlist
\item
  Series:
\end{enumerate}

\[
\sum_{n=0}^\infty x^n.
\]

Here \(c_n = 1\).

\[
Р = 1.
\]

Converges for \(|x| < 1\).

\begin{enumerate}
\def\labelenumi{\arabic{enumi}.}
\setcounter{enumi}{2}
\tightlist
\item
  Series:
\end{enumerate}

\[
\sum_{n=1}^\infty \frac{x^n}{n}.
\]

Ratio test:

\[
\lim_{n\to\infty} \left|\frac{(x^{n+1}/(n+1))}{(x^n/n)}\right| = |х|.
\]

Итак, \(R = 1\). Сходится для \(|x| < 1\), расходится для \(|x| > 1\). В
\(x=\pm 1\) проверьте отдельно.

\subsubsection{Интервал
сближения}\label{ux438ux43dux442ux435ux440ux432ux430ux43b-ux441ux431ux43bux438ux436ux435ux43dux438ux44f}

Набор значений \(x\), в которых ряд сходится, называется интервалом
сходимости.

\begin{itemize}
\tightlist
\item
  Всегда центрируется по адресу \(a\).
\item
  Расширяет блоки \(R\) в обоих направлениях.
\item
  Конечные точки \(x=a\pm R\) необходимо проверять индивидуально.
\end{itemize}

\subsubsection{Почему это важно- Радиус сходимости говорит нам, где
степенные ряды ведут себя как
функции.}\label{ux43fux43eux447ux435ux43cux443-ux44dux442ux43e-ux432ux430ux436ux43dux43e--ux440ux430ux434ux438ux443ux441-ux441ux445ux43eux434ux438ux43cux43eux441ux442ux438-ux433ux43eux432ux43eux440ux438ux442-ux43dux430ux43c-ux433ux434ux435-ux441ux442ux435ux43fux435ux43dux43dux44bux435-ux440ux44fux434ux44b-ux432ux435ux434ux443ux442-ux441ux435ux431ux44f-ux43aux430ux43a-ux444ux443ux43dux43aux446ux438ux438.}

\begin{itemize}
\tightlist
\item
  Необходим для практического использования разложений ряда Тейлора.
\item
  Определяет область применимости рядовых решений в физике и технике.
\end{itemize}

\subsubsection{Упражнения}\label{ux443ux43fux440ux430ux436ux43dux435ux43dux438ux44f-47}

\begin{enumerate}
\def\labelenumi{\arabic{enumi}.}
\tightlist
\item
  Найдите радиус сходимости \(\sum_{n=0}^\infty \frac{(x-3)^n}{n!}\).
\item
  Вычислите радиус сходимости \(\sum_{n=1}^\infty \frac{x^n}{n^2}\).
\item
  Используйте тест соотношения, чтобы найти \(R\) для
  \(\sum_{n=0}^\infty n!x^n\).
\item
  Определите интервал сходимости для
  \(\sum_{n=1}^\infty \frac{(x+1)^n}{n}\).
\item
  Объясните, почему показательный ряд сходится для всех \(x\), а
  геометрический ряд сходится только для \(|x|<1\).
\end{enumerate}

\subsection{13.3 Ряд Тейлора и
Маклорена}\label{ux440ux44fux434-ux442ux435ux439ux43bux43eux440ux430-ux438-ux43cux430ux43aux43bux43eux440ux435ux43dux430}

Степенные ряды становятся особенно эффективными, когда они используются
для представления знакомых функций. Это делается с помощью ряда Тейлора,
а особый случай с центром в 0 называется рядом Маклорена.

\subsubsection{Серия
Тейлор}\label{ux441ux435ux440ux438ux44f-ux442ux435ux439ux43bux43eux440}

Если функция \(f(x)\) бесконечно дифференцируема в \(x=a\), ее ряд
Тейлора относительно \(a\) равен

\[
f(x) = \sum_{n=0}^\infty \frac{f^{(n)}(a)}{n!}(x-a)^n.
\]

Здесь \(f^{(n)}(a)\) обозначает \(n\)-ю производную от \(f\) по адресу
\(a\).

\subsubsection{Серия
Маклорен}\label{ux441ux435ux440ux438ux44f-ux43cux430ux43aux43bux43eux440ux435ux43d}

Серия Тейлора с центром в \(a=0\):

\[
f(x) = \sum_{n=0}^\infty \frac{f^{(n)}(0)}{n!} x^n.
\]

\subsubsection{Примеры}\label{ux43fux440ux438ux43cux435ux440ux44b-27}

\begin{enumerate}
\def\labelenumi{\arabic{enumi}.}
\tightlist
\item
  Показательная функция
\end{enumerate}

\[
e^x = 1 + x + \frac{x^2}{2!} + \frac{x^3}{3!} + \cdots
\]

\begin{enumerate}
\def\labelenumi{\arabic{enumi}.}
\setcounter{enumi}{1}
\tightlist
\item
  Синус и косинус
\end{enumerate}

\[
\sin x = x - \frac{x^3}{3!} + \frac{x^5}{5!} - \cdots
\]

\[
\cos x = 1 - \frac{x^2}{2!} + \frac{x^4}{4!} - \cdots
\]

\begin{enumerate}
\def\labelenumi{\arabic{enumi}.}
\setcounter{enumi}{2}
\tightlist
\item
  Натуральный логарифм (для \(|x|<1\))
\end{enumerate}

\[
\ln(1+x) = x - \frac{x^2}{2} + \frac{x^3}{3} - \frac{x^4}{4} + \cdots
\]

\subsubsection{Полиномиальная аппроксимация
Тейлора}\label{ux43fux43eux43bux438ux43dux43eux43cux438ux430ux43bux44cux43dux430ux44f-ux430ux43fux43fux440ux43eux43aux441ux438ux43cux430ux446ux438ux44f-ux442ux435ux439ux43bux43eux440ux430}

Конечная сумма первых членов \(n\) представляет собой полином Тейлора
степени \(n\):

\[
P_n(x) = \sum_{k=0}^n \frac{f^{(k)}(a)}{k!}(x-a)^k.
\]

Этот полином приближается к \(f(x)\) рядом с \(x=a\).

\subsubsection{Остаток (термин
ошибки)}\label{ux43eux441ux442ux430ux442ux43eux43a-ux442ux435ux440ux43cux438ux43d-ux43eux448ux438ux431ux43aux438}

Разница между функцией и ее полиномом Тейлора равна остатку:

\[
R_n(x) = f(x) - P_n(x).
\]

Одна форма (форма Лагранжа)

\[
R_n(x) = \frac{f^{(n+1)}(c)}{(n+1)!}(x-a)^{n+1},
\]

для некоторых \(c\) между \(a\) и \(x\).

\subsubsection{Почему это
важно}\label{ux43fux43eux447ux435ux43cux443-ux44dux442ux43e-ux432ux430ux436ux43dux43e-29}

\begin{itemize}
\tightlist
\item
  Ряды Тейлора обеспечивают полиномиальную аппроксимацию сложных
  функций.
\item
  Они необходимы в численном анализе, физике и технике.
\item
  Разложение в ряд Маклорена дает простые формулы для показательных,
  тригонометрических и логарифмических функций.
\end{itemize}

\subsubsection{Упражнения}\label{ux443ux43fux440ux430ux436ux43dux435ux43dux438ux44f-48}

\begin{enumerate}
\def\labelenumi{\arabic{enumi}.}
\tightlist
\item
  Найдите серию Маклорена для \(f(x)=\cosh x = \tfrac{e^x+e^{-x}}{2}\).
\item
  Напишите ряд Тейлора для \(f(x)=e^x\) с центром в \(a=2\).
\item
  Вычислите полином Тейлора степени 3 для \(f(x)=\ln(1+x)\) в \(a=0\).4.
  Используйте ряд Маклорена для \(\sin x\), чтобы приблизить
  \(\sin(0.1)\).
\item
  Объясните, почему ряды Тейлора часто дают хорошие локальные
  аппроксимации, но могут расходиться при больших значениях \(|x|\).
\end{enumerate}

\subsection{13.4 Применение рядов
Тейлора}\label{ux43fux440ux438ux43cux435ux43dux435ux43dux438ux435-ux440ux44fux434ux43eux432-ux442ux435ux439ux43bux43eux440ux430}

Ряды Тейлора --- это не только теоретические инструменты --- они
используются для аппроксимации функций, решения уравнений и анализа
физических систем. Их приложения охватывают математику, естественные
науки и инженерию.

\subsubsection{Аппроксимация
функции}\label{ux430ux43fux43fux440ux43eux43aux441ux438ux43cux430ux446ux438ux44f-ux444ux443ux43dux43aux446ux438ux438}

Сложные функции можно аппроксимировать полиномами вблизи точки.

Пример: Приблизительно \(e^x\) рядом с \(x=0\) с использованием полинома
Маклорена 3-й степени:

\[
P_3(x) = 1 + x + \tfrac{x^2}{2} + \tfrac{x^3}{6}.
\]

Для небольшого \(x\) это дает точные оценки \(e^x\).

\subsubsection{Численные
методы}\label{ux447ux438ux441ux43bux435ux43dux43dux44bux435-ux43cux435ux442ux43eux434ux44b}

Ряды Тейлора составляют основу численных алгоритмов:

\begin{itemize}
\tightlist
\item
  Приближение квадратных корней, логарифмов и тригонометрических
  значений.
\item
  Оценка ошибки в остаточном члене.
\item
  Используется в итерационных методах, таких как метод Ньютона (где
  локальная линеаризация происходит из ряда Тейлора).
\end{itemize}

\subsubsection{Решение дифференциальных
уравнений}\label{ux440ux435ux448ux435ux43dux438ux435-ux434ux438ux444ux444ux435ux440ux435ux43dux446ux438ux430ux43bux44cux43dux44bux445-ux443ux440ux430ux432ux43dux435ux43dux438ux439}

Многие дифференциальные уравнения имеют решения, выраженные в виде рядов
Тейлора (или степенных).

Пример: Решением \(y'' + y = 0\) с \(y(0)=0, y'(0)=1\) является
\(\sin x\), который естественным образом возникает из серии Маклорена.

\subsubsection{Физика и
инженерия}\label{ux444ux438ux437ux438ux43aux430-ux438-ux438ux43dux436ux435ux43dux435ux440ux438ux44f}

\begin{itemize}
\item
  Малоугловое приближение:

  \[
  \sin x \approx x, \quad \cos x \approx 1 - \tfrac{x^2}{2}, \quad |x| \ll 1.
  \]

  Используется в маятниковом движении, оптике и волновой механике.
\item
  Теория относительности и квантовая механика: расширения Тейлора
  упрощают нелинейные выражения для практического использования.
\item
  Аппроксимация энергетических функций: В механике функции потенциальной
  энергии разлагаются вблизи точек равновесия.
\end{itemize}

\subsubsection{Вероятность и
статистика}\label{ux432ux435ux440ux43eux44fux442ux43dux43eux441ux442ux44c-ux438-ux441ux442ux430ux442ux438ux441ux442ux438ux43aux430}

\begin{itemize}
\tightlist
\item
  Функции, производящие момент, и характеристические функции используют
  степенные ряды.
\item
  Аппроксимации вероятностных распределений (например, нормальное
  приближение к биномиальному) используют разложения Тейлора.
\end{itemize}

\subsubsection{Почему это
важно}\label{ux43fux43eux447ux435ux43cux443-ux44dux442ux43e-ux432ux430ux436ux43dux43e-30}

\begin{itemize}
\tightlist
\item
  Ряды Тейлора служат мостом между точными формулами и практическими
  вычислениями.
\item
  Они позволяют нам сводить сложные проблемы к управляемым
  полиномиальным аппроксимациям.
\item
  Приложения делают их одним из важнейших инструментов прикладной
  математики.
\end{itemize}

\subsubsection{Упражнения}\label{ux443ux43fux440ux430ux436ux43dux435ux43dux438ux44f-49}

\begin{enumerate}
\def\labelenumi{\arabic{enumi}.}
\tightlist
\item
  Используйте серию Маклорена для \(e^x\), чтобы приблизить \(e^{0.1}\)
  до четырех десятичных знаков.
\item
  Примените приближение малого угла для оценки \(\sin(5^\circ)\).
\item
  Решите дифференциальное уравнение \(y'' = -y\), используя метод
  степенного ряда.
\item
  Расширьте \(\ln(1+x)\) до 4-й степени и используйте его для
  аппроксимации \(\ln(1.1)\).
\item
  Объясните, почему полиномиальные аппроксимации особенно полезны для
  компьютеров и калькуляторов.\# Приложения
\end{enumerate}

\subsection{Приложение A. Основы предварительного
расчета}\label{ux43fux440ux438ux43bux43eux436ux435ux43dux438ux435-a.-ux43eux441ux43dux43eux432ux44b-ux43fux440ux435ux434ux432ux430ux440ux438ux442ux435ux43bux44cux43dux43eux433ux43e-ux440ux430ux441ux447ux435ux442ux430}

\subsubsection{A.1 Курс повышения квалификации по
алгебре}\label{a.1-ux43aux443ux440ux441-ux43fux43eux432ux44bux448ux435ux43dux438ux44f-ux43aux432ux430ux43bux438ux444ux438ux43aux430ux446ux438ux438-ux43fux43e-ux430ux43bux433ux435ux431ux440ux435}

Прежде чем погрузиться в математический анализ, полезно просмотреть
некоторые навыки алгебры, которые будут возникать снова и снова. Это
«инструменты», которые вам понадобятся для работы с выражениями, решения
уравнений и упрощения результатов.

\paragraph{Показатели и
степени}\label{ux43fux43eux43aux430ux437ux430ux442ux435ux43bux438-ux438-ux441ux442ux435ux43fux435ux43dux438}

\begin{itemize}
\item
  Основные правила:

  \[
  a^m \cdot a^n = a^{m+n}, \quad \frac{a^m}{a^n} = a^{m-n}, \quad (a^m)^n = a^{mn}.
  \]
\item
  Отрицательные показатели:

  \[
  a^{-n} = \frac{1}{a^n}, \quad a \neq 0.
  \]
\item
  Дробные показатели:

  \[
  a^{1/n} = \sqrt[n]{a}, \quad a^{m/n} = \sqrt[n]{a^m}.
  \]
\end{itemize}

\paragraph{Факторинг}\label{ux444ux430ux43aux442ux43eux440ux438ux43dux433}

Факторинг упрощает выражения и помогает в решении уравнений.

\begin{enumerate}
\def\labelenumi{\arabic{enumi}.}
\item
  Общий фактор:

  \[
  6x^2+9x = 3x(2x+3).
  \]
\item
  Разность квадратов:

  \[
  a^2-b^2 = (a-b)(a+b).
  \]
\item
  Квадратичные трёхчлены:

  \[
  x^2+5x+6 = (x+2)(x+3).
  \]
\end{enumerate}

\paragraph{Полиномы}\label{ux43fux43eux43bux438ux43dux43eux43cux44b}

\begin{itemize}
\tightlist
\item
  Стандартная форма: \(P(x) = a_nx^n + a_{n-1}x^{n-1} + \cdots + a_0\).
\item
  Степень: наибольшая степень \(x\).
\item
  Длинное деление и синтетическое деление полезны для упрощения
  рациональных функций.
\end{itemize}

\paragraph{Рациональные
выражения}\label{ux440ux430ux446ux438ux43eux43dux430ux43bux44cux43dux44bux435-ux432ux44bux440ux430ux436ux435ux43dux438ux44f}

Упростите, разложив числитель и знаменатель на множители:

\[
\frac{x^2-1}{x^2-2x+1} = \frac{(x-1)(x+1)}{(x-1)^2} = \frac{x+1}{x-1}, \quad x \neq 1.
\]

\paragraph{Логарифмы}\label{ux43bux43eux433ux430ux440ux438ux444ux43cux44b}

\begin{itemize}
\item
  Определение: \(\log_a b = c\) означает \(a^c = b\).
\item
  Общие базы: натуральный логарифм (\(\ln x = \log_e x\)) и база 10
  (\(\log x\)).
\item
  Правила:

  \[
  \log(ab) = \log a + \log b, \quad \log\left(\frac{a}{b}\right) = \log a - \log b, \quad \log(a^n) = n\log a.
  \]
\end{itemize}

\paragraph{Уравнения}\label{ux443ux440ux430ux432ux43dux435ux43dux438ux44f}

\begin{itemize}
\item
  Линейный: решить \(ax+b=0\) → \(x=-b/a\).
\item
  Квадратичный: \(ax^2+bx+c=0\) имеет решения.

  \[
  x=\frac{-b\pm \sqrt{b^2-4ac}}{2a}.
  \]
\item
  Экспонента: \(e^x = k\) → \(x = \ln k\).
\end{itemize}

\subsubsection{A.2 Основы
тригонометрии}\label{a.2-ux43eux441ux43dux43eux432ux44b-ux442ux440ux438ux433ux43eux43dux43eux43cux435ux442ux440ux438ux438}

Тригонометрия дает язык углов и периодических явлений. Поскольку
исчисление часто имеет дело с колебаниями, движением и волнами,
необходимо четкое понимание тригонометрических функций и их свойств.

\paragraph{Единичный
круг}\label{ux435ux434ux438ux43dux438ux447ux43dux44bux439-ux43aux440ux443ux433}

\begin{itemize}
\item
  Определяется как круг радиуса 1 с центром в начале координат
  координатной плоскости.
\item
  Для угла \(\theta\), измеренного от положительной оси \(x\):

  \[
  (\cos \theta, \sin \theta)
  \]

  дает координаты точки на окружности.
\end{itemize}

Особые значения:

\begin{longtable}[]{@{}
  >{\raggedright\arraybackslash}p{(\linewidth - 6\tabcolsep) * \real{0.3333}}
  >{\raggedright\arraybackslash}p{(\linewidth - 6\tabcolsep) * \real{0.1667}}
  >{\raggedright\arraybackslash}p{(\linewidth - 6\tabcolsep) * \real{0.1667}}
  >{\raggedright\arraybackslash}p{(\linewidth - 6\tabcolsep) * \real{0.3333}}@{}}
\toprule\noalign{}
\begin{minipage}[b]{\linewidth}\raggedright
\(\theta\)
\end{minipage} & \begin{minipage}[b]{\linewidth}\raggedright
\(\sin \theta\)
\end{minipage} & \begin{minipage}[b]{\linewidth}\raggedright
\(\cos \theta\)
\end{minipage} & \begin{minipage}[b]{\linewidth}\raggedright
\(\tan \theta = \frac{\sin \theta}{\cos \theta}\)
\end{minipage} \\
\midrule\noalign{}
\endhead
\bottomrule\noalign{}
\endlastfoot
\(0\) & 0 & 1 & 0 \\
\(\pi/6\) & 1/2 & \(\sqrt{3}/2\) & \(1/\sqrt{3}\) \\
\(\pi/3\) & \(\sqrt{3}/2\) & 1/2 & \(\sqrt{3}\) \\
\(\pi/2\) & 1 & 0 & неопределенный \\
\end{longtable}

\paragraph{Основные
идентичности}\label{ux43eux441ux43dux43eux432ux43dux44bux435-ux438ux434ux435ux43dux442ux438ux447ux43dux43eux441ux442ux438}

\begin{enumerate}
\def\labelenumi{\arabic{enumi}.}
\tightlist
\item
  Пифагорейская идентичность
\end{enumerate}

\[
\sin^2\theta + \cos^2\theta = 1.
\]

\begin{enumerate}
\def\labelenumi{\arabic{enumi}.}
\setcounter{enumi}{1}
\tightlist
\item
  Фактортождества
\end{enumerate}

\[
\tan\theta = \frac{\sin\theta}{\cos\theta}, \quad \cot\theta = \frac{\cos\theta}{\sin\theta}.
\]

\begin{enumerate}
\def\labelenumi{\arabic{enumi}.}
\setcounter{enumi}{2}
\tightlist
\item
  Взаимные тождества
\end{enumerate}

\[
\sec\theta = \frac{1}{\cos\theta}, \quad \csc\theta = \frac{1}{\sin\theta}.
\]

\paragraph{Формулы сложения
углов}\label{ux444ux43eux440ux43cux443ux43bux44b-ux441ux43bux43eux436ux435ux43dux438ux44f-ux443ux433ux43bux43eux432}

\[
\sin(\alpha+\beta) = \sin\alpha\cos\beta + \cos\alpha\sin\beta,
\]

\[
\cos(\alpha+\beta) = \cos\alpha\cos\beta - \sin\alpha\sin\beta.
\]

Особые случаи:

\begin{itemize}
\item
  Двойной угол:

  \[
  \sin(2\theta) = 2\sin\theta\cos\theta, \quad
  \cos(2\theta) = \cos^2\theta - \sin^2\theta.
  \]
\end{itemize}

\paragraph{Графики}\label{ux433ux440ux430ux444ux438ux43aux438}

\begin{itemize}
\tightlist
\item
  \(\sin x\): волна начинается с 0, амплитуда 1, период \(2\pi\).
\item
  \(\cos x\): волна, начинающаяся с 1, амплитуда 1, период \(2\pi\).
\item
  \(\tan x\): повторяется каждый \(\pi\), неопределенный, с нечетным
  кратным \(\pi/2\).
\end{itemize}

\subsubsection{A.3 Координатная
геометрия}\label{a.3-ux43aux43eux43eux440ux434ux438ux43dux430ux442ux43dux430ux44f-ux433ux435ux43eux43cux435ux442ux440ux438ux44f}

Координатная геометрия связывает алгебру и геометрию, описывая
геометрические объекты (линии, круги, кривые) с помощью уравнений.
Исчисление в значительной степени опирается на эту структуру для
построения графиков функций, поиска наклонов и анализа кривых.

\paragraph{Декартова
плоскость}\label{ux434ux435ux43aux430ux440ux442ux43eux432ux430-ux43fux43bux43eux441ux43aux43eux441ux442ux44c}

\begin{itemize}
\item
  Точка представлена координатами \((x,y)\).
\item
  Расстояние между двумя точками \((x_1,y_1)\) и \((x_2,y_2)\):

  \[
  d = \sqrt{(x_2-x_1)^2 + (y_2-y_1)^2}.
  \]
\item
  Середина отрезка:

  \[
  M = \left(\frac{x_1+x_2}{2}, \frac{y_1+y_2}{2}\right).
  \]
\end{itemize}

\paragraph{Линии}\label{ux43bux438ux43dux438ux438}

\begin{enumerate}
\def\labelenumi{\arabic{enumi}.}
\item
  Формула наклона

  \[
  m = \frac{y_2-y_1}{x_2-x_1}.
  \]
\item
  Уравнение линии

  \begin{itemize}
  \item
    Точечно-наклонная форма:

    \[
    y-y_1 = m(x-x_1).
    \]
  \item
    Форма пересечения наклона:

    \[
    y = mx+b.
    \]
  \end{itemize}
\item
  Параллельные и перпендикулярные линии

  \begin{itemize}
  \tightlist
  \item
    Параллельные линии: одинаковый наклон.
  \item
    Перпендикулярные линии: уклоны соответствуют \(m_1m_2 = -1\).
  \end{itemize}
\end{enumerate}

\paragraph{Круги}\label{ux43aux440ux443ux433ux438}

Уравнение окружности с центром \((h,k)\) и радиусом \(r\):

\[
(x-h)^2+(y-k)^2 = r^2.
\]

Особый случай: единичный круг с центром в начале координат:

\[
x^2+y^2=1.
\]

\paragraph{Конические
сечения}\label{ux43aux43eux43dux438ux447ux435ux441ux43aux438ux435-ux441ux435ux447ux435ux43dux438ux44f}

\begin{enumerate}
\def\labelenumi{\arabic{enumi}.}
\item
  Парабола:

  \begin{itemize}
  \item
    Стандартная форма (открытие вверх/вниз):

    \[
    y = ax^2+bx+c.
    \]
  \end{itemize}
\item
  Эллипс (с центром в начале координат):

  \[
  \frac{x^2}{a^2}+\frac{y^2}{b^2}=1.
  \]
\item
  Гипербола (с центром в начале координат):

  \[
  \frac{x^2}{a^2}-\frac{y^2}{b^2}=1.
  \]
\end{enumerate}

\subsection{Приложение Б. Основные формулы и
таблицы}\label{ux43fux440ux438ux43bux43eux436ux435ux43dux438ux435-ux431.-ux43eux441ux43dux43eux432ux43dux44bux435-ux444ux43eux440ux43cux443ux43bux44b-ux438-ux442ux430ux431ux43bux438ux446ux44b}

\subsubsection{B.1 Таблица производныхПроизводные измеряют скорость
изменения и наклоны функций. Наличие таблицы быстрой справки помогает
учащимся избежать повторного вывода формул каждый
раз.}\label{b.1-ux442ux430ux431ux43bux438ux446ux430-ux43fux440ux43eux438ux437ux432ux43eux434ux43dux44bux445ux43fux440ux43eux438ux437ux432ux43eux434ux43dux44bux435-ux438ux437ux43cux435ux440ux44fux44eux442-ux441ux43aux43eux440ux43eux441ux442ux44c-ux438ux437ux43cux435ux43dux435ux43dux438ux44f-ux438-ux43dux430ux43aux43bux43eux43dux44b-ux444ux443ux43dux43aux446ux438ux439.-ux43dux430ux43bux438ux447ux438ux435-ux442ux430ux431ux43bux438ux446ux44b-ux431ux44bux441ux442ux440ux43eux439-ux441ux43fux440ux430ux432ux43aux438-ux43fux43eux43cux43eux433ux430ux435ux442-ux443ux447ux430ux449ux438ux43cux441ux44f-ux438ux437ux431ux435ux436ux430ux442ux44c-ux43fux43eux432ux442ux43eux440ux43dux43eux433ux43e-ux432ux44bux432ux43eux434ux430-ux444ux43eux440ux43cux443ux43b-ux43aux430ux436ux434ux44bux439-ux440ux430ux437.}

\paragraph{Основные
правила}\label{ux43eux441ux43dux43eux432ux43dux44bux435-ux43fux440ux430ux432ux438ux43bux430-1}

\begin{enumerate}
\def\labelenumi{\arabic{enumi}.}
\tightlist
\item
  Постоянное правило
\end{enumerate}

\[
\frac{d}{dx}[c] = 0
\]

\begin{enumerate}
\def\labelenumi{\arabic{enumi}.}
\setcounter{enumi}{1}
\tightlist
\item
  Правило власти
\end{enumerate}

\[
\frac{d}{dx}[x^n] = nx^{n-1}, \quad (n \in \mathbb{R})
\]

\begin{enumerate}
\def\labelenumi{\arabic{enumi}.}
\setcounter{enumi}{2}
\tightlist
\item
  Постоянное множественное правило
\end{enumerate}

\[
\frac{d}{dx}[c f(x)] = c f'(x)
\]

\begin{enumerate}
\def\labelenumi{\arabic{enumi}.}
\setcounter{enumi}{3}
\tightlist
\item
  Правило суммы и разности
\end{enumerate}

\[
\frac{d}{dx}[f(x)\pm g(x)] = f'(x)\pm g'(x)
\]

\paragraph{Тригонометрические
функции}\label{ux442ux440ux438ux433ux43eux43dux43eux43cux435ux442ux440ux438ux447ux435ux441ux43aux438ux435-ux444ux443ux43dux43aux446ux438ux438}

\[
\frac{d}{dx}[\sin x] = \cos x
\]

\[
\frac{d}{dx}[\cos x] = -\sin x
\]

\[
\frac{d}{dx}[\tan x] = \sec^2 x, \quad x \neq \tfrac{\pi}{2}+k\pi
\]

\[
\frac{d}{dx}[\cot x] = -\csc^2 x
\]

\[
\frac{d}{dx}[\sec x] = \sec x \tan x
\]

\[
\frac{d}{dx}[\csc x] = -\csc x \cot x
\]

\paragraph{Экспоненциальные и логарифмические
функции}\label{ux44dux43aux441ux43fux43eux43dux435ux43dux446ux438ux430ux43bux44cux43dux44bux435-ux438-ux43bux43eux433ux430ux440ux438ux444ux43cux438ux447ux435ux441ux43aux438ux435-ux444ux443ux43dux43aux446ux438ux438}

\[
\frac{d}{dx}[e^x] = e^x
\]

\[
\frac{d}{dx}[a^x] = a^x \ln a, \quad a>0, a\neq 1
\]

\[
\frac{d}{dx}[\ln x] = \frac{1}{x}, \quad x>0
\]

\[
\frac{d}{dx}[\log_a x] = \frac{1}{x\ln a}, \quad a>0, a\neq 1
\]

\paragraph{Обратные тригонометрические
функции}\label{ux43eux431ux440ux430ux442ux43dux44bux435-ux442ux440ux438ux433ux43eux43dux43eux43cux435ux442ux440ux438ux447ux435ux441ux43aux438ux435-ux444ux443ux43dux43aux446ux438ux438}

\[
\frac{d}{dx}[\arcsin x] = \frac{1}{\sqrt{1-x^2}}, \quad |x|<1
\]

\[
\frac{d}{dx}[\arccos x] = -\frac{1}{\sqrt{1-x^2}}, \quad |x|<1
\]

\[
\frac{d}{dx}[\arctan x] = \frac{1}{1+x^2}, \quad x \in \mathbb{R}
\]

\paragraph{Правила произведения, частного и
цепочки}\label{ux43fux440ux430ux432ux438ux43bux430-ux43fux440ux43eux438ux437ux432ux435ux434ux435ux43dux438ux44f-ux447ux430ux441ux442ux43dux43eux433ux43e-ux438-ux446ux435ux43fux43eux447ux43aux438}

\begin{enumerate}
\def\labelenumi{\arabic{enumi}.}
\tightlist
\item
  Правило продукта
\end{enumerate}

\[
\frac{d}{dx}[f(x)g(x)] = f'(x)g(x)+f(x)g'(x)
\]

\begin{enumerate}
\def\labelenumi{\arabic{enumi}.}
\setcounter{enumi}{1}
\tightlist
\item
  Правило частного
\end{enumerate}

\[
\frac{d}{dx}\left[\frac{f(x)}{g(x)}\right] = \frac{f'(x)g(x)-f(x)g'(x)}{[g(x)]^2}, \quad g(x)\neq 0
\]

\begin{enumerate}
\def\labelenumi{\arabic{enumi}.}
\setcounter{enumi}{2}
\tightlist
\item
  Правило цепочки
\end{enumerate}

\[
\frac{d}{dx}[f(g(x))] = f'(g(x))\cdot g'(x)
\]

\subsubsection{B.3 Расширения общих
серий}\label{b.3-ux440ux430ux441ux448ux438ux440ux435ux43dux438ux44f-ux43eux431ux449ux438ux445-ux441ux435ux440ux438ux439}

Степенные ряды позволяют выразить функции в виде бесконечных
многочленов. Эти расширения необходимы для аппроксимации, решения
дифференциальных уравнений и построения интуиции о функциях в
исчислении.

\paragraph{Геометрическая
серия}\label{ux433ux435ux43eux43cux435ux442ux440ux438ux447ux435ux441ux43aux430ux44f-ux441ux435ux440ux438ux44f}

\[
\frac{1}{1-x} = \sum_{n=0}^\infty x^n, \quad |x| < 1
\]

\paragraph{Экспоненциальная
функция}\label{ux44dux43aux441ux43fux43eux43dux435ux43dux446ux438ux430ux43bux44cux43dux430ux44f-ux444ux443ux43dux43aux446ux438ux44f}

\[
e^x = \sum_{n=0}^\infty \frac{x^n}{n!}
= 1 + x + \frac{x^2}{2!} + \frac{x^3}{3!} + \cdots
\]

Действительно для всех \(x\).

\paragraph{Тригонометрические
функции}\label{ux442ux440ux438ux433ux43eux43dux43eux43cux435ux442ux440ux438ux447ux435ux441ux43aux438ux435-ux444ux443ux43dux43aux446ux438ux438-1}

\[
\sin x = \sum_{n=0}^\infty (-1)^n \frac{x^{2n+1}}{(2n+1)!}
= x - \frac{x^3}{3!} + \frac{x^5}{5!} - \cdots
\]

\[
\cos x = \sum_{n=0}^\infty (-1)^n \frac{x^{2n}}{(2n)!}
= 1 - \frac{x^2}{2!} + \frac{x^4}{4!} - \cdots
\]

\[
\tan^{-1} x = \sum_{n=0}^\infty (-1)^n \frac{x^{2n+1}}{2n+1}, \quad |x|\leq 1
\]

\paragraph{Логарифм}\label{ux43bux43eux433ux430ux440ux438ux444ux43c}

\[
\ln(1+x) = \sum_{n=1}^\infty (-1)^{n+1} \frac{x^n}{n}, \quad -1 < x \leq 1
\]

\paragraph{Биномиальное разложение
(обобщенное)}\label{ux431ux438ux43dux43eux43cux438ux430ux43bux44cux43dux43eux435-ux440ux430ux437ux43bux43eux436ux435ux43dux438ux435-ux43eux431ux43eux431ux449ux435ux43dux43dux43eux435}

\[
(1+x)^r = \sum_{n=0}^\infty \binom{r}{n} x^n, \quad |x|<1
\]

где

\[\binom{r}{n} = \frac{r(r-1)(r-2)\cdots(r-n+1)}{n!}.
\]

\subsection{Appendix C. Proof
Sketches}\label{appendix-c.-proof-sketches}

\subsubsection{\texorpdfstring{C.1 Limit Laws and the
\(\varepsilon\)--\(\delta\)
Definition}{C.1 Limit Laws and the \textbackslash varepsilon--\textbackslash delta Definition}}\label{c.1-limit-laws-and-the-varepsilondelta-definition}

Calculus rests on the precise meaning of a limit. While intuition
(``values get closer and closer'') is helpful, a formal definition
ensures rigor and avoids paradoxes.

\paragraph{Intuitive Idea}\label{intuitive-idea}

We write

\[
\lim_{x \to a} f(x) = L
\]

to mean that as \(x\) gets arbitrarily close to \(a\), the values of
\(f(x)\) get arbitrarily close to \(L\).

\paragraph{\texorpdfstring{Formal (\(\varepsilon\)--\(\delta\))
Definition}{Formal (\textbackslash varepsilon--\textbackslash delta) Definition}}\label{formal-varepsilondelta-definition}

We say that

\[
\lim_{x \to a} f(x) = L
\]

if for every \(\varepsilon > 0\), there exists a \(\delta > 0\) such
that whenever

\[
0 < |x-a| <\дельта,
\]

we have

\[
|f(x) - L| <\варепсилон.
\]

\begin{itemize}
\tightlist
\item
  \(\varepsilon\): how close we want \(f(x)\) to be to \(L\).
\item
  \(\delta\): how close \(x\) must be to \(a\) to achieve that.
\end{itemize}

\paragraph{Example}\label{example}

Show that

\[
\lim_{x \to 2} (3x+1) = 7.
\]

\begin{itemize}
\tightlist
\item
  Let \(\varepsilon > 0\).
\item
  We want \(|(3x+1)-7| < \varepsilon\).
\item
  Simplify: \(|3x-6| = 3|x-2| < \varepsilon\).
\item
  This holds if we choose \(\delta = \varepsilon/3\).
\end{itemize}

Thus, by the definition, the limit is 7.

\paragraph{Limit Laws}\label{limit-laws}

If \(\lim_{x \to a} f(x) = L\) and \(\lim_{x \to a} g(x) = M\), then:

\begin{enumerate}
\def\labelenumi{\arabic{enumi}.}
\tightlist
\item
  Sum/Difference
\end{enumerate}

\[
\lim_{x \to a} [f(x) \pm g(x)] = L \pm M
\]

\begin{enumerate}
\def\labelenumi{\arabic{enumi}.}
\setcounter{enumi}{1}
\tightlist
\item
  Constant Multiple
\end{enumerate}

\[
\lim_{x \to a} [c f(x)] = cL
\]

\begin{enumerate}
\def\labelenumi{\arabic{enumi}.}
\setcounter{enumi}{2}
\tightlist
\item
  Product
\end{enumerate}

\[
\lim_{x \to a} [f(x)g(x)] = LM
\]

\begin{enumerate}
\def\labelenumi{\arabic{enumi}.}
\setcounter{enumi}{3}
\tightlist
\item
  Quotient (if \(M \neq 0\))
\end{enumerate}

\[
\lim_{x \to a} \frac{f(x)}{g(x)} = \frac{L}{M}
\]

\begin{enumerate}
\def\labelenumi{\arabic{enumi}.}
\setcounter{enumi}{4}
\tightlist
\item
  Powers and Roots
\end{enumerate}

\[
\lim_{x \to a} [f(x)]^n = L^n, \quad \lim_{x \to a} \sqrt[n]{f(x)} = \sqrt[n]{L} \ (\text{если определено}).
\]

\subsubsection{C.2 Proof Sketch: The Fundamental Theorem of
Calculus}\label{c.2-proof-sketch-the-fundamental-theorem-of-calculus}

The Fundamental Theorem of Calculus (FTC) links the two central
operations of calculus: differentiation and integration. It shows that
they are, in fact, inverse processes.

\paragraph{Statement of the Theorem}\label{statement-of-the-theorem}

Part I (Differentiation of an Integral): If \(f\) is continuous on
\([a,b]\) and we define

\[
F(x) = \int_a^x f(t)\,dt,
\]

then \(F\) is differentiable on \((a,b)\) and

\[
F'(х) = f(x).
\]

Part II (Evaluation of a Definite Integral): If \(F\) is any
antiderivative of \(f\) on \([a,b]\), then

\[
\int_a^b f(x)\,dx = F(b)-F(a).
\]

\paragraph{Proof Sketch of Part I}\label{proof-sketch-of-part-i}

\begin{enumerate}
\def\labelenumi{\arabic{enumi}.}
\item
  Start with the definition of the derivative:

  \[
  F'(x) = \lim_{h\to 0} \frac{F(x+h)-F(x)}{h}.
  \]
\item
  Substituting \(F(x) = \int_a^x f(t)\,dt\):

  \[
  F(x+h)-F(x) = \int_a^{x+h} f(t)\,dt - \int_a^x f(t)\,dt.
  \]
\item
  By the additivity of integrals:

  \[
  F(x+h)-F(x) = \int_x^{x+h} f(t)\,dt.
  \]
\item
  Therefore:

  \[
  \frac{F(x+h)-F(x)}{h} = \frac{1}{h}\int_x^{x+h} f(t)\,dt.
  \]5. По теореме о среднем значении для интегралов существует
  \(c \in [x,x+h]\) такой, что

  \[
  \frac{1}{h}\int_x^{x+h} f(t)\,dt = f(c).
  \]
\item
  Поскольку \(h \to 0\), \(c \to x\) и поскольку \(f\) является
  непрерывным:

  \[
  \lim_{h\to 0} f(c) = f(x).
  \]
\end{enumerate}

Таким образом, \(F'(x) = f(x)\).

\paragraph{Пробный эскиз части
II}\label{ux43fux440ux43eux431ux43dux44bux439-ux44dux441ux43aux438ux437-ux447ux430ux441ux442ux438-ii}

\begin{enumerate}
\def\labelenumi{\arabic{enumi}.}
\item
  Пусть \(F\) является производной от \(f\), поэтому \(F'(x) = f(x)\).
\item
  Согласно части I функция

  \[
  G(x) = \int_a^x f(t)\,dt
  \]

  также является производным от \(f\).
\item
  Поскольку \(F\) и \(G\) отличаются только константой,

  \[
  F(x) = G(x) + C.
  \]
\item
  Оценка на конечных точках:

  \[
  \int_a^b f(x)\,dx = G(b)-G(a) = (F(b)+C)-(F(a)+C) = F(b)-F(a).
  \]
\end{enumerate}

\subsubsection{C.3 Схема доказательства: сходимость геометрического
ряда}\label{c.3-ux441ux445ux435ux43cux430-ux434ux43eux43aux430ux437ux430ux442ux435ux43bux44cux441ux442ux432ux430-ux441ux445ux43eux434ux438ux43cux43eux441ux442ux44c-ux433ux435ux43eux43cux435ux442ux440ux438ux447ux435ux441ux43aux43eux433ux43e-ux440ux44fux434ux430}

Геометрическая серия --- одна из самых простых и важных бесконечных
серий. Он служит моделью для понимания конвергенции и является основой
для многих последующих результатов в исчислении.

\paragraph{Сериал}\label{ux441ux435ux440ux438ux430ux43b}

\[
\sum_{n=0}^\infty ar^n = a + ar + ar^2 + ar^3 + \cdots
\]

где \(a\) --- первый член, а \(r\) --- обычное соотношение.

\paragraph{Формула частичной
суммы}\label{ux444ux43eux440ux43cux443ux43bux430-ux447ux430ux441ux442ux438ux447ux43dux43eux439-ux441ux443ux43cux43cux44b}

\(n\)-я частичная сумма равна

\[
S_n = a + ar + ar^2 + \cdots + ar^n.
\]

Умножьте обе части на \(r\):

\[
rS_n = ar + ar^2 + \cdots + ar^{n+1}.
\]

Вычтите два уравнения:

\[
S_n - rS_n = a - ar^{n+1}.
\]

\[
S_n(1-r) = a(1-r^{n+1}).
\]

Итак

\[
S_n = \frac{a(1-r^{n+1})}{1-r}, \quad r \neq 1.
\]

\paragraph{Конвергенция}\label{ux43aux43eux43dux432ux435ux440ux433ux435ux43dux446ux438ux44f}

Примите лимит как \(n \to \infty\):

\begin{itemize}
\item
  Если \(|r| < 1\), то \(r^{n+1} \to 0\).

  \[
  \lim_{n\to\infty} S_n = \frac{a}{1-r}.
  \]
\item
  Если \(|r| \geq 1\), то \(r^{n+1}\) не переходит в 0. Ряд расходится.
\end{itemize}

\paragraph{Результат}\label{ux440ux435ux437ux443ux43bux44cux442ux430ux442}

\[
\sum_{n=0}^\infty ar^n =
\begin{cases}
\dfrac{a}{1-r}, & |r|<1, \\[6pt]
\text{diverges}, & |r|\geq 1.
\end{cases}
\]

\subsection{Приложение D. Приложения и
подключения}\label{ux43fux440ux438ux43bux43eux436ux435ux43dux438ux435-d.-ux43fux440ux438ux43bux43eux436ux435ux43dux438ux44f-ux438-ux43fux43eux434ux43aux43bux44eux447ux435ux43dux438ux44f}

\subsubsection{D.1 Физические связи: скорость, ускорение и
работа}\label{d.1-ux444ux438ux437ux438ux447ux435ux441ux43aux438ux435-ux441ux432ux44fux437ux438-ux441ux43aux43eux440ux43eux441ux442ux44c-ux443ux441ux43aux43eux440ux435ux43dux438ux435-ux438-ux440ux430ux431ux43eux442ux430}

Исчисление изначально было разработано для решения физических задач,
особенно движения и изменений. Вот некоторые из наиболее важных связей.

\paragraph{Положение, скорость и
ускорение}\label{ux43fux43eux43bux43eux436ux435ux43dux438ux435-ux441ux43aux43eux440ux43eux441ux442ux44c-ux438-ux443ux441ux43aux43eux440ux435ux43dux438ux435-1}

\begin{itemize}
\item
  Функция положения: \(s(t)\) определяет местоположение объекта в момент
  времени \(t\).
\item
  Скорость: производная положения.

  \[
  v(t) = s'(t) = \frac{ds}{dt}
  \]
\item
  Ускорение: производная скорости (или вторая производная положения).

  \[
  a(t) = v'(t) = s''(t) = \frac{d^2s}{dt^2}
  \]
\end{itemize}

Пример: Если \(s(t) = 4t^2\) метров, то:

\[
v(t) = 8t, \quad a(t) = 8.
\]

Таким образом, объект движется быстрее, линейно со временем, с
постоянным ускорением.

\paragraph{Работа и
сила}\label{ux440ux430ux431ux43eux442ux430-ux438-ux441ux438ux43bux430}

В физике работа --- это произведение силы и расстояния. Если сила
меняется в зависимости от положения, расчет дает:

\[W = \int_a^b F(x)\, dx
\]

where \(F(x)\) is the force at position \(x\), and the object moves from
\(x=a\) to \(x=b\).

Example: A spring with Hooke's law force \(F(x) = kx\) requires work

\[
W = \int_0^d kx\, dx = \frac{1}{2}kd^2
\]

to stretch the spring a distance \(d\).

\paragraph{Energy and Areas Under
Curves}\label{energy-and-areas-under-curves}

\begin{itemize}
\tightlist
\item
  Kinetic energy: \(E_k = \tfrac{1}{2}mv^2\).
\item
  Potential energy often involves integrals (e.g., gravitational
  potential energy from force of gravity).
\item
  In general, integrating a force function gives energy stored or work
  done.
\end{itemize}

\paragraph{Quick Practice}\label{quick-practice}

\begin{enumerate}
\def\labelenumi{\arabic{enumi}.}
\tightlist
\item
  If \(s(t) = t^3 - 3t\), find \(v(t)\) and \(a(t)\).
\item
  Compute the work done by a constant force of 10 N moving an object 5
  m.
\item
  A spring has constant \(k=200\). How much work is needed to stretch it
  0.1 m?
\item
  Show that acceleration is the second derivative of position.
\item
  Explain how the integral \(\int v(t)\, dt\) relates to displacement.
\end{enumerate}

\subsubsection{D.2 Probability and Statistics
Connections}\label{d.2-probability-and-statistics-connections}

Calculus is deeply connected with probability and statistics, especially
when dealing with continuous random variables. Integrals become
essential for defining probabilities, averages, and expectations.

\paragraph{Probability Density Functions
(PDFs)}\label{probability-density-functions-pdfs}

For a continuous random variable \(X\), probabilities are described by a
probability density function \(f(x)\):

\begin{enumerate}
\def\labelenumi{\arabic{enumi}.}
\item
  \(f(x) \geq 0\) for all \(x\).
\item
  Total probability equals 1:

  \[
  \int_{-\infty}^{\infty} f(x)\, dx = 1.
  \]
\end{enumerate}

The probability that \(X\) lies in an interval \([a,b]\) is

\[
P(a \leq X \leq b) = \int_a^b f(x)\, dx.
\]

\paragraph{Expected Value (Mean)}\label{expected-value-mean}

The expected value (average outcome) is

\[
E[X] = \int_{-\infty}^{\infty} x f(x)\, dx.
\]

This is the calculus version of a weighted average.

\paragraph{Variance}\label{variance}

Variance measures spread:

\[
\text{Var}(X) = E[(X-\mu)^2] = \int_{-\infty}^{\infty} (x-\mu)^2 f(x)\, dx,
\]

where \(\mu = E[X]\).

\paragraph{Common Distributions}\label{common-distributions}

\begin{enumerate}
\def\labelenumi{\arabic{enumi}.}
\item
  Uniform distribution on \([a,b]\):

  \[
  f(x) = \frac{1}{b-a}, \quad a \leq x \leq b.
  \]

  Mean: \(\frac{a+b}{2}\).
\item
  Exponential distribution with parameter \(\lambda > 0\):

  \[
  f(x) = \lambda e^{-\lambda x}, \quad x \geq 0.
  \]

  Mean: \(1/\lambda\).
\item
  Normal (Gaussian) distribution:

  \[
  f(x) = \frac{1}{\sqrt{2\pi\sigma^2}} e^{-(x-\mu)^2/(2\sigma^2)}.
  \]

  Интегралы этого распределения связаны с функцией ошибок.
\end{enumerate}

\paragraph{Почему это
важно}\label{ux43fux43eux447ux435ux43cux443-ux44dux442ux43e-ux432ux430ux436ux43dux43e-31}

\begin{itemize}
\tightlist
\item
  Интегралы превращают вероятности в площади под кривыми.
\item
  Ожидание и дисперсия связывают исчисление со средними значениями и
  изменчивостью.
\item
  Большинство реальных моделей данных (финансы, физика, биология,
  искусственный интеллект) используют эти непрерывные распределения
  вероятностей.
\end{itemize}

\paragraph{\texorpdfstring{Быстрая практика1. Для
\(f(x) = \tfrac{1}{2}\) на \([0,2]\) вычислите
\(P(0.5 \leq X \leq 1.5)\).}{Быстрая практика1. Для f(x) = \textbackslash tfrac\{1\}\{2\} на {[}0,2{]} вычислите P(0.5 \textbackslash leq X \textbackslash leq 1.5).}}\label{ux431ux44bux441ux442ux440ux430ux44f-ux43fux440ux430ux43aux442ux438ux43aux4301.-ux434ux43bux44f-fx-tfrac12-ux43dux430-02-ux432ux44bux447ux438ux441ux43bux438ux442ux435-p0.5-leq-x-leq-1.5.}

\begin{enumerate}
\def\labelenumi{\arabic{enumi}.}
\setcounter{enumi}{1}
\tightlist
\item
  Для экспоненциального распределения с \(\lambda = 2\) вычислите
  \(E[X]\).
\item
  Докажите, что общая площадь под стандартной нормальной кривой равна 1.
\item
  Найдите среднее значение равномерного распределения на \([3,7]\).
\item
  Объясните, почему для непрерывных переменных вероятности вычисляются с
  помощью интегралов, а не сумм.
\end{enumerate}

\subsubsection{D.3 Связь с информатикой: аппроксимации Тейлора в
алгоритмах}\label{d.3-ux441ux432ux44fux437ux44c-ux441-ux438ux43dux444ux43eux440ux43cux430ux442ux438ux43aux43eux439-ux430ux43fux43fux440ux43eux43aux441ux438ux43cux430ux446ux438ux438-ux442ux435ux439ux43bux43eux440ux430-ux432-ux430ux43bux433ux43eux440ux438ux442ux43cux430ux445}

Исчисление предназначено не только для физики --- оно также лежит в
основе многих инструментов и методов информатики. Один из самых
очевидных мостов --- через ряды Тейлора, которые обеспечивают
эффективные способы аппроксимации функций в численных вычислениях и
алгоритмах.

\paragraph{Аппроксимация функций для
вычислений}\label{ux430ux43fux43fux440ux43eux43aux441ux438ux43cux430ux446ux438ux44f-ux444ux443ux43dux43aux446ux438ux439-ux434ux43bux44f-ux432ux44bux447ux438ux441ux43bux435ux43dux438ux439}

Компьютеры не могут напрямую хранить или точно вычислять большинство
функций (например, \(e^x\), \(\sin x\) или \(\ln x\)). Вместо этого они
используют полиномиальные аппроксимации, полученные на основе разложений
Тейлора.

Пример: Чтобы приблизить \(e^x\), усеките ряд Маклорена:

\[
e^x \approx 1 + x + \frac{x^2}{2!} + \frac{x^3}{3!}.
\]

Для небольшого \(x\) этот полином дает точные результаты всего с
несколькими членами.

\paragraph{Эффективность в
алгоритмах}\label{ux44dux444ux444ux435ux43aux442ux438ux432ux43dux43eux441ux442ux44c-ux432-ux430ux43bux433ux43eux440ux438ux442ux43cux430ux445}

\begin{itemize}
\tightlist
\item
  Тригонометрические функции: алгоритмы для калькуляторов и процессоров
  часто используют разложение в ряд (или его вариации, такие как
  полиномы Чебышева).
\item
  Экспонента/логарифм: разложения Тейлора являются основой быстрых
  аппроксимаций в числовых библиотеках.
\item
  Нахождение корня: метод Ньютона основан на линейной аппроксимации,
  прямом применении ряда Тейлора (первая производная).
\end{itemize}

\paragraph{Численный
анализ}\label{ux447ux438ux441ux43bux435ux43dux43dux44bux439-ux430ux43dux430ux43bux438ux437}

Разложения Тейлора играют центральную роль в анализе ошибок:

\begin{itemize}
\item
  Аппроксимация члена ошибки с использованием формулы остатка:

  \[
  R_n(x) = \frac{f^{(n+1)}(c)}{(n+1)!}(x-a)^{n+1}.
  \]
\item
  Это говорит нам, сколько терминов необходимо для заданной точности.
\end{itemize}

\paragraph{Связи с машинным
обучением}\label{ux441ux432ux44fux437ux438-ux441-ux43cux430ux448ux438ux43dux43dux44bux43c-ux43eux431ux443ux447ux435ux43dux438ux435ux43c}

\begin{itemize}
\tightlist
\item
  Оптимизация на основе градиента (например, градиентный спуск)
  использует производные для эффективного обновления параметров.
\item
  Функции активации (например, \(\tanh x\) или
  \(\sigma(x)=1/(1+e^{-x})\)) часто аппроксимируются полиномами или
  кусочными функциями для повышения скорости.
\item
  Рядовые аппроксимации могут ускорить обучение и вывод в ограниченных
  средах.
\end{itemize}

\paragraph{Почему это
важно}\label{ux43fux43eux447ux435ux43cux443-ux44dux442ux43e-ux432ux430ux436ux43dux43e-32}

\begin{itemize}
\tightlist
\item
  Аппроксимации Тейлора соединяют непрерывную математику с дискретными
  вычислениями.
\item
  Они показывают, как концепции исчисления используются в алгоритмах,
  численных методах и машинном обучении.
\item
  Понимание приближений помогает избежать ошибок при использовании
  компьютеров для расчетов.
\end{itemize}

\paragraph{Быстрая
практика}\label{ux431ux44bux441ux442ux440ux430ux44f-ux43fux440ux430ux43aux442ux438ux43aux430}

\begin{enumerate}
\def\labelenumi{\arabic{enumi}.}
\tightlist
\item
  Приблизительный \(\sin(0.1)\), используя первые три члена ряда
  Маклорена.2. Используйте остаточный член, чтобы оценить ошибку при
  аппроксимации \(e^1\) полиномом 3-й степени.
\item
  Объясните, как метод Ньютона использует теорему Тейлора.
\item
  Почему компьютеры могут предпочитать полиномиальные приближения точным
  формулам для функций?
\item
  Почему в машинном обучении производная (градиент) так важна для
  оптимизации?
\end{enumerate}




\end{document}
