% Options for packages loaded elsewhere
\PassOptionsToPackage{unicode}{hyperref}
\PassOptionsToPackage{hyphens}{url}
\PassOptionsToPackage{dvipsnames,svgnames,x11names}{xcolor}
%
\documentclass[
  letterpaper,
  DIV=11,
  numbers=noendperiod]{scrartcl}

\usepackage{amsmath,amssymb}
\usepackage{iftex}
\ifPDFTeX
  \usepackage[T1]{fontenc}
  \usepackage[utf8]{inputenc}
  \usepackage{textcomp} % provide euro and other symbols
\else % if luatex or xetex
  \usepackage{unicode-math}
  \defaultfontfeatures{Scale=MatchLowercase}
  \defaultfontfeatures[\rmfamily]{Ligatures=TeX,Scale=1}
\fi
\usepackage{lmodern}
\ifPDFTeX\else  
    % xetex/luatex font selection
\fi
%% Support for zero-width non-joiner characters.
\makeatletter
\def\zerowidthnonjoiner{%
  % Prevent ligatures and adjust kerning, but still support hyphenating.
  \texorpdfstring{%
    \TextOrMath{\nobreak\discretionary{-}{}{\kern.03em}%
      \ifvmode\else\nobreak\hskip\z@skip\fi}{}%
  }{}%
}
\makeatother
\ifPDFTeX
  \DeclareUnicodeCharacter{200C}{\zerowidthnonjoiner}
\else
  \catcode`^^^^200c=\active
  \protected\def ^^^^200c{\zerowidthnonjoiner}
\fi
%% End of ZWNJ support
% Use upquote if available, for straight quotes in verbatim environments
\IfFileExists{upquote.sty}{\usepackage{upquote}}{}
\IfFileExists{microtype.sty}{% use microtype if available
  \usepackage[]{microtype}
  \UseMicrotypeSet[protrusion]{basicmath} % disable protrusion for tt fonts
}{}
\makeatletter
\@ifundefined{KOMAClassName}{% if non-KOMA class
  \IfFileExists{parskip.sty}{%
    \usepackage{parskip}
  }{% else
    \setlength{\parindent}{0pt}
    \setlength{\parskip}{6pt plus 2pt minus 1pt}}
}{% if KOMA class
  \KOMAoptions{parskip=half}}
\makeatother
\usepackage{xcolor}
\setlength{\emergencystretch}{3em} % prevent overfull lines
\setcounter{secnumdepth}{-\maxdimen} % remove section numbering
% Make \paragraph and \subparagraph free-standing
\makeatletter
\ifx\paragraph\undefined\else
  \let\oldparagraph\paragraph
  \renewcommand{\paragraph}{
    \@ifstar
      \xxxParagraphStar
      \xxxParagraphNoStar
  }
  \newcommand{\xxxParagraphStar}[1]{\oldparagraph*{#1}\mbox{}}
  \newcommand{\xxxParagraphNoStar}[1]{\oldparagraph{#1}\mbox{}}
\fi
\ifx\subparagraph\undefined\else
  \let\oldsubparagraph\subparagraph
  \renewcommand{\subparagraph}{
    \@ifstar
      \xxxSubParagraphStar
      \xxxSubParagraphNoStar
  }
  \newcommand{\xxxSubParagraphStar}[1]{\oldsubparagraph*{#1}\mbox{}}
  \newcommand{\xxxSubParagraphNoStar}[1]{\oldsubparagraph{#1}\mbox{}}
\fi
\makeatother


\providecommand{\tightlist}{%
  \setlength{\itemsep}{0pt}\setlength{\parskip}{0pt}}\usepackage{longtable,booktabs,array}
\usepackage{calc} % for calculating minipage widths
% Correct order of tables after \paragraph or \subparagraph
\usepackage{etoolbox}
\makeatletter
\patchcmd\longtable{\par}{\if@noskipsec\mbox{}\fi\par}{}{}
\makeatother
% Allow footnotes in longtable head/foot
\IfFileExists{footnotehyper.sty}{\usepackage{footnotehyper}}{\usepackage{footnote}}
\makesavenoteenv{longtable}
\usepackage{graphicx}
\makeatletter
\newsavebox\pandoc@box
\newcommand*\pandocbounded[1]{% scales image to fit in text height/width
  \sbox\pandoc@box{#1}%
  \Gscale@div\@tempa{\textheight}{\dimexpr\ht\pandoc@box+\dp\pandoc@box\relax}%
  \Gscale@div\@tempb{\linewidth}{\wd\pandoc@box}%
  \ifdim\@tempb\p@<\@tempa\p@\let\@tempa\@tempb\fi% select the smaller of both
  \ifdim\@tempa\p@<\p@\scalebox{\@tempa}{\usebox\pandoc@box}%
  \else\usebox{\pandoc@box}%
  \fi%
}
% Set default figure placement to htbp
\def\fps@figure{htbp}
\makeatother

\KOMAoption{captions}{tableheading}
\makeatletter
\@ifpackageloaded{caption}{}{\usepackage{caption}}
\AtBeginDocument{%
\ifdefined\contentsname
  \renewcommand*\contentsname{Table of contents}
\else
  \newcommand\contentsname{Table of contents}
\fi
\ifdefined\listfigurename
  \renewcommand*\listfigurename{List of Figures}
\else
  \newcommand\listfigurename{List of Figures}
\fi
\ifdefined\listtablename
  \renewcommand*\listtablename{List of Tables}
\else
  \newcommand\listtablename{List of Tables}
\fi
\ifdefined\figurename
  \renewcommand*\figurename{Figure}
\else
  \newcommand\figurename{Figure}
\fi
\ifdefined\tablename
  \renewcommand*\tablename{Table}
\else
  \newcommand\tablename{Table}
\fi
}
\@ifpackageloaded{float}{}{\usepackage{float}}
\floatstyle{ruled}
\@ifundefined{c@chapter}{\newfloat{codelisting}{h}{lop}}{\newfloat{codelisting}{h}{lop}[chapter]}
\floatname{codelisting}{Listing}
\newcommand*\listoflistings{\listof{codelisting}{List of Listings}}
\makeatother
\makeatletter
\makeatother
\makeatletter
\@ifpackageloaded{caption}{}{\usepackage{caption}}
\@ifpackageloaded{subcaption}{}{\usepackage{subcaption}}
\makeatother

\ifLuaTeX
\usepackage[bidi=basic]{babel}
\else
\usepackage[bidi=default]{babel}
\fi
\babelprovide[main,import]{sanskrit}
% get rid of language-specific shorthands (see #6817):
\let\LanguageShortHands\languageshorthands
\def\languageshorthands#1{}
\usepackage{bookmark}

\IfFileExists{xurl.sty}{\usepackage{xurl}}{} % add URL line breaks if available
\urlstyle{same} % disable monospaced font for URLs
\hypersetup{
  pdftitle={गणितस्य लघु पुस्तकम्},
  pdflang={sa},
  colorlinks=true,
  linkcolor={blue},
  filecolor={Maroon},
  citecolor={Blue},
  urlcolor={Blue},
  pdfcreator={LaTeX via pandoc}}


\title{गणितस्य लघु पुस्तकम्}
\author{}
\date{}

\begin{document}
\maketitle


\section{गणितस्य लघु
पुस्तकम्}\label{ux917ux923ux924ux938ux92f-ux932ux918-ux92aux938ux924ux915ux92e}

गणितस्य मूलविचारानाम् एकः संक्षिप्तः, आरम्भक-अनुकूलः परिचयः।

\subsection{प्रारूप}\label{ux92aux930ux930ux92a}

\begin{itemize}
\tightlist
\item
  \href{../artifacts/sa/book.pdf}{Download PDF} -- मुद्रण-सज्ज संस्करण
\item
  \href{../artifacts/sa/book.epub}{Download EPUB} -- ई-पाठक अनुकूल
\item
  \href{../artifacts/sa/book.tex}{View LaTeX} -- लेटेक्स स्रोत
\end{itemize}

\section{भाग 1. सीमा
व्युत्पन्न}\label{ux92dux917-1.-ux938ux92e-ux935ux92fux924ux92aux928ux928}

\section{अध्याय 1. कार्याणि
सीमाश्च}\label{ux905ux927ux92fux92f-1.-ux915ux930ux92fux923-ux938ux92eux936ux91a}

\subsection{1.1 कार्याणि}\label{ux915ux930ux92fux923}

गणितस्य मूलभूतवस्तूनाम् एकं फंक्शन् अस्ति । तस्य हृदये फंक्शन् एकः नियमः अस्ति यः एकं निवेशं
गृहीत्वा सम्यक् एकं आउटपुट् उत्पादयति । कार्याणि सम्बन्धानां वर्णनं कुर्मः, वास्तविक-जगतः
घटनानां प्रतिरूपणं कुर्मः, गणितस्य सम्पूर्णं यन्त्रं च निर्मामः ।

\subsubsection{परिभाषा}\label{ux92aux930ux92dux937}

औपचारिकरूपेण, \(f\) सेट् \(X\) (डोमेन् इति उच्यते) सेट् \(Y\) (कोडोमेन् इति उच्यते)
यावत् एकं फंक्शन् लिख्यते

\[
f : X \to Y.
\] इति

प्रत्येकं \(x \in X\) तत्त्वस्य कृते \(f(x) \in Y\) इति अद्वितीयं तत्त्वं भवति ।
\(f(x)\) इति मूल्यं \(f\) इत्यस्य अन्तर्गतं \(x\) इत्यस्य प्रतिबिम्बं कथ्यते ।

यदि \(y = f(x)\), तर्हि \(y\) निवेशस्य \(x\) इत्यस्य अनुरूपं आउटपुट् अस्ति ।
यथार्थतः दृश्यमानानां सर्वेषां निर्गमानाम् समुच्चयः श्रेणी (कोडोमेनस्य उपसमूहः) इति उच्यते
।

\subsubsection{उदाहरणम्}\label{ux909ux926ux939ux930ux923ux92e}

\begin{enumerate}
\def\labelenumi{\arabic{enumi}.}
\item
  \(f(x) = x^2\) इति फंक्शन् प्रत्येकं वास्तविकसङ्ख्यां \(x\) इत्यस्य वर्गे मैप् करोति ।

  \begin{itemize}
  \tightlist
  \item
    डोमेन: सर्वाणि वास्तविकसङ्ख्याः \(\mathbb{R}\)।
  \item
    कोडोमेन: सर्वाणि वास्तविकसङ्ख्याः \(\mathbb{R}\)।
  \item
    श्रेणी: सर्वाणि अऋणात्मकानि वास्तविकसङ्ख्यानि \([0, \infty)\)।
  \end{itemize}
\item
  \(g(x) = \dfrac{1}{x}\) इति फंक्शन् प्रत्येकं अशून्यं वास्तविकसङ्ख्यां तस्य पारस्परिकं
  नियुक्तं करोति ।

  \begin{itemize}
  \tightlist
  \item
    डोमेन: \(\mathbb{R} \setminus \{0\}\)।
  \item
    श्रेणी: \(\mathbb{R} \setminus \{0\}\)।
  \end{itemize}
\item
  एकं वास्तविक-जगतः उदाहरणम् : \(T(t)\) इति समये \(t\) (घण्टेषु) समये बहिः
  तापमानं (°C मध्ये) भवतु । एतत् ``दिनस्य समयात्'' ``तापमानम्'' यावत् कार्यम् अस्ति
  ।
\end{enumerate}

\subsubsection{कार्यों का प्रतिनिधित्व के
उपाय}\label{ux915ux930ux92f-ux915-ux92aux930ux924ux928ux927ux924ux935-ux915-ux909ux92aux92f}

कार्याणि अनेकैः उपयोगिभिः प्रकारैः प्रतिनिधितुं शक्यन्ते : १.

\begin{itemize}
\tightlist
\item
  सूत्रम् : यथा, \(f(x) = \sin x + x^2\)।- आलेखाः : निर्देशांकविमानस्य सर्वेषां
  बिन्दूनां \((x, f(x))\) प्लॉट् करणं ।
\item
  सारणी: आँकडानां असततसमूहानां कृते निवेशानां निर्गमानाञ्च युग्मनम्।
\item
  मौखिकवर्णनानि : ``प्रत्येकं छात्रं स्वस्य ग्रेडं नियुक्तं कुर्वन्तु।''
\end{itemize}

प्रत्येकं प्रतिनिधित्वं एकस्यैव कार्यस्य भिन्नान् पक्षान् प्रकाशयति ।

\subsubsection{शब्दावली}\label{ux936ux92cux926ux935ux932}

\begin{itemize}
\tightlist
\item
  स्वतन्त्रचरः: निवेशः (सामान्यतया \(x\) इति लिखितः)।
\item
  आश्रितः चरः: आउटपुट् (सामान्यतया \(y\) इति लिखितम्, यत्र \(y = f(x)\))।
\item
  फंक्शन संकेतनम्: \(f(x)\) पठ्यते ``\(x\) इत्यस्य \(f\)।''
\end{itemize}

\subsubsection{गणनायां कार्याणां महत्त्वं
किमर्थम्}\label{ux917ux923ux928ux92f-ux915ux930ux92fux923-ux92eux939ux924ux924ux935-ux915ux92eux930ux925ux92e}

कार्याणि कथं परिवर्तन्ते इति अध्ययनं गणितम् । व्युत्पन्नाः परिवर्तनस्य क्षणिकदरं
मापयन्ति, अभिन्नाः तु सञ्चितप्रभावं मापयन्ति । एतेषां विचाराणां निपुणतायै प्रथमं
कार्याणि के सन्ति, तेषां व्यवहारः कथं भवति इति ठोसबोधः आवश्यकः ।

\subsubsection{अभ्यास}\label{ux905ux92dux92fux938}

\begin{enumerate}
\def\labelenumi{\arabic{enumi}.}
\item
  \(f(x) = 3x - 2\) इति कार्यस्य कृते :

  \begin{itemize}
  \tightlist
  \item
    डोमेन्, कोडमेन्, रेन्ज् च ज्ञातव्यम् ।
  \end{itemize}
\item
  \(h(x) = \sqrt{x-1}\) इति फंक्शन् कस्य इनपुट् कृते परिभाषितम् अस्ति? तस्य परिधिः
  कः ?
\item
  स्वस्य दैनन्दिनजीवनात् कस्यचित् कार्यस्य वास्तविकं उदाहरणं ददातु। डोमेन् कोडोमेन् च
  स्पष्टतया वदतु।
\item
  \(f(x) = |x|\) इत्यस्य आलेखं रेखांकयन्तु। परिधिः किम् ?
\item
  मानातु \(g(x) = \dfrac{1}{x^2+1}\)। तस्य परिधिः \((0, 1]\) इति अन्तरालः
  किमर्थम् इति व्याख्यातव्यम् ।
\end{enumerate}

\subsection{1.2 आलेखाः परिवर्तनानि
च}\label{ux906ux932ux916-ux92aux930ux935ux930ux924ux928ux928-ux91a}

न केवलं सूत्रेण अपितु तस्य आलेखेन अपि कार्यं ज्ञातुं शक्यते । \(f\) इत्यस्य फंक्शन् इत्यस्य
आलेखः सर्वेषां क्रमबद्धयुग्मानां \((x, f(x))\) इत्यस्य समुच्चयः अस्ति, यत्र \(x\) \(f\)
इत्यस्य डोमेनस्य अन्तर्भवति । एतेषां युग्मानां समन्वयविमानस्य प्लॉट् करणेन कार्यस्य
व्यवहारः कथं भवति इति चित्रं प्राप्यते ।

\subsubsection{मूलभूत आलेख}\label{ux92eux932ux92dux924-ux906ux932ux916}

केचन आलेखाः एतावन्तः मौलिकाः सन्ति यत् ते कण्ठस्थं कर्तव्यम्-

\begin{itemize}
\tightlist
\item
  \(f(x) = x\): उत्पत्तिद्वारा एकः सीधा रेखा।
\item
  \(f(x) = x^2\): ऊर्ध्वं उद्घाटितं परवलयम्।
\item
  \(f(x) = |x|\): ``V''-आकारस्य आलेखः ।
\item
  \(f(x) = \frac{1}{x}\): शाखाद्वययुक्तः अतिशयोक्तिः ।- \(f(x) = \sin x\):
  एक तरङ्गसदृश आवधिक वक्र।
\end{itemize}

एते अधिकजटिलकार्यस्य निर्माणखण्डरूपेण कार्यं कुर्वन्ति ।

\subsubsection{परिवर्तनम्}\label{ux92aux930ux935ux930ux924ux928ux92e}

सरलनियमानां उपयोगेन आलेखान् स्थानान्तरयितुं, तानितुं, प्रतिबिम्बितुं वा शक्यते:

\begin{enumerate}
\def\labelenumi{\arabic{enumi}.}
\item
  ऊर्ध्वाधरशिफ्ट् : स्थिरांकं योजयित्वा आलेखः उपरि अधः वा गच्छति ।

  \[
  y = f(x) + c \quad \text{is } f(x) \text{ shifted upward by } c.
  \] इति
\item
  क्षैतिजशिफ्ट् : तर्कस्य अन्तः योजयित्वा आलेखः वामभागे वा दक्षिणभागे वा गच्छति ।

  \[
  y = f(x - c) \quad \text{is } f(x) \text{ shifted right by } c.
  \]
\item
  ऊर्ध्वाधरमापनम् : नित्येन गुणनेन आलेखः लम्बवत् खिन्नः वा संपीडितः वा भवति ।

  \[
  y = a f(x), \quad a > 1 \text{ stretches; } 0 < a < 1 \text{ compresses.}
  \] इति
\item
  क्षैतिज-मापनम् : तर्कस्य अन्तः गुणनं कृत्वा आलेखं क्षैतिजरूपेण तानयति वा संपीडयति वा ।

  \[
  y = f(bx), \quad b > 1 \text{ compresses toward the } y\text{-axis}.
  \] इति
\item
  प्रतिबिम्बाः : १.

  \begin{itemize}
  \tightlist
  \item
    \(y = -f(x)\): \(x\)-अक्षस्य पारं प्रतिबिम्बम्।
  \item
    \(y = f(-x)\): \(y\)-अक्षस्य पारं प्रतिबिम्बम्।
  \end{itemize}
\end{enumerate}

\subsubsection{परिवर्तनं
संयोजयति}\label{ux92aux930ux935ux930ux924ux928-ux938ux92fux91cux92fux924}

जटिलाः आलेखाः प्रायः क्रमेण अनेकविकारानाम् संयोजनेन आगच्छन्ति । उदाहरणतया:

\[
y = 2(x-1)^2 + 3
\] इति

परवलयम् \(y = x^2\) गृहीत्वा, 1 द्वारा दक्षिणं स्थलं कृत्वा, 2 द्वारा लम्बवत्
तानयित्वा, 3 द्वारा ऊर्ध्वं स्थानान्तरणं कृत्वा प्राप्यते ।

\subsubsection{अभ्यास}\label{ux905ux92dux92fux938-1}

\begin{enumerate}
\def\labelenumi{\arabic{enumi}.}
\tightlist
\item
  \(y = (x+2)^2 - 1\) इत्यस्य आलेखं रेखांकयन्तु। \(y = x^2\) तः परिवर्तनस्य क्रमं
  चिनुत ।
\item
  \(y = f(x)\) इत्यस्य आलेखस्य किं भवति यदि वयं \(x\) इत्यस्य स्थाने \(-x\) इति
  स्थापयामः? \(f(x) = \sqrt{x}\) इत्यनेन सह प्रयतस्व।
\item
  \(y = \sin x\) इत्येतत् \(y = 3\sin(x - \pi/4)\) इति परिणमयन्ति ये
  परिवर्तनाः तेषां वर्णनं कुरुत।
\item
  \(y = |x-1| + 2\) इत्यस्य आलेखं रचयन्तु। प्रत्येकस्य शाखायाः तस्य शिखरं प्रवणतां च
  वदतु।
\item
  \(y = \frac{1}{x-2}\) कृते \(y = \frac{1}{x}\) इत्यस्य आलेखः कथं परिवर्तितः
  इति व्याख्यातव्यम् ।
\end{enumerate}

\subsection{1.3 सीमानां सहजविचारःबहुषु परिस्थितिषु कस्मिन्चित् बिन्दौ फंक्शन् इत्यस्य
मूल्यं तस्य बिन्दुस्य समीपे गृह्णाति मूल्येभ्यः न्यूनं भवति । सीमायाः अवधारणा एतत् विचारं
गृह्णाति
।}\label{ux938ux92eux928-ux938ux939ux91cux935ux91aux930ux92cux939ux937-ux92aux930ux938ux925ux924ux937-ux915ux938ux92eux928ux91aux924-ux92cux928ux926-ux92bux915ux936ux928-ux907ux924ux92fux938ux92f-ux92eux932ux92f-ux924ux938ux92f-ux92cux928ux926ux938ux92f-ux938ux92eux92a-ux917ux939ux923ux924-ux92eux932ux92fux92dux92f-ux928ux92fux928-ux92dux935ux924-ux938ux92eux92f-ux905ux935ux927ux930ux923-ux90fux924ux924-ux935ux91aux930-ux917ux939ux923ux924}

\subsubsection{कस्यचित् मूल्यस्य
समीपगमनम्}\label{ux915ux938ux92fux91aux924-ux92eux932ux92fux938ux92f-ux938ux92eux92aux917ux92eux928ux92e}

कल्पयतु भित्तिं प्रति गच्छति। स्पर्शात् पूर्वमपि त्वं समीपं समीपं गच्छसि । तथैव यथा \(x\)
\(a\) सङ्ख्यायाः समीपं गच्छति तथा \(f(x)\) इत्यस्य मूल्यानि कस्यापि संख्यायाः \(L\)
इत्यस्य समीपं गन्तुं शक्नुवन्ति । तदा वयं वदामः-

\[
\lim_{x \to a} f(x) = L.
\] इति

एतेन एतत् विचारं व्यक्तं भवति यत् \(f(x)\) इत्येतत् यथा वयं \(L\) इच्छामः तथा समीपे
कर्तुं शक्यते, केवलं \(x\) इत्यस्य \(a\) इत्यस्य पर्याप्तं समीपं गृहीत्वा ।

\subsubsection{उदाहरणम्}\label{ux909ux926ux939ux930ux923ux92e-1}

\begin{enumerate}
\def\labelenumi{\arabic{enumi}.}
\item
  \(f(x) = 2x + 3\) कृते : यथा \(x \to 1\), \(f(x) \to 5\)।
\item
  \(f(x) = \dfrac{\sin x}{x}\) कृते : \(x \to 0\) इति रूपेण, फंक्शन् 1 इत्यस्य
  समीपं गच्छति, यद्यपि \(f(0)\) परिभाषितं नास्ति ।
\item
  \(f(x) = \dfrac{1}{x}\) कृते : यथा \(x \to 0^+\) (दक्षिणतः समीपं गच्छन्),
  \(f(x) \to +\infty\) । यथा \(x \to 0^-\) (वामतः उपसृत्य),
  \(f(x) \to -\infty\) । वामदक्षिणव्यवहारयोः भिन्नत्वात् 0 इत्यत्र सीमा नास्ति ।
\end{enumerate}

\subsubsection{सीमाओं का
महत्त्व}\label{ux938ux92eux913-ux915-ux92eux939ux924ux924ux935}

\begin{itemize}
\tightlist
\item
  ते अस्मान् तेषु बिन्दुषु कार्याणि परिभाषितुं शक्नुवन्ति यत्र ते मूलतः परिभाषिताः न सन्ति
  ।
\item
  ते असंतुलनानां एकलत्वानां च समीपे व्यवहारं गृह्णन्ति।
\item
  ते व्युत्पन्नस्य (परिवर्तनस्य तत्क्षणिकदराः) अभिन्नस्य (योगस्य सीमारूपेण क्षेत्राणि) च
  आधारं निर्मान्ति ।
\end{itemize}

\subsubsection{एकपक्षीय
सीमा}\label{ux90fux915ux92aux915ux937ux92f-ux938ux92e}

कदाचित् वामदक्षिणयोः व्यवहारस्य पृथक् पृथक् अध्ययनं करणीयम् : १.

\[
\lim_{x \to a^-} f(x), \quad \lim_{x \to a^+} f(x).
\] इति

यदि उभौ अपि सहमतौ तर्हि द्विपक्षीयसीमा विद्यते।

\subsubsection{अभ्यास}\label{ux905ux92dux92fux938-2}

\begin{enumerate}
\def\labelenumi{\arabic{enumi}.}
\tightlist
\item
  \(\lim_{x \to 2} (3x^2 - x)\) गणना करें।
\item
  \(\lim_{x \to 0} \frac{\sin x}{x}\) इति किम् ? \(\sin x\) इत्यस्य आलेखात्
  अन्तःकरणस्य उपयोगं कुर्वन्तु ।
\item
  \(\lim_{x \to 0} |x|/x\) मूल्याङ्कनं कुर्वन्तु। किं द्विपक्षीयसीमा विद्यते ?
\item
  \(\lim_{x \to \infty} \frac{1}{x}\) ज्ञातव्यम्। एतस्य परिणामस्य शब्दैः व्याख्यां
  कुरुत।5. \(f(x) = \frac{x^2-1}{x-1}\) कृते \(\lim_{x \to 1} f(x)\) इति
  किम्? \(f(1)\) इत्यस्य मूल्येन सह तुलनां कुर्वन्तु ।
\end{enumerate}

\subsection{1.4 सीमानां
औपचारिकपरिभाषा}\label{ux938ux92eux928-ux914ux92aux91aux930ux915ux92aux930ux92dux937}

एप्सिलॉन्--डेल्टा परिभाषायाः उपयोगेन सीमायाः सहजविचारः सटीकः कर्तुं शक्यते । एतेन
अस्मान् कठोरमार्गः प्राप्यते यत् \(f(x)\) मूल्यस्य \(L\) इत्यस्य समीपं गच्छति यतः
\(x\) \(a\) इत्यस्य समीपं गच्छति ।

\subsubsection{परिभाषा}\label{ux92aux930ux92dux937-1}

वयं लिखामः

\[
\lim_{x \to a} f(x) = L
\] इति

यदि निम्नलिखितशर्तः भवति : १.

प्रत्येकं \(\varepsilon > 0\) (किमपि लघु) कृते \(\delta > 0\) अस्ति यत् यदा कदापि

\[
0 < |x - a| < \delta,
\] इति

तदनुवर्तते

\[
|f(x) - L| < \varepsilon.
\] इति

शब्देषु: वयं \(f(x)\) इत्यस्य \(L\) इत्यस्य यथा इच्छेम समीपे कर्तुं शक्नुमः, बशर्ते \(x\)
\(a\) इत्यस्य पर्याप्तं समीपे अस्ति (किन्तु \(a\) इत्यस्य बराबरं न)।

\subsubsection{उदाहरणम् १ :
रेखीयफलनम्}\label{ux909ux926ux939ux930ux923ux92e-ux967-ux930ux916ux92fux92bux932ux928ux92e}

\(f(x) = 2x + 1\) कृते \(\lim_{x \to 3} f(x) = 7\) इति दर्शयतु ।

\begin{itemize}
\tightlist
\item
  वयं \(|f(x) - 7| < \varepsilon\) इच्छामः।
\item
  परन्तु \(f(x) - 7 = 2x + 1 - 7 = 2(x - 3)\)।
\item
  अतः \(|f(x) - 7| = 2|x - 3|\)।
\item
  यदि वयं \(\delta = \varepsilon / 2\) इति चिनोमः, तर्हि यदा कदापि
  \(|x - 3| < \delta\), अस्माकं \(|f(x) - 7| < \varepsilon\) भवति । एतेन
  सीमा सिद्धा भवति।
\end{itemize}

\subsubsection{उदाहरणम् २ : पारस्परिकं
कार्यम्}\label{ux909ux926ux939ux930ux923ux92e-ux968-ux92aux930ux938ux92aux930ux915-ux915ux930ux92fux92e}

\(f(x) = \frac{1}{x}\) कृते \(\lim_{x \to 2} f(x) = \tfrac{1}{2}\) इति
विचार्यताम् ।

\begin{itemize}
\tightlist
\item
  वयं \(\left|\frac{1}{x} - \frac{1}{2}\right| < \varepsilon\) इच्छामः।
\item
  अस्याः असमानतायाः बीजगणितीय-हेरफेरस्य आवश्यकता वर्तते, परन्तु \(\varepsilon\)
  इत्यस्य आधारेण \(\delta\) इति चयनेन तस्य तृप्तिः कर्तुं शक्यते । प्रक्रिया अधिका
  जटिला, परन्तु सिद्धान्तः समानः एव ।
\end{itemize}

\subsubsection{एतत् किमर्थं
महत्त्वपूर्णम्}\label{ux90fux924ux924-ux915ux92eux930ux925-ux92eux939ux924ux924ux935ux92aux930ux923ux92e}

\begin{itemize}
\tightlist
\item
  एप्सिलॉन्--डेल्टा परिभाषा गारण्टीं ददाति यत् सीमाः अस्पष्टाः न सन्ति अथवा केवलं
  अन्तःकरणस्य आधारेण न सन्ति।
\item
  निरन्तरतायाः, व्युत्पन्नस्य, अभिन्नस्य च आधारः अस्ति ।
\item
  यद्यपि आरम्भकानां कृते अमूर्तं दृश्यते तथापि सरल-उदाहरणैः सह कार्यं कृत्वा परिचितता
  निर्मीयते।
\end{itemize}

\subsubsection{अभ्यास}\label{ux905ux92dux92fux938-3}

\begin{enumerate}
\def\labelenumi{\arabic{enumi}.}
\tightlist
\item
  एप्सिलॉन--डेल्टा परिभाषायाः उपयोगेन \(\lim_{x \to 4} (x+1) = 5\) इति सिद्धं
  कुर्वन्तु।2. औपचारिकपरिभाषायाः उपयोगेन \(\lim_{x \to 0} 5x = 0\) इति दर्शयतु।
\item
  \(\lim_{x \to 0} \frac{1}{x}\) किमर्थं नास्ति इति व्याख्यातव्यम्।
\item
  \(f(x) = x^2\) कृते \(\lim_{x \to 2} f(x) = 4\) इति दर्शयतु।
\item
  स्वशब्देषु सीमापरिभाषायां \(\varepsilon\) तथा \(\delta\) इत्येतयोः भूमिकां
  व्याख्यातव्यम्।
\end{enumerate}

\subsection{1.5 निरन्तरता}\label{ux928ux930ux928ux924ux930ux924}

यदि तस्य आलेखः कागदात् भवतः पेन्सिलं न उत्थापयित्वा आकर्षितुं शक्यते तर्हि कार्यं निरन्तरं
भवति । अधिकं सटीकं वक्तुं शक्यते यत् निरन्तरता सुनिश्चितं करोति यत् निवेशे लघुपरिवर्तनानि
निर्गमस्य लघुपरिवर्तनानि उत्पद्यन्ते ।

\subsubsection{परिभाषा}\label{ux92aux930ux92dux937-2}

एकं फंक्शन् \(f\) एकस्मिन् बिन्दौ \(a\) निरन्तरं भवति यदि त्रीणि शर्ताः पूर्यन्ते:

\begin{enumerate}
\def\labelenumi{\arabic{enumi}.}
\tightlist
\item
  \(f(a)\) इति विवक्षितम्।
\item
  \(\lim_{x \to a} f(x)\) इति वर्तते।
\item
  \(\lim_{x \to a} f(x) = f(a)\) इति ।
\end{enumerate}

यदि कस्मिंश्चित् अन्तरालस्य प्रत्येकस्मिन् बिन्दौ फंक्शन् निरन्तरं भवति तर्हि तस्मिन् अन्तरे
निरन्तरम् इति वदामः ।

\subsubsection{उदाहरणम्}\label{ux909ux926ux939ux930ux923ux92e-2}

\begin{enumerate}
\def\labelenumi{\arabic{enumi}.}
\item
  बहुपदफलनम् : \(f(x) = x^2 + 3x - 5\) इत्यादीनि कार्याणि \(\mathbb{R}\)
  इत्यत्र सर्वत्र निरन्तराणि सन्ति ।
\item
  तर्कसंगतकार्यम् : \(f(x) = \frac{1}{x-1}\) \(x = 1\) इत्यत्र विहाय सर्वत्र
  निरन्तरम् अस्ति, यत्र अपरिभाषितम् अस्ति ।
\item
  खण्डवार कार्यम् : १.

  \[
  f(x) =
  \begin{cases}
  x^2 & x < 1, \\
  2 & x = 1, \\
  x+1 & x > 1,
  \end{cases}
  \] इति

  अस्य फंक्शन् इत्यस्य \(x = 1\) इत्यत्र ``jump'' अस्ति, अतः तत्र निरन्तरं नास्ति ।
\end{enumerate}

\subsubsection{असंतुलनानां
प्रकाराः}\label{ux905ux938ux924ux932ux928ux928-ux92aux930ux915ux930}

\begin{enumerate}
\def\labelenumi{\arabic{enumi}.}
\tightlist
\item
  हटनीयविच्छेदः : आलेखे एकः ``छिद्रः'' । उदाहरणम् : \(x=1\) इत्यत्र
  \(f(x) = \frac{x^2-1}{x-1}\) ।
\item
  कूर्दनविच्छेदः : वामहस्तस्य दक्षिणहस्तस्य च सीमा भिन्ना भवति ।
\item
  अनन्तविच्छेदः : फंक्शन् कस्यचित् बिन्दुस्य समीपे \(\pm\infty\) इति गच्छति, यथा
  \(x = 0\) इत्यस्य समीपे \(f(x) = 1/x\) इति ।
\end{enumerate}

\subsubsection{मध्यवर्ती मूल्य
प्रमेय}\label{ux92eux927ux92fux935ux930ux924-ux92eux932ux92f-ux92aux930ux92eux92f}

यदि कश्चन फंक्शन् \([a, b]\) अन्तरालस्य निरन्तरं भवति, तर्हि \(f(a)\) तथा
\(f(b)\) इत्येतयोः मध्ये कस्यापि संख्यायाः \(N\) कृते, तत्र केचन \(c \in [a, b]\)
सन्ति यत् \(f(c) = N\)समीकरणानां मूलस्य समाधानस्य च अस्तित्वं सिद्धयितुं एषः गुणः
महत्त्वपूर्णः अस्ति ।

\subsubsection{अभ्यास}\label{ux905ux92dux92fux938-4}

\begin{enumerate}
\def\labelenumi{\arabic{enumi}.}
\tightlist
\item
  \(x = 0\) इत्यत्र \(f(x) = |x|\) इति फंक्शन् निरन्तरं भवति वा इति निर्णयं कुर्वन्तु
  ।
\item
  \(f(x) = \frac{x+2}{x^2-1}\) कृते विच्छेदबिन्दून् चिनुत।
\item
  प्रत्येकं बहुपदफलनं सर्वत्र किमर्थं निरन्तरं भवति इति व्याख्यातव्यम्।
\item
  कूर्दनविच्छेदयुक्तस्य कार्यस्य उदाहरणं ददातु। तस्य आलेखं रेखांकयतु।
\item
  समीकरणस्य \(x^3 + x - 1 = 0\) इत्यस्य समाधानं 0 तः 1 पर्यन्तं भवति इति दर्शयितुं
  Intermediate Value Theorem इत्यस्य उपयोगं कुर्वन्तु ।
\end{enumerate}

\section{अध्याय 2.
व्युत्पन्न}\label{ux905ux927ux92fux92f-2.-ux935ux92fux924ux92aux928ux928}

\subsection{2.1 परिवर्तनस्य दररूपेण
व्युत्पन्नम्}\label{ux92aux930ux935ux930ux924ux928ux938ux92f-ux926ux930ux930ux92aux923-ux935ux92fux924ux92aux928ux928ux92e}

व्युत्पन्नं गणितस्य केन्द्रविचारेषु अन्यतमम् अस्ति । एतत् मापयति यत् यथा यथा कस्यचित्
कार्यस्य परिवर्तनं भवति तथा तथा तस्य परिवर्तनं भवति - अन्येषु शब्देषु, निवेशस्य विषये
निर्गमस्य परिवर्तनस्य दरः ।

\subsubsection{परिवर्तनस्य औसत
दरः}\label{ux92aux930ux935ux930ux924ux928ux938ux92f-ux914ux938ux924-ux926ux930}

\(f(x)\) इति फंक्शन् कृते \(x = a\) तथा \(x = b\) इत्येतयोः बिन्दुयोः मध्ये
परिवर्तनस्य औसतदरः भवति

\[
\frac{f(b) - f(a)}{b - a}.
\] इति

इदं \((a, f(a))\) तथा \((b, f(b))\) इति बिन्दुभिः माध्यमेन सेकण्ट् रेखायाः प्रवणता
अस्ति ।

\subsubsection{परिवर्तनस्य क्षणिक
दर}\label{ux92aux930ux935ux930ux924ux928ux938ux92f-ux915ux937ux923ux915-ux926ux930}

एकस्मिन् बिन्दौ \(f(x)\) कियत् शीघ्रं परिवर्तते इति मापनार्थं वयं अन्तरालं संकुचितुं
दद्मः:

\[
f'(a) = \lim_{h \to 0} \frac{f(a+h) - f(a)}{h}.
\] इति

इयं सीमा यदि अस्ति तर्हि \(a\) इत्यत्र \(f\) इत्यस्य व्युत्पन्नं उच्यते ।
ज्यामितीयदृष्ट्या \((a, f(a))\) इति बिन्दौ \(f\) इत्यस्य आलेखस्य स्पर्शरेखायाः
प्रवणता अस्ति ।

\subsubsection{संकेतन}\label{ux938ux915ux924ux928}

\begin{itemize}
\tightlist
\item
  \(f'(x)\): अभाज्य संकेतन।
\item
  \(\dfrac{dy}{dx}\): लाइब्निज् संकेतन, यदा \(y = f(x)\) प्रयुक्त।
\item
  \(Df(x)\): संचालक संकेतन।
\end{itemize}

एते सर्वे प्रतीकाः एकमेव अवधारणाम् निर्दिशन्ति ।

\subsubsection{उदाहरणम्}\label{ux909ux926ux939ux930ux923ux92e-3}

\begin{enumerate}
\def\labelenumi{\arabic{enumi}.}
\item
  \(f(x) = x^2\) कृते:

  \[
  f'(x) = \lim_{h \to 0} \frac{(x+h)^2 - x^2}{h} = \lim_{h \to 0} \frac{2xh + h^2}{h} = 2x.
  \] इति

  \(x\) इत्यत्र परवलयस्य प्रवणता \(2x\) अस्ति ।
\item
  \(f(x) = \sin x\) कृते:

  \[
  f'(x) = \cos x.
  \] इति3. \(f(x) = c\) (एकः नित्यः) कृते :

  \[
  f'(x) = 0.
  \] इति

  नित्यं कार्यं कदापि न परिवर्तते।
\end{enumerate}

\subsubsection{व्याख्या}\label{ux935ux92fux916ux92f}

\begin{itemize}
\tightlist
\item
  भौतिकशास्त्रे : यदि \(s(t)\) स्थितिः अस्ति तर्हि \(s'(t)\) वेगः अस्ति ।
\item
  अर्थशास्त्रे : यदि \(C(x)\) मूल्यं भवति तर्हि \(C'(x)\) सीमान्तव्ययः भवति ।
\item
  जीवविज्ञाने : यदि \(P(t)\) जनसंख्या अस्ति तर्हि \(P'(t)\) वृद्धिदरः अस्ति।
\end{itemize}

व्युत्पन्नं बहुषु सन्दर्भेषु ``परिवर्तनं'' सटीकं करोति ।

\subsubsection{अभ्यास}\label{ux905ux92dux92fux938-5}

\begin{enumerate}
\def\labelenumi{\arabic{enumi}.}
\tightlist
\item
  \(f(x) = 3x^2 - 2x + 1\) कृते \(f'(x)\) गणनां कुर्वन्तु।
\item
  \(x = 2\) इत्यत्र \(f(x) = x^3\) इत्यस्य स्पर्शरेखायाः प्रवणतां ज्ञातव्यम् ।
\item
  यदि \(s(t) = t^2 + 2t\) दूरं मीटर् मध्ये प्रतिनिधियति तर्हि \(t = 5\) इत्यत्र
  वेगः कः ?
\item
  \(f(x) = \frac{1}{x}\) इत्यस्य व्युत्पन्नस्य गणनायै सीमापरिभाषायाः उपयोगं कुर्वन्तु
  ।
\item
  \(y = x^2\) इत्यस्य आलेखं स्केच कृत्वा \(x = 1\) इत्यत्र स्पर्शरेखां आकर्षयन्तु।
\end{enumerate}

\subsection{2.2 भेद नियम}\label{ux92dux926-ux928ux92fux92e}

एकदा व्युत्पन्नं परिभाषितं जातं चेत् तस्य गणनायाः कुशलमार्गाः आवश्यकाः । भेदनियमाः
शॉर्टकट् सन्ति ये अस्मान् सीमापरिभाषां पुनः पुनः प्रयोक्तुं रक्षन्ति ।

\subsubsection{नित्य नियम}\label{ux928ux924ux92f-ux928ux92fux92e}

यदि \(f(x) = c\) यत्र \(c\) नित्यं तर्हि

\[
f'(x) = 0.
\] इति

\subsubsection{शक्ति नियम}\label{ux936ux915ux924-ux928ux92fux92e}

\(f(x) = x^n\) कृते यत्र \(n\) वास्तविकसङ्ख्या अस्ति,

\[
\frac{d}{dx} \big( x^n \big) = n x^{n-1}.
\] इति

उदाहरणानि : १.

\begin{itemize}
\tightlist
\item
  \(\frac{d}{dx}(x^2) = 2x\)।
\item
  \(\frac{d}{dx}(x^5) = 5x^4\)।
\item
  \(\frac{d}{dx}(\sqrt{x}) = \frac{1}{2\sqrt{x}}\)।
\end{itemize}

\subsubsection{नित्य बहुविधः
नियमः}\label{ux928ux924ux92f-ux92cux939ux935ux927-ux928ux92fux92e}

यदि \(f(x) = c \cdot g(x)\), तर्हि

\[
f'(x) = c \cdot g'(x).
\] इति

\subsubsection{योग भेद नियम}\label{ux92fux917-ux92dux926-ux928ux92fux92e}

\begin{itemize}
\tightlist
\item
  \((f + g)' = f' + g'\)।
\item
  \((f - g)' = f' - g'\)।
\end{itemize}

\subsubsection{उत्पाद नियम}\label{ux909ux924ux92aux926-ux928ux92fux92e}

\(f(x)\) तथा \(g(x)\) कृते:

\[
(fg)' = f'g + fg'.
\] इति

उदाहरणम् : यदि \(f(x) = x^2\), \(g(x) = \sin x\):

\[
(fg)' = (2x)(\sin x) + (x^2)(\cos x).
\] इति

\subsubsection{भागफल नियम}\label{ux92dux917ux92bux932-ux928ux92fux92e}

\(f(x)\) तथा \(g(x)\) कृते:

\[
\left(\frac{f}{g}\right)' = \frac{f'g - fg'}{g^2}, \quad g(x) \neq 0.
\] इति

उदाहरणम् : यदि \(f(x) = x^2\), \(g(x) = x+1\):

\[ इति\left(\frac{x^2}{x+1}\right)' = \frac{(2x)(x+1) - (x^2)(1)}{(x+1)^2}।
\]

\subsubsection{Derivatives of Common
Functions}\label{derivatives-of-common-functions}

\begin{itemize}
\tightlist
\item
  \(\frac{d}{dx}(\sin x) = \cos x\).
\item
  \(\frac{d}{dx}(\cos x) = -\sin x\).
\item
  \(\frac{d}{dx}(e^x) = e^x\).
\item
  \(\frac{d}{dx}(\ln x) = \frac{1}{x}, \quad x > 0\).
\end{itemize}

\subsubsection{Exercises}\label{exercises}

\begin{enumerate}
\def\labelenumi{\arabic{enumi}.}
\tightlist
\item
  Differentiate \(f(x) = 7x^3 - 4x + 9\).
\item
  Use the product rule to find the derivative of \(f(x) = x^2 e^x\).
\item
  Apply the quotient rule to \(f(x) = \frac{\sin x}{x}\).
\item
  Compute \(\frac{d}{dx}(\ln(x^2))\) using the chain of rules.
\item
  Show that the derivative of \(f(x) = \frac{1}{x}\) is
  \(-\frac{1}{x^2}\).
\end{enumerate}

\subsection{2.3 The Chain Rule}\label{the-chain-rule}

Often, functions are built by combining simpler functions together. To
differentiate such composite functions, we use the chain rule.

\subsubsection{The Rule}\label{the-rule}

If \(y = f(g(x))\), then

\[ इति
\frac{dy}{dx} = f'(g(x)) \cdot g'(x) ।
\]

In words: differentiate the outer function, keep the inside unchanged,
then multiply by the derivative of the inside.

\subsubsection{Examples}\label{examples}

\begin{enumerate}
\def\labelenumi{\arabic{enumi}.}
\item
  Square of a linear function

  \[ इति
  य = (3x+2)^2
  \]

  Outer function: \(f(u) = u^2\), inner function: \(g(x) = 3x+2\).

  \[
  य' = २(३x+२) \cdot ३ = ६(३x+२) ।
  \]
\item
  Exponential with quadratic inside

  \[ इति
  य = ई^{x^2}
  \]

  Outer function: \(f(u) = e^u\), inner function: \(g(x) = x^2\).

  \[ इति
  य' = ई^{x^2} \cdot 2x = 2x ई^{x^2}।
  \]
\item
  Logarithm with root inside

  \[ इति
  य = \ln(\sqrt{x})
  \]

  Outer: \(f(u) = \ln u\), inner: \(g(x) = \sqrt{x}\).

  \[ इति
  y' = \frac{1}{\sqrt{x}} \cdot \frac{1}{2\sqrt{x}} = \frac{1}{2x}।
  \]
\end{enumerate}

\subsubsection{Generalized Chain Rule}\label{generalized-chain-rule}

For multiple nested functions \(y = f(g(h(x)))\):

\[ इति
\frac{dy}{dx} = f'(g(h(x))) \cdot g'(h(x)) \cdot h'(x) ।
\] इति

एतत् स्वाभाविकतया गभीरतररचनासु विस्तृतं भवति ।

\subsubsection{श्रृङ्खला नियमः किमर्थं महत्त्वपूर्णः- एतत् प्रायः सर्वाणि
वास्तविक-जगतः प्रतिरूपाणि सम्पादयति यत्र एकः परिमाणः परोक्षरूपेण अन्यस्य उपरि
आश्रितः भवति
।}\label{ux936ux930ux919ux916ux932-ux928ux92fux92e-ux915ux92eux930ux925-ux92eux939ux924ux924ux935ux92aux930ux923--ux90fux924ux924-ux92aux930ux92f-ux938ux930ux935ux923-ux935ux938ux924ux935ux915-ux91cux917ux924-ux92aux930ux924ux930ux92aux923-ux938ux92eux92aux926ux92fux924-ux92fux924ux930-ux90fux915-ux92aux930ux92eux923-ux92aux930ux915ux937ux930ux92aux923-ux905ux928ux92fux938ux92f-ux909ux92aux930-ux906ux936ux930ux924-ux92dux935ux924}

\begin{itemize}
\tightlist
\item
  एतत् गणितं भौतिकशास्त्रेण सह संयोजयति (उदा., स्थानद्वारा कालस्य आधारेण वेगः) ।
\item
  अन्तर्निहितभेदे उन्नतविषयेषु च अत्यावश्यकम्।
\end{itemize}

\subsubsection{अभ्यास}\label{ux905ux92dux92fux938-6}

\begin{enumerate}
\def\labelenumi{\arabic{enumi}.}
\tightlist
\item
  \(y = (5x^2 + 1)^3\) इति भेदं कुरुत।
\item
  \(\frac{d}{dx}(\sin(3x))\) ज्ञातव्यम्।
\item
  \(\frac{d}{dx}(\ln(1+x^2))\) गणना करें।
\item
  \(y = \cos^2(x)\) इति भेदं कुरुत।
\item
  सामान्यीकृतशृङ्खलानियमं \(y = e^{\sin(x^2)}\) इत्यत्र प्रयोजयन्तु।
\end{enumerate}

\subsection{2.4 अन्तर्निहित
भेद}\label{ux905ux928ux924ux930ux928ux939ux924-ux92dux926}

न सर्वाणि कार्याणि \(y = f(x)\) इति रूपेण दत्तानि सन्ति । कदाचित् \(x\) तथा
\(y\) समीकरणेन सम्बद्धौ भवतः, \(y\) कृते स्पष्टतया समाधानं कठिनं वा असम्भवं वा भवति
। एतादृशेषु सति वयं अन्तर्निहितभेदस्य उपयोगं कुर्मः ।

\subsubsection{विचारः}\label{ux935ux91aux930}

यदि समीकरणे \(x\) तथा \(y\) इत्येतयोः द्वयोः अपि समावेशः भवति तर्हि वयं \(x\)
इत्यस्य विषये उभयपक्षयोः भेदं कर्तुं शक्नुमः, \(y\) इत्यस्य \(x\) इत्यस्य कार्यरूपेण
व्यवहारं कृत्वा प्रत्येकं समये वयं \(y\) इत्यनेन सह सम्बद्धं पदं भेदयामः तदा वयं
\(\frac{dy}{dx}\) इत्यनेन गुणयामः ।

\subsubsection{उदाहरणम् १ : एकं
वृत्तम्}\label{ux909ux926ux939ux930ux923ux92e-ux967-ux90fux915-ux935ux924ux924ux92e}

समीकरणम् : १.

\[
x^2 + y^2 = 25
\]

\(x\) इत्यस्य विषये भेदं कुर्वन्तु:

\[
2x + 2y \frac{dy}{dx} = 0.
\] इति

\(\frac{dy}{dx}\) कृते समाधानं कुरुत:

\[
\frac{dy}{dx} = -\frac{x}{y}.
\] इति

अनेन कस्मिन् अपि बिन्दौ वृत्तस्य स्पर्शरेखायाः प्रवणता प्राप्यते ।

\subsubsection{उदाहरणम् २ : चरानाम् एकः
उत्पादः}\label{ux909ux926ux939ux930ux923ux92e-ux968-ux91aux930ux928ux92e-ux90fux915-ux909ux924ux92aux926}

समीकरणम् : १.

\[
xy = 1
\] इति

भेदं कुरुत : १.

\[
x \frac{dy}{dx} + y = 0.
\] इति

अतः,

\[
\frac{dy}{dx} = -\frac{y}{x}.
\] इति

\subsubsection{उदाहरणम् 3: त्रिकोणमितीय
सम्बन्ध}\label{ux909ux926ux939ux930ux923ux92e-3-ux924ux930ux915ux923ux92eux924ux92f-ux938ux92eux92cux928ux927}

समीकरणम् : १.

\[
\sin(xy) = x
\] इति

भेदं कुरुत : १.

\[
\cos(xy) \cdot \Big(y + x\frac{dy}{dx}\Big) = 1.
\] इति

\(\frac{dy}{dx}\) कृते समाधानं कुरुत:

\[
\frac{dy}{dx} = \frac{1 - y\cos(xy)}{x\cos(xy)}.
\] इति

\subsubsection{अन्तर्निहितभेदः किमर्थं उपयोगी
भवति}\label{ux905ux928ux924ux930ux928ux939ux924ux92dux926-ux915ux92eux930ux925-ux909ux92aux92fux917-ux92dux935ux924}

\begin{itemize}
\tightlist
\item
  अनेकाः महत्त्वपूर्णाः वक्राः (वृत्ताः, दीर्घवृत्ताः, अतिपरवलयः) स्वाभाविकतया
  अन्तर्निहितरूपेण परिभाषिताः सन्ति ।
\item
  अस्मान् प्रथमं \(y\) कृते समाधानं विना समीकरणानां भेदं कर्तुं शक्नोति ।-
  सम्बन्धितदराणि, अवकलसमीकरणानि च इत्यादिषु अधिक उन्नतविषयेषु एतत् प्रमुखं सोपानम्
  अस्ति ।
\end{itemize}

\subsubsection{अभ्यास}\label{ux905ux92dux92fux938-7}

\begin{enumerate}
\def\labelenumi{\arabic{enumi}.}
\tightlist
\item
  वक्रस्य \(x^2 + xy + y^2 = 7\) कृते \(\frac{dy}{dx}\) इति ज्ञातव्यम् ।
\item
  \(\cos(x) + \cos(y) = 1\) इत्यस्य अन्तर्निहितरूपेण भेदं कुर्वन्तु।
\item
  \((1, 2)\) इति बिन्दौ \(x^3 + y^3 = 9\) यावत् स्पर्शरेखायाः प्रवणतां ज्ञातव्यम्
  ।
\item
  \(x^2 + y^2 = 10\) दत्त, \((x, y) = (1, 3)\) यदा \(\frac{dy}{dx}\)
  गणना।
\item
  \(\frac{dy}{dx}\) अन्वेष्टुं \(e^{xy} = x + y\) इति भेदं कुर्वन्तु।
\end{enumerate}

\subsection{2.5 उच्च-क्रम
व्युत्पन्न}\label{ux909ux91aux91a-ux915ux930ux92e-ux935ux92fux924ux92aux928ux928}

एतावता वयं प्रथमं व्युत्पन्नम् अधीतवन्तः, यत् कार्यस्य परिवर्तनस्य दरं मापयति । परन्तु
व्युत्पन्नानाम् अपि भेदः कर्तुं शक्यते, येन उच्चक्रमस्य व्युत्पन्नस्य जन्म भवति ।

\subsubsection{परिभाषा}\label{ux92aux930ux92dux937-3}

\begin{itemize}
\item
  \(f\) इत्यस्य द्वितीयं व्युत्पन्नं व्युत्पन्नस्य व्युत्पन्नम् अस्ति :

  \[
  f''(x) = \frac{d}{dx}\left(f'(x)\right).
  \] इति
\item
  अधिकसामान्यतया \(n\)-थ व्युत्पन्नं यथा लिख्यते

  \[
  f^{(n)}(x) = \frac{d^n}{dx^n} f(x).
  \] इति
\end{itemize}

\subsubsection{उदाहरणम्}\label{ux909ux926ux939ux930ux923ux92e-4}

\begin{enumerate}
\def\labelenumi{\arabic{enumi}.}
\item
  \(f(x) = x^3\) इति

  \begin{itemize}
  \tightlist
  \item
    प्रथम व्युत्पन्न : \(f'(x) = 3x^2\)।
  \item
    द्वितीय व्युत्पन्न: \(f''(x) = 6x\)।
  \item
    तृतीय व्युत्पन्न: \(f^{(3)}(x) = 6\)।
  \item
    चतुर्थ व्युत्पन्न: \(f^{(4)}(x) = 0\)।
  \end{itemize}
\item
  \(f(x) = \sin x\) इति

  \begin{itemize}
  \tightlist
  \item
    \(f'(x) = \cos x\)।
  \item
    \(f''(x) = -\sin x\)।
  \item
    \(f^{(3)}(x) = -\cos x\)।
  \item
    \(f^{(4)}(x) = \sin x\)। व्युत्पन्नाः दीर्घताचक्रे पुनरावृत्तिं कुर्वन्ति ४ ।
  \end{itemize}
\item
  \(f(x) = e^x\) इति

  \begin{itemize}
  \tightlist
  \item
    प्रत्येकं व्युत्पन्नं \(e^x\) भवति।
  \end{itemize}
\end{enumerate}

\subsubsection{अनुप्रयोग}\label{ux905ux928ux92aux930ux92fux917}

\begin{itemize}
\item
  अवतलता : \(f''(x)\) इत्यस्य चिह्नं वदति यत् \(f\) इत्यस्य आलेखः अवतलः उपरि
  (\(f'' > 0\)) अस्ति वा अवतलः अस्ति वा (\(f'' < 0\))।
\item
  विभक्तिबिन्दवः : बिन्दुः यत्र \(f''(x) = 0\) अवतलता च परिवर्तते ।
\item
  गतिः भौतिकशास्त्रे यदि \(s(t)\) स्थितिः भवति :

  \begin{itemize}
  \tightlist
  \item
    \(s'(t)\) = वेग, .
  \item
    \(s''(t)\) = त्वरण, .
  \item
    \(s^{(3)}(t)\) = झटका (त्वरण परिवर्तन की दर)।
  \end{itemize}
\item
  सन्निकर्षाः : उच्चक्रमस्य व्युत्पन्नाः टेलर श्रृङ्खलायां दृश्यन्ते, येषां उपयोगः कार्याणां
  अनुमानं कर्तुं भवति ।\#\#\# अभ्यास
\end{itemize}

\begin{enumerate}
\def\labelenumi{\arabic{enumi}.}
\tightlist
\item
  \(f(x) = \cos x\) इत्यस्य प्रथमचतुर्णां व्युत्पन्नानां गणनां कुरुत।
\item
  \(f(x) = x^4 - 2x^2 + 3\) कृते \(f''(x)\) ज्ञातव्यम्।
\item
  \(f(x) = e^{2x}\) कृते \(f^{(n)}(x) = 2^n e^{2x}\) इति दर्शयतु।
\item
  यत्र \(f(x) = x^3 - 3x\) अवतलः उपरि अवतलः च भवति तत्र अन्तरालं निर्धारयन्तु।
\item
  यदि \(s(t) = t^3 - 6t^2 + 9t\) तर्हि \(t = 2\) इत्यत्र वेगं त्वरणं च ज्ञातव्यम्।
\end{enumerate}

\section{अध्याय 3. व्युत्पन्न के
अनुप्रयोग}\label{ux905ux927ux92fux92f-3.-ux935ux92fux924ux92aux928ux928-ux915-ux905ux928ux92aux930ux92fux917}

\subsection{3.1 स्पर्शरेखा तथा
सामान्य}\label{ux938ux92aux930ux936ux930ux916-ux924ux925-ux938ux92eux928ux92f}

व्युत्पन्नस्य प्रथमप्रयोगेषु एकः वक्रस्य स्पर्शरेखायाः सामान्यरेखायाः च समीकरणानि अन्वेष्टुं
भवति । एताः रेखाः दत्तबिन्दौ कस्यचित् फंक्शन् इत्यस्य स्थानीयज्यामितिं गृह्णन्ति ।

\subsubsection{स्पर्शरेखा}\label{ux938ux92aux930ux936ux930ux916}

\((a, f(a))\) इति बिन्दौ वक्रस्य \(y = f(x)\) इत्यस्य स्पर्शरेखा सा रेखा अस्ति या
केवलं तत्रत्यां आलेखं ``स्पृशति'' तथा च वक्रस्य समानं प्रवणता भवति

स्पर्शरेखायाः प्रवणता व्युत्पन्नेन दीयते- १.

\[
m_{\text{tangent}} = f'(a).
\] इति

एवं \((a, f(a))\) इत्यत्र स्पर्शरेखायाः समीकरणं भवति

\[
y - f(a) = f'(a)(x - a).
\] इति

\subsubsection{सामान्य रेखा}\label{ux938ux92eux928ux92f-ux930ux916}

सामान्यरेखा तस्मिन् एव बिन्दौ स्पर्शरेखायाः लम्बवत् भवति । अस्य प्रवणता
स्पर्शरेखाप्रवणस्य ऋणात्मकः परस्परं भवति : १.

\[
m_{\text{normal}} = -\frac{1}{f'(a)}.
\] इति

अतः सामान्यरेखायाः समीकरणम् अस्ति

\[
y - f(a) = -\frac{1}{f'(a)} (x - a), \quad f'(a) \neq 0.
\] इति

\subsubsection{उदाहरणम्}\label{ux909ux926ux939ux930ux923ux92e-5}

\begin{enumerate}
\def\labelenumi{\arabic{enumi}.}
\item
  \(f(x) = x^2\) at \(x = 1\)।

  \begin{itemize}
  \tightlist
  \item
    \(f(1) = 1\), \(f'(x) = 2x\), अतः \(f'(1) = 2\)।
  \item
    स्पर्शरेखा: \(y - 1 = 2(x - 1)\), अथवा \(y = 2x - 1\)।
  \item
    सामान्यम्: प्रवणता = \(-\tfrac{1}{2}\), अतः समीकरणम्
    \(y - 1 = -\tfrac{1}{2}(x - 1)\) अस्ति।
  \end{itemize}
\item
  \(f(x) = \sin x\) at \(x = \tfrac{\pi}{4}\)।

  \begin{itemize}
  \tightlist
  \item
    \(f(\tfrac{\pi}{4}) = \tfrac{\sqrt{2}}{2}\),
    \(f'(\tfrac{\pi}{4}) = \cos(\tfrac{\pi}{4}) = \tfrac{\sqrt{2}}{2}\)।
  \item
    स्पर्शरेखा:
    \(y - \tfrac{\sqrt{2}}{2} = \tfrac{\sqrt{2}}{2}(x - \tfrac{\pi}{4})\)।
  \end{itemize}
\end{enumerate}

\subsubsection{स्पर्शरेखाः सामान्याः च किमर्थं महत्त्वपूर्णाः सन्ति- स्पर्शरेखाः वक्रस्य
स्थानीयरूपेण अनुमानं कुर्वन्ति
(रेखीयसन्निकर्षः)।}\label{ux938ux92aux930ux936ux930ux916-ux938ux92eux928ux92f-ux91a-ux915ux92eux930ux925-ux92eux939ux924ux924ux935ux92aux930ux923-ux938ux928ux924--ux938ux92aux930ux936ux930ux916-ux935ux915ux930ux938ux92f-ux938ux925ux928ux92fux930ux92aux923-ux905ux928ux92eux928-ux915ux930ux935ux928ux924-ux930ux916ux92fux938ux928ux928ux915ux930ux937}

\begin{itemize}
\tightlist
\item
  ज्यामितिः, प्रकाशिकी (प्रतिबिम्ब/अपवर्तन), यान्त्रिक (बलदिशा) च इत्यत्र
  सामान्याः उपयोगिनो भवन्ति ।
\item
  अनुकूलन-वक्रता-अध्ययनयोः द्वयोः अपि भूमिका अस्ति ।
\end{itemize}

\subsubsection{अभ्यास}\label{ux905ux92dux92fux938-8}

\begin{enumerate}
\def\labelenumi{\arabic{enumi}.}
\tightlist
\item
  \(x = 2\) इत्यत्र \(y = x^3\) इत्यस्य स्पर्शरेखाः सामान्यरेखाः च ज्ञातव्याः ।
\item
  \(x = 0\) इत्यत्र \(y = e^x\) इत्यस्य स्पर्शरेखाः सामान्यरेखाः च निर्धारयन्तु ।
\item
  \(y = \ln x\) कृते \(x = 1\) इत्यत्र स्पर्शरेखां गणयन्तु ।
\item
  \(x^2 + y^2 = 9\) इत्यनेन वृत्तं दीयते। \((0,3)\) इत्यत्र स्पर्शरेखायाः प्रवणतां
  ज्ञातुं अन्तर्निहितभेदस्य उपयोगं कुर्वन्तु ।
\item
  \(y = \sqrt{x}\) इत्यस्य आलेखं स्केच कृत्वा \(x = 4\) इत्यत्र स्पर्शरेखाः
  सामान्यरेखाः च आकर्षयन्तु ।
\end{enumerate}

\subsection{3.2 सम्बन्धित
दर}\label{ux938ux92eux92cux928ux927ux924-ux926ux930}

अनेकेषु वास्तविकसमस्यासु कालस्य विषये द्वौ वा अधिकौ परिमाणौ परिवर्तन्ते, तेषां
परिवर्तनस्य दराः च सम्बद्धाः भवन्ति । सम्बन्धितदरसमस्याः एतेषां सम्बन्धानां वर्णनार्थं
व्युत्पन्नानाम् उपयोगं कुर्वन्ति ।

\subsubsection{सामान्य
दृष्टिकोण}\label{ux938ux92eux928ux92f-ux926ux937ux91fux915ux923}

\begin{enumerate}
\def\labelenumi{\arabic{enumi}.}
\tightlist
\item
  काल \(t\) इत्यस्य उपरि निर्भराः चराः चिनुत।
\item
  चरसम्बद्धं समीकरणं लिखत।
\item
  श्रृङ्खलानियमं प्रयोज्य \(t\) इत्यस्य विषये उभयपक्षयोः भेदं कुर्वन्तु।
\item
  दत्तक्षणे ज्ञातानि मूल्यानि प्रतिस्थापयन्तु।
\item
  अज्ञातदरस्य कृते समाधानं कुरुत।
\end{enumerate}

\subsubsection{उदाहरणम् १ : वृत्तस्य
विस्तारः}\label{ux909ux926ux939ux930ux923ux92e-ux967-ux935ux924ux924ux938ux92f-ux935ux938ux924ux930}

वृत्तस्य त्रिज्या \(r\) भवति, या \(\frac{dr}{dt} = 2 \,\text{cm/s}\) इत्यस्य
गतिना वर्धते । \(r = 5\) इति समये \(A = \pi r^2\) इति क्षेत्रं यस्मिन् दरेन वर्धते
तत् ज्ञातव्यम् ।

भेदं कुरुत : १.

\[
\frac{dA}{dt} = 2\pi r \frac{dr}{dt}.
\]

पर्याय:

\[
\frac{dA}{dt} = 2\pi (5)(2) = 20\pi \,\text{cm}^2/\text{s}.
\] इति

\subsubsection{उदाहरणम् २ : स्लाइडिंग्
सीढी}\label{ux909ux926ux939ux930ux923ux92e-ux968-ux938ux932ux907ux921ux917-ux938ux922}

१० पादपरिमितं सीढी भित्तिं प्रति अवलम्बते । अधः
\(\frac{dx}{dt} = 1 \,\text{ft/s}\) इत्यत्र दूरं स्खलति । यदा अधः भित्तितः ६
पाददूरे भवति तदा उपरिभागः कियत् शीघ्रं अधः स्खलति?

समीकरणम् : \(x^2 + y^2 = 100\), यत्र \(y\) ऊर्ध्वता अस्ति।

भेदं कुरुत : १.

\[
2x \frac{dx}{dt} + 2y \frac{dy}{dt} = 0.
\] इति\(x = 6\) इत्यत्र \(y = 8\) इति । पर्याय:

\[
2(6)(1) + 2(8)\frac{dy}{dt} = 0 \quad \Rightarrow \quad \frac{dy}{dt} = -\tfrac{6}{8} = -\tfrac{3}{4}.
\]

अतः उपरिभागः \(0.75 \,\text{ft/s}\) इत्यत्र अधः स्लाइड् भवति ।

\subsubsection{उदाहरणम् ३ : शङ्कुस्थे
जलम्}\label{ux909ux926ux939ux930ux923ux92e-ux969-ux936ux919ux915ux938ux925-ux91cux932ux92e}

१२ से.मी.उच्चतायाः ६ से.मी.त्रिज्यायाः च शङ्कुमध्ये जलं पात्यते । यदा जलं ४ से.मी.गभीरं
भवति तदा जलस्तरः \(2 \,\text{cm/s}\) इत्यत्र वर्धमानः भवति । केन वेगेन आयतनं
वर्धते ?

समीकरणम् : \(V = \tfrac{1}{3}\pi r^2 h\)। सादृश्यस्य उपयोगेन
\(r = \tfrac{h}{2}\) इति । प्रतिस्थापनम् : १.

\[
V = \tfrac{1}{12}\pi h^3.
\] इति

भेदं कुरुत : १.

\[
\frac{dV}{dt} = \tfrac{1}{4}\pi h^2 \frac{dh}{dt}.
\] इति

\(h = 4\) इत्यत्र \(\frac{dh}{dt} = 2\): .

\[
\frac{dV}{dt} = \tfrac{1}{4}\pi (16)(2) = 8\pi \,\text{cm}^3/\text{s}.
\] इति

\subsubsection{सम्बन्धित दराः किमर्थं
महत्त्वपूर्णाः}\label{ux938ux92eux92cux928ux927ux924-ux926ux930-ux915ux92eux930ux925-ux92eux939ux924ux924ux935ux92aux930ux923}

\begin{itemize}
\tightlist
\item
  ते भौतिकशास्त्रे, अभियांत्रिकीशास्त्रे, जीवविज्ञाने च गतिं परिवर्तनं च वर्णयन्ति ।
\item
  ते कालनिर्भरप्रक्रियाभिः ज्यामितिं गणितेन सह संयोजयन्ति।
\item
  ते अस्मान् गतिशीलप्रणालीनां गणितीयरूपेण प्रतिरूपणं कर्तुं प्रशिक्षयन्ति।
\end{itemize}

\subsubsection{अभ्यास}\label{ux905ux92dux92fux938-9}

\begin{enumerate}
\def\labelenumi{\arabic{enumi}.}
\tightlist
\item
  एकं गुब्बारं फूत्कृतं भवति यथा तस्य त्रिज्या \(0.5 \,\text{cm/s}\) इत्यत्र वर्धते।
  यदा त्रिज्या १० से.मी.भवति तदा तस्य आयतनं कियत् शीघ्रं वर्धते इति ज्ञातव्यम् ।
\item
  एकं वाहनम् उत्तरदिशि 40 कि.मी./घण्टां, अपरं पूर्वदिशि 30 कि.मी. २ घण्टानन्तरं तेषां
  मध्ये दूरं कियत् शीघ्रं वर्धते ?
\item
  भित्तितः 20 मीटर् दूरे एकः स्पॉटलाइट् 1.5 मी/सेकण्ड् वेगेन दूरं गच्छन् 2 मीटर् ऊर्ध्वं
  पुरुषं प्रकाशते। भित्तिस्थस्य तस्य छायायाः दीर्घता कियत् शीघ्रं परिवर्तते यदा सः
  प्रकाशात् ५ मी.
\item
  घनस्य पार्श्वदीर्घता 2 से.मी./से. पार्श्वे ३ से.मी.
\item
  सदा ऊर्ध्वतायाः समं त्रिज्यायुक्तं शङ्कुं निर्माय राशेः उपरि वालुकायाः
  \hspace{0pt}\hspace{0pt}पातनं भवति। यदि ५ से.मी./सेकण्ड् यावत् ऊर्ध्वता वर्धते
  तर्हि १० से.मी.
\end{enumerate}

\subsection{3.3 अनुकूलनसमस्याःअनुकूलनसमस्याः प्रायः कतिपयेषु बाधासु, कार्यस्य अधिकतमं
न्यूनतमं वा मूल्यं अन्वेष्टुं व्युत्पन्नानाम् उपयोगं कुर्वन्ति । एताः समस्याः तान् परिस्थितयः
प्रतिरूपयन्ति यत्र वयं कार्यक्षमतां, लाभं, क्षेत्रं वा अधिकतमं कर्तुम् इच्छामः, अथवा मूल्यं,
दूरं, समयं वा न्यूनीकर्तुं इच्छामः
।}\label{ux905ux928ux915ux932ux928ux938ux92eux938ux92fux905ux928ux915ux932ux928ux938ux92eux938ux92f-ux92aux930ux92f-ux915ux924ux92aux92fux937-ux92cux927ux938-ux915ux930ux92fux938ux92f-ux905ux927ux915ux924ux92e-ux928ux92fux928ux924ux92e-ux935-ux92eux932ux92f-ux905ux928ux935ux937ux91f-ux935ux92fux924ux92aux928ux928ux928ux92e-ux909ux92aux92fux917-ux915ux930ux935ux928ux924-ux90fux924-ux938ux92eux938ux92f-ux924ux928-ux92aux930ux938ux925ux924ux92f-ux92aux930ux924ux930ux92aux92fux928ux924-ux92fux924ux930-ux935ux92f-ux915ux930ux92fux915ux937ux92eux924-ux932ux92d-ux915ux937ux924ux930-ux935-ux905ux927ux915ux924ux92e-ux915ux930ux924ux92e-ux907ux91aux91bux92e-ux905ux925ux935-ux92eux932ux92f-ux926ux930-ux938ux92eux92f-ux935-ux928ux92fux928ux915ux930ux924-ux907ux91aux91bux92e}

\subsubsection{सामान्य चरण}\label{ux938ux92eux928ux92f-ux91aux930ux923}

\begin{enumerate}
\def\labelenumi{\arabic{enumi}.}
\tightlist
\item
  समस्यां अवगच्छन्तु : अनुकूलितुं परिमाणं चिनुत।
\item
  फंक्शन् सह मॉडल् : एकस्य चरस्य दृष्ट्या उद्देश्यफंक्शनं लिखत।
\item
  बाधाः प्रयोजयन्तु : चरानाम् न्यूनीकरणाय दत्तानां शर्तानाम् उपयोगं कुर्वन्तु।
\item
  भेदं कुरुत : उद्देश्यफलनस्य व्युत्पन्नस्य गणनां कुरुत।
\item
  महत्त्वपूर्णबिन्दून् ज्ञातव्यः : \(f'(x) = 0\) अथवा यत्र \(f'(x)\) अपरिभाषितः
  अस्ति तत्र समाधानं कुर्वन्तु।
\item
  अधिकतम/न्यूनतमस्य परीक्षणम् : द्वितीयव्युत्पन्नपरीक्षायाः उपयोगं कुर्वन्तु अथवा
  अन्त्यबिन्दून् जाँचयन्तु।
\item
  परिणामस्य व्याख्यां कुरुत : उत्तरं मूलसन्दर्भे वदतु।
\end{enumerate}

\subsubsection{उदाहरणम् १ : आयतस्य अधिकतमं
क्षेत्रफलम्}\label{ux909ux926ux939ux930ux923ux92e-ux967-ux906ux92fux924ux938ux92f-ux905ux927ux915ux924ux92e-ux915ux937ux924ux930ux92bux932ux92e}

आयतस्य परिधिः 40. के आयामाः तस्य क्षेत्रफलं अधिकतमं कुर्वन्ति?

\begin{itemize}
\tightlist
\item
  लम्बाई \(x\), चौड़ाई \(y\) चलो। बाध्यता :
  \(2x + 2y = 40 \Rightarrow y = 20 - x\)।
\item
  क्षेत्र: \(A = xy = x(20 - x) = 20x - x^2\)।
\item
  व्युत्पन्न: \(A'(x) = 20 - 2x\)। 0 इत्यस्य बराबरं सेट् कुर्वन्तु: \(x = 10\) ।
\item
  अथ \(y = 10\)।
\item
  अधिकतम क्षेत्रफल: \(100\)। आयत इति वर्गः ।
\end{itemize}

\subsubsection{उदाहरणम् २ : दूरं
न्यूनीकर्तुं}\label{ux909ux926ux939ux930ux923ux92e-ux968-ux926ux930-ux928ux92fux928ux915ux930ux924}

\((0,3)\) इत्यस्य समीपस्थे परवलयस्य \(y = x^2\) इत्यस्य उपरि बिन्दुं ज्ञातव्यम् ।

\begin{itemize}
\tightlist
\item
  दूरी वर्गीकृत: \(D(x) = (x-0)^2 + (x^2 - 3)^2\)।
\item
  विस्तार:
  \(D(x) = x^2 + (x^2 - 3)^2 = x^2 + x^4 - 6x^2 + 9 = x^4 - 5x^2 + 9\)।
\item
  व्युत्पन्न: \(D'(x) = 4x^3 - 10x\)। समाधान : \(x(4x^2 - 10) = 0\)।
\item
  समाधान : \(x = 0\), \(x = \pm \sqrt{2.5}\)।
\item
  जाँचः \(x = \pm \sqrt{2.5}\) इत्यत्र न्यूनतमं दूरं ददाति ।
\end{itemize}

\subsubsection{उदाहरणम् ३ : अधिकतममात्रायुक्तः
पेटी}\label{ux909ux926ux939ux930ux923ux92e-ux969-ux905ux927ux915ux924ux92eux92eux924ux930ux92fux915ux924-ux92aux91f}

कोणेभ्यः समानवर्गान् छित्त्वा पार्श्वयोः उपरि गुञ्जयित्वा पार्श्वे २० से.मी. आयतनं
अधिकतमं कृत्वा कटस्य आकारं ज्ञातव्यम् ।- कट आकार = \(x\) चलो। ततः आयामाः:
\((20 - 2x) \times (20 - 2x) \times x\)। - खण्डः \(V(x) = x(20 - 2x)^2\)।
- व्युत्पन्न: \(V'(x) = (20 - 2x)(20 - 6x)\)। - महत्वपूर्ण बिन्दवः : \(x = 10\)
(शून्य आयतनं ददाति) अथवा \(x = \tfrac{20}{6} \approx 3.33\) । -
\(x \approx 3.33\) इत्यत्र आयतनं अधिकतमं भवति ।

\subsubsection{अनुकूलनं किमर्थं
महत्त्वपूर्णम्}\label{ux905ux928ux915ux932ux928-ux915ux92eux930ux925-ux92eux939ux924ux924ux935ux92aux930ux923ux92e}

\begin{itemize}
\tightlist
\item
  अभियंताः कुशलसंरचनानां डिजाइनं कर्तुं तस्य उपयोगं कुर्वन्ति।
\item
  व्यवसायाः अधिकतमं लाभं प्राप्तुं वा न्यूनतया व्ययार्थं वा तस्य उपयोगं कुर्वन्ति।
\item
  वैज्ञानिकाः तस्य उपयोगं प्राकृतिकव्यवस्थानां प्रतिरूपणार्थं कुर्वन्ति ये संतुलनं इच्छन्ति।
\end{itemize}

\subsubsection{अभ्यास}\label{ux905ux92dux92fux938-10}

\begin{enumerate}
\def\labelenumi{\arabic{enumi}.}
\tightlist
\item
  एकस्य कृषकस्य नदीपार्श्वे आयताकारक्षेत्रं परिवेष्टयितुं 100 मी.वेष्टनं भवति (अतः केवलं 3
  पार्श्वयोः वेष्टनस्य आवश्यकता भवति)। क्षेत्रफलं अधिकतमं कृत्वा आयामान् ज्ञातव्यम्।
\item
  धनात्मकसङ्ख्याद्वयं ज्ञातव्यं यस्य योगः 20 भवति तथा च यस्य गुणनफलं यथासम्भवं बृहत्
  भवति।
\item
  100 सेमी\(^2\) सामग्रीतः सिलिण्डरं निर्मातव्यम्। अधिकतम आयतनस्य आयामान्
  ज्ञातव्यम्।
\item
  10 मी.दीर्घं तारं द्वौ खण्डौ छिनत्ति, एकं वर्गं नतम्, अन्यं वृत्तं कृत्वा। परिवेष्टितं
  कुलक्षेत्रं अधिकतमं कर्तुं कथं तस्य छेदनं कर्तव्यम् ?
\item
  वर्गाकारमूलं 32 m\(^3\) आयतनं च युक्तं बन्दं पेटी निर्मातव्यम्। पृष्ठीयक्षेत्रं न्यूनीकृत्य
  आयामान् ज्ञातव्यम्।
\end{enumerate}

\subsection{3.4 अवतलता तथा विभक्ति
बिन्दु}\label{ux905ux935ux924ux932ux924-ux924ux925-ux935ux92dux915ux924-ux92cux928ux926}

व्युत्पन्नाः न केवलं प्रवणानाम् विषये अपितु आलेखस्य आकारस्य विषये अपि वदन्ति ।
अवतलतायाः अवगमने विभक्तिबिन्दुपरिचये च द्वितीयः व्युत्पन्नः विशेषतया उपयोगी भवति ।

\subsubsection{अवतलता}\label{ux905ux935ux924ux932ux924}

\begin{itemize}
\item
  एकं फंक्शन् \(f(x)\) एकस्मिन् अन्तरालस्य उपरि अवतलं भवति यदि \(f''(x) > 0\).
  आलेखः ऊर्ध्वं नमति, चषकः इव ।
\item
  एकं फंक्शन् \(f(x)\) एकस्मिन् अन्तरालस्य उपरि अवतलं भवति यदि \(f''(x) < 0\).
  आलेखः अधः नमति, भ्रूभङ्ग इव।
\end{itemize}

अवतलता वर्णयति यत् कार्यस्य प्रवणता कथं परिवर्तते: यदि प्रवणाः वर्धन्ते तर्हि आलेखः
उपरि अवतलः भवति; यदि प्रवणाः न्यूनाः भवन्ति तर्हि आलेखः अधः अवतलः भवति ।

\subsubsection{विभक्ति बिन्दुविभक्तिबिन्दुः आलेखे एकः बिन्दुः भवति यत्र अवतलता
परिवर्तते
।}\label{ux935ux92dux915ux924-ux92cux928ux926ux935ux92dux915ux924ux92cux928ux926-ux906ux932ux916-ux90fux915-ux92cux928ux926-ux92dux935ux924-ux92fux924ux930-ux905ux935ux924ux932ux924-ux92aux930ux935ux930ux924ux924}

\begin{itemize}
\tightlist
\item
  यदि \(f''(x) = 0\) अथवा \(f''(x)\) अपरिभाषितः अस्ति तर्हि बिन्दुः
  विभक्तिबिन्दुस्य अभ्यर्थी अस्ति ।
\item
  पुष्ट्यर्थं अवतलतायाः बिन्दुस्य उभयतः चिह्नं परिवर्तयितव्यम् ।
\end{itemize}

\subsubsection{उदाहरणम्}\label{ux909ux926ux939ux930ux923ux92e-6}

\begin{enumerate}
\def\labelenumi{\arabic{enumi}.}
\item
  \(f(x) = x^3\) इति

  \begin{itemize}
  \tightlist
  \item
    \(f''(x) = 6x\)।
  \item
    \(x = 0\), \(f''(0) = 0\) पर।
  \item
    \(x < 0\) कृते \(f''(x) < 0\) → अवतलः अधः ।
  \item
    \(x > 0\) कृते \(f''(x) > 0\) → अवतलः उपरि।
  \item
    एवं \((0,0)\) इति विभक्तिबिन्दुः ।
  \end{itemize}
\item
  \(f(x) = x^4\) इति

  \begin{itemize}
  \tightlist
  \item
    \(f''(x) = 12x^2\)।
  \item
    \(x = 0\) इत्यत्र, \(f''(0) = 0\) इत्यत्र, परन्तु अवतलता चिह्नं न परिवर्तयति
    (सदैव ≥ 0)।
  \item
    न विभक्तिबिन्दु।
  \end{itemize}
\end{enumerate}

\subsubsection{अवतलता एवं वक्र
रेखाचित्र}\label{ux905ux935ux924ux932ux924-ux90fux935-ux935ux915ux930-ux930ux916ux91aux924ux930}

\begin{itemize}
\tightlist
\item
  यदि \(f'(x) = 0\) तथा \(f''(x) > 0\), तर्हि \(f\) इत्यस्य स्थानीयं न्यूनतमं
  भवति ।
\item
  यदि \(f'(x) = 0\) तथा \(f''(x) < 0\), तर्हि \(f\) इत्यस्य स्थानीय अधिकतमं
  भवति ।
\item
  एतत् द्वितीयव्युत्पन्नपरीक्षा इति ज्ञायते ।
\end{itemize}

\subsubsection{एतत् किमर्थं
महत्त्वपूर्णम्}\label{ux90fux924ux924-ux915ux92eux930ux925-ux92eux939ux924ux924ux935ux92aux930ux923ux92e-1}

अवतलता, विभक्तिबिन्दवः च अस्मान् आलेखानां ``आकारं'' अवगन्तुं साहाय्यं कुर्वन्ति: ते कुत्र
नमन्ति, समतलं भवन्ति, वा भ्रमन्ति वा । एते विचाराः वक्रस्केचिंग्, भौतिकशास्त्रे
(त्वरणं), अर्थशास्त्रे (क्षीणप्रतिफलं) च केन्द्रस्थाः सन्ति ।

\subsubsection{अभ्यास}\label{ux905ux92dux92fux938-11}

\begin{enumerate}
\def\labelenumi{\arabic{enumi}.}
\tightlist
\item
  \(f(x) = x^3 - 3x\) कृते अवतलतायाः अन्तरालानि निर्धारयन्तु। तस्य विभक्तिबिन्दवः
  ज्ञातव्याः।
\item
  \(f(x) = \ln(x)\) कृते अवतलतां सम्भाव्यविभक्तिबिन्दून् च चिनुत।
\item
  महत्त्वपूर्णबिन्दून् वर्गीकरणार्थं \(f(x) = x^2 e^{-x}\) इत्यत्र
  द्वितीयव्युत्पन्नपरीक्षां प्रयोजयन्तु।
\item
  अवतलत्वस्य विभक्तिबिन्दुस्य च अन्तरालस्य चिह्नं कृत्वा \(f(x) = \sin x\) इति रेखांकनं
  कुर्वन्तु।
\item
  \(f(x) = e^x\) इत्यस्य विभक्तिबिन्दवः किमर्थं नास्ति इति व्याख्यातव्यम्।
\end{enumerate}

\subsection{3.5 वक्र
रेखाचित्रण}\label{ux935ux915ux930-ux930ux916ux91aux924ux930ux923}

वक्रस्केचिंग् इति कार्यस्य व्युत्पन्नानां सूचनानां उपयोगेन तस्य आलेखस्य आकर्षणस्य प्रक्रिया ।
अनेकबिन्दून् प्लॉट् कर्तुं न अपि तु वयं प्रमुखविशेषतानां विश्लेषणं कुर्मः: अवरोधाः,
असममिताः, वर्धमानाः/हतान्तः अन्तरालाः, अवतलता च ।

\subsubsection{वक्र स्केचिंग् कृते चरणाः1. डोमेन् : फंक्शन् कुत्र परिभाषितम् इति चिनुत
।}\label{ux935ux915ux930-ux938ux915ux91aux917-ux915ux924-ux91aux930ux9231.-ux921ux92eux928-ux92bux915ux936ux928-ux915ux924ux930-ux92aux930ux92dux937ux924ux92e-ux907ux924-ux91aux928ux924}

\begin{enumerate}
\def\labelenumi{\arabic{enumi}.}
\setcounter{enumi}{1}
\item
  अवरोधाः : आलेखः अक्षान् कुत्र पारयति इति ज्ञातव्यम्।
\item
  लक्षणहीनाः : १.

  \begin{itemize}
  \tightlist
  \item
    ऊर्ध्वाधर-लक्षणाः तत्र भवन्ति यत्र कार्यं अपरिभाषितं भवति, अनन्ततां प्रति प्रवृत्तं
    च भवति ।
  \item
    क्षैतिजः अथवा तिर्यक् लक्षणाः अन्त्यव्यवहारस्य वर्णनं \(x \to \pm\infty\) इति
    कुर्वन्ति ।
  \end{itemize}
\item
  प्रथम व्युत्पन्न \(f'(x)\): .

  \begin{itemize}
  \tightlist
  \item
    सकारात्मक → कार्यं वर्धमानम् अस्ति।
  \item
    नकारात्मकम् → कार्यं न्यूनं भवति।
  \item
    \(f'(x)\) → महत्वपूर्ण बिन्दु (संभव अधिकतम / न्यूनतम) के शून्य।
  \end{itemize}
\item
  द्वितीय व्युत्पन्न \(f''(x)\): .

  \begin{itemize}
  \tightlist
  \item
    सकारात्मक → अवतल ऊपर।
  \item
    नकारात्मक → अवतल अधः।
  \item
    शून्यानि वा अपरिभाषितानि → सम्भाव्यविभक्तिबिन्दवः।
  \end{itemize}
\item
  सूचनां संयोजयन्तु : स्पष्टं सटीकं च आलेखं रेखांकयितुं सर्वेषां परिणामानां उपयोगं कुर्वन्तु।
\end{enumerate}

\subsubsection{\texorpdfstring{उदाहरणम् १:
\(f(x) = x^3 - 3x\)}{उदाहरणम् १: f(x) = x\^{}3 - 3x}}\label{ux909ux926ux939ux930ux923ux92e-ux967-fx-x3---3x}

\begin{itemize}
\item
  डोमेन : सर्वाणि वास्तविकसङ्ख्यानि।
\item
  अवरोधयति: \((0,0)\) इत्यत्र।
\item
  व्युत्पन्न: \(f'(x) = 3x^2 - 3 = 3(x-1)(x+1)\)।

  \begin{itemize}
  \tightlist
  \item
    वर्धमानः \((-\infty, -1) \cup (1, \infty)\)।
  \item
    घटते: \((-1, 1)\)।
  \end{itemize}
\item
  द्वितीय व्युत्पन्न: \(f''(x) = 6x\)।

  \begin{itemize}
  \tightlist
  \item
    \(x < 0\) कृते अवतलं, \(x > 0\) कृते अवतलम्।
  \item
    \((0,0)\) पर विभक्ति बिन्दु।
  \end{itemize}
\item
  आकारः: \((-1, 2)\) इत्यत्र स्थानीय अधिकतमं, \((1, -2)\) इत्यत्र स्थानीयं न्यूनतमं
  च सह एकः S-वक्रः ।
\end{itemize}

\subsubsection{\texorpdfstring{उदाहरणम् २:
\(f(x) = \frac{1}{x}\)}{उदाहरणम् २: f(x) = \textbackslash frac\{1\}\{x\}}}\label{ux909ux926ux939ux930ux923ux92e-ux968-fx-frac1x}

\begin{itemize}
\item
  डोमेन: \(x \neq 0\)।
\item
  ऊर्ध्वाधर लक्षण: \(x = 0\)।
\item
  क्षैतिज असममित: \(y = 0\)।
\item
  व्युत्पन्न: \(f'(x) = -\frac{1}{x^2}\) (सदा नकारात्मक)। कार्यं सर्वदा न्यूनं
  भवति।
\item
  द्वितीय व्युत्पन्न: \(f''(x) = \frac{2}{x^3}\)।

  \begin{itemize}
  \tightlist
  \item
    \(x > 0\) कृते अवतलम्।
  \item
    \(x < 0\) कृते अवतलम्।
  \end{itemize}
\item
  आलेखः द्विशाखायुक्तः अतिशयोक्तिः।
\end{itemize}

\subsubsection{वक्र रेखाचित्रणं किमर्थं उपयोगी
अस्ति}\label{ux935ux915ux930-ux930ux916ux91aux924ux930ux923-ux915ux92eux930ux925-ux909ux92aux92fux917-ux905ux938ux924}

\begin{itemize}
\tightlist
\item
  सम्पूर्णगणना विना कार्याणां समग्रव्यवहारस्य अन्वेषणं प्रदाति।
\item
  गणितपरीक्षासु अनुप्रयुक्तसमस्यासु च आवश्यकम्।
\item
  बीजगणितीय विश्लेषणं ज्यामितीयबोधं च सेतुम् अङ्कयति।
\end{itemize}

\subsubsection{अभ्यास}\label{ux905ux92dux92fux938-12}

\begin{enumerate}
\def\labelenumi{\arabic{enumi}.}
\tightlist
\item
  \(f(x) = x^4 - 2x^2\) इत्यस्य वक्रस्य रेखांकनं कुरुत। अधिकतमं, न्यूनतमं,
  विभक्तिबिन्दून् च चिनुत।2. \(f(x) = \ln(x)\) विश्लेषणं कृत्वा रेखांकनं कुर्वन्तु।
  अवरोधाः, लक्षणाः, अवतलता च दर्शयन्तु।
\item
  \(f(x) = e^{-x}\) कृते वृद्धि/क्षयः, लक्षणं, अवतलता च वर्णयन्तु।
\item
  \((- \pi, \pi)\) अन्तरालस्य उपरि \(f(x) = \tan x\) इत्यस्य आलेखं रेखांकयन्तु।
  लक्षणं चिह्नितव्यम्।
\item
  \(f(x) = x^3 - 6x^2 + 9x\) इत्यस्य महत्त्वपूर्णबिन्दुवर्गीकरणार्थं
  प्रथमद्वितीयव्युत्पन्नपरीक्षायाः उपयोगं कुर्वन्तु।
\end{enumerate}

\section{द्वितीयः भागः।
अभिन्न}\label{ux926ux935ux924ux92f-ux92dux917-ux905ux92dux928ux928}

\section{अध्याय 4. व्युत्पन्न एवं निश्चित
अभिन्न}\label{ux905ux927ux92fux92f-4.-ux935ux92fux924ux92aux928ux928-ux90fux935-ux928ux936ux91aux924-ux905ux92dux928ux928}

\subsection{4.1 अनिश्चित
अभिन्न}\label{ux905ux928ux936ux91aux924-ux905ux92dux928ux928}

अनिश्चितः अभिन्नः भेदस्य विपरीतप्रक्रिया भवति । यदि व्युत्पन्नः परिवर्तनं मापयति
तर्हि अभिन्नः स्वस्य परिवर्तनस्य दरात् मूलकार्यं पुनः प्राप्नोति ।

\subsubsection{परिभाषा}\label{ux92aux930ux92dux937-4}

यदि \(F'(x) = f(x)\), तर्हि

\[
\int f(x)\,dx = F(x) + C,
\] इति

यत्र \(C\) एकीकरणस्य नित्यं भवति।

प्रत्येकं अनिश्चितं अभिन्नं केवलं नित्येन भिन्नं कार्यकुटुम्बं प्रतिनिधियति, यतः भेदेन नित्यं
निराकरणं भवति ।

\subsubsection{मूल नियम}\label{ux92eux932-ux928ux92fux92e}

\begin{enumerate}
\def\labelenumi{\arabic{enumi}.}
\tightlist
\item
  नित्यं नियमः
\end{enumerate}

\[
\int c\,dx = cx + C.
\] इति

\begin{enumerate}
\def\labelenumi{\arabic{enumi}.}
\setcounter{enumi}{1}
\tightlist
\item
  शक्तिनियमः
\end{enumerate}

\[
\int x^n\,dx = \frac{x^{n+1}}{n+1} + C, \quad n \neq -1.
\] इति

\begin{enumerate}
\def\labelenumi{\arabic{enumi}.}
\setcounter{enumi}{2}
\tightlist
\item
  योगनियमः
\end{enumerate}

\[
\int \big(f(x) + g(x)\big)\,dx = \int f(x)\,dx + \int g(x)\,dx.
\] इति

\begin{enumerate}
\def\labelenumi{\arabic{enumi}.}
\setcounter{enumi}{3}
\tightlist
\item
  नित्यं बहुविधः नियमः
\end{enumerate}

\[
\int c f(x)\,dx = c \int f(x)\,dx.
\] इति

\subsubsection{सामान्य
अभिन्न}\label{ux938ux92eux928ux92f-ux905ux92dux928ux928}

\begin{itemize}
\tightlist
\item
  \(\int e^x dx = e^x + C\)
\item
  \(\int \sin x dx = -\cos x + C\)
\item
  \(\int \cos x dx = \sin x + C\)
\item
  \(\int \frac{1}{x} dx = \ln|x| + C\)
\end{itemize}

\subsubsection{उदाहरणम्}\label{ux909ux926ux939ux930ux923ux92e-7}

\begin{enumerate}
\def\labelenumi{\arabic{enumi}.}
\item
  \(\int (3x^2 - 4)\,dx = x^3 - 4x + C\) इति ।
\item
  \(\int \cos(2x)\,dx = \tfrac{1}{2}\sin(2x) + C\) इति ।
\item
  \(\int \frac{1}{x}\,dx = \ln|x| + C\) इति ।
\end{enumerate}

\subsubsection{व्याख्या}\label{ux935ux92fux916ux92f-1}

\begin{itemize}
\tightlist
\item
  अनिश्चिताः अभिन्नाः प्रतिव्युत्पन्नाः भवन्ति।
\item
  ते निश्चिताभिन्नानाम् आधारः भवन्ति, ये क्षेत्रफलं, दूरं, द्रव्यमानं च इत्यादीनां
  सञ्चितमात्राणां मापनं कुर्वन्ति ।
\item
  अनुप्रयुक्तसन्दर्भेषु एकीकरणं अस्मान् दरात् पुनः कुलपर्यन्तं गन्तुं शक्नोति।
\end{itemize}

\subsubsection{अभ्यास}\label{ux905ux92dux92fux938-13}

\begin{enumerate}
\def\labelenumi{\arabic{enumi}.}
\tightlist
\item
  \(\int (5x^4 + 2x)\,dx\) ज्ञातव्यम्।2. \(\int (e^x + 3)\,dx\) गणना।
\item
  एकीकरणस्य उपयोगेन \(f'(x) = 6x\) इत्यस्य सामान्यसमाधानं ज्ञातव्यम्।
\item
  \(\int \frac{2}{x}\,dx\) मूल्याङ्कनं कुर्वन्तु।
\item
  यदि वेगः \(v(t) = 4t\) अस्ति तर्हि \(s(t)\) इति स्थितिकार्यं ज्ञातव्यम् ।
\end{enumerate}

\subsection{4.2 क्षेत्रत्वेन निश्चितः
अभिन्नः}\label{ux915ux937ux924ux930ux924ux935ux928-ux928ux936ux91aux924-ux905ux92dux928ux928}

अनिश्चिताः अभिन्नाः प्रतिव्युत्पन्नपरिवारानाम् प्रतिनिधित्वं कुर्वन्ति, निश्चितः अभिन्नः
संख्यात्मकं मूल्यं ददाति: द्वयोः बिन्दुयोः मध्ये वक्रस्य अधः सञ्चितः क्षेत्रः

\subsubsection{परिभाषा}\label{ux92aux930ux92dux937-5}

\([a, b]\) इत्यत्र परिभाषितस्य \(f(x)\) इत्यस्य फंक्शन् कृते निश्चितं अभिन्नं भवति

\[
\int_a^b f(x)\,dx = \lim_{n \to \infty} \sum_{i=1}^n f(x_i^-) \,\Delta x,
\] इति

यत्र \([a, b]\) अन्तरालः \(n\) विस्तारस्य \(\Delta x\) उपअन्तरालेषु विभक्तः
भवति, तथा च \(x_i^-\) प्रत्येकस्मिन् उपान्तरे नमूनाबिन्दुः भवति ।

एषा रीमैन् योगानाम् सीमा अस्ति ।

\subsubsection{ज्यामितीय
व्याख्या}\label{ux91cux92fux92eux924ux92f-ux935ux92fux916ux92f}

\begin{itemize}
\tightlist
\item
  यदि \([a, b]\) इत्यत्र \(f(x) \geq 0\), तर्हि \(\int_a^b f(x)\,dx\)
  \(x=a\) तः \(x=b\) पर्यन्तं वक्रस्य \(y = f(x)\) इत्यस्य अधः क्षेत्रस्य बराबरं
  भवति ।
\item
  यदि \(f(x)\) \(x\)-अक्षस्य अधः डुबति तर्हि अभिन्नः हस्ताक्षरितक्षेत्रस्य गणनां
  करोति: अक्षस्य अधः क्षेत्राणि ऋणात्मकरूपेण गण्यन्ते ।
\end{itemize}

\subsubsection{निश्चित अभिन्न के
गुण}\label{ux928ux936ux91aux924-ux905ux92dux928ux928-ux915-ux917ux923}

\begin{enumerate}
\def\labelenumi{\arabic{enumi}.}
\tightlist
\item
  अन्तरालेषु योजकता
\end{enumerate}

\[
\int_a^c f(x)\,dx = \int_a^b f(x)\,dx + \int_b^c f(x)\,dx.
\] इति

\begin{enumerate}
\def\labelenumi{\arabic{enumi}.}
\setcounter{enumi}{1}
\tightlist
\item
  सीमां विपर्ययम्
\end{enumerate}

\[
\int_a^b f(x)\,dx = -\int_b^a f(x)\,dx.
\] इति

\begin{enumerate}
\def\labelenumi{\arabic{enumi}.}
\setcounter{enumi}{2}
\tightlist
\item
  शून्य-विस्तार-अन्तरालम्
\end{enumerate}

\[
\int_a^a f(x)\,dx = 0.
\] इति

\begin{enumerate}
\def\labelenumi{\arabic{enumi}.}
\setcounter{enumi}{3}
\tightlist
\item
  रेखीयता
\end{enumerate}

\[
\int_a^b \big( cf(x) + g(x)\big)\,dx = c\int_a^b f(x)\,dx + \int_a^b g(x)\,dx.
\] इति

\subsubsection{उदाहरणम्}\label{ux909ux926ux939ux930ux923ux92e-8}

\begin{enumerate}
\def\labelenumi{\arabic{enumi}.}
\item
  \(\int_0^2 x\,dx = \left[\tfrac{1}{2}x^2\right]_0^2 = 2.\) इति इदं
  \(y=x\) रेखायाः अधः समकोणस्य क्षेत्रफलम् अस्ति ।
\item
  \(\int_{-1}^1 x^3\,dx = 0.\) इति विषम-फंक्शन् \(x^3\) इत्यस्य सममितक्षेत्राणि
  सन्ति ये रद्दं कुर्वन्ति ।
\item
  \(\int_0^\pi \sin x\,dx = 2.\) इति एतेन साइनवक्रस्य एकस्य तोरणस्य अधः
  क्षेत्रफलं समं भवति ।
\end{enumerate}

\subsubsection{एतत् किमर्थं
महत्त्वपूर्णम्}\label{ux90fux924ux924-ux915ux92eux930ux925-ux92eux939ux924ux924ux935ux92aux930ux923ux92e-2}

\begin{itemize}
\tightlist
\item
  निश्चिताः अभिन्नाः सञ्चितमात्राः मापयन्ति : दूरी, द्रव्यमानं, ऊर्जा, संभाव्यता।- ते
  बीजगणितीयगणनां ज्यामितीय-अन्तर्ज्ञानेन सह सेतुम् अकुर्वन् ।
\item
  अग्रिमः सोपानः गणितस्य मौलिकप्रमेयः अस्ति, यः निश्चिताभिन्नं प्रतिव्युत्पन्नैः सह
  संयोजयति ।
\end{itemize}

\subsubsection{अभ्यास}\label{ux905ux92dux92fux938-14}

\begin{enumerate}
\def\labelenumi{\arabic{enumi}.}
\tightlist
\item
  \(\int_0^3 (2x+1)\,dx\) गणना।
\item
  \(y = x^2\) तथा \(x\)-अक्षयोः मध्ये \(x = 0\) तः \(x = 2\) पर्यन्तं क्षेत्रं
  ज्ञातव्यम् ।
\item
  \(\int_{-2}^2 (x^2 - 1)\,dx\) मूल्याङ्कनं कुर्वन्तु।
\item
  \(\int_{-a}^a f(x)\,dx = 0\) इति दर्शयतु यदि \(f(x)\) विषमः अस्ति।
\item
  \(n=4\) उप-अन्तरालैः सह दक्षिण-अन्तबिन्दुभिः च सह Riemann योगस्य उपयोगेन
  \(\int_0^1 e^x\,dx\) इत्यस्य अनुमानं कुर्वन्तु ।
\end{enumerate}

\subsection{4.3 गणितस्य मौलिक
प्रमेयम्}\label{ux917ux923ux924ux938ux92f-ux92eux932ux915-ux92aux930ux92eux92fux92e}

गणितस्य मौलिकप्रमेयः (FTC) गणितस्य मुख्यविचारद्वयं एकीकृत्य भवति : भेदः एकीकरणं च ।
क्षेत्राणां अन्वेषणं परिवर्तनस्य दरं च अन्वेष्टुं एकस्यैव मुद्रायाः द्वौ पक्षौ इति दर्शयति ।

\subsubsection{भाग 1: अभिन्नस्य
भेदः}\label{ux92dux917-1-ux905ux92dux928ux928ux938ux92f-ux92dux926}

यदि \(f\) \([a, b]\) इत्यत्र निरन्तरं भवति तर्हि परिभाषयन्तु

\[
F(x) = \int_a^x f(t)\,dt.
\] इति

अथ \(F\) इति भेद्यम्, च

\[
F'(x) = f(x).
\] इति

शब्देषु : सञ्चितक्षेत्रफलनस्य व्युत्पन्नं मूलफलमेव ।

\subsubsection{भाग 2: निश्चित अभिन्नानाम्
मूल्याङ्कनम्}\label{ux92dux917-2-ux928ux936ux91aux924-ux905ux92dux928ux928ux928ux92e-ux92eux932ux92fux919ux915ux928ux92e}

यदि \(f\) \([a, b]\) इत्यत्र निरन्तरं भवति तथा च \(F\) \(f\) इत्यस्य कोऽपि
प्रतिव्युत्पन्नः अस्ति, तर्हि

\[
\int_a^b f(x)\,dx = F(b) - F(a).
\] इति

एतेन अस्मान् ज्ञायते यत् वयं केवलं प्रतिव्युत्पन्नं अन्विष्य निश्चिताभिन्नानाम् मूल्याङ्कनं कर्तुं
शक्नुमः, न तु रीमैन् योगानाम् सीमानां गणनां कृत्वा ।

\subsubsection{उदाहरणम्}\label{ux909ux926ux939ux930ux923ux92e-9}

\begin{enumerate}
\def\labelenumi{\arabic{enumi}.}
\item
  \(\int_0^2 x^2\,dx\) इति ।

  \begin{itemize}
  \tightlist
  \item
    व्युत्पन्नविरोधी: \(F(x) = \tfrac{1}{3}x^3\).
  \item
    FTC लागू करें: \(F(2) - F(0) = \tfrac{8}{3} - 0 = \tfrac{8}{3}.\)
  \end{itemize}
\item
  यदि \(F(x) = \int_1^x \cos t \, dt\), तर्हि \(F'(x) = \cos x\)।
\item
  \(\int_1^4 \frac{1}{x}\,dx\) इति ।

  \begin{itemize}
  \tightlist
  \item
    व्युत्पन्नविरोधी: \(\ln|x|\).
  \item
    FTC लागू करें: \(\ln 4 - \ln 1 = \ln 4.\)
  \end{itemize}
\end{enumerate}

\subsubsection{FTC किमर्थं
महत्त्वपूर्णम्}\label{ftc-ux915ux92eux930ux925-ux92eux939ux924ux924ux935ux92aux930ux923ux92e}

\begin{itemize}
\tightlist
\item
  सीमाप्रक्रियातः एकीकरणं व्यावहारिकगणनायां परिणमयति ।- भेदः एकीकरणं च
  विलोमक्रियाः इति पुष्टिं करोति ।
\item
  एतत् केन्द्रीयप्रमेयम् अस्ति यत् गणितं, विज्ञानं, अभियांत्रिकी च इत्यत्र गणितं उपयोगी
  करोति ।
\end{itemize}

\subsubsection{अभ्यास}\label{ux905ux92dux92fux938-15}

\begin{enumerate}
\def\labelenumi{\arabic{enumi}.}
\tightlist
\item
  FTC इत्यस्य उपयोगेन \(\int_0^3 (2x+1)\,dx\) इत्यस्य मूल्याङ्कनं कुर्वन्तु।
\item
  यदि \(F(x) = \int_0^x e^t\,dt\), \(F'(x)\) ज्ञातव्य।
\item
  \(\int_0^\pi \sin x \, dx\) गणना करें।
\item
  दर्शयतु यत् यदि \(f'(x) = g(x)\), तर्हि
  \(\int_a^b g(x)\,dx = f(b) - f(a)\)।
\item
  \(0\) तः \(\pi/2\) पर्यन्तं \(y = \cos x\) इत्यस्य अधः क्षेत्रं 1 इत्यस्य बराबरं
  किमर्थम् इति व्याख्यातुं FTC इत्यस्य उपयोगं कुर्वन्तु ।
\end{enumerate}

\subsection{4.4 अभिन्नस्य
गुणाः}\label{ux905ux92dux928ux928ux938ux92f-ux917ux923}

निश्चित अभिन्नस्य अनेकाः महत्त्वपूर्णाः गुणाः सन्ति ये अनुप्रयोगेषु लचीलाः शक्तिशाली च
भवन्ति । एते गुणाः योगस्य सीमारूपेण परिभाषातः गणितस्य मौलिकप्रमेयात् च अनुवर्तन्ते ।

\subsubsection{रेखीयता}\label{ux930ux916ux92fux924}

\(f(x)\) तथा \(g(x)\) इति कार्याणां कृते, तथा च \(c, d\) इति स्थिरांकानाम् कृते:

\[
\int_a^b \big(c f(x) + d g(x)\big)\,dx = c \int_a^b f(x)\,dx + d \int_a^b g(x)\,dx.
\] इति

एतेन जटिलानि अभिन्नं सरलतरेषु भागेषु भङ्गयितुं शक्नुमः ।

\subsubsection{अन्तरालस्य उपरि
योजकता}\label{ux905ux928ux924ux930ux932ux938ux92f-ux909ux92aux930-ux92fux91cux915ux924}

यदि \(a < c < b\), तर्हि

\[
\int_a^b f(x)\,dx = \int_a^c f(x)\,dx + \int_c^b f(x)\,dx.
\] इति

वयं खण्डखण्डे अभिन्नस्य गणनां कर्तुं शक्नुमः ।

\subsubsection{सीमाविपर्ययः}\label{ux938ux92eux935ux92aux930ux92fux92f}

\[
\int_a^b f(x)\,dx = -\int_b^a f(x)\,dx.
\] इति

सीमां स्वैपिंग कृत्वा अभिन्नस्य चिह्नं परिवर्तते ।

\subsubsection{तुलना
सम्पत्ति}\label{ux924ux932ux928-ux938ux92eux92aux924ux924}

यदि \([a, b]\) मध्ये सर्वेषां \(x\) कृते \(f(x) \leq g(x)\), तर्हि

\[
\int_a^b f(x)\,dx \leq \int_a^b g(x)\,dx.
\] इति

एतेन प्रत्यक्षगणना विना क्षेत्राणां तुलना कर्तुं शक्यते ।

\subsubsection{निरपेक्ष मूल्य
असमानता}\label{ux928ux930ux92aux915ux937-ux92eux932ux92f-ux905ux938ux92eux928ux924}

\[
\left| \int_a^b f(x)\,dx \right| \leq \int_a^b |f(x)|\,dx.
\] इति

विश्लेषणे अभिसरणपरीक्षासु च एषः गुणः अत्यावश्यकः अस्ति ।

\subsubsection{समरूपता}\label{ux938ux92eux930ux92aux924}

\begin{itemize}
\item
  यदि \(f(x)\) समः (\(y\)-अक्षस्य विषये सममितम्):

  \[
  \int_{-a}^a f(x)\,dx = 2\int_0^a f(x)\,dx.
  \] इति
\item
  यदि \(f(x)\) विषम (उत्पत्तिविषये सममित) अस्ति :

  \[
  \int_{-a}^a f(x)\,dx = 0.
  \] इति\#\#\# उदाहरणम्
\end{itemize}

\begin{enumerate}
\def\labelenumi{\arabic{enumi}.}
\item
  \(\int_0^2 (3x^2 + 4)\,dx = \int_0^2 3x^2\,dx + \int_0^2 4\,dx = 8 + 8 = 16.\)
  इति
\item
  \(f(x) = x^3\) विषमत्वात् \(\int_{-1}^1 x^3\,dx = 0.\) इति
\item
  \(f(x) = x^2\) समत्वात्
  \(\int_{-2}^2 x^2\,dx = 2\int_0^2 x^2\,dx = 2\cdot \tfrac{8}{3} = \tfrac{16}{3}.\)
  इति
\end{enumerate}

\subsubsection{एते गुणाः किमर्थं महत्त्वपूर्णाः
सन्ति}\label{ux90fux924-ux917ux923-ux915ux92eux930ux925-ux92eux939ux924ux924ux935ux92aux930ux923-ux938ux928ux924}

\begin{itemize}
\tightlist
\item
  ते गणनां सरलीकरोति।
\item
  ते कार्याणां ज्यामितीयसमरूपताविशेषतां प्रकाशयन्ति।
\item
  ते अधिक उन्नतविश्लेषणार्थं सैद्धान्तिकसाधनं प्रददति।
\end{itemize}

\subsubsection{अभ्यास}\label{ux905ux92dux92fux938-16}

\begin{enumerate}
\def\labelenumi{\arabic{enumi}.}
\tightlist
\item
  \(\int_{-5}^5 (x^4 - x^3)\,dx\) मूल्याङ्कनार्थं समरूपतायाः उपयोगं कुर्वन्तु।
\item
  \(\int_1^4 (2x+3)\,dx = \int_1^2 (2x+3)\,dx + \int_2^4 (2x+3)\,dx\)
  इति दर्शयतु।
\item
  \(\int_0^\pi \sin(x)\,dx\) इत्यस्य मूल्याङ्कनं कृत्वा
  \(\int_{-\pi}^\pi \sin(x)\,dx\) इत्यनेन सह तुलनां कुर्वन्तु।
\item
  सिद्धं कुरुत यत् यदि \([a, b]\) इत्यत्र \(f(x) \geq 0\) तर्हि
  \(\int_a^b f(x)\,dx \geq 0\) इति।
\item
  सम/विषम गुणानाम् उपयोगेन \(\int_{-3}^3 (x^2 + 1)\,dx\) गणनां कुर्वन्तु।
\end{enumerate}

\section{अध्याय 5. एकीकरण की
तकनीक}\label{ux905ux927ux92fux92f-5.-ux90fux915ux915ux930ux923-ux915-ux924ux915ux928ux915}

\subsection{5.1 प्रतिस्थापन}\label{ux92aux930ux924ux938ux925ux92aux928}

एकीकरणस्य एकः उपयोगी युक्तिः प्रतिस्थापनविधिः अस्ति, या -उ-प्रतिस्थापनम्- इति अपि
कथ्यते । व्युत्पन्नानां कृते श्रृङ्खलानियमस्य विपरीतप्रक्रिया अस्ति ।

\subsubsection{विचारः}\label{ux935ux91aux930-1}

यदि कस्मिन् अपि अभिन्नस्य समष्टिफलनं भवति तर्हि वयं चरं परिवर्त्य सरलीकर्तुं शक्नुमः ।

औपचारिकरूपेण यदि \(u = g(x)\) भेद्यकार्यं भवति तर्हि

\[
\int f(g(x)) g'(x)\,dx = \int f(u)\,du.
\] इति

एतेन प्रतिस्थापनेन अभिन्नस्य मूल्याङ्कनं सुलभं भवति ।

\subsubsection{प्रतिस्थापनार्थं
पदानि}\label{ux92aux930ux924ux938ux925ux92aux928ux930ux925-ux92aux926ux928}

\begin{enumerate}
\def\labelenumi{\arabic{enumi}.}
\tightlist
\item
  एकं आन्तरिकं फंक्शन् \(u = g(x)\) चिनोतु यस्य व्युत्पन्नं अपि अभिन्नस्य मध्ये दृश्यते ।
\item
  \(du = g'(x)\,dx\) गणना करें।
\item
  \(u\) इत्यस्य दृष्ट्या अभिन्नं पुनः लिखत।
\item
  \(u\) इत्यस्य विषये एकीकरणं कुर्वन्तु।
\item
  प्रतिस्थापनं पुनः \(u = g(x)\)।
\end{enumerate}

\subsubsection{उदाहरणम्}\label{ux909ux926ux939ux930ux923ux92e-10}

\begin{enumerate}
\def\labelenumi{\arabic{enumi}.}
\item
  सरलप्रतिस्थापनम्

  \[
  \int 2x \cos(x^2)\,dx
  \] इति

  अस्तु \(u = x^2\), अतः \(du = 2x\,dx\)। तदा अभिन्न
  \(\int \cos u \,du = \sin u + C = \sin(x^2) + C\) भवति।
\item
  लघुगणकीय प्रकरण

  \[ इति\int \frac{2x}{x^2+1}\,dx
  \]

  Let \(u = x^2 + 1\), so \(du = 2x\,dx\). Then integral becomes
  \(\int \frac{1}{u}\,du = \ln|u| + C = \ln(x^2+1) + C\).
\item
  Trigonometric substitution

  \[ इति
  \int \sin(3x)\,dx
  \]

  Let \(u = 3x\), so \(du = 3\,dx\), hence \(dx = \frac{du}{3}\).
  Integral becomes
  \(\tfrac{1}{3}\int \sin u\,du = -\tfrac{1}{3}\cos u + C = -\tfrac{1}{3}\cos(3x) + C\).
\end{enumerate}

\subsubsection{Definite Integrals with
Substitution}\label{definite-integrals-with-substitution}

When evaluating definite integrals, we must also change the limits:

\[ इति
\int_a^b f(g(x)) g'(x)\,dx = \int_{g(a)}^{g(b)} f(u)\,दु.
\]

Example:

\[
\int_0^1 2x ई^{x^2}\,dx.
\]

Let \(u = x^2\), \(du = 2x\,dx\). Limits: when \(x=0, u=0\); when
\(x=1, u=1\). So the integral becomes

\[ इति
\int_0^1 ई^उ\,दु = ई - 1.
\]

\subsubsection{Exercises}\label{exercises-1}

\begin{enumerate}
\def\labelenumi{\arabic{enumi}.}
\tightlist
\item
  Evaluate \(\int (x^2+1)^5 (2x)\,dx\).
\item
  Compute \(\int \frac{\cos x}{\sin x}\,dx\).
\item
  Evaluate \(\int_0^\pi \sin(2x)\,dx\) using substitution.
\item
  Find \(\int e^{3x}\,dx\).
\item
  Compute \(\int \frac{1}{\sqrt{1+x^2}}\,dx\) by letting \(u = 1+x^2\).
\end{enumerate}

\subsection{5.2 Integration by Parts}\label{integration-by-parts}

Integration by parts is a technique that comes from the product rule for
derivatives. It helps evaluate integrals involving products of functions
that are not easily handled by substitution alone.

\subsubsection{The Formula}\label{the-formula}

From the product rule:

\[ इति
\frac{d}{dx}[u(x)v(x)] = उ'(x)v(x) + उ(x)v'(x) ।
\]

Integrating both sides gives the integration by parts formula:

\[ इति
\इन्त उ\,द्व = उव - \इन्त व\,दु।
\]

Here:

\begin{itemize}
\tightlist
\item
  \(u\) = a function chosen to be differentiated,
\item
  \(dv\) = the remaining part of the integrand to be integrated.
\end{itemize}

\subsubsection{\texorpdfstring{Choosing \(u\) and
\(dv\)}{Choosing u and dv}}\label{choosing-u-and-dv}

A common guideline is LIATE (Logarithmic, Inverse trig, Algebraic,
Trigonometric, Exponential).

\begin{itemize}
\tightlist
\item
  Choose \(u\) from the earliest category present.
\item
  Choose \(dv\) as the rest.
\end{itemize}

\subsubsection{Examples}\label{examples-1}

\begin{enumerate}
\def\labelenumi{\arabic{enumi}.}
\tightlist
\item
  Polynomial × Exponential
\end{enumerate}

\[ इति
\int x e^x\,dx
\] इति\(u = x\), \(dv = e^x dx\) इति । ततः \(du = dx\), \(v = e^x\) इति
।

\[
\int x e^x\,dx = x e^x - \int e^x dx = x e^x - e^x + C.
\] इति

\begin{enumerate}
\def\labelenumi{\arabic{enumi}.}
\setcounter{enumi}{1}
\tightlist
\item
  बहुपद × त्रिग
\end{enumerate}

\[
\int x \cos x\,dx
\] इति

\(u = x\), \(dv = \cos x dx\) इति । अथ \(du = dx\), \(v = \sin x\) इति ।

\[
\int x \cos x\,dx = x \sin x - \int \sin x dx = x \sin x + \cos x + C.
\] इति

\begin{enumerate}
\def\labelenumi{\arabic{enumi}.}
\setcounter{enumi}{2}
\tightlist
\item
  लघुगणकम्
\end{enumerate}

\[
\int \ln x\,dx
\] इति

\(u = \ln x\), \(dv = dx\) इति । ततः \(du = \frac{1}{x}dx\), \(v = x\)
इति ।

\[
\int \ln x\,dx = x \ln x - \int 1 dx = x \ln x - x + C.
\]

\subsubsection{निश्चित अभिन्न
उदाहरण}\label{ux928ux936ux91aux924-ux905ux92dux928ux928-ux909ux926ux939ux930ux923}

\[
\int_0^1 x e^x\,dx
\] इति

पूर्वफलस्य उपयोगेन: \(\int x e^x dx = (x-1)e^x\). गणयति:

\[
\big[(x-1)e^x\big]_0^1 = (0)e^1 - (-1)e^0 = 0 + 1 = 1.
\] इति

\subsubsection{एतत् किमर्थं
महत्त्वपूर्णम्}\label{ux90fux924ux924-ux915ux92eux930ux925-ux92eux939ux924ux924ux935ux92aux930ux923ux92e-3}

यदा प्रतिस्थापनं विफलं भवति तदा भागैः एकीकरणं महत्त्वपूर्णं भवति, विशेषतः लघुगणकैः,
विलोमत्रिकोणमापीफलनैः, घातीयैः अथवा ट्रिग्फलनैः सह बहुपदैः सह सम्बद्धैः उत्पादैः सह

\subsubsection{अभ्यास}\label{ux905ux92dux92fux938-17}

\begin{enumerate}
\def\labelenumi{\arabic{enumi}.}
\tightlist
\item
  \(\int x \sin x\,dx\) मूल्याङ्कनं कुर्वन्तु।
\item
  \(\int e^x \cos x\,dx\) ज्ञातव्यम्।
\item
  \(\int_1^2 \ln x\,dx\) गणना करें।
\item
  \(\int x^2 e^x\,dx\) मूल्याङ्कनं कुर्वन्तु।
\item
  \(\int \arctan x\,dx = x\arctan x - \tfrac{1}{2}\ln(1+x^2) + C\)
  दर्शयितुं भागैः एकीकरणस्य उपयोगं कुर्वन्तु।
\end{enumerate}

\subsection{5.3 त्रिकोणमितीय अभिन्न एवं
प्रतिस्थापन}\label{ux924ux930ux915ux923ux92eux924ux92f-ux905ux92dux928ux928-ux90fux935-ux92aux930ux924ux938ux925ux92aux928}

अनेकाः अभिन्नाः त्रिकोणमितीयकार्यं सम्मिलितं कुर्वन्ति । एतेषां प्रायः तादात्मानां
उपयोगेन विशेषप्रतिस्थापनेन वा सरलीकरणं कर्तुं शक्यते ।

\subsubsection{त्रिकोणमितीय
अभिन्न}\label{ux924ux930ux915ux923ux92eux924ux92f-ux905ux92dux928ux928}

\begin{enumerate}
\def\labelenumi{\arabic{enumi}.}
\tightlist
\item
  साइनस्य कोसाइनस्य च शक्तिः
\end{enumerate}

\begin{itemize}
\tightlist
\item
  यदि sine इत्यस्य शक्तिः विषमः अस्ति: एकं \(\sin x\) रक्षन्तु, शेषं
  \(\sin^2x = 1 - \cos^2x\) इत्यनेन परिवर्तयन्तु, \(u = \cos x\) इत्यनेन
  प्रतिस्थापयन्तु ।
\item
  यदि कोसाइनस्य शक्तिः विषमः अस्ति: एकं \(\cos x\) रक्षन्तु, शेषं
  \(\cos^2x = 1 - \sin^2x\) इत्यनेन परिवर्तयन्तु, \(u = \sin x\) इत्यनेन
  प्रतिस्थापयन्तु ।
\item
  यदि उभयम् अपि समं भवति : अर्धकोणपरिचयानां प्रयोगं कुर्वन्तु।
\end{itemize}

उदाहरण:

\[
\int \sin^3x \cos x \, dx
\] इति

आस्तु \(u = \sin x\), \(du = \cos x\,dx\):

\(du = dx\) इति \int u\^{}3,du = \tfrac{u^4}{4} + C = \tfrac{\sin^4x}{4}
+ सी.\$\$

\begin{enumerate}
\def\labelenumi{\arabic{enumi}.}
\setcounter{enumi}{1}
\tightlist
\item
  Products of sine and cosine with different angles Use product-to-sum
  formulas:
\end{enumerate}

\[ इति
\sin A \cos B = \tfrac{1}{2}[\sin(A+B) + \sin(A-B)].
\]

Example:

\[ इति
\int \ sin(2x)\cos(3x)\,dx = \tfrac{1}{2}\int [\sin(5x) - \सिन(x)]\,dx.
\]

\begin{enumerate}
\def\labelenumi{\arabic{enumi}.}
\setcounter{enumi}{2}
\tightlist
\item
  Powers of secant and tangent
\end{enumerate}

\begin{itemize}
\tightlist
\item
  If the power of secant is even: save \(\sec^2x\), convert the rest
  with \(\sec^2x = 1 + \tan^2x\), and substitute \(u = \tan x\).
\item
  If the power of tangent is odd: save \(\sec^2x\), convert the rest
  with \(\tan^2x = \sec^2x - 1\), and substitute \(u = \tan x\).
\end{itemize}

Example:

\[
\int \tan^3x \sec^2x \, dx
\]

Let \(u = \tan x\), \(du = \sec^2x\,dx\):

\[ इति
\int उ^३\,दु = \त्फ्राक्{उ^४}{४} + सी = \त्फ्राक्{\तन^४x}{४} + सी.
\]

\subsubsection{Trigonometric
Substitutions}\label{trigonometric-substitutions}

For integrals involving \(\sqrt{a^2 - x^2}\), \(\sqrt{a^2 + x^2}\), or
\(\sqrt{x^2 - a^2}\), use special substitutions:

\begin{enumerate}
\def\labelenumi{\arabic{enumi}.}
\tightlist
\item
  \(x = a \sin \theta\), for \(\sqrt{a^2 - x^2}\).
\item
  \(x = a \tan \theta\), for \(\sqrt{a^2 + x^2}\).
\item
  \(x = a \sec \theta\), for \(\sqrt{x^2 - a^2}\).
\end{enumerate}

Example:

\[ इति
\int \sqrt{a^2 - x^2}\,dx
\]

Let \(x = a\sin\theta\), so \(dx = a\cos\theta\,d\theta\):

\[ इति
\int \sqrt{a^2 - a^2\sin^2\theta}(a\cos\theta\,d\theta) = \int a^2 \cos^2\theta \, d\theta.
\] इति

अर्धकोणपरिचयानां उपयोगं सरलीकरोतु।

\subsubsection{एतानि युक्तयः किमर्थं महत्त्वपूर्णाः
सन्ति}\label{ux90fux924ux928-ux92fux915ux924ux92f-ux915ux92eux930ux925-ux92eux939ux924ux924ux935ux92aux930ux923-ux938ux928ux924}

\begin{itemize}
\tightlist
\item
  ते कठिनबीजगणितरूपं प्रबन्धनीयत्रिकोणमितीयरूपेषु परिवर्तयन्ति।
\item
  क्षेत्राणि, आयतनं, चापदीर्घता च इत्यादिषु समस्यासु ते विशेषतया उपयोगिनो भवन्ति ।
\item
  ते उन्नतसमायोजनपद्धतीनां आधारं स्थापयन्ति।
\end{itemize}

\subsubsection{अभ्यास}\label{ux905ux92dux92fux938-18}

\begin{enumerate}
\def\labelenumi{\arabic{enumi}.}
\tightlist
\item
  \(\int \sin^4x \cos^2x \, dx\) मूल्याङ्कनं कुर्वन्तु।
\item
  \(\int \sin(5x)\cos(2x)\,dx\) गणना।
\item
  \(\int \tan^2x \sec^2x \, dx\) मूल्याङ्कनं कुर्वन्तु।
\item
  प्रतिस्थापनस्य उपयोगेन \(\int \sqrt{9 - x^2}\,dx\) ज्ञातव्यम्।
\item
  \(x = a\tan\theta\) इत्यस्य उपयोगेन
  \(\int \frac{dx}{\sqrt{x^2 + a^2}} = \ln|x + \sqrt{x^2 + a^2}| + C\)
  इति दर्शयतु।
\end{enumerate}

\subsection{5.4 आंशिक भिन्नतर्कसंगतफलनानां (बहुपदानां अनुपाताः) एकीकरणे एकः
शक्तिशाली विधिः आंशिकअंशविघटनम् अस्ति । एषा प्रविधिः जटिलं अंशं सरलतरअंशानां योगरूपेण
अभिव्यञ्जयति येषां एकीकरणं सुलभतरं भवति
।}\label{ux906ux936ux915-ux92dux928ux928ux924ux930ux915ux938ux917ux924ux92bux932ux928ux928-ux92cux939ux92aux926ux928-ux905ux928ux92aux924-ux90fux915ux915ux930ux923-ux90fux915-ux936ux915ux924ux936ux932-ux935ux927-ux906ux936ux915ux905ux936ux935ux918ux91fux928ux92e-ux905ux938ux924-ux90fux937-ux92aux930ux935ux927-ux91cux91fux932-ux905ux936-ux938ux930ux932ux924ux930ux905ux936ux928-ux92fux917ux930ux92aux923-ux905ux92dux935ux92fux91eux91cux92fux924-ux92fux937-ux90fux915ux915ux930ux923-ux938ux932ux92dux924ux930-ux92dux935ux924}

\subsubsection{विचारः}\label{ux935ux91aux930-2}

यदि \(R(x) = \frac{P(x)}{Q(x)}\) एकं तर्कसंगतं फंक्शन् अस्ति, यत्र \(P(x)\) इत्यस्य
डिग्री \(Q(x)\) इत्यस्य डिग्री इत्यस्मात् न्यूना भवति, तर्हि वयं \(R(x)\) इत्यस्य
सरलतर-अंशेषु विघटयितुं शक्नुमः

एते सरलतरखण्डाः हरस्य \(Q(x)\) इत्यस्य गुणनखण्डैः सह सङ्गच्छन्ति ।

\subsubsection{सामान्य रूप}\label{ux938ux92eux928ux92f-ux930ux92a}

\begin{enumerate}
\def\labelenumi{\arabic{enumi}.}
\tightlist
\item
  विशिष्ट रेखीय कारक यदि
\end{enumerate}

\[
\frac{1}{(x-a)(x-b)},
\] इति

ततः यथा यथा

\[
\frac{A}{x-a} + \frac{B}{x-b}.
\] इति

\begin{enumerate}
\def\labelenumi{\arabic{enumi}.}
\setcounter{enumi}{1}
\tightlist
\item
  पुनरावृत्ति रेखीय कारक यदि हरकस्य \((x-a)^n\) अस्ति तर्हि पदाः सन्ति
\end{enumerate}

\[
\frac{A_1}{x-a} + \frac{A_2}{(x-a)^2} + \dots + \frac{A_n}{(x-a)^n}.
\] इति

\begin{enumerate}
\def\labelenumi{\arabic{enumi}.}
\setcounter{enumi}{2}
\tightlist
\item
  अनिवृत्त द्विघातकारक यदि हरस्य \((x^2+bx+c)\) अस्ति, तर्हि गणकः रेखीयः अस्ति:
\end{enumerate}

\[
\frac{Ax+B}{x^2+bx+c}.
\] इति

\subsubsection{उदाहरणम् १ : विशिष्टाः
रेखीयकारकाः}\label{ux909ux926ux939ux930ux923ux92e-ux967-ux935ux936ux937ux91f-ux930ux916ux92fux915ux930ux915}

\[
\int \frac{1}{x^2 - 1}\,dx
\] इति

कारक हर: \((x-1)(x+1)\)। विघटनम् : १.

\[
\frac{1}{x^2-1} = \frac{1}{2}\left(\frac{1}{x-1} - \frac{1}{x+1}\right).
\] इति

एकीकृत्य : १.

\[
\int \frac{1}{x^2 - 1}\,dx = \tfrac{1}{2}\ln\left|\frac{x-1}{x+1}\right| + C.
\] इति

\subsubsection{उदाहरणम् २ : पुनरावृत्तिः
रेखीयकारकः}\label{ux909ux926ux939ux930ux923ux92e-ux968-ux92aux928ux930ux935ux924ux924-ux930ux916ux92fux915ux930ux915}

\[
\int \frac{1}{(x-1)^2}\,dx
\] इति

एतत् पूर्वमेव सरलम् अस्ति : १.

\[
\int (x-1)^{-2}\,dx = -\frac{1}{x-1} + C.
\]

\subsubsection{उदाहरण 3: अपरिमेय द्विघात
कारक}\label{ux909ux926ux939ux930ux923-3-ux905ux92aux930ux92eux92f-ux926ux935ux918ux924-ux915ux930ux915}

\[
\int \frac{x}{x^2+1}\,dx
\] इति

\(u = x^2+1\) इति प्रतिस्थापयन्तु, अथवा न्युमरेटर् हरस्य व्युत्पन्नम् इति ज्ञापयन्तु ।

\[
\int \frac{x}{x^2+1}\,dx = \tfrac{1}{2}\ln(x^2+1) + C.
\] इति

\subsubsection{आंशिक भिन्न अपघटन में
चरण}\label{ux906ux936ux915-ux92dux928ux928-ux905ux92aux918ux91fux928-ux92e-ux91aux930ux923}

\begin{enumerate}
\def\labelenumi{\arabic{enumi}.}
\tightlist
\item
  हरस्य गुणनखण्डं कुरुत।
\item
  सामान्यं आंशिकं भिन्नरूपं लिखत।
\item
  अंशान् स्वच्छं कर्तुं हरेन माध्यमेन गुणयन्तु।
\item
  अज्ञातनित्यानां कृते समाधानं कुरुत।
\item
  प्रत्येकं पदं एकीकृत्य स्थापयन्तु।\#\#\# एतत् किमर्थं महत्त्वपूर्णम्
\end{enumerate}

\begin{itemize}
\tightlist
\item
  जटिल तर्कसंगतफलनानि सरल लघुगणकीय अथवा चाप स्पर्शरेखारूपेषु परिवर्तयति ।
\item
  विशेषतया अवकलसमीकरणेषु तथा लैप्लेसरूपान्तरणेषु उपयोगी।
\item
  उन्नतगणनायां अभियांत्रिकीयां च मौलिकः।
\end{itemize}

\subsubsection{अभ्यास}\label{ux905ux92dux92fux938-19}

\begin{enumerate}
\def\labelenumi{\arabic{enumi}.}
\tightlist
\item
  \(\int \frac{3x+5}{x^2-1}\,dx\) विघट्य एकीकृत्य।
\item
  \(\int \frac{1}{x^2(x+1)}\,dx\) मूल्याङ्कनं कुर्वन्तु।
\item
  \(\int \frac{2x+1}{x^2+2x+2}\,dx\) गणना करें।
\item
  \(\int \frac{1}{x^3 - x}\,dx\) ज्ञातव्यम्।
\item
  आंशिकअंशानां अथवा प्रतिस्थापनस्य उपयोगेन
  \(\int \frac{dx}{x^2+1} = \arctan x + C\) इति दर्शयतु।
\end{enumerate}

\subsection{5.5 अनुचित
अभिन्न}\label{ux905ux928ux91aux924-ux905ux92dux928ux928}

केचन अभिन्नाः प्रत्यक्षतया मूल्याङ्कनं कर्तुं न शक्यन्ते यतोहि अन्तरालः अनन्तः अस्ति अथवा
अभिन्नः असीमः भवति । एते अनुचिता अभिन्नाः इति उच्यन्ते । सीमानां उपयोगेन ते
परिभाषिताः भवन्ति ।

\subsubsection{परिभाषा}\label{ux92aux930ux92dux937-6}

\begin{enumerate}
\def\labelenumi{\arabic{enumi}.}
\tightlist
\item
  अनन्तान्तरे
\end{enumerate}

\[
\int_a^\infty f(x)\,dx = \lim_{b \to \infty} \int_a^b f(x)\,dx.
\] इति

\[
\int_{-\infty}^a f(x)\,dx = \lim_{b \to -\infty} \int_b^a f(x)\,dx.
\] इति

\begin{enumerate}
\def\labelenumi{\arabic{enumi}.}
\setcounter{enumi}{1}
\tightlist
\item
  असीम अभिन्न यदि \(f(x)\) इत्यत्र \(c\) इत्यत्र ऊर्ध्वाधर-लक्षणं भवति तर्हि
\end{enumerate}

\[
\int_a^c f(x)\,dx = \lim_{t \to c^-} \int_a^t f(x)\,dx,
\] इति

\[
\int_c^b f(x)\,dx = \lim_{t \to c^+} \int_t^b f(x)\,dx.
\] इति

\subsubsection{अभिसरण एवं
विचलन}\label{ux905ux92dux938ux930ux923-ux90fux935-ux935ux91aux932ux928}

\begin{itemize}
\tightlist
\item
  यदि सीमा अस्ति परिमितं च तर्हि अनुचितं अभिन्नं अभिसरति।
\item
  यदि सीमा नास्ति अथवा अनन्तं भवति तर्हि अनुचितं अभिन्नं विचलति।
\end{itemize}

\subsubsection{उदाहरणम्}\label{ux909ux926ux939ux930ux923ux92e-11}

\begin{enumerate}
\def\labelenumi{\arabic{enumi}.}
\tightlist
\item
  घातीयक्षयः
\end{enumerate}

\[
\int_1^\infty \frac{1}{x^2}\,dx = \lim_{b \to \infty} \Big[-\tfrac{1}{x}\Big]_1^b = 1.
\] इति

एतत् अभिसरति ।

\begin{enumerate}
\def\labelenumi{\arabic{enumi}.}
\setcounter{enumi}{1}
\tightlist
\item
  हार्मोनिक कार्य
\end{enumerate}

\[
\int_1^\infty \frac{1}{x}\,dx = \lim_{b \to \infty} \ln b.
\] इति

एतत् अनन्तं यावत् विचलति।

\begin{enumerate}
\def\labelenumi{\arabic{enumi}.}
\setcounter{enumi}{2}
\tightlist
\item
  0 इत्यत्र लक्षणहीनः
\end{enumerate}

\[
\int_0^1 \frac{1}{\sqrt{x}}\,dx = \lim_{t \to 0^+} \int_t^1 x^{-1/2}\,dx.
\] इति

\[
= \lim_{t \to 0^+} [2\sqrt{x}]_t^1 = 2.
\] इति

एतत् अभिसरति ।

\begin{enumerate}
\def\labelenumi{\arabic{enumi}.}
\setcounter{enumi}{3}
\tightlist
\item
  0 (विचलनशील) इत्यत्र लक्षणहीनः .
\end{enumerate}

\[ इति\int_0^1 \frac{1}{x}\,dx = \lim_{t \to 0^+} \ln(1) - \ln(t).
\]

This diverges since \(\ln(t) \to -\infty\).

\subsubsection{Comparison Test for Improper
Integrals}\label{comparison-test-for-improper-integrals}

\begin{itemize}
\tightlist
\item
  If \(0 \leq f(x) \leq g(x)\) for large \(x\), and \(\int g(x)\,dx\)
  converges, then \(\int f(x)\,dx\) also converges.
\item
  If \(\int f(x)\,dx\) diverges and \(f(x) \geq g(x) \geq 0\), then
  \(\int g(x)\,dx\) also diverges.
\end{itemize}

\subsubsection{Why Improper Integrals
Matter}\label{why-improper-integrals-matter}

\begin{itemize}
\tightlist
\item
  They extend integration to infinite domains and unbounded functions.
\item
  They are essential in probability (continuous distributions), physics
  (gravitational/electric fields), and Fourier analysis.
\end{itemize}

\subsubsection{Exercises}\label{exercises-2}

\begin{enumerate}
\def\labelenumi{\arabic{enumi}.}
\tightlist
\item
  Determine whether \(\int_1^\infty \frac{1}{x^p}\,dx\) converges for
  various values of \(p\).
\item
  Evaluate \(\int_0^\infty e^{-x}\,dx\).
\item
  Test convergence of \(\int_0^1 \frac{1}{x^p}\,dx\) depending on \(p\).
\item
  Compute \(\int_{-\infty}^\infty \frac{1}{1+x^2}\,dx\).
\item
  Use the comparison test to show that
  \(\int_1^\infty \frac{1}{x^2+1}\,dx\) converges.
\end{enumerate}

\section{Chapter 6. Applications of
Integration}\label{chapter-6.-applications-of-integration}

\subsection{6.1 Areas and Volumes}\label{areas-and-volumes}

One of the most important applications of integration is finding areas
under curves and volumes of solids.

\subsubsection{Area Between Curves}\label{area-between-curves}

If \(f(x) \geq g(x)\) on \([a, b]\), then the area between the curves
\(y=f(x)\) and \(y=g(x)\) is

\[ इति
A = \int_a^b \बृहत्(f(x) - g(x)\बृहत्)\,dx.
\]

Example: Find the area between \(y=x^2\) and \(y=x\) on \([0,1]\).

\[ इति
A = \int_0^1 (x - x^2)\,dx = \वाम[\tfrac{1}{2}x^2 - \tfrac{1}{3}x^3\दक्षिण]_0^1 = \tfrac{1}{6}।
\]

\subsubsection{Volumes by Slicing}\label{volumes-by-slicing}

If a solid has cross-sectional area \(A(x)\) at position \(x\), then the
volume is

\[
V = \int_a^b A(x)\,dx.
\] इति

\subsubsection{क्रान्ति के
खण्ड}\label{ux915ux930ux928ux924-ux915-ux916ux923ux921}

यदा कश्चन प्रदेशः अक्षं परितः परिभ्रमति तदा परिणामी ठोसस्य आयतनं एकीकरणेन सह
प्राप्यते ।

\begin{enumerate}
\def\labelenumi{\arabic{enumi}.}
\tightlist
\item
  डिस्कविधिःयदि \(y=f(x)\), \(x\in[a,b]\) इत्यस्य अन्तर्गतः प्रदेशः \(x\)-अक्षस्य
  परितः परिभ्रमति:
\end{enumerate}

\[
V = \pi \int_a^b [f(x)]^2\,dx.
\] इति

\begin{enumerate}
\def\labelenumi{\arabic{enumi}.}
\setcounter{enumi}{1}
\tightlist
\item
  वॉशर विधि यदि \(y=f(x)\) तथा \(y=g(x)\) इत्येतयोः मध्ये प्रदेशः \(x\)-अक्षस्य
  परितः परिभ्रमति:
\end{enumerate}

\[
V = \pi \int_a^b \Big([f(x)]^2 - [g(x)]^2\Big)\,dx.
\] इति

\begin{enumerate}
\def\labelenumi{\arabic{enumi}.}
\setcounter{enumi}{2}
\tightlist
\item
  शंखविधिः यदि \(y=f(x)\) इत्यस्य अन्तर्गतः प्रदेशः \(y\)-अक्षस्य परितः परिभ्रमति:
\end{enumerate}

\[
V = 2\pi \int_a^b x f(x)\,dx.
\] इति

\subsubsection{उदाहरणम्}\label{ux909ux926ux939ux930ux923ux92e-12}

\begin{enumerate}
\def\labelenumi{\arabic{enumi}.}
\tightlist
\item
  डिस्कविधिः \(y=\sqrt{x}\), \(0 \leq x \leq 4\), \(x\)-अक्षस्य परितः घुमाव:
\end{enumerate}

\[
V = \pi \int_0^4 (\sqrt{x})^2\,dx = \pi \int_0^4 x\,dx = \pi \left[\tfrac{1}{2}x^2\right]_0^4 = 8\pi.
\] इति

\begin{enumerate}
\def\labelenumi{\arabic{enumi}.}
\setcounter{enumi}{1}
\tightlist
\item
  वॉशर विधि \(y=\sqrt{x}\) तथा \(y=1\), \(0 \leq x \leq 1\), \(x\)-अक्षस्य
  परितः क्षेत्रं परिभ्रमन्तु:
\end{enumerate}

\[
V = \pi \int_0^1 \big((\sqrt{x})^2 - (1)^2\big)\,dx = \pi \int_0^1 (x-1)\,dx = -\tfrac{\pi}{2}.
\] इति

(आयतनस्य निरपेक्षं मूल्यं गृह्यताम्: \(V = \tfrac{\pi}{2}\))।

\begin{enumerate}
\def\labelenumi{\arabic{enumi}.}
\setcounter{enumi}{2}
\tightlist
\item
  शंखविधिः \(y=x\), \(0 \leq x \leq 1\), \(y\)-अक्षस्य परितः क्षेत्रं परिभ्रमन्तु:
\end{enumerate}

\[
V = 2\pi \int_0^1 x(x)\,dx = 2\pi \int_0^1 x^2\,dx = 2\pi \cdot \tfrac{1}{3} = \tfrac{2\pi}{3}.
\] इति

\subsubsection{एतत् किमर्थं
महत्त्वपूर्णम्}\label{ux90fux924ux924-ux915ux92eux930ux925-ux92eux939ux924ux924ux935ux92aux930ux923ux92e-4}

\begin{itemize}
\tightlist
\item
  ज्यामितिशास्त्रे क्षेत्राणां आयतनानां च गणनायाः सटीकमार्गान् प्रदाति ।
\item
  भौतिकशास्त्रे, अभियांत्रिकीशास्त्रे, संभाव्यतायां च अत्यावश्यकम्।
\item
  एकीकरणेन सह ज्यामितीयचिन्तनस्य परिचयं करोति।
\end{itemize}

\subsubsection{अभ्यास}\label{ux905ux92dux92fux938-20}

\begin{enumerate}
\def\labelenumi{\arabic{enumi}.}
\tightlist
\item
  \([0, \pi/2]\) इत्यत्र \(y=\cos x\) तथा \(y=\sin x\) इत्येतयोः मध्ये क्षेत्रं
  ज्ञातव्यम्।
\item
  \(x\)-अक्षस्य परितः \(y=x^2\), \(0 \leq x \leq 1\), परिभ्रमयित्वा निर्मितस्य
  ठोसस्य आयतनं गणयन्तु।
\item
  \(y\)-अक्षस्य परितः \([0,1]\) इत्यत्र \(y=x\) तथा \(y=\sqrt{x}\) इत्येतयोः
  मध्ये क्षेत्रं परिभ्रमयित्वा निर्मितस्य ठोसस्य आयतनं ज्ञातव्यम्।
\item
  \(x\)-अक्षस्य परितः \(y=\sqrt{1-x^2}\) (एकं अर्धवृत्तं) परिभ्रमयित्वा निर्मितस्य
  ठोसस्य आयतनं गणयितुं वाशर पद्धतेः उपयोगं कुर्वन्तु।
\item
  \(y=x^2+1\) तथा \(y=3x\) इत्येतयोः मध्ये परिवेष्टितं क्षेत्रं ज्ञातव्यम्।
\end{enumerate}

\subsection{6.2 चापदीर्घता तथा पृष्ठक्षेत्रफलवक्राणां दीर्घतां, घूर्णनवक्रैः
उत्पद्यमानानां ठोसद्रव्याणां पृष्ठक्षेत्रं च मापनार्थं अपि एकीकरणस्य उपयोगः कर्तुं शक्यते
।}\label{ux91aux92aux926ux930ux918ux924-ux924ux925-ux92aux937ux920ux915ux937ux924ux930ux92bux932ux935ux915ux930ux923-ux926ux930ux918ux924-ux918ux930ux923ux928ux935ux915ux930-ux909ux924ux92aux926ux92fux92eux928ux928-ux920ux938ux926ux930ux935ux92fux923-ux92aux937ux920ux915ux937ux924ux930-ux91a-ux92eux92aux928ux930ux925-ux905ux92a-ux90fux915ux915ux930ux923ux938ux92f-ux909ux92aux92fux917-ux915ux930ux924-ux936ux915ux92fux924}

\subsubsection{चाप लम्बाई}\label{ux91aux92a-ux932ux92eux92cux908}

\([a,b]\) अन्तरालस्य \(y=f(x)\) इति स्निग्धवक्रस्य कृते वक्रस्य दीर्घता भवति

\[
L = \int_a^b \sqrt{1 + \big(f'(x)\big)^2}\,dx.
\]

एतत् रेखाखण्डैः सह वक्रस्य अनुमानं कृत्वा सीमां गृहीत्वा आगच्छति ।

उदाहरण: \(x=0\) तः \(x=4\) पर्यन्तं \(y=\tfrac{1}{2}x^{3/2}\) इत्यस्य दीर्घतां
ज्ञातव्यम् ।

\begin{itemize}
\tightlist
\item
  व्युत्पन्न: \(f'(x) = \tfrac{3}{4}\sqrt{x}\)।
\item
  सूत्रम् : १.
\end{itemize}

\[
L = \int_0^4 \sqrt{1 + \Big(\tfrac{3}{4}\sqrt{x}\Big)^2}\,dx
= \int_0^4 \sqrt{1 + \tfrac{9}{16}x}\,dx.
\] इति

अस्य अभिन्नस्य मूल्याङ्कनं प्रतिस्थापनस्य उपयोगेन कर्तुं शक्यते ।

\subsubsection{क्रान्ति के पृष्ठीय
क्षेत्रफल}\label{ux915ux930ux928ux924-ux915-ux92aux937ux920ux92f-ux915ux937ux924ux930ux92bux932}

यदि वक्रं \(y=f(x)\), \(a \leq x \leq b\), \(x\)-अक्षस्य परितः परिभ्रमति तर्हि
परिणामी ठोसस्य पृष्ठक्षेत्रं भवति

\[
S = 2\pi \int_a^b f(x)\sqrt{1 + \big(f'(x)\big)^2}\,dx.
\] इति

यदि \(y\)-अक्षस्य परितः परिभ्रमति:

\[
S = 2\pi \int_a^b x \sqrt{1 + \big(f'(x)\big)^2}\,dx.
\] इति

\subsubsection{उदाहरणम्}\label{ux909ux926ux939ux930ux923ux92e-13}

\begin{enumerate}
\def\labelenumi{\arabic{enumi}.}
\tightlist
\item
  रेखायाः चापदीर्घता \(y=x\) कृते \(0 \leq x \leq 3\):
\end{enumerate}

\[
L = \int_0^3 \sqrt{1+(1)^2}\,dx = \int_0^3 \sqrt{2}\,dx = 3\sqrt{2}.
\]

\begin{enumerate}
\def\labelenumi{\arabic{enumi}.}
\setcounter{enumi}{1}
\tightlist
\item
  गोलस्य पृष्ठीयक्षेत्रम् \(y = \sqrt{r^2 - x^2}\), \(-r \leq x \leq r\) गृहीत्वा
  \(x\)-अक्षं परितः परिभ्रमन्तु ।
\end{enumerate}

\[
S = 2\pi \int_{-r}^r \sqrt{r^2 - x^2}\sqrt{1+\left(\frac{-x}{\sqrt{r^2-x^2}}\right)^2}\,dx.
\] इति

सरलीकरणेन \(S = 4\pi r^2\) इति गोलस्य पृष्ठक्षेत्रस्य परिचितं सूत्रं प्राप्यते ।

\subsubsection{एतत् किमर्थं
महत्त्वपूर्णम्}\label{ux90fux924ux924-ux915ux92eux930ux925-ux92eux939ux924ux924ux935ux92aux930ux923ux92e-5}

\begin{itemize}
\tightlist
\item
  चापदीर्घता वक्रमार्गेषु दूरस्य विचारं विस्तारयति।
\item
  क्रान्तिस्य पृष्ठीयक्षेत्रस्य भौतिकशास्त्रे, अभियांत्रिकीशास्त्रे, डिजाइनशास्त्रे च
  अनुप्रयोगाः सन्ति ।
\item
  गणितस्य ज्यामितिस्य च मध्ये सेतुः प्रदाति ।
\end{itemize}

\subsubsection{अभ्यास}\label{ux905ux92dux92fux938-21}

\begin{enumerate}
\def\labelenumi{\arabic{enumi}.}
\tightlist
\item
  \(y=\sqrt{x}\) इत्यस्य चापदीर्घतां \(x=0\) तः \(x=4\) पर्यन्तं ज्ञातव्यम् ।2.
  \(x\)-अक्षस्य परितः \(y=x^2\), \(0 \leq x \leq 1\), परिभ्रमयित्वा प्राप्तस्य
  ठोसस्य पृष्ठक्षेत्रस्य गणनां कुरुत।
\item
  \(y=\ln(\cosh x)\) इत्यस्य चापदीर्घतां \(x=0\) तः \(x=1\) पर्यन्तं ज्ञातव्यम् ।
\item
  दर्शयतु यत् \(0\) तः \(r\) पर्यन्तं \(x\)-अक्षस्य परितः \(y=\sqrt{r^2 - x^2}\)
  परिभ्रमणेन गोलस्य पृष्ठक्षेत्रस्य आधा भागः प्राप्यते ।
\item
  रेखां परिभ्रमयित्वा शङ्कुस्य पृष्ठक्षेत्रस्य सूत्रं व्युत्पादयन्तु।
\end{enumerate}

\subsection{6.3 कार्य तथा
औसत}\label{ux915ux930ux92f-ux924ux925-ux914ux938ux924}

एकीकरणं केवलं ज्यामितिपर्यन्तं सीमितं नास्ति । बलेन कृतं कार्यं, अन्तरालस्य उपरि कार्यस्य
औसतमूल्यं च गणयितुं साहाय्यं करोति ।

\subsubsection{कार्यम्‌}\label{ux915ux930ux92fux92e}

यदि चरबलं \(F(x)\) \(x=a\) तः \(x=b\) यावत् सीधारेखायाः सह वस्तुं चालयति, तर्हि
कुलकार्यं भवति

\[
W = \int_a^b F(x)\,dx.
\] इति

एतत् सूत्रं नित्यबलस्य कृते सरलप्रकरणं \(W = F \cdot d\) सामान्यीकरणं करोति ।

उदाहरणम् १ : वसन्तबलम् (Hooke's Law) २. \(a\) तः \(b\) पर्यन्तं लम्बितस्य वसन्तस्य
कृते, बलेन \(F(x) = kx\):

\[
W = \int_a^b kx\,dx = \tfrac{1}{2}k(b^2-a^2).
\] इति

उदाहरणम् २ : जलं पम्पं करणम् यदि टङ्क्याः जलं बहिः निष्कासितम् अस्ति तर्हि आवश्यकं
कार्यं समानं भवति

\[
W = \int_a^b \text{(weight density)} \times \text{(cross-sectional area)} \times \text{(distance lifted)} \, dx.
\] इति

\subsubsection{कस्यचित् फलनस्य औसतं
मूल्यम्}\label{ux915ux938ux92fux91aux924-ux92bux932ux928ux938ux92f-ux914ux938ux924-ux92eux932ux92fux92e}

\([a,b]\) इत्यत्र निरन्तरफलनस्य \(f(x)\) इत्यस्य औसतमूल्यं भवति

\[
f_{\text{avg}} = \frac{1}{b-a} \int_a^b f(x)\,dx.
\] इति

एषः संख्यासूचिकायाः \hspace{0pt}\hspace{0pt}सरासरीकरणस्य निरन्तरः उपमा अस्ति ।

उदाहरणम् १ : १. \([0,2]\) इत्यत्र \(f(x)=x^2\) इत्यस्य कृते:

\[
f_{\text{avg}} = \tfrac{1}{2-0}\int_0^2 x^2 dx = \tfrac{1}{2}\cdot \tfrac{8}{3} = \tfrac{4}{3}.
\] इति

उदाहरणम् २ : १. यदि कणस्य वेगः \(v(t)\) भवति तर्हि \([a,b]\) इत्यस्य उपरि
औसतवेगः भवति

\[
v_{\text{avg}} = \frac{1}{b-a}\int_a^b v(t)\,dt.
\]

\subsubsection{एतत् किमर्थं
महत्त्वपूर्णम्}\label{ux90fux924ux924-ux915ux92eux930ux925-ux92eux939ux924ux924ux935ux92aux930ux923ux92e-6}

\begin{itemize}
\tightlist
\item
  भौतिकशास्त्रे, अभियांत्रिकीशास्त्रे, ऊर्जागणने च कार्य-अखण्डाः दृश्यन्ते ।- औसतमूल्यं
  भिन्नमात्राणां कृते एकां प्रतिनिधिसङ्ख्यां ददाति ।
\item
  उभयम् अपि गणनां गति-बल-दक्षतायाः वास्तविक-जगतः समस्याभिः सह सम्बध्दयति ।
\end{itemize}

\subsubsection{अभ्यास}\label{ux905ux92dux92fux938-22}

\begin{enumerate}
\def\labelenumi{\arabic{enumi}.}
\tightlist
\item
  यदि \(k=10\) तर्हि वसन्तस्य 2 मीटर् तः 5 मीटर् पर्यन्तं तानयितुं आवश्यकस्य कार्यस्य
  गणनां कुरुत।
\item
  गुरुत्वाकर्षणक्षेत्रे (\(g=9.8 \,\text{m/s}^2\)) 100 किलोग्रामभारस्य वस्तु 5 मी.
  कार्यं अभिन्नरूपेण व्यक्तं कृत्वा मूल्याङ्कनं कुर्वन्तु।
\item
  \([0,\pi]\) इत्यत्र \(f(x)=\sin x\) इत्यस्य औसतं मूल्यं ज्ञातव्यम् ।
\item
  24-घण्टादिने \(T(t)=20+5\cos(\tfrac{\pi t}{12})\) चेत् औसततापमानस्य गणनां
  कुर्वन्तु।
\item
  10 मीटर् गभीरतायाः टङ्की जलेन पूर्णा भवति। सर्वं जलं उपरि पम्पं कर्तुं आवश्यकं कार्यं
  गणयन्तु, जलस्य भारः \(9800 \,\text{N/m}^3\) इति दत्तम्।
\end{enumerate}

\subsection{6.4 संभाव्यता घनत्व एवं निरन्तर
वितरण}\label{ux938ux92dux935ux92fux924-ux918ux928ux924ux935-ux90fux935-ux928ux930ux928ux924ux930-ux935ux924ux930ux923}

संभाव्यतासिद्धान्ते अपि एकीकरणस्य केन्द्रभूमिका भवति, विशेषतः निरन्तरयादृच्छिकचरानाम्
कृते । असततपरिणामानां स्थाने वयं संभाव्यताघनत्वफलनानि (pdfs) इति कार्यैः सह
संभाव्यतानां वर्णनं कुर्मः ।

\subsubsection{संभाव्यता घनत्व
फलन}\label{ux938ux92dux935ux92fux924-ux918ux928ux924ux935-ux92bux932ux928}

एकं संभाव्यताघनत्वकार्यं \(f(x)\) द्वे शर्तौ पूरयितुं भवितुमर्हति:

\begin{enumerate}
\def\labelenumi{\arabic{enumi}.}
\item
  \(f(x) \geq 0\) सर्वेषां \(x\) कृते।
\item
  वक्रस्य अधः कुलक्षेत्रं 1:

  \[
  \int_{-\infty}^\infty f(x)\,dx = 1.
  \] इति
\end{enumerate}

यदि \(X\) pdf \(f(x)\) इत्यनेन सह निरन्तरं यादृच्छिकचरः अस्ति, तर्हि \(X\) \(a\)
तथा \(b\) इत्येतयोः मध्ये अस्ति इति संभावना अस्ति

\[
P(a \leq X \leq b) = \int_a^b f(x)\,dx.
\] इति

\subsubsection{संचयी वितरण
कार्य}\label{ux938ux91aux92f-ux935ux924ux930ux923-ux915ux930ux92f}

सञ्चितवितरणकार्यं (cdf) इति परिभाषितं भवति

\[
F(x) = \int_{-\infty}^x f(t)\,dt.
\]

एतत् संभावनां ददाति यत् यादृच्छिकचरः \(x\) इत्यस्मात् न्यूनः वा समानः वा अस्ति ।

\subsubsection{अपेक्षित मूल्य (मध्यम) 1
.}\label{ux905ux92aux915ux937ux924-ux92eux932ux92f-ux92eux927ux92fux92e-1-.}

निरन्तरस्य यादृच्छिकचरस्य अपेक्षितं मूल्यं भारितसरासरी भवति :

\[
E[X] = \int_{-\infty}^\infty x f(x)\,dx.
\] इति

\subsubsection{उदाहरणम्}\label{ux909ux926ux939ux930ux923ux92e-14}

\begin{enumerate}
\def\labelenumi{\arabic{enumi}.}
\tightlist
\item
  एकरूप वितरण\([a,b]\) इत्यत्र \(f(x) = \tfrac{1}{b-a}\) इत्यस्य कृते:
\end{enumerate}

\begin{itemize}
\item
  अन्तराल \([c,d]\) की संभावना:

  \[
  P(c \leq X \leq d) = \frac{d-c}{b-a}.
  \] इति
\item
  अपेक्षितं मूल्यम्: \(E[X] = \tfrac{a+b}{2}\)।
\end{itemize}

\begin{enumerate}
\def\labelenumi{\arabic{enumi}.}
\setcounter{enumi}{1}
\tightlist
\item
  घातीय वितरण \(f(x) = \lambda e^{-\lambda x}\) कृते \(x \geq 0\):
\end{enumerate}

\begin{itemize}
\tightlist
\item
  \(\int_0^\infty \lambda e^{-\lambda x}\,dx = 1\)।
\item
  अर्थ: \(E[X] = \tfrac{1}{\lambda}\)।
\end{itemize}

\begin{enumerate}
\def\labelenumi{\arabic{enumi}.}
\setcounter{enumi}{2}
\tightlist
\item
  सामान्यवितरणम् घण्टावक्रम् : १.
\end{enumerate}

\[
f(x) = \frac{1}{\sqrt{2\pi\sigma^2}} e^{-\frac{(x-\mu)^2}{2\sigma^2}}.
\] इति

एतत् १ मध्ये एकीकृत्य भवति, परन्तु उन्नत-तकनीकानां आवश्यकता वर्तते ।

\subsubsection{एतत् किमर्थं
महत्त्वपूर्णम्}\label{ux90fux924ux924-ux915ux92eux930ux925-ux92eux939ux924ux924ux935ux92aux930ux923ux92e-7}

\begin{itemize}
\tightlist
\item
  संभाव्यताघनत्वं विज्ञानं, अभियांत्रिकी, सांख्यिकी च अनिश्चिततायाः वर्णनं करोति ।
\item
  अभिन्नाः वक्रानाम् अधः क्षेत्राणि संभाव्यताभिः सह संयोजयन्ति।
\item
  निरन्तरवितरणं परिणामगणनायाः विचारं सामान्यीकृत्य अन्तरालेषु संभावनानां मापनं
  करोति।
\end{itemize}

\subsubsection{अभ्यास}\label{ux905ux92dux92fux938-23}

\begin{enumerate}
\def\labelenumi{\arabic{enumi}.}
\tightlist
\item
  दर्शयतु यत् \([a,b]\) इत्यत्र एकरूपं घनत्वं \(f(x) = \tfrac{1}{b-a}\) 1 - मध्ये
  एकीकृतं भवति ।
\item
  \(\lambda = 2\) इत्यनेन सह घातीयवितरणस्य कृते \(P(0 \leq X \leq 1)\) इति
  गणनां कुर्वन्तु ।
\item
  \([0,1]\) इत्यत्र \(f(x) = 3x^2\) चेत् \(X\) इत्यस्य अपेक्षितं मूल्यं ज्ञातव्यम्।
\item
  सत्यापयन्तु यत् औसत 0 तथा विचरण 1 युक्तस्य सामान्यवितरणस्य कुलसंभावना 1 अस्ति
  (पूर्णप्रमाणस्य आवश्यकता नास्ति, परन्तु किमर्थं धारयति इति व्याख्यातव्यम्)।
\item
  \([0,1]\) इत्यत्र एकरूपवितरणस्य cdf गणयन्तु ।
\end{enumerate}

\section{तृतीय भाग। बहुचर
गणित}\label{ux924ux924ux92f-ux92dux917-ux92cux939ux91aux930-ux917ux923ux924}

\section{अध्याय 7. सदिश फलन एवं
वक्र}\label{ux905ux927ux92fux92f-7.-ux938ux926ux936-ux92bux932ux928-ux90fux935-ux935ux915ux930}

\subsection{7.1 सदिश फलन एवं स्थान
वक्र}\label{ux938ux926ux936-ux92bux932ux928-ux90fux935-ux938ux925ux928-ux935ux915ux930}

बहुचरगणनायां कार्याणि संख्यानां स्थाने सदिशान् निर्गन्तुं शक्नुवन्ति । एते सदिशमूल्यानि
कार्याणि इति उच्यन्ते, ते च अन्तरिक्षे वक्रवर्णनार्थं अत्यावश्यकाः सन्ति ।

\subsubsection{परिभाषा}\label{ux92aux930ux92dux937-7}

सदिशकार्यं रूपस्य कार्यम् अस्ति

\[
\mathbf{r}(t) = \langle x(t), y(t), z(t) \rangle,
\] इति

यत्र \(x(t), y(t), z(t)\) वास्तविक-मूल्यकं कार्याणि सन्ति ।

\begin{itemize}
\tightlist
\item
  निवेशः \(t\) प्रायः पैरामीटर् इति उच्यते ।- आउटपुट् 2D अथवा 3D स्पेस इत्यत्र सदिशः
  भवति ।
\item
  3D इत्यस्मिन् सदिशफलनस्य आलेखः अन्तरिक्षवक्रः अस्ति ।
\end{itemize}

\subsubsection{उदाहरणम्}\label{ux909ux926ux939ux930ux923ux92e-15}

\begin{enumerate}
\def\labelenumi{\arabic{enumi}.}
\tightlist
\item
  रेखा
\end{enumerate}

\[
\mathbf{r}(t) = \langle 1+2t, \; 3-t, \; 4+5t \rangle.
\] इति

एतेन दिशासदिशेन \(\langle 2,-1,5 \rangle\) इत्यनेन सह \((1,3,4)\) इति
बिन्दुद्वारा एकां सीधारेखा वर्णिता अस्ति ।

\begin{enumerate}
\def\labelenumi{\arabic{enumi}.}
\setcounter{enumi}{1}
\tightlist
\item
  विमाने वृत्तं कुरुत
\end{enumerate}

\[
\mathbf{r}(t) = \langle \cos t, \; \sin t, \; 0 \rangle, \quad 0 \leq t < 2\pi.
\] इति

\begin{enumerate}
\def\labelenumi{\arabic{enumi}.}
\setcounter{enumi}{2}
\tightlist
\item
  हेलिक्स
\end{enumerate}

\[
\mathbf{r}(t) = \langle \cos t, \; \sin t, \; t \rangle.
\] इति

एषः \(z\)-अक्षस्य परितः उदयमानः सर्पिलः अस्ति ।

\subsubsection{सीमा एवं
निरन्तरता}\label{ux938ux92e-ux90fux935-ux928ux930ux928ux924ux930ux924}

एकं सदिशफलं \(t=a\) इत्यत्र निरन्तरं भवति यदि प्रत्येकं घटकं \(x(t), y(t), z(t)\)
\(t=a\) इत्यत्र निरन्तरं भवति ।

\[
\lim_{t \to a} \mathbf{r}(t) = \langle \lim_{t \to a} x(t), \; \lim_{t \to a} y(t), \; \lim_{t \to a} z(t) \rangle.
\] इति

\subsubsection{अन्तरिक्ष वक्रों की
ज्यामिति}\label{ux905ux928ux924ux930ux915ux937-ux935ux915ux930-ux915-ux91cux92fux92eux924}

\begin{itemize}
\tightlist
\item
  प्रत्येकं वक्रस्य व्युत्पन्नेन दत्ता स्पर्शरेखा भवति ।
\item
  अन्तरिक्षवक्राः गतिमार्गस्य, कणप्रक्षेपवक्रस्य, ज्यामितीयआकारस्य च प्रतिरूपणं कर्तुं
  शक्नुवन्ति ।
\end{itemize}

\subsubsection{एतत् किमर्थं
महत्त्वपूर्णम्}\label{ux90fux924ux924-ux915ux92eux930ux925-ux92eux939ux924ux924ux935ux92aux930ux923ux92e-8}

सदिशकार्यं बहुचरगणनायाः आधारः भवति, येन व्युत्पन्नस्य अभिन्नस्य च विचारान्
उच्चतरपरिमाणेषु विस्तारयितुं शक्यते । भौतिकशास्त्रे (3D मध्ये गतिः, विद्युत्चुम्बकत्वम्,
द्रवगतिविज्ञानम्) अपि ते स्वाभाविकतया दृश्यन्ते ।

\subsubsection{अभ्यास}\label{ux905ux92dux92fux938-24}

\begin{enumerate}
\def\labelenumi{\arabic{enumi}.}
\tightlist
\item
  \(\langle 3,-2,1 \rangle\) सदिशस्य समानान्तरेण \((0,1,2)\) इत्यस्य माध्यमेन
  रेखायाः कृते सदिशफलनं लिखन्तु ।
\item
  \(\mathbf{r}(t) = \langle 2\cos t, \; 2\sin t, \; 3 \rangle\) द्वारा
  दत्तं वक्रं वर्णयतु।
\item
  \(t=1\) इत्यत्र
  \(\mathbf{r}(t) = \langle e^t, \; \ln t, \; t^2 \rangle\) निरन्तरं भवति
  वा इति निर्धारयन्तु।
\item
  \(\mathbf{r}(t) = \langle \cos t, \; \sin t, \; 2t \rangle\) इति
  हेलिक्सस्य रेखाचित्रं कुरुत।
\item
  \(t=2\) वक्रतायां बिन्दु ज्ञातव्यं यदा \(t=2\)।
\end{enumerate}

\subsection{7.2 सदिश फलन के व्युत्पन्न एवं अभिन्नसदिशकार्यं साधारणकार्यवत् भिन्नं
एकीकृतं च कर्तुं शक्यते - वयं केवलं प्रत्येकं घटके ऑपरेशनं प्रयोजयामः । एतेन वयं गतिः, वेगः,
त्वरणं, उच्चतरपरिमाणेषु सञ्चयस्य च अध्ययनं कर्तुं शक्नुमः
।}\label{ux938ux926ux936-ux92bux932ux928-ux915-ux935ux92fux924ux92aux928ux928-ux90fux935-ux905ux92dux928ux928ux938ux926ux936ux915ux930ux92f-ux938ux927ux930ux923ux915ux930ux92fux935ux924-ux92dux928ux928-ux90fux915ux915ux924-ux91a-ux915ux930ux924-ux936ux915ux92fux924---ux935ux92f-ux915ux935ux932-ux92aux930ux924ux92fux915-ux918ux91fux915-ux911ux92aux930ux936ux928-ux92aux930ux92fux91cux92fux92e-ux90fux924ux928-ux935ux92f-ux917ux924-ux935ux917-ux924ux935ux930ux923-ux909ux91aux91aux924ux930ux92aux930ux92eux923ux937-ux938ux91eux91aux92fux938ux92f-ux91a-ux905ux927ux92fux92fux928-ux915ux930ux924-ux936ux915ux928ux92e}

\subsubsection{एक सदिश फलन के
व्युत्पन्न}\label{ux90fux915-ux938ux926ux936-ux92bux932ux928-ux915-ux935ux92fux924ux92aux928ux928}

यदि

\[
\mathbf{r}(t) = \langle x(t), y(t), z(t) \rangle,
\] इति

तदा

\[
\mathbf{r}'(t) = \langle x'(t), y'(t), z'(t) \rangle.
\] इति

इदं व्युत्पन्नसदिशं \(t\) पैरामीटर् इत्यत्र वक्रं प्रति स्पर्शरेखादिशि सूचयति ।

\begin{itemize}
\tightlist
\item
  वेगः : यदि \(\mathbf{r}(t)\) \(t\) समये कस्यचित् कणस्य स्थितिं ददाति तर्हि
  \(\mathbf{v}(t) = \mathbf{r}'(t)\) तस्य वेगसदिशः अस्ति ।
\item
  वेगः : परिमाणं \(|\mathbf{v}(t)|\) कणस्य वेगः अस्ति ।
\item
  त्वरण: \(\mathbf{a}(t) = \mathbf{v}'(t) = \mathbf{r}''(t)\)।
\end{itemize}

\subsubsection{उदाहरणम्}\label{ux909ux926ux939ux930ux923ux92e-16}

\begin{enumerate}
\def\labelenumi{\arabic{enumi}.}
\tightlist
\item
  हेलिक्स
\end{enumerate}

\[
\mathbf{r}(t) = \langle \cos t, \sin t, t \rangle.
\] इति

\begin{itemize}
\tightlist
\item
  वेगः \(\mathbf{v}(t) = \langle -\sin t, \cos t, 1 \rangle\)।
\item
  गतिः
  \(|\mathbf{v}(t)| = \sqrt{(-\sin t)^2 + (\cos t)^2 + 1^2} = \sqrt{2}\)।
\item
  त्वरण: \(\mathbf{a}(t) = \langle -\cos t, -\sin t, 0 \rangle\)।
\end{itemize}

\begin{enumerate}
\def\labelenumi{\arabic{enumi}.}
\setcounter{enumi}{1}
\tightlist
\item
  प्रक्षेप्यगतिः
\end{enumerate}

\[
\mathbf{r}(t) = \langle v_0 \cos\theta \cdot t, \; v_0 \sin\theta \cdot t - \tfrac{1}{2}gt^2 \rangle.
\] इति

एतेन गुरुत्वाकर्षणस्य अधीनं प्रक्षेप्यस्य परवलयमार्गस्य प्रतिरूपणं भवति ।

\subsubsection{एक सदिश फलन का
अभिन्न}\label{ux90fux915-ux938ux926ux936-ux92bux932ux928-ux915-ux905ux92dux928ux928}

यदि

\[
\mathbf{r}(t) = \langle x(t), y(t), z(t) \rangle,
\] इति

तदा

\[
\int \mathbf{r}(t)\,dt = \left\langle \int x(t)\,dt, \; \int y(t)\,dt, \; \int z(t)\,dt \right\rangle + \mathbf{C},
\] इति

यत्र \(\mathbf{C}\) नित्यसदिशः अस्ति ।

\subsubsection{उदाहरण}\label{ux909ux926ux939ux930ux923}

\[
\mathbf{r}(t) = \langle t, t^2, t^3 \rangle.
\]

\begin{itemize}
\tightlist
\item
  व्युत्पन्न: \(\mathbf{r}'(t) = \langle 1, 2t, 3t^2 \rangle\)।
\item
  अभिन्नः : १.
\end{itemize}

\[
\int \mathbf{r}(t)\,dt = \langle \tfrac{1}{2}t^2, \tfrac{1}{3}t^3, \tfrac{1}{4}t^4 \rangle + \mathbf{C}.
\] इति

\subsubsection{एतत् किमर्थं महत्त्वपूर्णम्- सदिशफलनानां व्युत्पन्नाः अन्तरिक्षे गतिं बलं च
वर्णयन्ति
।}\label{ux90fux924ux924-ux915ux92eux930ux925-ux92eux939ux924ux924ux935ux92aux930ux923ux92e--ux938ux926ux936ux92bux932ux928ux928-ux935ux92fux924ux92aux928ux928-ux905ux928ux924ux930ux915ux937-ux917ux924-ux92cux932-ux91a-ux935ux930ux923ux92fux928ux924}

\begin{itemize}
\tightlist
\item
  अभिन्नं विस्थापनं, कार्यं, सञ्चितमात्राः च ददति।
\item
  एते साधनानि गणितं प्रत्यक्षतया भौतिकशास्त्रेण अभियांत्रिकीशास्त्रेण च सम्बध्दयन्ति ।
\end{itemize}

\subsubsection{अभ्यास}\label{ux905ux92dux92fux938-25}

\begin{enumerate}
\def\labelenumi{\arabic{enumi}.}
\tightlist
\item
  \(\mathbf{r}(t) = \langle t, \cos t, \sin t \rangle\) कृते वेगं, वेगं, त्वरणं
  च ज्ञातव्यम् ।
\item
  \(\mathbf{r}(t) = \langle e^t, \ln t, t^2 \rangle\) कृते
  \(\mathbf{r}'(t)\) गणनां कुर्वन्तु।
\item
  \(\mathbf{r}(t) = \langle 1, t, t^2 \rangle\) एकीकृत करें।
\item
  कणस्य वेगः \(\mathbf{v}(t) = \langle t, 2, 0 \rangle\) भवति । यदि
  \(\mathbf{r}(0) = \langle 1, 0, 0 \rangle\) तर्हि तस्य स्थितिसदिशं ज्ञातव्यम्
  ।
\item
  \(\mathbf{r}(t) = \langle \cos t, \sin t, 0 \rangle\) इत्यस्य वेगः नित्यः
  इति दर्शयतु।
\end{enumerate}

\subsection{7.3 चापदीर्घता वक्रता
च}\label{ux91aux92aux926ux930ux918ux924-ux935ux915ux930ux924-ux91a}

सदिशगणना न केवलं वक्रेण अनुसृतं मार्गं अपितु कियत् तीक्ष्णतया नमति इति मापनार्थं
साधनानि प्रदाति । एते चापदीर्घतायाः वक्रतायाः च माध्यमेन व्यक्ताः भवन्ति ।

\subsubsection{एक स्पेस वक्र की चाप
लम्बाई}\label{ux90fux915-ux938ux92aux938-ux935ux915ux930-ux915-ux91aux92a-ux932ux92eux92cux908}

यदि वक्रं दीयते

\[
\mathbf{r}(t) = \langle x(t), y(t), z(t) \rangle, \quad a \leq t \leq b,
\] इति

तदा चापदीर्घता भवति

\[
L = \int_a^b |\mathbf{r}'(t)|\,dt,
\] इति

कुत्र

\[
|\mathbf{r}'(t)| = \sqrt{(x'(t))^2 + (y'(t))^2 + (z'(t))^2}.
\] इति

उदाहरण:
\(\mathbf{r}(t) = \langle \cos t, \sin t, t \rangle, \, 0 \leq t \leq 2\pi\)
इति हेलिक्सस्य कृते :

\begin{itemize}
\tightlist
\item
  वेगः \(\mathbf{r}'(t) = \langle -\sin t, \cos t, 1 \rangle\)।
\item
  गतिः
  \(|\mathbf{r}'(t)| = \sqrt{(-\sin t)^2 + (\cos t)^2 + 1^2} = \sqrt{2}\)।
\item
  चाप लम्बाई : १.
\end{itemize}

\[
L = \int_0^{2\pi} \sqrt{2}\,dt = 2\pi\sqrt{2}.
\] इति

\subsubsection{वक्रता}\label{ux935ux915ux930ux924}

वक्रतायाः मापनं भवति यत् वक्रः कियत् शीघ्रं दिशां परिवर्तयति ।

स्निग्धवक्रस्य कृते \(\mathbf{r}(t)\):

\[
\kappa(t) = \frac{|\mathbf{r}'(t) \times \mathbf{r}''(t)|}{|\mathbf{r}'(t)|^3}.
\] इति

\begin{itemize}
\tightlist
\item
  \(\kappa = 0\): सीधी रेखा।
\item
  बृहत्तरः \(\kappa\): वक्रः अधिकं तीक्ष्णतया झुकति।
\end{itemize}

उदाहरण: \(r\) त्रिज्यावृत्तस्य कृते : १.\[
\mathbf{r}(t) = \langle r\cos t, r\sin t \rangle.
\] इति

अथ \(\kappa = \tfrac{1}{r}\)। अतः वक्रता नित्यं त्रिज्यायाः विलोमानुपातिकं च
भवति।

\subsubsection{इकाई स्पर्शरेखा एवं सामान्य
सदिश}\label{ux907ux915ux908-ux938ux92aux930ux936ux930ux916-ux90fux935-ux938ux92eux928ux92f-ux938ux926ux936}

\begin{itemize}
\tightlist
\item
  स्पर्शरेखा सदिश: .
\end{itemize}

\[
\mathbf{T}(t) = \frac{\mathbf{r}'(t)}{|\mathbf{r}'(t)|}.
\]

\begin{itemize}
\tightlist
\item
  सामान्य सदिशः वक्रतायाः केन्द्रं प्रति सूचयति, यथा परिभाषितः
\end{itemize}

\[
\mathbf{N}(t) = \frac{\mathbf{T}'(t)}{|\mathbf{T}'(t)|}.
\] इति

एते सदिशाः गतिज्यामितिं वर्णयन्ति : यात्रायाः दिशा, भ्रमणस्य दिशा च ।

\subsubsection{एतत् किमर्थं
महत्त्वपूर्णम्}\label{ux90fux924ux924-ux915ux92eux930ux925-ux92eux939ux924ux924ux935ux92aux930ux923ux92e-9}

\begin{itemize}
\tightlist
\item
  चापदीर्घता अन्तरिक्षे वक्राणां दूरतायाः अवधारणां सामान्यीकरोति ।
\item
  वक्रता भौतिकशास्त्रे (केन्द्रीयत्वरणं), अभियांत्रिकी (मार्गाः, रोलरकोस्टराः),
  सङ्गणकचित्रकलायां च महत्त्वपूर्णं मोचनस्य वर्णनं करोति ।
\end{itemize}

\subsubsection{अभ्यास}\label{ux905ux92dux92fux938-26}

\begin{enumerate}
\def\labelenumi{\arabic{enumi}.}
\tightlist
\item
  \(t=0\) तः \(t=1\) पर्यन्तं \(\mathbf{r}(t) = \langle t, t^2, 0 \rangle\)
  इत्यस्य चापदीर्घतां ज्ञातव्यम् ।
\item
  \(\mathbf{r}(t) = \langle \cos t, \sin t \rangle\) वृत्तस्य वक्रता गणयतु।
\item
  \(\mathbf{r}(t) = \langle t, \cos t, \sin t \rangle\) कृते
  \(|\mathbf{r}'(t)|\) इति गणनां कुर्वन्तु।
\item
  दर्शयतु यत् एकस्याः ऋजुरेखायाः वक्रता \(\kappa = 0\) अस्ति।
\item
  \(t=0\) इत्यत्र \(\mathbf{r}(t) = \langle e^t, e^{-t}, t \rangle\) इत्यस्य
  स्पर्शरेखा सदिशं ज्ञातव्यम् ।
\end{enumerate}

\subsection{7.4 अन्तरिक्षे
गतिः}\label{ux905ux928ux924ux930ux915ux937-ux917ux924}

द्वित्रिमात्रायां गतिवर्णने सदिशकार्यं विशेषतया शक्तिशाली भवति । सदिशमूल्यकफलनस्य
व्युत्पन्नस्य अभिन्नस्य च उपयोगेन स्थितिः, वेगः, त्वरणं च स्वाभाविकतया व्यक्तं भवति ।

\subsubsection{स्थिति, वेग, त्वरण
च}\label{ux938ux925ux924-ux935ux917-ux924ux935ux930ux923-ux91a}

\begin{itemize}
\tightlist
\item
  स्थिति सदिश: .
\end{itemize}

\[
\mathbf{r}(t) = \langle x(t), y(t), z(t) \rangle
\] इति

\begin{itemize}
\tightlist
\item
  वेग सदिश (स्थिति का व्युत्पन्न): .
\end{itemize}

\[
\mathbf{v}(t) = \mathbf{r}'(t) = \langle x'(t), y'(t), z'(t) \rangle
\] इति

\begin{itemize}
\tightlist
\item
  गति (वेगस्य परिमाणम्): १.
\end{itemize}

\[
|\mathbf{v}(t)| = \sqrt{(x'(t))^2 + (y'(t))^2 + (z'(t))^2}
\] इति

\begin{itemize}
\tightlist
\item
  त्वरण सदिश (वेग का व्युत्पन्न): .
\end{itemize}

\[\mathbf{a}(t) = \mathbf{v}'(t) = \mathbf{r}''(t) ।
\]

\subsubsection{Tangential and Normal
Components}\label{tangential-and-normal-components}

Acceleration can be decomposed into two components:

\[ इति
\mathbf{a}(t) = a_T \mathbf{T}(t) + a_N \mathbf{N}(t),
\]

where:

\begin{itemize}
\tightlist
\item
  \(\mathbf{T}(t)\) = unit tangent vector,
\item
  \(\mathbf{N}(t)\) = principal normal vector,
\item
  \(a_T = \frac{d}{dt}|\mathbf{v}(t)|\) = tangential acceleration
  (change in speed),
\item
  \(a_N = \kappa |\mathbf{v}(t)|^2\) = normal acceleration (change in
  direction).
\end{itemize}

\subsubsection{Projectile Motion in 3D}\label{projectile-motion-in-3d}

With gravity acting in the \(-z\) direction:

\[ इति
\mathbf{r}(t) = \langle v_0 \cos\थीटा \cos\phi \cdot टी,\; v_0 \cos\theta \sin\phi \cdot t,\; v_0 \सिन\थेता \cdot t - \tfrac{1}{2}gt^2 \rangle,
\]

where \(v_0\) is initial speed, \(\theta\) launch angle, and \(\phi\)
azimuthal direction.

\subsubsection{Example: Helical Motion}\label{example-helical-motion}

\[
\mathbf{r}(t) = \langle \cos t, \sin t, त \angle
\] इति

\begin{itemize}
\tightlist
\item
  वेगः \(\mathbf{v}(t) = \langle -\sin t, \cos t, 1 \rangle\)।
\item
  गतिः \(|\mathbf{v}(t)| = \sqrt{2}\)।
\item
  त्वरण: \(\mathbf{a}(t) = \langle -\cos t, -\sin t, 0 \rangle\)।
\item
  गतिः वेगेन एकरूपा भवति, ऊर्ध्वं सर्पिलरूपेण गच्छति।
\end{itemize}

\subsubsection{एतत् किमर्थं
महत्त्वपूर्णम्}\label{ux90fux924ux924-ux915ux92eux930ux925-ux92eux939ux924ux924ux935ux92aux930ux923ux92e-10}

\begin{itemize}
\tightlist
\item
  वास्तविक-जगत्-गति-कृते गणितीय-भाषां प्रदाति ।
\item
  भौतिकशास्त्रे (बलाः, प्रक्षेपवक्राः, वृत्तगतिः) आवश्यकाः।
\item
  उन्नतयान्त्रिकस्य अभियांत्रिकीप्रतिमानस्य च आधारः।
\end{itemize}

\subsubsection{अभ्यास}\label{ux905ux92dux92fux938-27}

\begin{enumerate}
\def\labelenumi{\arabic{enumi}.}
\tightlist
\item
  कणः \(\mathbf{r}(t) = \langle t, t^2, t^3 \rangle\) इत्यनेन सह गच्छति।
  \(t=1\) इत्यत्र वेगं त्वरणं च ज्ञातव्यम् ।
\item
  \(\mathbf{r}(t) = \langle \cos t, \sin t, t \rangle\) हेलिक्सस्य कृते वेगः
  नित्यः इति दर्शयतु।
\item
  \(45^\circ\) कोणे \(v_0 = 20 \,\text{m/s}\) इत्यनेन सह प्रक्षेप्यः प्रक्षेप्यते ।
  ऊर्ध्वाधरविमाने गतिं गृहीत्वा तस्य स्थितिसदिशं लिखत।
\item
  \(\mathbf{r}(t) = \langle e^t, e^{-t}, t \rangle\) कृते
  \(\mathbf{v}(t)\) तथा \(\mathbf{a}(t)\) इति ज्ञातव्यम् ।5. त्वरणसदिशं
  त्रिज्या \(r\) वृत्ते गतिं कृते स्पर्शरेखा-सामान्यघटकयोः विघटनं कुर्वन्तु।
\end{enumerate}

\section{अध्याय 8. अनेक चर के
कार्य}\label{ux905ux927ux92fux92f-8.-ux905ux928ux915-ux91aux930-ux915-ux915ux930ux92f}

\subsection{8.1 अनेकचरयोः सीमाः निरन्तरता
च}\label{ux905ux928ux915ux91aux930ux92f-ux938ux92e-ux928ux930ux928ux924ux930ux924-ux91a}

बहुचरगणने कार्याणि द्वयोः वा अधिकयोः चरयोः उपरि निर्भरं भवितुम् अर्हन्ति, यथा
\(f(x,y)\) अथवा \(f(x,y,z)\) । सीमानां निरन्तरतायाश्च अवधारणाः एकचरगणनातः
स्वाभाविकतया विस्तृताः सन्ति, परन्तु ते सूक्ष्मतराः सन्ति यतोहि अस्माभिः सर्वेषां
सम्भाव्यमार्गाणां विचारः करणीयः

\subsubsection{द्वयोः चरयोः
सीमाः}\label{ux926ux935ux92f-ux91aux930ux92f-ux938ux92e}

\(f(x,y)\) इति फंक्शन् कृते वयं वदामः

\[
\lim_{(x,y) \to (a,b)} f(x,y) = L
\] इति

यदि \(f(x,y)\) मनमाना \(L\) इत्यस्य समीपं गच्छति यतः \((x,y)\) कस्मिन् अपि
मार्गे \((a,b)\) इत्यस्य समीपं गच्छति ।

यदि भिन्नाः मार्गाः भिन्नानि सीमामूल्यानि ददति तर्हि सीमा नास्ति ।

उदाहरणम् १ (सीमा अस्ति) : १.

\[
f(x,y) = x^2 + y^2, \quad \lim_{(x,y) \to (0,0)} f(x,y) = 0.
\] इति

उदाहरणम् २ (सीमा नास्ति) : १.

\[
f(x,y) = \frac{xy}{x^2+y^2}, \quad (x,y) \to (0,0).
\] इति

\begin{itemize}
\tightlist
\item
  \(y=0\) इत्यनेन सह कार्यं 0 अस्ति ।
\item
  \(y=x\) इत्यनेन सह कार्यं \(\tfrac{1}{2}\) अस्ति । भिन्नफलं → सीमा नास्ति।
\end{itemize}

\subsubsection{निरन्तरता}\label{ux928ux930ux928ux924ux930ux924-1}

एकं फंक्शन् \(f(x,y)\) \((a,b)\) इत्यत्र निरन्तरं भवति यदि

\[
\lim_{(x,y)\to(a,b)} f(x,y) = f(a,b).
\]

बहुपदं परिमेयफलं च (यत्र हरः ≠ 0) स्वक्षेत्रेषु सर्वत्र निरन्तरम् अस्ति ।

\subsubsection{त्रयः वा अधिकानि चराः यावत्
विस्तारः}\label{ux924ux930ux92f-ux935-ux905ux927ux915ux928-ux91aux930-ux92fux935ux924-ux935ux938ux924ux930}

\(f(x,y,z)\) कृते सीमाः निरन्तरता च समानरूपेण परिभाषिताः भवन्ति, परन्तु
\((a,b,c)\) इति बिन्दुः अन्तरिक्षे अनन्तदिशाभ्यः अवश्यमेव उपसर्गः करणीयः

\subsubsection{एतत् किमर्थं
महत्त्वपूर्णम्}\label{ux90fux924ux924-ux915ux92eux930ux925-ux92eux939ux924ux924ux935ux92aux930ux923ux92e-11}

\begin{itemize}
\tightlist
\item
  निरन्तरता बहुचरकार्येषु कूर्दनं, छिद्रं, लक्षणं वा न भवति इति सुनिश्चितं करोति ।
\item
  आंशिकव्युत्पन्नं बहु अभिन्नं च परिभाषितुं सीमाः मौलिकाः सन्ति ।
\item
  एताः अवधारणाः बहुचरगणनायाः निर्माणखण्डाः सन्ति ।
\end{itemize}

\subsubsection{\texorpdfstring{अभ्यास1.
\(\lim_{(x,y)\to(0,0)} (x^2+y^2)\) अस्ति वा इति
निर्धारयतु।}{अभ्यास1. \textbackslash lim\_\{(x,y)\textbackslash to(0,0)\} (x\^{}2+y\^{}2) अस्ति वा इति निर्धारयतु।}}\label{ux905ux92dux92fux9381.-lim_xyto00-x2y2-ux905ux938ux924-ux935-ux907ux924-ux928ux930ux927ux930ux92fux924}

\begin{enumerate}
\def\labelenumi{\arabic{enumi}.}
\setcounter{enumi}{1}
\tightlist
\item
  दर्शयतु यत् \(\lim_{(x,y)\to(0,0)} \frac{x^2y}{x^2+y^2} = 0\) सर्वेषु
  ऋजुरेखामार्गेषु \(y=mx\)।
\item
  \(f(x,y) = \frac{x^2-y^2}{x^2+y^2}\) इत्यस्य सीमा \((x,y)\to(0,0)\) इति
  रूपेण अस्ति वा?
\item
  द्वयोः चरयोः बहुपदाः सर्वत्र किमर्थं निरन्तराः इति व्याख्यातव्यम्।
\item
  द्वयोः चरयोः कार्यस्य उदाहरणं ददातु यत् एकस्मिन् बिन्दौ विच्छिन्नं भवति, तस्य कारणं
  च व्याख्यातव्यम् ।
\end{enumerate}

\subsection{8.2 आंशिक
व्युत्पन्न}\label{ux906ux936ux915-ux935ux92fux924ux92aux928ux928}

अनेकचरानाम् फंक्शन्स् मध्ये वयं प्रायः मापयितुम् इच्छामः यत् यदा केवलं एकः चरः परिवर्तते
अन्ये तु नित्यं धारयन्ति तदा फंक्शन् कथं परिवर्तते । अनेन आंशिकव्युत्पन्नस्य विचारः भवति ।

\subsubsection{परिभाषा}\label{ux92aux930ux92dux937-8}

\(f(x,y)\) इति फंक्शन् कृते \((a,b)\) इति बिन्दौ \(x\) इत्यस्य विषये आंशिकव्युत्पन्नं
भवति

\[
\frac{\partial f}{\partial x}(a,b) = \lim_{h \to 0} \frac{f(a+h, b) - f(a,b)}{h}.
\] इति

तथा \(y\) इत्यस्य विषये आंशिकव्युत्पन्नं भवति

\[
\frac{\partial f}{\partial y}(a,b) = \lim_{h \to 0} \frac{f(a, b+h) - f(a,b)}{h}.
\] इति

अन्येषां सर्वेषां चरानाम् वयं भेदं कुर्वन् नित्यं व्यवहरामः ।

\subsubsection{संकेतन}\label{ux938ux915ux924ux928-1}

\begin{itemize}
\tightlist
\item
  \(\frac{\partial f}{\partial x}\), \(f_x\), \(\partial_x f\)।
\item
  \(\frac{\partial f}{\partial y}\), \(f_y\), \(\partial_y f\)।
\end{itemize}

त्रयाणां चरानाम् \(f(x,y,z)\) कृते अस्माकं \(f_x, f_y, f_z\) अपि अस्ति ।

\subsubsection{उदाहरणम्}\label{ux909ux926ux939ux930ux923ux92e-17}

\begin{enumerate}
\def\labelenumi{\arabic{enumi}.}
\tightlist
\item
  \(f(x,y) = x^2y + y^3\) इति
\end{enumerate}

\begin{itemize}
\tightlist
\item
  \(f_x = 2xy\)।
\item
  \(f_y = x^2 + 3y^2\)।
\end{itemize}

\begin{enumerate}
\def\labelenumi{\arabic{enumi}.}
\setcounter{enumi}{1}
\tightlist
\item
  \(f(x,y) = e^{xy}\) इति
\end{enumerate}

\begin{itemize}
\tightlist
\item
  \(f_x = y e^{xy}\)।
\item
  \(f_y = x e^{xy}\)।
\end{itemize}

\begin{enumerate}
\def\labelenumi{\arabic{enumi}.}
\setcounter{enumi}{2}
\tightlist
\item
  \(f(x,y,z) = x^2 + yz\) इति
\end{enumerate}

\begin{itemize}
\tightlist
\item
  \(f_x = 2x\)।
\item
  \(f_y = z\)।
\item
  \(f_z = y\)।
\end{itemize}

\subsubsection{उच्च-क्रम आंशिक
व्युत्पन्न}\label{ux909ux91aux91a-ux915ux930ux92e-ux906ux936ux915-ux935ux92fux924ux92aux928ux928}

वयं आंशिकव्युत्पन्नं पुनः पुनः ग्रहीतुं शक्नुमः : १.

\begin{itemize}
\tightlist
\item
  \(f_{xx} = \frac{\partial}{\partial x}\Big(f_x\Big)\)।
\item
  \(f_{yy}, f_{xy}, f_{yx}\) इत्यादि।
\end{itemize}

Clairaut's Theorem: यदि \(f\) इत्यस्य निरन्तरद्वितीय आंशिकव्युत्पन्नाः सन्ति तर्हि

\[
f_{xy} = f_{yx}.
\] इति

\subsubsection{\texorpdfstring{ज्यामितीय अर्थ- \(f_x\): \(x\)-दिशि पृष्ठस्य
प्रवणता।}{ज्यामितीय अर्थ- f\_x: x-दिशि पृष्ठस्य प्रवणता।}}\label{ux91cux92fux92eux924ux92f-ux905ux930ux925--f_x-x-ux926ux936-ux92aux937ux920ux938ux92f-ux92aux930ux935ux923ux924}

\begin{itemize}
\tightlist
\item
  \(f_y\): \(y\)-दिशायां पृष्ठस्य प्रवणता।
\item
  ते मिलित्वा वर्णयन्ति यत् पृष्ठभागः कथं झुकति।
\end{itemize}

\subsubsection{एतत् किमर्थं
महत्त्वपूर्णम्}\label{ux90fux924ux924-ux915ux92eux930ux925-ux92eux939ux924ux924ux935ux92aux930ux923ux92e-12}

\begin{itemize}
\tightlist
\item
  आंशिकव्युत्पन्नाः बहुचरयोः ढालस्य, स्पर्शरेखाविमानस्य, अनुकूलनस्य च आधारः भवन्ति ।
\item
  भौतिकशास्त्रे, अभियांत्रिकीशास्त्रे, अर्थशास्त्रे च बहुभिः निवेशैः सह प्रणालीनां
  प्रतिरूपणार्थं तेषां व्यापकरूपेण उपयोगः भवति ।
\end{itemize}

\subsubsection{अभ्यास}\label{ux905ux92dux92fux938-28}

\begin{enumerate}
\def\labelenumi{\arabic{enumi}.}
\tightlist
\item
  \(f(x,y) = x^3y^2\) कृते \(f_x\) तथा \(f_y\) ज्ञातव्यम्।
\item
  \(f(x,y,z) = xyz + x^2\) कृते \(f_x, f_y, f_z\) गणनां कुर्वन्तु।
\item
  \(f(x,y) = x^2y + y^3\) कृते क्लेरौट् इत्यस्य प्रमेयस्य सत्यापनम्।
\item
  \(f(x,y) = \sqrt{x^2+y^2}\) कृते \(f_x\) तथा \(f_y\) इत्येतयोः अर्थः किम्
  इति ज्यामितीयरूपेण व्याख्यातव्यम्।
\item
  \(f(x,y) = e^{x^2+y^2}\) इत्यस्य सर्वे द्वितीयक्रमस्य आंशिकव्युत्पन्नाः ज्ञातव्याः।
\end{enumerate}

\subsection{8.3 ढाल एवं दिशात्मक
व्युत्पन्न}\label{ux922ux932-ux90fux935-ux926ux936ux924ux92eux915-ux935ux92fux924ux92aux928ux928}

आंशिकव्युत्पन्नाः निर्देशांक-अक्षैः सह परिवर्तनं मापयन्ति, परन्तु कदाचित् वयं कस्यापि दिशि
कस्यापि कार्यस्य परिवर्तनस्य दरं ज्ञातुम् इच्छामः । अनेन ढालस्य, दिग्व्युत्पन्नस्य च
अवधारणाः भवन्ति ।

\subsubsection{ढाल सदिश}\label{ux922ux932-ux938ux926ux936}

\(f(x,y)\) इति फंक्शन् कृते ग्रेडिएण्ट् सदिशः अस्ति

\[
\nabla f(x,y) = \left\langle \frac{\partial f}{\partial x}, \frac{\partial f}{\partial y} \right\rangle.
\] इति

त्रयाणां चरानाम् कृते \(f(x,y,z)\):

\[
\nabla f(x,y,z) = \left\langle f_x, f_y, f_z \right\rangle.
\] इति

ढालः कार्यस्य अधिकतमवृद्धेः दिशि सूचयति, तस्य परिमाणं च तीव्रतमं प्रवणं ददाति ।

\subsubsection{दिशात्मक
व्युत्पन्न}\label{ux926ux936ux924ux92eux915-ux935ux92fux924ux92aux928ux928}

एककसदिशस्य \(\mathbf{u} = \langle u_1, u_2 \rangle\) इत्यस्य दिशि एकस्मिन्
बिन्दौ \(f(x,y)\) इत्यस्य परिवर्तनस्य दरः अस्ति

\[
D_{\mathbf{u}} f(x,y) = \nabla f(x,y) \cdot \mathbf{u}.
\] इति

एषः दिशासदिशयुक्तस्य ढालस्य बिन्दुगुणः ।

\subsubsection{उदाहरणम्}\label{ux909ux926ux939ux930ux923ux92e-18}

\begin{enumerate}
\def\labelenumi{\arabic{enumi}.}
\tightlist
\item
  \(f(x,y) = x^2 + y^2\) इति
\end{enumerate}

\begin{itemize}
\tightlist
\item
  ढाल: \(\nabla f = \langle 2x, 2y \rangle\)।
\item
  at (1,2): \(\nabla f = \langle 2,4 \rangle\)।-
  \(\mathbf{u} = \langle \tfrac{3}{5}, \tfrac{4}{5} \rangle\) के साथ
  दिशात्मक व्युत्पन्न:
\end{itemize}

\[
D_{\mathbf{u}} f(1,2) = \langle 2,4 \rangle \cdot \langle \tfrac{3}{5}, \tfrac{4}{5} \rangle = \tfrac{26}{5}.
\] इति

\begin{enumerate}
\def\labelenumi{\arabic{enumi}.}
\setcounter{enumi}{1}
\tightlist
\item
  \(f(x,y,z) = x y z\) इति
\end{enumerate}

\begin{itemize}
\tightlist
\item
  ढाल: \(\nabla f = \langle yz, xz, xy \rangle\)।
\item
  at (1,1,1): \(\nabla f = \langle 1,1,1 \rangle\)।
\item
  अधिकतमं वृद्धिदिशा \(\langle 1,1,1 \rangle\) इत्यनेन सह भवति ।
\end{itemize}

\subsubsection{ज्यामितीय
व्याख्या}\label{ux91cux92fux92eux924ux92f-ux935ux92fux916ux92f-1}

\begin{itemize}
\tightlist
\item
  ढालसदिशः \(f\) इत्यस्य समतलवक्रयोः अथवा समतलपृष्ठयोः लम्बवत् (सामान्यः) भवति ।
\item
  दिशात्मकव्युत्पन्नाः मनमानादिशेषु प्रवणतां सामान्ययन्ति।
\end{itemize}

\subsubsection{एतत् किमर्थं
महत्त्वपूर्णम्}\label{ux90fux924ux924-ux915ux92eux930ux925-ux92eux939ux924ux924ux935ux92aux930ux923ux92e-13}

\begin{itemize}
\tightlist
\item
  अनुकूलने ढालः अस्मान् तीव्रतम-आरोहणाय वा अवरोहाय वा गन्तुं दिशां वदति ।
\item
  भौतिकशास्त्रे ढालाः तापप्रवाहः, विद्युत्विभवः इत्यादीनां क्षेत्राणां वर्णनं कुर्वन्ति ।
\item
  दिशात्मकव्युत्पन्नाः परिवर्तनस्य एकचर-बहुचर-दरं एकीकृतयन्ति ।
\end{itemize}

\subsubsection{अभ्यास}\label{ux905ux92dux92fux938-29}

\begin{enumerate}
\def\labelenumi{\arabic{enumi}.}
\tightlist
\item
  \(f(x,y) = e^{xy}\) कृते \(\nabla f(x,y)\) गणनां कुर्वन्तु।
\item
  \(f(x,y,z) = x^2+y^2+z^2\) इत्यस्य ढालं ज्ञात्वा (1,1,1) इत्यत्र मूल्याङ्कनं
  कुर्वन्तु ।
\item
  \(\mathbf{u} = \langle 0,1 \rangle\) इत्यस्य दिशि (2,1) इत्यत्र
  \(f(x,y) = x^2-y\) इत्यस्य दिशात्मकव्युत्पन्नस्य गणनां कुर्वन्तु।
\item
  \(f(x,y) = x^2+y^2\) इत्यस्य ढालः \(x^2+y^2=1\) वृत्तस्य लम्बः इति दर्शयतु ।
\item
  (1,2) इत्यत्र \(f(x,y) = xy\) इत्यस्य दिशात्मकव्युत्पन्नं अधिकतमं करोति इति
  एककसदिशदिशा ज्ञातव्यम् ।
\end{enumerate}

\subsection{8.4 स्पर्शरेखा विमान एवं रेखीय
सन्निकर्ष}\label{ux938ux92aux930ux936ux930ux916-ux935ux92eux928-ux90fux935-ux930ux916ux92f-ux938ux928ux928ux915ux930ux937}

एकचरगणने स्पर्शरेखा बिन्दुसमीपे वक्रस्य अनुमानं करोति । बहुचरगणनायां अनुरूपसंकल्पना
स्पर्शरेखाविमानं भवति, यत् बिन्दुसमीपे पृष्ठस्य रेखीयसन्निकर्षं प्रदाति ।

\subsubsection{एक पृष्ठ के स्पर्शरेखा
विमान}\label{ux90fux915-ux92aux937ux920-ux915-ux938ux92aux930ux936ux930ux916-ux935ux92eux928}

मानातु \(z = f(x,y)\) \((a,b)\) इत्यत्र भेद्यम् अस्ति । \((a,b,f(a,b))\) इत्यत्र
स्पर्शरेखाविमानं द्वारा दीयते

\[
z = f(a,b) + f_x(a,b)(x-a) + f_y(a,b)(y-b).
\] इतिएतत् विमानं बिन्दौ पृष्ठभागं स्पृशति, समीपे च तस्य सन्निकर्षं करोति ।

\subsubsection{उदाहरणम् 1:
पराबोलोइड}\label{ux909ux926ux939ux930ux923ux92e-1-ux92aux930ux92cux932ux907ux921}

\((1,2)\) इत्यत्र \(f(x,y) = x^2 + y^2\) इत्यस्य कृते:

\begin{itemize}
\tightlist
\item
  \(f(1,2) = 1^2+2^2=5\)।
\item
  \(f_x = 2x\), अतः \(f_x(1,2) = 2\)।
\item
  \(f_y = 2y\), अतः \(f_y(1,2) = 4\)।
\end{itemize}

स्पर्शरेखा विमानस्य समीकरणम् : १.

\[
z = 5 + 2(x-1) + 4(y-2).
\] इति

\subsubsection{रेखीय
सन्निकर्ष}\label{ux930ux916ux92f-ux938ux928ux928ux915ux930ux937}

स्पर्शरेखाविमानस्य उपयोगेन \((a,b)\) इत्यस्य समीपे \(f(x,y)\) इत्यस्य अनुमानं कर्तुं
शक्यते:

\[
f(x,y) \approx f(a,b) + f_x(a,b)(x-a) + f_y(a,b)(y-b).
\] इति

इदं \((a,b)\) इत्यत्र \(f\) इत्यस्य रेखीयकरणम् अस्ति ।

\subsubsection{उदाहरणम् २ : रेखीय
सन्निकर्षः}\label{ux909ux926ux939ux930ux923ux92e-ux968-ux930ux916ux92f-ux938ux928ux928ux915ux930ux937}

\((4,5)\) के पास अनुमानित \(f(x,y) = \sqrt{x+y}\)।

\begin{itemize}
\tightlist
\item
  \(f(4,5) = \sqrt{9} = 3\)।
\item
  \(f_x = \frac{1}{2\sqrt{x+y}}, \quad f_y = \frac{1}{2\sqrt{x+y}}\)।
\item
  at (4,5): \(f_x = f_y = \tfrac{1}{6}\)।
\end{itemize}

अतः,

\[
f(x,y) \approx 3 + \tfrac{1}{6}(x-4) + \tfrac{1}{6}(y-5).
\] इति

\subsubsection{एतत् किमर्थं
महत्त्वपूर्णम्}\label{ux90fux924ux924-ux915ux92eux930ux925-ux92eux939ux924ux924ux935ux92aux930ux923ux92e-14}

\begin{itemize}
\tightlist
\item
  स्पर्शरेखाविमानाः पृष्ठस्य उत्तमं रेखीयसन्निकर्षं ददति ।
\item
  रेखीयकरणेन गणनायाः कृते जटिलकार्यं सरलं भवति ।
\item
  संख्याविधिषु, भौतिकशास्त्रे, अर्थशास्त्रे च व्यापकरूपेण प्रयुक्तः ।
\end{itemize}

\subsubsection{अभ्यास}\label{ux905ux92dux92fux938-30}

\begin{enumerate}
\def\labelenumi{\arabic{enumi}.}
\tightlist
\item
  \((1,1)\) इत्यत्र \(z = x^2y + y^2\) इत्यस्य स्पर्शरेखाविमानं ज्ञातव्यम्।
\item
  \((0,0)\) के पास अनुमानित \(f(x,y) = e^{x+y}\)।
\item
  \((1,1)\) इत्यत्र \(z = \ln(x^2+y^2)\) इत्यस्य स्पर्शरेखा समीकरणं व्युत्पादयन्तु।
\item
  (4,6) इत्यस्य समीपे \(f(x,y) = \sqrt{x+y}\) इत्यस्य उपयोगेन \(\sqrt{10.1}\)
  इत्यस्य अनुमानं कर्तुं रेखीयसन्निकर्षस्य उपयोगं कुर्वन्तु ।
\item
  यथा यथा \((x,y)\) \((a,b)\) इत्यस्य समीपं गच्छति तथा तथा
  स्पर्शरेखाविमानसन्निकर्षः किमर्थं सुधरति इति व्याख्यातव्यम्।
\end{enumerate}

\subsection{8.5 अनेकचरयोः
अनुकूलनम्}\label{ux905ux928ux915ux91aux930ux92f-ux905ux928ux915ux932ux928ux92e}

बहुचरगणने अनुकूलनं अधिकतमस्य न्यूनतमस्य च विचारान् एकचरफलनात् द्वयोः वा अधिकचरयोः
कार्यपर्यन्तं विस्तारयति ।

\subsubsection{आलोचनात्मक
बिन्दु}\label{ux906ux932ux91aux928ux924ux92eux915-ux92cux928ux926}

\(f(x,y)\) कृते, एकः महत्त्वपूर्णः बिन्दुः भवति यत्र

\[
f_x(x,y) = 0 \quad \text{and} \quad f_y(x,y) = 0,
\] इति

यत्र वा आंशिकव्युत्पन्नाः न सन्ति।

\subsubsection{द्वितीय व्युत्पन्न परीक्षणमहत्त्वपूर्णबिन्दून् वर्गीकृत्य द्वितीयस्य
आंशिकव्युत्पन्नस्य गणनां कुर्वन्तु
:}\label{ux926ux935ux924ux92f-ux935ux92fux924ux92aux928ux928-ux92aux930ux915ux937ux923ux92eux939ux924ux924ux935ux92aux930ux923ux92cux928ux926ux928-ux935ux930ux917ux915ux924ux92f-ux926ux935ux924ux92fux938ux92f-ux906ux936ux915ux935ux92fux924ux92aux928ux928ux938ux92f-ux917ux923ux928-ux915ux930ux935ux928ux924}

\[
D = f_{xx}(a,b) f_{yy}(a,b) - \big(f_{xy}(a,b)\big)^2.
\] इति

\begin{itemize}
\tightlist
\item
  यदि \(D > 0\) तथा \(f_{xx}(a,b) > 0\): स्थानीय न्यूनतम।
\item
  यदि \(D > 0\) तथा \(f_{xx}(a,b) < 0\): स्थानीय अधिकतम।
\item
  यदि \(D < 0\): काठी बिन्दु।
\item
  यदि \(D = 0\): परीक्षणं निष्कर्षहीनं भवति।
\end{itemize}

\subsubsection{उदाहरणम् 1:
पराबोलोइड}\label{ux909ux926ux939ux930ux923ux92e-1-ux92aux930ux92cux932ux907ux921-1}

\(f(x,y) = x^2 + y^2\) इति ।

\begin{itemize}
\tightlist
\item
  \(f_x = 2x, f_y = 2y\)। (0,0) इत्यत्र आलोचनात्मकः बिन्दुः ।
\item
  \(f_{xx} = 2, f_{yy} = 2, f_{xy} = 0\)।
\item
  \(D = (2)(2) - 0 = 4 > 0\), तथा \(f_{xx} > 0\)।
\item
  अतः (0,0) इति स्थानीयं न्यूनतमम् अस्ति ।
\end{itemize}

\subsubsection{उदाहरणम् २ :
काठीबिन्दु}\label{ux909ux926ux939ux930ux923ux92e-ux968-ux915ux920ux92cux928ux926}

\(f(x,y) = x^2 - y^2\) इति ।

\begin{itemize}
\tightlist
\item
  \(f_x = 2x, f_y = -2y\)। (0,0) इत्यत्र आलोचनात्मकः बिन्दुः ।
\item
  \(f_{xx} = 2, f_{yy} = -2, f_{xy} = 0\)।
\item
  \(D = (2)(-2) - 0 = -4 < 0\)।
\item
  अतः (0,0) इति काठीबिन्दुः ।
\end{itemize}

\subsubsection{बाध्य अनुकूलनं तथा लैग्रेन्ज
गुणक}\label{ux92cux927ux92f-ux905ux928ux915ux932ux928-ux924ux925-ux932ux917ux930ux928ux91c-ux917ux923ux915}

कदाचित्, वयं \(g(x,y) = c\) इत्यस्य बाधायाः अधीनं \(f(x,y)\) इत्यस्य अनुकूलनं कर्तुम्
इच्छामः ।

लैग्रेन्ज गुणकानां विधिः : समाधानं कुर्वन्तु

\[
\nabla f(x,y) = \lambda \nabla g(x,y).
\] इति

उदाहरणम् : \(x^2+y^2=1\) इत्यस्य अधीनं \(f(x,y) = xy\) अधिकतमं कुर्वन्तु ।

\begin{itemize}
\tightlist
\item
  ढाल:
  \(\nabla f = \langle y,x \rangle, \quad \nabla g = \langle 2x,2y \rangle\)।
\item
  समीकरणम् : \(y = 2\lambda x, \, x = 2\lambda y\)।
\item
  समाधानं \((\pm \tfrac{1}{\sqrt{2}}, \pm \tfrac{1}{\sqrt{2}})\) इत्यत्र
  अधिकतमं प्रति नेति।
\end{itemize}

\subsubsection{एतत् किमर्थं
महत्त्वपूर्णम्}\label{ux90fux924ux924-ux915ux92eux930ux925-ux92eux939ux924ux924ux935ux92aux930ux923ux92e-15}

\begin{itemize}
\tightlist
\item
  अर्थशास्त्रे, अभियांत्रिकीशास्त्रे, यन्त्रशिक्षणे, भौतिकशास्त्रे च अनुकूलनं अत्यावश्यकम् ।
\item
  लैग्रेन्ज गुणकाः बाधाभिः सह अनुकूलनं अनुमन्यन्ते, यत् अनुप्रयुक्तगणितस्य प्रमुखं साधनम् अस्ति
  ।
\end{itemize}

\subsubsection{अभ्यास}\label{ux905ux92dux92fux938-31}

\begin{enumerate}
\def\labelenumi{\arabic{enumi}.}
\tightlist
\item
  \(f(x,y) = x^2+xy+y^2\) इत्यस्य महत्त्वपूर्णबिन्दून् ज्ञात्वा वर्गीकरणं कुर्वन्तु।
\item
  \(f(x,y) = x^3-y^3\) कृते बिन्दु (0,0) वर्गीकृत्य।
\item
  \(f(x,y) = x^4+y^4-4xy\) कृते द्वितीयव्युत्पन्नपरीक्षायाः उपयोगं कुर्वन्तु।
\item
  \(x^2+y^2=1\) इत्यस्य अधीनं \(f(x,y) = x+y\) अधिकतमं कुर्वन्तु।
\item
  \(x+y=1\) इत्यस्य अधीनं \(f(x,y) = x^2+2y^2\) इत्यस्य न्यूनीकरणं कुर्वन्तु।
\end{enumerate}

\section{अध्याय 9. बहु
अभिन्न}\label{ux905ux927ux92fux92f-9.-ux92cux939-ux905ux92dux928ux928}

\subsection{9.1 द्विगुण अभिन्नएकचरगणनायां निश्चितः अभिन्नः वक्रस्य अधः क्षेत्रं
ददाति । द्वयोः चरयोः द्विगुणः अभिन्नः पृष्ठस्य अधः आयतनस्य गणनां करोति (अथवा
अधिकसामान्यतया, क्षेत्रस्य उपरि मूल्यानां सञ्चयः)
।}\label{ux926ux935ux917ux923-ux905ux92dux928ux928ux90fux915ux91aux930ux917ux923ux928ux92f-ux928ux936ux91aux924-ux905ux92dux928ux928-ux935ux915ux930ux938ux92f-ux905ux927-ux915ux937ux924ux930-ux926ux926ux924-ux926ux935ux92f-ux91aux930ux92f-ux926ux935ux917ux923-ux905ux92dux928ux928-ux92aux937ux920ux938ux92f-ux905ux927-ux906ux92fux924ux928ux938ux92f-ux917ux923ux928-ux915ux930ux924-ux905ux925ux935-ux905ux927ux915ux938ux92eux928ux92fux924ux92f-ux915ux937ux924ux930ux938ux92f-ux909ux92aux930-ux92eux932ux92fux928-ux938ux91eux91aux92f}

\subsubsection{परिभाषा}\label{ux92aux930ux92dux937-9}

यदि \(f(x,y)\) कस्मिन्चित् प्रदेशे \(R\) निरन्तरं भवति तर्हि द्विगुणं अभिन्नं भवति

\[
\iint_R f(x,y)\, dA = \lim_{m,n \to \infty} \sum_{i=1}^m \sum_{j=1}^n f(x_{ij}^-, y_{ij}^-) \Delta A,
\]

यत्र \(R\) क्षेत्रफलस्य \(\Delta A\) लघु आयतेषु विभक्तम् अस्ति ।

\subsubsection{पुनरावर्तित
अभिन्न}\label{ux92aux928ux930ux935ux930ux924ux924-ux905ux92dux928ux928}

फुबिनी इत्यस्य प्रमेयेन वयं द्विगुणं अभिन्नं पुनरावृत्तं अभिन्नरूपेण गणयितुं शक्नुमः :

\[
\iint_R f(x,y)\, dA = \int_a^b \int_c^d f(x,y)\, dy\, dx,
\] इति

यदि \(R\) आयत \([a,b] \times [c,d]\) अस्ति।

एकीकरणस्य क्रमः प्रायः स्विच् कर्तुं शक्यते:

\[
\int_a^b \int_c^d f(x,y)\,dy\,dx = \int_c^d \int_a^b f(x,y)\,dx\,dy.
\] इति

\subsubsection{उदाहरणम्}\label{ux909ux926ux939ux930ux923ux92e-19}

\begin{enumerate}
\def\labelenumi{\arabic{enumi}.}
\tightlist
\item
  आयतप्रदेशः
\end{enumerate}

\[
\iint_R (x+y)\, dA, \quad R=[0,1]\times[0,2].
\] इति

\[
= \int_0^1 \int_0^2 (x+y)\,dy\,dx = \int_0^1 \Big[xy+\tfrac{1}{2}y^2\Big]_0^2 dx
= \int_0^1 (2x+2)dx = 3.
\] इति

\begin{enumerate}
\def\labelenumi{\arabic{enumi}.}
\setcounter{enumi}{1}
\tightlist
\item
  त्रिकोणप्रदेशः
\end{enumerate}

\[
R = \{(x,y): 0 \leq x \leq 1, 0 \leq y \leq x\}.
\] इति

\[
\iint_R (x+y)\, dA = \int_0^1 \int_0^x (x+y)\,dy\,dx.
\] इति

मूल्याङ्कनं कृत्वा \(\tfrac{2}{3}\) प्राप्यते ।

\subsubsection{अनुप्रयोग}\label{ux905ux928ux92aux930ux92fux917-1}

\begin{itemize}
\tightlist
\item
  एकस्य पृष्ठस्य अधः आयतनम् : १.
\end{itemize}

\[
V = \iint_R f(x,y)\, dA.
\] इति

\begin{itemize}
\tightlist
\item
  कस्यचित् क्षेत्रस्य उपरि कार्यस्य औसतं मूल्यम् : १.
\end{itemize}

\[
f_{\text{avg}} = \frac{1}{A(R)} \iint_R f(x,y)\, dA.
\] इति

\subsubsection{एतत् किमर्थं
महत्त्वपूर्णम्}\label{ux90fux924ux924-ux915ux92eux930ux925-ux92eux939ux924ux924ux935ux92aux930ux923ux92e-16}

द्विगुण-अखण्डाः एकीकरणं द्वि-आयामपर्यन्तं विस्तारयन्ति । भौतिकशास्त्रे (द्रव्यमानं,
संभाव्यतावितरणं), अर्थशास्त्रे (अपेक्षितमूल्यानि), अभियांत्रिकीशास्त्रे (सेंट्रॉइड्, प्रवाहः)
च ते अत्यावश्यकाः सन्ति ।

\subsubsection{अभ्यास}\label{ux905ux92dux92fux938-32}

\begin{enumerate}
\def\labelenumi{\arabic{enumi}.}
\tightlist
\item
  \(\iint_R (x^2+y^2)\, dA\) मूल्याङ्कनं कुर्वन्तु यत्र \(R=[0,1]\times[0,1]\)।
\item
  \(\iint_R xy\, dA\) गणयन्तु यत्र
  \(R=\{(x,y):0\leq x\leq2,0\leq y\leq x\}\)।3. इकाईवर्गस्य
  \([0,1]\times[0,1]\) इत्यस्य उपरि \(f(x,y) = x+y\) इत्यस्य औसतमूल्यं ज्ञातव्यम्
  ।
\item
  \(\iint_R f(x,y)\, dA\) इत्यस्य व्याख्या संभाव्यतायाः दृष्ट्या कुर्वन्तु यदि
  \(f(x,y)\) संभाव्यताघनत्वकार्यं भवति।
\item
  दर्शयतु यत् एकीकरणस्य स्विचिंग् क्रमः \(\iint_{[0,1]\times[0,2]} (x+y)\,dA\)
  कृते समानं परिणामं ददाति ।
\end{enumerate}

\subsection{9.2 त्रिगुण
अभिन्न}\label{ux924ux930ux917ux923-ux905ux92dux928ux928}

त्रिगुणात्मकाः अभिन्नाः एकीकरणस्य विचारं त्रयः चराः यावत् विस्तारयन्ति, येन अस्माभिः
त्रिविमप्रदेशेषु आयतनं, द्रव्यमानं, अन्यमात्राश्च गणयितुं शक्यते

\subsubsection{परिभाषा}\label{ux92aux930ux92dux937-10}

यदि \(f(x,y,z)\) ठोसप्रदेशे \(E\) इत्यत्र निरन्तरं भवति तर्हि त्रिगुणं अभिन्नं भवति

\[
\iiint_E f(x,y,z)\, dV = \lim_{m,n,p \to \infty} \sum f(x_{ijk}^-, y_{ijk}^-, z_{ijk}^-) \Delta V,
\] इति

यत्र प्रदेशः \(\Delta V\) आयतनस्य पेटीषु उपविभक्तः अस्ति ।

\subsubsection{पुनरावर्तित
अभिन्न}\label{ux92aux928ux930ux935ux930ux924ux924-ux905ux92dux928ux928-1}

फुबिनी इत्यस्य प्रमेयेन त्रिगुणा अभिन्नस्य गणना पुनरावर्तित अभिन्नरूपेण कर्तुं शक्यते :

\[
\iiint_E f(x,y,z)\, dV = \int_a^b \int_c^d \int_e^f f(x,y,z)\, dz\, dy\, dx,
\] इति

आयताकारपेटिकायाः \hspace{0pt}\hspace{0pt}कृते
\(E = [a,b]\times[c,d]\times[e,f]\) ।

एकीकरणस्य क्रमः सुविधायै चिन्वितुं शक्यते ।

\subsubsection{उदाहरणम्}\label{ux909ux926ux939ux930ux923ux92e-20}

\begin{enumerate}
\def\labelenumi{\arabic{enumi}.}
\tightlist
\item
  आयताकार पेटी
\end{enumerate}

\[
\iiint_E xyz\, dV, \quad E=[0,1]\times[0,2]\times[0,3].
\] इति

\[
= \int_0^1 \int_0^2 \int_0^3 xyz\,dz\,dy\,dx.
\] इति

प्रथमं \(z\) इत्यस्य उपरि एकीकृत्य:

\[
\int_0^3 xyz\,dz = xy \left[\tfrac{1}{2}z^2\right]_0^3 = \tfrac{9}{2}xy.
\] इति

अधुना \(y\) इत्यस्य उपरि एकीकृत्य:

\[
\int_0^2 \tfrac{9}{2}xy\,dy = \tfrac{9}{2}x \cdot \left[\tfrac{1}{2}y^2\right]_0^2 = 9x.
\]

अन्ते \(x\) इत्यस्य उपरि एकीकृत्य:

\[
\int_0^1 9x\,dx = \tfrac{9}{2}.
\] इति

\begin{enumerate}
\def\labelenumi{\arabic{enumi}.}
\setcounter{enumi}{1}
\tightlist
\item
  विमानैः परिसीमितः प्रदेशः
  \(E = \{(x,y,z) \mid 0 \leq x \leq 1, 0 \leq y \leq x, 0 \leq z \leq y\}\)
  इति ।
\end{enumerate}

\[
\iiint_E 1\,dV = \int_0^1 \int_0^x \int_0^y 1\,dz\,dy\,dx.
\] इति

गणयति:

\[
= \int_0^1 \int_0^x y\,dy\,dx = \int_0^1 \tfrac{1}{2}x^2\,dx = \tfrac{1}{6}.
\] इतिअतः अस्य त्रिकोणप्रदेशस्य आयतनं \(\tfrac{1}{6}\) अस्ति ।

\subsubsection{अनुप्रयोग}\label{ux905ux928ux92aux930ux92fux917-2}

\begin{itemize}
\item
  खण्डः \(V = \iiint_E 1 \, dV\)।
\item
  द्रव्यमानम् : यदि घनत्वं \(\rho(x,y,z)\) अस्ति, तर्हि

  \[
  M = \iiint_E \rho(x,y,z)\, dV.
  \] इति
\item
  औसतं मूल्यम् : १.

  \[
  f_{\text{avg}} = \frac{1}{V(E)} \iiint_E f(x,y,z)\,dV.
  \] इति
\end{itemize}

\subsubsection{एतत् किमर्थं
महत्त्वपूर्णम्}\label{ux90fux924ux924-ux915ux92eux930ux925-ux92eux939ux924ux924ux935ux92aux930ux923ux92e-17}

त्रिगुणाभिन्नं क्षेत्रफलं आयतनगणनां च मनमाना ठोसद्रव्येषु सामान्यीकरणं करोति ।
भौतिकशास्त्रे (द्रव्यमानवितरणं, द्रव्यमानकेन्द्रं, गुरुत्वाकर्षणक्षेत्राणि),
अभियांत्रिकीशास्त्रे, संभाव्यतायां च तेषां उपयोगः भवति ।

\subsubsection{अभ्यास}\label{ux905ux92dux92fux938-33}

\begin{enumerate}
\def\labelenumi{\arabic{enumi}.}
\tightlist
\item
  \(E=[0,1]\times[0,1]\times[0,1]\) घनस्य उपरि \(\iiint_E (x+y+z)\,dV\)
  गणनां कुर्वन्तु।
\item
  \(x=0, y=0, z=0, x+y+z=1\) द्वारा सीमाबद्ध चतुर्भुजस्य आयतनं ज्ञातव्यम्।
\item
  \(\iiint_E x^2 \, dV\) मूल्याङ्कनं कुर्वन्तु यत्र
  \(E=[0,2]\times[0,1]\times[0,1]\)।
\item
  दर्शयतु यत् \(\iiint_E 1\,dV\) \(E\) इत्यस्य ज्यामितीय आयतनस्य बराबरम् अस्ति।
\item
  यदि घनत्वं \(\rho(x,y,z)=x+y+z\) अस्ति तर्हि एककघनस्य द्रव्यमानं गणयन्तु ।
\end{enumerate}

\subsection{9.3 अनुप्रयोगाः : आयतन, द्रव्यमान,
संभावना}\label{ux905ux928ux92aux930ux92fux917-ux906ux92fux924ux928-ux926ux930ux935ux92fux92eux928-ux938ux92dux935ux928}

त्रिगुणाभिन्नाः शक्तिशालिनः भवन्ति यतोहि ते अस्मान् ठोसप्रदेशे मूल्यसञ्चयेन त्रिविमेषु
परिमाणानां गणनां कर्तुं शक्नुवन्ति ।

\subsubsection{मात्रा}\label{ux92eux924ux930}

सरलतमः अनुप्रयोगः \(E\) इति क्षेत्रस्य आयतनं अन्वेष्टुं भवति:

\[
V = \iiint_E 1 \, dV.
\] इति

उदाहरण: निर्देशांकविमानैः सीमाबद्धस्य ठोसस्य आयतनं तथा विमानं \(x+y+z=1\) ज्ञातव्यम्
।

\[
V = \iiint_E 1 \, dV = \int_0^1 \int_0^{1-x} \int_0^{1-x-y} 1 \, dz\, dy\, dx.
\]

मूल्याङ्कनं कृत्वा \(V = \tfrac{1}{6}\) प्राप्यते ।

\subsubsection{द्रव्यमान एवं
घनत्व}\label{ux926ux930ux935ux92fux92eux928-ux90fux935-ux918ux928ux924ux935}

यदि ठोसस्य घनत्वकार्यं \(\rho(x,y,z)\) भवति तर्हि तस्य द्रव्यमानं भवति

\[
M = \iiint_E \rho(x,y,z)\, dV.
\] इति

द्रव्यमानकेन्द्रं दत्तं भवति

\[
\bar{x} = \frac{1}{M}\iiint_E x\rho(x,y,z)\,dV, \quad
\bar{y} = \frac{1}{M}\iiint_E y\rho(x,y,z)\,dV, \quad
\bar{z} = \frac{1}{M}\iiint_E z\rho(x,y,z)\,dV.
\] इति

उदाहरण:\(\rho=1\) नित्यघनत्वयुक्तस्य एककघनस्य कृते द्रव्यमानस्य केन्द्रं
\((0.5,0.5,0.5)\) इत्यत्र भवति ।

\subsubsection{सम्भावना}\label{ux938ux92eux92dux935ux928}

यदि \(f(x,y,z)\) 3D मध्ये संभाव्यताघनत्वकार्यं भवति, तर्हि यादृच्छिकचरः \(E\) क्षेत्रे
निहितः इति संभावना अस्ति

\[
P(E) = \iiint_E f(x,y,z)\, dV,
\] इति

यत्र \(f(x,y,z) \geq 0\) तथा

\[
\iiint_{\mathbb{R}^3} f(x,y,z)\,dV = 1.
\] इति

उदाहरण: यदि \(x,y\) कृते \(f(x,y,z) = \tfrac{3}{4}z^2\), \(x,y\) इत्यत्र
एकरूपतया, तर्हि

\[
P(0 \leq z \leq 0.5) = \int_0^{0.5} \tfrac{3}{4}z^2 \, dz = \tfrac{1}{32}.
\] इति

\subsubsection{एतत् किमर्थं
महत्त्वपूर्णम्}\label{ux90fux924ux924-ux915ux92eux930ux925-ux92eux939ux924ux924ux935ux92aux930ux923ux92e-18}

\begin{itemize}
\tightlist
\item
  आयतनाः ज्यामितिम् अनियमितघनद्रव्येषु सामान्ययन्ति ।
\item
  द्रव्यमानस्य घनत्वस्य च अभिन्नाः गणितं भौतिकशास्त्रेण अभियांत्रिकीशास्त्रेण च
  सम्बध्दयन्ति ।
\item
  उच्चतरपरिमाणेषु संभाव्यताघनत्वकार्यस्य व्यापकरूपेण उपयोगः सांख्यिकीशास्त्रे,
  आँकडाविज्ञाने च भवति ।
\end{itemize}

\subsubsection{अभ्यास}\label{ux905ux92dux92fux938-34}

\begin{enumerate}
\def\labelenumi{\arabic{enumi}.}
\tightlist
\item
  \(x^2+y^2+z^2 \leq 1\) (एककगोल) द्वारा सीमाबद्धस्य ठोसस्य आयतनं ज्ञातव्यम् ।
\item
  \(z\) इत्यस्य आनुपातिकघनत्वयुक्तस्य शङ्कुस्य द्रव्यमानं गणयन्तु ।
\item
  \(x=0, y=0, z=0, x+y+z=1\) द्वारा सीमाबद्धस्य एकरूपचतुर्भुजस्य द्रव्यमानकेन्द्रं
  ज्ञातव्यम् ।
\item
  यदि घन \([0,2]\times[0,2]\times[0,2]\) इत्यत्र
  \(f(x,y,z) = \frac{1}{8}\) तर्हि सत्यापयन्तु यत् एतत् संभाव्यताघनत्वकार्यम् अस्ति
  ।
\item
  एककगोले यादृच्छिकरूपेण चयनितबिन्दुस्य \(z > 0\) इति संभावनायाः गणनाय त्रिगुणा
  अभिन्नस्य उपयोगं कुर्वन्तु ।
\end{enumerate}

\subsection{9.4 चर परिवर्तन : ध्रुवीय, बेलनाकार, गोलाकार
निर्देशांक}\label{ux91aux930-ux92aux930ux935ux930ux924ux928-ux927ux930ux935ux92f-ux92cux932ux928ux915ux930-ux917ux932ux915ux930-ux928ux930ux926ux936ux915}

अनेकाः अभिन्नाः प्रदेशस्य समरूपतायाः अनुरूपेषु समन्वयप्रणालीषु व्यक्तेषु सुलभाः भवन्ति ।
कार्टेशियन निर्देशांक \((x,y,z)\) इत्यस्य स्थाने वयं ध्रुवीय, बेलनाकार, अथवा गोलाकार
निर्देशांकस्य उपयोगं कर्तुं शक्नुमः ।

\subsubsection{ध्रुवीय निर्देशांक
(2D)}\label{ux927ux930ux935ux92f-ux928ux930ux926ux936ux915-2d}

द्वयोः चरयोः कार्याणां कृते वयं ध्रुवीयनिर्देशाङ्कं प्रति स्विच् कर्तुं शक्नुमः :

\[
x = r\cos\theta, \quad y = r\sin\theta, \quad r \geq 0, \; 0 \leq \theta < 2\pi.
\] इति

क्षेत्रतत्त्वं यथा परिणमति

\[
dA = r\,dr\,d\theta.
\] इति

उदाहरण:एककवृत्तस्य क्षेत्रफलं ज्ञातव्यम् ।

\[
A = \iint_{x^2+y^2\leq 1} 1\,dA = \int_0^{2\pi}\int_0^1 r\,dr\,d\theta = \pi.
\] इति

\subsubsection{बेलनाकार निर्देशांक
(3D)}\label{ux92cux932ux928ux915ux930-ux928ux930ux926ux936ux915-3d}

3D इत्यस्मिन् बेलनाकारनिर्देशाङ्काः \(z\) इत्यनेन सह ध्रुवीयनिर्देशाङ्कान् विस्तारयन्ति:

\[
x = r\cos\theta, \quad y = r\sin\theta, \quad z = z.
\]

आयतनतत्त्वम् अस्ति

\[
dV = r\,dr\,d\theta\,dz.
\] इति

उदाहरण: \(R\) त्रिज्यायुक्तस्य सिलिण्डरस्य आयतनं तथा ऊर्ध्वता \(h\):

\[
V = \int_0^h \int_0^{2\pi} \int_0^R r\,dr\,d\theta\,dz = \pi R^2 h.
\] इति

\subsubsection{गोलाकार निर्देशांक
(3D)}\label{ux917ux932ux915ux930-ux928ux930ux926ux936ux915-3d}

गोलाकारसमरूपतायाः कृते : १.

\[
x = \rho \sin\phi \cos\theta, \quad y = \rho \sin\phi \sin\theta, \quad z = \rho \cos\phi,
\] इति

कुत्र

\begin{itemize}
\tightlist
\item
  \(\rho \geq 0\) इति उत्पत्तितः दूरं, .
\item
  \(0 \leq \phi \leq \pi\) धनात्मक \(z\)-अक्षतः कोणः,
\item
  \(0 \leq \theta < 2\pi\) \(xy\)-विमानस्य कोणः ।
\end{itemize}

आयतनतत्त्वम् अस्ति

\[
dV = \rho^2 \sin\phi \, d\rho\, d\phi\, d\theta.
\] इति

उदाहरण: एककगोलस्य आयतनम् : १.

\[
V = \int_0^{2\pi} \int_0^\pi \int_0^1 \rho^2 \sin\phi \, d\rho\, d\phi\, d\theta.
\] इति

मूल्याङ्कनम् : १.

\[
V = \left(\int_0^1 \rho^2 d\rho\right)\left(\int_0^\pi \sin\phi d\phi\right)\left(\int_0^{2\pi} d\theta\right) = \tfrac{1}{3}(2)(2\pi) = \tfrac{4\pi}{3}.
\] इति

\subsubsection{एतत् किमर्थं
महत्त्वपूर्णम्}\label{ux90fux924ux924-ux915ux92eux930ux925-ux92eux939ux924ux924ux935ux92aux930ux923ux92e-19}

\begin{itemize}
\tightlist
\item
  ध्रुवीयनिर्देशाङ्काः वृत्तप्रदेशान् सरलीकरोति ।
\item
  बेलनाकारनिर्देशाङ्काः सिलिण्डरान् घूर्णनसमरूपतां च सम्पादयन्ति ।
\item
  गोलाकारनिर्देशाङ्काः गोलाकाराः, शङ्कुः, त्रिज्यासमस्याः च सरलीकरोति ।
\item
  चरानाम् एते परिवर्तनाः अन्यथा असम्भवाः अभिन्नाः प्रबन्धनीयाः भवन्ति ।
\end{itemize}

\subsubsection{अभ्यास}\label{ux905ux92dux92fux938-35}

\begin{enumerate}
\def\labelenumi{\arabic{enumi}.}
\tightlist
\item
  ध्रुवीय निर्देशांकस्य उपयोगेन \(\iint_{x^2+y^2\leq 4} (x^2+y^2)\,dA\) गणनां
  कुर्वन्तु।
\item
  बेलनाकार निर्देशांकस्य उपयोगेन \(h\) ऊर्ध्वतायां \(R\) त्रिज्यायां च शङ्कुस्य आयतनं
  ज्ञातव्यम् ।
\item
  त्रिज्या \(R\) इत्यस्य गोलस्य आयतनस्य मूल्याङ्कनार्थं गोलाकारनिर्देशाङ्कानां उपयोगं
  कुर्वन्तु ।
\item
  ध्रुवीयनिर्देशाङ्कानां कृते जैकोबियनगुणकः \(r\) इति दर्शयतु ।5. गोलाकार निर्देशांकस्य
  उपयोगेन उत्पत्तितः दूरीयाः आनुपातिकघनत्वस्य \(R\) त्रिज्यायाः ठोसगोलस्य द्रव्यमानं
  ज्ञातव्यम् ।
\end{enumerate}

\section{अध्याय 10. सदिश
गणित}\label{ux905ux927ux92fux92f-10.-ux938ux926ux936-ux917ux923ux924}

\subsection{10.1 सदिश क्षेत्र}\label{ux938ux926ux936-ux915ux937ux924ux930}

सदिशक्षेत्रं अन्तरिक्षे प्रत्येकं बिन्दुं प्रति सदिशं नियुक्तं करोति, यथा स्केलरफंक्शन् संख्यां
नियुक्तं करोति । प्रवाहस्य, बलानां, अन्येषां दिग्मात्राणां प्रतिरूपणार्थं सदिशक्षेत्राणां
उपयोगः भवति ।

\subsubsection{परिभाषा}\label{ux92aux930ux92dux937-11}

द्वयोः आयामयोः सदिशक्षेत्रं कार्यं भवति

\[
\mathbf{F}(x,y) = \langle P(x,y), Q(x,y) \rangle,
\] इति

यत्र \(P\) तथा \(Q\) स्केलर फंक्शन्स् सन्ति ।

त्रिविमेषु, २.

\[
\mathbf{F}(x,y,z) = \langle P(x,y,z), Q(x,y,z), R(x,y,z) \rangle.
\] इति

\subsubsection{उदाहरणम्}\label{ux909ux926ux939ux930ux923ux92e-21}

\begin{enumerate}
\def\labelenumi{\arabic{enumi}.}
\tightlist
\item
  त्रिज्याक्षेत्रम्
\end{enumerate}

\[
\mathbf{F}(x,y) = \langle x, y \rangle.
\] इति

सदिशाः उत्पत्तितः बहिः सूचयन्ति ।

\begin{enumerate}
\def\labelenumi{\arabic{enumi}.}
\setcounter{enumi}{1}
\tightlist
\item
  घूर्णनक्षेत्रम्
\end{enumerate}

\[
\mathbf{F}(x,y) = \langle -y, x \rangle.
\] इति

सदिशाः उत्पत्तिं परितः परिभ्रमन्ति ।

\begin{enumerate}
\def\labelenumi{\arabic{enumi}.}
\setcounter{enumi}{2}
\tightlist
\item
  गुरुत्वाकर्षणक्षेत्रम्
\end{enumerate}

\[
\mathbf{F}(x,y,z) = -\frac{GM}{r^3}\langle x,y,z \rangle, \quad r=\sqrt{x^2+y^2+z^2}.
\] इति

\subsubsection{सदिश क्षेत्रों का
दृश्यीकरण}\label{ux938ux926ux936-ux915ux937ux924ux930-ux915-ux926ux936ux92fux915ux930ux923}

\begin{itemize}
\tightlist
\item
  दिशां परिमाणं च सूचयितुं नमूनाबिन्दुषु लघुबाणान् आकर्षयन्तु।
\item
  सघनतराः बाणाः यत्र परिमाणं बृहत्तरं भवति।
\item
  प्रवाहरेखाः, प्रक्षेपवक्राः, बलानि च व्याख्यातुं उपयोगी ।
\end{itemize}

\subsubsection{प्रवाह रेखाएँ}\label{ux92aux930ux935ux939-ux930ux916ux90f}

सदिशक्षेत्रस्य प्रवाहरेखा (अथवा अभिन्नवक्रं) वक्रं \(\mathbf{r}(t)\) भवति यस्य
प्रत्येकस्मिन् बिन्दौ स्पर्शरेखासदिशः क्षेत्रेण सह मेलति:

\[
\mathbf{r}'(t) = \mathbf{F}(\mathbf{r}(t)).
\] इति

प्रवाहरेखाः वेगक्षेत्रे कणमार्गाणां वर्णनं कुर्वन्ति ।

\subsubsection{एतत् किमर्थं
महत्त्वपूर्णम्}\label{ux90fux924ux924-ux915ux92eux930ux925-ux92eux939ux924ux924ux935ux92aux930ux923ux92e-20}

\begin{itemize}
\tightlist
\item
  भौतिकशास्त्रे (द्रवप्रवाहः, विद्युत्चुम्बकत्वम्, गुरुत्वाकर्षणं) सदिशक्षेत्राणि मौलिकाः
  सन्ति ।
\item
  ते रेखा अभिन्नस्य, पृष्ठीय अभिन्नस्य, सदिशगणितस्य (Green, Stokes, Divergence) च
  बृहत् प्रमेयस्य आधारं भवन्ति ।
\item
  दिशात्मकमात्राणां प्रतिनिधित्वार्थं ज्यामितीयमार्गं प्रदातव्यम्।
\end{itemize}

\subsubsection{\texorpdfstring{अभ्यास1. सदिशक्षेत्रस्य
\(\mathbf{F}(x,y) = \langle y, -x \rangle\) इत्यस्य रेखाचित्रं
कुर्वन्तु।}{अभ्यास1. सदिशक्षेत्रस्य \textbackslash mathbf\{F\}(x,y) = \textbackslash langle y, -x \textbackslash rangle इत्यस्य रेखाचित्रं कुर्वन्तु।}}\label{ux905ux92dux92fux9381.-ux938ux926ux936ux915ux937ux924ux930ux938ux92f-mathbffxy-langle-y--x-rangle-ux907ux924ux92fux938ux92f-ux930ux916ux91aux924ux930-ux915ux930ux935ux928ux924}

\begin{enumerate}
\def\labelenumi{\arabic{enumi}.}
\setcounter{enumi}{1}
\tightlist
\item
  निर्धारयतु यत् \(\mathbf{F}(x,y) = \langle x,y \rangle\) इत्यस्य सदिशाः
  उत्पत्तिं प्रति वा दूरं वा दर्शयन्ति वा।
\item
  \(\mathbf{F}(x,y,z) = \langle y, z, x \rangle\) कृते
  \(\mathbf{F}(1,2,3)\) इति गणनां कुर्वन्तु।
\item
  \(\mathbf{F}(x,y) = \langle -y, x \rangle\) इत्यस्य प्रवाहरेखाः वर्णयन्तु।
\item
  गुरुत्वाकर्षणक्षेत्रं विद्युत्क्षेत्रं च त्रिज्यासदिशक्षेत्रस्य उदाहरणं किमर्थं भवति इति
  व्याख्यातव्यम्।
\end{enumerate}

\subsection{10.2 रेखा अभिन्न}\label{ux930ux916-ux905ux92dux928ux928}

रेखा अभिन्नः अभिन्नस्य विचारं वक्रेण सह मूल्याङ्कितकार्यं प्रति विस्तारयति । अन्तरालस्य
अथवा प्रदेशस्य उपरि एकीकरणस्य स्थाने वयं अन्तरिक्षे एकस्य मार्गस्य उपरि एकीकरणं कुर्मः ।

\subsubsection{परिभाषा: अदिश रेखा
अभिन्न}\label{ux92aux930ux92dux937-ux905ux926ux936-ux930ux916-ux905ux92dux928ux928}

यदि \(f(x,y)\) एकं स्केलर फंक्शन् अस्ति तथा च \(C\)
\(\mathbf{r}(t) = \langle x(t), y(t) \rangle, \; a \leq t \leq b\) द्वारा
पैरामीटरीकृतं वक्रं भवति, तर्हि रेखा अभिन्नं भवति

\[
\int_C f(x,y)\, ds = \int_a^b f(x(t),y(t)) \, |\mathbf{r}'(t)|\, dt,
\] इति

यत्र \(ds\) चापदीर्घता अस्ति।

एतेन वक्रस्य पार्श्वे \(f\) इत्यस्य सञ्चयः मापितः भवति ।

\subsubsection{परिभाषा: सदिश रेखा
अभिन्न}\label{ux92aux930ux92dux937-ux938ux926ux936-ux930ux916-ux905ux92dux928ux928}

सदिशक्षेत्रस्य \(\mathbf{F}(x,y) = \langle P(x,y), Q(x,y) \rangle\) कृते,
\(C\) इत्यनेन सह रेखा अभिन्नं भवति

\[
\int_C \mathbf{F} \cdot d\mathbf{r} = \int_a^b \mathbf{F}(\mathbf{r}(t)) \cdot \mathbf{r}'(t)\, dt.
\] इति

एतेन वक्रस्य पार्श्वे क्षेत्रेण कृतं कार्यं माप्यते ।

\subsubsection{उदाहरणम्}\label{ux909ux926ux939ux930ux923ux92e-22}

\begin{enumerate}
\def\labelenumi{\arabic{enumi}.}
\tightlist
\item
  अदिश रेखा अभिन्न
\end{enumerate}

\[
f(x,y) = x+y, \quad C: \mathbf{r}(t) = \langle t, t^2 \rangle, \; 0 \leq t \leq 1.
\] इति

तदा

\[
\int_C f(x,y)\, ds = \int_0^1 (t+t^2)\sqrt{(1)^2+(2t)^2}\, dt.
\] इति

\begin{enumerate}
\def\labelenumi{\arabic{enumi}.}
\setcounter{enumi}{1}
\tightlist
\item
  बलेन कृतं कार्यम्
\end{enumerate}

\[
\mathbf{F}(x,y) = \langle y, x \rangle, \quad C: \mathbf{r}(t) = \langle t, t^2 \rangle, \; 0 \leq t \leq 1.
\]

\[ इति
\int_C \mathbf{F} \cdot d\mathbf{r} = \int_0^1 \langle t^2, t \angle \cdot \langle 1, 2t \rangle\, dt = \int_0^1 (t^2 + 2t^2)\, dt = \int_0^1 3t^2\, dt = 1.\]

\subsubsection{Physical Interpretation}\label{physical-interpretation}

\begin{itemize}
\tightlist
\item
  Scalar line integral: accumulation of density along a wire.
\item
  Vector line integral: work done by a force moving an object along a
  path.
\end{itemize}

\subsubsection{Why This Matters}\label{why-this-matters}

\begin{itemize}
\tightlist
\item
  Line integrals connect vector fields with physical quantities like
  work and circulation.
\item
  They are building blocks for Green's Theorem and Stokes' Theorem.
\item
  Appear in physics (electric potential, fluid flow, mechanics).
\end{itemize}

\subsubsection{Exercises}\label{exercises-3}

\begin{enumerate}
\def\labelenumi{\arabic{enumi}.}
\tightlist
\item
  Compute \(\int_C (x^2+y^2)\, ds\) where \(C\) is the line segment from
  (0,0) to (1,1).
\item
  Evaluate \(\int_C \mathbf{F}\cdot d\mathbf{r}\) for
  \(\mathbf{F}(x,y) = \langle -y, x \rangle\) along the unit circle
  \(x^2+y^2=1\).
\item
  Interpret the meaning of \(\int_C 1\,ds\).
\item
  For \(\mathbf{F}(x,y,z) = \langle z,0,x \rangle\), compute the line
  integral along
  \(\mathbf{r}(t) = \langle t,t,1 \rangle, 0 \leq t \leq 1\).
\item
  Explain the difference between scalar and vector line integrals.
\end{enumerate}

\subsection{10.3 Surface Integrals}\label{surface-integrals}

A surface integral generalizes line integrals to two-dimensional
surfaces in three-dimensional space. They allow us to compute flux
through surfaces and accumulation of scalar fields over curved surfaces.

\subsubsection{Scalar Surface Integral}\label{scalar-surface-integral}

If a surface \(S\) is parameterized by

\[ इति
\mathbf{r}(उ,व) = \langle x(u,v), y(u,v), z(u,v) \rangle,
\]

then the surface integral of a scalar function \(f(x,y,z)\) is

\[ इति
\iint_S f(x,y,z)\, dS = \iint_D f(\mathbf{r}(u,v)) \, |\mathbf{r}_u \times \mathbf{r}_v| \, दु\,द्वि, ९.
\]

where \(\mathbf{r}_u\) and \(\mathbf{r}_v\) are partial derivatives of
\(\mathbf{r}(u,v)\), and \(D\) is the parameter domain.

\subsubsection{Vector Surface Integral
(Flux)}\label{vector-surface-integral-flux}

For a vector field \(\mathbf{F}(x,y,z)\), the flux through a surface
\(S\) is

\[
\iint_S \mathbf{F}\cdot d\mathbf{S} = \iint_S \mathbf{F}\cdot \mathbf{n}\, dS,
\] इतियत्र \(\mathbf{n}\) एककसामान्यसदिशः अस्ति । पैरामीटराइजेशनस्य उपयोगेन, .

\[
\iint_S \mathbf{F}\cdot d\mathbf{S} = \iint_D \mathbf{F}(\mathbf{r}(u,v)) \cdot (\mathbf{r}_u \times \mathbf{r}_v)\,du\,dv.
\] इति

\subsubsection{उदाहरणम्}\label{ux909ux926ux939ux930ux923ux92e-23}

\begin{enumerate}
\def\labelenumi{\arabic{enumi}.}
\tightlist
\item
  अदिश पृष्ठ अभिन्न पृष्ठभाग: इकाई डिस्क \(x^2+y^2 \leq 1\) उपरि विमान
  \(z=1\)।
\end{enumerate}

\[
\iint_S 1\, dS = \text{area of the disk} = \pi.
\] इति

\begin{enumerate}
\def\labelenumi{\arabic{enumi}.}
\setcounter{enumi}{1}
\tightlist
\item
  गोलस्य माध्यमेन प्रवाहः \(\mathbf{F}(x,y,z) = \langle x,y,z \rangle\), तथा
  \(S\) = त्रिज्या के गोल \(R\) । सामान्य सदिशः
  \(\mathbf{n} = \frac{1}{R}\langle x,y,z \rangle\) अस्ति ।
\end{enumerate}

\[
\mathbf{F}\cdot \mathbf{n} = \frac{x^2+y^2+z^2}{R} = R.
\] इति

अतः

\[
\iint_S \mathbf{F}\cdot d\mathbf{S} = \iint_S R\, dS = R \cdot 4\pi R^2 = 4\pi R^3.
\] इति

\subsubsection{एतत् किमर्थं
महत्त्वपूर्णम्}\label{ux90fux924ux924-ux915ux92eux930ux925-ux92eux939ux924ux924ux935ux92aux930ux923ux92e-21}

\begin{itemize}
\tightlist
\item
  अदिशपृष्ठीय अभिन्नाः क्षेत्रफलं पृष्ठवितरणं च मापयन्ति ।
\item
  सदिशपृष्ठस्य अभिन्नाः प्रवाहं मापयन्ति : पृष्ठतः गच्छन्तस्य क्षेत्रस्य परिमाणम् ।
\item
  अनुप्रयोगाः : विद्युत्चुम्बकत्वम्, द्रवप्रवाहः, तापसञ्चारः, इत्यादयः।
\end{itemize}

\subsubsection{अभ्यास}\label{ux905ux92dux92fux938-36}

\begin{enumerate}
\def\labelenumi{\arabic{enumi}.}
\tightlist
\item
  पार्श्वदीर्घतायाः घनस्य पृष्ठस्य कृते \(\iint_S 1\, dS\) गणयन्तु 2.
\item
  एककगोलस्य माध्यमेन \(\mathbf{F}(x,y,z) = \langle x,y,z \rangle\) इत्यस्य
  प्रवाहं ज्ञातव्यम्।
\item
  पराबोलोइडस्य \(z = x^2+y^2, \, z \leq 1\) कृते \(\iint_S z\, dS\) इत्यस्य
  मूल्याङ्कनं कुर्वन्तु।
\item
  \(\mathbf{F}(x,y,z) = \langle y,0,0 \rangle\) कृते \(x=1\),
  \(0 \leq y,z \leq 1\) इति विमानस्य माध्यमेन प्रवाहस्य गणनां कुर्वन्तु ।
\item
  यदि निमीलितपृष्ठद्वारा सदिशक्षेत्रस्य प्रवाहः शून्यः भवति तर्हि तस्य अर्थः किम् इति
  भौतिकरूपेण व्याख्यातव्यम्।
\end{enumerate}

\subsection{10.4 ग्रीनस्य
प्रमेयम्}\label{ux917ux930ux928ux938ux92f-ux92aux930ux92eux92fux92e}

ग्रीनस्य प्रमेयम् सदिशगणने मौलिकं परिणामं भवति यत् बन्दवक्रस्य परितः रेखा अभिन्नं तया
परिवेष्टितस्य प्रदेशस्य उपरि द्विगुण अभिन्नं प्रति संयोजयति स्टोक्सस्य प्रमेयस्य द्विविधं
संस्करणम् अस्ति ।

\subsubsection{\texorpdfstring{ग्रीन के प्रमेय का कथन\(C\) विमाने सकारात्मकरूपेण
उन्मुखं, सरलं, बन्दं वक्रं भवतु, \(R\) च तया परिवेष्टितः प्रदेशः भवतु । यदि
\(\mathbf{F}(x,y) = \langle P(x,y), Q(x,y) \rangle\) इत्यस्य \(R\) युक्ते
मुक्तक्षेत्रे निरन्तरं आंशिकव्युत्पन्नाः सन्ति,
तर्हि}{ग्रीन के प्रमेय का कथनC विमाने सकारात्मकरूपेण उन्मुखं, सरलं, बन्दं वक्रं भवतु, R च तया परिवेष्टितः प्रदेशः भवतु । यदि \textbackslash mathbf\{F\}(x,y) = \textbackslash langle P(x,y), Q(x,y) \textbackslash rangle इत्यस्य R युक्ते मुक्तक्षेत्रे निरन्तरं आंशिकव्युत्पन्नाः सन्ति, तर्हि}}\label{ux917ux930ux928-ux915-ux92aux930ux92eux92f-ux915-ux915ux925ux928c-ux935ux92eux928-ux938ux915ux930ux924ux92eux915ux930ux92aux923-ux909ux928ux92eux916-ux938ux930ux932-ux92cux928ux926-ux935ux915ux930-ux92dux935ux924-r-ux91a-ux924ux92f-ux92aux930ux935ux937ux91fux924-ux92aux930ux926ux936-ux92dux935ux924-ux92fux926-mathbffxy-langle-pxy-qxy-rangle-ux907ux924ux92fux938ux92f-r-ux92fux915ux924-ux92eux915ux924ux915ux937ux924ux930-ux928ux930ux928ux924ux930-ux906ux936ux915ux935ux92fux924ux92aux928ux928-ux938ux928ux924-ux924ux930ux939}

\[
\oint_C \mathbf{F} \cdot d\mathbf{r} = \oint_C P\,dx + Q\,dy = \iint_R \left( \frac{\partial Q}{\partial x} - \frac{\partial P}{\partial y} \right)\, dA.
\] इति

\subsubsection{व्याख्या}\label{ux935ux92fux916ux92f-2}

\begin{itemize}
\tightlist
\item
  \(C\) इत्यस्य परितः रेखा अभिन्नः सीमायाः सह सदिशक्षेत्रस्य परिसञ्चरणं मापयति ।
\item
  \(R\) इत्यस्य उपरि द्विगुणं अभिन्नं क्षेत्रस्य अन्तः क्षेत्रस्य कुल-कर्ल् (भ्रमणं) मापयति ।
\end{itemize}

\subsubsection{उदाहरणम् १ :
क्षेत्रसूत्रम्}\label{ux909ux926ux939ux930ux923ux92e-ux967-ux915ux937ux924ux930ux938ux924ux930ux92e}

यदि \(\mathbf{F} = \langle -y/2, x/2 \rangle\), तर्हि

\[
\frac{\partial Q}{\partial x} - \frac{\partial P}{\partial y} = 1.
\] इति

एवं ग्रीनस्य प्रमेयम् ददाति

\[
\text{Area}(R) = \iint_R 1\,dA = \oint_C \left(-\tfrac{y}{2}\,dx + \tfrac{x}{2}\,dy\right).
\] इति

एतेन रेखा अभिन्नस्य उपयोगेन क्षेत्रस्य गणनायाः उपायः प्राप्यते ।

\subsubsection{उदाहरणम् २ :
परिसञ्चरणम्}\label{ux909ux926ux939ux930ux923ux92e-ux968-ux92aux930ux938ux91eux91aux930ux923ux92e}

\(\mathbf{F}(x,y) = \langle -y, x \rangle\), \(C\) च एककवृत्तं भवतु ।

\begin{itemize}
\tightlist
\item
  \(P=-y, Q=x\)।
\item
  \(Q_x - P_y = 1 - (-1) = 2\)।
\item
  यूनिट् डिस्कस्य उपरि द्विगुणं अभिन्नम् : १.
\end{itemize}

\[
\iint_R 2\,dA = 2\pi (1^2) = 2\pi.
\] इति

अतः वृत्तस्य परितः परिसञ्चरणं \(2\pi\) अस्ति ।

\subsubsection{एतत् किमर्थं
महत्त्वपूर्णम्}\label{ux90fux924ux924-ux915ux92eux930ux925-ux92eux939ux924ux924ux935ux92aux930ux923ux92e-22}

\begin{itemize}
\tightlist
\item
  कठिनरेखा अभिन्नं द्विगुण अभिन्नं परिवर्तयति, अथवा तद्विपरीतम्।
\item
  स्थानीयगुणानां (curl) वैश्विकगुणानां (circulation) च मध्ये सेतुः प्रदाति ।
\item
  द्रवप्रवाहस्य, विद्युत्चुम्बकत्वस्य, समतलसदिशक्षेत्रस्य च कृते भौतिकशास्त्रे व्यापकरूपेण
  उपयुज्यते ।
\end{itemize}

\subsubsection{अभ्यास}\label{ux905ux92dux92fux938-37}

\begin{enumerate}
\def\labelenumi{\arabic{enumi}.}
\tightlist
\item
  दीर्घवृत्तस्य \(\frac{x^2}{a^2} + \frac{y^2}{b^2} = 1\) इत्यस्य अन्तः क्षेत्रस्य
  गणनां कर्तुं Green's Theorem इत्यस्य उपयोगं कुर्वन्तु ।
\item
  (0,0), (1,0), (1,1), (0,1) शिखरैः सह वर्गस्य सह
  \(\mathbf{F}(x,y) = \langle -y, x \rangle\) कृते Green's Theorem
  सत्यापयन्तु ।3. एककवृत्तस्य परितः
  \(\mathbf{F}(x,y) = \langle -y, x \rangle\) इत्यस्य परिसञ्चरणस्य गणनां कुरुत।
\item
  दर्शयतु यत् यदि \(\nabla \times \mathbf{F} = 0\), तर्हि कस्यापि बन्दवक्रस्य
  परितः \(\mathbf{F}\) इत्यस्य रेखा अभिन्नं शून्यं भवति ।
\item
  तत् दर्शयितुं Green's Theorem इत्यस्य उपयोगं कुर्वन्तु
\end{enumerate}

\[
\oint_C x\,dy = -\oint_C y\,dx
\] इति

कस्यापि निमीलितवक्रस्य \(C\) कृते।

\subsection{१०.५ स्टोक्सस्य
प्रमेयम्}\label{ux938ux91fux915ux938ux938ux92f-ux92aux930ux92eux92fux92e}

स्टोक्सस्य प्रमेयम् ग्रीनस्य प्रमेयस्य सामान्यीकरणं त्रिविमं करोति । एतत् पृष्ठस्य उपरि
सदिशक्षेत्रस्य कर्लस्य पृष्ठाभिन्नं तस्य पृष्ठस्य सीमां परितः सदिशक्षेत्रस्य रेखा अभिन्नं प्रति
सम्बद्धं करोति

\subsubsection{स्टोक्स के प्रमेय का
कथन}\label{ux938ux91fux915ux938-ux915-ux92aux930ux92eux92f-ux915-ux915ux925ux928}

\(S\) सीमावक्र \(C\) (सकारात्मक उन्मुख) सह उन्मुखं, चिकनी पृष्ठं भवतु । यदि
\(\mathbf{F}(x,y,z)\) निरन्तरं आंशिकव्युत्पन्नयुक्तं सदिशक्षेत्रं भवति तर्हि

\[
\iint_S (\nabla \times \mathbf{F}) \cdot d\mathbf{S} = \oint_C \mathbf{F} \cdot d\mathbf{r}.
\] इति

\begin{itemize}
\tightlist
\item
  वामपक्षः : पृष्ठद्वारा \(\mathbf{F}\) इत्यस्य कर्लस्य प्रवाहः ।
\item
  दक्षिणपक्षः सीमावक्रस्य सह \(\mathbf{F}\) इत्यस्य परिसञ्चरणम्।
\end{itemize}

\subsubsection{व्याख्या}\label{ux935ux92fux916ux92f-3}

\begin{itemize}
\tightlist
\item
  सीमायाः परितः रेखा अभिन्नः पृष्ठस्य अन्तः कुल ``भ्रमणस्य'' बराबरः भवति ।
\item
  Green's Theorem (यदा पृष्ठभागः विमाने भवति तदा विशेषः प्रकरणः) विस्तारयति ।
\end{itemize}

\subsubsection{उदाहरणम् १: विशेषप्रकरणरूपेण ग्रीनस्य
प्रमेयम्}\label{ux909ux926ux939ux930ux923ux92e-ux967-ux935ux936ux937ux92aux930ux915ux930ux923ux930ux92aux923-ux917ux930ux928ux938ux92f-ux92aux930ux92eux92fux92e}

यदि \(S\) \(xy\)-विमानस्य समतलप्रदेशः अस्ति तर्हि स्टोक्सस्य प्रमेयम् ग्रीनस्य
प्रमेयपर्यन्तं न्यूनीकरोति ।

\subsubsection{उदाहरणम् २ : गोलार्धे
परिसञ्चरणम्}\label{ux909ux926ux939ux930ux923ux92e-ux968-ux917ux932ux930ux927-ux92aux930ux938ux91eux91aux930ux923ux92e}

\(\mathbf{F}(x,y,z) = \langle -y, x, 0 \rangle\), \(S\) च त्रिज्या 1 इत्यस्य
ऊर्ध्वगोलार्धं भवतु ।

\begin{itemize}
\tightlist
\item
  सीमा \(C\): \(xy\)-विमान में इकाई वृत्त।
\item
  स्टोक्सस्य प्रमेयेन : १.
\end{itemize}

\[
\oint_C \mathbf{F}\cdot d\mathbf{r} = \iint_S (\nabla \times \mathbf{F})\cdot d\mathbf{S}.
\] इति

\begin{itemize}
\tightlist
\item
  कर्ल: \(\nabla \times \mathbf{F} = \langle 0,0,2 \rangle\)।
\item
  गोलार्धं प्रति सामान्यं बहिः सूचयति: \(\mathbf{n} = \langle 0,0,1 \rangle\)।
\item
  अतः अभिन्न = २.- गोलार्ध का क्षेत्रफल = \(2\pi (1^2)\)।
\end{itemize}

\[
\iint_S 2\, dS = 2 \cdot 2\pi = 4\pi.
\] इति

एवं विषुववृत्तं परितः परिसञ्चरणं \(4\pi\) भवति ।

\subsubsection{एतत् किमर्थं
महत्त्वपूर्णम्}\label{ux90fux924ux924-ux915ux92eux930ux925-ux92eux939ux924ux924ux935ux92aux930ux923ux92e-23}

\begin{itemize}
\tightlist
\item
  पृष्ठीय अभिन्नस्य रेखा अभिन्नस्य च मध्ये गहनं संयोजनं प्रदाति ।
\item
  सुविधाजनकपृष्ठानां चयनस्य अनुमतिं दत्त्वा गणनाः सरलीकरोति।
\item
  विद्युत्चुम्बकत्वे (Faraday's Law) द्रवगतिविज्ञाने च व्यापकरूपेण प्रयुक्तः ।
\end{itemize}

\subsubsection{अभ्यास}\label{ux905ux92dux92fux938-38}

\begin{enumerate}
\def\labelenumi{\arabic{enumi}.}
\tightlist
\item
  \(xy\)-विमानस्य यूनिट् डिस्कस्य उपरि
  \(\mathbf{F}(x,y,z) = \langle y, -x, 0 \rangle\) कृते Stokes' Theorem
  सत्यापयन्तु ।
\item
  \(\oint_C \mathbf{F}\cdot d\mathbf{r}\) गणनां कुरुत यत्र
  \(\mathbf{F}(x,y,z) = \langle z, 0, x \rangle\), तथा \(C\) शिखरयुक्तस्य
  त्रिकोणस्य सीमा (0,0,0), (1,0,0), (0,1,0) अस्ति ।
\item
  दर्शयतु यत् यदि \(\nabla \times \mathbf{F} = 0\) तर्हि कस्यापि निमीलितवक्रस्य
  परितः परिसञ्चरणं शून्यम् अस्ति।
\item
  \(z=0\) विमानस्य एककवर्गस्य सीमायाः परितः
  \(\mathbf{F}(x,y,z) = \langle -y, x, z \rangle\) इत्यस्य परिसञ्चरणस्य
  गणनाय Stokes' Theorem इत्यस्य प्रयोगं कुर्वन्तु ।
\item
  स्टोक्सस्य प्रमेयम् ग्रीनस्य प्रमेयस्य सामान्यीकरणं कथं करोति इति व्याख्यातव्यम्।
\end{enumerate}

\subsection{10.6 विचलन
प्रमेय}\label{ux935ux91aux932ux928-ux92aux930ux92eux92f}

विचलनप्रमेयः (Gauss's Theorem इति अपि उच्यते) सदिशक्षेत्रस्य प्रवाहं
निमीलितपृष्ठद्वारा पृष्ठस्य अन्तः क्षेत्रस्य विचलनस्य त्रिगुणाभिन्नेन सह सम्बद्धं करोति

\subsubsection{विचलन प्रमेय का
कथन}\label{ux935ux91aux932ux928-ux92aux930ux92eux92f-ux915-ux915ux925ux928}

\(E\) \(\mathbb{R}^3\) मध्ये सीमापृष्ठेन \(S\) (बहिः उन्मुख) सह ठोसः प्रदेशः भवतु
। यदि \(\mathbf{F}(x,y,z)\) \(E\) इत्यत्र निरन्तरं आंशिकव्युत्पन्नयुक्तं सदिशक्षेत्रं
भवति तर्हि

\[
\iint_S \mathbf{F} \cdot d\mathbf{S} = \iiint_E (\nabla \cdot \mathbf{F}) \, dV.
\]

\begin{itemize}
\tightlist
\item
  वामपक्षः : बन्दपृष्ठ \(S\) पार \(\mathbf{F}\) का प्रवाह।
\item
  दक्षिणपक्षः : क्षेत्रस्य अन्तः विचलनस्य त्रिगुणः अभिन्नः ।
\end{itemize}

\subsubsection{विचलन}\label{ux935ux91aux932ux928}

सदिशक्षेत्रस्य \(\mathbf{F}(x,y,z) = \langle P, Q, R \rangle\) इत्यस्य विचलनं
भवति

\[ इति\nabla \cdot \mathbf{F} = \frac{\ आंशिक P}{\आंशिक x} + \frac{\आंशिक Q}{\आंशिक y} + \frac{\आंशिक R}{\आंशिक z}।
\]

It measures the ``net outflow'' per unit volume at each point.

\subsubsection{Example 1: Flux of a Radial
Field}\label{example-1-flux-of-a-radial-field}

Let \(\mathbf{F}(x,y,z) = \langle x, y, z \rangle\), and let \(E\) be
the unit ball \(x^2+y^2+z^2 \leq 1\).

\begin{itemize}
\tightlist
\item
  Divergence: \(\nabla \cdot \mathbf{F} = 1+1+1 = 3\).
\item
  Volume of unit ball: \(\tfrac{4}{3}\pi\). So
\end{itemize}

\[ इति
\iiint_E (\nabla \cdot \mathbf{F})\, dV = 3 \cdot \tfrac{4}{3}\pi = 4\pi.
\] इति

एवं एककगोलस्य पारं प्रवाहः \(4\pi\) भवति ।

\subsubsection{उदाहरणम् २ :
नित्यक्षेत्रम्}\label{ux909ux926ux939ux930ux923ux92e-ux968-ux928ux924ux92fux915ux937ux924ux930ux92e}

\(\mathbf{F}(x,y,z) = \langle 1, 0, 0 \rangle\) इति ।

\begin{itemize}
\tightlist
\item
  विचलन: \(\nabla \cdot \mathbf{F} = 0\)।
\item
  अतः कस्यापि बन्दपृष्ठस्य माध्यमेन प्रवाहः शून्यः भवति, अन्तर्ज्ञानेन सह सङ्गतः
  (शुद्धबहिःप्रवाहः नास्ति)।
\end{itemize}

\subsubsection{एतत् किमर्थं
महत्त्वपूर्णम्}\label{ux90fux924ux924-ux915ux92eux930ux925-ux92eux939ux924ux924ux935ux92aux930ux923ux92e-24}

\begin{itemize}
\item
  पृष्ठीय अभिन्नं सरलतर आयतन अभिन्नं परिवर्तयति ।
\item
  भौतिकशास्त्रे प्रयुक्तः : विद्युत्चुम्बकत्वम्, द्रवप्रवाहः, तापसञ्चारः च इति विषये गाउस्
  नियमः ।
\item
  एकीकरणरूपरेखां सम्पूर्णं करोति : १.

  \begin{itemize}
  \tightlist
  \item
    ग्रीनस्य प्रमेयम् (2D curl ↔ circulation)
  \item
    Stokes' Theorem (3D curl ↔ पृष्ठेषु परिसञ्चरणम्)
  \item
    विचलन प्रमेय (3D विचलन ↔ बन्द पृष्ठों पर प्रवाह)
  \end{itemize}
\end{itemize}

\subsubsection{अभ्यास}\label{ux905ux92dux92fux938-39}

\begin{enumerate}
\def\labelenumi{\arabic{enumi}.}
\tightlist
\item
  त्रिज्या \(R\) इत्यस्य गोलस्य पृष्ठभागे
  \(\mathbf{F}(x,y,z) = \langle x,y,z \rangle\) इत्यस्य प्रवाहस्य गणनां कर्तुं
  Divergence Theorem इत्यस्य उपयोगं कुर्वन्तु ।
\item
  एककघन \([0,1]^3\) इत्यत्र
  \(\mathbf{F}(x,y,z) = \langle y, z, x \rangle\) इत्यस्य विचलनप्रमेयस्य
  सत्यापनम् ।
\item
  दर्शयतु यत् यदि \(\nabla \cdot \mathbf{F} = 0\) तर्हि कस्यापि बन्दपृष्ठस्य
  माध्यमेन कुलप्रवाहः शून्यः भवति ।
\item
  एककगोलस्य माध्यमेन \(\mathbf{F}(x,y,z) = \langle x^2, y^2, z^2 \rangle\)
  इत्यस्य प्रवाहस्य गणनां कुर्वन्तु।
\item
  विचलनप्रमेयः गणितस्य एकविमीयमूलप्रमेयस्य सामान्यीकरणं कथं करोति इति व्याख्यातव्यम्।
\end{enumerate}

\section{चतुर्थ भाग। अनन्त
प्रक्रियाएँ}\label{ux91aux924ux930ux925-ux92dux917-ux905ux928ux928ux924-ux92aux930ux915ux930ux92fux90f}

\section{अध्याय 11. अनुक्रमाः अभिसरणं च\#\# 11.1 परिभाषा उदाहरणानि
च}\label{ux905ux927ux92fux92f-11.-ux905ux928ux915ux930ux92e-ux905ux92dux938ux930ux923-ux91a-11.1-ux92aux930ux92dux937-ux909ux926ux939ux930ux923ux928-ux91a}

क्रमः संख्यानां क्रमबद्धसूची भवति, प्रायः इव लिख्यते

\[
a_1, a_2, a_3, \dots
\] इति

अथवा अधिकसामान्यतया \((a_n)_{n=1}^\infty\)। प्रत्येकं \(a_n\) क्रमस्य nth पदं
कथ्यते ।

\subsubsection{एकं अनुक्रमं
परिभाषयति}\label{ux90fux915-ux905ux928ux915ux930ux92e-ux92aux930ux92dux937ux92fux924}

क्रमः द्विधा परिभाषितुं शक्यते ।

\begin{enumerate}
\def\labelenumi{\arabic{enumi}.}
\item
  स्पष्टसूत्रम् -- नमपदस्य प्रत्यक्षं नियमं ददाति।

  \begin{itemize}
  \item
    उदाहरणम् : \(a_n = \frac{1}{n}\) क्रमं परिभाषयति

    \[
    1, \tfrac{1}{2}, \tfrac{1}{3}, \tfrac{1}{4}, \dots
    \] इति
  \end{itemize}
\item
  पुनरावर्तनीयपरिभाषा -- पूर्वपदानां उपयोगेन पदानाम् परिभाषां करोति।

  \begin{itemize}
  \item
    उदाहरणम् : फिबोनाची अनुक्रम : १.

    \[
    a_1 = 1, \quad a_2 = 1, \quad a_{n} = a_{n-1} + a_{n-2} \quad (n \geq 3).
    \] इति
  \end{itemize}
\end{enumerate}

\subsubsection{अनुक्रमस्य
उदाहरणानि}\label{ux905ux928ux915ux930ux92eux938ux92f-ux909ux926ux939ux930ux923ux928}

\begin{enumerate}
\def\labelenumi{\arabic{enumi}.}
\item
  अंकगणितीय क्रमः : १.

  \[
  a_n = a_1 + (n-1)d.
  \] इति

  उदाहरणम् : \(a_n = 2n+1\) → विषमसङ्ख्यानां क्रमः ।
\item
  ज्यामितीयक्रमः : १.

  \[
  a_n = a_1 r^{n-1}.
  \] इति

  उदाहरणम् : \(a_n = 2^n\) → 2 इत्यस्य शक्तिः।
\item
  हार्मोनिक क्रमः : १.

  \[
  a_n = \frac{1}{n}.
  \] इति
\item
  वैकल्पिकक्रमः : १.

  \[
  a_n = (-1)^n.
  \] इति
\end{enumerate}

\subsubsection{गणितस्य
क्रमाः}\label{ux917ux923ux924ux938ux92f-ux915ux930ux92e}

अनन्तप्रक्रियाणां आधारः अनुक्रमाः सन्ति : १.

\begin{itemize}
\tightlist
\item
  क्रमाणां सीमाः → अभिसरणं परिभाषयन्तु।
\item
  श्रृङ्खला → अनुक्रमेभ्यः निर्मिताः अनन्तयोगाः ।
\item
  अनुक्रमैः श्रृङ्खलाभिः च अनुमानिताः कार्याः।
\end{itemize}

\subsubsection{एतत् किमर्थं
महत्त्वपूर्णम्}\label{ux90fux924ux924-ux915ux92eux930ux925-ux92eux939ux924ux924ux935ux92aux930ux923ux92e-25}

\begin{itemize}
\tightlist
\item
  अनुक्रमाः अनन्तश्रृङ्खलानां सन्निकर्षाणां च निर्माणखण्डान् प्रदास्यन्ति ।
\item
  ते अस्मान् ``अनन्तं समीपं गच्छन्ति'' अभिसरणं च कठोररूपेण परिभाषितुं शक्नुवन्ति।
\item
  अनेकाः महत्त्वपूर्णाः कार्याणि (घातीयः, त्रिकोणमितीयः) अनुक्रमैः श्रृङ्खलाभिः च
  व्यक्तं कर्तुं शक्यते ।
\end{itemize}

\subsubsection{अभ्यास}\label{ux905ux92dux92fux938-40}

\begin{enumerate}
\def\labelenumi{\arabic{enumi}.}
\tightlist
\item
  \(a_n = \frac{n}{n+1}\) क्रमस्य प्रथमानि पञ्च पदानि लिखत।
\item
  निर्धारयतु यत् \(a_n = (-1)^n n\) सीमाबद्धा अस्ति वा।
\item
  \(2,4,8,16,\dots\) क्रमस्य कृते पुनरावर्तनीयपरिभाषा ददातु।
\item
  \(a_1=3\) तथा \(d=5\) सहित अंकगणितीय अनुक्रम का 10वें पद ज्ञात कीजिए।5.
  \(a_1=1\), \(a_{n+1}=2a_n\) इत्यनेन परिभाषितस्य क्रमस्य कृते स्पष्टं सूत्रं लिखत।
\end{enumerate}

\subsection{11.2 एकस्वर एवं सीमाबद्ध
अनुक्रम}\label{ux90fux915ux938ux935ux930-ux90fux935-ux938ux92eux92cux926ux927-ux905ux928ux915ux930ux92e}

क्रमः अभिसरति वा इति ज्ञातुं अस्माभिः तस्य व्यवहारस्य अध्ययनं करणीयम् यत् सः वर्धते,
न्यूनीभवति, सीमान्तरे तिष्ठति, अथवा सीमां विना वर्धते? एकरसता, सीमा च इति द्वौ
महत्त्वपूर्णौ अवधारणाौ ।

\subsubsection{एकरस
अनुक्रम}\label{ux90fux915ux930ux938-ux905ux928ux915ux930ux92e}

\((a_n)\) क्रमः एकस्वरः इति उच्यते यदि सः सर्वदा वर्धमानः अथवा सर्वदा न्यूनः भवति
।

\begin{itemize}
\item
  एकरसः वर्धमानः : १.

  \[
  a_{n+1} \geq a_n \quad \text{for all } n.
  \] इति
\item
  एकरसः न्यूनता : १.

  \[
  a_{n+1} \leq a_n \quad \text{for all } n.
  \] इति
\end{itemize}

उदाहरणानि : १.

\begin{enumerate}
\def\labelenumi{\arabic{enumi}.}
\tightlist
\item
  \(a_n = n\) एकरसः वर्धमानः अस्ति।
\item
  \(a_n = \frac{1}{n}\) एकस्वरस्य न्यूनता भवति।
\end{enumerate}

\subsubsection{सीमाबद्ध
अनुक्रम}\label{ux938ux92eux92cux926ux927-ux905ux928ux915ux930ux92e}

यदि सर्वेषां \(n\) कृते \(a_n \leq M\) इति संख्या अस्ति चेत् उपरि क्रमः सीमाबद्धः
भवति । यदि सर्वेषां \(n\) कृते \(a_n \geq m\) इति \(m\) अस्ति तर्हि अधः
सीमाबद्धम् अस्ति ।

यदि उभयस्थितिः धारयति तर्हि क्रमः सीमाबद्धः भवति ।

उदाहरणानि : १.

\begin{enumerate}
\def\labelenumi{\arabic{enumi}.}
\tightlist
\item
  \(a_n = \frac{1}{n}\) 0 तः 1 पर्यन्तं सीमां प्राप्नोति ।
\item
  \(a_n = (-1)^n\) -1 तथा 1 मध्ये सीमा अस्ति।
\item
  \(a_n = n\) इति सीमां न भवति।
\end{enumerate}

\subsubsection{एकरस अभिसरण
प्रमेय}\label{ux90fux915ux930ux938-ux905ux92dux938ux930ux923-ux92aux930ux92eux92f}

विश्लेषणे एकः मौलिकः परिणामः : १.

\begin{itemize}
\tightlist
\item
  प्रत्येकं एकरसवर्धनक्रमः यः उपरि सीमाबद्धः भवति सः अभिसरणं करोति।
\item
  प्रत्येकं एकरसः क्षीणक्रमः यः अधः सीमाबद्धः भवति सः अभिसरणं करोति ।
\end{itemize}

अयं प्रमेयः सीमां स्पष्टतया न अन्विष्य अभिसरणस्य गारण्टीं ददाति ।

\subsubsection{उदाहरण}\label{ux909ux926ux939ux930ux923-1}

\begin{enumerate}
\def\labelenumi{\arabic{enumi}.}
\item
  अनुक्रमः \(a_n = 1 - \frac{1}{n}\)।

  \begin{itemize}
  \tightlist
  \item
    वर्धमानः: यतः \(a_{n+1} - a_n = \frac{1}{n} - \frac{1}{n+1} > 0\)।
  \item
    उपरि 1 द्वारा सीमाबद्धः।
  \item
    अतः अभिसरति ।
  \item
    सीमा: \(\lim_{n\to\infty} a_n = 1\)।
  \end{itemize}
\end{enumerate}

\subsubsection{एतत् किमर्थं
महत्त्वपूर्णम्}\label{ux90fux924ux924-ux915ux92eux930ux925-ux92eux939ux924ux924ux935ux92aux930ux923ux92e-26}

\begin{itemize}
\tightlist
\item
  एकरसता सीमा च अभिसरणस्य शीघ्रपरीक्षाः ददति ।
\item
  प्रमाणेषु सीमानिर्माणे च कठोरतापूर्वकं ते अत्यावश्यकाः सन्ति।
\item
  एते विचाराः स्वाभाविकतया कार्याणि श्रृङ्खलाश्च यावत् विस्तारन्ते।\#\#\# अभ्यास
\end{itemize}

\begin{enumerate}
\def\labelenumi{\arabic{enumi}.}
\tightlist
\item
  \(a_n = \frac{n}{n+1}\) एकरसः सीमायुक्तः च अस्ति वा इति निर्धारयतु।
\item
  दर्शयतु यत् \(a_n = \sqrt{n}\) एकरसः वर्धमानः अस्ति किन्तु सीमाबद्धः नास्ति।
\item
  \(a_n = 2 - \frac{1}{n}\) अभिसरणं करोति इति सिद्धं कुरुत, तस्य सीमां च
  ज्ञातव्यम् ।
\item
  एकस्वरस्य सीमाबद्धस्य क्रमस्य उदाहरणं ददातु।
\item
  एकरस-अभिसरण-प्रमेयं \(a_n = \ln\!\big(1+\frac{1}{n}\big)\) इत्यत्र प्रयोजयन्तु
  ।
\end{enumerate}

\subsection{11.3 अनुक्रमस्य
सीमाः}\label{ux905ux928ux915ux930ux92eux938ux92f-ux938ux92e}

क्रमस्य विषये केन्द्रीयः प्रश्नः अस्ति यत् \(n\) वर्धमानेन तस्य पदाः एकस्य मूल्यस्य समीपं
गच्छन्ति वा इति । अनेन क्रमस्य सीमायाः अवधारणा भवति ।

\subsubsection{परिभाषा}\label{ux92aux930ux92dux937-12}

क्रमस्य \((a_n)\) सीमा \(L\) भवति यदि, प्रत्येकं \(\varepsilon > 0\) कृते,
पूर्णाङ्कः \(N\) अस्ति यत्\ldots{}

\[
|a_n - L| < \varepsilon \quad \text{whenever } n > N.
\] इति

ततः वयं लिखामः

\[
\lim_{n\to\infty} a_n = L.
\] इति

यदि तादृशः \(L\) नास्ति तर्हि क्रमः विचलति ।

\subsubsection{अंतरचेतना}\label{ux905ux924ux930ux91aux924ux928}

\begin{itemize}
\tightlist
\item
  क्रमस्य पदाः मनमाना \(L\) इत्यस्य समीपं गच्छन्ति यतः \(n\) बृहत् भवति ।
\item
  कस्यचित् सूचकाङ्कस्य \(N\) इत्यस्मात् परं, सर्वे पदाः \(L\) इत्यस्य परितः एकस्य
  लघुपट्टिकायाः \hspace{0pt}\hspace{0pt}अन्तः एव तिष्ठन्ति ।
\end{itemize}

\subsubsection{उदाहरणम्}\label{ux909ux926ux939ux930ux923ux92e-24}

\begin{enumerate}
\def\labelenumi{\arabic{enumi}.}
\item
  \(a_n = \frac{1}{n}\) इति । यथा यथा \(n\) वर्धते तथा तथा पदाः 0 प्रति
  संकुचन्ति ।

  \[
  \lim_{n\to\infty} \frac{1}{n} = 0.
  \] इति
\item
  \(a_n = (-1)^n\) इति । पदाः -१ तथा १ मध्ये क्रमेण भवन्ति, अतः एकः सीमा
  नास्ति । क्रमः विचलति।
\item
  \(a_n = \frac{n}{n+1}\) इति । यथा \(n \to \infty\), गणकः हरः च प्रायः
  समानौ भवतः, अतः

  \[
  \lim_{n\to\infty} \frac{n}{n+1} = 1.
  \] इति
\end{enumerate}

\subsubsection{सीमाओं के गुण}\label{ux938ux92eux913-ux915-ux917ux923}

यदि \(\lim a_n = A\) तथा \(\lim b_n = B\):

\begin{itemize}
\item
  \(\lim (a_n+b_n) = A+B\)।
\item
  \(\lim (a_n b_n) = AB\)।
\item
  \(\lim (c a_n) = cA\) नित्य \(c\) कृते।
\item
  यदि \(b_n \neq 0\) तथा \(B \neq 0\), तर्हि

  \[
  \lim \frac{a_n}{b_n} = \frac{A}{B}.
  \] इति
\end{itemize}

\subsubsection{प्रमेय : निचोड़
सिद्धान्त}\label{ux92aux930ux92eux92f-ux928ux91aux921-ux938ux926ux927ux928ux924}

यदि \(a_n \leq b_n \leq c_n\) सर्वेषां बृहत् \(n\) कृते, तथा च

\(N\) इति

तदा

\[ इति\lim_{n\to\infty} b_n = ल.
\]

Example:

\[ इति
a_n = -\tfrac{1}{n}, \quad b_n = \tfrac{\sin n}{n}, \quad c_n = \tfrac{1}{n}।
\]

Since \(-\tfrac{1}{n} \leq \tfrac{\sin n}{n} \leq \tfrac{1}{n}\) and
both bounding sequences go to 0,

\[ इति
\lim_{n\to\infty} \frac{\sin n}{न} = 0.
\]

\subsubsection{Why This Matters}\label{why-this-matters-1}

\begin{itemize}
\tightlist
\item
  Limits make rigorous the idea of sequences ``approaching'' a value.
\item
  Convergence of sequences underpins infinite series and continuity.
\item
  These concepts are essential in defining real numbers via limits.
\end{itemize}

\subsubsection{Exercises}\label{exercises-4}

\begin{enumerate}
\def\labelenumi{\arabic{enumi}.}
\tightlist
\item
  Find \(\lim_{n\to\infty} \frac{2n+1}{3n+4}\).
\item
  Determine if \(a_n = \sqrt{n+1} - \sqrt{n}\) converges.
\item
  Does \(a_n = \cos n\) converge? Why or why not?
\item
  Use the Squeeze Principle to show
  \(\lim_{n\to\infty} \frac{\sin n}{n} = 0\).
\item
  Prove that if \(\lim a_n = L\), then \(\lim |a_n| = |L|\).
\end{enumerate}

\section{Chapter 12. Infinite series}\label{chapter-12.-infinite-series}

\subsection{12.1 Series and Convergence}\label{series-and-convergence}

A series is the sum of the terms of a sequence. Instead of just listing
numbers, we add them together and study whether the infinite sum
approaches a finite value.

\subsubsection{Definition}\label{definition}

Given a sequence \((a_n)\), the corresponding series is

\[
\sum_{n=1}^\infty a_n = क_1 + क_2 + क_3 + \बिन्दवः
\]

We define the nth partial sum as

\[ इति
S_n = \sum_{k=1}^n a_k.
\]

If the sequence \((S_n)\) converges to a finite limit \(S\), then the
series converges and

\[ इति
\sum_{n=1}^\infty a_n = स.
\]

If \((S_n)\) diverges, then the series diverges.

\subsubsection{Examples}\label{examples-2}

\begin{enumerate}
\def\labelenumi{\arabic{enumi}.}
\tightlist
\item
  Geometric series
\end{enumerate}

\[ इति
\sum_{n=0}^\infty ar^n = \frac{a}{1-r}, \quad |r| < १.
\]

Example:

\[ इति
1 + \tfrac{1}{2} + \tfrac{1}{4} + \tfrac{1}{8} + \बिन्दवः = 2.
\]

\begin{enumerate}
\def\labelenumi{\arabic{enumi}.}
\setcounter{enumi}{1}
\tightlist
\item
  Harmonic series
\end{enumerate}

\[ इति
\sum_{n=1}^\infty \frac{1}{न}।
\]

This series diverges, even though the terms go to 0.

\begin{enumerate}
\def\labelenumi{\arabic{enumi}.}
\setcounter{enumi}{2}
\tightlist
\item
  p-series
\end{enumerate}

\[ इति
\sum_{n=1}^\infty \frac{1}{न^प}।
\] इति

\begin{itemize}
\tightlist
\item
  अभिसरणं करोति यदि \(p > 1\)।
\item
  विचलति यदि \(p \leq 1\)।\#\#\# अभिसरणार्थं आवश्यकी शर्तः
\end{itemize}

यदि \(\sum a_n\) अभिसरति तर्हि अवश्यमेव

\[
\lim_{n\to\infty} a_n = 0.
\] इति

यदि \(\lim a_n \neq 0\), श्रृङ्खला विचलति। परन्तु विपरीतम् सत्यं नास्ति :
\(\lim a_n = 0\) अभिसरणस्य गारण्टीं न ददाति (उदा., हार्मोनिक श्रृङ्खला)।

\subsubsection{एतत् किमर्थं
महत्त्वपूर्णम्}\label{ux90fux924ux924-ux915ux92eux930ux925-ux92eux939ux924ux924ux935ux92aux930ux923ux92e-27}

\begin{itemize}
\tightlist
\item
  श्रृङ्खला अनन्तप्रक्रियासु परिमितं परिवर्तनं विस्तारयति।
\item
  अभिसरणश्रृङ्खलानां उपयोगः कार्याणां अनुमानं कर्तुं, क्षेत्राणां गणनां कर्तुं,
  भौतिकप्रक्रियाणां प्रतिरूपणार्थं च भवति ।
\item
  श्रृङ्खलायाः अध्ययनेन शक्तिशालिनः अभिसरणपरीक्षाः भवन्ति ।
\end{itemize}

\subsubsection{अभ्यास}\label{ux905ux92dux92fux938-41}

\begin{enumerate}
\def\labelenumi{\arabic{enumi}.}
\tightlist
\item
  \(\sum_{n=1}^\infty \frac{2}{3^n}\) अभिसरणं करोति वा इति निर्धारयन्तु, तस्य
  योगं च ज्ञातव्यम् ।
\item
  \(\sum_{n=1}^\infty \frac{1}{n^2}\) अभिसरणं करोति इति दर्शयतु।
\item
  \(\sum_{n=1}^\infty \frac{1}{\sqrt{n}}\) अभिसरणं करोति वा ?
\item
  \(\sum_{n=1}^\infty \frac{1}{2^n}\) श्रृङ्खलायाः प्रथमचतुर्णां आंशिकयोगाः
  लिखत।
\item
  \(\lim a_n = 0\) आवश्यकं किन्तु अभिसरणार्थं किमर्थं पर्याप्तं न इति व्याख्यातव्यम्।
\end{enumerate}

\subsection{12.2
अभिसरणपरीक्षाः}\label{ux905ux92dux938ux930ux923ux92aux930ux915ux937}

यतः बहवः श्रृङ्खलाः प्रत्यक्षतया योगं कर्तुं न शक्यन्ते, तस्मात् गणितज्ञैः श्रृङ्खला अभिसरणं
करोति वा विचलितं वा इति निर्णयार्थं परीक्षणं विकसितवन्तः । एतानि परीक्षणानि
अनन्तयोगविश्लेषणस्य साधनानि सन्ति ।

\subsubsection{1. विचलनार्थं nth-Term
Test}\label{ux935ux91aux932ux928ux930ux925-nth-term-test}

यदि

\[
\lim_{n\to\infty} a_n \neq 0 \quad \text{or does not exist},
\] इति

तदा

\[
\sum a_n
\] इति

विचलति ।

यदि \(\lim a_n = 0\) तर्हि परीक्षणं निष्कर्षहीनं भवति।

\subsubsection{2. तुलना
परीक्षा}\label{ux924ux932ux928-ux92aux930ux915ux937}

सर्वेषां \(n\) कृते \(0 \leq a_n \leq b_n\) इति कल्पयतु ।

\begin{itemize}
\tightlist
\item
  यदि \(\sum b_n\) अभिसरणं करोति तर्हि \(\sum a_n\) अपि अभिसरति ।
\item
  यदि \(\sum a_n\) विचलति तर्हि \(\sum b_n\) अपि विचलति।
\end{itemize}

\subsubsection{3. सीमा तुलना
परीक्षा}\label{ux938ux92e-ux924ux932ux928-ux92aux930ux915ux937}

यदि \(a_n, b_n > 0\) तथा

\[
\lim_{n\to\infty} \frac{a_n}{b_n} = c,
\] इति

यत्र \(0 < c < \infty\), ततः \(\sum a_n\) तथा \(\sum b_n\) उभयम् अपि
अभिसृत्य वा उभयम् अपि विचलितं भवति ।

\subsubsection{4. अनुपात
परीक्षण}\label{ux905ux928ux92aux924-ux92aux930ux915ux937ux923}

\(\sum a_n\) कृते गणनां कुर्वन्तु

\[
L = \lim_{n\to\infty} \left| \frac{a_{n+1}}{a_n} \right|.
\] इति- यदि \(L < 1\), श्रृङ्खला सर्वथा अभिसरणं करोति। - यदि \(L > 1\) अथवा
\(L = \infty\), श्रृङ्खला विचलति। - यदि \(L = 1\) तर्हि परीक्षणं निष्कर्षहीनं
भवति।

\subsubsection{5. मूलपरीक्षा}\label{ux92eux932ux92aux930ux915ux937}

\(\sum a_n\) कृते गणनां कुर्वन्तु

\[
L = \lim_{n\to\infty} \sqrt[n]{|a_n|}.
\] इति

\begin{itemize}
\tightlist
\item
  यदि \(L < 1\), श्रृङ्खला सर्वथा अभिसरणं करोति।
\item
  यदि \(L > 1\), श्रृङ्खला विचलति।
\item
  यदि \(L = 1\), परीक्षणम् अनिर्णयात्मकम् अस्ति।
\end{itemize}

\subsubsection{6. वैकल्पिकश्रृङ्खलापरीक्षा (Leibniz's Test)
.}\label{ux935ux915ux932ux92aux915ux936ux930ux919ux916ux932ux92aux930ux915ux937-leibnizs-test-.}

रूपस्य श्रृङ्खलायाम्

\[
\sum (-1)^n b_n \quad \text{or} \quad \sum (-1)^{n+1} b_n,
\] इति

यदि

\begin{enumerate}
\def\labelenumi{\arabic{enumi}.}
\tightlist
\item
  \(b_{n+1} \leq b_n\) (ह्रासमानः), तथा
\item
  \(\lim_{n\to\infty} b_n = 0\), ९.
\end{enumerate}

ततः श्रृङ्खला अभिसरति।

\subsubsection{उदाहरणम्}\label{ux909ux926ux939ux930ux923ux92e-25}

\begin{enumerate}
\def\labelenumi{\arabic{enumi}.}
\tightlist
\item
  \(\sum \frac{1}{n^2}\): तुलनापरीक्षा → अभिसरणं करोति।
\item
  \(\sum \frac{1}{n}\): हार्मोनिक श्रृङ्खला → विचलति।
\item
  \(\sum \frac{(-1)^n}{n}\): वैकल्पिक श्रृङ्खला परीक्षण → अभिसरण।
\item
  \(\sum \frac{n!}{n^n}\): अनुपातपरीक्षा → अभिसरणं करोति।
\item
  \(\sum \frac{2^n}{n}\): मूलपरीक्षा → विचलति।
\end{enumerate}

\subsubsection{एतत् किमर्थं
महत्त्वपूर्णम्}\label{ux90fux924ux924-ux915ux92eux930ux925-ux92eux939ux924ux924ux935ux92aux930ux923ux92e-28}

\begin{itemize}
\tightlist
\item
  अभिसरणपरीक्षाः स्पष्टयोगानाम् आवश्यकतां विना श्रृङ्खलानां वर्गीकरणं कुर्मः ।
\item
  ते गणितस्य अनन्तप्रक्रियाणां निबन्धनस्य व्यवस्थितमार्गान् प्रददति।
\item
  ते शक्तिश्रृङ्खला, फूरियरश्रृङ्खला इत्यादीनां परवर्तीनां विषयाणां कृते महत्त्वपूर्णाः सन्ति
  ।
\end{itemize}

\subsubsection{अभ्यास}\label{ux905ux92dux92fux938-42}

\begin{enumerate}
\def\labelenumi{\arabic{enumi}.}
\tightlist
\item
  \(\sum \frac{1}{n^3}\) इत्यस्य अभिसरणस्य परीक्षणम्।
\item
  \(\sum \frac{3^n}{n!}\) कृते अनुपातपरीक्षायाः उपयोगं कुर्वन्तु ।
\item
  \(\sum \left(\frac{1}{2}\right)^n\) इत्यत्र मूलपरीक्षां प्रयोजयन्तु।
\item
  \(\sum \frac{(-1)^n}{\sqrt{n}}\) इत्यस्य अभिसरणं निर्धारयन्तु।
\item
  \(\sum \frac{1}{n^2+1}\) परीक्षणार्थं \(\frac{1}{n^2}\) इत्यनेन सह
  सीमातुलनापरीक्षायाः उपयोगं कुर्वन्तु ।
\end{enumerate}

\subsection{12.3 निरपेक्ष बनाम सशर्त
अभिसरण}\label{ux928ux930ux92aux915ux937-ux92cux928ux92e-ux938ux936ux930ux924-ux905ux92dux938ux930ux923}

चिह्नानां पर्यायेण सर्वाणि श्रृङ्खलानि समानरूपेण न वर्तन्ते । एतत् नियन्त्रयितुं वयं
निरपेक्षसमागमस्य सशर्तसमागमस्य च भेदं कुर्मः ।

\subsubsection{निरपेक्ष
अभिसरण}\label{ux928ux930ux92aux915ux937-ux905ux92dux938ux930ux923}

एकः श्रृङ्खला \(\sum a_n\) सर्वथा अभिसरणीयः भवति यदि

\[
\sum |a_n|
\] इति

अभिसरति ।प्रमेयम् : यदि श्रृङ्खला निरपेक्षतया अभिसरणं करोति तर्हि सा अपि अभिसरणं
करोति ।

उदाहरण:

\[
\sum \frac{(-1)^n}{n^2}.
\] इति

अत्र \(\sum \left|\frac{(-1)^n}{n^2}\right| = \sum \frac{1}{n^2}\)
अभिसरति (p-श्रृङ्खला, \(p=2\)) । अतः श्रृङ्खला सर्वथा अभिसरणात्मका अस्ति।

\subsubsection{सशर्त
अभिसरण}\label{ux938ux936ux930ux924-ux905ux92dux938ux930ux923}

\(\sum a_n\) श्रृङ्खला यदि अभिसरणं करोति तर्हि सशर्तरूपेण अभिसरणं भवति, परन्तु
सर्वथा न ।

उदाहरण:

\[
\sum \frac{(-1)^n}{n}.
\] इति

\begin{itemize}
\tightlist
\item
  वैकल्पिक श्रृङ्खला परीक्षण → अभिसरण।
\item
  परन्तु \(\sum \left|\frac{(-1)^n}{n}\right| = \sum \frac{1}{n}\) विचलति
  (हारमोनिक श्रृङ्खला)। अतः श्रृङ्खला सशर्तरूपेण अभिसरणीयः अस्ति।
\end{itemize}

\subsubsection{पुनर्व्यवस्था
प्रमेय}\label{ux92aux928ux930ux935ux92fux935ux938ux925-ux92aux930ux92eux92f}

सशर्तरूपेण अभिसरणश्रृङ्खलानां कृते पदानाम् पुनर्व्यवस्थापनेन योगः परिवर्तयितुं शक्यते - अपि
च भिन्नमूल्ये विचलनं वा अभिसरणं वा कर्तुं शक्यते ।

एतत् आश्चर्यजनकं परिणामं सशर्तसमागमस्य सुकुमारं स्वरूपं दर्शयति ।

\subsubsection{एतत् किमर्थं
महत्त्वपूर्णम्}\label{ux90fux924ux924-ux915ux92eux930ux925-ux92eux939ux924ux924ux935ux92aux930ux923ux92e-29}

\begin{itemize}
\tightlist
\item
  निरपेक्षं अभिसरणं अधिकं प्रबलं भवति, स्थिरतायाः गारण्टीं च ददाति।
\item
  सशर्त-अभिसरणम् अनन्त-योगेषु क्रमस्य महत्त्वं प्रकाशयति ।
\item
  व्यवहारे सम्मुखीकृताः बहवः वैकल्पिकश्रृङ्खलाः केवलं सशर्तरूपेण अभिसरणं कुर्वन्ति ।
\end{itemize}

\subsubsection{अभ्यास}\label{ux905ux92dux92fux938-43}

\begin{enumerate}
\def\labelenumi{\arabic{enumi}.}
\tightlist
\item
  दर्शयतु यत् \(\sum \frac{(-1)^n}{n^3}\) सर्वथा अभिसरणं करोति।
\item
  \(\sum \frac{(-1)^n}{n}\) सशर्तरूपेण अभिसरणम् इति दर्शयतु।
\item
  निरपेक्षस्य सशर्तस्य च अभिसरणस्य कृते \(\sum \frac{(-1)^n}{\sqrt{n}}\) परीक्षणं
  कुर्वन्तु।
\item
  निरपेक्षसमागमस्य अभिसरणं किमर्थं भवति, परन्तु विपरीतम् सत्यं नास्ति इति
  व्याख्यातव्यम्।
\item
  रीमैन् पुनर्व्यवस्थापनप्रमेयस्य विषये स्वशब्देषु शोधं कृत्वा सारांशं कुर्वन्तु।
\end{enumerate}

\section{अध्याय 13. शक्तिश्रृङ्खला तथा
विस्तार}\label{ux905ux927ux92fux92f-13.-ux936ux915ux924ux936ux930ux919ux916ux932-ux924ux925-ux935ux938ux924ux930}

\subsection{13.1 शक्ति
श्रृङ्खला}\label{ux936ux915ux924-ux936ux930ux919ux916ux932}

शक्तिश्रृङ्खला अनन्तश्रृङ्खला अस्ति यस्मिन् प्रत्येकं पदं चरस्य शक्तिं समावेशयति ।
शक्तिश्रृङ्खलाः गणनायां केन्द्रस्थाः सन्ति यतोहि ते अस्मान् कार्यान् अनन्तबहुपदरूपेण
प्रतिनिधितुं ददति ।

\subsubsection{सामान्य रूप}\label{ux938ux92eux928ux92f-ux930ux92a-1}

\(a\) इत्यत्र केन्द्रीकृतायाः शक्तिश्रृङ्खलायाः रूपं भवति

\[ इति\sum_{n=0}^\infty c_n (x-a) ^ n,
\]

where \(c_n\) are constants called the coefficients.

\begin{itemize}
\item
  If \(a=0\), the series is centered at the origin:

  \[ इति
  \sum_{n=0}^\infty c_n x^n.
  \]
\end{itemize}

\subsubsection{Examples}\label{examples-3}

\begin{enumerate}
\def\labelenumi{\arabic{enumi}.}
\tightlist
\item
  Geometric series
\end{enumerate}

\[ इति
\sum_{n=0}^\infty x^n = \frac{1}{1-x}, \क्वाड |x|<1।
\]

\begin{enumerate}
\def\labelenumi{\arabic{enumi}.}
\setcounter{enumi}{1}
\tightlist
\item
  Exponential function
\end{enumerate}

\[
e^x = \sum_{n=0}^\infty \frac{x^n}{न!}।
\]

\begin{enumerate}
\def\labelenumi{\arabic{enumi}.}
\setcounter{enumi}{2}
\tightlist
\item
  Sine and cosine
\end{enumerate}

\[ इति
\सिन x = \sum_{n=0}^\infty (-1)^n \frac{x^{2n+1}}{(2n+1)!}, \quad  
\cos x = \sum_{n=0}^\infty (-1)^n \frac{x^{2n}}{(2n)!}।
\]

\subsubsection{Interval of Convergence}\label{interval-of-convergence}

For each power series, there exists a radius of convergence \(R\) such
that:

\begin{itemize}
\tightlist
\item
  The series converges if \(|x-a| < R\).
\item
  The series diverges if \(|x-a| > R\).
\item
  At \(|x-a| = R\), convergence must be checked separately.
\end{itemize}

\subsubsection{Why This Matters}\label{why-this-matters-2}

\begin{itemize}
\tightlist
\item
  Power series allow us to approximate functions by polynomials.
\item
  They connect calculus with analysis and differential equations.
\item
  Many special functions in mathematics and physics are defined by their
  power series.
\end{itemize}

\subsubsection{Exercises}\label{exercises-5}

\begin{enumerate}
\def\labelenumi{\arabic{enumi}.}
\tightlist
\item
  Write the power series for \(\sum_{n=0}^\infty \frac{(x-2)^n}{n!}\).
\item
  Find the first four terms of the power series for \(e^x\).
\item
  Express \(\frac{1}{1+x}\) as a power series centered at 0.
\item
  Determine whether the series \(\sum_{n=0}^\infty n! x^n\) converges at
  \(x=0.1\).
\item
  Explain why power series are sometimes called ``infinite
  polynomials.''
\end{enumerate}

\subsection{13.2 Radius of Convergence}\label{radius-of-convergence}

Every power series converges for some values of \(x\) and diverges for
others. The boundary between these two behaviors is described by the
radius of convergence.

\subsubsection{Definition}\label{definition-1}

For a power series

\[ इति
\sum_{n=0}^\infty c_n (x-a) ^ n,
\] इति

तत्र \(R \geq 0\) (संभवतः अनन्त) सङ्ख्या विद्यते यथा:

\begin{itemize}
\tightlist
\item
  श्रृङ्खला नितान्तं अभिसरणं करोति यदि \(|x-a| < R\).
\item
  श्रृङ्खला विचलति यदि \(|x-a| > R\)।- \(|x-a| = R\) इत्यत्र अभिसरणस्य पृथक्
  जाँचः करणीयः ।
\end{itemize}

इयं संख्या \(R\) इति अभिसरणत्रिज्या उच्यते ।

\subsubsection{अभिसरण त्रिज्या
अन्वेषण}\label{ux905ux92dux938ux930ux923-ux924ux930ux91cux92f-ux905ux928ux935ux937ux923}

सामान्यौ विधिद्वयम् : १.

\begin{enumerate}
\def\labelenumi{\arabic{enumi}.}
\tightlist
\item
  अनुपातपरीक्षा
\end{enumerate}

\[
R = \lim_{n\to\infty} \left| \frac{c_n}{c_{n+1}} \right|.
\]

\begin{enumerate}
\def\labelenumi{\arabic{enumi}.}
\setcounter{enumi}{1}
\tightlist
\item
  मूलपरीक्षा
\end{enumerate}

\[
R = \frac{1}{\limsup_{n\to\infty} \sqrt[n]{|c_n|}}.
\] इति

\subsubsection{उदाहरणम्}\label{ux909ux926ux939ux930ux923ux92e-26}

\begin{enumerate}
\def\labelenumi{\arabic{enumi}.}
\tightlist
\item
  श्रृङ्खला : १.
\end{enumerate}

\[
\sum_{n=0}^\infty \frac{x^n}{n!}.
\] इति

अनुपातपरीक्षायाः उपयोगेन : १.

\[
\lim_{n\to\infty} \frac{1/(n!)}{1/((n+1)!)} = \infty.
\] इति

अतः \(R = \infty\) (सर्वस्य वास्तविकस्य \(x\) कृते अभिसरणं करोति) ।

\begin{enumerate}
\def\labelenumi{\arabic{enumi}.}
\setcounter{enumi}{1}
\tightlist
\item
  श्रृङ्खला : १.
\end{enumerate}

\[
\sum_{n=0}^\infty x^n.
\] इति

अत्र \(c_n = 1\) इति ।

\[
R = 1.
\] इति

\(|x| < 1\) कृते अभिसरति ।

\begin{enumerate}
\def\labelenumi{\arabic{enumi}.}
\setcounter{enumi}{2}
\tightlist
\item
  श्रृङ्खला : १.
\end{enumerate}

\[
\sum_{n=1}^\infty \frac{x^n}{n}.
\] इति

अनुपातपरीक्षाः १.

\[
\lim_{n\to\infty} \left|\frac{(x^{n+1}/(n+1))}{(x^n/n)}\right| = |x|.
\] इति

अतः \(R = 1\)। \(|x| < 1\) कृते अभिसरति, \(|x| > 1\) कृते विचलति ।
\(x=\pm 1\) इत्यत्र पृथक् परीक्षणं कुर्वन्तु ।

\subsubsection{अभिसरणस्य
अन्तराल}\label{ux905ux92dux938ux930ux923ux938ux92f-ux905ux928ux924ux930ux932}

यत्र श्रृङ्खला अभिसरणं करोति तत्र \(x\)-मूल्यानां समुच्चयः अभिसरणस्य अन्तरालः इति
कथ्यते ।

\begin{itemize}
\tightlist
\item
  सदैव \(a\) इत्यत्र केन्द्रितम्।
\item
  उभयदिशि \(R\) एककानां विस्तारं करोति ।
\item
  अन्त्यबिन्दवः \(x=a\pm R\) इत्यस्य व्यक्तिगतरूपेण जाँचः करणीयः ।
\end{itemize}

\subsubsection{एतत् किमर्थं
महत्त्वपूर्णम्}\label{ux90fux924ux924-ux915ux92eux930ux925-ux92eux939ux924ux924ux935ux92aux930ux923ux92e-30}

\begin{itemize}
\tightlist
\item
  अभिसरणस्य त्रिज्या अस्मान् वदति यत् शक्तिश्रृङ्खला कुत्र कार्यवत् वर्तन्ते।
\item
  व्यवहारे टेलर श्रृङ्खलाविस्तारस्य उपयोगाय आवश्यकम्।
\item
  भौतिकशास्त्रे अभियांत्रिकीशास्त्रे च श्रृङ्खलासमाधानस्य वैधतायाः क्षेत्रं निर्धारयति।
\end{itemize}

\subsubsection{अभ्यास}\label{ux905ux92dux92fux938-44}

\begin{enumerate}
\def\labelenumi{\arabic{enumi}.}
\tightlist
\item
  \(\sum_{n=0}^\infty \frac{(x-3)^n}{n!}\) की अभिसरण त्रिज्या ज्ञात कीजिए।
\item
  \(\sum_{n=1}^\infty \frac{x^n}{n^2}\) इत्यस्य अभिसरणत्रिज्यायाः गणनां कुरुत।
\item
  \(\sum_{n=0}^\infty n!x^n\) कृते \(R\) अन्वेष्टुं अनुपातपरीक्षायाः उपयोगं कुर्वन्तु
  ।
\item
  \(\sum_{n=1}^\infty \frac{(x+1)^n}{n}\) कृते अभिसरणान्तरं निर्धारयन्तु।
\item
  व्याख्यातव्यं यत् घातीयश्रृङ्खला सर्वेषां \(x\) कृते अभिसरणं करोति, यदा तु
  ज्यामितीयश्रृङ्खला केवलं \(|x|<1\) कृते अभिसरणं करोति।\#\# 13.3 टेलर एण्ड मैक्लेरिन्
  श्रृङ्खला
\end{enumerate}

शक्तिश्रृङ्खला तदा विशेषतया शक्तिशालिनः भवन्ति यदा तेषां उपयोगः परिचितकार्यस्य
प्रतिनिधित्वार्थं भवति । एतत् टेलर-श्रृङ्खलायाः माध्यमेन भवति, 0 इत्यत्र केन्द्रितः
विशेषः प्रकरणः च मैक्लोरिन्-श्रृङ्खला इति कथ्यते ।

\subsubsection{टेलर श्रृंखला}\label{ux91fux932ux930-ux936ux930ux916ux932}

यदि \(f(x)\) इति फंक्शन् \(x=a\) इत्यत्र अनन्तरूपेण विभेदनीयः अस्ति तर्हि \(a\)
इत्यस्य विषये तस्य Taylor श्रृङ्खला अस्ति

\[
f(x) = \sum_{n=0}^\infty \frac{f^{(n)}(a)}{n!}(x-a)^n.
\] इति

अत्र \(f^{(n)}(a)\) \(a\) इत्यत्र \(f\) इत्यस्य \(n\)-तमं व्युत्पन्नं सूचयति ।

\subsubsection{मैकलॉरिन
श्रृंखला}\label{ux92eux915ux932ux930ux928-ux936ux930ux916ux932}

\(a=0\) इत्यत्र केन्द्रीकृता एकः टेलर-श्रृङ्खला:

\[
f(x) = \sum_{n=0}^\infty \frac{f^{(n)}(0)}{n!} x^n.
\] इति

\subsubsection{उदाहरणम्}\label{ux909ux926ux939ux930ux923ux92e-27}

\begin{enumerate}
\def\labelenumi{\arabic{enumi}.}
\tightlist
\item
  घातीय फलनम्
\end{enumerate}

\[
e^x = 1 + x + \frac{x^2}{2!} + \frac{x^3}{3!} + \cdots
\] इति

\begin{enumerate}
\def\labelenumi{\arabic{enumi}.}
\setcounter{enumi}{1}
\tightlist
\item
  साइनः कोसाइनः च
\end{enumerate}

\[
\sin x = x - \frac{x^3}{3!} + \frac{x^5}{5!} - \cdots
\] इति

\[
\cos x = 1 - \frac{x^2}{2!} + \frac{x^4}{4!} - \cdots
\] इति

\begin{enumerate}
\def\labelenumi{\arabic{enumi}.}
\setcounter{enumi}{2}
\tightlist
\item
  प्राकृतिक लघुगणक (\(|x|<1\) कृते) .
\end{enumerate}

\[
\ln(1+x) = x - \frac{x^2}{2} + \frac{x^3}{3} - \frac{x^4}{4} + \cdots
\] इति

\subsubsection{टेलर बहुपद
सन्निकटन}\label{ux91fux932ux930-ux92cux939ux92aux926-ux938ux928ux928ux915ux91fux928}

प्रथमस्य \(n\) पदानाम् परिमितयोगः डिग्री \(n\) इत्यस्य टेलर बहुपदः अस्ति:

\[
P_n(x) = \sum_{k=0}^n \frac{f^{(k)}(a)}{k!}(x-a)^k.
\] इति

इदं बहुपदं \(x=a\) इत्यस्य समीपे \(f(x)\) इत्यस्य अनुमानं करोति ।

\subsubsection{शेष (त्रुटिपद)
२.}\label{ux936ux937-ux924ux930ux91fux92aux926-ux968.}

फलनस्य तस्य टेलर बहुपदस्य च भेदः शेषः अस्ति : १.

\[
R_n(x) = f(x) - P_n(x).
\] इति

एकं रूपं (Lagrange's form) अस्ति

\[
R_n(x) = \frac{f^{(n+1)}(c)}{(n+1)!}(x-a)^{n+1},
\] इति

\(a\) तथा \(x\) इत्येतयोः मध्ये केषाञ्चन \(c\) कृते ।

\subsubsection{एतत् किमर्थं
महत्त्वपूर्णम्}\label{ux90fux924ux924-ux915ux92eux930ux925-ux92eux939ux924ux924ux935ux92aux930ux923ux92e-31}

\begin{itemize}
\tightlist
\item
  टेलर श्रृङ्खला जटिलफलनानां बहुपदसन्निकर्षं प्रदाति।
\item
  संख्यात्मकविश्लेषणे, भौतिकशास्त्रे, अभियांत्रिकीशास्त्रे च ते अत्यावश्यकाः सन्ति।
\item
  मैकलॉरिन् श्रृङ्खलाविस्तारः घातीय, त्रिकोणमितीय, लघुगणकीय च फलनानां सरलसूत्राणि
  ददाति ।
\end{itemize}

\subsubsection{अभ्यास}\label{ux905ux92dux92fux938-45}

\begin{enumerate}
\def\labelenumi{\arabic{enumi}.}
\tightlist
\item
  \(f(x)=\cosh x = \tfrac{e^x+e^{-x}}{2}\) कृते Maclaurin श्रृङ्खलां
  ज्ञातव्यम्।2. \(a=2\) इत्यत्र केन्द्रीकृत्य \(f(x)=e^x\) इत्यस्य कृते Taylor श्रृङ्खलां
  लिखत।
\item
  \(a=0\) इत्यत्र \(f(x)=\ln(1+x)\) इत्यस्य कृते डिग्री-3 टेलर बहुपदस्य गणनां कुर्वन्तु
  ।
\item
  \(\sin(0.1)\) इत्यस्य अनुमानं कर्तुं \(\sin x\) इत्यस्य कृते Maclaurin श्रृङ्खलायाः
  उपयोगं कुर्वन्तु ।
\item
  व्याख्यातव्यं यत् टेलर श्रृङ्खला प्रायः उत्तमं स्थानीयसन्निकर्षं किमर्थं प्रदाति परन्तु बृहत्
  \(|x|\) कृते विचलितुं शक्नोति।
\end{enumerate}

\subsection{13.4 टेलर श्रृङ्खलायाः
अनुप्रयोगाः}\label{ux91fux932ux930-ux936ux930ux919ux916ux932ux92f-ux905ux928ux92aux930ux92fux917}

टेलर श्रृङ्खलाः केवलं सैद्धान्तिकसाधनाः न सन्ति - तेषां उपयोगः कार्याणां अनुमानं कर्तुं,
समीकरणानां समाधानार्थं, भौतिकतन्त्राणां विश्लेषणार्थं च भवति । तेषां अनुप्रयोगाः गणितं,
विज्ञानं, अभियांत्रिकीशास्त्रं च व्याप्नुवन्ति ।

\subsubsection{फ़ंक्शन
सन्निकर्ष}\label{ux92bux915ux936ux928-ux938ux928ux928ux915ux930ux937}

जटिलकार्यं बिन्दुसमीपे बहुपदैः अनुमानितुं शक्यते ।

उदाहरणम् : डिग्री-3 मैक्लेरिन् बहुपदस्य उपयोगेन \(x=0\) इत्यस्य समीपे अनुमानित
\(e^x\):

\[
P_3(x) = 1 + x + \tfrac{x^2}{2} + \tfrac{x^3}{6}.
\] इति

लघु \(x\) कृते, एतेन \(e^x\) इत्यस्य सटीकं अनुमानं प्राप्यते ।

\subsubsection{संख्यात्मक
विधियाँ}\label{ux938ux916ux92fux924ux92eux915-ux935ux927ux92f}

टेलर श्रृङ्खला संख्यात्मक-अल्गोरिदम् इत्यस्य आधारं प्रददाति : १.

\begin{itemize}
\tightlist
\item
  वर्गमूलानि, लघुगणकानि, त्रिकोणमितीयमूल्यानि च अनुमानयन् ।
\item
  शेषपदस्य माध्यमेन त्रुटिअनुमानम्।
\item
  न्यूटनस्य पद्धतिः इत्यादिषु पुनरावर्तनीयविधिषु प्रयुक्तः (यत्र स्थानीयरेखीयीकरणं
  टेलरश्रृङ्खलातः आगच्छति) ।
\end{itemize}

\subsubsection{अवकल समीकरणों का
समाधान}\label{ux905ux935ux915ux932-ux938ux92eux915ux930ux923-ux915-ux938ux92eux927ux928}

अनेकानाम् अवकलसमीकरणानां समाधानं टेलरः (अथवा शक्तिः) श्रृङ्खला इति व्यक्तं भवति ।

उदाहरणम् : \(y(0)=0, y'(0)=1\) इत्यनेन सह \(y'' + y = 0\) इत्यस्य समाधानं
\(\sin x\) अस्ति, यत् स्वाभाविकतया तस्य Maclaurin श्रृङ्खलातः उत्पद्यते ।

\subsubsection{भौतिकी एवं
अभियांत्रिकी}\label{ux92dux924ux915-ux90fux935-ux905ux92dux92fux924ux930ux915}

\begin{itemize}
\item
  लघु-कोण सन्निकर्षः : १.

  \[
  \sin x \approx x, \quad \cos x \approx 1 - \tfrac{x^2}{2}, \quad |x| \ll 1.
  \] इति

  लोलकगतिः, प्रकाशिकी, तरङ्गयान्त्रिकः च इति विषयेषु उपयुज्यते ।
\item
  सापेक्षता तथा क्वाण्टम यांत्रिकी : टेलर विस्तारः व्यावहारिकप्रयोगाय
  अरैखिकव्यञ्जनानि सरलीकरोति ।- ऊर्जाकार्यस्य अनुमानम् : यान्त्रिकशास्त्रे सम्भाव्य
  ऊर्जाकार्यं संतुलनबिन्दुसमीपे विस्तारितं भवति ।
\end{itemize}

\subsubsection{संभाव्यता एवं
सांख्यिकी}\label{ux938ux92dux935ux92fux924-ux90fux935-ux938ux916ux92fux915}

\begin{itemize}
\tightlist
\item
  क्षणजननकार्यं लक्षणकार्यं च शक्तिश्रृङ्खलायाः उपयोगं करोति ।
\item
  संभाव्यतावितरणस्य सन्निकर्षाः (उदा. द्विपदस्य सामान्यसन्निकर्षः) टेलरविस्तारस्य
  उपयोगं कुर्वन्ति ।
\end{itemize}

\subsubsection{एतत् किमर्थं
महत्त्वपूर्णम्}\label{ux90fux924ux924-ux915ux92eux930ux925-ux92eux939ux924ux924ux935ux92aux930ux923ux92e-32}

\begin{itemize}
\tightlist
\item
  टेलर श्रृङ्खला सटीकसूत्राणां व्यावहारिकगणनायाश्च मध्ये सेतुम् प्रददाति ।
\item
  ते अस्मान् जटिलसमस्यान् प्रबन्धनीयबहुपदसन्निकर्षेषु न्यूनीकर्तुं शक्नुवन्ति।
\item
  अनुप्रयोगाः तान् अनुप्रयुक्तगणितस्य महत्त्वपूर्णेषु साधनेषु अन्यतमं कुर्वन्ति।
\end{itemize}

\subsubsection{अभ्यास}\label{ux905ux92dux92fux938-46}

\begin{enumerate}
\def\labelenumi{\arabic{enumi}.}
\tightlist
\item
  \(e^x\) कृते Maclaurin श्रृङ्खलायाः उपयोगं कृत्वा \(e^{0.1}\) चतुर्णां
  दशमलवस्थानानां यावत् अनुमानं कुर्वन्तु ।
\item
  \(\sin(5^\circ)\) इत्यस्य अनुमानं कर्तुं लघु-कोणसन्निकर्षं प्रयोजयन्तु।
\item
  शक्तिश्रृङ्खलापद्धतेः उपयोगेन अवकलसमीकरणस्य \(y'' = -y\) समाधानं कुर्वन्तु।
\item
  \(\ln(1+x)\) इत्येतत् 4th degree पर्यन्तं विस्तारयन्तु तथा च \(\ln(1.1)\) इत्यस्य
  अनुमानं कर्तुं तस्य उपयोगं कुर्वन्तु ।
\item
  बहुपदसन्निकर्षाः सङ्गणकानां गणकयन्त्राणां च कृते विशेषतया किमर्थं उपयोगिनो भवन्ति
  इति व्याख्यातव्यम्।
\end{enumerate}

\section{परिशिष्टानि}\label{ux92aux930ux936ux937ux91fux928}

\subsection{परिशिष्ट क. पूर्वगणना
आवश्यक}\label{ux92aux930ux936ux937ux91f-ux915.-ux92aux930ux935ux917ux923ux928-ux906ux935ux936ux92fux915}

\subsubsection{क.1 बीजगणित
ताजगी}\label{ux915.1-ux92cux91cux917ux923ux924-ux924ux91cux917}

गणनायां गोतां कर्तुं पूर्वं केषाञ्चन बीजगणितकौशलानाम् समीक्षां कर्तुं साहाय्यं करोति ये पुनः
पुनः दृश्यन्ते । एते एव ``उपकरणाः'' भवद्भिः व्यञ्जनानां परिवर्तनार्थं, समीकरणानां
समाधानार्थं, परिणामानां सरलीकरणाय च आवश्यकाः भविष्यन्ति ।

\paragraph{घातांक एवं
शक्ति}\label{ux918ux924ux915-ux90fux935-ux936ux915ux924}

\begin{itemize}
\item
  मूलभूतनियमाः : १.

  \[
  a^m \cdot a^n = a^{m+n}, \quad \frac{a^m}{a^n} = a^{m-n}, \quad (a^m)^n = a^{mn}.
  \] इति
\item
  ऋणात्मक घातांक : १.

  \[
  a^{-n} = \frac{1}{a^n}, \quad a \neq 0.
  \] इति
\item
  भिन्नात्मक घातांक : १.

  \[
  a^{1/n} = \sqrt[n]{a}, \quad a^{m/n} = \sqrt[n]{a^m}.
  \] इति
\end{itemize}

\paragraph{कारकीकरण}\label{ux915ux930ux915ux915ux930ux923}

कारकीकरणं व्यञ्जनानि सरलीकरोति, समीकरणानां समाधानार्थं च सहायकं भवति ।

\begin{enumerate}
\def\labelenumi{\arabic{enumi}.}
\item
  सामान्य कारकः : १.

  \[
  6x^2+9x = 3x(2x+3).
  \] इति
\item
  वर्गानां भेदः : १.

  \[ इतिक^२-ख^२ = (क-ख)(क+ख)।
  \]
\item
  Quadratic trinomials:

  \[ इति
  x^2+5x+6 = (x+2)(x+3)।
  \]
\end{enumerate}

\paragraph{Polynomials}\label{polynomials}

\begin{itemize}
\tightlist
\item
  Standard form: \(P(x) = a_nx^n + a_{n-1}x^{n-1} + \cdots + a_0\).
\item
  Degree: the largest power of \(x\).
\item
  Long division and synthetic division are useful for simplifying
  rational functions.
\end{itemize}

\paragraph{Rational Expressions}\label{rational-expressions}

Simplify by factoring numerator and denominator:

\[ इति
\frac{x^2-1}{x^2-2x+1} = \frac{(x-1)(x+1)}{(x-1)^2} = \frac{x+1}{x-1}, \quad x \neq 1.
\]

\paragraph{Logarithms}\label{logarithms}

\begin{itemize}
\item
  Definition: \(\log_a b = c\) means \(a^c = b\).
\item
  Common bases: natural log (\(\ln x = \log_e x\)) and base 10
  (\(\log x\)).
\item
  Rules:

  \[
  \log(ab) = \log a + \log b, \quad \log\left(\frac{a}{b}\right) = \log a - \log b, \quad \log(a^n) = n\log a.
  \]
\end{itemize}

\paragraph{Equations}\label{equations}

\begin{itemize}
\item
  Linear: solve \(ax+b=0\) → \(x=-b/a\).
\item
  Quadratic: \(ax^2+bx+c=0\) has solutions

  \[ इति
  x=\frac{-b\pm \sqrt{b^2-4ac}}{2a}।
  \]
\item
  Exponential: \(e^x = k\) → \(x = \ln k\).
\end{itemize}

\subsubsection{A.2 Trigonometry Basics}\label{a.2-trigonometry-basics}

Trigonometry provides the language of angles and periodic phenomena.
Since calculus often deals with oscillations, motion, and waves, a solid
grasp of trigonometric functions and their properties is essential.

\paragraph{The Unit Circle}\label{the-unit-circle}

\begin{itemize}
\item
  Defined as the circle of radius 1 centered at the origin in the
  coordinate plane.
\item
  For an angle \(\theta\) measured from the positive \(x\)-axis:

  \[ इति
  (\कोस \थेता, \सिं \थेता) २.
  \] इति

  वृत्ते बिन्दुस्य निर्देशांकं ददाति।
\end{itemize}

विशेषमूल्यानि : १.

\begin{longtable}[]{@{}
  >{\raggedright\arraybackslash}p{(\linewidth - 6\tabcolsep) * \real{0.3333}}
  >{\raggedright\arraybackslash}p{(\linewidth - 6\tabcolsep) * \real{0.1667}}
  >{\raggedright\arraybackslash}p{(\linewidth - 6\tabcolsep) * \real{0.1667}}
  >{\raggedright\arraybackslash}p{(\linewidth - 6\tabcolsep) * \real{0.3333}}@{}}
\toprule\noalign{}
\begin{minipage}[b]{\linewidth}\raggedright
\(\theta\)
\end{minipage} & \begin{minipage}[b]{\linewidth}\raggedright
\(\sin \theta\)
\end{minipage} & \begin{minipage}[b]{\linewidth}\raggedright
\(\cos \theta\)
\end{minipage} & \begin{minipage}[b]{\linewidth}\raggedright
\(\tan \theta = \frac{\sin \theta}{\cos \theta}\)
\end{minipage} \\
\midrule\noalign{}
\endhead
\bottomrule\noalign{}
\endlastfoot
\(0\) & ० & १ & ० \\
\(\pi/6\) & १/२ & \(\sqrt{3}/2\) & \(1/\sqrt{3}\) \\
\(\pi/3\) & \(\sqrt{3}/2\) & १/२ & \(\sqrt{3}\) \\
\(\pi/2\) & १ & ० & अविवक्षित \\
\end{longtable}

\paragraph{मौलिक पहचान}\label{ux92eux932ux915-ux92aux939ux91aux928}

\begin{enumerate}
\def\labelenumi{\arabic{enumi}.}
\tightlist
\item
  पायथागोरस-परिचयः
\end{enumerate}

\[
\sin^2\theta + \cos^2\theta = 1.
\] इति

\begin{enumerate}
\def\labelenumi{\arabic{enumi}.}
\setcounter{enumi}{1}
\tightlist
\item
  भागफलपरिचयः
\end{enumerate}

\[
\tan\theta = \frac{\sin\theta}{\cos\theta}, \quad \cot\theta = \frac{\cos\theta}{\sin\theta}.
\] इति

\begin{enumerate}
\def\labelenumi{\arabic{enumi}.}
\setcounter{enumi}{2}
\tightlist
\item
  परस्परपरिचयः
\end{enumerate}

\[
\sec\theta = \frac{1}{\cos\theta}, \quad \csc\theta = \frac{1}{\sin\theta}.
\] इति

\paragraph{कोण योजन
सूत्र}\label{ux915ux923-ux92fux91cux928-ux938ux924ux930}

\[
\sin(\alpha+\beta) = \sin\alpha\cos\beta + \cos\alpha\sin\beta,
\] इति

\[
\cos(\alpha+\beta) = \cos\alpha\cos\beta - \sin\alpha\sin\beta.
\] इति

विशेषप्रकरणाः : १.

\begin{itemize}
\item
  द्विकोणः : १.

  \[
  \sin(2\theta) = 2\sin\theta\cos\theta, \quad
  \cos(2\theta) = \cos^2\theta - \sin^2\theta.
  \] इति
\end{itemize}

\paragraph{आलेख}\label{ux906ux932ux916}

\begin{itemize}
\tightlist
\item
  \(\sin x\): तरंग 0, आयाम 1, अवधि \(2\pi\) पर आरभ्यते।
\item
  \(\cos x\): तरंग 1, आयाम 1, अवधि \(2\pi\) पर आरभ्यते।
\item
  \(\tan x\): प्रत्येकं \(\pi\) पुनरावृत्तिं करोति, \(\pi/2\) इत्यस्य विषमगुणकेषु
  अपरिभाषितम् ।
\end{itemize}

\subsubsection{क.3 समन्वय
ज्यामिति}\label{ux915.3-ux938ux92eux928ux935ux92f-ux91cux92fux92eux924}

समीकरणानां उपयोगेन ज्यामितीयवस्तूनाम् (रेखाः, वृत्ताः, वक्राः) वर्णनं कृत्वा ज्यामितिः
बीजगणितं ज्यामितिं च सम्बध्दयति समन्वयं कुर्वन्तु । कार्याणां आलेखनिर्धारणाय, प्रवणानाम्
अन्वेषणाय, वक्रविश्लेषणाय च गणितः अस्मिन् ढाञ्चे बहुधा अवलम्बते ।

\paragraph{कार्टेशियन
विमान}\label{ux915ux930ux91fux936ux92fux928-ux935ux92eux928}

\begin{itemize}
\item
  एकः बिन्दुः निर्देशांकैः \((x,y)\) इत्यनेन प्रतिनिधितः भवति ।
\item
  \((x_1,y_1)\) तथा \((x_2,y_2)\) द्वयोः बिन्दुयोः मध्ये दूरी:

  \[
  d = \sqrt{(x_2-x_1)^2 + (y_2-y_1)^2}.
  \] इति
\item
  रेखाखण्डस्य मध्यबिन्दुः : १.

  \[
  M = \left(\frac{x_1+x_2}{2}, \frac{y_1+y_2}{2}\right).
  \] इति
\end{itemize}

\paragraph{रेखाएँ}\label{ux930ux916ux90f}

\begin{enumerate}
\def\labelenumi{\arabic{enumi}.}
\item
  प्रवणसूत्रम्

  \[
  m = \frac{y_2-y_1}{x_2-x_1}.
  \] इति
\item
  रेखायाः समीकरणम्

  \begin{itemize}
  \item
    बिन्दु-प्रवणरूपः : १.

    \[ इतिय-य_१ = म(x-x_1)।
    \]
  \item
    Slope-intercept form:

    \[ इति
    य = mx+b.
    \]
  \end{itemize}
\item
  Parallel and perpendicular lines

  \begin{itemize}
  \tightlist
  \item
    Parallel lines: same slope.
  \item
    Perpendicular lines: slopes satisfy \(m_1m_2 = -1\).
  \end{itemize}
\end{enumerate}

\paragraph{Circles}\label{circles}

Equation of a circle with center \((h,k)\) and radius \(r\):

\[ इति
(x-h) ^ 2+(y-k) ^ 2 = र ^ 2।
\]

Special case: unit circle centered at origin:

\[
x^2+y^2=1.
\]

\paragraph{Conic Sections}\label{conic-sections}

\begin{enumerate}
\def\labelenumi{\arabic{enumi}.}
\item
  Parabola:

  \begin{itemize}
  \item
    Standard form (opening up/down):

    \[ इति
    य = अक्ष^२+बक्स+ग.
    \]
  \end{itemize}
\item
  Ellipse (centered at origin):

  \[ इति
  \frac{x^2}{क^2}+\frac{y^2}{b^2}=1.
  \]
\item
  Hyperbola (centered at origin):

  \[ इति
  \frac{x^2}{क^2}-\frac{y^2}{b^2}=1.
  \]
\end{enumerate}

\subsection{Appendix B. Key Formulas and
Tables}\label{appendix-b.-key-formulas-and-tables}

\subsubsection{B.1 Derivative Table}\label{b.1-derivative-table}

Derivatives measure rates of change and slopes of functions. Having a
quick-reference table helps learners avoid re-deriving formulas each
time.

\paragraph{Basic Rules}\label{basic-rules}

\begin{enumerate}
\def\labelenumi{\arabic{enumi}.}
\tightlist
\item
  Constant rule
\end{enumerate}

\[ इति
\frac{d}{dx}[c] = 0
\]

\begin{enumerate}
\def\labelenumi{\arabic{enumi}.}
\setcounter{enumi}{1}
\tightlist
\item
  Power rule
\end{enumerate}

\[ इति
\frac{d}{dx}[x^n] = nx^{n-1}, \quad (n \in \mathbb{R})
\]

\begin{enumerate}
\def\labelenumi{\arabic{enumi}.}
\setcounter{enumi}{2}
\tightlist
\item
  Constant multiple rule
\end{enumerate}

\[ इति
\frac{d}{dx}[c f(x)] = c f'(x)
\]

\begin{enumerate}
\def\labelenumi{\arabic{enumi}.}
\setcounter{enumi}{3}
\tightlist
\item
  Sum and difference rule
\end{enumerate}

\[ इति
\frac{d}{dx}[f(x)\pm g(x)] = च'(x)\pm g'(x)
\]

\paragraph{Trigonometric Functions}\label{trigonometric-functions}

\[ इति
\frac{d}{dx}[\sin x] = \cos x
\]

\[ इति
\frac{d}{dx}[\cos x] = -\सिन x
\]

\[ इति
\frac{d}{dx}[\tan x] = \sec^2 x, \quad x \neq \tfrac{\pi}{2}+क\पि
\]

\[ इति
\frac{d}{dx}[\cot x] = -\csc^2 x
\]

\[ इति
\frac{d}{dx}[\sec x] = \sec x \tan x
\]

\[ इति
\frac{d}{dx}[\csc x] = -\csc x \cot x
\]

\paragraph{Exponential and Logarithmic
Functions}\label{exponential-and-logarithmic-functions}

\[
\frac{d}{dx}[e^x] ​​= ई^क्स
\]

\[ इति
\frac{d}{dx}[a^x] = a^x \ln a, \quad a>0, a\neq 1
\]

\[ इति
\frac{d}{dx}[\ln x] = \frac{1}{x}, \quad x>0
\]

\[ इति
\frac{d}{dx}[\log_a x] = \frac{1}{x\ln a}, \quad a>0, a\neq 1
\]

\paragraph{Inverse Trigonometric
Functions}\label{inverse-trigonometric-functions}

\[ इति\frac{d}{dx}[\arcsin x] = \frac{1}{\sqrt{1-x^2}}, \quad |x|<1
\]

\[ इति
\frac{d}{dx}[\arccos x] = -\frac{1}{\sqrt{1-x^2}}, \quad |x|<1
\]

\[ इति
\frac{d}{dx}[\arctan x] = \frac{1}{1+x^2}, \quad x \in \mathbb{R}
\]

\paragraph{Product, Quotient, and Chain
Rules}\label{product-quotient-and-chain-rules}

\begin{enumerate}
\def\labelenumi{\arabic{enumi}.}
\tightlist
\item
  Product Rule
\end{enumerate}

\[
\frac{d}{dx}[f(x)g(x)] = च'(x)g(x)+f(x)g'(x)
\]

\begin{enumerate}
\def\labelenumi{\arabic{enumi}.}
\setcounter{enumi}{1}
\tightlist
\item
  Quotient Rule
\end{enumerate}

\[ इति
\frac{d}{dx}\left[\frac{f(x)}{g(x)}\right] = \frac{f'(x)g(x)-f(x)g'(x)}{[g(x)]^2}, \quad g(x)\neq 0
\]

\begin{enumerate}
\def\labelenumi{\arabic{enumi}.}
\setcounter{enumi}{2}
\tightlist
\item
  Chain Rule
\end{enumerate}

\[ इति
\frac{d}{dx}[f(g(x))] = f'(g(x))\cdot g'(x)
\]

\subsubsection{B.3 Common Series
Expansions}\label{b.3-common-series-expansions}

Power series let us express functions as infinite polynomials. These
expansions are essential for approximations, solving differential
equations, and building intuition about functions in calculus.

\paragraph{Geometric Series}\label{geometric-series}

\[ इति
\frac{1}{1-x} = \sum_{n=0}^\infty x^n, \quad |x| < १
\]

\paragraph{Exponential Function}\label{exponential-function}

\[ इति
e^x = \sum_{n=0}^\infty \frac{x^n}{न!}
= 1 + x + \frac{x^2}{2!} + \frac{x^3}{3!} + \cdots
\]

Valid for all \(x\).

\paragraph{Trigonometric Functions}\label{trigonometric-functions-1}

\[ इति
\सिन x = \sum_{n=0}^\infty (-1)^n \frac{x^{2n+1}}{(2n+1)!}
= x - \frac{x^3}{3!} + \frac{x^5}{5!} - \cdots
\]

\[ इति
\cos x = \sum_{n=0}^\infty (-1)^n \frac{x^{2n}}{(2n)!}
= 1 - \frac{x^2}{2!} + \frac{x^4}{4!} - \cdots
\]

\[ इति
\tan^{-1} x = \sum_{n=0}^\infty (-1)^n \frac{x^{2n+1}}{2n+1}, \quad |x|\leq 1
\]

\paragraph{Logarithm}\label{logarithm}

\[ इति
\ln(1+x) = \sum_{n=1}^\infty (-1)^{n+1} \frac{x^n}{n}, \quad -1 < x \leq 1
\]

\paragraph{Binomial Expansion
(Generalized)}\label{binomial-expansion-generalized}

\[ इति
(1+x)^r = \sum_{n=0}^\infty \बिनोम{r}{न} x^n, \quad |x|<1
\]

where

\[ इति
\binom{r}{n} = \frac{r(r-1)(r-2)\cdots(r-n+1)}{न!}।
\] इति

\subsection{परिशिष्ट ग. प्रमाण
रेखाचित्र}\label{ux92aux930ux936ux937ux91f-ux917.-ux92aux930ux92eux923-ux930ux916ux91aux924ux930}

\subsubsection{\texorpdfstring{ग.1 सीमा नियमाः तथा
\(\varepsilon\)--\(\delta\) परिभाषागणितः सीमायाः सटीकार्थे अवलम्बते ।
अन्तर्ज्ञानं (``मूल्यानि समीपं समीपं गच्छन्ति'') सहायकं भवति चेदपि औपचारिकपरिभाषा
कठोरताम् सुनिश्चित्य विरोधाभासान् परिहरति
।}{ग.1 सीमा नियमाः तथा \textbackslash varepsilon--\textbackslash delta परिभाषागणितः सीमायाः सटीकार्थे अवलम्बते । अन्तर्ज्ञानं (``मूल्यानि समीपं समीपं गच्छन्ति'') सहायकं भवति चेदपि औपचारिकपरिभाषा कठोरताम् सुनिश्चित्य विरोधाभासान् परिहरति ।}}\label{ux917.1-ux938ux92e-ux928ux92fux92e-ux924ux925-varepsilondelta-ux92aux930ux92dux937ux917ux923ux924-ux938ux92eux92f-ux938ux91fux915ux930ux925-ux905ux935ux932ux92eux92cux924-ux905ux928ux924ux930ux91cux91eux928-ux92eux932ux92fux928-ux938ux92eux92a-ux938ux92eux92a-ux917ux91aux91bux928ux924-ux938ux939ux92fux915-ux92dux935ux924-ux91aux926ux92a-ux914ux92aux91aux930ux915ux92aux930ux92dux937-ux915ux920ux930ux924ux92e-ux938ux928ux936ux91aux924ux92f-ux935ux930ux927ux92dux938ux928-ux92aux930ux939ux930ux924}

\paragraph{सहज विचार}\label{ux938ux939ux91c-ux935ux91aux930}

वयं लिखामः

\[
\lim_{x \to a} f(x) = L
\] इति

अर्थात् यथा यथा \(x\) \(a\) इत्यस्य मनमाना समीपं गच्छति तथा तथा \(f(x)\) इत्यस्य
मूल्यानि \(L\) इत्यस्य मनमाना समीपं गच्छन्ति ।

\paragraph{\texorpdfstring{औपचारिक (\(\varepsilon\)--\(\delta\))
परिभाषा}{औपचारिक (\textbackslash varepsilon--\textbackslash delta) परिभाषा}}\label{ux914ux92aux91aux930ux915-varepsilondelta-ux92aux930ux92dux937}

इति वदामः

\[
\lim_{x \to a} f(x) = L
\] इति

यदि प्रत्येकं \(\varepsilon > 0\) कृते \(\delta > 0\) अस्ति यत् यदा कदापि

\[
0 < |x-a| < \delta,
\] इति

अस्माकं अस्ति

\[
|f(x) - L| < \varepsilon.
\] इति

\begin{itemize}
\tightlist
\item
  \(\varepsilon\): वयं इच्छामः यत् \(f(x)\) \(L\) इत्यस्य कियत् समीपे भवतु।
\item
  \(\delta\): तत् प्राप्तुं \(x\) \(a\) इत्यस्य कियत् समीपे भवितुमर्हति।
\end{itemize}

\paragraph{उदाहरण}\label{ux909ux926ux939ux930ux923-2}

तत् दर्शयतु

\[
\lim_{x \to 2} (3x+1) = 7.
\] इति

\begin{itemize}
\tightlist
\item
  \(\varepsilon > 0\) अस्तु।
\item
  वयं \(|(3x+1)-7| < \varepsilon\) इच्छामः।
\item
  सरलीकरण : \(|3x-6| = 3|x-2| < \varepsilon\)।
\item
  यदि वयं \(\delta = \varepsilon/3\) इति चिनोमः तर्हि एतत् धारयति ।
\end{itemize}

एवं परिभाषया सीमा ७ ।

\paragraph{सीमा नियम}\label{ux938ux92e-ux928ux92fux92e}

यदि \(\lim_{x \to a} f(x) = L\) तथा \(\lim_{x \to a} g(x) = M\), तर्हि:

\begin{enumerate}
\def\labelenumi{\arabic{enumi}.}
\tightlist
\item
  योग/अन्तरम्
\end{enumerate}

\[
\lim_{x \to a} [f(x) \pm g(x)] = L \pm M
\] इति

\begin{enumerate}
\def\labelenumi{\arabic{enumi}.}
\setcounter{enumi}{1}
\tightlist
\item
  नित्यं बहुविधम्
\end{enumerate}

\[
\lim_{x \to a} [c f(x)] = cL
\] इति

\begin{enumerate}
\def\labelenumi{\arabic{enumi}.}
\setcounter{enumi}{2}
\tightlist
\item
  उत्पादः
\end{enumerate}

\[
\lim_{x \to a} [f(x)g(x)] = LM
\] इति

\begin{enumerate}
\def\labelenumi{\arabic{enumi}.}
\setcounter{enumi}{3}
\tightlist
\item
  भागफल (यदि \(M \neq 0\)) .
\end{enumerate}

\[
\lim_{x \to a} \frac{f(x)}{g(x)} = \frac{L}{M}
\] इति

\begin{enumerate}
\def\labelenumi{\arabic{enumi}.}
\setcounter{enumi}{4}
\tightlist
\item
  शक्तिः मूलं च
\end{enumerate}

\[
\lim_{x \to a} [f(x)]^n = L^n, \quad \lim_{x \to a} \sqrt[n]{f(x)} = \sqrt[n]{L} \ (\text{if defined}).
\] इति

\subsubsection{ग.2 प्रमाण रेखाचित्र : गणित का मौलिक
प्रमेय}\label{ux917.2-ux92aux930ux92eux923-ux930ux916ux91aux924ux930-ux917ux923ux924-ux915-ux92eux932ux915-ux92aux930ux92eux92f}

गणितस्य मौलिकप्रमेयः (FTC) गणितस्य केन्द्रीयक्रियाद्वयं सम्बध्दयति : भेदभावः एकीकरणं च
। ते वस्तुतः विलोमप्रक्रियाः इति दर्शयति ।

\paragraph{प्रमेय का
कथन}\label{ux92aux930ux92eux92f-ux915-ux915ux925ux928}

प्रथमः भागः (एकस्य अभिन्नस्य भेदः): १. यदि \(f\) \([a,b]\) इत्यत्र निरन्तरं भवति
तथा च वयं परिभाषयामः

\[ इतिF(x) = \int_a^x f(t)\,dt, 2019।
\]

then \(F\) is differentiable on \((a,b)\) and

\[ इति
च'(x) = च(x) ।
\]

Part II (Evaluation of a Definite Integral): If \(F\) is any
antiderivative of \(f\) on \([a,b]\), then

\[ इति
\int_a^b f(x)\,dx = F(b)-F(a).
\]

\paragraph{Proof Sketch of Part I}\label{proof-sketch-of-part-i}

\begin{enumerate}
\def\labelenumi{\arabic{enumi}.}
\item
  Start with the definition of the derivative:

  \[
  F'(x) = \lim_{h\to 0} \frac{F(x+h)-F(x)}{ह}।
  \]
\item
  Substituting \(F(x) = \int_a^x f(t)\,dt\):

  \[ इति
  F(x+h)-F(x) = \int_a^{x+h} f(t)\,dt - \int_a^x f(t)\,dt.
  \]
\item
  By the additivity of integrals:

  \[ इति
  F(x+h)-F(x) = \int_x^{x+h} f(t)\,dt.
  \]
\item
  Therefore:

  \[ इति
  \frac{F(x+h)-F(x)}{h} = \frac{1}{h}\int_x^{x+h} f(t)\,dt.
  \]
\item
  By the Mean Value Theorem for integrals, there exists
  \(c \in [x,x+h]\) such that

  \[ इति
  \frac{1}{ह}\int_x^{x+h} च(ट)\,डट = च(ग)।
  \]
\item
  As \(h \to 0\), \(c \to x\), and since \(f\) is continuous:

  \[ इति
  \lim_{h\to 0} च(ग) = च(क्स) ।
  \]
\end{enumerate}

Thus, \(F'(x) = f(x)\).

\paragraph{Proof Sketch of Part II}\label{proof-sketch-of-part-ii}

\begin{enumerate}
\def\labelenumi{\arabic{enumi}.}
\item
  Let \(F\) be an antiderivative of \(f\), so \(F'(x) = f(x)\).
\item
  By Part I, the function

  \[ इति
  G(x) = \int_a^x f(t)\,dt
  \]

  is also an antiderivative of \(f\).
\item
  Since \(F\) and \(G\) differ only by a constant,

  \[ इति
  F(x) = G(x) + C.
  \]
\item
  Evaluating at the endpoints:

  \[ इति
  \int_a^b f(x)\,dx = G(b)-G(a) = (F(b)+C)-(F(a)+C) = F(b)-F(a).
  \]
\end{enumerate}

\subsubsection{C.3 Proof Sketch: Convergence of the Geometric
Series}\label{c.3-proof-sketch-convergence-of-the-geometric-series}

The geometric series is one of the simplest and most important infinite
series. It serves as a model for understanding convergence and is the
foundation for many later results in calculus.

\paragraph{The Series}\label{the-series}

\[ इति
\sum_{n=0}^\infty ar^n = अ + अर् + अर्^2 + अर्^3 + \cdots
\]

where \(a\) is the first term and \(r\) is the common ratio.

\paragraph{Partial Sum Formula}\label{partial-sum-formula}

The \(n\)-th partial sum is

\[ इतिS_n = अ + अर् + अर्^2 + \cdots + ar^n.
\]

Multiply both sides by \(r\):

\[ इति
rS_n = अर् + अर्^2 + \cdots + अर्^{n+1}।
\]

Subtract the two equations:

\[ इति
S_n - rS_n = अ - अर्^{न+1}।
\]

\[
S_n(1-r) = a(1-r^{n+1})।
\]

So

\[ इति
S_n = \frac{a(1-r^{n+1})}{1-r}, \quad r \neq 1.
\]

\paragraph{Convergence}\label{convergence}

Take the limit as \(n \to \infty\):

\begin{itemize}
\item
  If \(|r| < 1\), then \(r^{n+1} \to 0\).

  \[ इति
  \lim_{n\to\infty} S_n = \frac{a}{1-r}।
  \]
\item
  If \(|r| \geq 1\), then \(r^{n+1}\) does not go to 0. The series
  diverges.
\end{itemize}

\paragraph{Result}\label{result}

\[ इति
\sum_{n=0}^\infty ar^n =
\प्रारम्भ{प्रकरण}
\dfrac{a}{1-r}, & |r|<1, \\[6pt]
\text{विचलति}, & |r|\geq 1.
\अन्त{प्रकरण}
\]

\subsection{Appendix D. Applications and
Connections}\label{appendix-d.-applications-and-connections}

\subsubsection{D.1 Physics Connections: Velocity, Acceleration, and
Work}\label{d.1-physics-connections-velocity-acceleration-and-work}

Calculus was originally developed to solve problems in physics -
especially motion and change. Here are some of the most important
connections.

\paragraph{Position, Velocity, and
Acceleration}\label{position-velocity-and-acceleration}

\begin{itemize}
\item
  Position function: \(s(t)\) gives the location of an object at time
  \(t\).
\item
  Velocity: the derivative of position.

  \[ इति
  व(त) = स'(त) = \frac{ds}{dt}
  \]
\item
  Acceleration: the derivative of velocity (or second derivative of
  position).

  \[ इति
  a(t) = v'(t) = s''(t) = \frac{d^2s}{dt^2}
  \]
\end{itemize}

Example: If \(s(t) = 4t^2\) meters, then:

\[ इति
v(t) = 8t, \quad a(t) = 8.
\]

So the object moves faster linearly with time, under constant
acceleration.

\paragraph{Work and Force}\label{work-and-force}

In physics, work is the product of force and distance. If force varies
with position, calculus gives:

\[ इति
W = \int_a^b F(x)\, dx
\]

where \(F(x)\) is the force at position \(x\), and the object moves from
\(x=a\) to \(x=b\).

Example: A spring with Hooke's law force \(F(x) = kx\) requires work

\[ इति
W = \int_0^d kx\, dx = \frac{1}{2}kd^2
\] इति

वसन्तं दूरं तानयितुं \(d\) इति ।

\paragraph{\texorpdfstring{ऊर्जा तथा वक्र के अन्तर्गत क्षेत्र- गतिज ऊर्जा:
\(E_k = \tfrac{1}{2}mv^2\).}{ऊर्जा तथा वक्र के अन्तर्गत क्षेत्र- गतिज ऊर्जा: E\_k = \textbackslash tfrac\{1\}\{2\}mv\^{}2.}}\label{ux90aux930ux91c-ux924ux925-ux935ux915ux930-ux915-ux905ux928ux924ux930ux917ux924-ux915ux937ux924ux930--ux917ux924ux91c-ux90aux930ux91c-e_k-tfrac12mv2.}

\begin{itemize}
\tightlist
\item
  सम्भाव्य ऊर्जायां प्रायः अभिन्नाः (उदा. गुरुत्वाकर्षणबलात् गुरुत्वाकर्षणविभवशक्तिः)
  सम्मिलिताः भवन्ति ।
\item
  सामान्यतया बलकार्यस्य एकीकरणेन ऊर्जा संगृहीतं वा कृतं कार्यं वा प्राप्यते ।
\end{itemize}

\paragraph{त्वरित अभ्यास}\label{ux924ux935ux930ux924-ux905ux92dux92fux938}

\begin{enumerate}
\def\labelenumi{\arabic{enumi}.}
\tightlist
\item
  यदि \(s(t) = t^3 - 3t\) तर्हि \(v(t)\) तथा \(a(t)\) ज्ञातव्यम्।
\item
  10 N इत्यस्य नित्यबलेन कृतं कार्यं 5 मी.
\item
  वसन्तस्य नित्यं \(k=200\) भवति। ०.१ मी.पर्यन्तं प्रसारयितुं कियत् कार्यं आवश्यकम् ?
\item
  त्वरणं स्थानस्य द्वितीयव्युत्पन्नं इति दर्शयतु।
\item
  अभिन्न \(\int v(t)\, dt\) विस्थापनेन सह कथं सम्बद्धः इति व्याख्यातव्यम्।
\end{enumerate}

\subsubsection{D.2 संभाव्यता तथा सांख्यिकी
सम्बन्ध}\label{d.2-ux938ux92dux935ux92fux924-ux924ux925-ux938ux916ux92fux915-ux938ux92eux92cux928ux927}

विशेषतः निरन्तरयादृच्छिकचरैः सह व्यवहारं कुर्वन् गणितः संभाव्यतायाः सांख्यिकीयाः च सह
गहनतया सम्बद्धः अस्ति । संभाव्यतां, औसतं, अपेक्षां च परिभाषितुं अभिन्नाः अत्यावश्यकाः
भवन्ति ।

\paragraph{संभाव्यता घनत्व कार्य
(PDFs)}\label{ux938ux92dux935ux92fux924-ux918ux928ux924ux935-ux915ux930ux92f-pdfs}

निरन्तरयादृच्छिकचरस्य \(X\) कृते संभाव्यताः संभाव्यताघनत्वफलकेन \(f(x)\) द्वारा
वर्णिताः भवन्ति:

\begin{enumerate}
\def\labelenumi{\arabic{enumi}.}
\item
  \(f(x) \geq 0\) सर्वेषां \(x\) कृते।
\item
  कुलसंभावना 1 इत्यस्य बराबरम् अस्ति : .

  \[
  \int_{-\infty}^{\infty} f(x)\, dx = 1.
  \] इति
\end{enumerate}

\(X\) अन्तराल \([a,b]\) मध्ये अस्ति इति संभावना अस्ति

\[
P(a \leq X \leq b) = \int_a^b f(x)\, dx.
\] इति

\paragraph{अपेक्षित मूल्य
(मध्यम)}\label{ux905ux92aux915ux937ux924-ux92eux932ux92f-ux92eux927ux92fux92e}

अपेक्षितं मूल्यं (सरासरीफलं) अस्ति

\[
E[X] = \int_{-\infty}^{\infty} x f(x)\, dx.
\] इति

एतत् भारितसरासरीयाः गणितसंस्करणम् अस्ति ।

\paragraph{विचरण}\label{ux935ux91aux930ux923}

विचरणमापाः प्रसारिताः : १.

\[
\text{Var}(X) = E[(X-\mu)^2] = \int_{-\infty}^{\infty} (x-\mu)^2 f(x)\, dx,
\] इति

यत्र \(\mu = E[X]\)।

\paragraph{सामान्य
वितरण}\label{ux938ux92eux928ux92f-ux935ux924ux930ux923}

\begin{enumerate}
\def\labelenumi{\arabic{enumi}.}
\item
  \([a,b]\) इत्यत्र एकरूपं वितरणम् :

  \[
  f(x) = \frac{1}{b-a}, \quad a \leq x \leq b.
  \] इति

  अर्थ: \(\frac{a+b}{2}\)।
\item
  पैरामीटर् \(\lambda > 0\) इत्यनेन सह घातीयवितरणं:

  \[ इति
  f(x) = \लम्बदा ई^{-\लम्बदा x}, \quad x \geq 0.\]

  Mean: \(1/\lambda\).
\item
  Normal (Gaussian) distribution:

  \[ इति
  f(x) = \frac{1}{\sqrt{2\पि\सिग्मा^2}} ई^{-(एक्स-\मु)^2/(2\सिग्मा^2)}।
  \]

  Integrals of this distribution connect to the error function.
\end{enumerate}

\paragraph{Why This Matters}\label{why-this-matters-3}

\begin{itemize}
\tightlist
\item
  Integrals turn probabilities into areas under curves.
\item
  Expectation and variance link calculus to averages and variability.
\item
  Most real-world data models (finance, physics, biology, AI) use these
  continuous probability distributions.
\end{itemize}

\paragraph{Quick Practice}\label{quick-practice}

\begin{enumerate}
\def\labelenumi{\arabic{enumi}.}
\tightlist
\item
  For \(f(x) = \tfrac{1}{2}\) on \([0,2]\), compute
  \(P(0.5 \leq X \leq 1.5)\).
\item
  For exponential distribution with \(\lambda = 2\), compute \(E[X]\).
\item
  Show that the total area under the standard normal curve equals 1.
\item
  Find the mean of a uniform distribution on \([3,7]\).
\item
  Explain why probabilities are computed with integrals, not sums, for
  continuous variables.
\end{enumerate}

\subsubsection{D.3 Computer Science Connections: Taylor Approximations
in
Algorithms}\label{d.3-computer-science-connections-taylor-approximations-in-algorithms}

Calculus is not only for physics - it also underpins many tools and
techniques in computer science. One of the clearest bridges is through
Taylor series, which provide efficient ways to approximate functions in
numerical computing and algorithms.

\paragraph{Function Approximation for
Computing}\label{function-approximation-for-computing}

Computers cannot directly store or calculate most functions exactly
(like \(e^x\), \(\sin x\), or \(\ln x\)). Instead, they use polynomial
approximations derived from Taylor expansions.

Example: To approximate \(e^x\), truncate the Maclaurin series:

\[ इति
e^x \प्रायः १ + x + \frac{x^2}{2!} + \frac{x^3}{3!}।
\]

लघु \(x\) कृते, एतत् बहुपदं केवलं कतिपयैः पदैः सह सटीकं परिणामं ददाति ।

\paragraph{एल्गोरिदम्स् मध्ये
दक्षता}\label{ux90fux932ux917ux930ux926ux92eux938-ux92eux927ux92f-ux926ux915ux937ux924}

\begin{itemize}
\tightlist
\item
  त्रिकोणमितीयकार्यम् : गणकयंत्रस्य CPU-इत्यस्य च एल्गोरिदम् प्रायः श्रृङ्खलाविस्तारस्य
  (अथवा चेबिशेव-बहुपदवत् भिन्नतायाः) उपयोगं कुर्वन्ति ।- घातीय/लघुगणकम् :
  संख्यात्मकपुस्तकालयेषु द्रुतसन्निकर्षस्य आधारः टेलरविस्तारः भवति ।
\item
  मूलनिष्कर्षः : न्यूटनस्य पद्धतिः रेखीयसन्निकर्षे आधारिता अस्ति, यत् टेलरश्रृङ्खलायाः
  (प्रथमव्युत्पन्नस्य) प्रत्यक्षप्रयोगः अस्ति ।
\end{itemize}

\paragraph{संख्यात्मक
विश्लेषण}\label{ux938ux916ux92fux924ux92eux915-ux935ux936ux932ux937ux923}

त्रुटिविश्लेषणे टेलरविस्ताराः केन्द्रस्थाः सन्ति : १.

\begin{itemize}
\item
  शेषसूत्रस्य उपयोगेन त्रुटिपदस्य अनुमानं करणम् : १.

  \[
  R_n(x) = \frac{f^{(n+1)}(c)}{(n+1)!}(x-a)^{n+1}.
  \] इति
\item
  एतेन दत्तस्य सटीकतायै कति पदानाम् आवश्यकता भवति इति ज्ञायते ।
\end{itemize}

\paragraph{मशीन लर्निंग
कनेक्शन}\label{ux92eux936ux928-ux932ux930ux928ux917-ux915ux928ux915ux936ux928}

\begin{itemize}
\tightlist
\item
  ढाल-आधारितं अनुकूलनं (ढाल-अवरोह इव) मापदण्डान् कुशलतया अद्यतनीकर्तुं व्युत्पन्नानाम्
  उपयोगं करोति ।
\item
  सक्रियकरणकार्यं (यथा \(\tanh x\) अथवा \(\sigma(x)=1/(1+e^{-x})\)) प्रायः
  बहुपदैः अथवा गतिकृते खण्डवारकार्यैः अनुमानितं भवति ।
\item
  श्रृङ्खलासन्निकर्षाः बाध्यवातावरणेषु प्रशिक्षणं अनुमानं च त्वरितुं शक्नुवन्ति।
\end{itemize}

\paragraph{एतत् किमर्थं
महत्त्वपूर्णम्}\label{ux90fux924ux924-ux915ux92eux930ux925-ux92eux939ux924ux924ux935ux92aux930ux923ux92e-33}

\begin{itemize}
\tightlist
\item
  टेलर सन्निकर्षाः असततगणनायाः सह निरन्तरगणितस्य सेतुम् अकुर्वन् ।
\item
  ते दर्शयन्ति यत् एल्गोरिदम्, संख्यात्मकविधिः, यन्त्रशिक्षणं च कथं गणितसंकल्पनानां
  उपयोगः भवति ।
\item
  अनुमानानाम् अवगमनेन गणनायाः कृते सङ्गणकानां उपरि अवलम्ब्य जालस्य परिहाराय सहायकं
  भवति ।
\end{itemize}

\paragraph{त्वरित
अभ्यास}\label{ux924ux935ux930ux924-ux905ux92dux92fux938-1}

\begin{enumerate}
\def\labelenumi{\arabic{enumi}.}
\tightlist
\item
  तस्य Maclaurin श्रृङ्खलायाः प्रथमत्रिपदानां उपयोगेन \(\sin(0.1)\) इत्यस्य अनुमानं
  कुरुत ।
\item
  डिग्री-3 बहुपदेन सह \(e^1\) इत्यस्य अनुमानं कर्तुं दोषस्य अनुमानं कर्तुं शेषपदस्य उपयोगं
  कुर्वन्तु ।
\item
  न्यूटनस्य पद्धत्या टेलरस्य प्रमेयस्य उपयोगः कथं भवति इति व्याख्यातव्यम्।
\item
  सङ्गणकाः कार्याणां कृते सटीकसूत्रेभ्यः बहुपदसन्निकर्षं किमर्थं प्राधान्यं ददति?
\item
  यन्त्रशिक्षणे अनुकूलनार्थं व्युत्पन्नं (ढालम्) किमर्थम् एतावत् महत्त्वपूर्णम् अस्ति ?
\end{enumerate}




\end{document}
