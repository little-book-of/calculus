% Options for packages loaded elsewhere
\PassOptionsToPackage{unicode}{hyperref}
\PassOptionsToPackage{hyphens}{url}
\PassOptionsToPackage{dvipsnames,svgnames,x11names}{xcolor}
%
\documentclass[
  letterpaper,
  DIV=11,
  numbers=noendperiod]{scrartcl}

\usepackage{amsmath,amssymb}
\usepackage{iftex}
\ifPDFTeX
  \usepackage[T1]{fontenc}
  \usepackage[utf8]{inputenc}
  \usepackage{textcomp} % provide euro and other symbols
\else % if luatex or xetex
  \usepackage{unicode-math}
  \defaultfontfeatures{Scale=MatchLowercase}
  \defaultfontfeatures[\rmfamily]{Ligatures=TeX,Scale=1}
\fi
\usepackage{lmodern}
\ifPDFTeX\else  
    % xetex/luatex font selection
\fi
% Use upquote if available, for straight quotes in verbatim environments
\IfFileExists{upquote.sty}{\usepackage{upquote}}{}
\IfFileExists{microtype.sty}{% use microtype if available
  \usepackage[]{microtype}
  \UseMicrotypeSet[protrusion]{basicmath} % disable protrusion for tt fonts
}{}
\makeatletter
\@ifundefined{KOMAClassName}{% if non-KOMA class
  \IfFileExists{parskip.sty}{%
    \usepackage{parskip}
  }{% else
    \setlength{\parindent}{0pt}
    \setlength{\parskip}{6pt plus 2pt minus 1pt}}
}{% if KOMA class
  \KOMAoptions{parskip=half}}
\makeatother
\usepackage{xcolor}
\setlength{\emergencystretch}{3em} % prevent overfull lines
\setcounter{secnumdepth}{-\maxdimen} % remove section numbering
% Make \paragraph and \subparagraph free-standing
\makeatletter
\ifx\paragraph\undefined\else
  \let\oldparagraph\paragraph
  \renewcommand{\paragraph}{
    \@ifstar
      \xxxParagraphStar
      \xxxParagraphNoStar
  }
  \newcommand{\xxxParagraphStar}[1]{\oldparagraph*{#1}\mbox{}}
  \newcommand{\xxxParagraphNoStar}[1]{\oldparagraph{#1}\mbox{}}
\fi
\ifx\subparagraph\undefined\else
  \let\oldsubparagraph\subparagraph
  \renewcommand{\subparagraph}{
    \@ifstar
      \xxxSubParagraphStar
      \xxxSubParagraphNoStar
  }
  \newcommand{\xxxSubParagraphStar}[1]{\oldsubparagraph*{#1}\mbox{}}
  \newcommand{\xxxSubParagraphNoStar}[1]{\oldsubparagraph{#1}\mbox{}}
\fi
\makeatother


\providecommand{\tightlist}{%
  \setlength{\itemsep}{0pt}\setlength{\parskip}{0pt}}\usepackage{longtable,booktabs,array}
\usepackage{calc} % for calculating minipage widths
% Correct order of tables after \paragraph or \subparagraph
\usepackage{etoolbox}
\makeatletter
\patchcmd\longtable{\par}{\if@noskipsec\mbox{}\fi\par}{}{}
\makeatother
% Allow footnotes in longtable head/foot
\IfFileExists{footnotehyper.sty}{\usepackage{footnotehyper}}{\usepackage{footnote}}
\makesavenoteenv{longtable}
\usepackage{graphicx}
\makeatletter
\newsavebox\pandoc@box
\newcommand*\pandocbounded[1]{% scales image to fit in text height/width
  \sbox\pandoc@box{#1}%
  \Gscale@div\@tempa{\textheight}{\dimexpr\ht\pandoc@box+\dp\pandoc@box\relax}%
  \Gscale@div\@tempb{\linewidth}{\wd\pandoc@box}%
  \ifdim\@tempb\p@<\@tempa\p@\let\@tempa\@tempb\fi% select the smaller of both
  \ifdim\@tempa\p@<\p@\scalebox{\@tempa}{\usebox\pandoc@box}%
  \else\usebox{\pandoc@box}%
  \fi%
}
% Set default figure placement to htbp
\def\fps@figure{htbp}
\makeatother

\KOMAoption{captions}{tableheading}
\makeatletter
\@ifpackageloaded{caption}{}{\usepackage{caption}}
\AtBeginDocument{%
\ifdefined\contentsname
  \renewcommand*\contentsname{Table of contents}
\else
  \newcommand\contentsname{Table of contents}
\fi
\ifdefined\listfigurename
  \renewcommand*\listfigurename{List of Figures}
\else
  \newcommand\listfigurename{List of Figures}
\fi
\ifdefined\listtablename
  \renewcommand*\listtablename{List of Tables}
\else
  \newcommand\listtablename{List of Tables}
\fi
\ifdefined\figurename
  \renewcommand*\figurename{Figure}
\else
  \newcommand\figurename{Figure}
\fi
\ifdefined\tablename
  \renewcommand*\tablename{Table}
\else
  \newcommand\tablename{Table}
\fi
}
\@ifpackageloaded{float}{}{\usepackage{float}}
\floatstyle{ruled}
\@ifundefined{c@chapter}{\newfloat{codelisting}{h}{lop}}{\newfloat{codelisting}{h}{lop}[chapter]}
\floatname{codelisting}{Listing}
\newcommand*\listoflistings{\listof{codelisting}{List of Listings}}
\makeatother
\makeatletter
\makeatother
\makeatletter
\@ifpackageloaded{caption}{}{\usepackage{caption}}
\@ifpackageloaded{subcaption}{}{\usepackage{subcaption}}
\makeatother

\ifLuaTeX
\usepackage[bidi=basic]{babel}
\else
\usepackage[bidi=default]{babel}
\fi
\babelprovide[main,import]{vietnamese}
% get rid of language-specific shorthands (see #6817):
\let\LanguageShortHands\languageshorthands
\def\languageshorthands#1{}
\usepackage{bookmark}

\IfFileExists{xurl.sty}{\usepackage{xurl}}{} % add URL line breaks if available
\urlstyle{same} % disable monospaced font for URLs
\hypersetup{
  pdftitle={Cuốn Sách Giải Tích Nhỏ},
  pdflang={vi},
  colorlinks=true,
  linkcolor={blue},
  filecolor={Maroon},
  citecolor={Blue},
  urlcolor={Blue},
  pdfcreator={LaTeX via pandoc}}


\title{Cuốn Sách Giải Tích Nhỏ}
\author{}
\date{}

\begin{document}
\maketitle


\section{Cuốn sách nhỏ về phép
tính}\label{cuux1ed1n-suxe1ch-nhux1ecf-vux1ec1-phuxe9p-tuxednh}

Phần giới thiệu ngắn gọn, thân thiện với người mới bắt đầu về những ý
tưởng cốt lõi của phép tính.

\subsection{Định dạng}\label{ux111ux1ecbnh-dux1ea1ng}

\begin{itemize}
\tightlist
\item
  \href{../artifacts/vi/book.pdf}{Tải xuống PDF} -- phiên bản sẵn sàng
  để in
\item
  \href{../artifacts/vi/book.epub}{Tải xuống EPUB} -- thân thiện với
  trình đọc sách điện tử
\item
  \href{../artifacts/vi/book.tex}{Xem LaTeX} -- Nguồn latex
\end{itemize}

\section{Phần 1. Giới hạn và đạo
hàm}\label{phux1ea7n-1.-giux1edbi-hux1ea1n-vuxe0-ux111ux1ea1o-huxe0m}

\section{Chương 1. Chức năng và giới
hạn}\label{chux1b0ux1a1ng-1.-chux1ee9c-nux103ng-vuxe0-giux1edbi-hux1ea1n}

\subsection{1.1 Chức năng}\label{chux1ee9c-nux103ng}

Hàm số là một trong những đối tượng cơ bản nhất trong toán học. Về bản
chất, hàm là một quy tắc lấy đầu vào và tạo ra chính xác một đầu ra. Hàm
cho phép chúng ta mô tả các mối quan hệ, mô hình hóa các hiện tượng
trong thế giới thực và xây dựng toàn bộ bộ máy tính toán.

\subsubsection{Sự định nghĩa}\label{sux1ef1-ux111ux1ecbnh-nghux129a}

Về mặt hình thức, một hàm \(f\) từ một tập \(X\) (gọi là miền) đến một
tập \(Y\) (gọi là codomain) được viết

\[
f : X \to Y.
\]

Với mỗi phần tử \(x \in X\), có một phần tử duy nhất \(f(x) \in Y\). Giá
trị \(f(x)\) được gọi là ảnh của \(x\) dưới \(f\).

Nếu \(y = f(x)\) thì \(y\) là đầu ra tương ứng với \(x\) đầu vào. Tập
hợp tất cả các kết quả đầu ra thực sự xuất hiện được gọi là phạm vi (tập
hợp con của tên miền).

\subsubsection{Ví dụ}\label{vuxed-dux1ee5}

\begin{enumerate}
\def\labelenumi{\arabic{enumi}.}
\tightlist
\item
  Hàm \(f(x) = x^2\) ánh xạ từng số thực \(x\) vào bình phương của nó.
\end{enumerate}

\begin{itemize}
\tightlist
\item
  Miền: toàn số thực \(\mathbb{R}\).

  \begin{itemize}
  \tightlist
  \item
    Codomain: toàn số thực \(\mathbb{R}\).
  \item
    Phạm vi: tất cả các số thực không âm \([0, \infty)\).
  \end{itemize}
\end{itemize}

\begin{enumerate}
\def\labelenumi{\arabic{enumi}.}
\setcounter{enumi}{1}
\tightlist
\item
  Hàm \(g(x) = \dfrac{1}{x}\) gán cho mỗi số thực khác 0 số nghịch đảo
  của nó.
\end{enumerate}

\begin{itemize}
\tightlist
\item
  Tên miền: \(\mathbb{R} \setminus \{0\}\).

  \begin{itemize}
  \tightlist
  \item
    Phạm vi: \(\mathbb{R} \setminus \{0\}\).
  \end{itemize}
\end{itemize}

\begin{enumerate}
\def\labelenumi{\arabic{enumi}.}
\setcounter{enumi}{2}
\tightlist
\item
  Một ví dụ thực tế: Gọi \(T(t)\) là nhiệt độ bên ngoài (tính bằng °C)
  tại thời điểm \(t\) (tính bằng giờ). Đây là một chức năng từ ``thời
  gian trong ngày'' đến ``nhiệt độ''.
\end{enumerate}

\subsubsection{Các cách biểu diễn
hàm}\label{cuxe1c-cuxe1ch-biux1ec3u-diux1ec5n-huxe0m}

Các hàm có thể được biểu diễn theo nhiều cách hữu ích:

\begin{itemize}
\tightlist
\item
  Công thức: ví dụ: \(f(x) = \sin x + x^2\).
\item
  Đồ thị: vẽ đồ thị tất cả các điểm \((x, f(x))\) trong mặt phẳng tọa
  độ.
\item
  Bảng: ghép nối đầu vào và đầu ra cho các bộ dữ liệu rời rạc.
\item
  Mô tả bằng lời: ``Chấm điểm cho từng học sinh.''
\end{itemize}

Mỗi cách trình bày nêu bật các khía cạnh khác nhau của cùng một chức
năng.

\subsubsection{Thuật ngữ}\label{thuux1eadt-ngux1eef}

\begin{itemize}
\tightlist
\item
  Biến độc lập: đầu vào (thường viết là \(x\)).
\item
  Biến phụ thuộc: kết quả đầu ra (thường viết là \(y\), trong đó
  \(y = f(x)\)).
\item
  Ký hiệu hàm: \(f(x)\) đọc là ``\(f\) của \(x\).''
\end{itemize}

\subsubsection{Tại sao Hàm số lại quan trọng trong Giải
tích}\label{tux1ea1i-sao-huxe0m-sux1ed1-lux1ea1i-quan-trux1ecdng-trong-giux1ea3i-tuxedch}

Giải tích là nghiên cứu về cách các hàm số thay đổi. Công cụ phái sinh
đo lường tốc độ thay đổi tức thời, trong khi tích phân đo lường tác động
tích lũy. Để nắm vững những ý tưởng này, trước tiên chúng ta cần hiểu
biết vững chắc về chức năng là gì và cách chúng hoạt động.

\subsubsection{Bài tập}\label{buxe0i-tux1eadp}

\begin{enumerate}
\def\labelenumi{\arabic{enumi}.}
\tightlist
\item
  Đối với hàm \(f(x) = 3x - 2\):
\end{enumerate}

\begin{itemize}
\tightlist
\item
  Tìm tên miền, tên miền và phạm vi.
\end{itemize}

\begin{enumerate}
\def\labelenumi{\arabic{enumi}.}
\setcounter{enumi}{1}
\item
  Hàm \(h(x) = \sqrt{x-1}\) được xác định cho đầu vào nào? Phạm vi của
  nó là gì?
\item
  Cho một ví dụ thực tế về một chức năng trong cuộc sống hàng ngày của
  bạn. Nêu rõ tên miền và codomain.
\item
  Vẽ đồ thị của \(f(x) = |x|\). Phạm vi là gì?
\item
  Giả sử \(g(x) = \dfrac{1}{x^2+1}\). Giải thích tại sao phạm vi của nó
  là khoảng \((0, 1]\).
\end{enumerate}

\subsection{1.2 Đồ thị và các phép biến
đổi}\label{ux111ux1ed3-thux1ecb-vuxe0-cuxe1c-phuxe9p-biux1ebfn-ux111ux1ed5i}

Một hàm có thể được hiểu không chỉ bằng công thức mà còn bằng đồ thị của
nó. Đồ thị của hàm \(f\) là tập hợp tất cả các cặp có thứ tự
\((x, f(x))\), trong đó \(x\) thuộc tập xác định của \(f\). Việc vẽ các
cặp này trong mặt phẳng tọa độ sẽ cho ta một hình ảnh về cách hoạt động
của hàm.

\subsubsection{Đồ thị cơ
bản}\label{ux111ux1ed3-thux1ecb-cux1a1-bux1ea3n}

Một số biểu đồ rất cơ bản cần được ghi nhớ:

\begin{itemize}
\tightlist
\item
  \(f(x) = x\): đường thẳng đi qua gốc tọa độ.
\item
  \(f(x) = x^2\): một parabol hướng lên trên.
\item
  \(f(x) = |x|\): đồ thị hình chữ ``V''.
\item
  \(f(x) = \frac{1}{x}\): một hyperbol có hai nhánh.
\item
  \(f(x) = \sin x\): đường cong tuần hoàn dạng sóng.
\end{itemize}

Chúng đóng vai trò là khối xây dựng cho các chức năng phức tạp hơn.

\subsubsection{Biến đổi}\label{biux1ebfn-ux111ux1ed5i}

Đồ thị có thể được dịch chuyển, kéo dài hoặc phản ánh bằng các quy tắc
đơn giản:

\begin{enumerate}
\def\labelenumi{\arabic{enumi}.}
\item
  Dịch chuyển theo chiều dọc: Thêm một hằng số sẽ làm đồ thị di chuyển
  lên hoặc xuống.

  \[
  y = f(x) + c \quad \text{is } f(x) \text{ shifted upward by } c.
  \]
\item
  Dịch chuyển theo chiều ngang: Việc thêm vào bên trong đối số sẽ di
  chuyển đồ thị sang trái hoặc sang phải.

  \[
  y = f(x - c) \quad \text{is } f(x) \text{ shifted right by } c.
  \]
\item
  Chia tỷ lệ theo chiều dọc: Nhân với một hằng số kéo dài hoặc nén đồ
  thị theo chiều dọc.

  \[
  y = a f(x), \quad a > 1 \text{ stretches; } 0 < a < 1 \text{ compresses.}
  \]
\item
  Chia tỷ lệ theo chiều ngang: Nhân bên trong đối số sẽ kéo dài hoặc nén
  đồ thị theo chiều ngang.

  \[
  y = f(bx), \quad b > 1 \text{ compresses toward the } y\text{-axis}.
  \]
\item
  Suy ngẫm:
\end{enumerate}

\begin{itemize}
\tightlist
\item
  \(y = -f(x)\): phản chiếu qua trục \(x\).

  \begin{itemize}
  \tightlist
  \item
    \(y = f(-x)\): phản chiếu qua trục \(y\).
  \end{itemize}
\end{itemize}

\subsubsection{Kết hợp các phép biến
đổi}\label{kux1ebft-hux1ee3p-cuxe1c-phuxe9p-biux1ebfn-ux111ux1ed5i}

Đồ thị phức tạp thường đến từ việc kết hợp một số phép biến đổi theo
trình tự. Ví dụ:

\[
y = 2(x-1)^2 + 3
\]

thu được bằng cách lấy parabol \(y = x^2\), dịch chuyển sang phải 1, kéo
dài theo chiều dọc thêm 2 và dịch chuyển lên trên 3.

\subsubsection{Bài tập}\label{buxe0i-tux1eadp-1}

\begin{enumerate}
\def\labelenumi{\arabic{enumi}.}
\tightlist
\item
  Vẽ đồ thị của \(y = (x+2)^2 - 1\). Xác định trình tự các phép biến đổi
  từ \(y = x^2\).
\item
  Điều gì xảy ra với đồ thị \(y = f(x)\) nếu chúng ta thay \(x\) bằng
  \(-x\)? Hãy thử với \(f(x) = \sqrt{x}\).
\item
  Mô tả các phép biến đổi biến \(y = \sin x\) thành
  \(y = 3\sin(x - \pi/4)\).
\item
  Vẽ đồ thị của \(y = |x-1| + 2\). Nêu đỉnh và hệ số góc của mỗi nhánh.
\item
  Với \(y = \frac{1}{x-2}\), hãy giải thích cách biến đổi đồ thị của
  \(y = \frac{1}{x}\).
\end{enumerate}

\subsection{1.3 Ý tưởng trực quan về giới
hạn}\label{uxfd-tux1b0ux1edfng-trux1ef1c-quan-vux1ec1-giux1edbi-hux1ea1n}

Trong nhiều trường hợp, giá trị của hàm tại một điểm ít quan trọng hơn
các giá trị mà nó lấy ở gần điểm đó. Khái niệm giới hạn nắm bắt được ý
tưởng này.

\subsubsection{Tiếp cận một giá
trị}\label{tiux1ebfp-cux1eadn-mux1ed9t-giuxe1-trux1ecb}

Hãy tưởng tượng bạn đang đi về phía một bức tường. Ngay cả trước khi bạn
chạm vào nó, bạn càng ngày càng tiến gần hơn. Theo cách tương tự, khi
\(x\) tiến đến một số \(a\), các giá trị của \(f(x)\) có thể tiến đến
một số \(L\) nào đó. Sau đó chúng tôi nói:

\[
\lim_{x \to a} f(x) = L.
\]

Điều này thể hiện ý tưởng rằng \(f(x)\) có thể được làm gần đến mức
chúng ta muốn với \(L\), chỉ bằng cách lấy \(x\) đủ gần với \(a\).

\subsubsection{Ví dụ}\label{vuxed-dux1ee5-1}

\begin{enumerate}
\def\labelenumi{\arabic{enumi}.}
\item
  Với \(f(x) = 2x + 3\): Như \(x \to 1\), \(f(x) \to 5\).
\item
  Với \(f(x) = \dfrac{\sin x}{x}\): Khi \(x \to 0\), hàm tiến tới 1, mặc
  dù \(f(0)\) không được xác định.
\item
  Với \(f(x) = \dfrac{1}{x}\): Như \(x \to 0^+\) (tiến dần từ bên phải),
  \(f(x) \to +\infty\). Như \(x \to 0^-\) (tiếp cận từ bên trái),
  \(f(x) \to -\infty\). Vì hành vi bên trái và bên phải khác nhau nên
  giới hạn ở mức 0 không tồn tại.
\end{enumerate}

\subsubsection{Tầm quan trọng của giới
hạn}\label{tux1ea7m-quan-trux1ecdng-cux1ee7a-giux1edbi-hux1ea1n}

\begin{itemize}
\tightlist
\item
  Chúng cho phép chúng ta định nghĩa các hàm tại những điểm mà ban đầu
  chúng không được định nghĩa.
\item
  Chúng nắm bắt hành vi gần những điểm gián đoạn và điểm kỳ dị.
\item
  Chúng tạo thành nền tảng cho đạo hàm (tốc độ thay đổi tức thời) và
  tích phân (diện tích là giới hạn của tổng).
\end{itemize}

\subsubsection{Giới hạn một
phía}\label{giux1edbi-hux1ea1n-mux1ed9t-phuxeda}

Đôi khi hành vi từ bên trái và bên phải phải được nghiên cứu riêng biệt:

\[
\lim_{x \to a^-} f(x), \quad \lim_{x \to a^+} f(x).
\]

Nếu cả hai đều đồng ý thì tồn tại giới hạn hai phía.

\subsubsection{Bài tập}\label{buxe0i-tux1eadp-2}

\begin{enumerate}
\def\labelenumi{\arabic{enumi}.}
\tightlist
\item
  Tính \(\lim_{x \to 2} (3x^2 - x)\).
\item
  \(\lim_{x \to 0} \frac{\sin x}{x}\) là gì? Sử dụng trực giác từ biểu
  đồ của \(\sin x\).
\item
  Đánh giá \(\lim_{x \to 0} |x|/x\). Giới hạn hai phía có tồn tại không?
\item
  Tìm \(\lim_{x \to \infty} \frac{1}{x}\). Giải thích kết quả này bằng
  lời.
\item
  Với \(f(x) = \frac{x^2-1}{x-1}\), \(\lim_{x \to 1} f(x)\) là bao
  nhiêu? So sánh với giá trị của \(f(1)\).
\end{enumerate}

\subsection{1.4 Định nghĩa chính thức về giới
hạn}\label{ux111ux1ecbnh-nghux129a-chuxednh-thux1ee9c-vux1ec1-giux1edbi-hux1ea1n}

Ý tưởng trực quan về giới hạn có thể được thực hiện chính xác bằng cách
sử dụng định nghĩa epsilon--delta. Điều này cho chúng ta một cách chặt
chẽ để nói rằng \(f(x)\) tiến gần đến một giá trị \(L\) khi \(x\) tiến
gần đến \(a\).

\subsubsection{Định nghĩa}\label{ux111ux1ecbnh-nghux129a}

Chúng tôi viết

\[
\lim_{x \to a} f(x) = L
\]

nếu điều kiện sau đây xảy ra:

Với mỗi \(\varepsilon > 0\) (dù nhỏ đến đâu), tồn tại một \(\delta > 0\)
sao cho bất cứ khi nào

\[
0 < |x - a| < \delta,
\]

nó theo sau đó

\[
|f(x) - L| < \varepsilon.
\]

Nói cách khác: chúng ta có thể làm cho \(f(x)\) càng gần \(L\) càng tốt,
miễn là \(x\) đủ gần với \(a\) (nhưng không bằng \(a\)).

\subsubsection{Ví dụ 1: Hàm tuyến
tính}\label{vuxed-dux1ee5-1-huxe0m-tuyux1ebfn-tuxednh}

Với \(f(x) = 2x + 1\), hãy chứng minh rằng \(\lim_{x \to 3} f(x) = 7\).

\begin{itemize}
\tightlist
\item
  Chúng tôi muốn \(|f(x) - 7| < \varepsilon\).
\item
  Nhưng \(f(x) - 7 = 2x + 1 - 7 = 2(x - 3)\).
\item
  Vậy \(|f(x) - 7| = 2|x - 3|\).
\item
  Nếu chọn \(\delta = \varepsilon / 2\) thì bất cứ khi nào
  \(|x - 3| < \delta\), ta có \(|f(x) - 7| < \varepsilon\). Điều này
  chứng tỏ giới hạn.
\end{itemize}

\subsubsection{Ví dụ 2: Hàm nghịch
đảo}\label{vuxed-dux1ee5-2-huxe0m-nghux1ecbch-ux111ux1ea3o}

Với \(f(x) = \frac{1}{x}\), hãy xem xét
\(\lim_{x \to 2} f(x) = \tfrac{1}{2}\).

\begin{itemize}
\tightlist
\item
  Chúng tôi muốn
  \(\left|\frac{1}{x} - \frac{1}{2}\right| < \varepsilon\).
\item
  Bất đẳng thức này yêu cầu thao tác đại số, nhưng có thể thỏa mãn bằng
  cách chọn \(\delta\) tùy thuộc vào \(\varepsilon\). Quá trình này phức
  tạp hơn, nhưng nguyên tắc thì giống nhau.
\end{itemize}

\subsubsection{Tại sao điều này lại quan
trọng}\label{tux1ea1i-sao-ux111iux1ec1u-nuxe0y-lux1ea1i-quan-trux1ecdng}

\begin{itemize}
\tightlist
\item
  Định nghĩa epsilon--delta đảm bảo rằng các giới hạn không mơ hồ hoặc
  chỉ dựa trên trực giác.
\item
  Là nền tảng của tính liên tục, đạo hàm và tích phân.
\item
  Mặc dù những người mới bắt đầu có thể thấy nó trừu tượng, nhưng việc
  làm việc với các ví dụ đơn giản sẽ tạo nên sự quen thuộc.
\end{itemize}

\subsubsection{Bài tập}\label{buxe0i-tux1eadp-3}

\begin{enumerate}
\def\labelenumi{\arabic{enumi}.}
\tightlist
\item
  Sử dụng định nghĩa epsilon--delta, chứng minh rằng
  \(\lim_{x \to 4} (x+1) = 5\).
\item
  Chứng minh rằng \(\lim_{x \to 0} 5x = 0\) bằng cách sử dụng định nghĩa
  hình thức.
\item
  Giải thích tại sao \(\lim_{x \to 0} \frac{1}{x}\) không tồn tại.
\item
  Với \(f(x) = x^2\), hãy chứng minh rằng \(\lim_{x \to 2} f(x) = 4\).
\item
  Bằng lời của bạn, hãy giải thích vai trò của \(\varepsilon\) và
  \(\delta\) trong định nghĩa về giới hạn.
\end{enumerate}

\subsection{1.5 Tính liên tục}\label{tuxednh-liuxean-tux1ee5c}

Một hàm số là liên tục nếu đồ thị của nó có thể vẽ được mà không cần
nhấc bút chì lên khỏi giấy. Chính xác hơn, tính liên tục đảm bảo rằng
những thay đổi nhỏ ở đầu vào sẽ tạo ra những thay đổi nhỏ ở đầu ra.

\subsubsection{Sự định nghĩa}\label{sux1ef1-ux111ux1ecbnh-nghux129a-1}

Hàm \(f\) liên tục tại điểm \(a\) nếu ba điều kiện được thỏa mãn:

\begin{enumerate}
\def\labelenumi{\arabic{enumi}.}
\tightlist
\item
  \(f(a)\) được xác định.
\item
  \(\lim_{x \to a} f(x)\) tồn tại.
\item
  \(\lim_{x \to a} f(x) = f(a)\).
\end{enumerate}

Nếu một hàm số liên tục tại mọi điểm trong một khoảng thì ta nói nó liên
tục trên khoảng đó.

\subsubsection{Ví dụ}\label{vuxed-dux1ee5-2}

\begin{enumerate}
\def\labelenumi{\arabic{enumi}.}
\item
  Hàm đa thức: Các hàm như \(f(x) = x^2 + 3x - 5\) liên tục ở mọi nơi
  trên \(\mathbb{R}\).
\item
  Hàm hữu tỷ: \(f(x) = \frac{1}{x-1}\) liên tục ở mọi nơi ngoại trừ tại
  \(x = 1\), nơi nó không được xác định.
\item
  Chức năng từng phần:

  \[
  f(x) =
  \begin{cases}
  x^2 & x < 1, \\
  2 & x = 1, \\
  x+1 & x > 1,
  \end{cases}
  \]
\end{enumerate}

Hàm này có một bước nhảy tại \(x = 1\), vì vậy nó không liên tục ở đó.

\subsubsection{Các loại gián
đoạn}\label{cuxe1c-loux1ea1i-giuxe1n-ux111oux1ea1n}

\begin{enumerate}
\def\labelenumi{\arabic{enumi}.}
\tightlist
\item
  Gián đoạn có thể tháo rời: Một ``lỗ hổng'' trên biểu đồ. Ví dụ:
  \(f(x) = \frac{x^2-1}{x-1}\) tại \(x=1\).
\item
  Nhảy không liên tục: Giới hạn bên trái và bên phải là khác nhau.
\item
  Gián đoạn vô hạn: Hàm tiến tới \(\pm\infty\) gần một điểm, như với
  \(f(x) = 1/x\) gần \(x = 0\).
\end{enumerate}

\subsubsection{Định lý giá trị trung
gian}\label{ux111ux1ecbnh-luxfd-giuxe1-trux1ecb-trung-gian}

Nếu một hàm số liên tục trên một khoảng \([a, b]\), thì với bất kỳ số
\(N\) nào nằm giữa \(f(a)\) và \(f(b)\), tồn tại một số \(c \in [a, b]\)
sao cho \(f(c) = N\).

Tính chất này rất quan trọng trong việc chứng minh sự tồn tại nghiệm và
nghiệm của phương trình.

\subsubsection{Bài tập}\label{buxe0i-tux1eadp-4}

\begin{enumerate}
\def\labelenumi{\arabic{enumi}.}
\tightlist
\item
  Quyết định xem hàm \(f(x) = |x|\) có liên tục tại \(x = 0\) hay không.
\item
  Xác định các điểm gián đoạn của \(f(x) = \frac{x+2}{x^2-1}\).
\item
  Giải thích tại sao mọi hàm đa thức đều liên tục tại mọi điểm.
\item
  Cho ví dụ về hàm số có bước nhảy gián đoạn. Hãy phác họa đồ thị của
  nó.
\item
  Sử dụng Định lý Giá trị Trung gian để chứng minh rằng phương trình
  \(x^3 + x - 1 = 0\) có nghiệm nằm trong khoảng từ 0 đến 1.
\end{enumerate}

\section{Chương 2. Công cụ phái
sinh}\label{chux1b0ux1a1ng-2.-cuxf4ng-cux1ee5-phuxe1i-sinh}

\subsection{2.1 Đạo hàm dưới dạng tỷ lệ thay
đổi}\label{ux111ux1ea1o-huxe0m-dux1b0ux1edbi-dux1ea1ng-tux1ef7-lux1ec7-thay-ux111ux1ed5i}

Đạo hàm là một trong những ý tưởng trung tâm của phép tính. Nó đo lường
cách một hàm thay đổi khi đầu vào của nó thay đổi - nói cách khác là tốc
độ thay đổi của đầu ra so với đầu vào.

\subsubsection{Tỷ lệ thay đổi trung
bình}\label{tux1ef7-lux1ec7-thay-ux111ux1ed5i-trung-buxecnh}

Đối với hàm \(f(x)\), tốc độ thay đổi trung bình giữa hai điểm \(x = a\)
và \(x = b\) là

\[
\frac{f(b) - f(a)}{b - a}.
\]

Đây là độ dốc của đường cát tuyến đi qua các điểm \((a, f(a))\) và
\((b, f(b))\).

\subsubsection{Tốc độ thay đổi tức
thời}\label{tux1ed1c-ux111ux1ed9-thay-ux111ux1ed5i-tux1ee9c-thux1eddi}

Để đo mức độ \(f(x)\) thay đổi nhanh như thế nào tại một điểm, chúng ta
để khoảng co lại:

\[
f'(a) = \lim_{h \to 0} \frac{f(a+h) - f(a)}{h}.
\]

Giới hạn này, nếu nó tồn tại, được gọi là đạo hàm của \(f\) tại \(a\).
Về mặt hình học, nó là độ dốc của đường tiếp tuyến với đồ thị \(f\) tại
điểm \((a, f(a))\).

\subsubsection{Ký hiệu}\label{kuxfd-hiux1ec7u}

\begin{itemize}
\tightlist
\item
  \(f'(x)\): ký hiệu nguyên tố.
\item
  \(\dfrac{dy}{dx}\): ký hiệu Leibniz, được sử dụng khi \(y = f(x)\).
\item
  \(Df(x)\): ký hiệu toán tử.
\end{itemize}

Tất cả những biểu tượng này đề cập đến cùng một khái niệm.

\subsubsection{Ví dụ}\label{vuxed-dux1ee5-3}

\begin{enumerate}
\def\labelenumi{\arabic{enumi}.}
\item
  Với \(f(x) = x^2\):

  \[
  f'(x) = \lim_{h \to 0} \frac{(x+h)^2 - x^2}{h} = \lim_{h \to 0} \frac{2xh + h^2}{h} = 2x.
  \]
\end{enumerate}

Độ dốc của parabol tại \(x\) là \(2x\).

\begin{enumerate}
\def\labelenumi{\arabic{enumi}.}
\setcounter{enumi}{1}
\item
  Với \(f(x) = \sin x\):

  \[
  f'(x) = \cos x.
  \]
\item
  Với \(f(x) = c\) (hằng số):

  \[
  f'(x) = 0.
  \]
\end{enumerate}

Hàm hằng không bao giờ thay đổi.

\subsubsection{Phiên dịch}\label{phiuxean-dux1ecbch}

\begin{itemize}
\tightlist
\item
  Trong vật lý: Nếu \(s(t)\) là vị trí thì \(s'(t)\) là vận tốc.
\item
  Trong kinh tế học: Nếu \(C(x)\) là chi phí thì \(C'(x)\) là chi phí
  biên.
\item
  Trong sinh học: Nếu \(P(t)\) là dân số thì \(P'(t)\) là tốc độ tăng
  trưởng.
\end{itemize}

Đạo hàm làm cho ``sự thay đổi'' trở nên chính xác trong nhiều bối cảnh.

\subsubsection{Bài tập}\label{buxe0i-tux1eadp-5}

\begin{enumerate}
\def\labelenumi{\arabic{enumi}.}
\tightlist
\item
  Tính \(f'(x)\) cho \(f(x) = 3x^2 - 2x + 1\).
\item
  Tìm độ dốc của đường tiếp tuyến với \(f(x) = x^3\) tại \(x = 2\).
\item
  Nếu \(s(t) = t^2 + 2t\) biểu thị khoảng cách tính bằng mét thì vận tốc
  tại \(t = 5\) là bao nhiêu?
\item
  Sử dụng định nghĩa giới hạn để tính đạo hàm của
  \(f(x) = \frac{1}{x}\).
\item
  Vẽ đồ thị \(y = x^2\) và vẽ tiếp tuyến tại \(x = 1\).
\end{enumerate}

\#\#2.2 Quy tắc phân biệt

Khi đạo hàm được xác định, chúng ta cần những cách hiệu quả để tính toán
nó. Các quy tắc lấy vi phân là những lối tắt giúp chúng ta không phải áp
dụng nhiều lần định nghĩa giới hạn.

\subsubsection{Quy tắc không
đổi}\label{quy-tux1eafc-khuxf4ng-ux111ux1ed5i}

Nếu \(f(x) = c\) trong đó \(c\) là hằng số thì

\[
f'(x) = 0.
\]

\subsubsection{Quy tắc quyền
lực}\label{quy-tux1eafc-quyux1ec1n-lux1ef1c}

Với \(f(x) = x^n\) trong đó \(n\) là số thực,

\[
\frac{d}{dx} \big( x^n \big) = n x^{n-1}.
\]

Ví dụ:

\begin{itemize}
\tightlist
\item
  \(\frac{d}{dx}(x^2) = 2x\).
\item
  \(\frac{d}{dx}(x^5) = 5x^4\).
\item
  \(\frac{d}{dx}(\sqrt{x}) = \frac{1}{2\sqrt{x}}\).
\end{itemize}

\subsubsection{Quy tắc bội số không
đổi}\label{quy-tux1eafc-bux1ed9i-sux1ed1-khuxf4ng-ux111ux1ed5i}

Nếu \(f(x) = c \cdot g(x)\), thì

\[
f'(x) = c \cdot g'(x).
\]

\subsubsection{Quy tắc tính tổng và
hiệu}\label{quy-tux1eafc-tuxednh-tux1ed5ng-vuxe0-hiux1ec7u}

\begin{itemize}
\tightlist
\item
  \((f + g)' = f' + g'\).
\item
  \((f - g)' = f' - g'\).
\end{itemize}

\subsubsection{Quy tắc sản phẩm}\label{quy-tux1eafc-sux1ea3n-phux1ea9m}

Đối với \(f(x)\) và \(g(x)\):

\[
(fg)' = f'g + fg'.
\]

Ví dụ: Nếu \(f(x) = x^2\), \(g(x) = \sin x\):

\[
(fg)' = (2x)(\sin x) + (x^2)(\cos x).
\]

\subsubsection{Quy tắc thương
số}\label{quy-tux1eafc-thux1b0ux1a1ng-sux1ed1}

Đối với \(f(x)\) và \(g(x)\):

\[
\left(\frac{f}{g}\right)' = \frac{f'g - fg'}{g^2}, \quad g(x) \neq 0.
\]

Ví dụ: Nếu \(f(x) = x^2\), \(g(x) = x+1\):

\[
\left(\frac{x^2}{x+1}\right)' = \frac{(2x)(x+1) - (x^2)(1)}{(x+1)^2}.
\]

\subsubsection{Đạo hàm của hàm số
chung}\label{ux111ux1ea1o-huxe0m-cux1ee7a-huxe0m-sux1ed1-chung}

\begin{itemize}
\tightlist
\item
  \(\frac{d}{dx}(\sin x) = \cos x\).
\item
  \(\frac{d}{dx}(\cos x) = -\sin x\).
\item
  \(\frac{d}{dx}(e^x) = e^x\).
\item
  \(\frac{d}{dx}(\ln x) = \frac{1}{x}, \quad x > 0\).
\end{itemize}

\subsubsection{Bài tập}\label{buxe0i-tux1eadp-6}

\begin{enumerate}
\def\labelenumi{\arabic{enumi}.}
\tightlist
\item
  Đạo hàm \(f(x) = 7x^3 - 4x + 9\).
\item
  Sử dụng quy tắc tích để tìm đạo hàm của \(f(x) = x^2 e^x\).
\item
  Áp dụng quy tắc thương cho \(f(x) = \frac{\sin x}{x}\).
\item
  Tính \(\frac{d}{dx}(\ln(x^2))\) bằng cách sử dụng chuỗi quy tắc.
\item
  Chứng minh rằng đạo hàm của \(f(x) = \frac{1}{x}\) là
  \(-\frac{1}{x^2}\).
\end{enumerate}

\subsection{2.3 Quy tắc dây
chuyền}\label{quy-tux1eafc-duxe2y-chuyux1ec1n}

Thông thường, các hàm được xây dựng bằng cách kết hợp các hàm đơn giản
hơn với nhau. Để phân biệt các hàm tổng hợp như vậy, chúng tôi sử dụng
quy tắc dây chuyền.

\subsubsection{Quy tắc}\label{quy-tux1eafc}

Nếu \(y = f(g(x))\), thì

\[
\frac{dy}{dx} = f'(g(x)) \cdot g'(x).
\]

Nói cách khác: lấy hàm số ngoài, giữ nguyên phần bên trong, sau đó nhân
với đạo hàm của hàm bên trong.

\subsubsection{Ví dụ}\label{vuxed-dux1ee5-4}

\begin{enumerate}
\def\labelenumi{\arabic{enumi}.}
\item
  Bình phương của hàm tuyến tính

  \[
  y = (3x+2)^2
  \]
\end{enumerate}

Hàm ngoài: \(f(u) = u^2\), hàm trong: \(g(x) = 3x+2\).

\[
   y' = 2(3x+2) \cdot 3 = 6(3x+2).
   \]

\begin{enumerate}
\def\labelenumi{\arabic{enumi}.}
\setcounter{enumi}{1}
\item
  Hàm mũ với phương trình bậc hai bên trong

  \[
  y = e^{x^2}
  \]
\end{enumerate}

Hàm ngoài: \(f(u) = e^u\), hàm trong: \(g(x) = x^2\).

\[
   y' = e^{x^2} \cdot 2x = 2x e^{x^2}.
   \]

\begin{enumerate}
\def\labelenumi{\arabic{enumi}.}
\setcounter{enumi}{2}
\item
  Logarit có gốc bên trong

  \[
  y = \ln(\sqrt{x})
  \]
\end{enumerate}

Bên ngoài: \(f(u) = \ln u\), bên trong: \(g(x) = \sqrt{x}\).

\[
   y' = \frac{1}{\sqrt{x}} \cdot \frac{1}{2\sqrt{x}} = \frac{1}{2x}.
   \]

\subsubsection{Quy tắc chuỗi tổng
quát}\label{quy-tux1eafc-chuux1ed7i-tux1ed5ng-quuxe1t}

Đối với nhiều hàm lồng nhau \(y = f(g(h(x)))\):

\[
\frac{dy}{dx} = f'(g(h(x))) \cdot g'(h(x)) \cdot h'(x).
\]

Điều này mở rộng một cách tự nhiên đến các tác phẩm sâu hơn.

\subsubsection{Tại sao Quy tắc Chuỗi lại quan
trọng}\label{tux1ea1i-sao-quy-tux1eafc-chuux1ed7i-lux1ea1i-quan-trux1ecdng}

\begin{itemize}
\tightlist
\item
  Nó xử lý gần như tất cả các mô hình trong thế giới thực trong đó một
  đại lượng phụ thuộc vào một đại lượng khác một cách gián tiếp.
\item
  Nó kết nối phép tính với vật lý (ví dụ: vận tốc phụ thuộc vào thời
  gian qua vị trí).
\item
  Cần thiết trong việc phân biệt ngầm và các chủ đề nâng cao.
\end{itemize}

\subsubsection{Bài tập}\label{buxe0i-tux1eadp-7}

\begin{enumerate}
\def\labelenumi{\arabic{enumi}.}
\tightlist
\item
  Đạo hàm \(y = (5x^2 + 1)^3\).
\item
  Tìm \(\frac{d}{dx}(\sin(3x))\).
\item
  Tính \(\frac{d}{dx}(\ln(1+x^2))\).
\item
  Đạo hàm \(y = \cos^2(x)\).
\item
  Áp dụng quy tắc chuỗi tổng quát cho \(y = e^{\sin(x^2)}\).
\end{enumerate}

\subsection{2.4 Sự khác biệt tiềm
ẩn}\label{sux1ef1-khuxe1c-biux1ec7t-tiux1ec1m-ux1ea9n}

Không phải tất cả các hàm đều có dạng \(y = f(x)\). Đôi khi \(x\) và
\(y\) có liên hệ với nhau bằng một phương trình, và việc giải \(y\) một
cách rõ ràng là khó hoặc không thể. Trong những trường hợp như vậy,
chúng tôi sử dụng vi phân ngầm.

\subsubsection{Ý tưởng}\label{uxfd-tux1b0ux1edfng}

Nếu một phương trình bao gồm cả \(x\) và \(y\), chúng ta có thể lấy đạo
hàm cả hai vế theo \(x\), coi \(y\) là một hàm của \(x\). Mỗi lần chúng
tôi phân biệt một số hạng liên quan đến \(y\), chúng tôi nhân với
\(\frac{dy}{dx}\).

\subsubsection{Ví dụ 1: Hình
tròn}\label{vuxed-dux1ee5-1-huxecnh-truxf2n}

phương trình:

\[
x^2 + y^2 = 25
\]

Đạo hàm theo \(x\):

\[
2x + 2y \frac{dy}{dx} = 0.
\]

Giải \(\frac{dy}{dx}\):

\[
\frac{dy}{dx} = -\frac{x}{y}.
\]

Điều này cho biết độ dốc của tiếp tuyến với đường tròn tại bất kỳ điểm
nào.

\subsubsection{Ví dụ 2: Tích của
biến}\label{vuxed-dux1ee5-2-tuxedch-cux1ee7a-biux1ebfn}

phương trình:

\[
xy = 1
\]

Phân biệt:

\[
x \frac{dy}{dx} + y = 0.
\]

Vì thế,

\[
\frac{dy}{dx} = -\frac{y}{x}.
\]

\subsubsection{Ví dụ 3: Quan hệ lượng
giác}\label{vuxed-dux1ee5-3-quan-hux1ec7-lux1b0ux1ee3ng-giuxe1c}

phương trình:

\[
\sin(xy) = x
\]

Phân biệt:

\[
\cos(xy) \cdot \Big(y + x\frac{dy}{dx}\Big) = 1.
\]

Giải \(\frac{dy}{dx}\):

\[
\frac{dy}{dx} = \frac{1 - y\cos(xy)}{x\cos(xy)}.
\]

\subsubsection{Tại sao sự khác biệt tiềm ẩn lại hữu
ích}\label{tux1ea1i-sao-sux1ef1-khuxe1c-biux1ec7t-tiux1ec1m-ux1ea9n-lux1ea1i-hux1eefu-uxedch}

\begin{itemize}
\tightlist
\item
  Nhiều đường cong quan trọng (hình tròn, hình elip, hyperbol) được xác
  định ngầm một cách tự nhiên.
\item
  Nó cho phép chúng ta phân tích các phương trình mà không cần giải
  \(y\) trước tiên.
\item
  Đây là một bước quan trọng trong các chủ đề nâng cao hơn như tỷ lệ
  liên quan và phương trình vi phân.
\end{itemize}

\subsubsection{Bài tập}\label{buxe0i-tux1eadp-8}

\begin{enumerate}
\def\labelenumi{\arabic{enumi}.}
\tightlist
\item
  Cho đường cong \(x^2 + xy + y^2 = 7\), hãy tìm \(\frac{dy}{dx}\).
\item
  Đạo hàm ngầm định \(\cos(x) + \cos(y) = 1\).
\item
  Tìm hệ số góc của tiếp tuyến với \(x^3 + y^3 = 9\) tại điểm
  \((1, 2)\).
\item
  Cho \(x^2 + y^2 = 10\), hãy tính \(\frac{dy}{dx}\) khi
  \((x, y) = (1, 3)\).
\item
  Đạo hàm \(e^{xy} = x + y\) để tìm \(\frac{dy}{dx}\).
\end{enumerate}

\subsection{2.5 Đạo hàm bậc cao}\label{ux111ux1ea1o-huxe0m-bux1eadc-cao}

Cho đến nay, chúng ta đã nghiên cứu đạo hàm bậc nhất, đo lường tốc độ
thay đổi của hàm số. Nhưng bản thân các đạo hàm cũng có thể vi phân, tạo
ra các đạo hàm bậc cao hơn.

\subsubsection{Sự định nghĩa}\label{sux1ef1-ux111ux1ecbnh-nghux129a-2}

\begin{itemize}
\item
  Đạo hàm bậc hai của \(f\) là đạo hàm của đạo hàm:

  \[
  f''(x) = \frac{d}{dx}\left(f'(x)\right).
  \]
\item
  Tổng quát hơn, đạo hàm thứ \(n\)-th được viết là

  \[
  f^{(n)}(x) = \frac{d^n}{dx^n} f(x).
  \]
\end{itemize}

\subsubsection{Ví dụ}\label{vuxed-dux1ee5-5}

\begin{enumerate}
\def\labelenumi{\arabic{enumi}.}
\tightlist
\item
  \(f(x) = x^3\)
\end{enumerate}

\begin{itemize}
\tightlist
\item
  Đạo hàm cấp 1: \(f'(x) = 3x^2\).

  \begin{itemize}
  \tightlist
  \item
    Đạo hàm bậc hai: \(f''(x) = 6x\).
  \item
    Đạo hàm bậc ba: \(f^{(3)}(x) = 6\).
  \item
    Đạo hàm cấp 4: \(f^{(4)}(x) = 0\).
  \end{itemize}
\end{itemize}

\begin{enumerate}
\def\labelenumi{\arabic{enumi}.}
\setcounter{enumi}{1}
\tightlist
\item
  \(f(x) = \sin x\)
\end{enumerate}

\begin{itemize}
\tightlist
\item
  \(f'(x) = \cos x\).

  \begin{itemize}
  \tightlist
  \item
    \(f''(x) = -\sin x\).
  \item
    \(f^{(3)}(x) = -\cos x\).
  \item
    \(f^{(4)}(x) = \sin x\). Các đạo hàm lặp lại trong một chu kỳ có độ
    dài 4.
  \end{itemize}
\end{itemize}

\begin{enumerate}
\def\labelenumi{\arabic{enumi}.}
\setcounter{enumi}{2}
\tightlist
\item
  \(f(x) = e^x\)
\end{enumerate}

\begin{itemize}
\tightlist
\item
  Mọi đạo hàm đều là \(e^x\).
\end{itemize}

\subsubsection{Ứng dụng}\label{ux1ee9ng-dux1ee5ng}

\begin{itemize}
\item
  Tính lõm: Dấu của \(f''(x)\) cho biết đồ thị của \(f\) là lõm lên
  (\(f'' > 0\)) hay lõm xuống (\(f'' < 0\)).
\item
  Điểm uốn: Những điểm có \(f''(x) = 0\) và độ lõm thay đổi.
\item
  Chuyển động: Trong vật lý, nếu \(s(t)\) là vị trí:
\item
  \(s'(t)\) = vận tốc,

  \begin{itemize}
  \tightlist
  \item
    \(s''(t)\) = gia tốc,
  \item
    \(s^{(3)}(t)\) = giật (tốc độ thay đổi gia tốc).
  \end{itemize}
\item
  Xấp xỉ: Đạo hàm bậc cao xuất hiện trong chuỗi Taylor, dùng để xấp xỉ
  các hàm số.
\end{itemize}

\subsubsection{Bài tập}\label{buxe0i-tux1eadp-9}

\begin{enumerate}
\def\labelenumi{\arabic{enumi}.}
\tightlist
\item
  Tính bốn đạo hàm đầu tiên của \(f(x) = \cos x\).
\item
  Tìm \(f''(x)\) cho \(f(x) = x^4 - 2x^2 + 3\).
\item
  Với \(f(x) = e^{2x}\), hãy chứng minh rằng
  \(f^{(n)}(x) = 2^n e^{2x}\).
\item
  Xác định các khoảng trong đó \(f(x) = x^3 - 3x\) lõm lên và lõm xuống.
\item
  Nếu \(s(t) = t^3 - 6t^2 + 9t\), tìm vận tốc và gia tốc tại \(t = 2\).
\end{enumerate}

\section{Chương 3. Ứng dụng của công cụ phái
sinh}\label{chux1b0ux1a1ng-3.-ux1ee9ng-dux1ee5ng-cux1ee7a-cuxf4ng-cux1ee5-phuxe1i-sinh}

\subsection{3.1 Tiếp tuyến và pháp
tuyến}\label{tiux1ebfp-tuyux1ebfn-vuxe0-phuxe1p-tuyux1ebfn}

Một trong những ứng dụng đầu tiên của đạo hàm là tìm phương trình tiếp
tuyến và pháp tuyến của một đường cong. Những đường này nắm bắt hình
dạng cục bộ của hàm tại một điểm nhất định.

\subsubsection{Đường tiếp
tuyến}\label{ux111ux1b0ux1eddng-tiux1ebfp-tuyux1ebfn}

Đường tiếp tuyến của một đường cong \(y = f(x)\) tại một điểm
\((a, f(a))\) là đường thẳng ``chạm'' vào đồ thị ở đó và có cùng độ dốc
với đường cong.

Độ dốc của đường tiếp tuyến được cho bởi đạo hàm:

\[
m_{\text{tangent}} = f'(a).
\]

Do đó, phương trình của đường tiếp tuyến tại \((a, f(a))\) là

\[
y - f(a) = f'(a)(x - a).
\]

\subsubsection{Đường thường}\label{ux111ux1b0ux1eddng-thux1b0ux1eddng}

Đường thẳng vuông góc với đường tiếp tuyến tại cùng một điểm. Độ dốc của
nó là nghịch đảo âm của độ dốc tiếp tuyến:

\[
m_{\text{normal}} = -\frac{1}{f'(a)}.
\]

Vậy phương trình của đường chuẩn là

\[
y - f(a) = -\frac{1}{f'(a)} (x - a), \quad f'(a) \neq 0.
\]

\subsubsection{Ví dụ}\label{vuxed-dux1ee5-6}

\begin{enumerate}
\def\labelenumi{\arabic{enumi}.}
\tightlist
\item
  \(f(x) = x^2\) tại \(x = 1\).
\end{enumerate}

\begin{itemize}
\tightlist
\item
  \(f(1) = 1\), \(f'(x) = 2x\), do đó \(f'(1) = 2\).

  \begin{itemize}
  \tightlist
  \item
    Tiếp tuyến: \(y - 1 = 2(x - 1)\), hoặc \(y = 2x - 1\).
  \item
    Bình thường: độ dốc = \(-\tfrac{1}{2}\), nên phương trình là
    \(y - 1 = -\tfrac{1}{2}(x - 1)\).
  \end{itemize}
\end{itemize}

\begin{enumerate}
\def\labelenumi{\arabic{enumi}.}
\setcounter{enumi}{1}
\tightlist
\item
  \(f(x) = \sin x\) tại \(x = \tfrac{\pi}{4}\).
\end{enumerate}

\begin{itemize}
\tightlist
\item
  \(f(\tfrac{\pi}{4}) = \tfrac{\sqrt{2}}{2}\),
  \(f'(\tfrac{\pi}{4}) = \cos(\tfrac{\pi}{4}) = \tfrac{\sqrt{2}}{2}\).

  \begin{itemize}
  \tightlist
  \item
    Tiếp tuyến:
    \(y - \tfrac{\sqrt{2}}{2} = \tfrac{\sqrt{2}}{2}(x - \tfrac{\pi}{4})\).
  \end{itemize}
\end{itemize}

\subsubsection{Tại sao tiếp tuyến và chuẩn mực lại quan
trọng}\label{tux1ea1i-sao-tiux1ebfp-tuyux1ebfn-vuxe0-chuux1ea9n-mux1ef1c-lux1ea1i-quan-trux1ecdng}

\begin{itemize}
\tightlist
\item
  Tiếp tuyến xấp xỉ đường cong cục bộ (xấp xỉ tuyến tính).
\item
  Định mức rất hữu ích trong hình học, quang học (phản xạ/khúc xạ) và cơ
  học (hướng lực).
\item
  Cả hai đều có vai trò trong việc tối ưu hóa và nghiên cứu độ cong.
\end{itemize}

\subsubsection{Bài tập}\label{buxe0i-tux1eadp-10}

\begin{enumerate}
\def\labelenumi{\arabic{enumi}.}
\tightlist
\item
  Tìm các tiếp tuyến và pháp tuyến của \(y = x^3\) tại \(x = 2\).
\item
  Xác định tiếp tuyến và pháp tuyến của \(y = e^x\) tại \(x = 0\).
\item
  Với \(y = \ln x\), hãy tính đường tiếp tuyến tại \(x = 1\).
\item
  Đường tròn có \(x^2 + y^2 = 9\). Sử dụng đạo hàm ẩn để tìm độ dốc của
  tiếp tuyến tại \((0,3)\).
\item
  Vẽ đồ thị \(y = \sqrt{x}\) và vẽ các tiếp tuyến và pháp tuyến tại
  \(x = 4\).
\end{enumerate}

\subsection{3.2 Giá liên quan}\label{giuxe1-liuxean-quan}

Trong nhiều bài toán thực tế, hai hoặc nhiều đại lượng thay đổi theo
thời gian và tốc độ thay đổi của chúng có mối liên hệ với nhau. Các bài
toán về tỷ giá liên quan sử dụng đạo hàm để mô tả các mối quan hệ này.

\subsubsection{Cách tiếp cận
chung}\label{cuxe1ch-tiux1ebfp-cux1eadn-chung}

\begin{enumerate}
\def\labelenumi{\arabic{enumi}.}
\tightlist
\item
  Xác định các biến phụ thuộc vào thời gian \(t\).
\item
  Viết phương trình liên hệ các biến.
\item
  Tìm đạo hàm cả hai vế đối với \(t\), áp dụng quy tắc dây chuyền.
\item
  Thay thế các giá trị đã biết tại thời điểm đã cho.
\item
  Giải tìm tỷ lệ chưa biết.
\end{enumerate}

\subsubsection{Ví dụ 1: Mở rộng vòng
tròn}\label{vuxed-dux1ee5-1-mux1edf-rux1ed9ng-vuxf2ng-truxf2n}

Một đường tròn có bán kính \(r\), bán kính tăng với tốc độ
\(\frac{dr}{dt} = 2 \,\text{cm/s}\). Tìm tốc độ tăng diện tích
\(A = \pi r^2\) khi \(r = 5\).

Phân biệt:

\[
\frac{dA}{dt} = 2\pi r \frac{dr}{dt}.
\]

Thay thế:

\[
\frac{dA}{dt} = 2\pi (5)(2) = 20\pi \,\text{cm}^2/\text{s}.
\]

\subsubsection{Ví dụ 2: Thang
trượt}\label{vuxed-dux1ee5-2-thang-trux1b0ux1ee3t}

Một chiếc thang cao 10 ft dựa vào tường. Đáy trượt đi ở mức
\(\frac{dx}{dt} = 1 \,\text{ft/s}\). Đỉnh trượt xuống nhanh như thế nào
khi đáy cách tường 6 ft?

Phương trình: \(x^2 + y^2 = 100\), trong đó \(y\) là chiều cao.

Phân biệt:

\[
2x \frac{dx}{dt} + 2y \frac{dy}{dt} = 0.
\]

Tại \(x = 6\), \(y = 8\). Thay thế:

\[
2(6)(1) + 2(8)\frac{dy}{dt} = 0 \quad \Rightarrow \quad \frac{dy}{dt} = -\tfrac{6}{8} = -\tfrac{3}{4}.
\]

Vì vậy, đỉnh trượt xuống ở mức \(0,75 \,\text{ft/s}\).

\subsubsection{Ví dụ 3: Nước trong hình
nón}\label{vuxed-dux1ee5-3-nux1b0ux1edbc-trong-huxecnh-nuxf3n}

Người ta đổ nước vào một hình nón có chiều cao 12 cm và bán kính 6 cm.
Khi nước sâu 4 cm, mực nước đang dâng lên \(2 \,\text{cm/s}\). Khối
lượng đang tăng với tốc độ bao nhiêu?

Phương trình: \(V = \tfrac{1}{3}\pi r^2 h\). Sử dụng tính tương tự,
\(r = \tfrac{h}{2}\). Thay thế:

\[
V = \tfrac{1}{12}\pi h^3.
\]

Phân biệt:

\[
\frac{dV}{dt} = \tfrac{1}{4}\pi h^2 \frac{dh}{dt}.
\]

Tại \(h = 4\), \(\frac{dh}{dt} = 2\):

\[
\frac{dV}{dt} = \tfrac{1}{4}\pi (16)(2) = 8\pi \,\text{cm}^3/\text{s}.
\]

\subsubsection{Tại sao Giá liên quan lại quan
trọng}\label{tux1ea1i-sao-giuxe1-liuxean-quan-lux1ea1i-quan-trux1ecdng}

\begin{itemize}
\tightlist
\item
  Chúng mô tả chuyển động và sự thay đổi trong vật lý, kỹ thuật và sinh
  học.
\item
  Họ kết nối hình học với phép tính thông qua các quá trình phụ thuộc
  vào thời gian.
\item
  Họ đào tạo chúng tôi mô hình hóa các hệ thống động về mặt toán học.
\end{itemize}

\subsubsection{Bài tập}\label{buxe0i-tux1eadp-11}

\begin{enumerate}
\def\labelenumi{\arabic{enumi}.}
\tightlist
\item
  Một quả bóng bay được thổi phồng lên để bán kính của nó tăng lên
  \(0,5 \,\text{cm/s}\). Tìm thể tích của nó tăng nhanh như thế nào khi
  bán kính là 10 cm.
\item
  Một ô tô chạy về phía bắc với vận tốc 40 km/h và một ô tô khác chạy về
  phía đông với vận tốc 30 km/h. Khoảng cách giữa họ tăng nhanh như thế
  nào sau 2 giờ?
\item
  Một đèn chiếu cách bức tường 20 m chiếu vào một người cao 2 m đang
  bước đi với vận tốc 1,5 m/s. Độ dài bóng của người đó trên tường thay
  đổi nhanh như thế nào khi người đó cách nguồn sáng 5m?
\item
  Chiều dài cạnh của hình lập phương tăng với tốc độ 2 cm/s. Diện tích
  bề mặt tăng nhanh như thế nào khi cạnh 3 cm?
\item
  Cát được đổ lên đống tạo thành hình nón có bán kính luôn bằng chiều
  cao. Nếu chiều cao tăng với tốc độ 5 cm/s thì thể tích tăng với tốc độ
  bao nhiêu khi chiều cao là 10 cm?
\end{enumerate}

\subsection{3.3 Vấn đề tối ưu
hóa}\label{vux1ea5n-ux111ux1ec1-tux1ed1i-ux1b0u-huxf3a}

Các bài toán tối ưu hóa sử dụng đạo hàm để tìm giá trị lớn nhất hoặc nhỏ
nhất của hàm, thường dưới những ràng buộc nhất định. Những vấn đề này mô
hình hóa các tình huống mà chúng ta muốn tối đa hóa hiệu quả, lợi nhuận
hoặc diện tích hoặc giảm thiểu chi phí, khoảng cách hoặc thời gian.

\subsubsection{Các bước chung}\label{cuxe1c-bux1b0ux1edbc-chung}

\begin{enumerate}
\def\labelenumi{\arabic{enumi}.}
\tightlist
\item
  Hiểu rõ vấn đề: Xác định số lượng cần tối ưu.
\item
  Mô hình hàm: Viết hàm mục tiêu theo một biến.
\item
  Áp dụng ràng buộc: Sử dụng các điều kiện đã cho để rút gọn các biến.
\item
  Đạo hàm: Tính đạo hàm của hàm mục tiêu.
\item
  Tìm các điểm tới hạn: Giải \(f'(x) = 0\) hoặc trong đó \(f'(x)\) không
  xác định.
\item
  Kiểm tra cực đại/cực tiểu: Sử dụng phép kiểm tra đạo hàm bậc hai hoặc
  kiểm tra điểm cuối.
\item
  Giải thích kết quả: Nêu câu trả lời trong ngữ cảnh ban đầu.
\end{enumerate}

\subsubsection{Ví dụ 1: Diện tích tối đa của hình chữ
nhật}\label{vuxed-dux1ee5-1-diux1ec7n-tuxedch-tux1ed1i-ux111a-cux1ee7a-huxecnh-chux1eef-nhux1eadt}

Một hình chữ nhật có chu vi là 40. Kích thước nào giúp diện tích của nó
lớn nhất?

\begin{itemize}
\tightlist
\item
  Gọi chiều dài \(x\), chiều rộng \(y\). Ràng buộc:
  \(2x + 2y = 40 \Rightarrow y = 20 - x\).
\item
  Diện tích: \(A = xy = x(20 - x) = 20x - x^2\).
\item
  Đạo hàm: \(A'(x) = 20 - 2x\). Đặt bằng 0: \(x = 10\).
\item
  Khi đó \(y = 10\).
\item
  Diện tích tối đa: \(100\). Hình chữ nhật là hình vuông.
\end{itemize}

\subsubsection{Ví dụ 2: Giảm thiểu khoảng
cách}\label{vuxed-dux1ee5-2-giux1ea3m-thiux1ec3u-khoux1ea3ng-cuxe1ch}

Tìm điểm trên parabol \(y = x^2\) gần \((0,3)\) nhất.

\begin{itemize}
\tightlist
\item
  Khoảng cách bình phương: \(D(x) = (x-0)^2 + (x^2 - 3)^2\).
\item
  Khai triển:
  \(D(x) = x^2 + (x^2 - 3)^2 = x^2 + x^4 - 6x^2 + 9 = x^4 - 5x^2 + 9\).
\item
  Đạo hàm: \(D'(x) = 4x^3 - 10x\). Giải: \(x(4x^2 - 10) = 0\).
\item
  Lời giải: \(x = 0\), \(x = \pm \sqrt{2.5}\).
\item
  Kiểm tra sẽ đưa ra khoảng cách tối thiểu là \(x = \pm \sqrt{2.5}\).
\end{itemize}

\subsubsection{Ví dụ 3: Hộp có âm lượng tối
đa}\label{vuxed-dux1ee5-3-hux1ed9p-cuxf3-uxe2m-lux1b0ux1ee3ng-tux1ed1i-ux111a}

Một chiếc hộp không có phần trên được làm từ một miếng bìa cứng hình
vuông có cạnh 20 cm bằng cách cắt các hình vuông bằng nhau ở các góc và
gấp các cạnh lại. Tìm kích thước của vết cắt để tối đa hóa khối lượng.

\begin{itemize}
\tightlist
\item
  Cho kích thước cắt = \(x\). Khi đó kích thước:
  \((20 - 2x) \times (20 - 2x) \times x\).
\item
  Khối lượng: \(V(x) = x(20 - 2x)^2\).
\item
  Đạo hàm: \(V'(x) = (20 - 2x)(20 - 6x)\).
\item
  Điểm tới hạn: \(x = 10\) (khối lượng bằng 0) hoặc
  \(x = \tfrac{20}{6} \approx 3.33\).
\item
  Ở mức \(x \approx 3,33\), khối lượng đạt mức tối đa.
\end{itemize}

\subsubsection{Tại sao tối ưu hóa lại quan
trọng}\label{tux1ea1i-sao-tux1ed1i-ux1b0u-huxf3a-lux1ea1i-quan-trux1ecdng}

\begin{itemize}
\tightlist
\item
  Các kỹ sư sử dụng nó để thiết kế các kết cấu hiệu quả.
\item
  Doanh nghiệp sử dụng nó để tối đa hóa lợi nhuận hoặc giảm thiểu chi
  phí.
\item
  Các nhà khoa học sử dụng nó để mô hình hóa các hệ thống tự nhiên tìm
  kiếm sự cân bằng.
\end{itemize}

\subsubsection{Bài tập}\label{buxe0i-tux1eadp-12}

\begin{enumerate}
\def\labelenumi{\arabic{enumi}.}
\tightlist
\item
  Một người nông dân có 100 m hàng rào để bao quanh một thửa ruộng hình
  chữ nhật ven sông (vì vậy chỉ cần rào 3 mặt). Tìm kích thước tối đa
  hóa diện tích.
\item
  Tìm hai số dương có tổng bằng 20 và tích của chúng càng lớn càng tốt.
\item
  Một hình trụ được làm từ vật liệu có kích thước 100 cm\(^2\). Tìm kích
  thước thể tích lớn nhất.
\item
  Một sợi dây dài 10 m được cắt thành hai đoạn, một đoạn uốn thành hình
  vuông, đoạn kia uốn thành hình tròn. Nên cắt nó như thế nào để tối đa
  hóa tổng diện tích được bao bọc?
\item
  Xây một cái hộp kín có đáy vuông, thể tích 32 m\(^3\). Tìm các kích
  thước có diện tích bề mặt nhỏ nhất.
\end{enumerate}

\subsection{3.4 Điểm lõm và điểm
uốn}\label{ux111iux1ec3m-luxf5m-vuxe0-ux111iux1ec3m-uux1ed1n}

Đạo hàm không chỉ cho chúng ta biết về hệ số góc mà còn về hình dạng của
đồ thị. Đạo hàm thứ hai đặc biệt hữu ích trong việc tìm hiểu tính lõm và
xác định các điểm uốn.

\subsubsection{Độ lõm}\label{ux111ux1ed9-luxf5m}

\begin{itemize}
\item
  Hàm \(f(x)\) lõm trên một khoảng nếu \(f''(x) > 0\). Đồ thị uốn cong
  lên trên, giống như một cái cốc.
\item
  Hàm \(f(x)\) lõm xuống trên một khoảng nếu \(f''(x) < 0\). Đồ thị uốn
  cong xuống, giống như một cái cau mày.
\end{itemize}

Độ lõm mô tả độ dốc của hàm số thay đổi như thế nào: nếu độ dốc tăng thì
đồ thị lõm lên; nếu độ dốc giảm dần thì đồ thị lõm xuống.

\subsubsection{Điểm uốn}\label{ux111iux1ec3m-uux1ed1n}

Điểm uốn là một điểm trên đồ thị có độ lõm thay đổi.

\begin{itemize}
\tightlist
\item
  Nếu \(f''(x) = 0\) hoặc \(f''(x)\) không được xác định, thì điểm đó là
  ứng cử viên cho điểm uốn.
\item
  Để khẳng định mặt lõm phải đổi dấu ở hai bên điểm.
\end{itemize}

\subsubsection{Ví dụ}\label{vuxed-dux1ee5-7}

\begin{enumerate}
\def\labelenumi{\arabic{enumi}.}
\tightlist
\item
  \(f(x) = x^3\)
\end{enumerate}

\begin{itemize}
\tightlist
\item
  \(f''(x) = 6x\).

  \begin{itemize}
  \tightlist
  \item
    Tại \(x = 0\), \(f''(0) = 0\).
  \item
    Với \(x < 0\), \(f''(x) < 0\) → lõm xuống.
  \item
    Với \(x > 0\), \(f''(x) > 0\) → lõm lên.
  \item
    Như vậy \((0,0)\) là một điểm uốn.
  \end{itemize}
\end{itemize}

\begin{enumerate}
\def\labelenumi{\arabic{enumi}.}
\setcounter{enumi}{1}
\tightlist
\item
  \(f(x) = x^4\)
\end{enumerate}

\begin{itemize}
\tightlist
\item
  \(f''(x) = 12x^2\).

  \begin{itemize}
  \tightlist
  \item
    Tại \(x = 0\), \(f''(0) = 0\) nhưng độ lõm không đổi dấu (luôn ≥ 0).
  \item
    Không có điểm uốn.
  \end{itemize}
\end{itemize}

\subsubsection{Phác thảo lõm và đường
cong}\label{phuxe1c-thux1ea3o-luxf5m-vuxe0-ux111ux1b0ux1eddng-cong}

\begin{itemize}
\tightlist
\item
  Nếu \(f'(x) = 0\) và \(f''(x) > 0\), thì \(f\) có mức tối thiểu cục
  bộ.
\item
  Nếu \(f'(x) = 0\) và \(f''(x) < 0\), thì \(f\) có mức cực đại cục bộ.
\item
  Đây được gọi là phép thử đạo hàm bậc hai.
\end{itemize}

\subsubsection{Tại sao điều này lại quan
trọng}\label{tux1ea1i-sao-ux111iux1ec1u-nuxe0y-lux1ea1i-quan-trux1ecdng-1}

Điểm lõm và điểm uốn giúp chúng ta hiểu được ``hình dạng'' của đồ thị:
nơi chúng uốn cong, làm phẳng hoặc xoay. Những ý tưởng này là trọng tâm
trong việc phác thảo đường cong, vật lý (gia tốc) và kinh tế học (hiệu
suất giảm dần).

\subsubsection{Bài tập}\label{buxe0i-tux1eadp-13}

\begin{enumerate}
\def\labelenumi{\arabic{enumi}.}
\tightlist
\item
  Xác định khoảng lõm của \(f(x) = x^3 - 3x\). Tìm điểm uốn của nó.
\item
  Với \(f(x) = \ln(x)\), hãy xác định độ lõm và các điểm uốn có thể có.
\item
  Áp dụng phép thử đạo hàm bậc hai cho \(f(x) = x^2 e^{-x}\) để phân
  loại các điểm tới hạn.
\item
  Vẽ \(f(x) = \sin x\), đánh dấu các khoảng lõm và điểm uốn.
\item
  Giải thích tại sao \(f(x) = e^x\) không có điểm uốn.
\end{enumerate}

\subsection{3.5 Phác thảo đường
cong}\label{phuxe1c-thux1ea3o-ux111ux1b0ux1eddng-cong}

Vẽ đường cong là quá trình vẽ đồ thị của hàm số bằng cách sử dụng thông
tin từ đạo hàm của nó. Thay vì vẽ đồ thị nhiều điểm, chúng tôi phân tích
các đặc điểm chính: điểm chặn, đường tiệm cận, khoảng tăng/giảm và độ
lõm.

\subsubsection{Các bước vẽ đường
cong}\label{cuxe1c-bux1b0ux1edbc-vux1ebd-ux111ux1b0ux1eddng-cong}

\begin{enumerate}
\def\labelenumi{\arabic{enumi}.}
\tightlist
\item
  Miền: Xác định nơi hàm được xác định.
\item
  Điểm giao nhau: Tìm nơi đồ thị đi qua các trục.
\item
  Đường tiệm cận:
\end{enumerate}

\begin{itemize}
\tightlist
\item
  Các tiệm cận đứng xảy ra khi hàm số không xác định và có xu hướng tiến
  tới vô cùng.

  \begin{itemize}
  \tightlist
  \item
    Các tiệm cận ngang hoặc nghiêng mô tả hành vi cuối cùng là
    \(x \to \pm\infty\).
  \end{itemize}
\end{itemize}

\begin{enumerate}
\def\labelenumi{\arabic{enumi}.}
\setcounter{enumi}{3}
\tightlist
\item
  Đạo hàm bậc nhất \(f'(x)\):
\end{enumerate}

\begin{itemize}
\tightlist
\item
  Tích cực → hàm số ngày càng tăng.

  \begin{itemize}
  \tightlist
  \item
    Âm → hàm số giảm dần.
  \item
    Điểm 0 của \(f'(x)\) → điểm tới hạn (có thể là cực đại/cực tiểu).
  \end{itemize}
\end{itemize}

\begin{enumerate}
\def\labelenumi{\arabic{enumi}.}
\setcounter{enumi}{4}
\tightlist
\item
  Đạo hàm bậc hai \(f''(x)\):
\end{enumerate}

\begin{itemize}
\tightlist
\item
  Dương → lõm lên.

  \begin{itemize}
  \tightlist
  \item
    Âm → lõm xuống.
  \item
    Số không hoặc không xác định → điểm uốn có thể.
  \end{itemize}
\end{itemize}

\begin{enumerate}
\def\labelenumi{\arabic{enumi}.}
\setcounter{enumi}{5}
\tightlist
\item
  Kết hợp thông tin: Sử dụng tất cả các kết quả để phác họa một biểu đồ
  rõ ràng và chính xác.
\end{enumerate}

\subsubsection{\texorpdfstring{Ví dụ 1:
\(f(x) = x^3 - 3x\)}{Ví dụ 1: f(x) = x\^{}3 - 3x}}\label{vuxed-dux1ee5-1-fx-x3---3x}

\begin{itemize}
\item
  Tên miền: toàn số thực.
\item
  Chặn: tại \((0,0)\).
\item
  Đạo hàm: \(f'(x) = 3x^2 - 3 = 3(x-1)(x+1)\).
\item
  Tăng: \((-\infty, -1) \cup (1, \infty)\).

  \begin{itemize}
  \tightlist
  \item
    Giảm: \((-1, 1)\).
  \end{itemize}
\item
  Đạo hàm bậc hai: \(f''(x) = 6x\).
\item
  Lõm xuống khi \(x < 0\), lõm lên khi \(x > 0\).

  \begin{itemize}
  \tightlist
  \item
    Điểm uốn tại \((0,0)\).
  \end{itemize}
\item
  Hình dạng: đường cong chữ S với cực đại cục bộ tại \((-1, 2)\), cực
  tiểu cục bộ tại \((1, -2)\).
\end{itemize}

\subsubsection{\texorpdfstring{Ví dụ 2:
\(f(x) = \frac{1}{x}\)}{Ví dụ 2: f(x) = \textbackslash frac\{1\}\{x\}}}\label{vuxed-dux1ee5-2-fx-frac1x}

\begin{itemize}
\item
  Tên miền: \(x \neq 0\).
\item
  Đường tiệm cận đứng: \(x = 0\).
\item
  Tiệm cận ngang: \(y = 0\).
\item
  Đạo hàm: \(f'(x) = -\frac{1}{x^2}\) (luôn âm). Chức năng luôn giảm.
\item
  Đạo hàm bậc hai: \(f''(x) = \frac{2}{x^3}\).
\item
  Lõm lên khi \(x > 0\).

  \begin{itemize}
  \tightlist
  \item
    Lõm xuống khi \(x < 0\).
  \end{itemize}
\item
  Đồ thị: hyperbol có hai nhánh.
\end{itemize}

\subsubsection{Tại sao việc phác thảo đường cong lại hữu
ích}\label{tux1ea1i-sao-viux1ec7c-phuxe1c-thux1ea3o-ux111ux1b0ux1eddng-cong-lux1ea1i-hux1eefu-uxedch}

\begin{itemize}
\tightlist
\item
  Cung cấp cái nhìn sâu sắc về hành vi tổng thể của các chức năng mà
  không cần tính toán toàn diện.
\item
  Cần thiết trong các kỳ thi giải tích và các bài toán ứng dụng.
\item
  Cầu nối phân tích đại số và hiểu biết hình học.
\end{itemize}

\subsubsection{Bài tập}\label{buxe0i-tux1eadp-14}

\begin{enumerate}
\def\labelenumi{\arabic{enumi}.}
\tightlist
\item
  Vẽ đường cong \(f(x) = x^4 - 2x^2\). Xác định điểm cực đại, điểm cực
  tiểu và điểm uốn.
\item
  Phân tích và phác họa \(f(x) = \ln(x)\). Hiển thị các điểm chặn, tiệm
  cận và độ lõm.
\item
  Với \(f(x) = e^{-x}\), hãy mô tả sự tăng trưởng/phân rã, tiệm cận và
  độ lõm.
\item
  Vẽ đồ thị \(f(x) = \tan x\) trên đoạn \((- \pi, \pi)\). Đánh dấu các
  tiệm cận.
\item
  Sử dụng phép kiểm định đạo hàm cấp một và cấp hai để phân loại các
  điểm tới hạn của \(f(x) = x^3 - 6x^2 + 9x\).
\end{enumerate}

\#Phần II. tích phân

\section{Chương 4. Nguyên hàm và tích phân xác
định}\label{chux1b0ux1a1ng-4.-nguyuxean-huxe0m-vuxe0-tuxedch-phuxe2n-xuxe1c-ux111ux1ecbnh}

\subsection{4.1 Tích phân không xác
định}\label{tuxedch-phuxe2n-khuxf4ng-xuxe1c-ux111ux1ecbnh}

Tích phân không xác định là quá trình vi phân ngược lại. Nếu đạo hàm có
số đo thay đổi thì tích phân sẽ khôi phục hàm ban đầu từ tốc độ thay đổi
của nó.

\subsubsection{Sự định nghĩa}\label{sux1ef1-ux111ux1ecbnh-nghux129a-3}

Nếu \(F'(x) = f(x)\), thì

\[
\int f(x)\,dx = F(x) + C,
\]

trong đó \(C\) là hằng số tích phân.

Mọi tích phân không xác định biểu diễn một họ hàm số chỉ khác nhau một
hằng số, vì vi phân loại bỏ hằng số.

\subsubsection{Quy tắc cơ bản}\label{quy-tux1eafc-cux1a1-bux1ea3n}

\begin{enumerate}
\def\labelenumi{\arabic{enumi}.}
\tightlist
\item
  Quy tắc không đổi
\end{enumerate}

\[
\int c\,dx = cx + C.
\]

\begin{enumerate}
\def\labelenumi{\arabic{enumi}.}
\setcounter{enumi}{1}
\tightlist
\item
  Quy tắc quyền lực
\end{enumerate}

\[
\int x^n\,dx = \frac{x^{n+1}}{n+1} + C, \quad n \neq -1.
\]

\begin{enumerate}
\def\labelenumi{\arabic{enumi}.}
\setcounter{enumi}{2}
\tightlist
\item
  Quy tắc tính tổng
\end{enumerate}

\[
\int \big(f(x) + g(x)\big)\,dx = \int f(x)\,dx + \int g(x)\,dx.
\]

\begin{enumerate}
\def\labelenumi{\arabic{enumi}.}
\setcounter{enumi}{3}
\tightlist
\item
  Quy tắc bội số không đổi
\end{enumerate}

\[
\int c f(x)\,dx = c \int f(x)\,dx.
\]

\subsubsection{Tích phân chung}\label{tuxedch-phuxe2n-chung}

\begin{itemize}
\tightlist
\item
  \(\int e^x dx = e^x + C\)
\item
  \(\int \sin x dx = -\cos x + C\)
\item
  \(\int \cos x dx = \sin x + C\)
\item
  \(\int \frac{1}{x} dx = \ln|x| + C\)
\end{itemize}

\subsubsection{Ví dụ}\label{vuxed-dux1ee5-8}

\begin{enumerate}
\def\labelenumi{\arabic{enumi}.}
\item
  \(\int (3x^2 - 4)\,dx = x^3 - 4x + C\).
\item
  \(\int \cos(2x)\,dx = \tfrac{1}{2}\sin(2x) + C\).
\item
  \(\int \frac{1}{x}\,dx = \ln|x| + C\).
\end{enumerate}

\subsubsection{Phiên dịch}\label{phiuxean-dux1ecbch-1}

\begin{itemize}
\tightlist
\item
  Tích phân không xác định là nguyên hàm.
\item
  Chúng là nền tảng cho tích phân xác định, đo các đại lượng tích lũy
  như diện tích, khoảng cách và khối lượng.
\item
  Trong bối cảnh ứng dụng, tích hợp cho phép chúng ta chuyển từ tỷ lệ
  trở lại tổng số.
\end{itemize}

\subsubsection{Bài tập}\label{buxe0i-tux1eadp-15}

\begin{enumerate}
\def\labelenumi{\arabic{enumi}.}
\tightlist
\item
  Tìm \(\int (5x^4 + 2x)\,dx\).
\item
  Tính \(\int (e^x + 3)\,dx\).
\item
  Tìm nghiệm tổng quát của \(f'(x) = 6x\) bằng cách sử dụng tích phân.
\item
  Đánh giá \(\int \frac{2}{x}\,dx\).
\item
  Nếu vận tốc là \(v(t) = 4t\), hãy tìm hàm vị trí \(s(t)\).
\end{enumerate}

\subsection{4.2 Tích phân xác định dưới dạng diện
tích}\label{tuxedch-phuxe2n-xuxe1c-ux111ux1ecbnh-dux1b0ux1edbi-dux1ea1ng-diux1ec7n-tuxedch}

Trong khi tích phân không xác định đại diện cho họ các nguyên hàm thì
tích phân xác định cho một giá trị bằng số: diện tích tích lũy dưới
đường cong giữa hai điểm.

\subsubsection{Sự định nghĩa}\label{sux1ef1-ux111ux1ecbnh-nghux129a-4}

Đối với hàm \(f(x)\) được xác định trên \([a, b]\), tích phân xác định
là

\[
\int_a^b f(x)\,dx = \lim_{n \to \infty} \sum_{i=1}^n f(x_i^-) \,\Delta x,
\]

trong đó khoảng \([a, b]\) được chia thành các khoảng con \(n\) có chiều
rộng \(\Delta x\) và \(x_i^-\) là một điểm mẫu trong mỗi khoảng con.

Đây là giới hạn của tổng Riemann.

\subsubsection{Giải thích hình
học}\label{giux1ea3i-thuxedch-huxecnh-hux1ecdc}

\begin{itemize}
\tightlist
\item
  Nếu \(f(x) \geq 0\) trên \([a, b]\), thì \(\int_a^b f(x)\,dx\) bằng
  diện tích dưới đường cong \(y = f(x)\) từ \(x=a\) đến \(x=b\).
\item
  Nếu \(f(x)\) giảm xuống dưới trục \(x\), tích phân sẽ tính diện tích
  có dấu: các vùng bên dưới trục được tính là âm.
\end{itemize}

\subsubsection{Thuộc tính của Tích phân xác
định}\label{thuux1ed9c-tuxednh-cux1ee7a-tuxedch-phuxe2n-xuxe1c-ux111ux1ecbnh}

\begin{enumerate}
\def\labelenumi{\arabic{enumi}.}
\tightlist
\item
  Cộng tính theo khoảng thời gian
\end{enumerate}

\[
\int_a^c f(x)\,dx = \int_a^b f(x)\,dx + \int_b^c f(x)\,dx.
\]

\begin{enumerate}
\def\labelenumi{\arabic{enumi}.}
\setcounter{enumi}{1}
\tightlist
\item
  Đảo ngược giới hạn
\end{enumerate}

\[
\int_a^b f(x)\,dx = -\int_b^a f(x)\,dx.
\]

\begin{enumerate}
\def\labelenumi{\arabic{enumi}.}
\setcounter{enumi}{2}
\tightlist
\item
  Khoảng cách có độ rộng bằng 0
\end{enumerate}

\[
\int_a^a f(x)\,dx = 0.
\]

\begin{enumerate}
\def\labelenumi{\arabic{enumi}.}
\setcounter{enumi}{3}
\tightlist
\item
  Tuyến tính
\end{enumerate}

\[
\int_a^b \big( cf(x) + g(x)\big)\,dx = c\int_a^b f(x)\,dx + \int_a^b g(x)\,dx.
\]

\subsubsection{Ví dụ}\label{vuxed-dux1ee5-9}

\begin{enumerate}
\def\labelenumi{\arabic{enumi}.}
\item
  \(\int_0^2 x\,dx = \left[\tfrac{1}{2}x^2\right]_0^2 = 2.\) Đây là diện
  tích của tam giác vuông dưới đường thẳng \(y=x\).
\item
  \(\int_{-1}^1 x^3\,dx = 0.\) Hàm lẻ \(x^3\) có diện tích đối xứng
  triệt tiêu nhau.
\item
  \(\int_0^\pi \sin x\,dx = 2.\) Giá trị này bằng diện tích dưới một vòm
  của đường cong hình sin.
\end{enumerate}

\subsubsection{Tại sao điều này lại quan
trọng}\label{tux1ea1i-sao-ux111iux1ec1u-nuxe0y-lux1ea1i-quan-trux1ecdng-2}

\begin{itemize}
\tightlist
\item
  Tích phân xác định đo các đại lượng tích lũy: khoảng cách, khối lượng,
  năng lượng, xác suất.
\item
  Họ kết nối tính toán đại số với trực giác hình học.
\item
  Bước tiếp theo là Định lý cơ bản của Giải tích, kết nối tích phân xác
  định với nguyên hàm.
\end{itemize}

\subsubsection{Bài tập}\label{buxe0i-tux1eadp-16}

\begin{enumerate}
\def\labelenumi{\arabic{enumi}.}
\tightlist
\item
  Tính \(\int_0^3 (2x+1)\,dx\).
\item
  Tìm diện tích giữa \(y = x^2\) và trục \(x\) từ \(x = 0\) đến
  \(x = 2\).
\item
  Tính giá \(\int_{-2}^2 (x^2 - 1)\,dx\).
\item
  Chứng minh rằng \(\int_{-a}^a f(x)\,dx = 0\) nếu \(f(x)\) là số lẻ.
\item
  Ước tính \(\int_0^1 e^x\,dx\) bằng cách sử dụng tổng Riemann với các
  khoảng con \(n=4\) và điểm cuối bên phải.
\end{enumerate}

\subsection{4.3 Định lý cơ bản của phép
tính}\label{ux111ux1ecbnh-luxfd-cux1a1-bux1ea3n-cux1ee7a-phuxe9p-tuxednh}

Định lý cơ bản của phép tính (FTC) hợp nhất hai ý tưởng chính của phép
tính: vi phân và tích phân. Nó cho thấy rằng việc tìm diện tích và tìm
tỷ lệ thay đổi là hai mặt của một đồng tiền.

\subsubsection{Phần 1: Đạo hàm của một tích
phân}\label{phux1ea7n-1-ux111ux1ea1o-huxe0m-cux1ee7a-mux1ed9t-tuxedch-phuxe2n}

Nếu \(f\) liên tục trên \([a, b]\), hãy xác định

\[
F(x) = \int_a^x f(t)\,dt.
\]

Khi đó \(F\) khả vi và

\[
F'(x) = f(x).
\]

Nói cách khác: đạo hàm của hàm diện tích tích lũy chính là hàm ban đầu.

\subsubsection{Phần 2: Đánh giá tích phân xác
định}\label{phux1ea7n-2-ux111uxe1nh-giuxe1-tuxedch-phuxe2n-xuxe1c-ux111ux1ecbnh}

Nếu \(f\) liên tục trên \([a, b]\) và \(F\) là nguyên hàm bất kỳ của
\(f\), thì

\[
\int_a^b f(x)\,dx = F(b) - F(a).
\]

Điều này cho chúng ta biết rằng chúng ta có thể tính tích phân xác định
đơn giản bằng cách tìm nguyên hàm, thay vì tính giới hạn của tổng
Riemann.

\subsubsection{Ví dụ}\label{vuxed-dux1ee5-10}

\begin{enumerate}
\def\labelenumi{\arabic{enumi}.}
\tightlist
\item
  \(\int_0^2 x^2\,dx\).
\end{enumerate}

\begin{itemize}
\tightlist
\item
  Nguyên hàm: \(F(x) = \tfrac{1}{3}x^3\).

  \begin{itemize}
  \tightlist
  \item
    Áp dụng FTC: \(F(2) - F(0) = \tfrac{8}{3} - 0 = \tfrac{8}{3}.\)
  \end{itemize}
\end{itemize}

\begin{enumerate}
\def\labelenumi{\arabic{enumi}.}
\setcounter{enumi}{1}
\item
  Nếu \(F(x) = \int_1^x \cos t \, dt\), thì \(F'(x) = \cos x\).
\item
  \(\int_1^4 \frac{1}{x}\,dx\).
\end{enumerate}

\begin{itemize}
\tightlist
\item
  Nguyên hàm: \(\ln|x|\).

  \begin{itemize}
  \tightlist
  \item
    Áp dụng FTC: \(\ln 4 - \ln 1 = \ln 4.\)
  \end{itemize}
\end{itemize}

\subsubsection{Tại sao FTC lại quan
trọng}\label{tux1ea1i-sao-ftc-lux1ea1i-quan-trux1ecdng}

\begin{itemize}
\tightlist
\item
  Nó biến đổi tích phân từ một quá trình giới hạn thành một tính toán
  thực tế.
\item
  Khẳng định phép vi phân và tích phân là hai phép toán nghịch đảo.
\item
  Định lý trung tâm làm cho phép tính vi phân trở nên hữu ích trong toán
  học, khoa học và kỹ thuật.
\end{itemize}

\subsubsection{Bài tập}\label{buxe0i-tux1eadp-17}

\begin{enumerate}
\def\labelenumi{\arabic{enumi}.}
\tightlist
\item
  Đánh giá \(\int_0^3 (2x+1)\,dx\) bằng cách sử dụng FTC.
\item
  Nếu \(F(x) = \int_0^x e^t\,dt\), hãy tìm \(F'(x)\).
\item
  Tính \(\int_0^\pi \sin x \, dx\).
\item
  Chứng minh rằng nếu \(f'(x) = g(x)\) thì
  \(\int_a^b g(x)\,dx = f(b) - f(a)\).
\item
  Sử dụng FTC để giải thích tại sao diện tích bên dưới \(y = \cos x\) từ
  \(0\) đến \(\pi/2\) bằng 1.
\end{enumerate}

\subsection{4.4 Tính chất của tích
phân}\label{tuxednh-chux1ea5t-cux1ee7a-tuxedch-phuxe2n}

Tích phân xác định có một số tính chất quan trọng giúp nó linh hoạt và
hiệu quả trong các ứng dụng. Các tính chất này tuân theo định nghĩa là
giới hạn của tổng và từ Định lý cơ bản của Giải tích.

\subsubsection{Tính tuyến tính}\label{tuxednh-tuyux1ebfn-tuxednh}

Đối với các hàm \(f(x)\) và \(g(x)\) và các hằng số \(c, d\):

\[
\int_a^b \big(c f(x) + d g(x)\big)\,dx = c \int_a^b f(x)\,dx + d \int_a^b g(x)\,dx.
\]

Điều này cho phép chúng ta chia các tích phân phức tạp thành các phần
đơn giản hơn.

\subsubsection{Tính cộng theo khoảng thời
gian}\label{tuxednh-cux1ed9ng-theo-khoux1ea3ng-thux1eddi-gian}

Nếu \(a < c < b\) thì

\[
\int_a^b f(x)\,dx = \int_a^c f(x)\,dx + \int_c^b f(x)\,dx.
\]

Chúng ta có thể tính tích phân từng phần một.

\subsubsection{Đảo ngược giới
hạn}\label{ux111ux1ea3o-ngux1b0ux1ee3c-giux1edbi-hux1ea1n}

\[
\int_a^b f(x)\,dx = -\int_b^a f(x)\,dx.
\]

Hoán đổi giới hạn làm thay đổi dấu của tích phân.

\subsubsection{Thuộc tính so sánh}\label{thuux1ed9c-tuxednh-so-suxe1nh}

Nếu \(f(x) \leq g(x)\) với mọi \(x\) trong \([a, b]\), thì

\[
\int_a^b f(x)\,dx \leq \int_a^b g(x)\,dx.
\]

Điều này cho phép chúng ta so sánh các khu vực mà không cần tính toán
trực tiếp.

\subsubsection{Bất bình đẳng về giá trị tuyệt
đối}\label{bux1ea5t-buxecnh-ux111ux1eb3ng-vux1ec1-giuxe1-trux1ecb-tuyux1ec7t-ux111ux1ed1i}

\[
\left| \int_a^b f(x)\,dx \right| \leq \int_a^b |f(x)|\,dx.
\]

Tính chất này rất cần thiết trong phân tích và kiểm tra độ hội tụ.

\subsubsection{Tính đối xứng}\label{tuxednh-ux111ux1ed1i-xux1ee9ng}

\begin{itemize}
\item
  Nếu \(f(x)\) chẵn (đối xứng qua trục \(y\)):

  \[
  \int_{-a}^a f(x)\,dx = 2\int_0^a f(x)\,dx.
  \]
\item
  Nếu \(f(x)\) lẻ (đối xứng qua gốc tọa độ):

  \[
  \int_{-a}^a f(x)\,dx = 0.
  \]
\end{itemize}

\subsubsection{Ví dụ}\label{vuxed-dux1ee5-11}

\begin{enumerate}
\def\labelenumi{\arabic{enumi}.}
\item
  \(\int_0^2 (3x^2 + 4)\,dx = \int_0^2 3x^2\,dx + \int_0^2 4\,dx = 8 + 8 = 16.\)
\item
  Vì \(f(x) = x^3\) là số lẻ nên \(\int_{-1}^1 x^3\,dx = 0.\)
\item
  Vì \(f(x) = x^2\) là số chẵn nên
  \(\int_{-2}^2 x^2\,dx = 2\int_0^2 x^2\,dx = 2\cdot \tfrac{8}{3} = \tfrac{16}{3}.\)
\end{enumerate}

\subsubsection{Tại sao những thuộc tính này lại quan
trọng}\label{tux1ea1i-sao-nhux1eefng-thuux1ed9c-tuxednh-nuxe0y-lux1ea1i-quan-trux1ecdng}

\begin{itemize}
\tightlist
\item
  Họ đơn giản hóa việc tính toán.
\item
  Chúng bộc lộ những đặc điểm hình học và tính đối xứng của hàm số.
\item
  Họ cung cấp các công cụ lý thuyết để phân tích nâng cao hơn.
\end{itemize}

\subsubsection{Bài tập}\label{buxe0i-tux1eadp-18}

\begin{enumerate}
\def\labelenumi{\arabic{enumi}.}
\tightlist
\item
  Sử dụng tính đối xứng để đánh giá \(\int_{-5}^5 (x^4 - x^3)\,dx\).
\item
  Chứng minh rằng
  \(\int_1^4 (2x+3)\,dx = \int_1^2 (2x+3)\,dx + \int_2^4 (2x+3)\,dx\).
\item
  Tính giá \(\int_0^\pi \sin(x)\,dx\) và so sánh với
  \(\int_{-\pi}^\pi \sin(x)\,dx\).
\item
  Chứng minh rằng nếu \(f(x) \geq 0\) trên \([a, b]\), thì
  \(\int_a^b f(x)\,dx \geq 0\).
\item
  Tính \(\int_{-3}^3 (x^2 + 1)\,dx\) bằng cách sử dụng thuộc tính
  chẵn/lẻ.
\end{enumerate}

\section{Chương 5. Kỹ thuật tích
hợp}\label{chux1b0ux1a1ng-5.-kux1ef9-thuux1eadt-tuxedch-hux1ee3p}

\subsection{5.1 Thay người}\label{thay-ngux1b0ux1eddi}

Một trong những kỹ thuật tích phân hữu ích nhất là phương pháp thay thế,
còn được gọi là -u-substitution-. Đó là quá trình ngược lại của quy tắc
dây chuyền đối với các công cụ phái sinh.

\subsubsection{Ý tưởng}\label{uxfd-tux1b0ux1edfng-1}

Nếu tích phân chứa hàm tổng hợp, chúng ta có thể đơn giản hóa nó bằng
cách thay đổi các biến.

Về mặt hình thức, nếu \(u = g(x)\) là một hàm khả vi, thì

\[
\int f(g(x)) g'(x)\,dx = \int f(u)\,du.
\]

Sự thay thế này làm cho tích phân dễ tính hơn.

\subsubsection{Các bước thay
thế}\label{cuxe1c-bux1b0ux1edbc-thay-thux1ebf}

\begin{enumerate}
\def\labelenumi{\arabic{enumi}.}
\tightlist
\item
  Xác định hàm bên trong \(u = g(x)\) có đạo hàm cũng xuất hiện trong
  tích phân.
\item
  Tính \(du = g'(x)\,dx\).
\item
  Viết lại tích phân theo \(u\).
\item
  Tích phân đối với \(u\).
\item
  Thay lại \(u = g(x)\).
\end{enumerate}

\subsubsection{Ví dụ}\label{vuxed-dux1ee5-12}

\begin{enumerate}
\def\labelenumi{\arabic{enumi}.}
\item
  Thay thế đơn giản

  \[
  \int 2x \cos(x^2)\,dx
  \]
\end{enumerate}

Đặt \(u = x^2\), do đó \(du = 2x\,dx\). Khi đó tích phân trở thành
\(\int \cos u \,du = \sin u + C = \sin(x^2) + C\).

\begin{enumerate}
\def\labelenumi{\arabic{enumi}.}
\setcounter{enumi}{1}
\item
  Trường hợp logarit

  \[
  \int \frac{2x}{x^2+1}\,dx
  \]
\end{enumerate}

Đặt \(u = x^2 + 1\), do đó \(du = 2x\,dx\). Khi đó tích phân trở thành
\(\int \frac{1}{u}\,du = \ln|u| + C = \ln(x^2+1) + C\).

\begin{enumerate}
\def\labelenumi{\arabic{enumi}.}
\setcounter{enumi}{2}
\item
  Thay thế lượng giác

  \[
  \int \sin(3x)\,dx
  \]
\end{enumerate}

Cho \(u = 3x\), do đó \(du = 3\,dx\), do đó \(dx = \frac{du}{3}\). Tích
phân trở thành
\(\tfrac{1}{3}\int \sin u\,du = -\tfrac{1}{3}\cos u + C = -\tfrac{1}{3}\cos(3x) + C\).

\subsubsection{Tích phân xác định có thể thay
thế}\label{tuxedch-phuxe2n-xuxe1c-ux111ux1ecbnh-cuxf3-thux1ec3-thay-thux1ebf}

Khi tính tích phân xác định, chúng ta cũng phải thay đổi các giới hạn:

\[
\int_a^b f(g(x)) g'(x)\,dx = \int_{g(a)}^{g(b)} f(u)\,du.
\]

Ví dụ:

\[
\int_0^1 2x e^{x^2}\,dx.
\]

Cho \(u = x^2\), \(du = 2x\,dx\). Giới hạn: khi \(x=0, u=0\); khi
\(x=1, u=1\). Vậy tích phân trở thành

\[
\int_0^1 e^u\,du = e - 1.
\]

\subsubsection{Bài tập}\label{buxe0i-tux1eadp-19}

\begin{enumerate}
\def\labelenumi{\arabic{enumi}.}
\tightlist
\item
  Tính giá \(\int (x^2+1)^5 (2x)\,dx\).
\item
  Tính \(\int \frac{\cos x}{\sin x}\,dx\).
\item
  Đánh giá \(\int_0^\pi \sin(2x)\,dx\) bằng cách sử dụng thay thế.
\item
  Tìm \(\int e^{3x}\,dx\).
\item
  Tính \(\int \frac{1}{\sqrt{1+x^2}}\,dx\) bằng cách cho \(u = 1+x^2\).
\end{enumerate}

\subsection{5.2 Tích hợp theo bộ
phận}\label{tuxedch-hux1ee3p-theo-bux1ed9-phux1eadn}

Tích hợp từng phần là một kỹ thuật xuất phát từ quy tắc tích phân cho
đạo hàm. Nó giúp đánh giá các tích phân liên quan đến tích của các hàm
không dễ dàng được xử lý chỉ bằng phép thế.

\subsubsection{Công thức}\label{cuxf4ng-thux1ee9c}

Từ quy tắc sản phẩm:

\[
\frac{d}{dx}[u(x)v(x)] = u'(x)v(x) + u(x)v'(x).
\]

Tích phân cả hai vế cho công thức tích phân từng phần:

\[
\int u\,dv = uv - \int v\,du.
\]

Đây:

\begin{itemize}
\tightlist
\item
  \(u\) = hàm được chọn để lấy đạo hàm,
\item
  \(dv\) = phần còn lại của số nguyên cần tích phân.
\end{itemize}

\subsubsection{\texorpdfstring{Chọn \(u\) và
\(dv\)}{Chọn u và dv}}\label{chux1ecdn-u-vuxe0-dv}

Một hướng dẫn phổ biến là LIATE (Logarit, nghịch đảo lượng giác, đại số,
lượng giác, hàm mũ).

\begin{itemize}
\tightlist
\item
  Chọn \(u\) từ danh mục có mặt sớm nhất.
\item
  Chọn \(dv\) làm phần còn lại.
\end{itemize}

\subsubsection{Ví dụ}\label{vuxed-dux1ee5-13}

\begin{enumerate}
\def\labelenumi{\arabic{enumi}.}
\tightlist
\item
  Đa thức × hàm mũ
\end{enumerate}

\[
\int x e^x\,dx
\]

Giả sử \(u = x\), \(dv = e^x dx\). Khi đó \(du = dx\), \(v = e^x\).

\[
\int x e^x\,dx = x e^x - \int e^x dx = x e^x - e^x + C.
\]

\begin{enumerate}
\def\labelenumi{\arabic{enumi}.}
\setcounter{enumi}{1}
\tightlist
\item
  Đa thức × Lượng giác
\end{enumerate}

\[
\int x \cos x\,dx
\]

Cho \(u = x\), \(dv = \cos x dx\). Khi đó \(du = dx\), \(v = \sin x\).

\[
\int x \cos x\,dx = x \sin x - \int \sin x dx = x \sin x + \cos x + C.
\]

\begin{enumerate}
\def\labelenumi{\arabic{enumi}.}
\setcounter{enumi}{2}
\tightlist
\item
  Logarit
\end{enumerate}

\[
\int \ln x\,dx
\]

Cho \(u = \ln x\), \(dv = dx\). Khi đó \(du = \frac{1}{x}dx\),
\(v = x\).

\[
\int \ln x\,dx = x \ln x - \int 1 dx = x \ln x - x + C.
\]

\subsubsection{Ví dụ tích phân xác
định}\label{vuxed-dux1ee5-tuxedch-phuxe2n-xuxe1c-ux111ux1ecbnh}

\[
\int_0^1 x e^x\,dx
\]

Sử dụng kết quả trước đó: \(\int x e^x dx = (x-1)e^x\). Đánh giá:

\[
\big[(x-1)e^x\big]_0^1 = (0)e^1 - (-1)e^0 = 0 + 1 = 1.
\]

\subsubsection{Tại sao điều này lại quan
trọng}\label{tux1ea1i-sao-ux111iux1ec1u-nuxe0y-lux1ea1i-quan-trux1ecdng-3}

Việc tích hợp từng phần là rất quan trọng khi việc thay thế không thành
công, đặc biệt là với logarit, hàm lượng giác nghịch đảo và các tích
liên quan đến đa thức với hàm mũ hoặc hàm lượng giác.

\subsubsection{Bài tập}\label{buxe0i-tux1eadp-20}

\begin{enumerate}
\def\labelenumi{\arabic{enumi}.}
\tightlist
\item
  Tính giá \(\int x \sin x\,dx\).
\item
  Tìm \(\int e^x \cos x\,dx\).
\item
  Tính \(\int_1^2 \ln x\,dx\).
\item
  Tính giá \(\int x^2 e^x\,dx\).
\item
  Sử dụng tích phân từng phần để hiển thị
  \(\int \arctan x\,dx = x\arctan x - \tfrac{1}{2}\ln(1+x^2) + C\).
\end{enumerate}

\subsection{5.3 Tích phân lượng giác và Thay
thế}\label{tuxedch-phuxe2n-lux1b0ux1ee3ng-giuxe1c-vuxe0-thay-thux1ebf}

Nhiều tích phân liên quan đến các hàm lượng giác. Những điều này thường
có thể được đơn giản hóa bằng cách sử dụng danh tính hoặc bằng cách thực
hiện các phép thay thế đặc biệt.

\subsubsection{Tích phân lượng
giác}\label{tuxedch-phuxe2n-lux1b0ux1ee3ng-giuxe1c}

\begin{enumerate}
\def\labelenumi{\arabic{enumi}.}
\tightlist
\item
  lũy thừa của sin và cos
\end{enumerate}

\begin{itemize}
\tightlist
\item
  Nếu lũy thừa của sin là số lẻ: lưu một \(\sin x\), đổi phần còn lại
  bằng \(\sin^2x = 1 - \cos^2x\), và thay \(u = \cos x\).
\item
  Nếu lũy thừa cosin lẻ: lưu một \(\cos x\), đổi phần còn lại thành
  \(\cos^2x = 1 - \sin^2x\), và thay \(u = \sin x\).
\item
  Nếu cả hai đều chẵn: sử dụng đẳng thức nửa góc.
\end{itemize}

Ví dụ:

\[
\int \sin^3x \cos x \, dx
\]

Cho \(u = \sin x\), \(du = \cos x\,dx\):

\[
\int u^3\,du = \tfrac{u^4}{4} + C = \tfrac{\sin^4x}{4} + C.
\]

\begin{enumerate}
\def\labelenumi{\arabic{enumi}.}
\setcounter{enumi}{1}
\tightlist
\item
  Tích các hàm sin và cos có các góc khác nhau Sử dụng công thức tính
  tổng:
\end{enumerate}

\[
\sin A \cos B = \tfrac{1}{2}[\sin(A+B) + \sin(A-B)].
\]

Ví dụ:

\[
\int \sin(2x)\cos(3x)\,dx = \tfrac{1}{2}\int [\sin(5x) - \sin(x)]\,dx.
\]

\begin{enumerate}
\def\labelenumi{\arabic{enumi}.}
\setcounter{enumi}{2}
\tightlist
\item
  lũy thừa cát tuyến và tiếp tuyến
\end{enumerate}

\begin{itemize}
\tightlist
\item
  Nếu lũy thừa cát tuyến chẵn: lưu \(\sec^2x\), quy đổi phần còn lại
  bằng \(\sec^2x = 1 + \tan^2x\), và thay \(u = \tan x\).
\item
  Nếu lũy thừa của tiếp tuyến là số lẻ: lưu \(\sec^2x\), quy đổi phần
  còn lại bằng \(\tan^2x = \sec^2x - 1\), và thay \(u = \tan x\).
\end{itemize}

Ví dụ:

\[
\int \tan^3x \sec^2x \, dx
\]

Cho \(u = \tan x\), \(du = \sec^2x\,dx\):

\[
\int u^3\,du = \tfrac{u^4}{4} + C = \tfrac{\tan^4x}{4} + C.
\]

\subsubsection{Thay thế lượng
giác}\label{thay-thux1ebf-lux1b0ux1ee3ng-giuxe1c}

Đối với các tích phân liên quan đến \(\sqrt{a^2 - x^2}\),
\(\sqrt{a^2 + x^2}\) hoặc \(\sqrt{x^2 - a^2}\), hãy sử dụng các phép
thay thế đặc biệt:

\begin{enumerate}
\def\labelenumi{\arabic{enumi}.}
\tightlist
\item
  \(x = a \sin \theta\), với \(\sqrt{a^2 - x^2}\).
\item
  \(x = a \tan \theta\), với \(\sqrt{a^2 + x^2}\).
\item
  \(x = a \sec \theta\), với \(\sqrt{x^2 - a^2}\).
\end{enumerate}

Ví dụ:

\[
\int \sqrt{a^2 - x^2}\,dx
\]

Đặt \(x = a\sin\theta\), vì vậy \(dx = a\cos\theta\,d\theta\):

\[
\int \sqrt{a^2 - a^2\sin^2\theta}(a\cos\theta\,d\theta) = \int a^2 \cos^2\theta \, d\theta.
\]

Đơn giản hóa việc sử dụng danh tính nửa góc.

\subsubsection{Tại sao những kỹ thuật này lại quan
trọng}\label{tux1ea1i-sao-nhux1eefng-kux1ef9-thuux1eadt-nuxe0y-lux1ea1i-quan-trux1ecdng}

\begin{itemize}
\tightlist
\item
  Chuyển các dạng đại số khó thành dạng lượng giác dễ học.
\item
  Chúng đặc biệt hữu ích trong các bài toán liên quan đến diện tích, thể
  tích và độ dài cung.
\item
  Đặt nền móng cho các phương pháp tích hợp tiên tiến.
\end{itemize}

\subsubsection{Bài tập}\label{buxe0i-tux1eadp-21}

\begin{enumerate}
\def\labelenumi{\arabic{enumi}.}
\tightlist
\item
  Tính giá \(\int \sin^4x \cos^2x \, dx\).
\item
  Tính \(\int \sin(5x)\cos(2x)\,dx\).
\item
  Tính giá \(\int \tan^2x \sec^2x \, dx\).
\item
  Tìm \(\int \sqrt{9 - x^2}\,dx\) bằng cách sử dụng phép thay thế.
\item
  Chứng minh rằng
  \(\int \frac{dx}{\sqrt{x^2 + a^2}} = \ln|x + \sqrt{x^2 + a^2}| + C\)
  sử dụng \(x = a\tan\theta\).
\end{enumerate}

\subsection{5.4 Phân số một
phần}\label{phuxe2n-sux1ed1-mux1ed9t-phux1ea7n}

Khi tích phân các hàm hữu tỷ (tỷ lệ của đa thức), một phương pháp hiệu
quả là phân rã từng phần. Kỹ thuật này biểu diễn một phân số phức tạp
dưới dạng tổng của các phân số đơn giản hơn, dễ tích phân hơn.

\subsubsection{Ý tưởng}\label{uxfd-tux1b0ux1edfng-2}

Nếu \(R(x) = \frac{P(x)}{Q(x)}\) là một hàm hữu tỉ, trong đó bậc của
\(P(x)\) nhỏ hơn bậc của \(Q(x)\), thì chúng ta có thể phân tách
\(R(x)\) thành các phân số đơn giản hơn.

Những phần đơn giản hơn này tương ứng với các thừa số của mẫu số
\(Q(x)\).

\subsubsection{Các dạng thông
dụng}\label{cuxe1c-dux1ea1ng-thuxf4ng-dux1ee5ng}

\begin{enumerate}
\def\labelenumi{\arabic{enumi}.}
\tightlist
\item
  Yếu tố tuyến tính riêng biệt Nếu
\end{enumerate}

\[
\frac{1}{(x-a)(x-b)},
\]

sau đó phân hủy thành

\[
\frac{A}{x-a} + \frac{B}{x-b}.
\]

\begin{enumerate}
\def\labelenumi{\arabic{enumi}.}
\setcounter{enumi}{1}
\tightlist
\item
  Hệ số tuyến tính lặp lại Nếu mẫu số có \((x-a)^n\), thì các số hạng là
\end{enumerate}

\[
\frac{A_1}{x-a} + \frac{A_2}{(x-a)^2} + \dots + \frac{A_n}{(x-a)^n}.
\]

\begin{enumerate}
\def\labelenumi{\arabic{enumi}.}
\setcounter{enumi}{2}
\tightlist
\item
  Hệ số bậc hai bất khả quy Nếu mẫu số có \((x^2+bx+c)\), thì tử số là
  tuyến tính:
\end{enumerate}

\[
\frac{Ax+B}{x^2+bx+c}.
\]

\subsubsection{Ví dụ 1: Hệ số tuyến tính riêng
biệt}\label{vuxed-dux1ee5-1-hux1ec7-sux1ed1-tuyux1ebfn-tuxednh-riuxeang-biux1ec7t}

\[
\int \frac{1}{x^2 - 1}\,dx
\]

Mẫu số nhân tố: \((x-1)(x+1)\). Phân hủy:

\[
\frac{1}{x^2-1} = \frac{1}{2}\left(\frac{1}{x-1} - \frac{1}{x+1}\right).
\]

Tích hợp:

\[
\int \frac{1}{x^2 - 1}\,dx = \tfrac{1}{2}\ln\left|\frac{x-1}{x+1}\right| + C.
\]

\subsubsection{Ví dụ 2: Hệ số tuyến tính
lặp}\label{vuxed-dux1ee5-2-hux1ec7-sux1ed1-tuyux1ebfn-tuxednh-lux1eb7p}

\[
\int \frac{1}{(x-1)^2}\,dx
\]

Việc này đã đơn giản rồi:

\[
\int (x-1)^{-2}\,dx = -\frac{1}{x-1} + C.
\]

\subsubsection{Ví dụ 3: Hệ số bậc hai bất khả
quy}\label{vuxed-dux1ee5-3-hux1ec7-sux1ed1-bux1eadc-hai-bux1ea5t-khux1ea3-quy}

\[
\int \frac{x}{x^2+1}\,dx
\]

Thay \(u = x^2+1\), hoặc nhận ra tử số là đạo hàm của mẫu số.

\[
\int \frac{x}{x^2+1}\,dx = \tfrac{1}{2}\ln(x^2+1) + C.
\]

\subsubsection{Các bước phân tách từng
phần}\label{cuxe1c-bux1b0ux1edbc-phuxe2n-tuxe1ch-tux1eebng-phux1ea7n}

\begin{enumerate}
\def\labelenumi{\arabic{enumi}.}
\tightlist
\item
  Phân tích mẫu số.
\item
  Viết dạng phân số tổng quát.
\item
  Nhân với mẫu số để xóa phân số.
\item
  Giải các hằng số chưa biết.
\item
  Tích hợp từng thuật ngữ.
\end{enumerate}

\subsubsection{Tại sao điều này lại quan
trọng}\label{tux1ea1i-sao-ux111iux1ec1u-nuxe0y-lux1ea1i-quan-trux1ecdng-4}

\begin{itemize}
\tightlist
\item
  Chuyển đổi các hàm số hữu tỷ phức tạp thành dạng logarit hoặc arctang
  đơn giản.
\item
  Đặc biệt hữu ích trong các phương trình vi phân và phép biến đổi
  Laplace.
\item
  Cơ bản về tính toán và kỹ thuật nâng cao.
\end{itemize}

\subsubsection{Bài tập}\label{buxe0i-tux1eadp-22}

\begin{enumerate}
\def\labelenumi{\arabic{enumi}.}
\tightlist
\item
  Phân tách và tích phân \(\int \frac{3x+5}{x^2-1}\,dx\).
\item
  Tính giá \(\int \frac{1}{x^2(x+1)}\,dx\).
\item
  Tính \(\int \frac{2x+1}{x^2+2x+2}\,dx\).
\item
  Tìm \(\int \frac{1}{x^3 - x}\,dx\).
\item
  Chứng minh \(\int \frac{dx}{x^2+1} = \arctan x + C\) bằng cách sử dụng
  phân số một phần hoặc thay thế.
\end{enumerate}

\subsection{5.5 Tích phân không
đúng}\label{tuxedch-phuxe2n-khuxf4ng-ux111uxfang}

Một số tích phân không thể được tính trực tiếp vì khoảng là vô hạn hoặc
tích phân trở nên không bị chặn. Chúng được gọi là tích phân không đúng.
Chúng được xác định bằng cách sử dụng các giới hạn.

\subsubsection{Sự định nghĩa}\label{sux1ef1-ux111ux1ecbnh-nghux129a-5}

\begin{enumerate}
\def\labelenumi{\arabic{enumi}.}
\tightlist
\item
  Khoảng thời gian vô hạn
\end{enumerate}

\[
\int_a^\infty f(x)\,dx = \lim_{b \to \infty} \int_a^b f(x)\,dx.
\]

\[
\int_{-\infty}^a f(x)\,dx = \lim_{b \to -\infty} \int_b^a f(x)\,dx.
\]

\begin{enumerate}
\def\labelenumi{\arabic{enumi}.}
\setcounter{enumi}{1}
\tightlist
\item
  Tích phân không giới hạn Nếu \(f(x)\) có tiệm cận đứng tại \(c\), thì
\end{enumerate}

\[
\int_a^c f(x)\,dx = \lim_{t \to c^-} \int_a^t f(x)\,dx,
\]

\[
\int_c^b f(x)\,dx = \lim_{t \to c^+} \int_t^b f(x)\,dx.
\]

\subsubsection{Hội tụ và phân
kỳ}\label{hux1ed9i-tux1ee5-vuxe0-phuxe2n-kux1ef3}

\begin{itemize}
\tightlist
\item
  Nếu giới hạn tồn tại và hữu hạn thì tích phân suy rộng sẽ hội tụ.
\item
  Nếu giới hạn không tồn tại hoặc vô hạn thì tích phân suy rộng sẽ phân
  kỳ.
\end{itemize}

\subsubsection{Ví dụ}\label{vuxed-dux1ee5-14}

\begin{enumerate}
\def\labelenumi{\arabic{enumi}.}
\tightlist
\item
  Phân rã theo cấp số nhân
\end{enumerate}

\[
\int_1^\infty \frac{1}{x^2}\,dx = \lim_{b \to \infty} \Big[-\tfrac{1}{x}\Big]_1^b = 1.
\]

Điều này hội tụ.

\begin{enumerate}
\def\labelenumi{\arabic{enumi}.}
\setcounter{enumi}{1}
\tightlist
\item
  Chức năng điều hòa
\end{enumerate}

\[
\int_1^\infty \frac{1}{x}\,dx = \lim_{b \to \infty} \ln b.
\]

Điều này phân kỳ đến vô cùng.

\begin{enumerate}
\def\labelenumi{\arabic{enumi}.}
\setcounter{enumi}{2}
\tightlist
\item
  Đường tiệm cận tại 0
\end{enumerate}

\[
\int_0^1 \frac{1}{\sqrt{x}}\,dx = \lim_{t \to 0^+} \int_t^1 x^{-1/2}\,dx.
\]

\[
= \lim_{t \to 0^+} [2\sqrt{x}]_t^1 = 2.
\]

Điều này hội tụ.

\begin{enumerate}
\def\labelenumi{\arabic{enumi}.}
\setcounter{enumi}{3}
\tightlist
\item
  Đường tiệm cận tại 0 (phân kỳ)
\end{enumerate}

\[
\int_0^1 \frac{1}{x}\,dx = \lim_{t \to 0^+} \ln(1) - \ln(t).
\]

Điều này phân kỳ vì \(\ln(t) \to -\infty\).

\subsubsection{Kiểm tra so sánh các tích phân không
đúng}\label{kiux1ec3m-tra-so-suxe1nh-cuxe1c-tuxedch-phuxe2n-khuxf4ng-ux111uxfang}

\begin{itemize}
\tightlist
\item
  Nếu \(0 \leq f(x) \leq g(x)\) đối với \(x\) lớn và \(\int g(x)\,dx\)
  hội tụ, thì \(\int f(x)\,dx\) cũng hội tụ.
\item
  Nếu \(\int f(x)\,dx\) phân kỳ và \(f(x) \geq g(x) \geq 0\), thì
  \(\int g(x)\,dx\) cũng phân kỳ.
\end{itemize}

\subsubsection{Tại sao tích phân không đúng lại quan
trọng}\label{tux1ea1i-sao-tuxedch-phuxe2n-khuxf4ng-ux111uxfang-lux1ea1i-quan-trux1ecdng}

\begin{itemize}
\tightlist
\item
  Chúng mở rộng tích hợp tới các miền vô hạn và các chức năng không giới
  hạn.
\item
  Chúng rất cần thiết trong xác suất (phân bố liên tục), vật lý (trường
  hấp dẫn/điện trường) và phân tích Fourier.
\end{itemize}

\subsubsection{Bài tập}\label{buxe0i-tux1eadp-23}

\begin{enumerate}
\def\labelenumi{\arabic{enumi}.}
\tightlist
\item
  Xác định xem \(\int_1^\infty \frac{1}{x^p}\,dx\) có hội tụ với các giá
  trị khác nhau của \(p\) hay không.
\item
  Tính giá \(\int_0^\infty e^{-x}\,dx\).
\item
  Kiểm tra sự hội tụ của \(\int_0^1 \frac{1}{x^p}\,dx\) tùy thuộc vào
  \(p\).
\item
  Tính \(\int_{-\infty}^\infty \frac{1}{1+x^2}\,dx\).
\item
  Sử dụng phép kiểm tra so sánh để chứng tỏ rằng
  \(\int_1^\infty \frac{1}{x^2+1}\,dx\) hội tụ.
\end{enumerate}

\section{Chương 6. Ứng dụng của Tích
hợp}\label{chux1b0ux1a1ng-6.-ux1ee9ng-dux1ee5ng-cux1ee7a-tuxedch-hux1ee3p}

\subsection{6.1 Khu vực và Khối
lượng}\label{khu-vux1ef1c-vuxe0-khux1ed1i-lux1b0ux1ee3ng}

Một trong những ứng dụng quan trọng nhất của tích phân là tìm diện tích
dưới đường cong và thể tích của chất rắn.

\subsubsection{Vùng giữa các đường
cong}\label{vuxf9ng-giux1eefa-cuxe1c-ux111ux1b0ux1eddng-cong}

Nếu \(f(x) \geq g(x)\) trên \([a, b]\), thì diện tích giữa các đường
cong \(y=f(x)\) và \(y=g(x)\) là

\[
A = \int_a^b \big(f(x) - g(x)\big)\,dx.
\]

Ví dụ: Tìm diện tích giữa \(y=x^2\) và \(y=x\) trên \([0,1]\).

\[
A = \int_0^1 (x - x^2)\,dx = \left[\tfrac{1}{2}x^2 - \tfrac{1}{3}x^3\right]_0^1 = \tfrac{1}{6}.
\]

\subsubsection{Tập bằng cách
cắt}\label{tux1eadp-bux1eb1ng-cuxe1ch-cux1eaft}

Nếu một vật rắn có diện tích tiết diện \(A(x)\) tại vị trí \(x\) thì thể
tích là

\[
V = \int_a^b A(x)\,dx.
\]

\subsubsection{Tập cách mạng}\label{tux1eadp-cuxe1ch-mux1ea1ng}

Khi một vùng được quay quanh một trục, thể tích của chất rắn thu được có
thể được tìm thấy bằng tích phân.

\begin{enumerate}
\def\labelenumi{\arabic{enumi}.}
\tightlist
\item
  Phương pháp đĩa Nếu vùng bên dưới \(y=f(x)\), \(x\in[a,b]\), được xoay
  quanh trục \(x\):
\end{enumerate}

\[
V = \pi \int_a^b [f(x)]^2\,dx.
\]

\begin{enumerate}
\def\labelenumi{\arabic{enumi}.}
\setcounter{enumi}{1}
\tightlist
\item
  Phương pháp giặt Nếu vùng giữa \(y=f(x)\) và \(y=g(x)\) được xoay
  quanh trục \(x\):
\end{enumerate}

\[
V = \pi \int_a^b \Big([f(x)]^2 - [g(x)]^2\Big)\,dx.
\]

\begin{enumerate}
\def\labelenumi{\arabic{enumi}.}
\setcounter{enumi}{2}
\tightlist
\item
  Phương pháp vỏ Nếu vùng bên dưới \(y=f(x)\) được xoay quanh trục
  \(y\):
\end{enumerate}

\[
V = 2\pi \int_a^b x f(x)\,dx.
\]

\subsubsection{Ví dụ}\label{vuxed-dux1ee5-15}

\begin{enumerate}
\def\labelenumi{\arabic{enumi}.}
\tightlist
\item
  Phương pháp đĩa Xoay \(y=\sqrt{x}\), \(0 \leq x \leq 4\), quanh trục
  \(x\):
\end{enumerate}

\[
V = \pi \int_0^4 (\sqrt{x})^2\,dx = \pi \int_0^4 x\,dx = \pi \left[\tfrac{1}{2}x^2\right]_0^4 = 8\pi.
\]

\begin{enumerate}
\def\labelenumi{\arabic{enumi}.}
\setcounter{enumi}{1}
\tightlist
\item
  Phương pháp giặt Xoay vùng giữa \(y=\sqrt{x}\) và \(y=1\),
  \(0 \leq x \leq 1\), quanh trục \(x\):
\end{enumerate}

\[
V = \pi \int_0^1 \big((\sqrt{x})^2 - (1)^2\big)\,dx = \pi \int_0^1 (x-1)\,dx = -\tfrac{\pi}{2}.
\]

(Lấy giá trị tuyệt đối cho khối lượng: \(V = \tfrac{\pi}{2}\)).

\begin{enumerate}
\def\labelenumi{\arabic{enumi}.}
\setcounter{enumi}{2}
\tightlist
\item
  Phương pháp vỏ Xoay vùng dưới \(y=x\), \(0 \leq x \leq 1\), quanh trục
  \(y\):
\end{enumerate}

\[
V = 2\pi \int_0^1 x(x)\,dx = 2\pi \int_0^1 x^2\,dx = 2\pi \cdot \tfrac{1}{3} = \tfrac{2\pi}{3}.
\]

\subsubsection{Tại sao điều này lại quan
trọng}\label{tux1ea1i-sao-ux111iux1ec1u-nuxe0y-lux1ea1i-quan-trux1ecdng-5}

\begin{itemize}
\tightlist
\item
  Cung cấp các cách chính xác để tính diện tích và thể tích trong hình
  học.
\item
  Cần thiết trong vật lý, kỹ thuật và xác suất.
\item
  Giới thiệu tư duy hình học tích hợp.
\end{itemize}

\subsubsection{Bài tập}\label{buxe0i-tux1eadp-24}

\begin{enumerate}
\def\labelenumi{\arabic{enumi}.}
\tightlist
\item
  Tìm diện tích giữa \(y=\cos x\) và \(y=\sin x\) trên \([0, \pi/2]\).
\item
  Tính thể tích của vật rắn được hình thành bằng cách quay \(y=x^2\),
  \(0 \leq x \leq 1\), quanh trục \(x\).
\item
  Tìm thể tích của vật rắn được hình thành bằng cách quay vùng giữa
  \(y=x\) và \(y=\sqrt{x}\) trên \([0,1]\) quanh trục \(y\).
\item
  Sử dụng phương pháp vòng đệm để tính thể tích của vật rắn được hình
  thành bằng cách quay \(y=\sqrt{1-x^2}\) (hình bán nguyệt) quanh trục
  \(x\).
\item
  Tìm diện tích nằm giữa \(y=x^2+1\) và \(y=3x\).
\end{enumerate}

\subsection{6.2 Chiều dài cung và diện tích bề
mặt}\label{chiux1ec1u-duxe0i-cung-vuxe0-diux1ec7n-tuxedch-bux1ec1-mux1eb7t}

Tích phân cũng có thể được sử dụng để đo chiều dài của đường cong và
diện tích bề mặt của chất rắn được tạo ra bởi các đường cong quay.

\subsubsection{Độ dài cung}\label{ux111ux1ed9-duxe0i-cung}

Đối với một đường cong trơn \(y=f(x)\) trên đoạn \([a,b]\), độ dài của
đường cong là

\[
L = \int_a^b \sqrt{1 + \big(f'(x)\big)^2}\,dx.
\]

Điều này xuất phát từ việc xấp xỉ đường cong với các đoạn đường và lấy
giới hạn.

Ví dụ: Tìm độ dài của \(y=\tfrac{1}{2}x^{3/2}\) từ \(x=0\) đến \(x=4\).

\begin{itemize}
\tightlist
\item
  Đạo hàm: \(f'(x) = \tfrac{3}{4}\sqrt{x}\).
\item
  Công thức:
\end{itemize}

\[
L = \int_0^4 \sqrt{1 + \Big(\tfrac{3}{4}\sqrt{x}\Big)^2}\,dx
= \int_0^4 \sqrt{1 + \tfrac{9}{16}x}\,dx.
\]

Tích phân này có thể được đánh giá bằng cách thay thế.

\subsubsection{Diện tích bề mặt cách
mạng}\label{diux1ec7n-tuxedch-bux1ec1-mux1eb7t-cuxe1ch-mux1ea1ng}

Nếu một đường cong \(y=f(x)\), \(a \leq x \leq b\), được xoay quanh trục
\(x\), thì diện tích bề mặt của vật rắn thu được là

\[
S = 2\pi \int_a^b f(x)\sqrt{1 + \big(f'(x)\big)^2}\,dx.
\]

Nếu xoay quanh trục \(y\):

\[
S = 2\pi \int_a^b x \sqrt{1 + \big(f'(x)\big)^2}\,dx.
\]

\subsubsection{Ví dụ}\label{vuxed-dux1ee5-16}

\begin{enumerate}
\def\labelenumi{\arabic{enumi}.}
\tightlist
\item
  Độ dài cung của một đường thẳng Với \(y=x\), \(0 \leq x \leq 3\):
\end{enumerate}

\[
L = \int_0^3 \sqrt{1+(1)^2}\,dx = \int_0^3 \sqrt{2}\,dx = 3\sqrt{2}.
\]

\begin{enumerate}
\def\labelenumi{\arabic{enumi}.}
\setcounter{enumi}{1}
\tightlist
\item
  Diện tích bề mặt của hình cầu Lấy \(y = \sqrt{r^2 - x^2}\),
  \(-r \leq x \leq r\) và xoay quanh trục \(x\).
\end{enumerate}

\[
S = 2\pi \int_{-r}^r \sqrt{r^2 - x^2}\sqrt{1+\left(\frac{-x}{\sqrt{r^2-x^2}}\right)^2}\,dx.
\]

Đơn giản hóa ta có \(S = 4\pi r^2\), công thức quen thuộc tính diện tích
bề mặt của hình cầu.

\subsubsection{Tại sao điều này lại quan
trọng}\label{tux1ea1i-sao-ux111iux1ec1u-nuxe0y-lux1ea1i-quan-trux1ecdng-6}

\begin{itemize}
\tightlist
\item
  Độ dài cung mở rộng ý tưởng về khoảng cách tới những đường cong.
\item
  Diện tích bề mặt cách mạng có ứng dụng trong vật lý, kỹ thuật và thiết
  kế.
\item
  Cung cấp một cầu nối giữa giải tích và hình học.
\end{itemize}

\subsubsection{Bài tập}\label{buxe0i-tux1eadp-25}

\begin{enumerate}
\def\labelenumi{\arabic{enumi}.}
\tightlist
\item
  Tìm độ dài cung của \(y=\sqrt{x}\) từ \(x=0\) đến \(x=4\).
\item
  Tính diện tích bề mặt của vật rắn thu được bằng cách quay \(y=x^2\),
  \(0 \leq x \leq 1\), quanh trục \(x\).
\item
  Tìm độ dài cung của \(y=\ln(\cosh x)\) từ \(x=0\) đến \(x=1\).
\item
  Chứng minh rằng việc quay \(y=\sqrt{r^2 - x^2}\) từ \(0\) đến \(r\)
  quanh trục \(x\) sẽ có một nửa diện tích bề mặt của hình cầu.
\item
  Suy ra công thức tính diện tích toàn phần của hình nón bằng cách quay
  một đường thẳng.
\end{enumerate}

\subsection{6.3 Công việc và Trung
bình}\label{cuxf4ng-viux1ec7c-vuxe0-trung-buxecnh}

Tích hợp không giới hạn ở hình học. Nó cũng giúp tính toán công do một
lực thực hiện và giá trị trung bình của hàm số trong một khoảng.

\subsubsection{Công việc}\label{cuxf4ng-viux1ec7c}

Nếu một lực thay đổi \(F(x)\) di chuyển một vật dọc theo một đường thẳng
từ \(x=a\) đến \(x=b\), thì tổng công là

\[
W = \int_a^b F(x)\,dx.
\]

Công thức này khái quát hóa trường hợp đơn giản \(W = F \cdot d\) cho
lực không đổi.

Ví dụ 1: Lực lò xo (Định luật Hooke) Đối với một lò xo kéo dài từ độ dài
\(a\) đến \(b\), với lực \(F(x) = kx\):

\[
W = \int_a^b kx\,dx = \tfrac{1}{2}k(b^2-a^2).
\]

Ví dụ 2: Bơm nước Nếu bơm nước ra khỏi bể thì công cần thiết bằng

\[
W = \int_a^b \text{(weight density)} \times \text{(cross-sectional area)} \times \text{(distance lifted)} \, dx.
\]

\subsubsection{Giá trị trung bình của
hàm}\label{giuxe1-trux1ecb-trung-buxecnh-cux1ee7a-huxe0m}

Giá trị trung bình của hàm liên tục \(f(x)\) trên \([a,b]\) là

\[
f_{\text{avg}} = \frac{1}{b-a} \int_a^b f(x)\,dx.
\]

Đây là sự tương tự liên tục của việc tính trung bình một danh sách các
số.

Ví dụ 1: Với \(f(x)=x^2\) trên \([0,2]\):

\[
f_{\text{avg}} = \tfrac{1}{2-0}\int_0^2 x^2 dx = \tfrac{1}{2}\cdot \tfrac{8}{3} = \tfrac{4}{3}.
\]

Ví dụ 2: Nếu vận tốc của một hạt là \(v(t)\) thì vận tốc trung bình trên
\([a,b]\) là

\[
v_{\text{avg}} = \frac{1}{b-a}\int_a^b v(t)\,dt.
\]

\subsubsection{Tại sao điều này lại quan
trọng}\label{tux1ea1i-sao-ux111iux1ec1u-nuxe0y-lux1ea1i-quan-trux1ecdng-7}

\begin{itemize}
\tightlist
\item
  Tích phân công xuất hiện trong các tính toán vật lý, kỹ thuật và năng
  lượng.
\item
  Giá trị trung bình cho một số đại diện duy nhất cho các đại lượng khác
  nhau.
\item
  Cả hai đều kết nối phép tính với các bài toán thực tế về chuyển động,
  lực và hiệu suất.
\end{itemize}

\subsubsection{Bài tập}\label{buxe0i-tux1eadp-26}

\begin{enumerate}
\def\labelenumi{\arabic{enumi}.}
\tightlist
\item
  Tính công cần thiết để kéo dãn một lò xo từ 2 m lên 5 m nếu \(k=10\).
\item
  Một vật có khối lượng 100 kg được nâng thẳng đứng lên cao 5 m trong
  trường hấp dẫn (\(g=9,8 \,\text{m/s}^2\)). Thể hiện công việc như một
  phần không thể thiếu và đánh giá.
\item
  Tìm giá trị trung bình của \(f(x)=\sin x\) trên \([0,\pi]\).
\item
  Tính nhiệt độ trung bình nếu \(T(t)=20+5\cos(\tfrac{\pi t}{12})\)
  trong 24 giờ một ngày.
\item
  Một bể có độ sâu 10 m chứa đầy nước. Tính công cần thiết để bơm toàn
  bộ nước lên trên, biết nước nặng \(9800 \,\text{N/m}^3\).
\end{enumerate}

\subsection{6.4 Mật độ xác suất và phân bố liên
tục}\label{mux1eadt-ux111ux1ed9-xuxe1c-suux1ea5t-vuxe0-phuxe2n-bux1ed1-liuxean-tux1ee5c}

Tích phân cũng đóng vai trò trung tâm trong lý thuyết xác suất, đặc biệt
đối với các biến ngẫu nhiên liên tục. Thay vì các kết quả riêng biệt,
chúng tôi mô tả xác suất bằng các hàm gọi là hàm mật độ xác suất (pdf).

\subsubsection{Hàm mật độ xác
suất}\label{huxe0m-mux1eadt-ux111ux1ed9-xuxe1c-suux1ea5t}

Hàm mật độ xác suất \(f(x)\) phải thỏa mãn hai điều kiện:

\begin{enumerate}
\def\labelenumi{\arabic{enumi}.}
\item
  \(f(x) \geq 0\) với mọi \(x\).
\item
  Tổng diện tích dưới đường cong là 1:

  \[
  \int_{-\infty}^\infty f(x)\,dx = 1.
  \]
\end{enumerate}

Nếu \(X\) là một biến ngẫu nhiên liên tục với pdf \(f(x)\), thì xác suất
để \(X\) nằm giữa \(a\) và \(b\) là

\[
P(a \leq X \leq b) = \int_a^b f(x)\,dx.
\]

\subsubsection{Hàm phân phối tích
lũy}\label{huxe0m-phuxe2n-phux1ed1i-tuxedch-lux169y}

Hàm phân phối tích lũy (cdf) được định nghĩa là

\[
F(x) = \int_{-\infty}^x f(t)\,dt.
\]

Nó cho biết xác suất để biến ngẫu nhiên nhỏ hơn hoặc bằng \(x\).

\subsubsection{Giá trị mong đợi (Trung
bình)}\label{giuxe1-trux1ecb-mong-ux111ux1ee3i-trung-buxecnh}

Giá trị kỳ vọng của một biến ngẫu nhiên liên tục là giá trị trung bình
có trọng số:

\[
E[X] = \int_{-\infty}^\infty x f(x)\,dx.
\]

\subsubsection{Ví dụ}\label{vuxed-dux1ee5-17}

\begin{enumerate}
\def\labelenumi{\arabic{enumi}.}
\tightlist
\item
  Phân phối thống nhất Với \(f(x) = \tfrac{1}{b-a}\) trên \([a,b]\):
\end{enumerate}

\begin{itemize}
\item
  Xác suất của khoảng \([c,d]\):

  \[
  P(c \leq X \leq d) = \frac{d-c}{b-a}.
  \]
\item
  Expected value: \(E[X] = \tfrac{a+b}{2}\).
\end{itemize}

\begin{enumerate}
\def\labelenumi{\arabic{enumi}.}
\setcounter{enumi}{1}
\tightlist
\item
  Phân phối theo cấp số nhân Với \(f(x) = \lambda e^{-\lambda x}\),
  \(x \geq 0\):
\end{enumerate}

\begin{itemize}
\tightlist
\item
  \(\int_0^\infty \lambda e^{-\lambda x}\,dx = 1\).
\item
  Giá trị trung bình: \(E[X] = \tfrac{1}{\lambda}\).
\end{itemize}

\begin{enumerate}
\def\labelenumi{\arabic{enumi}.}
\setcounter{enumi}{2}
\tightlist
\item
  Phân phối chuẩn Đường cong hình chuông:
\end{enumerate}

\[
f(x) = \frac{1}{\sqrt{2\pi\sigma^2}} e^{-\frac{(x-\mu)^2}{2\sigma^2}}.
\]

Nó tích hợp thành 1, nhưng đòi hỏi kỹ thuật tiên tiến.

\subsubsection{Tại sao điều này lại quan
trọng}\label{tux1ea1i-sao-ux111iux1ec1u-nuxe0y-lux1ea1i-quan-trux1ecdng-8}

\begin{itemize}
\tightlist
\item
  Mật độ xác suất mô tả sự không chắc chắn trong khoa học, kỹ thuật và
  thống kê.
\item
  Tích phân kết nối các khu vực dưới đường cong với xác suất.
\item
  Phân phối liên tục khái quát hóa ý tưởng tính kết quả để đo lường khả
  năng xảy ra trong các khoảng thời gian.
\end{itemize}

\subsubsection{Bài tập}\label{buxe0i-tux1eadp-27}

\begin{enumerate}
\def\labelenumi{\arabic{enumi}.}
\tightlist
\item
  Chứng minh rằng mật độ đồng nhất \(f(x) = \tfrac{1}{b-a}\) trên
  \([a,b]\) tích phân thành 1.
\item
  Đối với phân phối mũ với \(\lambda = 2\), hãy tính
  \(P(0 \leq X \leq 1)\).
\item
  Tìm giá trị kỳ vọng của \(X\) nếu \(f(x) = 3x^2\) trên \([0,1]\).
\item
  Xác minh rằng phân phối chuẩn với trung bình 0 và phương sai 1 có tổng
  xác suất là 1 (không cần bằng chứng đầy đủ nhưng hãy giải thích lý do
  tại sao nó đúng).
\item
  Tính cdf của phân bố đều trên \([0,1]\).
\end{enumerate}

\#Phần III. Phép tính đa biến

\section{Chương 7. Hàm vectơ và đường
cong}\label{chux1b0ux1a1ng-7.-huxe0m-vectux1a1-vuxe0-ux111ux1b0ux1eddng-cong}

\subsection{7.1 Hàm vectơ và đường cong không
gian}\label{huxe0m-vectux1a1-vuxe0-ux111ux1b0ux1eddng-cong-khuxf4ng-gian}

Trong phép tính nhiều biến, các hàm có thể xuất ra vectơ thay vì số.
Chúng được gọi là các hàm có giá trị vectơ và chúng rất cần thiết để mô
tả các đường cong trong không gian.

\subsubsection{Sự định nghĩa}\label{sux1ef1-ux111ux1ecbnh-nghux129a-6}

Hàm vectơ là hàm có dạng

\[
\mathbf{r}(t) = \langle x(t), y(t), z(t) \rangle,
\]

trong đó \(x(t), y(t), z(t)\) là các hàm có giá trị thực.

\begin{itemize}
\tightlist
\item
  Đầu vào \(t\) thường được gọi là tham số.
\item
  Đầu ra là một vector trong không gian 2D hoặc 3D.
\item
  Đồ thị của hàm vectơ trong không gian 3D là đường cong không gian.
\end{itemize}

\subsubsection{Ví dụ}\label{vuxed-dux1ee5-18}

\begin{enumerate}
\def\labelenumi{\arabic{enumi}.}
\tightlist
\item
  Đường dây
\end{enumerate}

\[
\mathbf{r}(t) = \langle 1+2t, \; 3-t, \; 4+5t \rangle.
\]

Điều này mô tả một đường thẳng đi qua điểm \((1,3,4)\) với vectơ chỉ
phương \(\langle 2,-1,5 \rangle\).

\begin{enumerate}
\def\labelenumi{\arabic{enumi}.}
\setcounter{enumi}{1}
\tightlist
\item
  Vòng tròn trong mặt phẳng
\end{enumerate}

\[
\mathbf{r}(t) = \langle \cos t, \; \sin t, \; 0 \rangle, \quad 0 \leq t < 2\pi.
\]

\begin{enumerate}
\def\labelenumi{\arabic{enumi}.}
\setcounter{enumi}{2}
\tightlist
\item
  Chuỗi xoắn
\end{enumerate}

\[
\mathbf{r}(t) = \langle \cos t, \; \sin t, \; t \rangle.
\]

Đây là một đường xoắn ốc tăng lên xung quanh trục \(z\).

\subsubsection{Giới hạn và tính liên
tục}\label{giux1edbi-hux1ea1n-vuxe0-tuxednh-liuxean-tux1ee5c}

Một hàm vectơ liên tục tại \(t=a\) nếu mỗi thành phần
\(x(t), y(t), z(t)\) liên tục tại \(t=a\).

\[
\lim_{t \to a} \mathbf{r}(t) = \langle \lim_{t \to a} x(t), \; \lim_{t \to a} y(t), \; \lim_{t \to a} z(t) \rangle.
\]

\subsubsection{Hình học của đường cong không
gian}\label{huxecnh-hux1ecdc-cux1ee7a-ux111ux1b0ux1eddng-cong-khuxf4ng-gian}

\begin{itemize}
\tightlist
\item
  Mỗi đường cong có một hướng tiếp tuyến được cho bởi đạo hàm.
\item
  Đường cong không gian có thể mô hình hóa đường chuyển động, quỹ đạo
  hạt và hình dạng hình học.
\end{itemize}

\subsubsection{Tại sao điều này lại quan
trọng}\label{tux1ea1i-sao-ux111iux1ec1u-nuxe0y-lux1ea1i-quan-trux1ecdng-9}

Hàm vectơ là nền tảng cho phép tính nhiều biến, cho phép chúng ta mở
rộng ý tưởng về đạo hàm và tích phân sang các chiều cao hơn. Chúng cũng
xuất hiện một cách tự nhiên trong vật lý (chuyển động trong không gian
3D, điện từ, động lực học chất lỏng).

\subsubsection{Bài tập}\label{buxe0i-tux1eadp-28}

\begin{enumerate}
\def\labelenumi{\arabic{enumi}.}
\tightlist
\item
  Viết hàm vectơ cho đường thẳng đi qua \((0,1,2)\) song song với vectơ
  \(\langle 3,-2,1 \rangle\).
\item
  Hãy mô tả đường cong cho bởi
  \(\mathbf{r}(t) = \langle 2\cos t, \; 2\sin t, \; 3 \rangle\).
\item
  Xác định xem \(\mathbf{r}(t) = \langle e^t, \; \lnt, \; t^2 \rangle\)
  liên tục tại \(t=1\).
\item
  Vẽ đường xoắn
  \(\mathbf{r}(t) = \langle \cos t, \; \sin t, \; 2t \rangle\).
\item
  Tìm điểm trên đường cong
  \(\mathbf{r}(t) = \langle t, \; t^2, \; t^3 \rangle\) khi \(t=2\).
\end{enumerate}

\subsection{7.2 Đạo hàm và tích phân của hàm
vectơ}\label{ux111ux1ea1o-huxe0m-vuxe0-tuxedch-phuxe2n-cux1ee7a-huxe0m-vectux1a1}

Các hàm vectơ có thể được vi phân và tích hợp giống như các hàm thông
thường - chúng ta chỉ cần áp dụng thao tác cho từng thành phần. Điều này
cho phép chúng ta nghiên cứu chuyển động, vận tốc, gia tốc và sự tích
lũy ở các chiều cao hơn.

\subsubsection{Đạo hàm của hàm
vectơ}\label{ux111ux1ea1o-huxe0m-cux1ee7a-huxe0m-vectux1a1}

Nếu như

\[
\mathbf{r}(t) = \langle x(t), y(t), z(t) \rangle,
\]

sau đó

\[
\mathbf{r}'(t) = \langle x'(t), y'(t), z'(t) \rangle.
\]

Vectơ đạo hàm này hướng tiếp tuyến với đường cong tại tham số \(t\).

\begin{itemize}
\tightlist
\item
  Vận tốc: Nếu \(\mathbf{r}(t)\) cho biết vị trí của một hạt tại thời
  điểm \(t\) thì \(\mathbf{v}(t) = \mathbf{r}'(t)\) là vectơ vận tốc của
  hạt.
\item
  Tốc độ: Độ lớn \(|\mathbf{v}(t)|\) là tốc độ của hạt.
\item
  Gia tốc: \(\mathbf{a}(t) = \mathbf{v}'(t) = \mathbf{r}''(t)\).
\end{itemize}

\subsubsection{Ví dụ}\label{vuxed-dux1ee5-19}

\begin{enumerate}
\def\labelenumi{\arabic{enumi}.}
\tightlist
\item
  Đường xoắn ốc
\end{enumerate}

\[
\mathbf{r}(t) = \langle \cos t, \sin t, t \rangle.
\]

\begin{itemize}
\tightlist
\item
  Vận tốc: \(\mathbf{v}(t) = \langle -\sin t, \cos t, 1 \rangle\).
\item
  Tốc độ:
  \(|\mathbf{v}(t)| = \sqrt{(-\sin t)^2 + (\cos t)^2 + 1^2} = \sqrt{2}\).
\item
  Gia tốc: \(\mathbf{a}(t) = \langle -\cos t, -\sin t, 0 \rangle\).
\end{itemize}

\begin{enumerate}
\def\labelenumi{\arabic{enumi}.}
\setcounter{enumi}{1}
\tightlist
\item
  Chuyển động của đạn
\end{enumerate}

\[
\mathbf{r}(t) = \langle v_0 \cos\theta \cdot t, \; v_0 \sin\theta \cdot t - \tfrac{1}{2}gt^2 \rangle.
\]

Điều này mô hình hóa đường parabol của một viên đạn dưới tác dụng của
trọng lực.

\subsubsection{Tích phân của hàm
vectơ}\label{tuxedch-phuxe2n-cux1ee7a-huxe0m-vectux1a1}

Nếu như

\[
\mathbf{r}(t) = \langle x(t), y(t), z(t) \rangle,
\]

sau đó

\[
\int \mathbf{r}(t)\,dt = \left\langle \int x(t)\,dt, \; \int y(t)\,dt, \; \int z(t)\,dt \right\rangle + \mathbf{C},
\]

trong đó \(\mathbf{C}\) là một vectơ không đổi.

\subsubsection{Ví dụ}\label{vuxed-dux1ee5-20}

\[
\mathbf{r}(t) = \langle t, t^2, t^3 \rangle.
\]

\begin{itemize}
\tightlist
\item
  Đạo hàm: \(\mathbf{r}'(t) = \langle 1, 2t, 3t^2 \rangle\).
\item
  Tích phân:
\end{itemize}

\[
\int \mathbf{r}(t)\,dt = \langle \tfrac{1}{2}t^2, \tfrac{1}{3}t^3, \tfrac{1}{4}t^4 \rangle + \mathbf{C}.
\]

\subsubsection{Tại sao điều này lại quan
trọng}\label{tux1ea1i-sao-ux111iux1ec1u-nuxe0y-lux1ea1i-quan-trux1ecdng-10}

\begin{itemize}
\tightlist
\item
  Đạo hàm của hàm vectơ mô tả chuyển động và các lực trong không gian.
\item
  Tích phân cho biết độ dịch chuyển, công và các đại lượng tích lũy.
\item
  Những công cụ này kết nối trực tiếp phép tính với vật lý và kỹ thuật.
\end{itemize}

\subsubsection{Bài tập}\label{buxe0i-tux1eadp-29}

\begin{enumerate}
\def\labelenumi{\arabic{enumi}.}
\tightlist
\item
  Với \(\mathbf{r}(t) = \langle t, \cos t, \sin t \rangle\), tìm vận
  tốc, vận tốc và gia tốc.
\item
  Tính \(\mathbf{r}'(t)\) với
  \(\mathbf{r}(t) = \langle e^t, \ln t, t^2 \rangle\).
\item
  Tích phân \(\mathbf{r}(t) = \langle 1, t, t^2 \rangle\).
\item
  Một hạt có vận tốc \(\mathbf{v}(t) = \langle t, 2, 0 \rangle\). Tìm
  vectơ vị trí của nó nếu \(\mathbf{r}(0) = \langle 1, 0, 0 \rangle\).
\item
  Chứng minh rằng tốc độ của
  \(\mathbf{r}(t) = \langle \cos t, \sin t, 0 \rangle\) là không đổi.
\end{enumerate}

\subsection{7.3 Chiều dài và độ cong của
cung}\label{chiux1ec1u-duxe0i-vuxe0-ux111ux1ed9-cong-cux1ee7a-cung}

Phép tính vectơ cung cấp các công cụ để đo không chỉ đường đi được vạch
ra bởi một đường cong mà còn đo độ cong của nó. Chúng được thể hiện
thông qua chiều dài cung và độ cong.

\subsubsection{Độ dài cung của đường cong không
gian}\label{ux111ux1ed9-duxe0i-cung-cux1ee7a-ux111ux1b0ux1eddng-cong-khuxf4ng-gian}

Nếu một đường cong được cho bởi

\[
\mathbf{r}(t) = \langle x(t), y(t), z(t) \rangle, \quad a \leq t \leq b,
\]

thì độ dài cung là

\[
L = \int_a^b |\mathbf{r}'(t)|\,dt,
\]

Ở đâu

\[
|\mathbf{r}'(t)| = \sqrt{(x'(t))^2 + (y'(t))^2 + (z'(t))^2}.
\]

Ví dụ: Đối với chuỗi xoắn
\(\mathbf{r}(t) = \langle \cos t, \sin t, t \rangle, \, 0 \leq t \leq 2\pi\):

\begin{itemize}
\tightlist
\item
  Vận tốc: \(\mathbf{r}'(t) = \langle -\sin t, \cos t, 1 \rangle\).
\item
  Tốc độ:
  \(|\mathbf{r}'(t)| = \sqrt{(-\sin t)^2 + (\cos t)^2 + 1^2} = \sqrt{2}\).
\item
  Độ dài cung:
\end{itemize}

\[
L = \int_0^{2\pi} \sqrt{2}\,dt = 2\pi\sqrt{2}.
\]

\subsubsection{Độ cong}\label{ux111ux1ed9-cong}

Độ cong đo tốc độ đường cong thay đổi hướng.

Đối với một đường cong trơn \(\mathbf{r}(t)\):

\[
\kappa(t) = \frac{|\mathbf{r}'(t) \times \mathbf{r}''(t)|}{|\mathbf{r}'(t)|^3}.
\]

\begin{itemize}
\tightlist
\item
  \(\kappa = 0\): đường thẳng.
\item
  \(\kappa\) càng lớn: đường cong càng cong.
\end{itemize}

Ví dụ: Đối với đường tròn bán kính \(r\):

\[
\mathbf{r}(t) = \langle r\cos t, r\sin t \rangle.
\]

Khi đó \(\kappa = \tfrac{1}{r}\). Vì vậy độ cong không đổi và tỷ lệ
nghịch với bán kính.

\subsubsection{Đơn vị tiếp tuyến và vectơ pháp
tuyến}\label{ux111ux1a1n-vux1ecb-tiux1ebfp-tuyux1ebfn-vuxe0-vectux1a1-phuxe1p-tuyux1ebfn}

\begin{itemize}
\tightlist
\item
  Vectơ tiếp tuyến:
\end{itemize}

\[
\mathbf{T}(t) = \frac{\mathbf{r}'(t)}{|\mathbf{r}'(t)|}.
\]

\begin{itemize}
\tightlist
\item
  Vector pháp tuyến: hướng về tâm cong, được định nghĩa là
\end{itemize}

\[
\mathbf{N}(t) = \frac{\mathbf{T}'(t)}{|\mathbf{T}'(t)|}.
\]

Các vectơ này mô tả hình học của chuyển động: hướng di chuyển và hướng
quay.

\subsubsection{Tại sao điều này lại quan
trọng}\label{tux1ea1i-sao-ux111iux1ec1u-nuxe0y-lux1ea1i-quan-trux1ecdng-11}

\begin{itemize}
\tightlist
\item
  Độ dài cung khái quát khái niệm khoảng cách tới các đường cong trong
  không gian.
\item
  Độ cong mô tả sự uốn cong, rất quan trọng trong vật lý (gia tốc hướng
  tâm), kỹ thuật (đường, tàu lượn siêu tốc) và đồ họa máy tính.
\end{itemize}

\subsubsection{Bài tập}\label{buxe0i-tux1eadp-30}

\begin{enumerate}
\def\labelenumi{\arabic{enumi}.}
\tightlist
\item
  Tìm độ dài cung của \(\mathbf{r}(t) = \langle t, t^2, 0 \rangle\) từ
  \(t=0\) đến \(t=1\).
\item
  Tính độ cong của đường tròn
  \(\mathbf{r}(t) = \langle \cos t, \sin t \rangle\).
\item
  Với \(\mathbf{r}(t) = \langle t, \cos t, \sin t \rangle\), hãy tính
  \(|\mathbf{r}'(t)|\).
\item
  Chứng minh đường thẳng có độ cong \(\kappa = 0\).
\item
  Tìm vectơ tiếp tuyến của
  \(\mathbf{r}(t) = \langle e^t, e^{-t}, t \rangle\) tại \(t=0\).
\end{enumerate}

\subsection{7.4 Chuyển động trong không
gian}\label{chuyux1ec3n-ux111ux1ed9ng-trong-khuxf4ng-gian}

Hàm vectơ đặc biệt mạnh mẽ trong việc mô tả chuyển động theo hai hoặc ba
chiều. Vị trí, vận tốc và gia tốc được biểu diễn một cách tự nhiên bằng
cách sử dụng đạo hàm và tích phân của các hàm có giá trị vectơ.

\subsubsection{Vị trí, Vận tốc và Gia
tốc}\label{vux1ecb-truxed-vux1eadn-tux1ed1c-vuxe0-gia-tux1ed1c}

\begin{itemize}
\tightlist
\item
  Vectơ vị trí:
\end{itemize}

\[
\mathbf{r}(t) = \langle x(t), y(t), z(t) \rangle
\]

\begin{itemize}
\tightlist
\item
  Vector vận tốc (đạo hàm vị trí):
\end{itemize}

\[
\mathbf{v}(t) = \mathbf{r}'(t) = \langle x'(t), y'(t), z'(t) \rangle
\]

\begin{itemize}
\tightlist
\item
  Tốc độ (độ lớn của vận tốc):
\end{itemize}

\[
|\mathbf{v}(t)| = \sqrt{(x'(t))^2 + (y'(t))^2 + (z'(t))^2}
\]

\begin{itemize}
\tightlist
\item
  Vectơ gia tốc (đạo hàm của vận tốc):
\end{itemize}

\[
\mathbf{a}(t) = \mathbf{v}'(t) = \mathbf{r}''(t).
\]

\subsubsection{Thành phần tiếp tuyến và pháp
tuyến}\label{thuxe0nh-phux1ea7n-tiux1ebfp-tuyux1ebfn-vuxe0-phuxe1p-tuyux1ebfn}

Gia tốc có thể được phân tách thành hai thành phần:

\[
\mathbf{a}(t) = a_T \mathbf{T}(t) + a_N \mathbf{N}(t),
\]

Ở đâu:

\begin{itemize}
\tightlist
\item
  \(\mathbf{T}(t)\) = vectơ tiếp tuyến đơn vị,
\item
  \(\mathbf{N}(t)\) = vectơ pháp tuyến chính,
\item
  \(a_T = \frac{d}{dt}|\mathbf{v}(t)|\) = gia tốc tiếp tuyến (thay đổi
  tốc độ),
\item
  \(a_N = \kappa |\mathbf{v}(t)|^2\) = gia tốc bình thường (đổi hướng).
\end{itemize}

\subsubsection{Chuyển động của đạn ở dạng
3D}\label{chuyux1ec3n-ux111ux1ed9ng-cux1ee7a-ux111ux1ea1n-ux1edf-dux1ea1ng-3d}

Với trọng lực tác dụng theo hướng \(-z\):

\[
\mathbf{r}(t) = \langle v_0 \cos\theta \cos\phi \cdot t,\; v_0 \cos\theta \sin\phi \cdot t,\; v_0 \sin\theta \cdot t - \tfrac{1}{2}gt^2 \rangle,
\]

trong đó \(v_0\) là tốc độ ban đầu, \(\theta\) góc phóng và \(\phi\)
hướng phương vị.

\subsubsection{Ví dụ: Chuyển động xoắn
ốc}\label{vuxed-dux1ee5-chuyux1ec3n-ux111ux1ed9ng-xoux1eafn-ux1ed1c}

\[
\mathbf{r}(t) = \langle \cos t, \sin t, t \rangle
\]

\begin{itemize}
\tightlist
\item
  Vận tốc: \(\mathbf{v}(t) = \langle -\sin t, \cos t, 1 \rangle\).
\item
  Tốc độ: \(|\mathbf{v}(t)| = \sqrt{2}\).
\item
  Gia tốc: \(\mathbf{a}(t) = \langle -\cos t, -\sin t, 0 \rangle\).
\item
  Chuyển động có tốc độ đều, chuyển động xoắn ốc hướng lên trên.
\end{itemize}

\subsubsection{Tại sao điều này lại quan
trọng}\label{tux1ea1i-sao-ux111iux1ec1u-nuxe0y-lux1ea1i-quan-trux1ecdng-12}

\begin{itemize}
\tightlist
\item
  Cung cấp ngôn ngữ toán học cho chuyển động trong thế giới thực.
\item
  Cần thiết trong vật lý (lực, quỹ đạo, chuyển động tròn).
\item
  Nền tảng cho các mô hình cơ khí và kỹ thuật tiên tiến.
\end{itemize}

\subsubsection{Bài tập}\label{buxe0i-tux1eadp-31}

\begin{enumerate}
\def\labelenumi{\arabic{enumi}.}
\tightlist
\item
  Một hạt chuyển động dọc theo
  \(\mathbf{r}(t) = \langle t, t^2, t^3 \rangle\). Tìm vận tốc và gia
  tốc tại \(t=1\).
\item
  Chứng minh rằng tốc độ không đổi đối với đường xoắn ốc
  \(\mathbf{r}(t) = \langle \cos t, \sin t, t \rangle\).
\item
  Một viên đạn được phóng với góc \(v_0 = 20 \,\text{m/s}\) với góc
  \(45^\circ\). Viết vectơ vị trí của nó giả sử chuyển động trong mặt
  phẳng thẳng đứng.
\item
  Với \(\mathbf{r}(t) = \langle e^t, e^{-t}, t \rangle\), tìm
  \(\mathbf{v}(t)\) và \(\mathbf{a}(t)\).
\item
  Phân tích vectơ gia tốc thành các thành phần tiếp tuyến và pháp tuyến
  đối với chuyển động dọc theo đường tròn bán kính \(r\).
\end{enumerate}

\section{Chương 8. Hàm nhiều
biến}\label{chux1b0ux1a1ng-8.-huxe0m-nhiux1ec1u-biux1ebfn}

\subsection{8.1 Giới hạn và tính liên tục của một số
biến}\label{giux1edbi-hux1ea1n-vuxe0-tuxednh-liuxean-tux1ee5c-cux1ee7a-mux1ed9t-sux1ed1-biux1ebfn}

Trong phép tính nhiều biến, các hàm có thể phụ thuộc vào hai hoặc nhiều
biến, chẳng hạn như \(f(x,y)\) hoặc \(f(x,y,z)\). Các khái niệm về giới
hạn và tính liên tục mở rộng một cách tự nhiên từ phép tính một biến,
nhưng chúng tinh tế hơn vì chúng ta phải xem xét tất cả các cách tiếp
cận có thể có.

\subsubsection{Giới hạn của hai
biến}\label{giux1edbi-hux1ea1n-cux1ee7a-hai-biux1ebfn}

Đối với hàm \(f(x,y)\), ta nói

\[
\lim_{(x,y) \to (a,b)} f(x,y) = L
\]

nếu \(f(x,y)\) tùy ý tiến gần đến \(L\) khi \((x,y)\) tiến đến \((a,b)\)
dọc theo bất kỳ đường dẫn nào.

Nếu các đường dẫn khác nhau cho các giá trị giới hạn khác nhau thì giới
hạn đó không tồn tại.

Ví dụ 1 (tồn tại giới hạn):

\[
f(x,y) = x^2 + y^2, \quad \lim_{(x,y) \to (0,0)} f(x,y) = 0.
\]

Ví dụ 2 (không tồn tại giới hạn):

\[
f(x,y) = \frac{xy}{x^2+y^2}, \quad (x,y) \to (0,0).
\]

\begin{itemize}
\tightlist
\item
  Cùng \(y=0\) thì hàm số bằng 0.
\item
  Cùng với \(y=x\), hàm số là \(\tfrac{1}{2}\). Kết quả khác nhau → giới
  hạn không tồn tại.
\end{itemize}

\subsubsection{Tính liên tục}\label{tuxednh-liuxean-tux1ee5c-1}

Hàm \(f(x,y)\) liên tục tại \((a,b)\) nếu

\[
\lim_{(x,y)\to(a,b)} f(x,y) = f(a,b).
\]

Đa thức và hàm hữu tỷ (trong đó mẫu số ≠ 0) liên tục ở mọi nơi trong
miền của chúng.

\subsubsection{Mở rộng thành ba biến trở
lên}\label{mux1edf-rux1ed9ng-thuxe0nh-ba-biux1ebfn-trux1edf-luxean}

Đối với \(f(x,y,z)\), giới hạn và tính liên tục được xác định theo cùng
một cách, nhưng điểm \((a,b,c)\) phải được tiếp cận từ vô số hướng trong
không gian.

\subsubsection{Tại sao điều này lại quan
trọng}\label{tux1ea1i-sao-ux111iux1ec1u-nuxe0y-lux1ea1i-quan-trux1ecdng-13}

\begin{itemize}
\tightlist
\item
  Tính liên tục đảm bảo không có bước nhảy, lỗ trống hoặc tiệm cận trong
  hàm số nhiều biến.
\item
  Giới hạn là cơ sở để xác định đạo hàm riêng và tích phân bội.
\item
  Những khái niệm này là nền tảng cho phép tính nhiều biến.
\end{itemize}

\subsubsection{Bài tập}\label{buxe0i-tux1eadp-32}

\begin{enumerate}
\def\labelenumi{\arabic{enumi}.}
\tightlist
\item
  Xác định xem \(\lim_{(x,y)\to(0,0)} (x^2+y^2)\) có tồn tại hay không.
\item
  Chứng minh rằng \(\lim_{(x,y)\to(0,0)} \frac{x^2y}{x^2+y^2} = 0\) dọc
  theo tất cả các đường thẳng \(y=mx\).
\item
  Có tồn tại giới hạn cho \(f(x,y) = \frac{x^2-y^2}{x^2+y^2}\) là
  \((x,y)\to(0,0)\) không?
\item
  Giải thích tại sao đa thức hai biến liên tục ở mọi nơi.
\item
  Cho ví dụ về hàm hai biến không liên tục tại một điểm và giải thích
  tại sao.
\end{enumerate}

\subsection{8.2 Đạo hàm riêng
phần}\label{ux111ux1ea1o-huxe0m-riuxeang-phux1ea7n}

Trong hàm nhiều biến, chúng ta thường muốn đo xem hàm này thay đổi như
thế nào khi chỉ có một biến thay đổi trong khi các biến khác không đổi.
Điều này dẫn đến ý tưởng về đạo hàm riêng.

\subsubsection{Sự định nghĩa}\label{sux1ef1-ux111ux1ecbnh-nghux129a-7}

Đối với hàm \(f(x,y)\), đạo hàm riêng đối với \(x\) tại một điểm
\((a,b)\) là

\[
\frac{\partial f}{\partial x}(a,b) = \lim_{h \to 0} \frac{f(a+h, b) - f(a,b)}{h}.
\]

Tương tự, đạo hàm riêng đối với \(y\) là

\[
\frac{\partial f}{\partial y}(a,b) = \lim_{h \to 0} \frac{f(a, b+h) - f(a,b)}{h}.
\]

Chúng ta coi tất cả các biến khác là hằng số khi lấy đạo hàm.

\subsubsection{Ký hiệu}\label{kuxfd-hiux1ec7u-1}

\begin{itemize}
\tightlist
\item
  \(\frac{\partial f}{\partial x}\), \(f_x\), \(\partial_x f\).
\item
  \(\frac{\partial f}{\partial y}\), \(f_y\), \(\partial_y f\).
\end{itemize}

Với ba biến \(f(x,y,z)\), chúng ta cũng có \(f_x, f_y, f_z\).

\subsubsection{Ví dụ}\label{vuxed-dux1ee5-21}

\begin{enumerate}
\def\labelenumi{\arabic{enumi}.}
\tightlist
\item
  \(f(x,y) = x^2y + y^3\)
\end{enumerate}

\begin{itemize}
\tightlist
\item
  \(f_x = 2xy\).
\item
  \(f_y = x^2 + 3y^2\).
\end{itemize}

\begin{enumerate}
\def\labelenumi{\arabic{enumi}.}
\setcounter{enumi}{1}
\tightlist
\item
  \(f(x,y) = e^{xy}\)
\end{enumerate}

\begin{itemize}
\tightlist
\item
  \(f_x = y e^{xy}\).
\item
  \(f_y = x e^{xy}\).
\end{itemize}

\begin{enumerate}
\def\labelenumi{\arabic{enumi}.}
\setcounter{enumi}{2}
\tightlist
\item
  \(f(x,y,z) = x^2 + yz\)
\end{enumerate}

\begin{itemize}
\tightlist
\item
  \(f_x = 2x\).
\item
  \(f_y = z\).
\item
  \(f_z = y\).
\end{itemize}

\subsubsection{Đạo hàm từng phần bậc cao
hơn}\label{ux111ux1ea1o-huxe0m-tux1eebng-phux1ea7n-bux1eadc-cao-hux1a1n}

Chúng ta có thể lấy đạo hàm riêng nhiều lần:

\begin{itemize}
\tightlist
\item
  \(f_{xx} = \frac{\partial}{\partial x}\Big(f_x\Big)\).
\item
  \(f_{yy}, f_{xy}, f_{yx}\), v.v.
\end{itemize}

Định lý Clairaut: Nếu \(f\) có đạo hàm riêng cấp hai liên tục thì

\[
f_{xy} = f_{yx}.
\]

\subsubsection{Ý nghĩa hình học}\label{uxfd-nghux129a-huxecnh-hux1ecdc}

\begin{itemize}
\tightlist
\item
  \(f_x\): độ dốc của bề mặt theo phương \(x\).
\item
  \(f_y\): độ dốc của bề mặt theo phương \(y\).
\item
  Họ cùng nhau mô tả bề mặt nghiêng như thế nào.
\end{itemize}

\subsubsection{Tại sao điều này lại quan
trọng}\label{tux1ea1i-sao-ux111iux1ec1u-nuxe0y-lux1ea1i-quan-trux1ecdng-14}

\begin{itemize}
\tightlist
\item
  Đạo hàm riêng là nền tảng của gradient, mặt phẳng tiếp tuyến và tối ưu
  hóa đa biến.
\item
  Chúng được sử dụng rộng rãi trong vật lý, kỹ thuật và kinh tế để mô
  hình hóa các hệ thống có nhiều đầu vào.
\end{itemize}

\subsubsection{Bài tập}\label{buxe0i-tux1eadp-33}

\begin{enumerate}
\def\labelenumi{\arabic{enumi}.}
\tightlist
\item
  Tìm \(f_x\) và \(f_y\) với \(f(x,y) = x^3y^2\).
\item
  Tính \(f_x, f_y, f_z\) cho \(f(x,y,z) = xyz + x^2\).
\item
  Chứng minh định lý Clairaut cho \(f(x,y) = x^2y + y^3\).
\item
  Giải thích về mặt hình học ý nghĩa của \(f_x\) và \(f_y\) đối với
  \(f(x,y) = \sqrt{x^2+y^2}\).
\item
  Tìm tất cả các đạo hàm riêng bậc hai của \(f(x,y) = e^{x^2+y^2}\).
\end{enumerate}

\subsection{8.3 Đạo hàm theo độ dốc và
hướng}\label{ux111ux1ea1o-huxe0m-theo-ux111ux1ed9-dux1ed1c-vuxe0-hux1b0ux1edbng}

Đạo hàm riêng đo sự thay đổi dọc theo trục tọa độ, nhưng đôi khi chúng
ta muốn biết tốc độ thay đổi của hàm số theo bất kỳ hướng nào. Điều này
dẫn đến các khái niệm về độ dốc và đạo hàm có hướng.

\subsubsection{Vectơ chuyển màu}\label{vectux1a1-chuyux1ec3n-muxe0u}

Đối với hàm \(f(x,y)\), gradient là vectơ

\[
\nabla f(x,y) = \left\langle \frac{\partial f}{\partial x}, \frac{\partial f}{\partial y} \right\rangle.
\]

Đối với ba biến \(f(x,y,z)\):

\[
\nabla f(x,y,z) = \left\langle f_x, f_y, f_z \right\rangle.
\]

Các điểm gradient theo hướng tăng tối đa của hàm và độ lớn của nó cho độ
dốc lớn nhất.

\subsubsection{Đạo hàm định
hướng}\label{ux111ux1ea1o-huxe0m-ux111ux1ecbnh-hux1b0ux1edbng}

Tốc độ thay đổi của \(f(x,y)\) tại một điểm theo hướng của vectơ đơn vị
\(\mathbf{u} = \langle u_1, u_2 \rangle\) là

\[
D_{\mathbf{u}} f(x,y) = \nabla f(x,y) \cdot \mathbf{u}.
\]

Đây là tích số chấm của gradient với vectơ chỉ phương.

\subsubsection{Ví dụ}\label{vuxed-dux1ee5-22}

\begin{enumerate}
\def\labelenumi{\arabic{enumi}.}
\tightlist
\item
  \(f(x,y) = x^2 + y^2\)
\end{enumerate}

\begin{itemize}
\tightlist
\item
  Độ dốc: \(\nabla f = \langle 2x, 2y \rangle\).
\item
  Tại (1,2): \(\nabla f = \langle 2,4 \rangle\).
\item
  Đạo hàm có hướng dọc theo
  \(\mathbf{u} = \langle \tfrac{3}{5}, \tfrac{4}{5} \rangle\):
\end{itemize}

\[
D_{\mathbf{u}} f(1,2) = \langle 2,4 \rangle \cdot \langle \tfrac{3}{5}, \tfrac{4}{5} \rangle = \tfrac{26}{5}.
\]

\begin{enumerate}
\def\labelenumi{\arabic{enumi}.}
\setcounter{enumi}{1}
\tightlist
\item
  \(f(x,y,z) = x y z\)
\end{enumerate}

\begin{itemize}
\tightlist
\item
  Độ dốc: \(\nabla f = \langle yz, xz, xy \rangle\).
\item
  Tại (1,1,1): \(\nabla f = \langle 1,1,1 \rangle\).
\item
  Hướng tăng tối đa dọc theo \(\langle 1,1,1 \rangle\).
\end{itemize}

\subsubsection{Giải thích hình
học}\label{giux1ea3i-thuxedch-huxecnh-hux1ecdc-1}

\begin{itemize}
\tightlist
\item
  Vector gradient vuông góc (chuẩn tắc) với các đường cong mức hoặc các
  mặt phẳng của \(f\).
\item
  Đạo hàm có hướng khái quát hóa độ dốc theo các hướng tùy ý.
\end{itemize}

\subsubsection{Tại sao điều này lại quan
trọng}\label{tux1ea1i-sao-ux111iux1ec1u-nuxe0y-lux1ea1i-quan-trux1ecdng-15}

\begin{itemize}
\tightlist
\item
  Trong tối ưu hóa, độ dốc cho chúng ta biết hướng di chuyển để đi lên
  hoặc đi xuống dốc nhất.
\item
  Trong vật lý, gradient mô tả các trường như dòng nhiệt và điện thế.
\item
  Đạo hàm định hướng thống nhất tỷ lệ thay đổi một biến và đa biến.
\end{itemize}

\subsubsection{Bài tập}\label{buxe0i-tux1eadp-34}

\begin{enumerate}
\def\labelenumi{\arabic{enumi}.}
\tightlist
\item
  Tính \(\nabla f(x,y)\) cho \(f(x,y) = e^{xy}\).
\item
  Tìm gradient của \(f(x,y,z) = x^2+y^2+z^2\) và ước tính tại (1,1,1).
\item
  Tính đạo hàm có hướng của \(f(x,y) = x^2-y\) tại (2,1) theo hướng
  \(\mathbf{u} = \langle 0,1 \rangle\).
\item
  Chứng minh rằng gradient của \(f(x,y) = x^2+y^2\) vuông góc với đường
  tròn \(x^2+y^2=1\).
\item
  Tìm hướng vectơ đơn vị làm cực đại hóa đạo hàm có hướng của
  \(f(x,y) = xy\) tại (1,2).
\end{enumerate}

\subsection{8.4 Mặt phẳng tiếp tuyến và xấp xỉ tuyến
tính}\label{mux1eb7t-phux1eb3ng-tiux1ebfp-tuyux1ebfn-vuxe0-xux1ea5p-xux1ec9-tuyux1ebfn-tuxednh}

Trong phép tính một biến, đường tiếp tuyến gần giống một đường cong gần
một điểm. Trong phép tính nhiều biến, khái niệm tương tự là mặt phẳng
tiếp tuyến, cung cấp một phép tính gần đúng tuyến tính cho một bề mặt
gần một điểm.

\subsubsection{Mặt phẳng tiếp tuyến với một bề
mặt}\label{mux1eb7t-phux1eb3ng-tiux1ebfp-tuyux1ebfn-vux1edbi-mux1ed9t-bux1ec1-mux1eb7t}

Giả sử \(z = f(x,y)\) khả vi tại \((a,b)\). Mặt phẳng tiếp tuyến tại
\((a,b,f(a,b))\) được cho bởi

\[
z = f(a,b) + f_x(a,b)(x-a) + f_y(a,b)(y-b).
\]

Mặt phẳng này chạm vào bề mặt tại điểm và xấp xỉ nó ở gần đó.

\subsubsection{Ví dụ 1: Paraboloid}\label{vuxed-dux1ee5-1-paraboloid}

Với \(f(x,y) = x^2 + y^2\) tại \((1,2)\):

\begin{itemize}
\tightlist
\item
  \(f(1,2) = 1^2+2^2=5\).
\item
  \(f_x = 2x\) nên \(f_x(1,2) = 2\).
\item
  \(f_y = 2y\), nên \(f_y(1,2) = 4\).
\end{itemize}

Phương trình mặt phẳng tiếp tuyến:

\[
z = 5 + 2(x-1) + 4(y-2).
\]

\subsubsection{Xấp xỉ tuyến
tính}\label{xux1ea5p-xux1ec9-tuyux1ebfn-tuxednh}

Mặt phẳng tiếp tuyến có thể được sử dụng để tính gần đúng \(f(x,y)\) gần
\((a,b)\):

\[
f(x,y) \approx f(a,b) + f_x(a,b)(x-a) + f_y(a,b)(y-b).
\]

Đây là sự tuyến tính hóa của \(f\) tại \((a,b)\).

\subsubsection{Ví dụ 2: Xấp xỉ tuyến
tính}\label{vuxed-dux1ee5-2-xux1ea5p-xux1ec9-tuyux1ebfn-tuxednh}

Gần đúng \(f(x,y) = \sqrt{x+y}\) gần \((4,5)\).

\begin{itemize}
\tightlist
\item
  \(f(4,5) = \sqrt{9} = 3\).
\item
  \(f_x = \frac{1}{2\sqrt{x+y}}, \quad f_y = \frac{1}{2\sqrt{x+y}}\).
\item
  Tại (4,5): \(f_x = f_y = \tfrac{1}{6}\).
\end{itemize}

Vì thế,

\[
f(x,y) \approx 3 + \tfrac{1}{6}(x-4) + \tfrac{1}{6}(y-5).
\]

\subsubsection{Tại sao điều này lại quan
trọng}\label{tux1ea1i-sao-ux111iux1ec1u-nuxe0y-lux1ea1i-quan-trux1ecdng-16}

\begin{itemize}
\tightlist
\item
  Các mặt phẳng tiếp tuyến cho phép tính gần đúng tuyến tính tốt nhất
  của một bề mặt.
\item
  Tuyến tính hóa đơn giản hóa các hàm phức tạp để tính toán.
\item
  Được sử dụng rộng rãi trong các phương pháp số, vật lý và kinh tế.
\end{itemize}

\subsubsection{Bài tập}\label{buxe0i-tux1eadp-35}

\begin{enumerate}
\def\labelenumi{\arabic{enumi}.}
\tightlist
\item
  Tìm mặt phẳng tiếp tuyến của \(z = x^2y + y^2\) tại \((1,1)\).
\item
  Gần đúng \(f(x,y) = e^{x+y}\) gần \((0,0)\).
\item
  Suy ra phương trình mặt phẳng tiếp tuyến của \(z = \ln(x^2+y^2)\) tại
  \((1,1)\).
\item
  Sử dụng phép tính gần đúng tuyến tính để ước tính \(\sqrt{10.1}\) bằng
  cách sử dụng \(f(x,y) = \sqrt{x+y}\) gần (4,6).
\item
  Giải thích tại sao phép tính gần đúng của mặt phẳng tiếp tuyến được
  cải thiện khi \((x,y)\) tiến gần hơn đến \((a,b)\).
\end{enumerate}

\subsection{8.5 Tối ưu hóa nhiều
biến}\label{tux1ed1i-ux1b0u-huxf3a-nhiux1ec1u-biux1ebfn}

Tối ưu hóa trong phép tính nhiều biến mở rộng ý tưởng về cực đại và cực
tiểu từ hàm một biến đến hàm hai biến trở lên.

\subsubsection{Điểm quan trọng}\label{ux111iux1ec3m-quan-trux1ecdng}

Đối với \(f(x,y)\), một điểm tới hạn xảy ra ở đó

\[
f_x(x,y) = 0 \quad \text{and} \quad f_y(x,y) = 0,
\]

hoặc khi đạo hàm riêng không tồn tại.

\subsubsection{Kiểm tra đạo hàm thứ
hai}\label{kiux1ec3m-tra-ux111ux1ea1o-huxe0m-thux1ee9-hai}

Để phân loại các điểm tới hạn, hãy tính đạo hàm từng phần thứ hai:

\[
D = f_{xx}(a,b) f_{yy}(a,b) - \big(f_{xy}(a,b)\big)^2.
\]

\begin{itemize}
\tightlist
\item
  Nếu \(D > 0\) và \(f_{xx}(a,b) > 0\): mức tối thiểu cục bộ.
\item
  Nếu \(D > 0\) và \(f_{xx}(a,b) < 0\): mức tối đa cục bộ.
\item
  Nếu \(D < 0\): điểm yên ngựa.
\item
  Nếu \(D = 0\): kiểm định không thuyết phục.
\end{itemize}

\subsubsection{Ví dụ 1: Paraboloid}\label{vuxed-dux1ee5-1-paraboloid-1}

\(f(x,y) = x^2 + y^2\).

\begin{itemize}
\tightlist
\item
  \(f_x = 2x, f_y = 2y\). Điểm tới hạn tại (0,0).
\item
  \(f_{xx} = 2, f_{yy} = 2, f_{xy} = 0\).
\item
  \(D = (2)(2) - 0 = 4 > 0\), và \(f_{xx} > 0\).
\item
  Vậy (0,0) là cực tiểu địa phương.
\end{itemize}

\subsubsection{Ví dụ 2: Điểm yên
ngựa}\label{vuxed-dux1ee5-2-ux111iux1ec3m-yuxean-ngux1ef1a}

\(f(x,y) = x^2 - y^2\).

\begin{itemize}
\tightlist
\item
  \(f_x = 2x, f_y = -2y\). Điểm tới hạn tại (0,0).
\item
  \(f_{xx} = 2, f_{yy} = -2, f_{xy} = 0\).
\item
  \(D = (2)(-2) - 0 = -4 < 0\).
\item
  Vậy (0,0) là điểm yên ngựa.
\end{itemize}

\subsubsection{Tối ưu hóa ràng buộc và Hệ số
Lagrange}\label{tux1ed1i-ux1b0u-huxf3a-ruxe0ng-buux1ed9c-vuxe0-hux1ec7-sux1ed1-lagrange}

Đôi khi, chúng ta muốn tối ưu hóa \(f(x,y)\) tuân theo một ràng buộc
\(g(x,y) = c\).

Phương pháp nhân Lagrange: giải

\[
\nabla f(x,y) = \lambda \nabla g(x,y).
\]

Ví dụ: Tối đa hóa \(f(x,y) = xy\) tùy thuộc vào \(x^2+y^2=1\).

\begin{itemize}
\tightlist
\item
  Độ dốc:
  \(\nabla f = \langle y,x \rangle, \quad \nabla g = \langle 2x,2y \rangle\).
\item
  Phương trình: \(y = 2\lambda x, \, x = 2\lambda y\).
\item
  Các giải pháp dẫn đến giá trị tối đa ở mức
  \((\pm \tfrac{1}{\sqrt{2}}, \pm \tfrac{1}{\sqrt{2}})\).
\end{itemize}

\subsubsection{Tại sao điều này lại quan
trọng}\label{tux1ea1i-sao-ux111iux1ec1u-nuxe0y-lux1ea1i-quan-trux1ecdng-17}

\begin{itemize}
\tightlist
\item
  Tối ưu hóa là điều cần thiết trong kinh tế, kỹ thuật, học máy và vật
  lý.
\item
  Hệ số nhân Lagrange cho phép tối ưu hóa có ràng buộc, một công cụ quan
  trọng trong toán học ứng dụng.
\end{itemize}

\subsubsection{Bài tập}\label{buxe0i-tux1eadp-36}

\begin{enumerate}
\def\labelenumi{\arabic{enumi}.}
\tightlist
\item
  Tìm và phân loại các điểm tới hạn của \(f(x,y) = x^2+xy+y^2\).
\item
  Phân loại điểm (0,0) cho \(f(x,y) = x^3-y^3\).
\item
  Sử dụng phép thử đạo hàm bậc hai cho \(f(x,y) = x^4+y^4-4xy\).
\item
  Tối đa hóa \(f(x,y) = x+y\) với \(x^2+y^2=1\).
\item
  Giảm thiểu \(f(x,y) = x^2+2y^2\) theo \(x+y=1\).
\end{enumerate}

\section{Chương 9. Tích phân
bội}\label{chux1b0ux1a1ng-9.-tuxedch-phuxe2n-bux1ed9i}

\subsection{9.1 Tích phân kép}\label{tuxedch-phuxe2n-kuxe9p}

Trong phép tính một biến, tích phân xác định cho diện tích dưới một
đường cong. Trong hai biến, tích phân kép tính thể tích dưới một bề mặt
(hay nói chung hơn là tích lũy các giá trị trên một vùng).

\subsubsection{Sự định nghĩa}\label{sux1ef1-ux111ux1ecbnh-nghux129a-8}

Nếu \(f(x,y)\) liên tục trên vùng \(R\), thì tích phân kép là

\[
\iint_R f(x,y)\, dA = \lim_{m,n \to \infty} \sum_{i=1}^m \sum_{j=1}^n f(x_{ij}^-, y_{ij}^-) \Delta A,
\]

trong đó \(R\) được chia thành các hình chữ nhật nhỏ có diện tích
\(\Delta A\).

\subsubsection{Tích phân lặp}\label{tuxedch-phuxe2n-lux1eb7p}

Theo Định lý Fubini, chúng ta có thể tính tích phân kép dưới dạng tích
phân lặp:

\[
\iint_R f(x,y)\, dA = \int_a^b \int_c^d f(x,y)\, dy\, dx,
\]

nếu \(R\) là hình chữ nhật \([a,b] \times [c,d]\).

Thứ tự tích hợp thường có thể được chuyển đổi:

\[
\int_a^b \int_c^d f(x,y)\,dy\,dx = \int_c^d \int_a^b f(x,y)\,dx\,dy.
\]

\subsubsection{Ví dụ}\label{vuxed-dux1ee5-23}

\begin{enumerate}
\def\labelenumi{\arabic{enumi}.}
\tightlist
\item
  Vùng hình chữ nhật
\end{enumerate}

\[
\iint_R (x+y)\, dA, \quad R=[0,1]\times[0,2].
\]

\[
= \int_0^1 \int_0^2 (x+y)\,dy\,dx = \int_0^1 \Big[xy+\tfrac{1}{2}y^2\Big]_0^2 dx
= \int_0^1 (2x+2)dx = 3.
\]

\begin{enumerate}
\def\labelenumi{\arabic{enumi}.}
\setcounter{enumi}{1}
\tightlist
\item
  Vùng tam giác
\end{enumerate}

\[
R = \{(x,y): 0 \leq x \leq 1, 0 \leq y \leq x\}.
\]

\[
\iint_R (x+y)\, dA = \int_0^1 \int_0^x (x+y)\,dy\,dx.
\]

Đánh giá mang lại \(\tfrac{2}{3}\).

\subsubsection{Ứng dụng}\label{ux1ee9ng-dux1ee5ng-1}

\begin{itemize}
\tightlist
\item
  Thể tích dưới bề mặt:
\end{itemize}

\[
V = \iint_R f(x,y)\, dA.
\]

\begin{itemize}
\tightlist
\item
  Giá trị trung bình của hàm số trên một vùng:
\end{itemize}

\[
f_{\text{avg}} = \frac{1}{A(R)} \iint_R f(x,y)\, dA.
\]

\subsubsection{Tại sao điều này lại quan
trọng}\label{tux1ea1i-sao-ux111iux1ec1u-nuxe0y-lux1ea1i-quan-trux1ecdng-18}

Tích phân kép mở rộng tích phân sang hai chiều. Chúng rất cần thiết
trong vật lý (khối lượng, phân bố xác suất), kinh tế (giá trị kỳ vọng)
và kỹ thuật (trọng tâm, thông lượng).

\subsubsection{Bài tập}\label{buxe0i-tux1eadp-37}

\begin{enumerate}
\def\labelenumi{\arabic{enumi}.}
\tightlist
\item
  Tính giá \(\iint_R (x^2+y^2)\, dA\) trong đó \(R=[0,1]\times[0,1]\).
\item
  Tính \(\iint_R xy\, dA\) trong đó
  \(R=\{(x,y):0\leq x\leq2,0\leq y\leq x\}\).
\item
  Tìm giá trị trung bình của \(f(x,y) = x+y\) trên bình phương đơn vị
  \([0,1]\times[0,1]\).
\item
  Giải thích \(\iint_R f(x,y)\, dA\) về mặt xác suất nếu \(f(x,y)\) là
  hàm mật độ xác suất.
\item
  Chứng minh rằng việc chuyển đổi thứ tự tích phân sẽ cho kết quả tương
  tự đối với \(\iint_{[0,1]\times[0,2]} (x+y)\,dA\).
\end{enumerate}

\subsection{9.2 Tích phân ba lớp}\label{tuxedch-phuxe2n-ba-lux1edbp}

Tích phân ba lớp mở rộng ý tưởng tích phân cho ba biến, cho phép chúng
ta tính thể tích, khối lượng và các đại lượng khác trong vùng ba chiều.

\subsubsection{Sự định nghĩa}\label{sux1ef1-ux111ux1ecbnh-nghux129a-9}

Nếu \(f(x,y,z)\) liên tục trên một vùng đặc \(E\), thì tích phân bội ba
là

\[
\iiint_E f(x,y,z)\, dV = \lim_{m,n,p \to \infty} \sum f(x_{ijk}^-, y_{ijk}^-, z_{ijk}^-) \Delta V,
\]

trong đó vùng được chia thành các hộp có tập \(\Delta V\).

\subsubsection{Tích phân lặp}\label{tuxedch-phuxe2n-lux1eb7p-1}

Theo Định lý Fubini, tích phân bội ba có thể được tính dưới dạng tích
phân lặp:

\[
\iiint_E f(x,y,z)\, dV = \int_a^b \int_c^d \int_e^f f(x,y,z)\, dz\, dy\, dx,
\]

cho hình hộp chữ nhật \(E = [a,b]\times[c,d]\times[e,f]\).

Thứ tự tích hợp có thể được lựa chọn để thuận tiện.

\subsubsection{Ví dụ}\label{vuxed-dux1ee5-24}

\begin{enumerate}
\def\labelenumi{\arabic{enumi}.}
\tightlist
\item
  Hộp hình chữ nhật
\end{enumerate}

\[
\iiint_E xyz\, dV, \quad E=[0,1]\times[0,2]\times[0,3].
\]

\[
= \int_0^1 \int_0^2 \int_0^3 xyz\,dz\,dy\,dx.
\]

Tích hợp đầu tiên trên \(z\):

\[
\int_0^3 xyz\,dz = xy \left[\tfrac{1}{2}z^2\right]_0^3 = \tfrac{9}{2}xy.
\]

Bây giờ tích phân trên \(y\):

\[
\int_0^2 \tfrac{9}{2}xy\,dy = \tfrac{9}{2}x \cdot \left[\tfrac{1}{2}y^2\right]_0^2 = 9x.
\]

Cuối cùng lấy tích phân trên \(x\):

\[
\int_0^1 9x\,dx = \tfrac{9}{2}.
\]

\begin{enumerate}
\def\labelenumi{\arabic{enumi}.}
\setcounter{enumi}{1}
\tightlist
\item
  Vùng giới hạn bởi mặt phẳng Đặt
  \(E = \{(x,y,z) \mid 0 \leq x \leq 1, 0 \leq y \leq x, 0 \leq z \leq y\}\).
\end{enumerate}

\[
\iiint_E 1\,dV = \int_0^1 \int_0^x \int_0^y 1\,dz\,dy\,dx.
\]

Đánh giá:

\[
= \int_0^1 \int_0^x y\,dy\,dx = \int_0^1 \tfrac{1}{2}x^2\,dx = \tfrac{1}{6}.
\]

Vậy thể tích của vùng tam giác này là \(\tfrac{1}{6}\).

\subsubsection{Ứng dụng}\label{ux1ee9ng-dux1ee5ng-2}

\begin{itemize}
\item
  Khối lượng: \(V = \iiint_E 1 \, dV\).
\item
  Khối lượng: Nếu mật độ là \(\rho(x,y,z)\) thì

  \[
  M = \iiint_E \rho(x,y,z)\, dV.
  \]
\item
  Average value:

  \[
  f_{\text{avg}} = \frac{1}{V(E)} \iiint_E f(x,y,z)\,dV.
  \]
\end{itemize}

\subsubsection{Tại sao điều này lại quan
trọng}\label{tux1ea1i-sao-ux111iux1ec1u-nuxe0y-lux1ea1i-quan-trux1ecdng-19}

Tích phân ba tổng quát hóa các phép tính diện tích và thể tích cho các
chất rắn tùy ý. Chúng được sử dụng trong vật lý (phân bố khối lượng, tâm
khối lượng, trường hấp dẫn), kỹ thuật và xác suất.

\subsubsection{Bài tập}\label{buxe0i-tux1eadp-38}

\begin{enumerate}
\def\labelenumi{\arabic{enumi}.}
\tightlist
\item
  Tính \(\iiint_E (x+y+z)\,dV\) trên khối lập phương
  \(E=[0,1]\times[0,1]\times[0,1]\).
\item
  Tìm thể tích của tứ diện giới hạn bởi \(x=0, y=0, z=0, x+y+z=1\).
\item
  Tính giá \(\iiint_E x^2 \, dV\) trong đó
  \(E=[0,2]\times[0,1]\times[0,1]\).
\item
  Chứng minh rằng \(\iiint_E 1\,dV\) bằng thể tích hình học của \(E\).
\item
  Nếu mật độ là \(\rho(x,y,z)=x+y+z\), hãy tính khối lượng của khối lập
  phương đơn vị.
\end{enumerate}

\subsection{9.3 Ứng dụng: Khối lượng, Khối lượng, Xác
suất}\label{ux1ee9ng-dux1ee5ng-khux1ed1i-lux1b0ux1ee3ng-khux1ed1i-lux1b0ux1ee3ng-xuxe1c-suux1ea5t}

Tích phân ba chiều rất mạnh vì chúng cho phép chúng ta tính các đại
lượng trong không gian ba chiều bằng cách tích lũy các giá trị trên một
vùng đặc.

\subsubsection{Âm lượng}\label{uxe2m-lux1b0ux1ee3ng}

Ứng dụng đơn giản nhất là tìm thể tích của vùng \(E\):

\[
V = \iiint_E 1 \, dV.
\]

Ví dụ: Tìm thể tích của hình khối giới hạn bởi các mặt phẳng tọa độ và
mặt phẳng \(x+y+z=1\).

\[
V = \iiint_E 1 \, dV = \int_0^1 \int_0^{1-x} \int_0^{1-x-y} 1 \, dz\, dy\, dx.
\]

Đánh giá cho \(V = \tfrac{1}{6}\).

\subsubsection{Khối lượng và mật
độ}\label{khux1ed1i-lux1b0ux1ee3ng-vuxe0-mux1eadt-ux111ux1ed9}

Nếu một vật rắn có hàm mật độ \(\rho(x,y,z)\) thì khối lượng của nó là

\[
M = \iiint_E \rho(x,y,z)\, dV.
\]

Khối tâm được cho bởi

\[
\bar{x} = \frac{1}{M}\iiint_E x\rho(x,y,z)\,dV, \quad
\bar{y} = \frac{1}{M}\iiint_E y\rho(x,y,z)\,dV, \quad
\bar{z} = \frac{1}{M}\iiint_E z\rho(x,y,z)\,dV.
\]

Ví dụ: Đối với một khối lập phương đơn vị có mật độ không đổi
\(\rho=1\), khối tâm là \((0,5,0,5,0,5)\).

\subsubsection{Xác suất}\label{xuxe1c-suux1ea5t}

Nếu \(f(x,y,z)\) là hàm mật độ xác suất trong không gian 3D, thì xác
suất để biến ngẫu nhiên nằm trong vùng \(E\) là

\[
P(E) = \iiint_E f(x,y,z)\, dV,
\]

trong đó \(f(x,y,z) \geq 0\) và

\[
\iiint_{\mathbb{R}^3} f(x,y,z)\,dV = 1.
\]

Ví dụ: Nếu \(f(x,y,z) = \tfrac{3}{4}z^2\) cho \(0 \leq z \leq 1\), thống
nhất trong \(x,y\), thì

\[
P(0 \leq z \leq 0.5) = \int_0^{0.5} \tfrac{3}{4}z^2 \, dz = \tfrac{1}{32}.
\]

\subsubsection{Tại sao điều này lại quan
trọng}\label{tux1ea1i-sao-ux111iux1ec1u-nuxe0y-lux1ea1i-quan-trux1ecdng-20}

\begin{itemize}
\tightlist
\item
  Khối lượng khái quát hình học cho chất rắn không đều.
\item
  Tích phân khối lượng và mật độ kết nối phép tính với vật lý và kỹ
  thuật.
\item
  Hàm mật độ xác suất ở các chiều cao hơn được sử dụng rộng rãi trong
  thống kê và khoa học dữ liệu.
\end{itemize}

\subsubsection{Bài tập}\label{buxe0i-tux1eadp-39}

\begin{enumerate}
\def\labelenumi{\arabic{enumi}.}
\tightlist
\item
  Tìm thể tích của vật rắn giới hạn bởi \(x^2+y^2+z^2 \leq 1\) (hình cầu
  đơn vị).
\item
  Tính khối lượng của một hình nón có mật độ tỉ lệ với \(z\).
\item
  Tìm khối tâm của một tứ diện đều giới hạn bởi
  \(x=0, y=0, z=0, x+y+z=1\).
\item
  Nếu \(f(x,y,z) = \frac{1}{8}\) trên khối
  \([0,2]\times[0,2]\times[0,2]\), hãy xác minh rằng đó là hàm mật độ
  xác suất.
\item
  Sử dụng tích phân bội ba để tính xác suất để một điểm được chọn ngẫu
  nhiên trong hình cầu đơn vị có \(z > 0\).
\end{enumerate}

\subsection{9.4 Thay đổi các biến: Tọa độ cực, tọa độ trụ, tọa độ
cầu}\label{thay-ux111ux1ed5i-cuxe1c-biux1ebfn-tux1ecda-ux111ux1ed9-cux1ef1c-tux1ecda-ux111ux1ed9-trux1ee5-tux1ecda-ux111ux1ed9-cux1ea7u}

Nhiều tích phân trở nên dễ dàng hơn khi được biểu diễn trong hệ tọa độ
phù hợp với tính đối xứng của vùng. Thay vì tọa độ Descartes
\((x,y,z)\), chúng ta có thể sử dụng tọa độ cực, hình trụ hoặc hình cầu.

\subsubsection{Tọa độ cực (2D)}\label{tux1ecda-ux111ux1ed9-cux1ef1c-2d}

Đối với hàm hai biến, chúng ta có thể chuyển sang tọa độ cực:

\[
x = r\cos\theta, \quad y = r\sin\theta, \quad r \geq 0, \; 0 \leq \theta < 2\pi.
\]

Phần tử diện tích biến đổi thành

\[
dA = r\,dr\,d\theta.
\]

Ví dụ: Tìm diện tích của hình tròn đơn vị.

\[
A = \iint_{x^2+y^2\leq 1} 1\,dA = \int_0^{2\pi}\int_0^1 r\,dr\,d\theta = \pi.
\]

\subsubsection{Tọa độ hình trụ
(3D)}\label{tux1ecda-ux111ux1ed9-huxecnh-trux1ee5-3d}

Trong 3D, tọa độ trụ mở rộng tọa độ cực với \(z\):

\[
x = r\cos\theta, \quad y = r\sin\theta, \quad z = z.
\]

Phần tử khối lượng là

\[
dV = r\,dr\,d\theta\,dz.
\]

Ví dụ: Thể tích hình trụ có bán kính \(R\) và chiều cao \(h\):

\[
V = \int_0^h \int_0^{2\pi} \int_0^R r\,dr\,d\theta\,dz = \pi R^2 h.
\]

\subsubsection{Tọa độ cầu (3D)}\label{tux1ecda-ux111ux1ed9-cux1ea7u-3d}

Đối với sự đối xứng hình cầu, sử dụng:

\[
x = \rho \sin\phi \cos\theta, \quad y = \rho \sin\phi \sin\theta, \quad z = \rho \cos\phi,
\]

Ở đâu

\begin{itemize}
\tightlist
\item
  \(\rho \geq 0\) là khoảng cách từ gốc tọa độ,
\item
  \(0 \leq \phi \leq \pi\) là góc tính từ trục \(z\) dương,
\item
  \(0 \leq \theta < 2\pi\) là góc trong mặt phẳng \(xy\).
\end{itemize}

Phần tử khối lượng là

\[
dV = \rho^2 \sin\phi \, d\rho\, d\phi\, d\theta.
\]

Ví dụ: Thể tích khối cầu đơn vị:

\[
V = \int_0^{2\pi} \int_0^\pi \int_0^1 \rho^2 \sin\phi \, d\rho\, d\phi\, d\theta.
\]

Đánh giá:

\[
V = \left(\int_0^1 \rho^2 d\rho\right)\left(\int_0^\pi \sin\phi d\phi\right)\left(\int_0^{2\pi} d\theta\right) = \tfrac{1}{3}(2)(2\pi) = \tfrac{4\pi}{3}.
\]

\subsubsection{Tại sao điều này lại quan
trọng}\label{tux1ea1i-sao-ux111iux1ec1u-nuxe0y-lux1ea1i-quan-trux1ecdng-21}

\begin{itemize}
\tightlist
\item
  Tọa độ cực đơn giản hóa các vùng hình tròn.
\item
  Tọa độ trụ xử lý hình trụ và đối xứng quay.
\item
  Tọa độ cầu đơn giản hóa các bài toán về hình cầu, hình nón và bán
  kính.
\item
  Những thay đổi của các biến này làm cho các tích phân không thể quản
  lý được.
\end{itemize}

\subsubsection{Bài tập}\label{buxe0i-tux1eadp-40}

\begin{enumerate}
\def\labelenumi{\arabic{enumi}.}
\tightlist
\item
  Tính \(\iint_{x^2+y^2\leq 4} (x^2+y^2)\,dA\) sử dụng tọa độ cực.
\item
  Tìm thể tích hình nón có chiều cao \(h\) và bán kính \(R\) bằng tọa độ
  trụ.
\item
  Sử dụng tọa độ cầu để tính thể tích của quả cầu bán kính \(R\).
\item
  Chứng minh rằng hệ số Jacobian cho tọa độ cực là \(r\).
\item
  Tìm khối lượng của một quả cầu đặc có bán kính \(R\) với mật độ tỉ lệ
  với khoảng cách từ gốc tọa độ cầu.
\end{enumerate}

\section{Chương 10. Phép tính
vectơ}\label{chux1b0ux1a1ng-10.-phuxe9p-tuxednh-vectux1a1}

\subsection{10.1 Trường vectơ}\label{trux1b0ux1eddng-vectux1a1}

Trường vectơ gán một vectơ cho mỗi điểm trong không gian, giống như hàm
vô hướng gán một số. Trường vectơ được sử dụng để mô hình hóa dòng chảy,
lực và các đại lượng định hướng khác.

\subsubsection{Sự định nghĩa}\label{sux1ef1-ux111ux1ecbnh-nghux129a-10}

Trong hai chiều, trường vectơ là một hàm

\[
\mathbf{F}(x,y) = \langle P(x,y), Q(x,y) \rangle,
\]

trong đó \(P\) và \(Q\) là các hàm vô hướng.

Trong ba chiều,

\[
\mathbf{F}(x,y,z) = \langle P(x,y,z), Q(x,y,z), R(x,y,z) \rangle.
\]

\subsubsection{Ví dụ}\label{vuxed-dux1ee5-25}

\begin{enumerate}
\def\labelenumi{\arabic{enumi}.}
\tightlist
\item
  Trường xuyên tâm
\end{enumerate}

\[
\mathbf{F}(x,y) = \langle x, y \rangle.
\]

Các vectơ hướng ra ngoài từ gốc tọa độ.

\begin{enumerate}
\def\labelenumi{\arabic{enumi}.}
\setcounter{enumi}{1}
\tightlist
\item
  Trường quay
\end{enumerate}

\[
\mathbf{F}(x,y) = \langle -y, x \rangle.
\]

Các vectơ quay quanh gốc tọa độ.

\begin{enumerate}
\def\labelenumi{\arabic{enumi}.}
\setcounter{enumi}{2}
\tightlist
\item
  Trường hấp dẫn
\end{enumerate}

\[
\mathbf{F}(x,y,z) = -\frac{GM}{r^3}\langle x,y,z \rangle, \quad r=\sqrt{x^2+y^2+z^2}.
\]

\subsubsection{Trực quan hóa các trường
Vector}\label{trux1ef1c-quan-huxf3a-cuxe1c-trux1b0ux1eddng-vector}

\begin{itemize}
\tightlist
\item
  Vẽ các mũi tên nhỏ tại các điểm mẫu để chỉ hướng và độ lớn.
\item
  Mũi tên dày đặc hơn nơi cường độ lớn hơn.
\item
  Hữu ích cho việc giải thích các đường dòng chảy, quỹ đạo và lực.
\end{itemize}

\subsubsection{Đường dòng
chảy}\label{ux111ux1b0ux1eddng-duxf2ng-chux1ea3y}

Một đường dòng (hoặc đường cong tích phân) của trường vectơ là một đường
cong \(\mathbf{r}(t)\) có vectơ tiếp tuyến tại mỗi điểm khớp với trường:

\[
\mathbf{r}'(t) = \mathbf{F}(\mathbf{r}(t)).
\]

Các đường dòng mô tả đường đi của hạt trong trường vận tốc.

\subsubsection{Tại sao điều này lại quan
trọng}\label{tux1ea1i-sao-ux111iux1ec1u-nuxe0y-lux1ea1i-quan-trux1ecdng-22}

\begin{itemize}
\tightlist
\item
  Trường vectơ là cơ sở của vật lý (dòng chất lỏng, điện từ, lực hấp
  dẫn).
\item
  Chúng tạo thành cơ sở của tích phân đường, tích phân mặt và các định
  lý lớn của phép tính vectơ (Green, Stokes, Divergence).
\item
  Cung cấp một cách hình học để biểu diễn các đại lượng có hướng.
\end{itemize}

\subsubsection{Bài tập}\label{buxe0i-tux1eadp-41}

\begin{enumerate}
\def\labelenumi{\arabic{enumi}.}
\tightlist
\item
  Vẽ trường vectơ \(\mathbf{F}(x,y) = \langle y, -x \rangle\).
\item
  Xác định xem các vectơ của \(\mathbf{F}(x,y) = \langle x,y \rangle\)
  hướng về phía hay ra xa gốc tọa độ.
\item
  Với \(\mathbf{F}(x,y,z) = \langle y, z, x \rangle\), hãy tính
  \(\mathbf{F}(1,2,3)\).
\item
  Mô tả các đường dòng của \(\mathbf{F}(x,y) = \langle -y, x \rangle\).
\item
  Giải thích tại sao trường hấp dẫn và điện trường là ví dụ của trường
  vectơ hướng tâm.
\end{enumerate}

\subsection{10.2 Tích phân
đường}\label{tuxedch-phuxe2n-ux111ux1b0ux1eddng}

Tích phân đường mở rộng ý tưởng về tích phân cho các hàm được tính dọc
theo một đường cong. Thay vì tích phân trên một khoảng hoặc vùng, chúng
ta tích phân trên một đường trong không gian.

\subsubsection{Định nghĩa: Tích phân đường vô
hướng}\label{ux111ux1ecbnh-nghux129a-tuxedch-phuxe2n-ux111ux1b0ux1eddng-vuxf4-hux1b0ux1edbng}

Nếu \(f(x,y)\) là hàm vô hướng và \(C\) là đường cong được tham số hóa
bởi \(\mathbf{r}(t) = \langle x(t), y(t) \rangle, \; a \leq t \leq b\)
thì tích phân đường là

\[
\int_C f(x,y)\, ds = \int_a^b f(x(t),y(t)) \, |\mathbf{r}'(t)|\, dt,
\]

trong đó \(ds\) là độ dài cung.

Điều này đo lường sự tích lũy của \(f\) dọc theo đường cong.

\subsubsection{Định nghĩa: Tích phân đường
vectơ}\label{ux111ux1ecbnh-nghux129a-tuxedch-phuxe2n-ux111ux1b0ux1eddng-vectux1a1}

Với trường vectơ \(\mathbf{F}(x,y) = \langle P(x,y), Q(x,y) \rangle\),
tích phân đường dọc theo \(C\) là

\[
\int_C \mathbf{F} \cdot d\mathbf{r} = \int_a^b \mathbf{F}(\mathbf{r}(t)) \cdot \mathbf{r}'(t)\, dt.
\]

Điều này đo công được thực hiện bởi trường dọc theo đường cong.

\subsubsection{Ví dụ}\label{vuxed-dux1ee5-26}

\begin{enumerate}
\def\labelenumi{\arabic{enumi}.}
\tightlist
\item
  Tích phân đường vô hướng
\end{enumerate}

\[
f(x,y) = x+y, \quad C: \mathbf{r}(t) = \langle t, t^2 \rangle, \; 0 \leq t \leq 1.
\]

Sau đó

\[
\int_C f(x,y)\, ds = \int_0^1 (t+t^2)\sqrt{(1)^2+(2t)^2}\, dt.
\]

\begin{enumerate}
\def\labelenumi{\arabic{enumi}.}
\setcounter{enumi}{1}
\tightlist
\item
  Công việc được thực hiện bởi một lực lượng
\end{enumerate}

\[
\mathbf{F}(x,y) = \langle y, x \rangle, \quad C: \mathbf{r}(t) = \langle t, t^2 \rangle, \; 0 \leq t \leq 1.
\]

\[
\int_C \mathbf{F} \cdot d\mathbf{r} = \int_0^1 \langle t^2, t \rangle \cdot \langle 1, 2t \rangle\, dt = \int_0^1 (t^2 + 2t^2)\, dt = \int_0^1 3t^2\, dt = 1.
\]

\subsubsection{Giải thích vật
lý}\label{giux1ea3i-thuxedch-vux1eadt-luxfd}

\begin{itemize}
\tightlist
\item
  Tích phân đường vô hướng: tích lũy mật độ dọc theo một dây.
\item
  Tích phân đường vectơ: công được thực hiện bởi một lực di chuyển một
  vật dọc theo một đường đi.
\end{itemize}

\subsubsection{Tại sao điều này lại quan
trọng}\label{tux1ea1i-sao-ux111iux1ec1u-nuxe0y-lux1ea1i-quan-trux1ecdng-23}

\begin{itemize}
\tightlist
\item
  Tích phân đường kết nối các trường vectơ với các đại lượng vật lý như
  công và tuần hoàn.
\item
  Chúng là những khối xây dựng cho Định lý Green và Định lý Stokes.
\item
  Xuất hiện trong vật lý (điện thế, dòng chất lỏng, cơ học).
\end{itemize}

\subsubsection{Bài tập}\label{buxe0i-tux1eadp-42}

\begin{enumerate}
\def\labelenumi{\arabic{enumi}.}
\tightlist
\item
  Tính \(\int_C (x^2+y^2)\, ds\) trong đó \(C\) là đoạn thẳng từ (0,0)
  đến (1,1).
\item
  Đánh giá \(\int_C \mathbf{F}\cdot d\mathbf{r}\) với
  \(\mathbf{F}(x,y) = \langle -y, x \rangle\) dọc theo vòng tròn đơn vị
  \(x^2+y^2=1\).
\item
  Giải thích ý nghĩa của \(\int_C 1\,ds\).
\item
  Với \(\mathbf{F}(x,y,z) = \langle z,0,x \rangle\), hãy tính tích phân
  đường dọc theo
  \(\mathbf{r}(t) = \langle t,t,1 \rangle, 0 \leq t \leq 1\).
\item
  Giải thích sự khác biệt giữa tích phân đường vô hướng và tích phân
  đường vectơ.
\end{enumerate}

\subsection{10.3 Tích phân bề
mặt}\label{tuxedch-phuxe2n-bux1ec1-mux1eb7t}

Tích phân mặt khái quát hóa tích phân đường cho các bề mặt hai chiều
trong không gian ba chiều. Chúng cho phép chúng ta tính toán thông lượng
qua các bề mặt và sự tích lũy của trường vô hướng trên các bề mặt cong.

\subsubsection{Tích phân bề mặt vô
hướng}\label{tuxedch-phuxe2n-bux1ec1-mux1eb7t-vuxf4-hux1b0ux1edbng}

Nếu một bề mặt \(S\) được tham số hóa bởi

\[
\mathbf{r}(u,v) = \langle x(u,v), y(u,v), z(u,v) \rangle,
\]

thì tích phân bề mặt của hàm vô hướng \(f(x,y,z)\) là

\[
\iint_S f(x,y,z)\, dS = \iint_D f(\mathbf{r}(u,v)) \, |\mathbf{r}_u \times \mathbf{r}_v| \, du\,dv,
\]

trong đó \(\mathbf{r></u\) và \(\mathbf{r__v\) là đạo hàm riêng của
\(\mathbf{r}(u,v)\) và \(D\) là miền tham số.

\subsubsection{Tích phân bề mặt vectơ (Thông
lượng)}\label{tuxedch-phuxe2n-bux1ec1-mux1eb7t-vectux1a1-thuxf4ng-lux1b0ux1ee3ng}

Đối với trường vectơ \(\mathbf{F}(x,y,z)\), thông lượng qua một bề mặt
\(S\) là

\[
\iint_S \mathbf{F}\cdot d\mathbf{S} = \iint_S \mathbf{F}\cdot \mathbf{n}\, dS,
\]

trong đó \(\mathbf{n}\) là vectơ pháp tuyến đơn vị. Sử dụng tham số hóa,

\[
\iint_S \mathbf{F}\cdot d\mathbf{S} = \iint_D \mathbf{F}(\mathbf{r}(u,v)) \cdot (\mathbf{r}_u \times \mathbf{r}_v)\,du\,dv.
\]

\subsubsection{Ví dụ}\label{vuxed-dux1ee5-27}

\begin{enumerate}
\def\labelenumi{\arabic{enumi}.}
\tightlist
\item
  Tích phân bề mặt vô hướng Bề mặt: mặt phẳng \(z=1\) trên đĩa đơn
  \(x^2+y^2 \leq 1\).
\end{enumerate}

\[
\iint_S 1\, dS = \text{area of the disk} = \pi.
\]

\begin{enumerate}
\def\labelenumi{\arabic{enumi}.}
\setcounter{enumi}{1}
\tightlist
\item
  Dòng chảy qua một quả cầu Đặt
  \(\mathbf{F}(x,y,z) = \langle x,y,z \rangle\), và \(S\) = hình cầu bán
  kính \(R\). Vectơ pháp tuyến là
  \(\mathbf{n} = \frac{1}{R}\langle x,y,z \rangle\).
\end{enumerate}

\[
\mathbf{F}\cdot \mathbf{n} = \frac{x^2+y^2+z^2}{R} = R.
\]

Vì thế

\[
\iint_S \mathbf{F}\cdot d\mathbf{S} = \iint_S R\, dS = R \cdot 4\pi R^2 = 4\pi R^3.
\]

\subsubsection{Tại sao điều này lại quan
trọng}\label{tux1ea1i-sao-ux111iux1ec1u-nuxe0y-lux1ea1i-quan-trux1ecdng-24}

\begin{itemize}
\tightlist
\item
  Tích phân bề mặt vô hướng đo diện tích và phân bố bề mặt.
\item
  Tích phân mặt vector đo thông lượng: lượng trường truyền qua một mặt.
\item
  Ứng dụng: điện từ, dòng chất lỏng, truyền nhiệt, v.v.
\end{itemize}

\subsubsection{Bài tập}\label{buxe0i-tux1eadp-43}

\begin{enumerate}
\def\labelenumi{\arabic{enumi}.}
\tightlist
\item
  Tính \(\iint_S 1\, dS\) cho bề mặt của hình lập phương có cạnh 2.
\item
  Tìm thông lượng của \(\mathbf{F}(x,y,z) = \langle x,y,z \rangle\) qua
  mặt cầu đơn vị.
\item
  Tính \(\iint_S z\, dS\) cho paraboloid \(z = x^2+y^2, \, z \leq 1\).
\item
  Với \(\mathbf{F}(x,y,z) = \langle y,0,0 \rangle\), tính thông lượng
  qua mặt phẳng \(x=1\), \(0 \leq y,z \leq 1\).
\item
  Giải thích về mặt vật lý ý nghĩa của thông lượng của trường vectơ qua
  một bề mặt kín bằng không.
\end{enumerate}

\subsection{10.4 Định lý Green}\label{ux111ux1ecbnh-luxfd-green}

Định lý Green là một kết quả cơ bản trong phép tính vectơ nối tích phân
đường xung quanh một đường cong kín với tích phân kép trên vùng mà nó
bao quanh. Đây là phiên bản hai chiều của Định lý Stokes.

\subsubsection{Tuyên bố Định lý
Green}\label{tuyuxean-bux1ed1-ux111ux1ecbnh-luxfd-green}

Giả sử \(C\) là một đường cong khép kín, đơn giản, có hướng dương trong
mặt phẳng và gọi \(R\) là vùng mà nó bao quanh. Nếu
\(\mathbf{F}(x,y) = \langle P(x,y), Q(x,y) \rangle\) có đạo hàm riêng
liên tục trên một vùng mở chứa \(R\), thì

\[
\oint_C \mathbf{F} \cdot d\mathbf{r} = \oint_C P\,dx + Q\,dy = \iint_R \left( \frac{\partial Q}{\partial x} - \frac{\partial P}{\partial y} \right)\, dA.
\]

\subsubsection{Phiên dịch}\label{phiuxean-dux1ecbch-2}

\begin{itemize}
\tightlist
\item
  Tích phân đường quanh \(C\) đo sự hoàn lưu của trường vectơ dọc theo
  đường biên.
\item
  Tích phân kép trên \(R\) đo tổng độ cong (xoay) của trường bên trong
  vùng.
\end{itemize}

\subsubsection{Ví dụ 1: Công thức tính diện
tích}\label{vuxed-dux1ee5-1-cuxf4ng-thux1ee9c-tuxednh-diux1ec7n-tuxedch}

Nếu \(\mathbf{F} = \langle -y/2, x/2 \rangle\), thì

\[
\frac{\partial Q}{\partial x} - \frac{\partial P}{\partial y} = 1.
\]

Do đó, Định lý Green cho

\[
\text{Area}(R) = \iint_R 1\,dA = \oint_C \left(-\tfrac{y}{2}\,dx + \tfrac{x}{2}\,dy\right).
\]

Điều này cung cấp một cách tính diện tích bằng cách sử dụng tích phân
đường.

\subsubsection{Ví dụ 2: Lưu
thông}\label{vuxed-dux1ee5-2-lux1b0u-thuxf4ng}

Đặt \(\mathbf{F}(x,y) = \langle -y, x \rangle\), và \(C\) là vòng tròn
đơn vị.

\begin{itemize}
\tightlist
\item
  \(P=-y, Q=x\).
\item
  \(Q_x - P_y = 1 - (-1) = 2\).
\item
  Tích phân kép trên đĩa đơn vị:
\end{itemize}

\[
\iint_R 2\,dA = 2\pi (1^2) = 2\pi.
\]

Vậy vòng tuần hoàn quanh vòng tròn là \(2\pi\).

\subsubsection{Tại sao điều này lại quan
trọng}\label{tux1ea1i-sao-ux111iux1ec1u-nuxe0y-lux1ea1i-quan-trux1ecdng-25}

\begin{itemize}
\tightlist
\item
  Chuyển tích phân đường khó thành tích phân kép hoặc ngược lại.
\item
  Cung cấp cầu nối giữa các thuộc tính cục bộ (curl) và các thuộc tính
  toàn cầu (lưu thông).
\item
  Được sử dụng rộng rãi trong vật lý cho dòng chất lỏng, điện từ và
  trường vectơ phẳng.
\end{itemize}

\subsubsection{Bài tập}\label{buxe0i-tux1eadp-44}

\begin{enumerate}
\def\labelenumi{\arabic{enumi}.}
\tightlist
\item
  Sử dụng Định lý Green để tính diện tích bên trong hình elip
  \(\frac{x^2}{a^2} + \frac{y^2}{b^2} = 1\).
\item
  Chứng minh Định lý Green cho
  \(\mathbf{F}(x,y) = \langle -y, x \rangle\) dọc theo hình vuông có các
  đỉnh (0,0), (1,0), (1,1), (0,1).
\item
  Tính chu kỳ của \(\mathbf{F}(x,y) = \langle -y, x \rangle\) xung quanh
  vòng tròn đơn vị.
\item
  Chứng minh rằng nếu \(\nabla \times \mathbf{F} = 0\) thì tích phân
  đường của \(\mathbf{F}\) xung quanh bất kỳ đường cong kín nào đều bằng
  0.
\item
  Sử dụng Định lý Green để chứng minh rằng
\end{enumerate}

\[
\oint_C x\,dy = -\oint_C y\,dx
\]

với mọi đường cong khép kín \(C\).

\subsection{Định lý 10,5 Stokes}\label{ux111ux1ecbnh-luxfd-105-stokes}

Định lý Stokes khái quát Định lý Green thành ba chiều. Nó liên hệ tích
phân bề mặt của độ cong của trường vectơ trên một bề mặt với tích phân
đường của trường vectơ xung quanh biên của bề mặt đó.

\subsubsection{Phát biểu định lý
Stokes}\label{phuxe1t-biux1ec3u-ux111ux1ecbnh-luxfd-stokes}

Cho \(S\) là một bề mặt nhẵn, có định hướng với đường cong biên \(C\)
(có hướng dương). Nếu \(\mathbf{F}(x,y,z)\) là trường vectơ có đạo hàm
riêng liên tục, thì

\[
\iint_S (\nabla \times \mathbf{F}) \cdot d\mathbf{S} = \oint_C \mathbf{F} \cdot d\mathbf{r}.
\]

\begin{itemize}
\tightlist
\item
  Phía bên trái: dòng chuyển động của đường cong \(\mathbf{F}\) qua bề
  mặt.
\item
  Bên phải: sự tuần hoàn của \(\mathbf{F}\) dọc theo đường cong biên.
\end{itemize}

\subsubsection{Phiên dịch}\label{phiuxean-dux1ecbch-3}

\begin{itemize}
\tightlist
\item
  Tích phân đường xung quanh biên bằng tổng ``phép quay'' bên trong bề
  mặt.
\item
  Mở rộng Định lý Green (trường hợp đặc biệt khi bề mặt nằm trong mặt
  phẳng).
\end{itemize}

\subsubsection{Ví dụ 1: Định lý Green là trường hợp đặc
biệt}\label{vuxed-dux1ee5-1-ux111ux1ecbnh-luxfd-green-luxe0-trux1b0ux1eddng-hux1ee3p-ux111ux1eb7c-biux1ec7t}

Nếu \(S\) là một vùng phẳng trong mặt phẳng \(xy\), Định lý Stokes rút
gọn thành Định lý Green.

\subsubsection{Ví dụ 2: Hoàn lưu trên bán
cầu}\label{vuxed-dux1ee5-2-houxe0n-lux1b0u-truxean-buxe1n-cux1ea7u}

Đặt \(\mathbf{F}(x,y,z) = \langle -y, x, 0 \rangle\), và \(S\) là bán
cầu trên của bán kính 1.

\begin{itemize}
\tightlist
\item
  Biên \(C\): đường tròn đơn vị trong mặt phẳng \(xy\).
\item
  Theo định lý Stokes:
\end{itemize}

\[
\oint_C \mathbf{F}\cdot d\mathbf{r} = \iint_S (\nabla \times \mathbf{F})\cdot d\mathbf{S}.
\]

\begin{itemize}
\tightlist
\item
  Curl: \(\nabla \times \mathbf{F} = \langle 0,0,2 \rangle\).
\item
  Điểm bình thường đến bán cầu hướng ra ngoài:
  \(\mathbf{n} = \langle 0,0,1 \rangle\).
\item
  Vậy tích phân = 2.
\item
  Diện tích bán cầu = \(2\pi (1^2)\).
\end{itemize}

\[
\iint_S 2\, dS = 2 \cdot 2\pi = 4\pi.
\]

Do đó, lưu thông quanh xích đạo là \(4\pi\).

\subsubsection{Tại sao điều này lại quan
trọng}\label{tux1ea1i-sao-ux111iux1ec1u-nuxe0y-lux1ea1i-quan-trux1ecdng-26}

\begin{itemize}
\tightlist
\item
  Cung cấp một kết nối sâu sắc giữa tích phân bề mặt và tích phân đường.
\item
  Đơn giản hóa việc tính toán bằng cách cho phép lựa chọn các bề mặt
  thuận tiện.
\item
  Được sử dụng rộng rãi trong điện từ học (Định luật Faraday) và động
  lực học chất lỏng.
\end{itemize}

\subsubsection{Bài tập}\label{buxe0i-tux1eadp-45}

\begin{enumerate}
\def\labelenumi{\arabic{enumi}.}
\tightlist
\item
  Kiểm chứng Định lý Stokes cho
  \(\mathbf{F}(x,y,z) = \langle y, -x, 0 \rangle\) trên đĩa đơn vị trong
  mặt phẳng \(xy\).
\item
  Tính \(\oint_C \mathbf{F}\cdot d\mathbf{r}\) trong đó
  \(\mathbf{F}(x,y,z) = \langle z, 0, x \rangle\), và \(C\) là ranh giới
  của tam giác với các đỉnh (0,0,0), (1,0,0), (0,1,0).
\item
  Chứng minh rằng nếu \(\nabla \times \mathbf{F} = 0\), thì vòng tuần
  hoàn quanh bất kỳ đường cong kín nào đều bằng không.
\item
  Áp dụng Định lý Stokes để tính chu trình của
  \(\mathbf{F}(x,y,z) = \langle -y, x, z \rangle\) xung quanh ranh giới
  của hình vuông đơn vị trong mặt phẳng \(z=0\).
\item
  Giải thích Định lý Stokes khái quát hóa Định lý Green như thế nào.
\end{enumerate}

\subsection{10.6 Định lý phân
kỳ}\label{ux111ux1ecbnh-luxfd-phuxe2n-kux1ef3}

Định lý phân kỳ (còn gọi là Định lý Gauss) liên hệ dòng của một trường
vectơ qua một bề mặt kín với tích phân ba lớp của sự phân kỳ của trường
bên trong bề mặt.

\subsubsection{Phát biểu về Định lý phân
kỳ}\label{phuxe1t-biux1ec3u-vux1ec1-ux111ux1ecbnh-luxfd-phuxe2n-kux1ef3}

Cho \(E\) là một vùng đặc trong \(\mathbb{R}^3\) với bề mặt biên \(S\)
(hướng ra ngoài). Nếu \(\mathbf{F}(x,y,z)\) là trường vectơ có đạo hàm
riêng liên tục trên \(E\), thì

\[
\iint_S \mathbf{F} \cdot d\mathbf{S} = \iiint_E (\nabla \cdot \mathbf{F}) \, dV.
\]

\begin{itemize}
\tightlist
\item
  Vế trái: thông lượng \(\mathbf{F}\) đi qua mặt kín \(S\).
\item
  Vế phải: tích phân bội ba của phân kỳ bên trong miền.
\end{itemize}

\#\#\#Sự khác biệt

Sự phân kỳ của trường vectơ
\(\mathbf{F}(x,y,z) = \langle P, Q, R \rangle\) là

\[
\nabla \cdot \mathbf{F} = \frac{\partial P}{\partial x} + \frac{\partial Q}{\partial y} + \frac{\partial R}{\partial z}.
\]

Nó đo ``dòng chảy ra ròng'' trên một đơn vị thể tích tại mỗi điểm.

\subsubsection{Ví dụ 1: Thông lượng của trường xuyên
tâm}\label{vuxed-dux1ee5-1-thuxf4ng-lux1b0ux1ee3ng-cux1ee7a-trux1b0ux1eddng-xuyuxean-tuxe2m}

Đặt \(\mathbf{F}(x,y,z) = \langle x, y, z \rangle\), và gọi \(E\) là quả
cầu đơn vị \(x^2+y^2+z^2 \leq 1\).

\begin{itemize}
\tightlist
\item
  Phân kỳ: \(\nabla \cdot \mathbf{F} = 1+1+1 = 3\).
\item
  Thể tích đơn vị bóng: \(\tfrac{4}{3}\pi\). Vì thế
\end{itemize}

\[
\iiint_E (\nabla \cdot \mathbf{F})\, dV = 3 \cdot \tfrac{4}{3}\pi = 4\pi.
\]

Do đó, thông lượng trên hình cầu đơn vị là \(4\pi\).

\subsubsection{Ví dụ 2: Trường
hằng}\label{vuxed-dux1ee5-2-trux1b0ux1eddng-hux1eb1ng}

Đặt \(\mathbf{F}(x,y,z) = \langle 1, 0, 0 \rangle\).

\begin{itemize}
\tightlist
\item
  Phân kỳ: \(\nabla \cdot \mathbf{F} = 0\).
\item
  Vậy dòng thông qua bất kỳ bề mặt kín nào đều bằng 0, phù hợp với trực
  giác (không có dòng chảy ròng).
\end{itemize}

\subsubsection{Tại sao điều này lại quan
trọng}\label{tux1ea1i-sao-ux111iux1ec1u-nuxe0y-lux1ea1i-quan-trux1ecdng-27}

\begin{itemize}
\item
  Chuyển tích phân mặt thành tích phân thể tích đơn giản hơn.
\item
  Ứng dụng trong vật lý: Định luật Gauss trong điện từ, dòng chất lỏng
  và truyền nhiệt.
\item
  Hoàn thiện khuôn khổ thống nhất:
\item
  Định lý Green (2D ↔ tuần hoàn)

  \begin{itemize}
  \tightlist
  \item
    Định lý Stokes (độ cong 3D ↔ tuần hoàn trên bề mặt)
  \item
    Định lý phân kỳ (phân kỳ 3D ↔ thông lượng trên các bề mặt kín)
  \end{itemize}
\end{itemize}

\subsubsection{Bài tập}\label{buxe0i-tux1eadp-46}

\begin{enumerate}
\def\labelenumi{\arabic{enumi}.}
\tightlist
\item
  Sử dụng Định lý phân kỳ để tính dòng
  \(\mathbf{F}(x,y,z) = \langle x,y,z \rangle\) trên bề mặt của một hình
  cầu có bán kính \(R\).
\item
  Chứng minh Định lý phân kỳ cho
  \(\mathbf{F}(x,y,z) = \langle y, z, x \rangle\) trên khối lập phương
  đơn vị \([0,1]^3\).
\item
  Chứng minh rằng nếu \(\nabla \cdot \mathbf{F} = 0\) thì tổng thông
  lượng qua bất kỳ bề mặt kín nào đều bằng không.
\item
  Tính thông lượng của
  \(\mathbf{F}(x,y,z) = \langle x^2, y^2, z^2 \rangle\) qua hình cầu đơn
  vị.
\item
  Giải thích cách Định lý Phân kỳ khái quát hóa Định lý Cơ bản một chiều
  của Giải tích.
\end{enumerate}

\#Phần IV. Quy trình vô hạn

\section{Chương 11. Trình tự và sự hội
tụ}\label{chux1b0ux1a1ng-11.-truxecnh-tux1ef1-vuxe0-sux1ef1-hux1ed9i-tux1ee5}

\subsection{11.1 Định nghĩa và ví
dụ}\label{ux111ux1ecbnh-nghux129a-vuxe0-vuxed-dux1ee5}

Dãy số là một danh sách các số có thứ tự, thường được viết dưới dạng

\[
a_1, a_2, a_3, \dots
\]

hay nói chung hơn là \((a_n)_{n=1}^\infty\). Mỗi \(a_n\) được gọi là số
hạng thứ n của dãy.

\subsubsection{Xác định trình
tự}\label{xuxe1c-ux111ux1ecbnh-truxecnh-tux1ef1}

Một chuỗi có thể được xác định theo hai cách:

\begin{enumerate}
\def\labelenumi{\arabic{enumi}.}
\tightlist
\item
  Công thức rõ ràng -- đưa ra quy tắc trực tiếp cho số hạng thứ n.
\end{enumerate}

\begin{itemize}
\item
  Ví dụ: \(a_n = \frac{1}{n}\) xác định dãy

  \[
     1, \tfrac{1}{2}, \tfrac{1}{3}, \tfrac{1}{4}, \dots
     \]
\end{itemize}

\begin{enumerate}
\def\labelenumi{\arabic{enumi}.}
\setcounter{enumi}{1}
\tightlist
\item
  Định nghĩa đệ quy -- định nghĩa các thuật ngữ sử dụng các thuật ngữ
  trước đó.
\end{enumerate}

\begin{itemize}
\item
  Ví dụ: Dãy Fibonacci:

  \[
     a_1 = 1, \quad a_2 = 1, \quad a_{n} = a_{n-1} + a_{n-2} \quad (n \geq 3).
     \]
\end{itemize}

\subsubsection{Ví dụ về trình
tự}\label{vuxed-dux1ee5-vux1ec1-truxecnh-tux1ef1}

\begin{enumerate}
\def\labelenumi{\arabic{enumi}.}
\item
  Dãy số học:

  \[
  a_n = a_1 + (n-1)d.
  \]
\end{enumerate}

Ví dụ: \(a_n = 2n+1\) → dãy số lẻ.

\begin{enumerate}
\def\labelenumi{\arabic{enumi}.}
\setcounter{enumi}{1}
\item
  Trình tự hình học:

  \[
  a_n = a_1 r^{n-1}.
  \]
\end{enumerate}

Ví dụ: \(a_n = 2^n\) → lũy thừa của 2.

\begin{enumerate}
\def\labelenumi{\arabic{enumi}.}
\setcounter{enumi}{2}
\item
  Chuỗi hài:

  \[
  a_n = \frac{1}{n}.
  \]
\item
  Trình tự luân phiên:

  \[
  a_n = (-1)^n.
  \]
\end{enumerate}

\subsubsection{Dãy số trong phép
tính}\label{duxe3y-sux1ed1-trong-phuxe9p-tuxednh}

Trình tự là nền tảng cho các quá trình vô hạn:

\begin{itemize}
\tightlist
\item
  Giới hạn của dãy → xác định sự hội tụ.
\item
  Chuỗi → tổng vô hạn được xây dựng từ chuỗi.
\item
  Các hàm xấp xỉ theo trình tự và chuỗi.
\end{itemize}

\subsubsection{Tại sao điều này lại quan
trọng}\label{tux1ea1i-sao-ux111iux1ec1u-nuxe0y-lux1ea1i-quan-trux1ecdng-28}

\begin{itemize}
\tightlist
\item
  Chuỗi cung cấp các khối xây dựng cho chuỗi vô hạn và chuỗi xấp xỉ.
\item
  Chúng cho phép chúng ta định nghĩa chặt chẽ việc ``tiến tới vô cực''
  và sự hội tụ.
\item
  Nhiều hàm số quan trọng (lũy thừa, lượng giác) có thể được biểu diễn
  thông qua dãy số và chuỗi.
\end{itemize}

\subsubsection{Bài tập}\label{buxe0i-tux1eadp-47}

\begin{enumerate}
\def\labelenumi{\arabic{enumi}.}
\tightlist
\item
  Viết năm số hạng đầu tiên của dãy \(a_n = \frac{n}{n+1}\).
\item
  Xác định xem \(a_n = (-1)^n n\) có bị chặn hay không.
\item
  Đưa ra định nghĩa đệ quy cho dãy \(2,4,8,16,\dots\).
\item
  Tìm số hạng thứ 10 của dãy số học với \(a_1=3\) và \(d=5\).
\item
  Viết công thức rõ ràng cho dãy được xác định bởi \(a_1=1\),
  \(a_{n+1}=2a_n\).
\end{enumerate}

\subsection{11.2 Trình tự đơn điệu và có giới
hạn}\label{truxecnh-tux1ef1-ux111ux1a1n-ux111iux1ec7u-vuxe0-cuxf3-giux1edbi-hux1ea1n}

Để hiểu liệu một chuỗi có hội tụ hay không, chúng ta cần nghiên cứu hành
vi của nó: nó tăng, giảm, nằm trong giới hạn hay tăng không giới hạn?
Hai khái niệm quan trọng là tính đơn điệu và tính giới hạn.

\subsubsection{Trình tự đơn
điệu}\label{truxecnh-tux1ef1-ux111ux1a1n-ux111iux1ec7u}

Dãy \((a_n)\) được gọi là đơn điệu nếu nó luôn tăng hoặc luôn giảm.

\begin{itemize}
\item
  Đơn điệu tăng dần:

  \[
  a_{n+1} \geq a_n \quad \text{for all } n.
  \]
\item
  Giảm đơn điệu:

  \[
  a_{n+1} \leq a_n \quad \text{for all } n.
  \]
\end{itemize}

Ví dụ:

\begin{enumerate}
\def\labelenumi{\arabic{enumi}.}
\tightlist
\item
  \(a_n = n\) tăng đơn điệu.
\item
  \(a_n = \frac{1}{n}\) đơn điệu giảm dần.
\end{enumerate}

\subsubsection{Chuỗi giới hạn}\label{chuux1ed7i-giux1edbi-hux1ea1n}

Một dãy được giới hạn ở trên nếu tồn tại một số \(M\) sao cho
\(a_n \leq M\) với mọi \(n\). Nó được giới hạn bên dưới nếu tồn tại
\(m\) sao cho \(a_n \geq m\) với mọi \(n\).

Nếu cả hai điều kiện đều đúng thì chuỗi bị chặn.

Ví dụ:

\begin{enumerate}
\def\labelenumi{\arabic{enumi}.}
\tightlist
\item
  \(a_n = \frac{1}{n}\) được giới hạn trong khoảng từ 0 đến 1.
\item
  \(a_n = (-1)^n\) được giới hạn giữa -1 và 1.
\item
  \(a_n = n\) không bị giới hạn.
\end{enumerate}

\subsubsection{Định lý hội tụ đơn
điệu}\label{ux111ux1ecbnh-luxfd-hux1ed9i-tux1ee5-ux111ux1a1n-ux111iux1ec7u}

Một kết quả cơ bản trong phân tích:

\begin{itemize}
\tightlist
\item
  Mọi dãy tăng đơn điệu bị chặn ở trên đều hội tụ.
\item
  Mọi dãy giảm đơn điệu bị chặn dưới đây đều hội tụ.
\end{itemize}

Định lý này đảm bảo sự hội tụ mà không cần tìm giới hạn một cách rõ
ràng.

\subsubsection{Ví dụ}\label{vuxed-dux1ee5-28}

\begin{enumerate}
\def\labelenumi{\arabic{enumi}.}
\tightlist
\item
  Dãy số: \(a_n = 1 - \frac{1}{n}\).
\end{enumerate}

\begin{itemize}
\tightlist
\item
  Tăng: vì \(a_{n+1} - a_n = \frac{1}{n} - \frac{1}{n+1} > 0\).

  \begin{itemize}
  \tightlist
  \item
    Bị chặn ở trên bởi 1.
  \item
    Do đó nó hội tụ.
  \item
    Giới hạn: \(\lim_{n\to\infty} a_n = 1\).
  \end{itemize}
\end{itemize}

\subsubsection{Tại sao điều này lại quan
trọng}\label{tux1ea1i-sao-ux111iux1ec1u-nuxe0y-lux1ea1i-quan-trux1ecdng-29}

\begin{itemize}
\tightlist
\item
  Tính đơn điệu và tính bị chặn cho phép kiểm tra nhanh sự hội tụ.
\item
  Chúng rất cần thiết trong việc chứng minh và xây dựng các giới hạn một
  cách chặt chẽ.
\item
  Những ý tưởng này mở rộng một cách tự nhiên sang các hàm số và chuỗi.
\end{itemize}

\subsubsection{Bài tập}\label{buxe0i-tux1eadp-48}

\begin{enumerate}
\def\labelenumi{\arabic{enumi}.}
\tightlist
\item
  Xác định xem \(a_n = \frac{n}{n+1}\) có đơn điệu và bị chặn hay không.
\item
  Chứng minh \(a_n = \sqrt{n}\) tăng đơn điệu nhưng không bị chặn.
\item
  Chứng minh rằng \(a_n = 2 - \frac{1}{n}\) hội tụ và tìm giới hạn của
  nó.
\item
  Cho ví dụ về dãy giới hạn không đơn điệu.
\item
  Áp dụng định lý hội tụ đơn điệu cho
  \(a_n = \ln\!\big(1+\frac{1}{n}\big)\).
\end{enumerate}

\subsection{11.3 Giới hạn của
dãy}\label{giux1edbi-hux1ea1n-cux1ee7a-duxe3y}

Câu hỏi trọng tâm về một dãy là liệu các số hạng của nó có tiến tới một
giá trị duy nhất khi \(n\) tăng lên hay không. Điều này dẫn đến khái
niệm giới hạn của dãy.

\subsubsection{Sự định nghĩa}\label{sux1ef1-ux111ux1ecbnh-nghux129a-11}

Một dãy \((a_n)\) có một giới hạn \(L\) nếu, với mỗi
\(\varepsilon > 0\), tồn tại một số nguyên \(N\) sao cho

\[
|a_n - L| < \varepsilon \quad \text{whenever } n > N.
\]

Sau đó chúng tôi viết

\[
\lim_{n\to\infty} a_n = L.
\]

Nếu không có \(L\) như vậy tồn tại thì chuỗi sẽ phân kỳ.

\subsubsection{Trực giác}\label{trux1ef1c-giuxe1c}

\begin{itemize}
\tightlist
\item
  Các số hạng của dãy tiến gần tùy ý tới \(L\) khi \(n\) trở nên lớn.
\item
  Ngoài một số chỉ số \(N\), tất cả các số hạng đều nằm trong một biên
  độ nhỏ quanh \(L\).
\end{itemize}

\subsubsection{Ví dụ}\label{vuxed-dux1ee5-29}

\begin{enumerate}
\def\labelenumi{\arabic{enumi}.}
\item
  \(a_n = \frac{1}{n}\). Khi \(n\) tăng lên, số hạng co lại về 0.

  \[
  \lim_{n\to\infty} \frac{1}{n} = 0.
  \]
\item
  \(a_n = (-1)^n\). Các số hạng xen kẽ giữa -1 và 1, do đó không tồn tại
  một giới hạn nào. Trình tự phân kỳ.
\item
  \(a_n = \frac{n}{n+1}\). Vì \(n \to \infty\), tử số và mẫu số gần bằng
  nhau, vì vậy

  \[
  \lim_{n\to\infty} \frac{n}{n+1} = 1.
  \]
\end{enumerate}

\subsubsection{Thuộc tính của giới
hạn}\label{thuux1ed9c-tuxednh-cux1ee7a-giux1edbi-hux1ea1n}

Nếu \(\lim a_n = A\) và \(\lim b_n = B\):

\begin{itemize}
\item
  \(\lim (a_n+b_n) = A+B\).
\item
  \(\lim (a_n b_n) = AB\).
\item
  \(\lim (c a_n) = cA\) cho \(c\) không đổi.
\item
  Nếu \(b_n \neq 0\) và \(B \neq 0\) thì

  \[
  \lim \frac{a_n}{b_n} = \frac{A}{B}.
  \]
\end{itemize}

\subsubsection{Định lý: Nguyên lý
nén}\label{ux111ux1ecbnh-luxfd-nguyuxean-luxfd-nuxe9n}

Nếu \(a_n \leq b_n \leq c_n\) cho tất cả \(n\) lớn, và

\[
\lim_{n\to\infty} a_n = \lim_{n\to\infty} c_n = L,
\]

sau đó

\[
\lim_{n\to\infty} b_n = L.
\]

Ví dụ:

\[
a_n = -\tfrac{1}{n}, \quad b_n = \tfrac{\sin n}{n}, \quad c_n = \tfrac{1}{n}.
\]

Vì \(-\tfrac{1}{n} \leq \tfrac{\sin n}{n} \leq \tfrac{1}{n}\) và cả hai
chuỗi giới hạn đều tiến về 0,

\[
\lim_{n\to\infty} \frac{\sin n}{n} = 0.
\]

\subsubsection{Tại sao điều này lại quan
trọng}\label{tux1ea1i-sao-ux111iux1ec1u-nuxe0y-lux1ea1i-quan-trux1ecdng-30}

\begin{itemize}
\tightlist
\item
  Các giới hạn làm cho ý tưởng về chuỗi ``tiếp cận'' một giá trị trở nên
  khắt khe hơn.
\item
  Sự hội tụ của các chuỗi củng cố chuỗi vô hạn và tính liên tục.
\item
  Những khái niệm này rất cần thiết trong việc định nghĩa số thực thông
  qua giới hạn.
\end{itemize}

\subsubsection{Bài tập}\label{buxe0i-tux1eadp-49}

\begin{enumerate}
\def\labelenumi{\arabic{enumi}.}
\tightlist
\item
  Tìm \(\lim_{n\to\infty} \frac{2n+1}{3n+4}\).
\item
  Xác định xem \(a_n = \sqrt{n+1} - \sqrt{n}\) có hội tụ hay không.
\item
  \(a_n = \cos n\) có hội tụ không? Tại sao hoặc tại sao không?
\item
  Sử dụng Nguyên lý nén để chứng minh
  \(\lim_{n\to\infty} \frac{\sin n}{n} = 0\).
\item
  Chứng minh rằng nếu \(\lim a_n = L\) thì \(\lim |a_n| = |L|\).
\end{enumerate}

\section{Chương 12. Chuỗi vô
tận}\label{chux1b0ux1a1ng-12.-chuux1ed7i-vuxf4-tux1eadn}

\subsection{12.1 Chuỗi và Hội
tụ}\label{chuux1ed7i-vuxe0-hux1ed9i-tux1ee5}

Một chuỗi là tổng các số hạng của một dãy. Thay vì chỉ liệt kê các số,
chúng tôi cộng chúng lại với nhau và nghiên cứu xem tổng vô hạn có tiến
tới giá trị hữu hạn hay không.

\subsubsection{Sự định nghĩa}\label{sux1ef1-ux111ux1ecbnh-nghux129a-12}

Cho một dãy \((a_n)\), chuỗi tương ứng là

\[
\sum_{n=1}^\infty a_n = a_1 + a_2 + a_3 + \dots
\]

Chúng tôi xác định tổng một phần thứ n là

\[
S_n = \sum_{k=1}^n a_k.
\]

Nếu dãy \((S_n)\) hội tụ đến một giới hạn hữu hạn \(S\), thì chuỗi hội
tụ và

\[
\sum_{n=1}^\infty a_n = S.
\]

Nếu \((S_n)\) phân kỳ thì chuỗi phân kỳ.

\subsubsection{Ví dụ}\label{vuxed-dux1ee5-30}

\begin{enumerate}
\def\labelenumi{\arabic{enumi}.}
\tightlist
\item
  Chuỗi hình học
\end{enumerate}

\[
\sum_{n=0}^\infty ar^n = \frac{a}{1-r}, \quad |r| < 1.
\]

Ví dụ:

\[
1 + \tfrac{1}{2} + \tfrac{1}{4} + \tfrac{1}{8} + \dots = 2.
\]

\begin{enumerate}
\def\labelenumi{\arabic{enumi}.}
\setcounter{enumi}{1}
\tightlist
\item
  Chuỗi sóng hài
\end{enumerate}

\[
\sum_{n=1}^\infty \frac{1}{n}.
\]

Chuỗi này phân kỳ, mặc dù các số hạng tiến tới 0.

\begin{enumerate}
\def\labelenumi{\arabic{enumi}.}
\setcounter{enumi}{2}
\tightlist
\item
  dòng p
\end{enumerate}

\[
\sum_{n=1}^\infty \frac{1}{n^p}.
\]

\begin{itemize}
\tightlist
\item
  Hội tụ nếu \(p > 1\).
\item
  Phân kỳ nếu \(p \leq 1\).
\end{itemize}

\subsubsection{Điều kiện cần để hội
tụ}\label{ux111iux1ec1u-kiux1ec7n-cux1ea7n-ux111ux1ec3-hux1ed9i-tux1ee5}

Nếu \(\sum a_n\) hội tụ thì nhất thiết

\[
\lim_{n\to\infty} a_n = 0.
\]

Nếu \(\lim a_n \neq 0\), chuỗi sẽ phân kỳ. Nhưng điều ngược lại không
đúng: \(\lim a_n = 0\) không đảm bảo sự hội tụ (ví dụ: chuỗi hài).

\subsubsection{Tại sao điều này lại quan
trọng}\label{tux1ea1i-sao-ux111iux1ec1u-nuxe0y-lux1ea1i-quan-trux1ecdng-31}

\begin{itemize}
\tightlist
\item
  Chuỗi mở rộng phép cộng hữu hạn cho quá trình vô hạn.
\item
  Chuỗi hội tụ được sử dụng để tính gần đúng các hàm, tính toán vùng và
  mô hình hóa các quá trình vật lý.
\item
  Việc nghiên cứu chuỗi dẫn tới các kiểm định hội tụ mạnh mẽ.
\end{itemize}

\subsubsection{Bài tập}\label{buxe0i-tux1eadp-50}

\begin{enumerate}
\def\labelenumi{\arabic{enumi}.}
\tightlist
\item
  Xác định xem \(\sum_{n=1}^\infty \frac{2}{3^n}\) có hội tụ hay không
  và tìm tổng của nó.
\item
  Chứng minh rằng \(\sum_{n=1}^\infty \frac{1}{n^2}\) hội tụ.
\item
  \(\sum_{n=1}^\infty \frac{1}{\sqrt{n}}\) có hội tụ không?
\item
  Viết bốn tổng riêng đầu tiên của chuỗi
  \(\sum_{n=1}^\infty \frac{1}{2^n}\).
\item
  Giải thích tại sao \(\lim a_n = 0\) là cần thiết nhưng chưa đủ để hội
  tụ.
\end{enumerate}

\subsection{12.2 Kiểm tra hội tụ}\label{kiux1ec3m-tra-hux1ed9i-tux1ee5}

Vì nhiều chuỗi không thể tính tổng một cách trực tiếp nên các nhà toán
học đã phát triển các bài kiểm tra để quyết định xem một chuỗi đó hội tụ
hay phân kỳ. Những bài kiểm tra này là công cụ để phân tích số tiền vô
hạn.

\subsubsection{1. Bài kiểm tra phân kỳ học kỳ thứ
n}\label{buxe0i-kiux1ec3m-tra-phuxe2n-kux1ef3-hux1ecdc-kux1ef3-thux1ee9-n}

Nếu như

\[
\lim_{n\to\infty} a_n \neq 0 \quad \text{or does not exist},
\]

sau đó

\[
\sum a_n
\]

phân kỳ.

Nếu \(\lim a_n = 0\), phép thử không có kết quả.

\subsubsection{2. Kiểm tra so sánh}\label{kiux1ec3m-tra-so-suxe1nh}

Giả sử \(0 \leq a_n \leq b_n\) cho tất cả \(n\).

\begin{itemize}
\tightlist
\item
  Nếu \(\sum b_n\) hội tụ thì \(\sum a_n\) cũng hội tụ.
\item
  Nếu \(\sum a_n\) phân kỳ thì \(\sum b_n\) cũng phân kỳ.
\end{itemize}

\subsubsection{3. Kiểm tra so sánh giới
hạn}\label{kiux1ec3m-tra-so-suxe1nh-giux1edbi-hux1ea1n}

Nếu \(a_n, b_n > 0\) và

\[
\lim_{n\to\infty} \frac{a_n}{b_n} = c,
\]

trong đó \(0 < c < \infty\), thì \(\sum a_n\) và \(\sum b_n\) đều hội tụ
hoặc phân kỳ.

\subsubsection{4. Kiểm tra tỷ lệ}\label{kiux1ec3m-tra-tux1ef7-lux1ec7}

Với \(\sum a_n\), hãy tính

\[
L = \lim_{n\to\infty} \left| \frac{a_{n+1}}{a_n} \right|.
\]

\begin{itemize}
\tightlist
\item
  Nếu \(L < 1\) thì chuỗi hội tụ tuyệt đối.
\item
  Nếu \(L > 1\) hoặc \(L = \infty\) thì chuỗi phân kỳ.
\item
  Nếu \(L = 1\), phép kiểm định không thuyết phục.
\end{itemize}

\subsubsection{5. Kiểm tra gốc}\label{kiux1ec3m-tra-gux1ed1c}

Với \(\sum a_n\), hãy tính

\[
L = \lim_{n\to\infty} \sqrt[n]{|a_n|}.
\]

\begin{itemize}
\tightlist
\item
  Nếu \(L < 1\) thì chuỗi hội tụ tuyệt đối.
\item
  Nếu \(L > 1\) thì chuỗi phân kỳ.
\item
  Nếu \(L = 1\), phép kiểm định không thuyết phục.
\end{itemize}

\subsubsection{6. Bài kiểm tra nối tiếp xen kẽ (Bài kiểm tra của
Leibniz)}\label{buxe0i-kiux1ec3m-tra-nux1ed1i-tiux1ebfp-xen-kux1ebd-buxe0i-kiux1ec3m-tra-cux1ee7a-leibniz}

Đối với loạt mẫu

\[
\sum (-1)^n b_n \quad \text{or} \quad \sum (-1)^{n+1} b_n,
\]

nếu như

\begin{enumerate}
\def\labelenumi{\arabic{enumi}.}
\tightlist
\item
  \(b_{n+1} \leq b_n\) (giảm) và
\item
  \(\lim_{n\to\infty} b_n = 0\),
\end{enumerate}

thì chuỗi hội tụ.

\subsubsection{Ví dụ}\label{vuxed-dux1ee5-31}

\begin{enumerate}
\def\labelenumi{\arabic{enumi}.}
\tightlist
\item
  \(\sum \frac{1}{n^2}\): Kiểm tra so sánh → hội tụ.
\item
  \(\sum \frac{1}{n}\): Chuỗi sóng hài → phân kỳ.
\item
  \(\sum \frac{(-1)^n}{n}\): Kiểm tra chuỗi xen kẽ → hội tụ.
\item
  \(\sum \frac{n!}{n^n}\): Kiểm tra tỷ lệ → hội tụ.
\item
  \(\sum \frac{2^n}{n}\): Kiểm tra gốc → phân kỳ.
\end{enumerate}

\subsubsection{Tại sao điều này lại quan
trọng}\label{tux1ea1i-sao-ux111iux1ec1u-nuxe0y-lux1ea1i-quan-trux1ecdng-32}

\begin{itemize}
\tightlist
\item
  Kiểm định hội tụ cho phép chúng ta phân loại chuỗi mà không cần tính
  tổng rõ ràng.
\item
  Chúng cung cấp những cách có hệ thống để xử lý các quá trình vô hạn
  trong giải tích.
\item
  Chúng rất quan trọng cho các chủ đề sau này như chuỗi lũy thừa và
  chuỗi Fourier.
\end{itemize}

\subsubsection{Bài tập}\label{buxe0i-tux1eadp-51}

\begin{enumerate}
\def\labelenumi{\arabic{enumi}.}
\tightlist
\item
  Kiểm tra sự hội tụ của \(\sum \frac{1}{n^3}\).
\item
  Sử dụng phép kiểm tra tỷ lệ cho \(\sum \frac{3^n}{n!}\).
\item
  Áp dụng kiểm tra gốc cho \(\sum \left(\frac{1}{2}\right)^n\).
\item
  Xác định sự hội tụ của \(\sum \frac{(-1)^n}{\sqrt{n}}\).
\item
  Sử dụng phép kiểm tra so sánh giới hạn với \(\frac{1}{n^2}\) để kiểm
  tra \(\sum \frac{1}{n^2+1}\).
\end{enumerate}

\subsection{12.3 Hội tụ tuyệt đối và có điều
kiện}\label{hux1ed9i-tux1ee5-tuyux1ec7t-ux111ux1ed1i-vuxe0-cuxf3-ux111iux1ec1u-kiux1ec7n}

Không phải tất cả các chuỗi đều hoạt động giống nhau khi các dấu hiệu
thay thế. Để giải quyết vấn đề này, chúng ta phân biệt giữa hội tụ tuyệt
đối và hội tụ có điều kiện.

\subsubsection{Hội tụ tuyệt
đối}\label{hux1ed9i-tux1ee5-tuyux1ec7t-ux111ux1ed1i}

Chuỗi \(\sum a_n\) hội tụ tuyệt đối nếu

\[
\sum |a_n|
\]

hội tụ.

Định lý: Nếu một chuỗi hội tụ tuyệt đối thì nó cũng hội tụ.

Ví dụ:

\[
\sum \frac{(-1)^n}{n^2}.
\]

Ở đây \(\sum \left|\frac{(-1)^n}{n^2}\right| = \sum \frac{1}{n^2}\) hội
tụ (p-series, \(p=2\)). Vậy chuỗi này hội tụ tuyệt đối.

\subsubsection{Hội tụ có điều
kiện}\label{hux1ed9i-tux1ee5-cuxf3-ux111iux1ec1u-kiux1ec7n}

Chuỗi \(\sum a_n\) hội tụ có điều kiện nếu nó hội tụ, nhưng không hội tụ
tuyệt đối.

Ví dụ:

\[
\sum \frac{(-1)^n}{n}.
\]

\begin{itemize}
\tightlist
\item
  Kiểm tra chuỗi xen kẽ → hội tụ.
\item
  Nhưng \(\sum \left|\frac{(-1)^n}{n}\right| = \sum \frac{1}{n}\) phân
  kỳ (chuỗi sóng hài). Vậy chuỗi hội tụ có điều kiện.
\end{itemize}

\subsubsection{Định lý sắp xếp
lại}\label{ux111ux1ecbnh-luxfd-sux1eafp-xux1ebfp-lux1ea1i}

Đối với chuỗi hội tụ có điều kiện, việc sắp xếp lại các số hạng có thể
thay đổi tổng - thậm chí làm cho tổng phân kỳ hoặc hội tụ về một giá trị
khác.

Kết quả đáng ngạc nhiên này cho thấy bản chất mong manh của sự hội tụ có
điều kiện.

\subsubsection{Tại sao điều này lại quan
trọng}\label{tux1ea1i-sao-ux111iux1ec1u-nuxe0y-lux1ea1i-quan-trux1ecdng-33}

\begin{itemize}
\tightlist
\item
  Sự hội tụ tuyệt đối mạnh hơn và đảm bảo sự ổn định.
\item
  Sự hội tụ có điều kiện nêu bật tầm quan trọng của trật tự trong tổng
  vô hạn.
\item
  Nhiều chuỗi xen kẽ gặp trong thực tế chỉ hội tụ có điều kiện.
\end{itemize}

\subsubsection{Bài tập}\label{buxe0i-tux1eadp-52}

\begin{enumerate}
\def\labelenumi{\arabic{enumi}.}
\tightlist
\item
  Chứng minh rằng \(\sum \frac{(-1)^n}{n^3}\) hội tụ tuyệt đối.
\item
  Chứng minh rằng \(\sum \frac{(-1)^n}{n}\) hội tụ có điều kiện.
\item
  Kiểm tra \(\sum \frac{(-1)^n}{\sqrt{n}}\) về sự hội tụ tuyệt đối và có
  điều kiện.
\item
  Giải thích tại sao sự hội tụ tuyệt đối kéo theo sự hội tụ, nhưng điều
  ngược lại là không đúng.
\item
  Nghiên cứu và tóm tắt định lý sắp xếp lại Riemann bằng lời nói của
  bạn.
\end{enumerate}

\section{Chương 13. Chuỗi sức mạnh và mở
rộng}\label{chux1b0ux1a1ng-13.-chuux1ed7i-sux1ee9c-mux1ea1nh-vuxe0-mux1edf-rux1ed9ng}

\subsection{Dòng điện 13.1}\label{duxf2ng-ux111iux1ec7n-13.1}

Chuỗi lũy thừa là chuỗi vô hạn trong đó mỗi số hạng liên quan đến lũy
thừa của biến. Chuỗi lũy thừa là trung tâm của phép tính vì chúng cho
phép chúng ta biểu diễn các hàm số dưới dạng đa thức vô hạn.

\subsubsection{Mẫu chung}\label{mux1eabu-chung}

Chuỗi lũy thừa có tâm tại \(a\) có dạng

\[
\sum_{n=0}^\infty c_n (x-a)^n,
\]

trong đó \(c_n\) là các hằng số được gọi là hệ số.

\begin{itemize}
\item
  Nếu \(a=0\) thì chuỗi có tâm tại gốc tọa độ:

  \[
  \sum_{n=0}^\infty c_n x^n.
  \]
\end{itemize}

\subsubsection{Ví dụ}\label{vuxed-dux1ee5-32}

\begin{enumerate}
\def\labelenumi{\arabic{enumi}.}
\tightlist
\item
  Chuỗi hình học
\end{enumerate}

\[
\sum_{n=0}^\infty x^n = \frac{1}{1-x}, \quad |x|<1.
\]

\begin{enumerate}
\def\labelenumi{\arabic{enumi}.}
\setcounter{enumi}{1}
\tightlist
\item
  Hàm mũ
\end{enumerate}

\[
e^x = \sum_{n=0}^\infty \frac{x^n}{n!}.
\]

\begin{enumerate}
\def\labelenumi{\arabic{enumi}.}
\setcounter{enumi}{2}
\tightlist
\item
  Sin và cosin
\end{enumerate}

\[
\sin x = \sum_{n=0}^\infty (-1)^n \frac{x^{2n+1}}{(2n+1)!}, \quad  
\cos x = \sum_{n=0}^\infty (-1)^n \frac{x^{2n}}{(2n)!}.
\]

\subsubsection{Khoảng hội tụ}\label{khoux1ea3ng-hux1ed9i-tux1ee5}

Với mỗi chuỗi lũy thừa tồn tại bán kính hội tụ \(R\) sao cho:

\begin{itemize}
\tightlist
\item
  Chuỗi hội tụ nếu \(|x-a| < R\).
\item
  Chuỗi phân kỳ nếu \(|x-a| > R\).
\item
  Tại \(|x-a| = R\), sự hội tụ phải được kiểm tra riêng.
\end{itemize}

\subsubsection{Tại sao điều này lại quan
trọng}\label{tux1ea1i-sao-ux111iux1ec1u-nuxe0y-lux1ea1i-quan-trux1ecdng-34}

\begin{itemize}
\tightlist
\item
  Chuỗi lũy thừa cho phép ta tính gần đúng hàm số bằng đa thức.
\item
  Họ kết nối phép tính với phân tích và phương trình vi phân.
\item
  Nhiều hàm đặc biệt trong toán học và vật lý được xác định bằng chuỗi
  lũy thừa.
\end{itemize}

\subsubsection{Bài tập}\label{buxe0i-tux1eadp-53}

\begin{enumerate}
\def\labelenumi{\arabic{enumi}.}
\tightlist
\item
  Viết chuỗi lũy thừa của \(\sum_{n=0}^\infty \frac{(x-2)^n}{n!}\).
\item
  Tìm bốn số hạng đầu tiên của chuỗi lũy thừa của \(e^x\).
\item
  Biểu thị \(\frac{1}{1+x}\) dưới dạng chuỗi lũy thừa có tâm tại 0.
\item
  Xác định xem chuỗi \(\sum_{n=0}^\infty n! x^n\) hội tụ tại \(x=0,1\).
\item
  Giải thích tại sao chuỗi lũy thừa đôi khi được gọi là ``đa thức vô
  hạn''.
\end{enumerate}

\subsection{13,2 Bán kính hội tụ}\label{buxe1n-kuxednh-hux1ed9i-tux1ee5}

Mọi chuỗi lũy thừa đều hội tụ đối với một số giá trị của \(x\) và phân
kỳ đối với các giá trị khác. Ranh giới giữa hai hành vi này được mô tả
bằng bán kính hội tụ.

\subsubsection{Sự định nghĩa}\label{sux1ef1-ux111ux1ecbnh-nghux129a-13}

Đối với chuỗi điện

\[
\sum_{n=0}^\infty c_n (x-a)^n,
\]

tồn tại một số \(R \geq 0\) (có thể là vô hạn) sao cho:

\begin{itemize}
\tightlist
\item
  Chuỗi hội tụ tuyệt đối nếu \(|x-a| < R\).
\item
  Chuỗi phân kỳ nếu \(|x-a| > R\).
\item
  Tại \(|x-a| = R\), sự hội tụ phải được kiểm tra riêng.
\end{itemize}

Số \(R\) này được gọi là bán kính hội tụ.

\subsubsection{Tìm bán kính hội
tụ}\label{tuxecm-buxe1n-kuxednh-hux1ed9i-tux1ee5}

Hai phương pháp phổ biến:

\begin{enumerate}
\def\labelenumi{\arabic{enumi}.}
\tightlist
\item
  Kiểm tra tỷ lệ
\end{enumerate}

\[
R = \lim_{n\to\infty} \left| \frac{c_n}{c_{n+1}} \right|.
\]

\begin{enumerate}
\def\labelenumi{\arabic{enumi}.}
\setcounter{enumi}{1}
\tightlist
\item
  Kiểm tra gốc
\end{enumerate}

\[
R = \frac{1}{\limsup_{n\to\infty} \sqrt[n]{|c_n|}}.
\]

\subsubsection{Ví dụ}\label{vuxed-dux1ee5-33}

\begin{enumerate}
\def\labelenumi{\arabic{enumi}.}
\tightlist
\item
  Dòng sản phẩm:
\end{enumerate}

\[
\sum_{n=0}^\infty \frac{x^n}{n!}.
\]

Sử dụng kiểm tra tỷ lệ:

\[
\lim_{n\to\infty} \frac{1/(n!)}{1/((n+1)!)} = \infty.
\]

Vậy \(R = \infty\) (hội tụ với mọi \(x\) thực).

\begin{enumerate}
\def\labelenumi{\arabic{enumi}.}
\setcounter{enumi}{1}
\tightlist
\item
  Chuỗi:
\end{enumerate}

\[
\sum_{n=0}^\infty x^n.
\]

Ở đây \(c_n = 1\).

\[
R = 1.
\]

Hội tụ cho \(|x| < 1\).

\begin{enumerate}
\def\labelenumi{\arabic{enumi}.}
\setcounter{enumi}{2}
\tightlist
\item
  Dòng sản phẩm:
\end{enumerate}

\[
\sum_{n=1}^\infty \frac{x^n}{n}.
\]

Kiểm tra tỷ lệ:

\[
\lim_{n\to\infty} \left|\frac{(x^{n+1}/(n+1))}{(x^n/n)}\right| = |x|.
\]

Vậy \(R = 1\). Hội tụ cho \(|x| < 1\), phân kỳ cho
\$\textbar x\textbar{} \textgreater{} 1 đô la. Tại \(x=\pm 1\), hãy kiểm
tra riêng.

\subsubsection{Khoảng hội tụ}\label{khoux1ea3ng-hux1ed9i-tux1ee5-1}

Tập hợp các giá trị \(x\) trong đó chuỗi hội tụ được gọi là khoảng hội
tụ.

\begin{itemize}
\tightlist
\item
  Luôn tập trung tại \(a\).
\item
  Mở rộng đơn vị \(R\) theo cả hai hướng.
\item
  Điểm cuối \(x=a\pm R\) phải được kiểm tra riêng lẻ.
\end{itemize}

\subsubsection{Tại sao điều này lại quan
trọng}\label{tux1ea1i-sao-ux111iux1ec1u-nuxe0y-lux1ea1i-quan-trux1ecdng-35}

\begin{itemize}
\tightlist
\item
  Bán kính hội tụ cho chúng ta biết chuỗi lũy thừa hoạt động giống như
  hàm số ở đâu.
\item
  Cần thiết cho việc sử dụng khai triển chuỗi Taylor trong thực tế.
\item
  Xác định phạm vi nghiệm của dãy nghiệm trong vật lý và kỹ thuật.
\end{itemize}

\subsubsection{Bài tập}\label{buxe0i-tux1eadp-54}

\begin{enumerate}
\def\labelenumi{\arabic{enumi}.}
\tightlist
\item
  Tìm bán kính hội tụ của \(\sum_{n=0}^\infty \frac{(x-3)^n}{n!}\).
\item
  Tính bán kính hội tụ của \(\sum_{n=1}^\infty \frac{x^n}{n^2}\).
\item
  Sử dụng phép kiểm tra tỷ lệ để tìm \(R\) cho
  \(\sum_{n=0}^\infty n!x^n\).
\item
  Xác định khoảng hội tụ của \(\sum_{n=1}^\infty \frac{(x+1)^n}{n}\).
\item
  Giải thích tại sao chuỗi số mũ hội tụ với mọi \(x\), trong khi chuỗi
  hình học chỉ hội tụ với \(|x|<1\).
\end{enumerate}

\subsection{13.3 Dòng Taylor và
Maclaurin}\label{duxf2ng-taylor-vuxe0-maclaurin}

Chuỗi lũy thừa trở nên đặc biệt mạnh mẽ khi chúng được sử dụng để biểu
diễn các hàm quen thuộc. Điều này được thực hiện thông qua chuỗi Taylor
và trường hợp đặc biệt có tâm tại 0 được gọi là chuỗi Maclaurin.

\#\#\#Dòng Taylor

Nếu một hàm \(f(x)\) khả vi vô hạn tại \(x=a\) thì chuỗi Taylor của nó
về \(a\) là

\[
f(x) = \sum_{n=0}^\infty \frac{f^{(n)}(a)}{n!}(x-a)^n.
\]

Ở đây \(f^{(n)}(a)\) biểu thị đạo hàm thứ \(n\) của \(f\) tại \(a\).

\subsubsection{Dòng Maclaurin}\label{duxf2ng-maclaurin}

Chuỗi Taylor có tâm tại \(a=0\):

\[
f(x) = \sum_{n=0}^\infty \frac{f^{(n)}(0)}{n!} x^n.
\]

\subsubsection{Ví dụ}\label{vuxed-dux1ee5-34}

\begin{enumerate}
\def\labelenumi{\arabic{enumi}.}
\tightlist
\item
  Hàm mũ
\end{enumerate}

\[
e^x = 1 + x + \frac{x^2}{2!} + \frac{x^3}{3!} + \cdots
\]

\begin{enumerate}
\def\labelenumi{\arabic{enumi}.}
\setcounter{enumi}{1}
\tightlist
\item
  Sin và cosin
\end{enumerate}

\[
\sin x = x - \frac{x^3}{3!} + \frac{x^5}{5!} - \cdots
\]

\[
\cos x = 1 - \frac{x^2}{2!} + \frac{x^4}{4!} - \cdots
\]

\begin{enumerate}
\def\labelenumi{\arabic{enumi}.}
\setcounter{enumi}{2}
\tightlist
\item
  Logarit tự nhiên (với \(|x|<1\))
\end{enumerate}

\[
\ln(1+x) = x - \frac{x^2}{2} + \frac{x^3}{3} - \frac{x^4}{4} + \cdots
\]

\subsubsection{Xấp xỉ đa thức
Taylor}\label{xux1ea5p-xux1ec9-ux111a-thux1ee9c-taylor}

Tổng hữu hạn của các số hạng \(n\) đầu tiên là đa thức Taylor bậc \(n\):

\[
P_n(x) = \sum_{k=0}^n \frac{f^{(k)}(a)}{k!}(x-a)^k.
\]

Đa thức này xấp xỉ \(f(x)\) gần \(x=a\).

\subsubsection{Phần dư (Thuật ngữ
Lỗi)}\label{phux1ea7n-dux1b0-thuux1eadt-ngux1eef-lux1ed7i}

Hiệu giữa hàm số và đa thức Taylor của nó là phần dư:

\[
R_n(x) = f(x) - P_n(x).
\]

Một dạng (dạng Lagrange) là

\[
R_n(x) = \frac{f^{(n+1)}(c)}{(n+1)!}(x-a)^{n+1},
\]

đối với một số \(c\) nằm giữa \(a\) và \(x\).

\subsubsection{Tại sao điều này lại quan
trọng}\label{tux1ea1i-sao-ux111iux1ec1u-nuxe0y-lux1ea1i-quan-trux1ecdng-36}

\begin{itemize}
\tightlist
\item
  Chuỗi Taylor cung cấp các phép tính gần đúng đa thức cho các hàm phức
  tạp.
\item
  Chúng rất cần thiết trong phân tích số, vật lý và kỹ thuật.
\item
  Khai triển chuỗi Maclaurin đưa ra các công thức đơn giản cho các hàm
  số mũ, lượng giác và logarit.
\end{itemize}

\subsubsection{Bài tập}\label{buxe0i-tux1eadp-55}

\begin{enumerate}
\def\labelenumi{\arabic{enumi}.}
\tightlist
\item
  Tìm chuỗi Maclaurin của \(f(x)=\cosh x = \tfrac{e^x+e^{-x}}{2}\).
\item
  Viết chuỗi Taylor cho \(f(x)=e^x\) có tâm tại \(a=2\).
\item
  Tính đa thức Taylor bậc 3 cho \(f(x)=\ln(1+x)\) tại \(a=0\).
\item
  Sử dụng chuỗi Maclaurin cho \(\sin x\) để xấp xỉ \(\sin(0,1)\).
\item
  Giải thích tại sao chuỗi Taylor thường cho kết quả gần đúng cục bộ tốt
  nhưng có thể phân kỳ đối với \(|x|\) lớn.
\end{enumerate}

\#\#13.4 Ứng dụng của dòng Taylor

Chuỗi Taylor không chỉ là công cụ lý thuyết - chúng còn được sử dụng để
tính gần đúng các hàm, giải phương trình và phân tích các hệ vật lý. Ứng
dụng của họ bao gồm toán học, khoa học và kỹ thuật.

\subsubsection{Xấp xỉ hàm}\label{xux1ea5p-xux1ec9-huxe0m}

Các hàm phức tạp có thể được tính gần đúng bằng các đa thức gần một
điểm.

Ví dụ: Ước tính \(e^x\) gần \(x=0\) bằng cách sử dụng đa thức Maclaurin
bậc 3:

\[
P_3(x) = 1 + x + \tfrac{x^2}{2} + \tfrac{x^3}{6}.
\]

Đối với \(x\) nhỏ, điều này đưa ra ước tính chính xác về \(e^x\).

\subsubsection{Phương pháp số}\label{phux1b0ux1a1ng-phuxe1p-sux1ed1}

Chuỗi Taylor cung cấp cơ sở cho các thuật toán số:

\begin{itemize}
\tightlist
\item
  Xấp xỉ căn bậc hai, logarit và giá trị lượng giác.
\item
  Ước lượng sai số qua số hạng còn lại.
\item
  Được sử dụng trong các phương pháp lặp như phương pháp Newton (trong
  đó tuyến tính hóa cục bộ lấy từ chuỗi Taylor).
\end{itemize}

\subsubsection{Giải phương trình vi
phân}\label{giux1ea3i-phux1b0ux1a1ng-truxecnh-vi-phuxe2n}

Nhiều phương trình vi phân có nghiệm được biểu thị dưới dạng chuỗi
Taylor (hoặc lũy thừa).

Ví dụ: Nghiệm của \(y'' + y = 0\) với \(y(0)=0, y'(0)=1\) là \(\sin x\),
xuất hiện một cách tự nhiên từ chuỗi Maclaurin của nó.

\subsubsection{Vật lý và Kỹ
thuật}\label{vux1eadt-luxfd-vuxe0-kux1ef9-thuux1eadt}

\begin{itemize}
\item
  Xấp xỉ góc nhỏ:

  \[
  \sin x \approx x, \quad \cos x \approx 1 - \tfrac{x^2}{2}, \quad |x| \ll 1.
  \]
\end{itemize}

Được sử dụng trong chuyển động con lắc, quang học và cơ học sóng.

\begin{itemize}
\item
  Thuyết tương đối và cơ học lượng tử: Khai triển Taylor đơn giản hóa
  các biểu thức phi tuyến để sử dụng trong thực tế.
\item
  Xấp xỉ hàm năng lượng: Trong cơ học, hàm thế năng được khai triển gần
  điểm cân bằng.
\end{itemize}

\subsubsection{Xác suất và Thống
kê}\label{xuxe1c-suux1ea5t-vuxe0-thux1ed1ng-kuxea}

\begin{itemize}
\tightlist
\item
  Hàm sinh mômen và hàm đặc tính sử dụng chuỗi lũy thừa.
\item
  Các phép tính gần đúng của phân bố xác suất (ví dụ: xấp xỉ chuẩn với
  nhị thức) sử dụng khai triển Taylor.
\end{itemize}

\subsubsection{Tại sao điều này lại quan
trọng}\label{tux1ea1i-sao-ux111iux1ec1u-nuxe0y-lux1ea1i-quan-trux1ecdng-37}

\begin{itemize}
\tightlist
\item
  Chuỗi Taylor cung cấp cầu nối giữa các công thức chính xác và tính
  toán thực tế.
\item
  Chúng cho phép chúng ta giảm các vấn đề phức tạp thành các xấp xỉ đa
  thức có thể quản lý được.
\item
  Các ứng dụng làm cho chúng trở thành một trong những công cụ quan
  trọng nhất trong toán học ứng dụng.
\end{itemize}

\subsubsection{Bài tập}\label{buxe0i-tux1eadp-56}

\begin{enumerate}
\def\labelenumi{\arabic{enumi}.}
\tightlist
\item
  Sử dụng chuỗi Maclaurin cho \(e^x\) để tính gần đúng \(e^{0.1}\) tới
  bốn chữ số thập phân.
\item
  Áp dụng phép tính gần đúng góc nhỏ để ước tính \(\sin(5^\circ)\).
\item
  Giải phương trình vi phân \(y'' = -y\) bằng cách sử dụng phương pháp
  chuỗi lũy thừa.
\item
  Khai triển \(\ln(1+x)\) lên đến bậc 4 và sử dụng nó để tính gần đúng
  \(\ln(1.1)\).
\item
  Giải thích tại sao phép tính gần đúng đa thức đặc biệt hữu ích cho máy
  tính và máy tính bỏ túi.
\end{enumerate}

\#Phụ lục

\subsection{Phụ lục A. Những kiến \hspace{0pt}\hspace{0pt}thức cơ bản
trước khi tính
toán}\label{phux1ee5-lux1ee5c-a.-nhux1eefng-kiux1ebfn-thux1ee9c-cux1a1-bux1ea3n-trux1b0ux1edbc-khi-tuxednh-touxe1n}

\subsubsection{Ôn lại đại số
A.1}\label{uxf4n-lux1ea1i-ux111ux1ea1i-sux1ed1-a.1}

Trước khi đi sâu vào giải tích, việc ôn lại một số kỹ năng đại số sẽ
xuất hiện lặp đi lặp lại sẽ rất hữu ích. Đây là những ``công cụ'' bạn sẽ
cần để thao tác với biểu thức, giải phương trình và đơn giản hóa kết
quả.

\paragraph{Số mũ và lũy
thừa}\label{sux1ed1-mux169-vuxe0-lux169y-thux1eeba}

\begin{itemize}
\item
  Các quy tắc cơ bản:

  \[
  a^m \cdot a^n = a^{m+n}, \quad \frac{a^m}{a^n} = a^{m-n}, \quad (a^m)^n = a^{mn}.
  \]
\item
  Negative exponents:

  \[
  a^{-n} = \frac{1}{a^n}, \quad a \neq 0.
  \]
\item
  Fractional exponents:

  \[
  a^{1/n} = \sqrt[n]{a}, \quad a^{m/n} = \sqrt[n]{a^m}.
  \]
\end{itemize}

\paragraph{Bao thanh toán}\label{bao-thanh-touxe1n}

Phân tích nhân tử đơn giản hóa các biểu thức và giúp giải các phương
trình.

\begin{enumerate}
\def\labelenumi{\arabic{enumi}.}
\item
  Yếu tố chung:

  \[
  6x^2+9x = 3x(2x+3).
  \]
\item
  Sự khác biệt của hình vuông:

  \[
  a^2-b^2 = (a-b)(a+b).
  \]
\item
  Tam thức bậc hai:

  \[
  x^2+5x+6 = (x+2)(x+3).
  \]
\end{enumerate}

\paragraph{Đa thức}\label{ux111a-thux1ee9c}

\begin{itemize}
\tightlist
\item
  Dạng chuẩn: \(P(x) = a_nx^n + a_{n-1}x^{n-1} + \cdots + a_0\).
\item
  Bậc: lũy thừa lớn nhất của \(x\).
\item
  Phép chia dài và phép chia tổng hợp rất hữu ích trong việc đơn giản
  hóa hàm số hữu tỉ.
\end{itemize}

\paragraph{Biểu thức hữu tỉ}\label{biux1ec3u-thux1ee9c-hux1eefu-tux1ec9}

Rút gọn bằng cách phân tích nhân tử và mẫu số:

\[
\frac{x^2-1}{x^2-2x+1} = \frac{(x-1)(x+1)}{(x-1)^2} = \frac{x+1}{x-1}, \quad x \neq 1.
\]

\paragraph{Logarit}\label{logarit}

\begin{itemize}
\item
  Định nghĩa: \(\log_a b = c\) có nghĩa là \(a^c = b\).
\item
  Các cơ số chung: log tự nhiên (\(\ln x = \log_e x\)) và cơ số 10
  (\(\log x\)).
\item
  Quy tắc:

  \[
  \log(ab) = \log a + \log b, \quad \log\left(\frac{a}{b}\right) = \log a - \log b, \quad \log(a^n) = n\log a.
  \]
\end{itemize}

\paragraph{Phương trình}\label{phux1b0ux1a1ng-truxecnh}

\begin{itemize}
\item
  Tuyến tính: giải \(ax+b=0\) → \(x=-b/a\).
\item
  Bậc hai: \(ax^2+bx+c=0\) có nghiệm

  \[
  x=\frac{-b\pm \sqrt{b^2-4ac}}{2a}.
  \]
\item
  Exponential: \(e^x = k\) → \(x = \ln k\).
\end{itemize}

\subsubsection{A.2 Cơ bản về lượng
giác}\label{a.2-cux1a1-bux1ea3n-vux1ec1-lux1b0ux1ee3ng-giuxe1c}

Lượng giác cung cấp ngôn ngữ của các góc và hiện tượng tuần hoàn. Vì
phép tính thường xử lý các dao động, chuyển động và sóng nên việc nắm
vững các hàm lượng giác và các tính chất của chúng là điều cần thiết.

\paragraph{Vòng tròn đơn vị}\label{vuxf2ng-truxf2n-ux111ux1a1n-vux1ecb}

\begin{itemize}
\item
  Được xác định là đường tròn bán kính 1 có tâm tại gốc tọa độ.
\item
  Đối với góc \(\theta\) đo từ trục \(x\) dương:

  \[
  (\cos \theta, \sin \theta)
  \]
\end{itemize}

cho biết tọa độ của điểm trên đường tròn.

Giá trị đặc biệt:

\begin{longtable}[]{@{}
  >{\raggedright\arraybackslash}p{(\linewidth - 6\tabcolsep) * \real{0.3333}}
  >{\raggedright\arraybackslash}p{(\linewidth - 6\tabcolsep) * \real{0.1667}}
  >{\raggedright\arraybackslash}p{(\linewidth - 6\tabcolsep) * \real{0.1667}}
  >{\raggedright\arraybackslash}p{(\linewidth - 6\tabcolsep) * \real{0.3333}}@{}}
\toprule\noalign{}
\begin{minipage}[b]{\linewidth}\raggedright
\(\theta\)
\end{minipage} & \begin{minipage}[b]{\linewidth}\raggedright
\(\sin \theta\)
\end{minipage} & \begin{minipage}[b]{\linewidth}\raggedright
\(\cos \theta\)
\end{minipage} & \begin{minipage}[b]{\linewidth}\raggedright
\(\tan \theta = \frac{\sin \theta}{\cos \theta}\)
\end{minipage} \\
\midrule\noalign{}
\endhead
\bottomrule\noalign{}
\endlastfoot
\(0\) & 0 & 1 & 0 \\
\(\pi/6\) & 1/2 & \(\sqrt{3}/2\) & \(1/\sqrt{3}\) \\
\(\pi/4\) & \(\sqrt{2}/2\) & \(\sqrt{2}/2\) & 1 \\
\(\pi/3\) & \(\sqrt{3}/2\) & 1/2 & \(\sqrt{3}\) \\
\(\pi/2\) & 1 & 0 & không xác định \\
\end{longtable}

\paragraph{Danh tính cơ bản}\label{danh-tuxednh-cux1a1-bux1ea3n}

\begin{enumerate}
\def\labelenumi{\arabic{enumi}.}
\tightlist
\item
  Nhận dạng Pythagore
\end{enumerate}

\[
\sin^2\theta + \cos^2\theta = 1.
\]

\begin{enumerate}
\def\labelenumi{\arabic{enumi}.}
\setcounter{enumi}{1}
\tightlist
\item
  Danh tính thương
\end{enumerate}

\[
\tan\theta = \frac{\sin\theta}{\cos\theta}, \quad \cot\theta = \frac{\cos\theta}{\sin\theta}.
\]

\begin{enumerate}
\def\labelenumi{\arabic{enumi}.}
\setcounter{enumi}{2}
\tightlist
\item
  Bản sắc tương hỗ
\end{enumerate}

\[
\sec\theta = \frac{1}{\cos\theta}, \quad \csc\theta = \frac{1}{\sin\theta}.
\]

\paragraph{Công thức cộng góc}\label{cuxf4ng-thux1ee9c-cux1ed9ng-guxf3c}

\[
\sin(\alpha+\beta) = \sin\alpha\cos\beta + \cos\alpha\sin\beta,
\]

\[
\cos(\alpha+\beta) = \cos\alpha\cos\beta - \sin\alpha\sin\beta.
\]

Các trường hợp đặc biệt:

\begin{itemize}
\item
  Góc đôi:

  \[
  \sin(2\theta) = 2\sin\theta\cos\theta, \quad
  \cos(2\theta) = \cos^2\theta - \sin^2\theta.
  \]
\end{itemize}

\paragraph{Đồ thị}\label{ux111ux1ed3-thux1ecb}

\begin{itemize}
\tightlist
\item
  \(\sin x\): sóng bắt đầu từ 0, biên độ 1, chu kỳ \(2\pi\).
\item
  \(\cos x\): sóng bắt đầu tại 1, biên độ 1, chu kỳ \(2\pi\).
\item
  \(\tan x\): lặp lại mọi \(\pi\), không xác định ở bội số lẻ của
  \(\pi/2\).
\end{itemize}

\subsubsection{A.3 Hình học tọa
độ}\label{a.3-huxecnh-hux1ecdc-tux1ecda-ux111ux1ed9}

Phối hợp hình học liên kết đại số và hình học bằng cách mô tả các đối
tượng hình học (đường thẳng, hình tròn, đường cong) bằng các phương
trình. Giải tích phụ thuộc rất nhiều vào khuôn khổ này để vẽ đồ thị các
hàm số, tìm hệ số góc và phân tích các đường cong.

\paragraph{Mặt phẳng Descartes}\label{mux1eb7t-phux1eb3ng-descartes}

\begin{itemize}
\item
  Một điểm được biểu diễn bằng tọa độ \((x,y)\).
\item
  Khoảng cách giữa hai điểm \((x_1,y_1)\) và \((x_2,y_2)\):

  \[
  d = \sqrt{(x_2-x_1)^2 + (y_2-y_1)^2}.
  \]
\item
  Midpoint of a line segment:

  \[
  M = \left(\frac{x_1+x_2}{2}, \frac{y_1+y_2}{2}\right).
  \]
\end{itemize}

\paragraph{Dòng}\label{duxf2ng}

\begin{enumerate}
\def\labelenumi{\arabic{enumi}.}
\item
  Công thức độ dốc

  \[
  m = \frac{y_2-y_1}{x_2-x_1}.
  \]
\item
  Phương trình đường thẳng
\end{enumerate}

\begin{itemize}
\item
  Dạng độ dốc điểm:

  \[
     y-y_1 = m(x-x_1).
     \]

  \begin{itemize}
  \item
    Slope-intercept form:

    \[
    y = mx+b.
    \]
  \end{itemize}
\end{itemize}

\begin{enumerate}
\def\labelenumi{\arabic{enumi}.}
\setcounter{enumi}{2}
\tightlist
\item
  Đường song song và vuông góc
\end{enumerate}

\begin{itemize}
\tightlist
\item
  Các đường song song: cùng độ dốc.

  \begin{itemize}
  \tightlist
  \item
    Đường vuông góc: hệ số góc thỏa mãn \(m_1m_2 = -1\).
  \end{itemize}
\end{itemize}

\paragraph{Vòng kết nối}\label{vuxf2ng-kux1ebft-nux1ed1i}

Phương trình đường tròn có tâm \((h,k)\) và bán kính \(r\):

\[
(x-h)^2+(y-k)^2 = r^2.
\]

Trường hợp đặc biệt: đường tròn đơn vị có tâm tại gốc tọa độ:

\[
x^2+y^2=1.
\]

\paragraph{Đoạn hình nón}\label{ux111oux1ea1n-huxecnh-nuxf3n}

\begin{enumerate}
\def\labelenumi{\arabic{enumi}.}
\tightlist
\item
  Parabol:
\end{enumerate}

\begin{itemize}
\item
  Dạng chuẩn (mở lên/xuống):

  \[
     y = ax^2+bx+c.
     \]
\end{itemize}

\begin{enumerate}
\def\labelenumi{\arabic{enumi}.}
\setcounter{enumi}{1}
\item
  Hình elip (tập trung vào gốc):

  \[
  \frac{x^2}{a^2}+\frac{y^2}{b^2}=1.
  \]
\item
  Hyperbol (tập trung tại gốc tọa độ):

  \[
  \frac{x^2}{a^2}-\frac{y^2}{b^2}=1.
  \]
\end{enumerate}

\subsection{Phụ lục B. Các công thức và bảng
chính}\label{phux1ee5-lux1ee5c-b.-cuxe1c-cuxf4ng-thux1ee9c-vuxe0-bux1ea3ng-chuxednh}

\subsubsection{B.1 Bảng đạo
hàm}\label{b.1-bux1ea3ng-ux111ux1ea1o-huxe0m}

Đạo hàm đo lường tốc độ thay đổi và độ dốc của hàm số. Việc có bảng tra
cứu nhanh giúp người học tránh phải học lại công thức mỗi lần học.

\paragraph{Quy tắc cơ bản}\label{quy-tux1eafc-cux1a1-bux1ea3n-1}

\begin{enumerate}
\def\labelenumi{\arabic{enumi}.}
\tightlist
\item
  Quy tắc bất biến
\end{enumerate}

\[
\frac{d}{dx}[c] = 0
\]

\begin{enumerate}
\def\labelenumi{\arabic{enumi}.}
\setcounter{enumi}{1}
\tightlist
\item
  Quy tắc quyền lực
\end{enumerate}

\[
\frac{d}{dx}[x^n] = nx^{n-1}, \quad (n \in \mathbb{R})
\]

\begin{enumerate}
\def\labelenumi{\arabic{enumi}.}
\setcounter{enumi}{2}
\tightlist
\item
  Quy tắc bội số không đổi
\end{enumerate}

\[
\frac{d}{dx}[c f(x)] = c f'(x)
\]

\begin{enumerate}
\def\labelenumi{\arabic{enumi}.}
\setcounter{enumi}{3}
\tightlist
\item
  Quy tắc tính tổng và hiệu
\end{enumerate}

\[
\frac{d}{dx}[f(x)\pm g(x)] = f'(x)\pm g'(x)
\]

\paragraph{Hàm lượng giác}\label{huxe0m-lux1b0ux1ee3ng-giuxe1c}

\[
\frac{d}{dx}[\sin x] = \cos x
\]

\[
\frac{d}{dx}[\cos x] = -\sin x
\]

\[
\frac{d}{dx}[\tan x] = \sec^2 x, \quad x \neq \tfrac{\pi}{2}+k\pi
\]

\[
\frac{d}{dx}[\cot x] = -\csc^2 x
\]

\[
\frac{d}{dx}[\sec x] = \sec x \tan x
\]

\[
\frac{d}{dx}[\csc x] = -\csc x \cot x
\]

\paragraph{Hàm số mũ và hàm
logarit}\label{huxe0m-sux1ed1-mux169-vuxe0-huxe0m-logarit}

\[
\frac{d}{dx}[e^x] = e^x
\]

\[
\frac{d}{dx}[a^x] = a^x \ln a, \quad a>0, a\neq 1
\]

\[
\frac{d}{dx}[\ln x] = \frac{1}{x}, \quad x>0
\]

\[
\frac{d}{dx}[\log_a x] = \frac{1}{x\ln a}, \quad a>0, a\neq 1
\]

\paragraph{Hàm lượng giác nghịch
đảo}\label{huxe0m-lux1b0ux1ee3ng-giuxe1c-nghux1ecbch-ux111ux1ea3o}

\[
\frac{d}{dx}[\arcsin x] = \frac{1}{\sqrt{1-x^2}}, \quad |x|<1
\]

\[
\frac{d}{dx}[\arccos x] = -\frac{1}{\sqrt{1-x^2}}, \quad |x|<1
\]

\[
\frac{d}{dx}[\arctan x] = \frac{1}{1+x^2}, \quad x \in \mathbb{R}
\]

\paragraph{Quy tắc sản phẩm, thương số và
chuỗi}\label{quy-tux1eafc-sux1ea3n-phux1ea9m-thux1b0ux1a1ng-sux1ed1-vuxe0-chuux1ed7i}

\begin{enumerate}
\def\labelenumi{\arabic{enumi}.}
\tightlist
\item
  Quy tắc sản phẩm
\end{enumerate}

\[
\frac{d}{dx}[f(x)g(x)] = f'(x)g(x)+f(x)g'(x)
\]

\begin{enumerate}
\def\labelenumi{\arabic{enumi}.}
\setcounter{enumi}{1}
\tightlist
\item
  Quy tắc thương số
\end{enumerate}

\[
\frac{d}{dx}\left[\frac{f(x)}{g(x)}\right] = \frac{f'(x)g(x)-f(x)g'(x)}{[g(x)]^2}, \quad g(x)\neq 0
\]

\begin{enumerate}
\def\labelenumi{\arabic{enumi}.}
\setcounter{enumi}{2}
\tightlist
\item
  Quy tắc dây chuyền
\end{enumerate}

\[
\frac{d}{dx}[f(g(x))] = f'(g(x))\cdot g'(x)
\]

\subsubsection{B.3 Mở rộng chuỗi thông
thường}\label{b.3-mux1edf-rux1ed9ng-chuux1ed7i-thuxf4ng-thux1b0ux1eddng}

Chuỗi lũy thừa cho phép chúng ta biểu diễn các hàm dưới dạng đa thức vô
hạn. Những mở rộng này rất cần thiết cho các phép tính gần đúng, giải
phương trình vi phân và xây dựng trực giác về các hàm trong phép tính.

\paragraph{Chuỗi hình học}\label{chuux1ed7i-huxecnh-hux1ecdc}

\[
\frac{1}{1-x} = \sum_{n=0}^\infty x^n, \quad |x| < 1
\]

\paragraph{Hàm số mũ}\label{huxe0m-sux1ed1-mux169}

\[
e^x = \sum_{n=0}^\infty \frac{x^n}{n!}
= 1 + x + \frac{x^2}{2!} + \frac{x^3}{3!} + \cdots
\]

Hợp lệ cho tất cả \(x\).

\paragraph{Hàm lượng giác}\label{huxe0m-lux1b0ux1ee3ng-giuxe1c-1}

\[
\sin x = \sum_{n=0}^\infty (-1)^n \frac{x^{2n+1}}{(2n+1)!}
= x - \frac{x^3}{3!} + \frac{x^5}{5!} - \cdots
\]

\[
\cos x = \sum_{n=0}^\infty (-1)^n \frac{x^{2n}}{(2n)!}
= 1 - \frac{x^2}{2!} + \frac{x^4}{4!} - \cdots
\]

\[
\tan^{-1} x = \sum_{n=0}^\infty (-1)^n \frac{x^{2n+1}}{2n+1}, \quad |x|\leq 1
\]

\paragraph{Lôgarit}\label{luxf4garit}

\[
\ln(1+x) = \sum_{n=1}^\infty (-1)^{n+1} \frac{x^n}{n}, \quad -1 < x \leq 1
\]

\paragraph{Mở rộng nhị thức (Tổng
quát)}\label{mux1edf-rux1ed9ng-nhux1ecb-thux1ee9c-tux1ed5ng-quuxe1t}

\[
(1+x)^r = \sum_{n=0}^\infty \binom{r}{n} x^n, \quad |x|<1
\]

Ở đâu

\[
\binom{r}{n} = \frac{r(r-1)(r-2)\cdots(r-n+1)}{n!}.
\]

\subsection{Phụ lục C. Bản phác thảo chứng
minh}\label{phux1ee5-lux1ee5c-c.-bux1ea3n-phuxe1c-thux1ea3o-chux1ee9ng-minh}

\subsubsection{\texorpdfstring{C.1 Luật giới hạn và định nghĩa
\(\varepsilon\)--\(\delta\)}{C.1 Luật giới hạn và định nghĩa \textbackslash varepsilon--\textbackslash delta}}\label{c.1-luux1eadt-giux1edbi-hux1ea1n-vuxe0-ux111ux1ecbnh-nghux129a-varepsilondelta}

Phép tính dựa trên ý nghĩa chính xác của giới hạn. Mặc dù trực giác
(``các giá trị ngày càng gần gũi hơn'') là hữu ích nhưng một định nghĩa
hình thức sẽ đảm bảo tính chặt chẽ và tránh được những nghịch lý.

\paragraph{Ý tưởng trực quan}\label{uxfd-tux1b0ux1edfng-trux1ef1c-quan}

Chúng tôi viết

\[
\lim_{x \to a} f(x) = L
\]

có nghĩa là khi \(x\) tiến gần tùy ý đến \(a\), thì các giá trị của
\(f(x)\) sẽ tiến gần tùy ý đến \(L\).

\paragraph{\texorpdfstring{Trang trọng (\(\varepsilon\)--\(\delta\))
Định
nghĩa}{Trang trọng (\textbackslash varepsilon--\textbackslash delta) Định nghĩa}}\label{trang-trux1ecdng-varepsilondelta-ux111ux1ecbnh-nghux129a}

Chúng tôi nói rằng

\[
\lim_{x \to a} f(x) = L
\]

nếu với mọi \(\varepsilon > 0\), tồn tại một \(\delta > 0\) sao cho bất
cứ khi nào

\[
0 < |x-a| < \delta,
\]

chúng tôi có

\[
|f(x) - L| < \varepsilon.
\]

\begin{itemize}
\tightlist
\item
  \(\varepsilon\): chúng ta muốn \(f(x)\) tiến gần đến \(L\) đến mức
  nào.
\item
  \(\delta\): \(x\) phải gần \(a\) đến mức nào để đạt được điều đó.
\end{itemize}

\paragraph{Ví dụ}\label{vuxed-dux1ee5-35}

Cho thấy điều đó

\[
\lim_{x \to 2} (3x+1) = 7.
\]

\begin{itemize}
\tightlist
\item
  Cho \(\varepsilon > 0\).
\item
  Chúng tôi muốn \(|(3x+1)-7| < \varepsilon\).
\item
  Rút gọn: \(|3x-6| = 3|x-2| < \varepsilon\).
\item
  Điều này đúng nếu chúng ta chọn \(\delta = \varepsilon/3\).
\end{itemize}

Vì vậy, theo định nghĩa, giới hạn là 7.

\paragraph{Luật giới hạn}\label{luux1eadt-giux1edbi-hux1ea1n}

Nếu \(\lim_{x \to a} f(x) = L\) và \(\lim_{x \to a} g(x) = M\), thì:

\begin{enumerate}
\def\labelenumi{\arabic{enumi}.}
\tightlist
\item
  Tổng/Chênh lệch
\end{enumerate}

\[
\lim_{x \to a} [f(x) \pm g(x)] = L \pm M
\]

\begin{enumerate}
\def\labelenumi{\arabic{enumi}.}
\setcounter{enumi}{1}
\tightlist
\item
  Bội số không đổi
\end{enumerate}

\[
\lim_{x \to a} [c f(x)] = cL
\]

\begin{enumerate}
\def\labelenumi{\arabic{enumi}.}
\setcounter{enumi}{2}
\tightlist
\item
  Sản phẩm
\end{enumerate}

\[
\lim_{x \to a} [f(x)g(x)] = LM
\]

\begin{enumerate}
\def\labelenumi{\arabic{enumi}.}
\setcounter{enumi}{3}
\tightlist
\item
  Thương số (nếu \(M \neq 0\))
\end{enumerate}

\[
\lim_{x \to a} \frac{f(x)}{g(x)} = \frac{L}{M}
\]

\begin{enumerate}
\def\labelenumi{\arabic{enumi}.}
\setcounter{enumi}{4}
\tightlist
\item
  Quyền lực và cội nguồn
\end{enumerate}

\[
\lim_{x \to a} [f(x)]^n = L^n, \quad \lim_{x \to a} \sqrt[n]{f(x)} = \sqrt[n]{L} \ (\text{if defined}).
\]

\subsubsection{C.2 Phác thảo chứng minh: Định lý cơ bản của phép
tính}\label{c.2-phuxe1c-thux1ea3o-chux1ee9ng-minh-ux111ux1ecbnh-luxfd-cux1a1-bux1ea3n-cux1ee7a-phuxe9p-tuxednh}

Định lý cơ bản của phép tính (FTC) liên kết hai phép tính trung tâm của
phép tính: vi phân và tích phân. Nó cho thấy rằng trên thực tế, chúng là
những quá trình nghịch đảo.

\paragraph{Phát biểu của Định
lý}\label{phuxe1t-biux1ec3u-cux1ee7a-ux111ux1ecbnh-luxfd}

Phần I (Vi phân của tích phân): Nếu \(f\) liên tục trên \([a,b]\) và ta
xác định

\[
F(x) = \int_a^x f(t)\,dt,
\]

thì \(F\) khả vi trên \((a,b)\) và

\[
F'(x) = f(x).
\]

Phần II (Đánh giá tích phân xác định): Nếu \(F\) là nguyên hàm bất kỳ
của \(f\) trên \([a,b]\), thì

\[
\int_a^b f(x)\,dx = F(b)-F(a).
\]

\#\#\#\#Bản phác thảo chứng minh của Phần I

\begin{enumerate}
\def\labelenumi{\arabic{enumi}.}
\item
  Bắt đầu với định nghĩa đạo hàm:

  \[
  F'(x) = \lim_{h\to 0} \frac{F(x+h)-F(x)}{h}.
  \]
\item
  Thay \(F(x) = \int_a^x f(t)\,dt\):

  \[
  F(x+h)-F(x) = \int_a^{x+h} f(t)\,dt - \int_a^x f(t)\,dt.
  \]
\item
  Bằng tính cộng của tích phân:

  \[
  F(x+h)-F(x) = \int_x^{x+h} f(t)\,dt.
  \]
\item
  Vì vậy:

  \[
  \frac{F(x+h)-F(x)}{h} = \frac{1}{h}\int_x^{x+h} f(t)\,dt.
  \]
\item
  Theo Định lý Giá trị Trung bình của tích phân, tồn tại
  \(c \in [x,x+h]\) sao cho

  \[
  \frac{1}{h}\int_x^{x+h} f(t)\,dt = f(c).
  \]
\item
  Vì \(h \to 0\), \(c \to x\), và vì \(f\) là liên tục:

  \[
  \lim_{h\to 0} f(c) = f(x).
  \]
\end{enumerate}

Do đó, \(F'(x) = f(x)\).

\#\#\#\#Bản phác thảo chứng minh phần II

\begin{enumerate}
\def\labelenumi{\arabic{enumi}.}
\item
  Cho \(F\) là nguyên hàm của \(f\), vì vậy \(F'(x) = f(x)\).
\item
  Theo Phần I, hàm

  \[
  G(x) = \int_a^x f(t)\,dt
  \]
\end{enumerate}

cũng là nguyên hàm của \(f\).

\begin{enumerate}
\def\labelenumi{\arabic{enumi}.}
\setcounter{enumi}{2}
\item
  Vì \(F\) và \(G\) chỉ khác nhau một hằng số,

  \[
  F(x) = G(x) + C.
  \]
\item
  Đánh giá ở điểm cuối:

  \[
  \int_a^b f(x)\,dx = G(b)-G(a) = (F(b)+C)-(F(a)+C) = F(b)-F(a).
  \]
\end{enumerate}

\subsubsection{C.3 Phác thảo chứng minh: Sự hội tụ của chuỗi hình
học}\label{c.3-phuxe1c-thux1ea3o-chux1ee9ng-minh-sux1ef1-hux1ed9i-tux1ee5-cux1ee7a-chuux1ed7i-huxecnh-hux1ecdc}

Chuỗi hình học là một trong những chuỗi vô hạn đơn giản và quan trọng
nhất. Nó phục vụ như một mô hình để hiểu sự hội tụ và là nền tảng cho
nhiều kết quả sau này trong giải tích.

\paragraph{Bộ truyện}\label{bux1ed9-truyux1ec7n}

\[
\sum_{n=0}^\infty ar^n = a + ar + ar^2 + ar^3 + \cdots
\]

trong đó \(a\) là số hạng đầu tiên và \(r\) là tỉ số chung.

\paragraph{Công thức tính tổng một
phần}\label{cuxf4ng-thux1ee9c-tuxednh-tux1ed5ng-mux1ed9t-phux1ea7n}

Tổng từng phần \(n\)-th là

\[
S_n = a + ar + ar^2 + \cdots + ar^n.
\]

Nhân cả hai vế với \(r\):

\[
rS_n = ar + ar^2 + \cdots + ar^{n+1}.
\]

Trừ hai phương trình:

\[
S_n - rS_n = a - ar^{n+1}.
\]

\[
S_n(1-r) = a(1-r^{n+1}).
\]

Vì thế

\[
S_n = \frac{a(1-r^{n+1})}{1-r}, \quad r \neq 1.
\]

\paragraph{Hội tụ}\label{hux1ed9i-tux1ee5}

Lấy giới hạn là \(n \to \infty\):

\begin{itemize}
\item
  Nếu \(|r| < 1\) thì \(r^{n+1} \to 0\).

  \[
  \lim_{n\to\infty} S_n = \frac{a}{1-r}.
  \]
\item
  Nếu \(|r| \geq 1\), thì \(r^{n+1}\) không tiến về 0. Chuỗi phân kỳ.
\end{itemize}

\paragraph{Kết quả}\label{kux1ebft-quux1ea3}

\[
\sum_{n=0}^\infty ar^n =
\begin{cases}
\dfrac{a}{1-r}, & |r|<1, \\[6pt]
\text{diverges}, & |r|\geq 1.
\end{cases}
\]

\subsection{Phụ lục D. Ứng dụng và Kết
nối}\label{phux1ee5-lux1ee5c-d.-ux1ee9ng-dux1ee5ng-vuxe0-kux1ebft-nux1ed1i}

\subsubsection{D.1 Các kết nối Vật lý: Vận tốc, Gia tốc và
Công}\label{d.1-cuxe1c-kux1ebft-nux1ed1i-vux1eadt-luxfd-vux1eadn-tux1ed1c-gia-tux1ed1c-vuxe0-cuxf4ng}

Giải tích ban đầu được phát triển để giải các bài toán vật lý - đặc biệt
là chuyển động và sự biến đổi. Dưới đây là một số kết nối quan trọng
nhất.

\paragraph{Vị trí, Vận tốc và Gia
tốc}\label{vux1ecb-truxed-vux1eadn-tux1ed1c-vuxe0-gia-tux1ed1c-1}

\begin{itemize}
\item
  Hàm định vị: \(s(t)\) cho biết vị trí của một vật tại thời điểm \(t\).
\item
  Vận tốc: đạo hàm của vị trí.

  \[
  v(t) = s'(t) = \frac{ds}{dt}
  \]
\item
  Acceleration: the derivative of velocity (or second derivative of
  position).

  \[
  a(t) = v'(t) = s''(t) = \frac{d^2s}{dt^2}
  \]
\end{itemize}

Ví dụ: Nếu \(s(t) = 4t^2\) mét thì:

\[
v(t) = 8t, \quad a(t) = 8.
\]

Vì vậy vật chuyển động tuyến tính nhanh hơn theo thời gian với gia tốc
không đổi.

\paragraph{Công việc và sức
ép}\label{cuxf4ng-viux1ec7c-vuxe0-sux1ee9c-uxe9p}

Trong vật lý, công là tích của lực và khoảng cách. Nếu lực thay đổi theo
vị trí, phép tính sẽ cho:

\[
W = \int_a^b F(x)\, dx
\]

trong đó \(F(x)\) là lực tại vị trí \(x\), và vật chuyển động từ \(x=a\)
đến \(x=b\).

Ví dụ: Một lò xo chịu lực định luật Hooke \(F(x) = kx\) cần công

\[
W = \int_0^d kx\, dx = \frac{1}{2}kd^2
\]

để kéo dãn lò xo một đoạn \(d\).

\paragraph{Năng lượng và Diện tích Dưới Đường
cong}\label{nux103ng-lux1b0ux1ee3ng-vuxe0-diux1ec7n-tuxedch-dux1b0ux1edbi-ux111ux1b0ux1eddng-cong}

\begin{itemize}
\tightlist
\item
  Động năng: \(E_k = \tfrac{1}{2}mv^2\).
\item
  Thế năng thường liên quan đến tích phân (ví dụ thế năng hấp dẫn từ lực
  hấp dẫn).
\item
  Nói chung, việc tích phân hàm lực sẽ cho năng lượng được tích trữ hoặc
  thực hiện công.
\end{itemize}

\#\#\#\#Thực hành nhanh

\begin{enumerate}
\def\labelenumi{\arabic{enumi}.}
\tightlist
\item
  Nếu \(s(t) = t^3 - 3t\), hãy tìm \(v(t)\) và \(a(t)\).
\item
  Tính công do một lực không đổi 10 N thực hiện làm vật chuyển động được
  5 m.
\item
  Một lò xo có hằng số \(k=200\). Cần bao nhiêu công để kéo nó ra 0,1 m?
\item
  Chứng minh rằng gia tốc là đạo hàm bậc hai của vị trí.
\item
  Giải thích tích phân \(\int v(t)\, dt\) liên quan như thế nào đến độ
  dịch chuyển.
\end{enumerate}

\subsubsection{D.2 Kết nối xác suất và thống
kê}\label{d.2-kux1ebft-nux1ed1i-xuxe1c-suux1ea5t-vuxe0-thux1ed1ng-kuxea}

Giải tích có mối liên hệ sâu sắc với xác suất và thống kê, đặc biệt khi
xử lý các biến ngẫu nhiên liên tục. Tích phân trở nên cần thiết để xác
định xác suất, mức trung bình và kỳ vọng.

\paragraph{Hàm mật độ xác suất
(PDF)}\label{huxe0m-mux1eadt-ux111ux1ed9-xuxe1c-suux1ea5t-pdf}

Đối với biến ngẫu nhiên liên tục \(X\), xác suất được mô tả bằng hàm mật
độ xác suất \(f(x)\):

\begin{enumerate}
\def\labelenumi{\arabic{enumi}.}
\item
  \(f(x) \geq 0\) với mọi \(x\).
\item
  Tổng xác suất bằng 1:

  \[
  \int_{-\infty}^{\infty} f(x)\, dx = 1.
  \]
\end{enumerate}

Xác suất để \(X\) nằm trong khoảng \([a,b]\) là

\[
P(a \leq X \leq b) = \int_a^b f(x)\, dx.
\]

\paragraph{Giá trị mong đợi (Trung
bình)}\label{giuxe1-trux1ecb-mong-ux111ux1ee3i-trung-buxecnh-1}

Giá trị kỳ vọng (kết quả trung bình) là

\[
E[X] = \int_{-\infty}^{\infty} x f(x)\, dx.
\]

Đây là phiên bản tính toán của mức trung bình có trọng số.

\paragraph{Phương sai}\label{phux1b0ux1a1ng-sai}

Các biện pháp phương sai lây lan:

\[
\text{Var}(X) = E[(X-\mu)^2] = \int_{-\infty}^{\infty} (x-\mu)^2 f(x)\, dx,
\]

trong đó \(\mu = E[X]\).

\paragraph{Phân phối chung}\label{phuxe2n-phux1ed1i-chung}

\begin{enumerate}
\def\labelenumi{\arabic{enumi}.}
\item
  Phân phối đồng đều trên \([a,b]\):

  \[
  f(x) = \frac{1}{b-a}, \quad a \leq x \leq b.
  \]
\end{enumerate}

Giá trị trung bình: \(\frac{a+b}{2}\).

\begin{enumerate}
\def\labelenumi{\arabic{enumi}.}
\setcounter{enumi}{1}
\item
  Phân phối lũy thừa với tham số \(\lambda > 0\):

  \[
  f(x) = \lambda e^{-\lambda x}, \quad x \geq 0.
  \]
\end{enumerate}

Trung bình: \(1/\lambda\).

\begin{enumerate}
\def\labelenumi{\arabic{enumi}.}
\setcounter{enumi}{2}
\item
  Phân phối chuẩn (Gaussian):

  \[
  f(x) = \frac{1}{\sqrt{2\pi\sigma^2}} e^{-(x-\mu)^2/(2\sigma^2)}.
  \]
\end{enumerate}

Tích phân của phân phối này kết nối với hàm lỗi.

\paragraph{Tại sao điều này lại quan
trọng}\label{tux1ea1i-sao-ux111iux1ec1u-nuxe0y-lux1ea1i-quan-trux1ecdng-38}

\begin{itemize}
\tightlist
\item
  Tích phân biến xác suất thành diện tích dưới đường cong.
\item
  Phép tính liên kết kỳ vọng và phương sai với giá trị trung bình và độ
  biến thiên.
\item
  Hầu hết các mô hình dữ liệu trong thế giới thực (tài chính, vật lý,
  sinh học, AI) đều sử dụng các phân bố xác suất liên tục này.
\end{itemize}

\#\#\#\#Thực hành nhanh

\begin{enumerate}
\def\labelenumi{\arabic{enumi}.}
\tightlist
\item
  Với \(f(x) = \tfrac{1}{2}\) trên \([0,2]\), hãy tính
  \(P(0.5 \leq X \leq 1.5)\).
\item
  Để phân phối theo cấp số nhân với \(\lambda = 2\), hãy tính \(E[X]\).
\item
  Chứng minh rằng tổng diện tích dưới đường chuẩn chuẩn bằng 1.
\item
  Tìm giá trị trung bình của phân bố đều trên \([3,7]\).
\item
  Giải thích tại sao xác suất được tính bằng tích phân chứ không phải
  tổng cho các biến liên tục.
\end{enumerate}

\subsubsection{D.3 Kết nối khoa học máy tính: Xấp xỉ Taylor trong thuật
toán}\label{d.3-kux1ebft-nux1ed1i-khoa-hux1ecdc-muxe1y-tuxednh-xux1ea5p-xux1ec9-taylor-trong-thuux1eadt-touxe1n}

Giải tích không chỉ dành cho vật lý - nó còn là nền tảng của nhiều công
cụ và kỹ thuật trong khoa học máy tính. Một trong những cầu nối rõ ràng
nhất là thông qua chuỗi Taylor, cung cấp những cách hiệu quả để tính gần
đúng các hàm trong tính toán số và thuật toán.

\paragraph{Xấp xỉ hàm cho máy
tính}\label{xux1ea5p-xux1ec9-huxe0m-cho-muxe1y-tuxednh}

Máy tính không thể trực tiếp lưu trữ hoặc tính toán chính xác hầu hết
các hàm (như \(e^x\), \(\sin x\) hoặc \(\ln x\)). Thay vào đó, họ sử
dụng các phép tính gần đúng đa thức rút ra từ khai triển Taylor.

Ví dụ: Để ước chừng \(e^x\), hãy cắt bớt chuỗi Maclaurin:

\[
e^x \approx 1 + x + \frac{x^2}{2!} + \frac{x^3}{3!}.
\]

Đối với \(x\) nhỏ, đa thức này cho kết quả chính xác chỉ với một vài số
hạng.

\paragraph{Hiệu quả trong thuật
toán}\label{hiux1ec7u-quux1ea3-trong-thuux1eadt-touxe1n}

\begin{itemize}
\tightlist
\item
  Hàm lượng giác: Các thuật toán cho máy tính và CPU thường sử dụng khai
  triển chuỗi (hoặc các biến thể như đa thức Chebyshev).
\item
  Hàm mũ/logarit: Khai triển Taylor là nền tảng của phép tính gần đúng
  nhanh trong các thư viện số.
\item
  Tìm nghiệm: Phương pháp Newton dựa trên phép gần đúng tuyến tính, ứng
  dụng trực tiếp chuỗi Taylor (đạo hàm bậc nhất).
\end{itemize}

\paragraph{Phân tích số}\label{phuxe2n-tuxedch-sux1ed1}

Khai triển Taylor là trọng tâm trong phân tích lỗi:

\begin{itemize}
\item
  Xấp xỉ sai số bằng công thức tính số dư:

  \[
  R_n(x) = \frac{f^{(n+1)}(c)}{(n+1)!}(x-a)^{n+1}.
  \]
\item
  This tells us how many terms are needed for a given accuracy.
\end{itemize}

\paragraph{Kết nối máy học}\label{kux1ebft-nux1ed1i-muxe1y-hux1ecdc}

\begin{itemize}
\tightlist
\item
  Tối ưu hóa dựa trên độ dốc (như giảm độ dốc) sử dụng đạo hàm để cập
  nhật các tham số một cách hiệu quả.
\item
  Các hàm kích hoạt (như \(\tanh x\) hoặc \(\sigma(x)=1/(1+e^{-x})\))
  thường được xấp xỉ bằng các đa thức hoặc các hàm từng phần cho tốc độ.
\item
  Chuỗi xấp xỉ có thể tăng tốc độ đào tạo và suy luận trong môi trường
  hạn chế.
\end{itemize}

\paragraph{Tại sao điều này lại quan
trọng}\label{tux1ea1i-sao-ux111iux1ec1u-nuxe0y-lux1ea1i-quan-trux1ecdng-39}

\begin{itemize}
\tightlist
\item
  Phép tính gần đúng Taylor kết nối toán học liên tục với tính toán rời
  rạc.
\item
  Chúng cho thấy các khái niệm tính toán được sử dụng như thế nào trong
  các thuật toán, phương pháp số và học máy.
\item
  Hiểu được các phép tính gần đúng giúp tránh được những cạm bẫy khi dựa
  vào máy tính để tính toán.
\end{itemize}

\#\#\#\#Thực hành nhanh

\begin{enumerate}
\def\labelenumi{\arabic{enumi}.}
\tightlist
\item
  Tính gần đúng \(\sin(0,1)\) bằng cách sử dụng ba số hạng đầu tiên của
  chuỗi Maclaurin.
\item
  Sử dụng số hạng còn lại để ước tính sai số khi xấp xỉ \(e^1\) với đa
  thức bậc 3.
\item
  Giải thích cách sử dụng định lý Taylor trong phương pháp Newton.
\item
  Tại sao máy tính có thể thích các phép tính gần đúng đa thức hơn là
  các công thức chính xác của hàm số?
\item
  Trong học máy, tại sao đạo hàm (độ dốc) lại quan trọng đối với việc
  tối ưu hóa?
\end{enumerate}




\end{document}
