% Options for packages loaded elsewhere
\PassOptionsToPackage{unicode}{hyperref}
\PassOptionsToPackage{hyphens}{url}
\PassOptionsToPackage{dvipsnames,svgnames,x11names}{xcolor}
\documentclass[
  12pt,
  a4paper,
]{article}
\usepackage{xcolor}
\usepackage[a4paper,margin=2.5cm,heightrounded]{geometry}
\usepackage{amsmath,amssymb}
\setcounter{secnumdepth}{5}
\usepackage{iftex}
\ifPDFTeX
  \usepackage[T1]{fontenc}
  \usepackage[utf8]{inputenc}
  \usepackage{textcomp} % provide euro and other symbols
\else % if luatex or xetex
  \usepackage{unicode-math} % this also loads fontspec
  \defaultfontfeatures{Scale=MatchLowercase}
  \defaultfontfeatures[\rmfamily]{Ligatures=TeX,Scale=1}
\fi
\usepackage{lmodern}
\ifPDFTeX\else
  % xetex/luatex font selection
  \setmainfont[]{Noto Serif}
  \setmonofont[]{Noto Sans Mono}
\fi
% Use upquote if available, for straight quotes in verbatim environments
\IfFileExists{upquote.sty}{\usepackage{upquote}}{}
\IfFileExists{microtype.sty}{% use microtype if available
  \usepackage[]{microtype}
  \UseMicrotypeSet[protrusion]{basicmath} % disable protrusion for tt fonts
}{}
\makeatletter
\@ifundefined{KOMAClassName}{% if non-KOMA class
  \IfFileExists{parskip.sty}{%
    \usepackage{parskip}
  }{% else
    \setlength{\parindent}{0pt}
    \setlength{\parskip}{6pt plus 2pt minus 1pt}}
}{% if KOMA class
  \KOMAoptions{parskip=half}}
\makeatother
\usepackage{longtable,booktabs,array}
\usepackage{calc} % for calculating minipage widths
% Correct order of tables after \paragraph or \subparagraph
\usepackage{etoolbox}
\makeatletter
\patchcmd\longtable{\par}{\if@noskipsec\mbox{}\fi\par}{}{}
\makeatother
% Allow footnotes in longtable head/foot
\IfFileExists{footnotehyper.sty}{\usepackage{footnotehyper}}{\usepackage{footnote}}
\makesavenoteenv{longtable}
\setlength{\emergencystretch}{3em} % prevent overfull lines
\providecommand{\tightlist}{%
  \setlength{\itemsep}{0pt}\setlength{\parskip}{0pt}}
\usepackage{amsmath}   % already pulled in by Pandoc, but safe
\renewenvironment{equation}{\begin{equation*}}{\end{equation*}}
\usepackage{microtype}
\newcommand{\docversion}{0.1.1}
\setcounter{secnumdepth}{0}
\setcounter{tocdepth}{2} % keep TOC depth as desired
\makeatletter
\renewcommand{\sectionmark}[1]{\markboth{#1}{}}
\renewcommand{\subsectionmark}[1]{\markright{#1}}
\makeatother
\usepackage{fancyhdr}
\fancyhf{}%
\fancyhead[LE,RO]{\thepage}%
\fancyhead[LO,RE]{\nouppercase{\leftmark}}%
\fancyfoot[C]{Version \docversion}%
\renewcommand{\headrulewidth}{0.4pt}%
\setlength{\headheight}{13.6pt}%
}
\usepackage{etoolbox}
\AtBeginDocument{\thispagestyle{plain}\pagestyle{content}}
\let\oldsection\section
\renewcommand{\section}{\clearpage\oldsection}
\usepackage{bookmark}
\IfFileExists{xurl.sty}{\usepackage{xurl}}{} % add URL line breaks if available
\urlstyle{same}
\hypersetup{
  pdftitle={The Little Book of Calculus},
  pdfauthor={Duc-Tam Nguyen},
  colorlinks=true,
  linkcolor={blue},
  filecolor={Maroon},
  citecolor={Blue},
  urlcolor={blue},
  pdfcreator={LaTeX via pandoc}}

\title{The Little Book of Calculus}
\usepackage{etoolbox}
\makeatletter
\providecommand{\subtitle}[1]{% add subtitle to \maketitle
  \apptocmd{\@title}{\par {\large #1 \par}}{}{}
}
\makeatother
\subtitle{Version 0.1.0}
\author{Duc-Tam Nguyen}
\date{\today}

\begin{document}
\maketitle

{
\hypersetup{linkcolor=}
\setcounter{tocdepth}{2}
\tableofcontents
}
\section{Part 1. Limits and
derivatives}\label{part-1-limits-and-derivatives}

\section{\texorpdfstring{Chapter 1. Functions and limits
}{Chapter 1. Functions and limits }}\label{chapter-1-functions-and-limits}

\subsection{1.1 Functions}\label{11-functions}

A function is one of the most basic objects in mathematics. At its
heart, a function is a rule that takes an input and produces exactly one
output. Functions let us describe relationships, model real-world
phenomena, and build the entire machinery of calculus.

\subsubsection{Definition}\label{definition}

Formally, a function from a set (called the domain) to a set (called the
codomain) is written

\[f : X \to Y.\]

For every element , there is a unique element . The value is called the
image of under .

If , then is the output corresponding to the input . The set of all
outputs that actually appear is called the range (a subset of the
codomain).

\subsubsection{Examples}\label{examples}

\begin{enumerate}
\def\labelenumi{\arabic{enumi}.}
\item
  The function maps each real number to its square.

  \begin{itemize}
  \item
    Domain: all real numbers .
  \item
    Codomain: all real numbers .
  \item
    Range: all nonnegative real numbers .
  \end{itemize}
\item
  The function assigns to each nonzero real number its reciprocal.

  \begin{itemize}
  \item
    Domain: .
  \item
    Range: .
  \end{itemize}
\item
  A real-world example: Let be the outside temperature (in °C) at time
  (in hours). This is a function from ``time of day'' to
  ``temperature.''
\end{enumerate}

\subsubsection{Ways of Representing
Functions}\label{ways-of-representing-functions}

Functions can be represented in several useful ways:

\begin{itemize}
\item
  Formulas: e.g., .
\item
  Graphs: plotting all points in the coordinate plane.
\item
  Tables: pairing inputs and outputs for discrete sets of data.
\item
  Verbal descriptions: ``Assign to each student their grade.''
\end{itemize}

Each representation highlights different aspects of the same function.

\subsubsection{Terminology}\label{terminology}

\begin{itemize}
\item
  Independent variable: the input (usually written ).
\item
  Dependent variable: the output (usually written , where ).
\item
  Function notation: is read `` of .''
\end{itemize}

\subsubsection{Why Functions Matter in
Calculus}\label{why-functions-matter-in-calculus}

Calculus is the study of how functions change. Derivatives measure
instantaneous rates of change, while integrals measure accumulated
effects. To master these ideas, we first need a solid understanding of
what functions are and how they behave.

\subsubsection{Exercises}\label{exercises}

\begin{enumerate}
\def\labelenumi{\arabic{enumi}.}
\item
  For the function :

  \begin{itemize}
  \item
    Find the domain, codomain, and range.
  \end{itemize}
\item
  The function is defined for which inputs? What is its range?
\item
  Give a real-world example of a function from your daily life. Clearly
  state the domain and codomain.
\item
  Sketch the graph of . What is the range?
\item
  Suppose . Explain why its range is the interval .
\end{enumerate}

\subsection{1.2 Graphs and
Transformations}\label{12-graphs-and-transformations}

A function can be understood not only by formulas but also by its graph.
The graph of a function is the set of all ordered pairs , where belongs
to the domain of . Plotting these pairs in the coordinate plane gives a
picture of how the function behaves.

\subsubsection{Basic Graphs}\label{basic-graphs}

Some graphs are so fundamental that they should be memorized:

\begin{itemize}
\item
  : a straight line through the origin.
\item
  : a parabola opening upward.
\item
  : a ``V''-shaped graph.
\item
  : a hyperbola with two branches.
\item
  : a wave-like periodic curve.
\end{itemize}

These serve as the building blocks for more complicated functions.

\subsubsection{Transformations}\label{transformations}

Graphs can be shifted, stretched, or reflected using simple rules:

\begin{enumerate}
\def\labelenumi{\arabic{enumi}.}
\item
  Vertical shifts: Adding a constant moves the graph up or down.

  \[y = f(x) + c \quad \text{is } f(x) \text{ shifted upward by } c.\]
\item
  Horizontal shifts: Adding inside the argument moves the graph left or
  right.

  \[y = f(x - c) \quad \text{is } f(x) \text{ shifted right by } c.\]
\item
  Vertical scaling: Multiplying by a constant stretches or compresses
  the graph vertically.

  \[y = a f(x), \quad a > 1 \text{ stretches; } 0 < a < 1 \text{ compresses.}\]
\item
  Horizontal scaling: Multiplying inside the argument stretches or
  compresses the graph horizontally.

  \[y = f(bx), \quad b > 1 \text{ compresses toward the } y\text{-axis}.\]
\item
  Reflections:

  \begin{itemize}
  \item
    : reflection across the -axis.
  \item
    : reflection across the -axis.
  \end{itemize}
\end{enumerate}

\subsubsection{Combining
Transformations}\label{combining-transformations}

Complex graphs often come from combining several transformations in
sequence. For example:

\[y = 2(x-1)^2 + 3\]

is obtained by taking the parabola , shifting right by 1, stretching
vertically by 2, and shifting upward by 3.

\subsubsection{Exercises}\label{exercises-2}

\begin{enumerate}
\def\labelenumi{\arabic{enumi}.}
\item
  Sketch the graph of . Identify the sequence of transformations from .
\item
  What happens to the graph of if we replace by ? Try it with .
\item
  Describe the transformations that turn into .
\item
  Draw the graph of . State its vertex and slope of each branch.
\item
  For , explain how the graph of has been transformed.
\end{enumerate}

\subsection{1.3 Intuitive Idea of
Limits}\label{13-intuitive-idea-of-limits}

In many situations, a function's value at a point is less important than
the values it takes near that point. The concept of a limit captures
this idea.

\subsubsection{Approaching a Value}\label{approaching-a-value}

Imagine walking toward a wall. Even before you touch it, you get closer
and closer. In the same way, as approaches a number , the values of may
approach some number . We then say:

\[\lim_{x \to a} f(x) = L.\]

This expresses the idea that can be made as close as we want to , simply
by taking close enough to .

\subsubsection{Examples}\label{examples-2}

\begin{enumerate}
\def\labelenumi{\arabic{enumi}.}
\item
  For :\\
  As , .
\item
  For :\\
  As , the function approaches 1, even though is not defined.
\item
  For :\\
  As (approaching from the right), .\\
  As (approaching from the left), .\\
  Since the left and right behaviors differ, the limit at 0 does not
  exist.
\end{enumerate}

\subsubsection{Importance of Limits}\label{importance-of-limits}

\begin{itemize}
\item
  They allow us to define functions at points where they are not
  originally defined.
\item
  They capture behavior near discontinuities and singularities.
\item
  They form the foundation for derivatives (instantaneous rates of
  change) and integrals (areas as limits of sums).
\end{itemize}

\subsubsection{One-Sided Limits}\label{one-sided-limits}

Sometimes the behavior from the left and right must be studied
separately:

\[\lim_{x \to a^-} f(x), \quad \lim_{x \to a^+} f(x).\]

If both agree, then the two-sided limit exists.

\subsubsection{Exercises}\label{exercises-3}

\begin{enumerate}
\def\labelenumi{\arabic{enumi}.}
\item
  Compute .
\item
  What is ? Use intuition from the graph of .
\item
  Evaluate . Does the two-sided limit exist?
\item
  Find . Interpret this result in words.
\item
  For , what is ? Compare with the value of .
\end{enumerate}

\subsection{1.4 Formal Definition of
Limits}\label{14-formal-definition-of-limits}

The intuitive idea of a limit can be made precise using the
epsilon--delta definition. This gives us a rigorous way to say that gets
close to a value as gets close to .

\subsubsection{The Definition}\label{the-definition}

We write

\[\lim_{x \to a} f(x) = L\]

if the following condition holds:

For every (no matter how small), there exists a such that whenever

\[0 < |x - a| < \delta,\]

it follows that

\[|f(x) - L| < \varepsilon.\]

In words: we can make as close as we like to , provided is close enough
to (but not equal to ).

\subsubsection{Example 1: Linear
Function}\label{example-1-linear-function}

For , show that .

\begin{itemize}
\item
  We want .
\item
  But .
\item
  So .
\item
  If we choose , then whenever , we have .\\
  This proves the limit.
\end{itemize}

\subsubsection{Example 2: Reciprocal
Function}\label{example-2-reciprocal-function}

For , consider .

\begin{itemize}
\item
  We want .
\item
  This inequality requires algebraic manipulation, but it can be
  satisfied by choosing depending on .\\
  The process is more complicated, but the principle is the same.
\end{itemize}

\subsubsection{Why This Matters}\label{why-this-matters}

\begin{itemize}
\item
  The epsilon--delta definition guarantees that limits are not vague or
  based only on intuition.
\item
  It is the foundation for continuity, derivatives, and integrals.
\item
  Though beginners may find it abstract, working with simple examples
  builds familiarity.
\end{itemize}

\subsubsection{Exercises}\label{exercises-4}

\begin{enumerate}
\def\labelenumi{\arabic{enumi}.}
\item
  Using the epsilon--delta definition, prove that .
\item
  Show that using the formal definition.
\item
  Explain why does not exist.
\item
  For , show that .
\item
  In your own words, explain the role of and in the definition of a
  limit.
\end{enumerate}

\subsection{1.5 Continuity}\label{15-continuity}

A function is continuous if its graph can be drawn without lifting your
pencil from the paper. More precisely, continuity ensures that small
changes in the input produce small changes in the output.

\subsubsection{Definition}\label{definition-2}

A function is continuous at a point if three conditions are satisfied:

\begin{enumerate}
\def\labelenumi{\arabic{enumi}.}
\item
  is defined.
\item
  exists.
\item
  .
\end{enumerate}

If a function is continuous at every point in an interval, we say it is
continuous on that interval.

\subsubsection{Examples}\label{examples-3}

\begin{enumerate}
\def\labelenumi{\arabic{enumi}.}
\item
  Polynomial functions: Functions like are continuous everywhere on .
\item
  Rational functions: is continuous everywhere except at , where it is
  undefined.
\item
  Piecewise functions:

  \[f(x) =
  \begin{cases}
  x^2 & x < 1, \\
  2 & x = 1, \\
  x+1 & x > 1,
  \end{cases}\]

  This function has a ``jump'' at , so it is not continuous there.
\end{enumerate}

\subsubsection{Types of Discontinuities}\label{types-of-discontinuities}

\begin{enumerate}
\def\labelenumi{\arabic{enumi}.}
\item
  Removable discontinuity: A ``hole'' in the graph. Example: at .
\item
  Jump discontinuity: The left-hand and right-hand limits are different.
\item
  Infinite discontinuity: The function goes to near a point, as with
  near .
\end{enumerate}

\subsubsection{The Intermediate Value
Theorem}\label{the-intermediate-value-theorem}

If a function is continuous on an interval , then for any number between
and , there exists some such that .

This property is crucial in proving the existence of roots and solutions
to equations.

\subsubsection{Exercises}\label{exercises-5}

\begin{enumerate}
\def\labelenumi{\arabic{enumi}.}
\item
  Decide whether the function is continuous at .
\item
  Identify the points of discontinuity for .
\item
  Explain why every polynomial function is continuous everywhere.
\item
  Give an example of a function with a jump discontinuity. Sketch its
  graph.
\item
  Use the Intermediate Value Theorem to show that the equation has a
  solution between 0 and 1.
\end{enumerate}

\section{\texorpdfstring{Chapter 2. Derivatives
}{Chapter 2. Derivatives }}\label{chapter-2-derivatives}

\subsection{2.1 The Derivative as a Rate of
Change}\label{21-the-derivative-as-a-rate-of-change}

The derivative is one of the central ideas of calculus. It measures how
a function changes as its input changes - in other words, the rate of
change of the output with respect to the input.

\subsubsection{Average Rate of Change}\label{average-rate-of-change}

For a function , the average rate of change between two points and is

\[\frac{f(b) - f(a)}{b - a}.\]

This is the slope of the secant line through the points and .

\subsubsection{Instantaneous Rate of
Change}\label{instantaneous-rate-of-change}

To measure how fast is changing at a single point, we let the interval
shrink:

\[f'(a) = \lim_{h \to 0} \frac{f(a+h) - f(a)}{h}.\]

This limit, if it exists, is called the derivative of at .
Geometrically, it is the slope of the tangent line to the graph of at
the point .

\subsubsection{Notation}\label{notation}

\begin{itemize}
\item
  : prime notation.
\item
  : Leibniz notation, used when .
\item
  : operator notation.
\end{itemize}

All these symbols refer to the same concept.

\subsubsection{Examples}\label{examples-4}

\begin{enumerate}
\def\labelenumi{\arabic{enumi}.}
\item
  For :

  \[f'(x) = \lim_{h \to 0} \frac{(x+h)^2 - x^2}{h} = \lim_{h \to 0} \frac{2xh + h^2}{h} = 2x.\]

  The slope of the parabola at is .
\item
  For :

  \[f'(x) = \cos x.\]
\item
  For (a constant):

  \[f'(x) = 0.\]

  A constant function never changes.
\end{enumerate}

\subsubsection{Interpretation}\label{interpretation}

\begin{itemize}
\item
  In physics: If is position, then is velocity.
\item
  In economics: If is cost, then is marginal cost.
\item
  In biology: If is population, then is growth rate.
\end{itemize}

The derivative makes ``change'' precise in many contexts.

\subsubsection{Exercises}\label{exercises-6}

\begin{enumerate}
\def\labelenumi{\arabic{enumi}.}
\item
  Compute for .
\item
  Find the slope of the tangent line to at .
\item
  If represents distance in meters, what is the velocity at ?
\item
  Use the limit definition to compute the derivative of .
\item
  Sketch the graph of and draw the tangent line at .
\end{enumerate}

\subsection{2.2 Differentiation Rules}\label{22-differentiation-rules}

Once the derivative is defined, we need efficient ways to compute it.
The differentiation rules are shortcuts that save us from repeatedly
applying the limit definition.

\subsubsection{The Constant Rule}\label{the-constant-rule}

If where is a constant, then

\[f'(x) = 0.\]

\subsubsection{The Power Rule}\label{the-power-rule}

For where is a real number,

\[\frac{d}{dx} \big( x^n \big) = n x^{n-1}.\]

Examples:

\begin{itemize}
\item
  .
\item
  .
\item
  .
\end{itemize}

\subsubsection{The Constant Multiple
Rule}\label{the-constant-multiple-rule}

If , then

\[f'(x) = c \cdot g'(x).\]

\subsubsection{The Sum and Difference
Rules}\label{the-sum-and-difference-rules}

\begin{itemize}
\item
  .
\item
  .
\end{itemize}

\subsubsection{The Product Rule}\label{the-product-rule}

For and :

\[(fg)' = f'g + fg'.\]

Example: If , :

\[(fg)' = (2x)(\sin x) + (x^2)(\cos x).\]

\subsubsection{The Quotient Rule}\label{the-quotient-rule}

For and :

\[\left(\frac{f}{g}\right)' = \frac{f'g - fg'}{g^2}, \quad g(x) \neq 0.\]

Example: If , :

\[\left(\frac{x^2}{x+1}\right)' = \frac{(2x)(x+1) - (x^2)(1)}{(x+1)^2}.\]

\subsubsection{Derivatives of Common
Functions}\label{derivatives-of-common-functions}

\begin{itemize}
\item
  .
\item
  .
\item
  .
\item
  .
\end{itemize}

\subsubsection{Exercises}\label{exercises-7}

\begin{enumerate}
\def\labelenumi{\arabic{enumi}.}
\item
  Differentiate .
\item
  Use the product rule to find the derivative of .
\item
  Apply the quotient rule to .
\item
  Compute using the chain of rules.
\item
  Show that the derivative of is .
\end{enumerate}

\subsection{2.3 The Chain Rule}\label{23-the-chain-rule}

Often, functions are built by combining simpler functions together. To
differentiate such composite functions, we use the chain rule.

\subsubsection{The Rule}\label{the-rule}

If , then

\[\frac{dy}{dx} = f'(g(x)) \cdot g'(x).\]

In words: differentiate the outer function, keep the inside unchanged,
then multiply by the derivative of the inside.

\subsubsection{Examples}\label{examples-5}

\begin{enumerate}
\def\labelenumi{\arabic{enumi}.}
\item
  Square of a linear function

  \[y = (3x+2)^2\]

  Outer function: , inner function: .

  \[y' = 2(3x+2) \cdot 3 = 6(3x+2).\]
\item
  Exponential with quadratic inside

  \[y = e^{x^2}\]

  Outer function: , inner function: .

  \[y' = e^{x^2} \cdot 2x = 2x e^{x^2}.\]
\item
  Logarithm with root inside

  \[y = \ln(\sqrt{x})\]

  Outer: , inner: .

  \[y' = \frac{1}{\sqrt{x}} \cdot \frac{1}{2\sqrt{x}} = \frac{1}{2x}.\]
\end{enumerate}

\subsubsection{Generalized Chain Rule}\label{generalized-chain-rule}

For multiple nested functions :

\[\frac{dy}{dx} = f'(g(h(x))) \cdot g'(h(x)) \cdot h'(x).\]

This extends naturally to deeper compositions.

\subsubsection{Why the Chain Rule
Matters}\label{why-the-chain-rule-matters}

\begin{itemize}
\item
  It handles nearly all real-world models where one quantity depends on
  another indirectly.
\item
  It connects calculus with physics (e.g., velocity depending on time
  through position).
\item
  It is essential in implicit differentiation and advanced topics.
\end{itemize}

\subsubsection{Exercises}\label{exercises-8}

\begin{enumerate}
\def\labelenumi{\arabic{enumi}.}
\item
  Differentiate .
\item
  Find .
\item
  Compute .
\item
  Differentiate .
\item
  Apply the generalized chain rule to .
\end{enumerate}

\subsection{2.4 Implicit
Differentiation}\label{24-implicit-differentiation}

Not all functions are given in the form . Sometimes and are related by
an equation, and solving explicitly for is difficult or impossible. In
such cases, we use implicit differentiation.

\subsubsection{The Idea}\label{the-idea}

If an equation involves both and , we can differentiate both sides with
respect to , treating as a function of . Each time we differentiate a
term involving , we multiply by .

\subsubsection{Example 1: A Circle}\label{example-1-a-circle}

Equation:

\[x^2 + y^2 = 25\]

Differentiate with respect to :

\[2x + 2y \frac{dy}{dx} = 0.\]

Solve for :

\[\frac{dy}{dx} = -\frac{x}{y}.\]

This gives the slope of the tangent to the circle at any point.

\subsubsection{Example 2: A Product of
Variables}\label{example-2-a-product-of-variables}

Equation:

\[xy = 1\]

Differentiate:

\[x \frac{dy}{dx} + y = 0.\]

So,

\[\frac{dy}{dx} = -\frac{y}{x}.\]

\subsubsection{Example 3: Trigonometric
Relation}\label{example-3-trigonometric-relation}

Equation:

\[\sin(xy) = x\]

Differentiate:

\[\cos(xy) \cdot \Big(y + x\frac{dy}{dx}\Big) = 1.\]

Solve for :

\[\frac{dy}{dx} = \frac{1 - y\cos(xy)}{x\cos(xy)}.\]

\subsubsection{Why Implicit Differentiation is
Useful}\label{why-implicit-differentiation-is-useful}

\begin{itemize}
\item
  Many important curves (circles, ellipses, hyperbolas) are naturally
  defined implicitly.
\item
  It allows us to differentiate equations without first solving for .
\item
  It is a key step in more advanced topics such as related rates and
  differential equations.
\end{itemize}

\subsubsection{Exercises}\label{exercises-9}

\begin{enumerate}
\def\labelenumi{\arabic{enumi}.}
\item
  For the curve , find .
\item
  Differentiate implicitly.
\item
  Find the slope of the tangent line to at the point .
\item
  Given , compute when .
\item
  Differentiate to find .
\end{enumerate}

\subsection{2.5 Higher-Order
Derivatives}\label{25-higher-order-derivatives}

So far, we have studied the first derivative, which measures the rate of
change of a function. But derivatives themselves can also be
differentiated, giving rise to higher-order derivatives.

\subsubsection{Definition}\label{definition-3}

\begin{itemize}
\item
  The second derivative of is the derivative of the derivative:

  \[f''(x) = \frac{d}{dx}\left(f'(x)\right).\]
\item
  More generally, the -th derivative is written as

  \[f^{(n)}(x) = \frac{d^n}{dx^n} f(x).\]
\end{itemize}

\subsubsection{Examples}\label{examples-6}

\begin{enumerate}
\def\labelenumi{\arabic{enumi}.}
\item
  \begin{itemize}
  \item
    First derivative: .
  \item
    Second derivative: .
  \item
    Third derivative: .
  \item
    Fourth derivative: .
  \end{itemize}
\item
  \begin{itemize}
  \item
    .
  \item
    .
  \item
    .
  \item
    .\\
    The derivatives repeat in a cycle of length 4.
  \end{itemize}
\item
  \begin{itemize}
  \item
    Every derivative is .
  \end{itemize}
\end{enumerate}

\subsubsection{Applications}\label{applications}

\begin{itemize}
\item
  Concavity: The sign of tells whether the graph of is concave up () or
  concave down ().
\item
  Inflection points: Points where and the concavity changes.
\item
  Motion: In physics, if is position:

  \begin{itemize}
  \item
    = velocity,
  \item
    = acceleration,
  \item
    = jerk (rate of change of acceleration).
  \end{itemize}
\item
  Approximations: Higher-order derivatives appear in Taylor series, used
  to approximate functions.
\end{itemize}

\subsubsection{Exercises}\label{exercises-10}

\begin{enumerate}
\def\labelenumi{\arabic{enumi}.}
\item
  Compute the first four derivatives of .
\item
  Find for .
\item
  For , show that .
\item
  Determine the intervals where is concave up and concave down.
\item
  If , find the velocity and acceleration at .
\end{enumerate}

\section{Chapter 3. Applications of
Derivatives}\label{chapter-3-applications-of-derivatives}

\subsection{3.1 Tangents and Normals}\label{31-tangents-and-normals}

One of the first applications of derivatives is finding the equations of
tangent and normal lines to a curve. These lines capture the local
geometry of a function at a given point.

\subsubsection{Tangent Line}\label{tangent-line}

The tangent line to a curve at a point is the line that just ``touches''
the graph there and has the same slope as the curve.

The slope of the tangent line is given by the derivative:

\[m_{\text{tangent}} = f'(a).\]

Thus, the equation of the tangent line at is

\[y - f(a) = f'(a)(x - a).\]

\subsubsection{Normal Line}\label{normal-line}

The normal line is perpendicular to the tangent line at the same point.
Its slope is the negative reciprocal of the tangent slope:

\[m_{\text{normal}} = -\frac{1}{f'(a)}.\]

So the equation of the normal line is

\[y - f(a) = -\frac{1}{f'(a)} (x - a), \quad f'(a) \neq 0.\]

\subsubsection{Examples}\label{examples-7}

\begin{enumerate}
\def\labelenumi{\arabic{enumi}.}
\item
  at .

  \begin{itemize}
  \item
    , , so .
  \item
    Tangent: , or .
  \item
    Normal: slope = , so equation is .
  \end{itemize}
\item
  at .

  \begin{itemize}
  \item
    , .
  \item
    Tangent: .
  \end{itemize}
\end{enumerate}

\subsubsection{Why Tangents and Normals
Matter}\label{why-tangents-and-normals-matter}

\begin{itemize}
\item
  Tangents approximate the curve locally (linear approximation).
\item
  Normals are useful in geometry, optics (reflection/refraction), and
  mechanics (force directions).
\item
  Both play a role in optimization and curvature studies.
\end{itemize}

\subsubsection{Exercises}\label{exercises-11}

\begin{enumerate}
\def\labelenumi{\arabic{enumi}.}
\item
  Find the tangent and normal lines to at .
\item
  Determine the tangent and normal lines to at .
\item
  For , compute the tangent line at .
\item
  A circle is given by . Use implicit differentiation to find the slope
  of the tangent at .
\item
  Sketch the graph of and draw the tangent and normal lines at .
\end{enumerate}

\subsection{3.2 Related Rates}\label{32-related-rates}

In many real-world problems, two or more quantities change with respect
to time, and their rates of change are connected. Related rates problems
use derivatives to describe these relationships.

\subsubsection{General Approach}\label{general-approach}

\begin{enumerate}
\def\labelenumi{\arabic{enumi}.}
\item
  Identify the variables that depend on time .
\item
  Write an equation relating the variables.
\item
  Differentiate both sides with respect to , applying the chain rule.
\item
  Substitute the known values at the given instant.
\item
  Solve for the unknown rate.
\end{enumerate}

\subsubsection{Example 1: Expanding
Circle}\label{example-1-expanding-circle}

A circle has radius , which increases at the rate of . Find the rate at
which the area increases when .

Differentiate:

\[\frac{dA}{dt} = 2\pi r \frac{dr}{dt}.\]

Substitute:

\[\frac{dA}{dt} = 2\pi (5)(2) = 20\pi \,\text{cm}^2/\text{s}.\]

\subsubsection{Example 2: Sliding
Ladder}\label{example-2-sliding-ladder}

A 10 ft ladder leans against a wall. The bottom slides away at . How
fast is the top sliding down when the bottom is 6 ft from the wall?

Equation: , where is the height.

Differentiate:

\[2x \frac{dx}{dt} + 2y \frac{dy}{dt} = 0.\]

At , . Substitute:

\[2(6)(1) + 2(8)\frac{dy}{dt} = 0 \quad \Rightarrow \quad \frac{dy}{dt} = -\tfrac{6}{8} = -\tfrac{3}{4}.\]

So the top slides down at .

\subsubsection{Example 3: Water in a
Cone}\label{example-3-water-in-a-cone}

Water is poured into a cone of height 12 cm and radius 6 cm. When the
water is 4 cm deep, the water level is rising at . At what rate is the
volume increasing?

Equation: . Using similarity, . Substituting:

\[V = \tfrac{1}{12}\pi h^3.\]

Differentiate:

\[\frac{dV}{dt} = \tfrac{1}{4}\pi h^2 \frac{dh}{dt}.\]

At , :

\[\frac{dV}{dt} = \tfrac{1}{4}\pi (16)(2) = 8\pi \,\text{cm}^3/\text{s}.\]

\subsubsection{Why Related Rates Matter}\label{why-related-rates-matter}

\begin{itemize}
\item
  They describe motion and change in physics, engineering, and biology.
\item
  They connect geometry with calculus through time-dependent processes.
\item
  They train us to model dynamic systems mathematically.
\end{itemize}

\subsubsection{Exercises}\label{exercises-12}

\begin{enumerate}
\def\labelenumi{\arabic{enumi}.}
\item
  A balloon is inflated so that its radius increases at . Find how fast
  its volume increases when the radius is 10 cm.
\item
  A car drives north at 40 km/h and another east at 30 km/h. How fast is
  the distance between them increasing 2 hours later?
\item
  A spotlight 20 m from a wall shines on a man 2 m tall walking away at
  1.5 m/s. How fast does the length of his shadow on the wall change
  when he is 5 m from the light?
\item
  A cube's side length grows at 2 cm/s. How fast is the surface area
  increasing when the side is 3 cm?
\item
  Sand is poured onto a pile forming a cone with radius always equal to
  the height. If the height increases at 5 cm/s, at what rate is the
  volume increasing when the height is 10 cm?
\end{enumerate}

\subsection{3.3 Optimization Problems}\label{33-optimization-problems}

Optimization problems use derivatives to find the maximum or minimum
values of a function, often under certain constraints. These problems
model situations where we want to maximize efficiency, profit, or area,
or minimize cost, distance, or time.

\subsubsection{General Steps}\label{general-steps}

\begin{enumerate}
\def\labelenumi{\arabic{enumi}.}
\item
  Understand the problem: Identify the quantity to optimize.
\item
  Model with a function: Write the objective function in terms of one
  variable.
\item
  Apply constraints: Use given conditions to reduce variables.
\item
  Differentiate: Compute the derivative of the objective function.
\item
  Find critical points: Solve or where is undefined.
\item
  Test for maxima/minima: Use the second derivative test or check
  endpoints.
\item
  Interpret the result: State the answer in the original context.
\end{enumerate}

\subsubsection{Example 1: Maximum Area of a
Rectangle}\label{example-1-maximum-area-of-a-rectangle}

A rectangle has perimeter 40. What dimensions maximize its area?

\begin{itemize}
\item
  Let length , width . Constraint: .
\item
  Area: .
\item
  Derivative: . Set equal to 0: .
\item
  Then .
\item
  Maximum area: . The rectangle is a square.
\end{itemize}

\subsubsection{Example 2: Minimizing
Distance}\label{example-2-minimizing-distance}

Find the point on the parabola closest to .

\begin{itemize}
\item
  Distance squared: .
\item
  Expand: .
\item
  Derivative: . Solve: .
\item
  Solutions: , .
\item
  Checking gives the minimum distance at .
\end{itemize}

\subsubsection{Example 3: Box with Maximum
Volume}\label{example-3-box-with-maximum-volume}

A box with no top is to be made from a square piece of cardboard 20 cm
on a side by cutting out equal squares from the corners and folding up
the sides. Find the size of the cut that maximizes volume.

\begin{itemize}
\item
  Let cut size = . Then dimensions: .
\item
  Volume: .
\item
  Derivative: .
\item
  Critical points: (gives zero volume) or .
\item
  At , volume is maximized.
\end{itemize}

\subsubsection{Why Optimization Matters}\label{why-optimization-matters}

\begin{itemize}
\item
  Engineers use it to design efficient structures.
\item
  Businesses use it to maximize profit or minimize costs.
\item
  Scientists use it to model natural systems that seek equilibrium.
\end{itemize}

\subsubsection{Exercises}\label{exercises-13}

\begin{enumerate}
\def\labelenumi{\arabic{enumi}.}
\item
  A farmer has 100 m of fencing to enclose a rectangular field along a
  river (so only 3 sides need fencing). Find dimensions maximizing area.
\item
  Find two positive numbers whose sum is 20 and whose product is as
  large as possible.
\item
  A cylinder is to be made from 100 cm of material. Find dimensions of
  maximum volume.
\item
  A wire 10 m long is cut into two pieces, one bent into a square, the
  other into a circle. How should it be cut to maximize total area
  enclosed?
\item
  A closed box with square base and volume 32 m is to be built. Find
  dimensions minimizing surface area.
\end{enumerate}

\subsection{3.4 Concavity and Inflection
Points}\label{34-concavity-and-inflection-points}

Derivatives not only tell us about slopes but also about the shape of a
graph. The second derivative is especially useful in understanding
concavity and identifying inflection points.

\subsubsection{Concavity}\label{concavity}

\begin{itemize}
\item
  A function is concave up on an interval if .\\
  The graph bends upward, like a cup.
\item
  A function is concave down on an interval if .\\
  The graph bends downward, like a frown.
\end{itemize}

Concavity describes how the slope of a function is changing: if slopes
are increasing, the graph is concave up; if slopes are decreasing, the
graph is concave down.

\subsubsection{Inflection Points}\label{inflection-points}

An inflection point is a point on the graph where concavity changes.

\begin{itemize}
\item
  If or is undefined, the point is a candidate for an inflection point.
\item
  To confirm, the concavity must change sign on either side of the
  point.
\end{itemize}

\subsubsection{Examples}\label{examples-8}

\begin{enumerate}
\def\labelenumi{\arabic{enumi}.}
\item
  \begin{itemize}
  \item
    .
  \item
    At , .
  \item
    For , → concave down.
  \item
    For , → concave up.
  \item
    Thus, is an inflection point.
  \end{itemize}
\item
  \begin{itemize}
  \item
    .
  \item
    At , , but concavity does not change sign (always ≥ 0).
  \item
    No inflection point.
  \end{itemize}
\end{enumerate}

\subsubsection{Concavity and Curve
Sketching}\label{concavity-and-curve-sketching}

\begin{itemize}
\item
  If and , then has a local minimum.
\item
  If and , then has a local maximum.
\item
  This is known as the second derivative test.
\end{itemize}

\subsubsection{Why This Matters}\label{why-this-matters-2}

Concavity and inflection points help us understand the ``shape'' of
graphs: where they bend, flatten, or turn. These ideas are central in
curve sketching, physics (acceleration), and economics (diminishing
returns).

\subsubsection{Exercises}\label{exercises-14}

\begin{enumerate}
\def\labelenumi{\arabic{enumi}.}
\item
  Determine intervals of concavity for . Find its inflection points.
\item
  For , identify concavity and possible inflection points.
\item
  Apply the second derivative test to to classify critical points.
\item
  Sketch , marking intervals of concavity and inflection points.
\item
  Explain why has no inflection points.
\end{enumerate}

\subsection{3.5 Curve Sketching}\label{35-curve-sketching}

Curve sketching is the process of drawing the graph of a function by
using information from its derivatives. Rather than plotting many
points, we analyze key features: intercepts, asymptotes,
increasing/decreasing intervals, and concavity.

\subsubsection{Steps for Curve
Sketching}\label{steps-for-curve-sketching}

\begin{enumerate}
\def\labelenumi{\arabic{enumi}.}
\item
  Domain: Identify where the function is defined.
\item
  Intercepts: Find where the graph crosses the axes.
\item
  Asymptotes:

  \begin{itemize}
  \item
    Vertical asymptotes occur where the function is undefined and tends
    to infinity.
  \item
    Horizontal or slant asymptotes describe end behavior as .
  \end{itemize}
\item
  First derivative :

  \begin{itemize}
  \item
    Positive → function is increasing.
  \item
    Negative → function is decreasing.
  \item
    Zeros of → critical points (possible maxima/minima).
  \end{itemize}
\item
  Second derivative :

  \begin{itemize}
  \item
    Positive → concave up.
  \item
    Negative → concave down.
  \item
    Zeros or undefined → possible inflection points.
  \end{itemize}
\item
  Combine information: Use all results to sketch a clear and accurate
  graph.
\end{enumerate}

\subsubsection{\texorpdfstring{Example 1:
}{Example 1: }}\label{example-1--f--x---x-3-ux2212-3-x}

\begin{itemize}
\item
  Domain: all real numbers.
\item
  Intercepts: at .
\item
  Derivative: .

  \begin{itemize}
  \item
    Increasing: .
  \item
    Decreasing: .
  \end{itemize}
\item
  Second derivative: .

  \begin{itemize}
  \item
    Concave down for , concave up for .
  \item
    Inflection point at .
  \end{itemize}
\item
  Shape: an S-curve with local max at , local min at .
\end{itemize}

\subsubsection{\texorpdfstring{Example 2:
}{Example 2: }}\label{example-2--f--x---1-x}

\begin{itemize}
\item
  Domain: .
\item
  Vertical asymptote: .
\item
  Horizontal asymptote: .
\item
  Derivative: (always negative). Function is always decreasing.
\item
  Second derivative: .

  \begin{itemize}
  \item
    Concave up for .
  \item
    Concave down for .
  \end{itemize}
\item
  Graph: hyperbola with two branches.
\end{itemize}

\subsubsection{Why Curve Sketching is
Useful}\label{why-curve-sketching-is-useful}

\begin{itemize}
\item
  Provides insight into overall behavior of functions without exhaustive
  computation.
\item
  Essential in calculus exams and applied problems.
\item
  Bridges algebraic analysis and geometric understanding.
\end{itemize}

\subsubsection{Exercises}\label{exercises-15}

\begin{enumerate}
\def\labelenumi{\arabic{enumi}.}
\item
  Sketch the curve of . Identify maxima, minima, and inflection points.
\item
  Analyze and sketch . Show intercepts, asymptotes, and concavity.
\item
  For , describe growth/decay, asymptotes, and concavity.
\item
  Sketch the graph of on the interval . Mark asymptotes.
\item
  Use the first and second derivative tests to classify critical points
  of .
\end{enumerate}

\section{\texorpdfstring{Part II. Integrals
}{Part II. Integrals }}\label{part-ii-integrals}

\section{\texorpdfstring{Chapter 4. Antiderivatives and Definite
Integrals
}{Chapter 4. Antiderivatives and Definite Integrals }}\label{chapter-4-antiderivatives-and-definite-integrals}

\subsection{4.1 Indefinite Integrals}\label{41-indefinite-integrals}

An indefinite integral is the reverse process of differentiation. If a
derivative measures change, then an integral recovers the original
function from its rate of change.

\subsubsection{Definition}\label{definition-4}

If , then

\[\int f(x)\,dx = F(x) + C,\]

where is the constant of integration.

Every indefinite integral represents a family of functions that differ
only by a constant, since differentiation eliminates constants.

\subsubsection{Basic Rules}\label{basic-rules}

\begin{enumerate}
\def\labelenumi{\arabic{enumi}.}
\item
  Constant Rule
\end{enumerate}

\[\int c\,dx = cx + C.\]

\begin{enumerate}
\def\labelenumi{\arabic{enumi}.}
\item
  Power Rule
\end{enumerate}

\[\int x^n\,dx = \frac{x^{n+1}}{n+1} + C, \quad n \neq -1.\]

\begin{enumerate}
\def\labelenumi{\arabic{enumi}.}
\item
  Sum Rule
\end{enumerate}

\[\int \big(f(x) + g(x)\big)\,dx = \int f(x)\,dx + \int g(x)\,dx.\]

\begin{enumerate}
\def\labelenumi{\arabic{enumi}.}
\item
  Constant Multiple Rule
\end{enumerate}

\[\int c f(x)\,dx = c \int f(x)\,dx.\]

\subsubsection{Common Integrals}\label{common-integrals}

\begin{itemize}
\item
\item
\item
\item
\end{itemize}

\subsubsection{Examples}\label{examples-9}

\begin{enumerate}
\def\labelenumi{\arabic{enumi}.}
\item
  .
\item
  .
\item
  .
\end{enumerate}

\subsubsection{Interpretation}\label{interpretation-2}

\begin{itemize}
\item
  Indefinite integrals are antiderivatives.
\item
  They are the foundation for definite integrals, which measure
  accumulated quantities like area, distance, and mass.
\item
  In applied contexts, integration allows us to move from rates back to
  totals.
\end{itemize}

\subsubsection{Exercises}\label{exercises-16}

\begin{enumerate}
\def\labelenumi{\arabic{enumi}.}
\item
  Find .
\item
  Compute .
\item
  Find the general solution of using integration.
\item
  Evaluate .
\item
  If velocity is , find the position function .
\end{enumerate}

\subsection{4.2 The Definite Integral as
Area}\label{42-the-definite-integral-as-area}

While indefinite integrals represent families of antiderivatives, the
definite integral gives a numerical value: the accumulated area under a
curve between two points.

\subsubsection{Definition}\label{definition-5}

For a function defined on , the definite integral is

\[\int_a^b f(x)\,dx = \lim_{n \to \infty} \sum_{i=1}^n f(x_i^-) \,\Delta x,\]

where the interval is divided into subintervals of width , and is a
sample point in each subinterval.

This is the limit of Riemann sums.

\subsubsection{Geometric Interpretation}\label{geometric-interpretation}

\begin{itemize}
\item
  If on , then equals the area under the curve from to .
\item
  If dips below the -axis, the integral computes signed area: regions
  below the axis count as negative.
\end{itemize}

\subsubsection{Properties of the Definite
Integral}\label{properties-of-the-definite-integral}

\begin{enumerate}
\def\labelenumi{\arabic{enumi}.}
\item
  Additivity over intervals
\end{enumerate}

\[\int_a^c f(x)\,dx = \int_a^b f(x)\,dx + \int_b^c f(x)\,dx.\]

\begin{enumerate}
\def\labelenumi{\arabic{enumi}.}
\item
  Reversing limits
\end{enumerate}

\[\int_a^b f(x)\,dx = -\int_b^a f(x)\,dx.\]

\begin{enumerate}
\def\labelenumi{\arabic{enumi}.}
\item
  Zero-width interval
\end{enumerate}

\[\int_a^a f(x)\,dx = 0.\]

\begin{enumerate}
\def\labelenumi{\arabic{enumi}.}
\item
  Linearity
\end{enumerate}

\[\int_a^b \big( cf(x) + g(x)\big)\,dx = c\int_a^b f(x)\,dx + \int_a^b g(x)\,dx.\]

\subsubsection{Examples}\label{examples-10}

\begin{enumerate}
\def\labelenumi{\arabic{enumi}.}
\item
  \strut \\
  This is the area of a right triangle under the line .
\item
  \strut \\
  The odd function has symmetric areas that cancel.
\item
  \strut \\
  This equals the area under one arch of the sine curve.
\end{enumerate}

\subsubsection{Why This Matters}\label{why-this-matters-3}

\begin{itemize}
\item
  Definite integrals measure accumulated quantities: distance, mass,
  energy, probability.
\item
  They bridge algebraic computation with geometric intuition.
\item
  The next step is the Fundamental Theorem of Calculus, which connects
  definite integrals with antiderivatives.
\end{itemize}

\subsubsection{Exercises}\label{exercises-17}

\begin{enumerate}
\def\labelenumi{\arabic{enumi}.}
\item
  Compute .
\item
  Find the area between and the -axis from to .
\item
  Evaluate .
\item
  Show that if is odd.
\item
  Approximate using a Riemann sum with subintervals and right endpoints.
\end{enumerate}

\subsection{4.3 The Fundamental Theorem of
Calculus}\label{43-the-fundamental-theorem-of-calculus}

The Fundamental Theorem of Calculus (FTC) unites the two main ideas of
calculus: differentiation and integration. It shows that finding areas
and finding rates of change are two sides of the same coin.

\subsubsection{Part 1: Differentiation of an
Integral}\label{part-1-differentiation-of-an-integral}

If is continuous on , define

\[F(x) = \int_a^x f(t)\,dt.\]

Then is differentiable, and

\[F'(x) = f(x).\]

In words: the derivative of the accumulated area function is the
original function itself.

\subsubsection{Part 2: Evaluation of Definite
Integrals}\label{part-2-evaluation-of-definite-integrals}

If is continuous on and is any antiderivative of , then

\[\int_a^b f(x)\,dx = F(b) - F(a).\]

This tells us we can evaluate definite integrals simply by finding an
antiderivative, rather than by computing limits of Riemann sums.

\subsubsection{Examples}\label{examples-11}

\begin{enumerate}
\def\labelenumi{\arabic{enumi}.}
\item
  .

  \begin{itemize}
  \item
    Antiderivative: .
  \item
    Apply FTC:
  \end{itemize}
\item
  If , then .
\item
  .

  \begin{itemize}
  \item
    Antiderivative: .
  \item
    Apply FTC:
  \end{itemize}
\end{enumerate}

\subsubsection{Why the FTC Matters}\label{why-the-ftc-matters}

\begin{itemize}
\item
  It transforms integration from a limit process into a practical
  computation.
\item
  It confirms that differentiation and integration are inverse
  operations.
\item
  It is the central theorem that makes calculus useful in mathematics,
  science, and engineering.
\end{itemize}

\subsubsection{Exercises}\label{exercises-18}

\begin{enumerate}
\def\labelenumi{\arabic{enumi}.}
\item
  Evaluate using the FTC.
\item
  If , find .
\item
  Compute .
\item
  Show that if , then .
\item
  Use the FTC to explain why the area under from to equals 1.
\end{enumerate}

\subsection{4.4 Properties of
Integrals}\label{44-properties-of-integrals}

The definite integral has several important properties that make it
flexible and powerful in applications. These properties follow from the
definition as a limit of sums and from the Fundamental Theorem of
Calculus.

\subsubsection{Linearity}\label{linearity}

For functions and , and constants :

\[\int_a^b \big(c f(x) + d g(x)\big)\,dx = c \int_a^b f(x)\,dx + d \int_a^b g(x)\,dx.\]

This allows us to break complicated integrals into simpler parts.

\subsubsection{Additivity over
Intervals}\label{additivity-over-intervals}

If , then

\[\int_a^b f(x)\,dx = \int_a^c f(x)\,dx + \int_c^b f(x)\,dx.\]

We can compute integrals piece by piece.

\subsubsection{Reversal of Limits}\label{reversal-of-limits}

\[\int_a^b f(x)\,dx = -\int_b^a f(x)\,dx.\]

Swapping the bounds changes the sign of the integral.

\subsubsection{Comparison Property}\label{comparison-property}

If for all in , then

\[\int_a^b f(x)\,dx \leq \int_a^b g(x)\,dx.\]

This lets us compare areas without direct computation.

\subsubsection{Absolute Value
Inequality}\label{absolute-value-inequality}

\[\left| \int_a^b f(x)\,dx \right| \leq \int_a^b |f(x)|\,dx.\]

This property is essential in analysis and convergence tests.

\subsubsection{Symmetry}\label{symmetry}

\begin{itemize}
\item
  If is even (symmetric about the -axis):

  \[\int_{-a}^a f(x)\,dx = 2\int_0^a f(x)\,dx.\]
\item
  If is odd (symmetric about the origin):

  \[\int_{-a}^a f(x)\,dx = 0.\]
\end{itemize}

\subsubsection{Examples}\label{examples-12}

\begin{enumerate}
\def\labelenumi{\arabic{enumi}.}
\item
\item
  Since is odd,
\item
  Since is even,
\end{enumerate}

\subsubsection{Why These Properties
Matter}\label{why-these-properties-matter}

\begin{itemize}
\item
  They simplify calculations.
\item
  They reveal geometric and symmetry features of functions.
\item
  They provide theoretical tools for more advanced analysis.
\end{itemize}

\subsubsection{Exercises}\label{exercises-19}

\begin{enumerate}
\def\labelenumi{\arabic{enumi}.}
\item
  Use symmetry to evaluate .
\item
  Show that .
\item
  Evaluate and compare with .
\item
  Prove that if on , then .
\item
  Compute using even/odd properties.
\end{enumerate}

\section{Chapter 5. Techniques of
Integration}\label{chapter-5-techniques-of-integration}

\subsection{5.1 Substitution}\label{51-substitution}

One of the most useful techniques of integration is the substitution
method, also called -u-substitution-. It is the reverse process of the
chain rule for derivatives.

\subsubsection{The Idea}\label{the-idea-2}

If an integral contains a composite function, we can simplify it by
changing variables.

Formally, if is a differentiable function, then

\[\int f(g(x)) g'(x)\,dx = \int f(u)\,du.\]

This substitution makes the integral easier to evaluate.

\subsubsection{Steps for Substitution}\label{steps-for-substitution}

\begin{enumerate}
\def\labelenumi{\arabic{enumi}.}
\item
  Identify an inner function whose derivative also appears in the
  integrand.
\item
  Compute .
\item
  Rewrite the integral in terms of .
\item
  Integrate with respect to .
\item
  Substitute back .
\end{enumerate}

\subsubsection{Examples}\label{examples-13}

\begin{enumerate}
\def\labelenumi{\arabic{enumi}.}
\item
  Simple substitution

  \[\int 2x \cos(x^2)\,dx\]

  Let , so .\\
  Then integral becomes .
\item
  Logarithmic case

  \[\int \frac{2x}{x^2+1}\,dx\]

  Let , so .\\
  Then integral becomes .
\item
  Trigonometric substitution

  \[\int \sin(3x)\,dx\]

  Let , so , hence .\\
  Integral becomes .
\end{enumerate}

\subsubsection{Definite Integrals with
Substitution}\label{definite-integrals-with-substitution}

When evaluating definite integrals, we must also change the limits:

\[\int_a^b f(g(x)) g'(x)\,dx = \int_{g(a)}^{g(b)} f(u)\,du.\]

Example:

\[\int_0^1 2x e^{x^2}\,dx.\]

Let , . Limits: when ; when .\\
So the integral becomes

\[\int_0^1 e^u\,du = e - 1.\]

\subsubsection{Exercises}\label{exercises-20}

\begin{enumerate}
\def\labelenumi{\arabic{enumi}.}
\item
  Evaluate .
\item
  Compute .
\item
  Evaluate using substitution.
\item
  Find .
\item
  Compute by letting .
\end{enumerate}

\subsection{5.2 Integration by Parts}\label{52-integration-by-parts}

Integration by parts is a technique that comes from the product rule for
derivatives. It helps evaluate integrals involving products of functions
that are not easily handled by substitution alone.

\subsubsection{The Formula}\label{the-formula}

From the product rule:

\[\frac{d}{dx}[u(x)v(x)] = u'(x)v(x) + u(x)v'(x).\]

Integrating both sides gives the integration by parts formula:

\[\int u\,dv = uv - \int v\,du.\]

Here:

\begin{itemize}
\item
  = a function chosen to be differentiated,
\item
  = the remaining part of the integrand to be integrated.
\end{itemize}

\subsubsection{\texorpdfstring{Choosing and
}{Choosing  and }}\label{choosing--u--and--d-v}

A common guideline is LIATE (Logarithmic, Inverse trig, Algebraic,
Trigonometric, Exponential).

\begin{itemize}
\item
  Choose from the earliest category present.
\item
  Choose as the rest.
\end{itemize}

\subsubsection{Examples}\label{examples-14}

\begin{enumerate}
\def\labelenumi{\arabic{enumi}.}
\item
  Polynomial × Exponential
\end{enumerate}

\[\int x e^x\,dx\]

Let , . Then , .

\[\int x e^x\,dx = x e^x - \int e^x dx = x e^x - e^x + C.\]

\begin{enumerate}
\def\labelenumi{\arabic{enumi}.}
\item
  Polynomial × Trig
\end{enumerate}

\[\int x \cos x\,dx\]

Let , . Then , .

\[\int x \cos x\,dx = x \sin x - \int \sin x dx = x \sin x + \cos x + C.\]

\begin{enumerate}
\def\labelenumi{\arabic{enumi}.}
\item
  Logarithm
\end{enumerate}

\[\int \ln x\,dx\]

Let , . Then , .

\[\int \ln x\,dx = x \ln x - \int 1 dx = x \ln x - x + C.\]

\subsubsection{Definite Integral
Example}\label{definite-integral-example}

\[\int_0^1 x e^x\,dx\]

Using the earlier result: .\\
Evaluate:

\[\big[(x-1)e^x\big]_0^1 = (0)e^1 - (-1)e^0 = 0 + 1 = 1.\]

\subsubsection{Why This Matters}\label{why-this-matters-4}

Integration by parts is crucial when substitution fails, especially with
logarithms, inverse trigonometric functions, and products involving
polynomials with exponentials or trig functions.

\subsubsection{Exercises}\label{exercises-21}

\begin{enumerate}
\def\labelenumi{\arabic{enumi}.}
\item
  Evaluate .
\item
  Find .
\item
  Compute .
\item
  Evaluate .
\item
  Use integration by parts to show .
\end{enumerate}

\subsection{5.3 Trigonometric Integrals and
Substitutions}\label{53-trigonometric-integrals-and-substitutions}

Many integrals involve trigonometric functions. These can often be
simplified using identities or by making special substitutions.

\subsubsection{Trigonometric Integrals}\label{trigonometric-integrals}

\begin{enumerate}
\def\labelenumi{\arabic{enumi}.}
\item
  Powers of sine and cosine
\end{enumerate}

\begin{itemize}
\item
  If the power of sine is odd: save one , convert the rest with , and
  substitute .
\item
  If the power of cosine is odd: save one , convert the rest with , and
  substitute .
\item
  If both are even: use half-angle identities.
\end{itemize}

Example:

\[\int \sin^3x \cos x \, dx\]

Let , :

\[\int u^3\,du = \tfrac{u^4}{4} + C = \tfrac{\sin^4x}{4} + C.\]

\begin{enumerate}
\def\labelenumi{\arabic{enumi}.}
\item
  Products of sine and cosine with different angles\\
  Use product-to-sum formulas:
\end{enumerate}

\[\sin A \cos B = \tfrac{1}{2}[\sin(A+B) + \sin(A-B)].\]

Example:

\[\int \sin(2x)\cos(3x)\,dx = \tfrac{1}{2}\int [\sin(5x) - \sin(x)]\,dx.\]

\begin{enumerate}
\def\labelenumi{\arabic{enumi}.}
\item
  Powers of secant and tangent
\end{enumerate}

\begin{itemize}
\item
  If the power of secant is even: save , convert the rest with , and
  substitute .
\item
  If the power of tangent is odd: save , convert the rest with , and
  substitute .
\end{itemize}

Example:

\[\int \tan^3x \sec^2x \, dx\]

Let , :

\[\int u^3\,du = \tfrac{u^4}{4} + C = \tfrac{\tan^4x}{4} + C.\]

\subsubsection{Trigonometric
Substitutions}\label{trigonometric-substitutions}

For integrals involving , , or , use special substitutions:

\begin{enumerate}
\def\labelenumi{\arabic{enumi}.}
\item
  , for .
\item
  , for .
\item
  , for .
\end{enumerate}

Example:

\[\int \sqrt{a^2 - x^2}\,dx\]

Let , so :

\[\int \sqrt{a^2 - a^2\sin^2\theta}(a\cos\theta\,d\theta) = \int a^2 \cos^2\theta \, d\theta.\]

Simplify using half-angle identities.

\subsubsection{Why These Techniques
Matter}\label{why-these-techniques-matter}

\begin{itemize}
\item
  They convert difficult algebraic forms into manageable trigonometric
  ones.
\item
  They are especially useful in problems involving areas, volumes, and
  arc lengths.
\item
  They lay groundwork for advanced integration methods.
\end{itemize}

\subsubsection{Exercises}\label{exercises-22}

\begin{enumerate}
\def\labelenumi{\arabic{enumi}.}
\item
  Evaluate .
\item
  Compute .
\item
  Evaluate .
\item
  Find using substitution.
\item
  Show that using .
\end{enumerate}

\subsection{5.4 Partial Fractions}\label{54-partial-fractions}

When integrating rational functions (ratios of polynomials), one
powerful method is partial fraction decomposition. This technique
expresses a complicated fraction as a sum of simpler fractions that are
easier to integrate.

\subsubsection{The Idea}\label{the-idea-3}

If is a rational function, where the degree of is less than the degree
of , we can decompose into simpler fractions.

These simpler pieces correspond to the factors of the denominator .

\subsubsection{Common Forms}\label{common-forms}

\begin{enumerate}
\def\labelenumi{\arabic{enumi}.}
\item
  Distinct linear factors\\
  If
\end{enumerate}

\[\frac{1}{(x-a)(x-b)},\]

then decompose as

\[\frac{A}{x-a} + \frac{B}{x-b}.\]

\begin{enumerate}
\def\labelenumi{\arabic{enumi}.}
\item
  Repeated linear factors\\
  If denominator has , then terms are
\end{enumerate}

\[\frac{A_1}{x-a} + \frac{A_2}{(x-a)^2} + \dots + \frac{A_n}{(x-a)^n}.\]

\begin{enumerate}
\def\labelenumi{\arabic{enumi}.}
\item
  Irreducible quadratic factors\\
  If denominator has , then numerator is linear:
\end{enumerate}

\[\frac{Ax+B}{x^2+bx+c}.\]

\subsubsection{Example 1: Distinct Linear
Factors}\label{example-1-distinct-linear-factors}

\[\int \frac{1}{x^2 - 1}\,dx\]

Factor denominator: .\\
Decompose:

\[\frac{1}{x^2-1} = \frac{1}{2}\left(\frac{1}{x-1} - \frac{1}{x+1}\right).\]

Integrate:

\[\int \frac{1}{x^2 - 1}\,dx = \tfrac{1}{2}\ln\left|\frac{x-1}{x+1}\right| + C.\]

\subsubsection{Example 2: Repeated Linear
Factor}\label{example-2-repeated-linear-factor}

\[\int \frac{1}{(x-1)^2}\,dx\]

This is already simple:

\[\int (x-1)^{-2}\,dx = -\frac{1}{x-1} + C.\]

\subsubsection{Example 3: Irreducible Quadratic
Factor}\label{example-3-irreducible-quadratic-factor}

\[\int \frac{x}{x^2+1}\,dx\]

Substitute , or recognize numerator is derivative of denominator.

\[\int \frac{x}{x^2+1}\,dx = \tfrac{1}{2}\ln(x^2+1) + C.\]

\subsubsection{Steps in Partial Fraction
Decomposition}\label{steps-in-partial-fraction-decomposition}

\begin{enumerate}
\def\labelenumi{\arabic{enumi}.}
\item
  Factor the denominator.
\item
  Write the general partial fraction form.
\item
  Multiply through by the denominator to clear fractions.
\item
  Solve for unknown constants.
\item
  Integrate each term.
\end{enumerate}

\subsubsection{Why This Matters}\label{why-this-matters-5}

\begin{itemize}
\item
  Converts complex rational functions into simple logarithmic or
  arctangent forms.
\item
  Especially useful in differential equations and Laplace transforms.
\item
  Fundamental in advanced calculus and engineering.
\end{itemize}

\subsubsection{Exercises}\label{exercises-23}

\begin{enumerate}
\def\labelenumi{\arabic{enumi}.}
\item
  Decompose and integrate .
\item
  Evaluate .
\item
  Compute .
\item
  Find .
\item
  Show that using partial fractions or substitution.
\end{enumerate}

\subsection{5.5 Improper Integrals}\label{55-improper-integrals}

Some integrals cannot be evaluated directly because the interval is
infinite or the integrand becomes unbounded. These are called improper
integrals. They are defined using limits.

\subsubsection{Definition}\label{definition-6}

\begin{enumerate}
\def\labelenumi{\arabic{enumi}.}
\item
  Infinite interval
\end{enumerate}

\[\int_a^\infty f(x)\,dx = \lim_{b \to \infty} \int_a^b f(x)\,dx.\]

\[\int_{-\infty}^a f(x)\,dx = \lim_{b \to -\infty} \int_b^a f(x)\,dx.\]

\begin{enumerate}
\def\labelenumi{\arabic{enumi}.}
\item
  Unbounded integrand\\
  If has a vertical asymptote at , then
\end{enumerate}

\[\int_a^c f(x)\,dx = \lim_{t \to c^-} \int_a^t f(x)\,dx,\]

\[\int_c^b f(x)\,dx = \lim_{t \to c^+} \int_t^b f(x)\,dx.\]

\subsubsection{Convergence and
Divergence}\label{convergence-and-divergence}

\begin{itemize}
\item
  If the limit exists and is finite, the improper integral converges.
\item
  If the limit does not exist or is infinite, the improper integral
  diverges.
\end{itemize}

\subsubsection{Examples}\label{examples-15}

\begin{enumerate}
\def\labelenumi{\arabic{enumi}.}
\item
  Exponential decay
\end{enumerate}

\[\int_1^\infty \frac{1}{x^2}\,dx = \lim_{b \to \infty} \Big[-\tfrac{1}{x}\Big]_1^b = 1.\]

This converges.

\begin{enumerate}
\def\labelenumi{\arabic{enumi}.}
\item
  Harmonic function
\end{enumerate}

\[\int_1^\infty \frac{1}{x}\,dx = \lim_{b \to \infty} \ln b.\]

This diverges to infinity.

\begin{enumerate}
\def\labelenumi{\arabic{enumi}.}
\item
  Asymptote at 0
\end{enumerate}

\[\int_0^1 \frac{1}{\sqrt{x}}\,dx = \lim_{t \to 0^+} \int_t^1 x^{-1/2}\,dx.\]

\[= \lim_{t \to 0^+} [2\sqrt{x}]_t^1 = 2.\]

This converges.

\begin{enumerate}
\def\labelenumi{\arabic{enumi}.}
\item
  Asymptote at 0 (divergent)
\end{enumerate}

\[\int_0^1 \frac{1}{x}\,dx = \lim_{t \to 0^+} \ln(1) - \ln(t).\]

This diverges since .

\subsubsection{Comparison Test for Improper
Integrals}\label{comparison-test-for-improper-integrals}

\begin{itemize}
\item
  If for large , and converges, then also converges.
\item
  If diverges and , then also diverges.
\end{itemize}

\subsubsection{Why Improper Integrals
Matter}\label{why-improper-integrals-matter}

\begin{itemize}
\item
  They extend integration to infinite domains and unbounded functions.
\item
  They are essential in probability (continuous distributions), physics
  (gravitational/electric fields), and Fourier analysis.
\end{itemize}

\subsubsection{Exercises}\label{exercises-24}

\begin{enumerate}
\def\labelenumi{\arabic{enumi}.}
\item
  Determine whether converges for various values of .
\item
  Evaluate .
\item
  Test convergence of depending on .
\item
  Compute .
\item
  Use the comparison test to show that converges.
\end{enumerate}

\section{\texorpdfstring{Chapter 6. Applications of Integration
}{Chapter 6. Applications of Integration }}\label{chapter-6-applications-of-integration}

\subsection{6.1 Areas and Volumes}\label{61-areas-and-volumes}

One of the most important applications of integration is finding areas
under curves and volumes of solids.

\subsubsection{Area Between Curves}\label{area-between-curves}

If on , then the area between the curves and is

\[A = \int_a^b \big(f(x) - g(x)\big)\,dx.\]

Example:\\
Find the area between and on .

\[A = \int_0^1 (x - x^2)\,dx = \left[\tfrac{1}{2}x^2 - \tfrac{1}{3}x^3\right]_0^1 = \tfrac{1}{6}.\]

\subsubsection{Volumes by Slicing}\label{volumes-by-slicing}

If a solid has cross-sectional area at position , then the volume is

\[V = \int_a^b A(x)\,dx.\]

\subsubsection{Volumes of Revolution}\label{volumes-of-revolution}

When a region is revolved around an axis, the resulting solid's volume
can be found with integration.

\begin{enumerate}
\def\labelenumi{\arabic{enumi}.}
\item
  Disk Method\\
  If the region under , , is revolved around the -axis:
\end{enumerate}

\[V = \pi \int_a^b [f(x)]^2\,dx.\]

\begin{enumerate}
\def\labelenumi{\arabic{enumi}.}
\item
  Washer Method\\
  If region between and is revolved around the -axis:
\end{enumerate}

\[V = \pi \int_a^b \Big([f(x)]^2 - [g(x)]^2\Big)\,dx.\]

\begin{enumerate}
\def\labelenumi{\arabic{enumi}.}
\item
  Shell Method\\
  If region under is revolved around the -axis:
\end{enumerate}

\[V = 2\pi \int_a^b x f(x)\,dx.\]

\subsubsection{Examples}\label{examples-16}

\begin{enumerate}
\def\labelenumi{\arabic{enumi}.}
\item
  Disk method\\
  Revolve , , around the -axis:
\end{enumerate}

\[V = \pi \int_0^4 (\sqrt{x})^2\,dx = \pi \int_0^4 x\,dx = \pi \left[\tfrac{1}{2}x^2\right]_0^4 = 8\pi.\]

\begin{enumerate}
\def\labelenumi{\arabic{enumi}.}
\item
  Washer method\\
  Revolve region between and , , around -axis:
\end{enumerate}

\[V = \pi \int_0^1 \big((\sqrt{x})^2 - (1)^2\big)\,dx = \pi \int_0^1 (x-1)\,dx = -\tfrac{\pi}{2}.\]

(Take absolute value for volume: ).

\begin{enumerate}
\def\labelenumi{\arabic{enumi}.}
\item
  Shell method\\
  Revolve region under , , around the -axis:
\end{enumerate}

\[V = 2\pi \int_0^1 x(x)\,dx = 2\pi \int_0^1 x^2\,dx = 2\pi \cdot \tfrac{1}{3} = \tfrac{2\pi}{3}.\]

\subsubsection{Why This Matters}\label{why-this-matters-6}

\begin{itemize}
\item
  Provides exact ways to compute areas and volumes in geometry.
\item
  Essential in physics, engineering, and probability.
\item
  Introduces geometric thinking with integration.
\end{itemize}

\subsubsection{Exercises}\label{exercises-25}

\begin{enumerate}
\def\labelenumi{\arabic{enumi}.}
\item
  Find the area between and on .
\item
  Compute the volume of the solid formed by revolving , , around the
  -axis.
\item
  Find the volume of the solid formed by revolving the region between
  and on around the -axis.
\item
  Use the washer method to compute the volume of the solid formed by
  revolving (a semicircle) around the -axis.
\item
  Find the area enclosed between and .
\end{enumerate}

\subsection{6.2 Arc Length and Surface
Area}\label{62-arc-length-and-surface-area}

Integration can also be used to measure the length of curves and the
surface area of solids generated by revolving curves.

\subsubsection{Arc Length}\label{arc-length}

For a smooth curve on the interval , the length of the curve is

\[L = \int_a^b \sqrt{1 + \big(f'(x)\big)^2}\,dx.\]

This comes from approximating the curve with line segments and taking
the limit.

Example:\\
Find the length of from to .

\begin{itemize}
\item
  Derivative: .
\item
  Formula:
\end{itemize}

\[L = \int_0^4 \sqrt{1 + \Big(\tfrac{3}{4}\sqrt{x}\Big)^2}\,dx 
= \int_0^4 \sqrt{1 + \tfrac{9}{16}x}\,dx.\]

This integral can be evaluated using substitution.

\subsubsection{Surface Area of
Revolution}\label{surface-area-of-revolution}

If a curve , , is revolved around the -axis, the surface area of the
resulting solid is

\[S = 2\pi \int_a^b f(x)\sqrt{1 + \big(f'(x)\big)^2}\,dx.\]

If revolved around the -axis:

\[S = 2\pi \int_a^b x \sqrt{1 + \big(f'(x)\big)^2}\,dx.\]

\subsubsection{Examples}\label{examples-17}

\begin{enumerate}
\def\labelenumi{\arabic{enumi}.}
\item
  Arc length of a line\\
  For , :
\end{enumerate}

\[L = \int_0^3 \sqrt{1+(1)^2}\,dx = \int_0^3 \sqrt{2}\,dx = 3\sqrt{2}.\]

\begin{enumerate}
\def\labelenumi{\arabic{enumi}.}
\item
  Surface area of a sphere\\
  Take , , and revolve around the -axis.
\end{enumerate}

\[S = 2\pi \int_{-r}^r \sqrt{r^2 - x^2}\sqrt{1+\left(\frac{-x}{\sqrt{r^2-x^2}}\right)^2}\,dx.\]

Simplification gives , the familiar formula for the surface area of a
sphere.

\subsubsection{Why This Matters}\label{why-this-matters-7}

\begin{itemize}
\item
  Arc length extends the idea of distance to curved paths.
\item
  Surface area of revolution has applications in physics, engineering,
  and design.
\item
  Provides a bridge between calculus and geometry.
\end{itemize}

\subsubsection{Exercises}\label{exercises-26}

\begin{enumerate}
\def\labelenumi{\arabic{enumi}.}
\item
  Find the arc length of from to .
\item
  Compute the surface area of the solid obtained by revolving , , around
  the -axis.
\item
  Find the arc length of from to .
\item
  Show that revolving from to around the -axis gives half the surface
  area of a sphere.
\item
  Derive the formula for the surface area of a cone by revolving a line.
\end{enumerate}

\subsection{6.3 Work and Averages}\label{63-work-and-averages}

Integration is not limited to geometry. It also helps calculate work
done by a force and the average value of a function over an interval.

\subsubsection{Work}\label{work}

If a variable force moves an object along a straight line from to , then
the total work is

\[W = \int_a^b F(x)\,dx.\]

This formula generalizes the simple case for constant force.

Example 1: Spring Force (Hooke's Law)\\
For a spring stretched from length to , with force :

\[W = \int_a^b kx\,dx = \tfrac{1}{2}k(b^2-a^2).\]

Example 2: Pumping Water\\
If water is pumped out of a tank, the work required equals

\[W = \int_a^b \text{(weight density)} \times \text{(cross-sectional area)} \times \text{(distance lifted)} \, dx.\]

\subsubsection{Average Value of a
Function}\label{average-value-of-a-function}

The average value of a continuous function on is

\[f_{\text{avg}} = \frac{1}{b-a} \int_a^b f(x)\,dx.\]

This is the continuous analog of averaging a list of numbers.

Example 1:\\
For on :

\[f_{\text{avg}} = \tfrac{1}{2-0}\int_0^2 x^2 dx = \tfrac{1}{2}\cdot \tfrac{8}{3} = \tfrac{4}{3}.\]

Example 2:\\
If the velocity of a particle is , then the average velocity over is

\[v_{\text{avg}} = \frac{1}{b-a}\int_a^b v(t)\,dt.\]

\subsubsection{Why This Matters}\label{why-this-matters-8}

\begin{itemize}
\item
  Work integrals appear in physics, engineering, and energy
  calculations.
\item
  Average value gives a single representative number for varying
  quantities.
\item
  Both connect calculus to real-world problems of motion, force, and
  efficiency.
\end{itemize}

\subsubsection{Exercises}\label{exercises-27}

\begin{enumerate}
\def\labelenumi{\arabic{enumi}.}
\item
  Compute the work required to stretch a spring from 2 m to 5 m if .
\item
  A 100 kg object is lifted vertically 5 m in a gravitational field ().
  Express the work as an integral and evaluate.
\item
  Find the average value of on .
\item
  Compute the average temperature if over a 24-hour day.
\item
  A tank of depth 10 m is full of water. Compute the work required to
  pump all the water to the top, given water weighs .
\end{enumerate}

\subsection{6.4 Probability Densities and Continuous
Distributions}\label{64-probability-densities-and-continuous-distributions}

Integration also plays a central role in probability theory, especially
for continuous random variables. Instead of discrete outcomes, we
describe probabilities with functions called probability density
functions (pdfs).

\subsubsection{Probability Density
Functions}\label{probability-density-functions}

A probability density function must satisfy two conditions:

\begin{enumerate}
\def\labelenumi{\arabic{enumi}.}
\item
  for all .
\item
  The total area under the curve is 1:

  \[\int_{-\infty}^\infty f(x)\,dx = 1.\]
\end{enumerate}

If is a continuous random variable with pdf , then the probability that
lies between and is

\[P(a \leq X \leq b) = \int_a^b f(x)\,dx.\]

\subsubsection{Cumulative Distribution
Function}\label{cumulative-distribution-function}

The cumulative distribution function (cdf) is defined as

\[F(x) = \int_{-\infty}^x f(t)\,dt.\]

It gives the probability that the random variable is less than or equal
to .

\subsubsection{Expected Value (Mean)}\label{expected-value-mean}

The expected value of a continuous random variable is the weighted
average:

\[E[X] = \int_{-\infty}^\infty x f(x)\,dx.\]

\subsubsection{Examples}\label{examples-18}

\begin{enumerate}
\def\labelenumi{\arabic{enumi}.}
\item
  Uniform Distribution\\
  For on :
\end{enumerate}

\begin{itemize}
\item
  Probability of interval :

  \[P(c \leq X \leq d) = \frac{d-c}{b-a}.\]
\item
  Expected value: .
\end{itemize}

\begin{enumerate}
\def\labelenumi{\arabic{enumi}.}
\item
  Exponential Distribution\\
  For , :
\end{enumerate}

\begin{itemize}
\item
  .
\item
  Mean: .
\end{itemize}

\begin{enumerate}
\def\labelenumi{\arabic{enumi}.}
\item
  Normal Distribution\\
  The bell curve:
\end{enumerate}

\[f(x) = \frac{1}{\sqrt{2\pi\sigma^2}} e^{-\frac{(x-\mu)^2}{2\sigma^2}}.\]

It integrates to 1, but requires advanced techniques.

\subsubsection{Why This Matters}\label{why-this-matters-9}

\begin{itemize}
\item
  Probability densities describe uncertainty in science, engineering,
  and statistics.
\item
  Integrals connect areas under curves to probabilities.
\item
  Continuous distributions generalize the idea of counting outcomes to
  measuring likelihoods over intervals.
\end{itemize}

\subsubsection{Exercises}\label{exercises-28}

\begin{enumerate}
\def\labelenumi{\arabic{enumi}.}
\item
  Show that the uniform density on integrates to 1.
\item
  For the exponential distribution with , compute .
\item
  Find the expected value of if on .
\item
  Verify that the normal distribution with mean 0 and variance 1 has
  total probability 1 (no need for full proof, but explain why it
  holds).
\item
  Compute the cdf of the uniform distribution on .
\end{enumerate}

\section{\texorpdfstring{Part III. Multivariable Calculus
}{Part III. Multivariable Calculus }}\label{part-iii-multivariable-calculus}

\section{Chapter 7. Vector Functions and
Curves}\label{chapter-7-vector-functions-and-curves}

\subsection{7.1 Vector Functions and Space
Curves}\label{71-vector-functions-and-space-curves}

In multivariable calculus, functions can output vectors instead of
numbers. These are called vector-valued functions, and they are
essential for describing curves in space.

\subsubsection{Definition}\label{definition-7}

A vector function is a function of the form

\[\mathbf{r}(t) = \langle x(t), y(t), z(t) \rangle,\]

where are real-valued functions.

\begin{itemize}
\item
  The input is often called the parameter.
\item
  The output is a vector in 2D or 3D space.
\item
  The graph of a vector function in 3D is a space curve.
\end{itemize}

\subsubsection{Examples}\label{examples-19}

\begin{enumerate}
\def\labelenumi{\arabic{enumi}.}
\item
  Line
\end{enumerate}

\[\mathbf{r}(t) = \langle 1+2t, \; 3-t, \; 4+5t \rangle.\]

This describes a straight line through the point with direction vector .

\begin{enumerate}
\def\labelenumi{\arabic{enumi}.}
\item
  Circle in the plane
\end{enumerate}

\[\mathbf{r}(t) = \langle \cos t, \; \sin t, \; 0 \rangle, \quad 0 \leq t < 2\pi.\]

\begin{enumerate}
\def\labelenumi{\arabic{enumi}.}
\item
  Helix
\end{enumerate}

\[\mathbf{r}(t) = \langle \cos t, \; \sin t, \; t \rangle.\]

This is a spiral rising around the -axis.

\subsubsection{Limits and Continuity}\label{limits-and-continuity}

A vector function is continuous at if each component is continuous at .

\[\lim_{t \to a} \mathbf{r}(t) = \langle \lim_{t \to a} x(t), \; \lim_{t \to a} y(t), \; \lim_{t \to a} z(t) \rangle.\]

\subsubsection{Geometry of Space Curves}\label{geometry-of-space-curves}

\begin{itemize}
\item
  Each curve has a tangent direction given by the derivative.
\item
  Space curves can model motion paths, particle trajectories, and
  geometric shapes.
\end{itemize}

\subsubsection{Why This Matters}\label{why-this-matters-10}

Vector functions are the foundation for multivariable calculus, allowing
us to extend the ideas of derivatives and integrals into higher
dimensions. They also appear naturally in physics (motion in 3D,
electromagnetism, fluid dynamics).

\subsubsection{Exercises}\label{exercises-29}

\begin{enumerate}
\def\labelenumi{\arabic{enumi}.}
\item
  Write a vector function for a line through parallel to the vector .
\item
  Describe the curve given by .
\item
  Determine whether is continuous at .
\item
  Sketch the helix .
\item
  Find the point on the curve when .
\end{enumerate}

\subsection{7.2 Derivatives and Integrals of Vector
Functions}\label{72-derivatives-and-integrals-of-vector-functions}

Vector functions can be differentiated and integrated just like ordinary
functions - we simply apply the operation to each component. This allows
us to study motion, velocity, acceleration, and accumulation in higher
dimensions.

\subsubsection{Derivative of a Vector
Function}\label{derivative-of-a-vector-function}

If

\[\mathbf{r}(t) = \langle x(t), y(t), z(t) \rangle,\]

then

\[\mathbf{r}'(t) = \langle x'(t), y'(t), z'(t) \rangle.\]

This derivative vector points in the tangent direction to the curve at
parameter .

\begin{itemize}
\item
  Velocity: If gives the position of a particle at time , then is its
  velocity vector.
\item
  Speed: The magnitude is the particle's speed.
\item
  Acceleration: .
\end{itemize}

\subsubsection{Examples}\label{examples-20}

\begin{enumerate}
\def\labelenumi{\arabic{enumi}.}
\item
  Helix
\end{enumerate}

\[\mathbf{r}(t) = \langle \cos t, \sin t, t \rangle.\]

\begin{itemize}
\item
  Velocity: .
\item
  Speed: .
\item
  Acceleration: .
\end{itemize}

\begin{enumerate}
\def\labelenumi{\arabic{enumi}.}
\item
  Projectile Motion
\end{enumerate}

\[\mathbf{r}(t) = \langle v_0 \cos\theta \cdot t, \; v_0 \sin\theta \cdot t - \tfrac{1}{2}gt^2 \rangle.\]

This models the parabolic path of a projectile under gravity.

\subsubsection{Integral of a Vector
Function}\label{integral-of-a-vector-function}

If

\[\mathbf{r}(t) = \langle x(t), y(t), z(t) \rangle,\]

then

\[\int \mathbf{r}(t)\,dt = \left\langle \int x(t)\,dt, \; \int y(t)\,dt, \; \int z(t)\,dt \right\rangle + \mathbf{C},\]

where is a constant vector.

\subsubsection{Example}\label{example}

\[\mathbf{r}(t) = \langle t, t^2, t^3 \rangle.\]

\begin{itemize}
\item
  Derivative: .
\item
  Integral:
\end{itemize}

\[\int \mathbf{r}(t)\,dt = \langle \tfrac{1}{2}t^2, \tfrac{1}{3}t^3, \tfrac{1}{4}t^4 \rangle + \mathbf{C}.\]

\subsubsection{Why This Matters}\label{why-this-matters-11}

\begin{itemize}
\item
  Derivatives of vector functions describe motion and forces in space.
\item
  Integrals give displacement, work, and accumulated quantities.
\item
  These tools connect calculus directly to physics and engineering.
\end{itemize}

\subsubsection{Exercises}\label{exercises-30}

\begin{enumerate}
\def\labelenumi{\arabic{enumi}.}
\item
  For , find velocity, speed, and acceleration.
\item
  Compute for .
\item
  Integrate .
\item
  A particle has velocity . Find its position vector if .
\item
  Show that the speed of is constant.
\end{enumerate}

\subsection{7.3 Arc Length and
Curvature}\label{73-arc-length-and-curvature}

Vector calculus provides tools to measure not only the path traced by a
curve but also how sharply it bends. These are expressed through arc
length and curvature.

\subsubsection{Arc Length of a Space
Curve}\label{arc-length-of-a-space-curve}

If a curve is given by

\[\mathbf{r}(t) = \langle x(t), y(t), z(t) \rangle, \quad a \leq t \leq b,\]

then the arc length is

\[L = \int_a^b |\mathbf{r}'(t)|\,dt,\]

where

\[|\mathbf{r}'(t)| = \sqrt{(x'(t))^2 + (y'(t))^2 + (z'(t))^2}.\]

Example:\\
For the helix :

\begin{itemize}
\item
  Velocity: .
\item
  Speed: .
\item
  Arc length:
\end{itemize}

\[L = \int_0^{2\pi} \sqrt{2}\,dt = 2\pi\sqrt{2}.\]

\subsubsection{Curvature}\label{curvature}

Curvature measures how quickly a curve changes direction.

For a smooth curve :

\[\kappa(t) = \frac{|\mathbf{r}'(t) \times \mathbf{r}''(t)|}{|\mathbf{r}'(t)|^3}.\]

\begin{itemize}
\item
  : straight line.
\item
  Larger : curve bends more sharply.
\end{itemize}

Example:\\
For a circle of radius :

\[\mathbf{r}(t) = \langle r\cos t, r\sin t \rangle.\]

Then .\\
So curvature is constant and inversely proportional to radius.

\subsubsection{Unit Tangent and Normal
Vectors}\label{unit-tangent-and-normal-vectors}

\begin{itemize}
\item
  Tangent vector:
\end{itemize}

\[\mathbf{T}(t) = \frac{\mathbf{r}'(t)}{|\mathbf{r}'(t)|}.\]

\begin{itemize}
\item
  Normal vector: points toward the center of curvature, defined as
\end{itemize}

\[\mathbf{N}(t) = \frac{\mathbf{T}'(t)}{|\mathbf{T}'(t)|}.\]

These vectors describe the geometry of motion: direction of travel and
direction of turning.

\subsubsection{Why This Matters}\label{why-this-matters-12}

\begin{itemize}
\item
  Arc length generalizes the concept of distance to curves in space.
\item
  Curvature describes bending, crucial in physics (centripetal
  acceleration), engineering (roads, roller coasters), and computer
  graphics.
\end{itemize}

\subsubsection{Exercises}\label{exercises-31}

\begin{enumerate}
\def\labelenumi{\arabic{enumi}.}
\item
  Find the arc length of from to .
\item
  Compute the curvature of the circle .
\item
  For , calculate .
\item
  Show that a straight line has curvature .
\item
  Find the tangent vector to at .
\end{enumerate}

\subsection{7.4 Motion in Space}\label{74-motion-in-space}

Vector functions are especially powerful in describing motion in two or
three dimensions. Position, velocity, and acceleration are naturally
expressed using derivatives and integrals of vector-valued functions.

\subsubsection{Position, Velocity, and
Acceleration}\label{position-velocity-and-acceleration}

\begin{itemize}
\item
  Position vector:
\end{itemize}

\[\mathbf{r}(t) = \langle x(t), y(t), z(t) \rangle\]

\begin{itemize}
\item
  Velocity vector (derivative of position):
\end{itemize}

\[\mathbf{v}(t) = \mathbf{r}'(t) = \langle x'(t), y'(t), z'(t) \rangle\]

\begin{itemize}
\item
  Speed (magnitude of velocity):
\end{itemize}

\[|\mathbf{v}(t)| = \sqrt{(x'(t))^2 + (y'(t))^2 + (z'(t))^2}\]

\begin{itemize}
\item
  Acceleration vector (derivative of velocity):
\end{itemize}

\[\mathbf{a}(t) = \mathbf{v}'(t) = \mathbf{r}''(t).\]

\subsubsection{Tangential and Normal
Components}\label{tangential-and-normal-components}

Acceleration can be decomposed into two components:

\[\mathbf{a}(t) = a_T \mathbf{T}(t) + a_N \mathbf{N}(t),\]

where:

\begin{itemize}
\item
  = unit tangent vector,
\item
  = principal normal vector,
\item
  = tangential acceleration (change in speed),
\item
  = normal acceleration (change in direction).
\end{itemize}

\subsubsection{Projectile Motion in 3D}\label{projectile-motion-in-3d}

With gravity acting in the direction:

\[\mathbf{r}(t) = \langle v_0 \cos\theta \cos\phi \cdot t,\; v_0 \cos\theta \sin\phi \cdot t,\; v_0 \sin\theta \cdot t - \tfrac{1}{2}gt^2 \rangle,\]

where is initial speed, launch angle, and azimuthal direction.

\subsubsection{Example: Helical Motion}\label{example-helical-motion}

\[\mathbf{r}(t) = \langle \cos t, \sin t, t \rangle\]

\begin{itemize}
\item
  Velocity: .
\item
  Speed: .
\item
  Acceleration: .
\item
  Motion is uniform in speed, spiraling upward.
\end{itemize}

\subsubsection{Why This Matters}\label{why-this-matters-13}

\begin{itemize}
\item
  Provides mathematical language for real-world motion.
\item
  Essential in physics (forces, trajectories, circular motion).
\item
  Foundation for advanced mechanics and engineering models.
\end{itemize}

\subsubsection{Exercises}\label{exercises-32}

\begin{enumerate}
\def\labelenumi{\arabic{enumi}.}
\item
  A particle moves along . Find velocity and acceleration at .
\item
  Show that speed is constant for the helix .
\item
  A projectile is launched with at angle . Write its position vector
  assuming motion in a vertical plane.
\item
  For , find and .
\item
  Decompose the acceleration vector into tangential and normal
  components for motion along a circle of radius .
\end{enumerate}

\section{\texorpdfstring{Chapter 8. Functions of Several Variables
}{Chapter 8. Functions of Several Variables }}\label{chapter-8-functions-of-several-variables}

\subsection{8.1 Limits and Continuity in Several
Variables}\label{81-limits-and-continuity-in-several-variables}

In multivariable calculus, functions may depend on two or more
variables, such as or . The concepts of limits and continuity extend
naturally from single-variable calculus, but they are more subtle
because we must consider all possible paths of approach.

\subsubsection{Limits in Two Variables}\label{limits-in-two-variables}

For a function , we say

\[\lim_{(x,y) \to (a,b)} f(x,y) = L\]

if gets arbitrarily close to as approaches along any path.

If different paths give different limit values, then the limit does not
exist.

Example 1 (limit exists):

\[f(x,y) = x^2 + y^2, \quad \lim_{(x,y) \to (0,0)} f(x,y) = 0.\]

Example 2 (limit does not exist):

\[f(x,y) = \frac{xy}{x^2+y^2}, \quad (x,y) \to (0,0).\]

\begin{itemize}
\item
  Along , the function is 0.
\item
  Along , the function is .\\
  Different results → limit does not exist.
\end{itemize}

\subsubsection{Continuity}\label{continuity}

A function is continuous at if

\[\lim_{(x,y)\to(a,b)} f(x,y) = f(a,b).\]

Polynomials and rational functions (where denominator ≠ 0) are
continuous everywhere in their domains.

\subsubsection{Extension to Three or More
Variables}\label{extension-to-three-or-more-variables}

For , limits and continuity are defined the same way, but the point must
be approached from infinitely many directions in space.

\subsubsection{Why This Matters}\label{why-this-matters-14}

\begin{itemize}
\item
  Continuity ensures no jumps, holes, or asymptotes in multivariable
  functions.
\item
  Limits are fundamental for defining partial derivatives and multiple
  integrals.
\item
  These concepts are building blocks for multivariable calculus.
\end{itemize}

\subsubsection{Exercises}\label{exercises-33}

\begin{enumerate}
\def\labelenumi{\arabic{enumi}.}
\item
  Determine whether exists.
\item
  Show that along all straight-line paths .
\item
  Does the limit exist for as ?
\item
  Explain why polynomials in two variables are continuous everywhere.
\item
  Give an example of a function of two variables that is discontinuous
  at a point, and explain why.
\end{enumerate}

\subsection{8.2 Partial Derivatives}\label{82-partial-derivatives}

In functions of several variables, we often want to measure how the
function changes when only one variable changes while the others are
held constant. This leads to the idea of partial derivatives.

\subsubsection{Definition}\label{definition-8}

For a function , the partial derivative with respect to at a point is

\[\frac{\partial f}{\partial x}(a,b) = \lim_{h \to 0} \frac{f(a+h, b) - f(a,b)}{h}.\]

Similarly, the partial derivative with respect to is

\[\frac{\partial f}{\partial y}(a,b) = \lim_{h \to 0} \frac{f(a, b+h) - f(a,b)}{h}.\]

We treat all other variables as constants when differentiating.

\subsubsection{Notation}\label{notation-2}

\begin{itemize}
\item
  , , .
\item
  , , .
\end{itemize}

For three variables , we also have .

\subsubsection{Examples}\label{examples-21}

\begin{enumerate}
\def\labelenumi{\arabic{enumi}.}
\item
\end{enumerate}

\begin{itemize}
\item
  .
\item
  .
\end{itemize}

\begin{enumerate}
\def\labelenumi{\arabic{enumi}.}
\item
\end{enumerate}

\begin{itemize}
\item
  .
\item
  .
\end{itemize}

\begin{enumerate}
\def\labelenumi{\arabic{enumi}.}
\item
\end{enumerate}

\begin{itemize}
\item
  .
\item
  .
\item
  .
\end{itemize}

\subsubsection{Higher-Order Partial
Derivatives}\label{higher-order-partial-derivatives}

We can take partial derivatives repeatedly:

\begin{itemize}
\item
  .
\item
  , etc.
\end{itemize}

Clairaut's Theorem: If has continuous second partial derivatives, then

\[f_{xy} = f_{yx}.\]

\subsubsection{Geometric Meaning}\label{geometric-meaning}

\begin{itemize}
\item
  : slope of the surface in the -direction.
\item
  : slope of the surface in the -direction.
\item
  Together they describe how the surface tilts.
\end{itemize}

\subsubsection{Why This Matters}\label{why-this-matters-15}

\begin{itemize}
\item
  Partial derivatives are the foundation of gradients, tangent planes,
  and optimization in multiple variables.
\item
  They are widely used in physics, engineering, and economics to model
  systems with several inputs.
\end{itemize}

\subsubsection{Exercises}\label{exercises-34}

\begin{enumerate}
\def\labelenumi{\arabic{enumi}.}
\item
  Find and for .
\item
  Compute for .
\item
  Verify Clairaut's theorem for .
\item
  Interpret geometrically what and mean for .
\item
  Find all second-order partial derivatives of .
\end{enumerate}

\subsection{8.3 Gradient and Directional
Derivatives}\label{83-gradient-and-directional-derivatives}

Partial derivatives measure change along the coordinate axes, but
sometimes we want to know the rate of change of a function in any
direction. This leads to the concepts of the gradient and directional
derivatives.

\subsubsection{Gradient Vector}\label{gradient-vector}

For a function , the gradient is the vector

\[\nabla f(x,y) = \left\langle \frac{\partial f}{\partial x}, \frac{\partial f}{\partial y} \right\rangle.\]

For three variables :

\[\nabla f(x,y,z) = \left\langle f_x, f_y, f_z \right\rangle.\]

The gradient points in the direction of maximum increase of the
function, and its magnitude gives the steepest slope.

\subsubsection{Directional Derivatives}\label{directional-derivatives}

The rate of change of at a point in the direction of a unit vector is

\[D_{\mathbf{u}} f(x,y) = \nabla f(x,y) \cdot \mathbf{u}.\]

This is the dot product of the gradient with the direction vector.

\subsubsection{Examples}\label{examples-22}

\begin{enumerate}
\def\labelenumi{\arabic{enumi}.}
\item
\end{enumerate}

\begin{itemize}
\item
  Gradient: .
\item
  At (1,2): .
\item
  Directional derivative along :
\end{itemize}

\[D_{\mathbf{u}} f(1,2) = \langle 2,4 \rangle \cdot \langle \tfrac{3}{5}, \tfrac{4}{5} \rangle = \tfrac{26}{5}.\]

\begin{enumerate}
\def\labelenumi{\arabic{enumi}.}
\item
\end{enumerate}

\begin{itemize}
\item
  Gradient: .
\item
  At (1,1,1): .
\item
  Maximum increase direction is along .
\end{itemize}

\subsubsection{Geometric
Interpretation}\label{geometric-interpretation-2}

\begin{itemize}
\item
  The gradient vector is perpendicular (normal) to level curves or level
  surfaces of .
\item
  Directional derivatives generalize slope in arbitrary directions.
\end{itemize}

\subsubsection{Why This Matters}\label{why-this-matters-16}

\begin{itemize}
\item
  In optimization, the gradient tells us the direction to move for
  steepest ascent or descent.
\item
  In physics, gradients describe fields like heat flow and electric
  potential.
\item
  Directional derivatives unify single-variable and multivariable rates
  of change.
\end{itemize}

\subsubsection{Exercises}\label{exercises-35}

\begin{enumerate}
\def\labelenumi{\arabic{enumi}.}
\item
  Compute for .
\item
  Find the gradient of and evaluate at (1,1,1).
\item
  Calculate the directional derivative of at (2,1) in the direction of .
\item
  Show that the gradient of is perpendicular to the circle .
\item
  Find the unit vector direction that maximizes the directional
  derivative of at (1,2).
\end{enumerate}

\subsection{8.4 Tangent Planes and Linear
Approximations}\label{84-tangent-planes-and-linear-approximations}

In single-variable calculus, the tangent line approximates a curve near
a point. In multivariable calculus, the analogous concept is the tangent
plane, which provides a linear approximation to a surface near a point.

\subsubsection{Tangent Plane to a
Surface}\label{tangent-plane-to-a-surface}

Suppose is differentiable at . The tangent plane at is given by

\[z = f(a,b) + f_x(a,b)(x-a) + f_y(a,b)(y-b).\]

This plane touches the surface at the point and approximates it nearby.

\subsubsection{Example 1: Paraboloid}\label{example-1-paraboloid}

For at :

\begin{itemize}
\item
  .
\item
  , so .
\item
  , so .
\end{itemize}

Equation of tangent plane:

\[z = 5 + 2(x-1) + 4(y-2).\]

\subsubsection{Linear Approximation}\label{linear-approximation}

The tangent plane can be used to approximate near :

\[f(x,y) \approx f(a,b) + f_x(a,b)(x-a) + f_y(a,b)(y-b).\]

This is the linearization of at .

\subsubsection{Example 2: Linear
Approximation}\label{example-2-linear-approximation}

Approximate near .

\begin{itemize}
\item
  .
\item
  .
\item
  At (4,5): .
\end{itemize}

So,

\[f(x,y) \approx 3 + \tfrac{1}{6}(x-4) + \tfrac{1}{6}(y-5).\]

\subsubsection{Why This Matters}\label{why-this-matters-17}

\begin{itemize}
\item
  Tangent planes give the best linear approximation to a surface.
\item
  Linearization simplifies complex functions for computation.
\item
  Widely used in numerical methods, physics, and economics.
\end{itemize}

\subsubsection{Exercises}\label{exercises-36}

\begin{enumerate}
\def\labelenumi{\arabic{enumi}.}
\item
  Find the tangent plane to at .
\item
  Approximate near .
\item
  Derive the tangent plane equation for at .
\item
  Use linear approximation to estimate using near (4,6).
\item
  Explain why the tangent plane approximation improves as gets closer to
  .
\end{enumerate}

\subsection{8.5 Optimization in Several
Variables}\label{85-optimization-in-several-variables}

Optimization in multivariable calculus extends the ideas of maxima and
minima from single-variable functions to functions of two or more
variables.

\subsubsection{Critical Points}\label{critical-points}

For , a critical point occurs where

\[f_x(x,y) = 0 \quad \text{and} \quad f_y(x,y) = 0,\]

or where the partial derivatives do not exist.

\subsubsection{Second Derivative Test}\label{second-derivative-test}

To classify critical points, compute the second partial derivatives:

\[D = f_{xx}(a,b) f_{yy}(a,b) - \big(f_{xy}(a,b)\big)^2.\]

\begin{itemize}
\item
  If and : local minimum.
\item
  If and : local maximum.
\item
  If : saddle point.
\item
  If : test is inconclusive.
\end{itemize}

\subsubsection{Example 1: Paraboloid}\label{example-1-paraboloid-2}

.

\begin{itemize}
\item
  . Critical point at (0,0).
\item
  .
\item
  , and .
\item
  So (0,0) is a local minimum.
\end{itemize}

\subsubsection{Example 2: Saddle Point}\label{example-2-saddle-point}

.

\begin{itemize}
\item
  . Critical point at (0,0).
\item
  .
\item
  .
\item
  So (0,0) is a saddle point.
\end{itemize}

\subsubsection{Constrained Optimization and Lagrange
Multipliers}\label{constrained-optimization-and-lagrange-multipliers}

Sometimes, we want to optimize subject to a constraint .

Method of Lagrange multipliers: solve

\[\nabla f(x,y) = \lambda \nabla g(x,y).\]

Example: Maximize subject to .

\begin{itemize}
\item
  Gradients: .
\item
  Equations: .
\item
  Solutions lead to max at .
\end{itemize}

\subsubsection{Why This Matters}\label{why-this-matters-18}

\begin{itemize}
\item
  Optimization is essential in economics, engineering, machine learning,
  and physics.
\item
  Lagrange multipliers allow optimization with constraints, a key tool
  in applied mathematics.
\end{itemize}

\subsubsection{Exercises}\label{exercises-37}

\begin{enumerate}
\def\labelenumi{\arabic{enumi}.}
\item
  Find and classify the critical points of .
\item
  Classify the point (0,0) for .
\item
  Use the second derivative test for .
\item
  Maximize subject to .
\item
  Minimize subject to .
\end{enumerate}

\section{\texorpdfstring{Chapter 9. Multiple Integrals
}{Chapter 9. Multiple Integrals }}\label{chapter-9-multiple-integrals}

\subsection{9.1 Double Integrals}\label{91-double-integrals}

In single-variable calculus, a definite integral gives the area under a
curve. In two variables, a double integral computes the volume under a
surface (or more generally, the accumulation of values over a region).

\subsubsection{Definition}\label{definition-9}

If is continuous on a region , the double integral is

\[\iint_R f(x,y)\, dA = \lim_{m,n \to \infty} \sum_{i=1}^m \sum_{j=1}^n f(x_{ij}^-, y_{ij}^-) \Delta A,\]

where is divided into small rectangles of area .

\subsubsection{Iterated Integrals}\label{iterated-integrals}

By Fubini's Theorem, we can compute a double integral as an iterated
integral:

\[\iint_R f(x,y)\, dA = \int_a^b \int_c^d f(x,y)\, dy\, dx,\]

if is a rectangle .

Order of integration can often be switched:

\[\int_a^b \int_c^d f(x,y)\,dy\,dx = \int_c^d \int_a^b f(x,y)\,dx\,dy.\]

\subsubsection{Examples}\label{examples-23}

\begin{enumerate}
\def\labelenumi{\arabic{enumi}.}
\item
  Rectangle region
\end{enumerate}

\[\iint_R (x+y)\, dA, \quad R=[0,1]\times[0,2].\]

\[= \int_0^1 \int_0^2 (x+y)\,dy\,dx = \int_0^1 \Big[xy+\tfrac{1}{2}y^2\Big]_0^2 dx
= \int_0^1 (2x+2)dx = 3.\]

\begin{enumerate}
\def\labelenumi{\arabic{enumi}.}
\item
  Triangular region
\end{enumerate}

\[R = \{(x,y): 0 \leq x \leq 1, 0 \leq y \leq x\}.\]

\[\iint_R (x+y)\, dA = \int_0^1 \int_0^x (x+y)\,dy\,dx.\]

Evaluating gives .

\subsubsection{Applications}\label{applications-2}

\begin{itemize}
\item
  Volume under a surface:
\end{itemize}

\[V = \iint_R f(x,y)\, dA.\]

\begin{itemize}
\item
  Average value of a function over a region:
\end{itemize}

\[f_{\text{avg}} = \frac{1}{A(R)} \iint_R f(x,y)\, dA.\]

\subsubsection{Why This Matters}\label{why-this-matters-19}

Double integrals extend integration to two dimensions. They are
essential in physics (mass, probability distributions), economics
(expected values), and engineering (centroids, flux).

\subsubsection{Exercises}\label{exercises-38}

\begin{enumerate}
\def\labelenumi{\arabic{enumi}.}
\item
  Evaluate where .
\item
  Compute where .
\item
  Find the average value of over the unit square .
\item
  Interpret in terms of probability if is a probability density
  function.
\item
  Show that switching order of integration gives the same result for .
\end{enumerate}

\subsection{9.2 Triple Integrals}\label{92-triple-integrals}

Triple integrals extend the idea of integration to three variables,
allowing us to compute volumes, masses, and other quantities in
three-dimensional regions.

\subsubsection{Definition}\label{definition-10}

If is continuous on a solid region , the triple integral is

\[\iiint_E f(x,y,z)\, dV = \lim_{m,n,p \to \infty} \sum f(x_{ijk}^-, y_{ijk}^-, z_{ijk}^-) \Delta V,\]

where the region is subdivided into boxes of volume .

\subsubsection{Iterated Integrals}\label{iterated-integrals-2}

By Fubini's Theorem, a triple integral can be computed as an iterated
integral:

\[\iiint_E f(x,y,z)\, dV = \int_a^b \int_c^d \int_e^f f(x,y,z)\, dz\, dy\, dx,\]

for a rectangular box .

The order of integration can be chosen for convenience.

\subsubsection{Examples}\label{examples-24}

\begin{enumerate}
\def\labelenumi{\arabic{enumi}.}
\item
  Rectangular box
\end{enumerate}

\[\iiint_E xyz\, dV, \quad E=[0,1]\times[0,2]\times[0,3].\]

\[= \int_0^1 \int_0^2 \int_0^3 xyz\,dz\,dy\,dx.\]

First integrate over :

\[\int_0^3 xyz\,dz = xy \left[\tfrac{1}{2}z^2\right]_0^3 = \tfrac{9}{2}xy.\]

Now integrate over :

\[\int_0^2 \tfrac{9}{2}xy\,dy = \tfrac{9}{2}x \cdot \left[\tfrac{1}{2}y^2\right]_0^2 = 9x.\]

Finally integrate over :

\[\int_0^1 9x\,dx = \tfrac{9}{2}.\]

\begin{enumerate}
\def\labelenumi{\arabic{enumi}.}
\item
  Region bounded by planes\\
  Let .
\end{enumerate}

\[\iiint_E 1\,dV = \int_0^1 \int_0^x \int_0^y 1\,dz\,dy\,dx.\]

Evaluate:

\[= \int_0^1 \int_0^x y\,dy\,dx = \int_0^1 \tfrac{1}{2}x^2\,dx = \tfrac{1}{6}.\]

So the volume of this triangular region is .

\subsubsection{Applications}\label{applications-3}

\begin{itemize}
\item
  Volume: .
\item
  Mass: If density is , then

  \[M = \iiint_E \rho(x,y,z)\, dV.\]
\item
  Average value:

  \[f_{\text{avg}} = \frac{1}{V(E)} \iiint_E f(x,y,z)\,dV.\]
\end{itemize}

\subsubsection{Why This Matters}\label{why-this-matters-20}

Triple integrals generalize area and volume calculations to arbitrary
solids. They are used in physics (mass distributions, center of mass,
gravitational fields), engineering, and probability.

\subsubsection{Exercises}\label{exercises-39}

\begin{enumerate}
\def\labelenumi{\arabic{enumi}.}
\item
  Compute over the cube .
\item
  Find the volume of the tetrahedron bounded by .
\item
  Evaluate where .
\item
  Show that equals the geometric volume of .
\item
  If density is , compute the mass of the unit cube.
\end{enumerate}

\subsection{9.3 Applications: Volume, Mass,
Probability}\label{93-applications-volume-mass-probability}

Triple integrals are powerful because they allow us to compute
quantities in three dimensions by accumulating values over a solid
region.

\subsubsection{Volume}\label{volume}

The simplest application is finding the volume of a region :

\[V = \iiint_E 1 \, dV.\]

Example:\\
Find the volume of the solid bounded by the coordinate planes and the
plane .

\[V = \iiint_E 1 \, dV = \int_0^1 \int_0^{1-x} \int_0^{1-x-y} 1 \, dz\, dy\, dx.\]

Evaluating gives .

\subsubsection{Mass and Density}\label{mass-and-density}

If a solid has density function , its mass is

\[M = \iiint_E \rho(x,y,z)\, dV.\]

The center of mass is given by

\[\bar{x} = \frac{1}{M}\iiint_E x\rho(x,y,z)\,dV, \quad 
\bar{y} = \frac{1}{M}\iiint_E y\rho(x,y,z)\,dV, \quad 
\bar{z} = \frac{1}{M}\iiint_E z\rho(x,y,z)\,dV.\]

Example:\\
For a unit cube with constant density , the center of mass is at .

\subsubsection{Probability}\label{probability}

If is a probability density function in 3D, then the probability that
the random variable lies in a region is

\[P(E) = \iiint_E f(x,y,z)\, dV,\]

where and

\[\iiint_{\mathbb{R}^3} f(x,y,z)\,dV = 1.\]

Example:\\
If for , uniformly in , then

\[P(0 \leq z \leq 0.5) = \int_0^{0.5} \tfrac{3}{4}z^2 \, dz = \tfrac{1}{32}.\]

\subsubsection{Why This Matters}\label{why-this-matters-21}

\begin{itemize}
\item
  Volumes generalize geometry to irregular solids.
\item
  Mass and density integrals connect calculus to physics and
  engineering.
\item
  Probability density functions in higher dimensions are widely used in
  statistics and data science.
\end{itemize}

\subsubsection{Exercises}\label{exercises-40}

\begin{enumerate}
\def\labelenumi{\arabic{enumi}.}
\item
  Find the volume of the solid bounded by (the unit sphere).
\item
  Compute the mass of a cone with density proportional to .
\item
  Find the center of mass of a uniform tetrahedron bounded by .
\item
  If on the cube , verify that it is a probability density function.
\item
  Use a triple integral to compute the probability that a randomly
  chosen point in the unit sphere has .
\end{enumerate}

\subsection{9.4 Change of Variables: Polar, Cylindrical, Spherical
Coordinates}\label{94-change-of-variables-polar-cylindrical-spherical-coordinates}

Many integrals become easier when expressed in coordinate systems that
match the symmetry of the region. Instead of Cartesian coordinates , we
can use polar, cylindrical, or spherical coordinates.

\subsubsection{Polar Coordinates (2D)}\label{polar-coordinates-2d}

For functions of two variables, we can switch to polar coordinates:

\[x = r\cos\theta, \quad y = r\sin\theta, \quad r \geq 0, \; 0 \leq \theta < 2\pi.\]

The area element transforms as

\[dA = r\,dr\,d\theta.\]

Example:\\
Find the area of the unit circle.

\[A = \iint_{x^2+y^2\leq 1} 1\,dA = \int_0^{2\pi}\int_0^1 r\,dr\,d\theta = \pi.\]

\subsubsection{Cylindrical Coordinates
(3D)}\label{cylindrical-coordinates-3d}

In 3D, cylindrical coordinates extend polar coordinates with :

\[x = r\cos\theta, \quad y = r\sin\theta, \quad z = z.\]

The volume element is

\[dV = r\,dr\,d\theta\,dz.\]

Example:\\
Volume of a cylinder of radius and height :

\[V = \int_0^h \int_0^{2\pi} \int_0^R r\,dr\,d\theta\,dz = \pi R^2 h.\]

\subsubsection{Spherical Coordinates
(3D)}\label{spherical-coordinates-3d}

For spherical symmetry, use:

\[x = \rho \sin\phi \cos\theta, \quad y = \rho \sin\phi \sin\theta, \quad z = \rho \cos\phi,\]

where

\begin{itemize}
\item
  is the distance from the origin,
\item
  is the angle from the positive -axis,
\item
  is the angle in the -plane.
\end{itemize}

The volume element is

\[dV = \rho^2 \sin\phi \, d\rho\, d\phi\, d\theta.\]

Example:\\
Volume of the unit sphere:

\[V = \int_0^{2\pi} \int_0^\pi \int_0^1 \rho^2 \sin\phi \, d\rho\, d\phi\, d\theta.\]

Evaluating:

\[V = \left(\int_0^1 \rho^2 d\rho\right)\left(\int_0^\pi \sin\phi d\phi\right)\left(\int_0^{2\pi} d\theta\right) = \tfrac{1}{3}(2)(2\pi) = \tfrac{4\pi}{3}.\]

\subsubsection{Why This Matters}\label{why-this-matters-22}

\begin{itemize}
\item
  Polar coordinates simplify circular regions.
\item
  Cylindrical coordinates handle cylinders and rotational symmetry.
\item
  Spherical coordinates simplify spheres, cones, and radial problems.
\item
  These changes of variables make otherwise impossible integrals
  manageable.
\end{itemize}

\subsubsection{Exercises}\label{exercises-41}

\begin{enumerate}
\def\labelenumi{\arabic{enumi}.}
\item
  Compute using polar coordinates.
\item
  Find the volume of a cone of height and radius using cylindrical
  coordinates.
\item
  Use spherical coordinates to evaluate the volume of a ball of radius .
\item
  Show that the Jacobian factor for polar coordinates is .
\item
  Find the mass of a solid sphere of radius with density proportional to
  distance from the origin using spherical coordinates.
\end{enumerate}

\section{\texorpdfstring{Chapter 10. Vector Calculus
}{Chapter 10. Vector Calculus }}\label{chapter-10-vector-calculus}

\subsection{10.1 Vector Fields}\label{101-vector-fields}

A vector field assigns a vector to each point in space, much like a
scalar function assigns a number. Vector fields are used to model flows,
forces, and other directional quantities.

\subsubsection{Definition}\label{definition-11}

In two dimensions, a vector field is a function

\[\mathbf{F}(x,y) = \langle P(x,y), Q(x,y) \rangle,\]

where and are scalar functions.

In three dimensions,

\[\mathbf{F}(x,y,z) = \langle P(x,y,z), Q(x,y,z), R(x,y,z) \rangle.\]

\subsubsection{Examples}\label{examples-25}

\begin{enumerate}
\def\labelenumi{\arabic{enumi}.}
\item
  Radial field
\end{enumerate}

\[\mathbf{F}(x,y) = \langle x, y \rangle.\]

Vectors point outward from the origin.

\begin{enumerate}
\def\labelenumi{\arabic{enumi}.}
\item
  Rotational field
\end{enumerate}

\[\mathbf{F}(x,y) = \langle -y, x \rangle.\]

Vectors circulate around the origin.

\begin{enumerate}
\def\labelenumi{\arabic{enumi}.}
\item
  Gravitational field
\end{enumerate}

\[\mathbf{F}(x,y,z) = -\frac{GM}{r^3}\langle x,y,z \rangle, \quad r=\sqrt{x^2+y^2+z^2}.\]

\subsubsection{Visualizing Vector
Fields}\label{visualizing-vector-fields}

\begin{itemize}
\item
  Draw small arrows at sample points to indicate direction and
  magnitude.
\item
  Denser arrows where magnitudes are larger.
\item
  Useful for interpreting flow lines, trajectories, and forces.
\end{itemize}

\subsubsection{Flow Lines}\label{flow-lines}

A flow line (or integral curve) of a vector field is a curve whose
tangent vector at each point matches the field:

\[\mathbf{r}'(t) = \mathbf{F}(\mathbf{r}(t)).\]

Flow lines describe particle paths in a velocity field.

\subsubsection{Why This Matters}\label{why-this-matters-23}

\begin{itemize}
\item
  Vector fields are fundamental in physics (fluid flow,
  electromagnetism, gravitation).
\item
  They form the basis of line integrals, surface integrals, and the big
  theorems of vector calculus (Green, Stokes, Divergence).
\item
  Provide a geometric way to represent directional quantities.
\end{itemize}

\subsubsection{Exercises}\label{exercises-42}

\begin{enumerate}
\def\labelenumi{\arabic{enumi}.}
\item
  Sketch the vector field .
\item
  Determine whether the vectors of point toward or away from the origin.
\item
  For , compute .
\item
  Describe the flow lines of .
\item
  Explain why gravitational and electric fields are examples of radial
  vector fields.
\end{enumerate}

\subsection{10.2 Line Integrals}\label{102-line-integrals}

A line integral extends the idea of an integral to functions evaluated
along a curve. Instead of integrating over an interval or region, we
integrate over a path in space.

\subsubsection{Definition: Scalar Line
Integral}\label{definition-scalar-line-integral}

If is a scalar function and is a curve parameterized by , then the line
integral is

\[\int_C f(x,y)\, ds = \int_a^b f(x(t),y(t)) \, |\mathbf{r}'(t)|\, dt,\]

where is arc length.

This measures the accumulation of along the curve.

\subsubsection{Definition: Vector Line
Integral}\label{definition-vector-line-integral}

For a vector field , the line integral along is

\[\int_C \mathbf{F} \cdot d\mathbf{r} = \int_a^b \mathbf{F}(\mathbf{r}(t)) \cdot \mathbf{r}'(t)\, dt.\]

This measures the work done by the field along the curve.

\subsubsection{Examples}\label{examples-26}

\begin{enumerate}
\def\labelenumi{\arabic{enumi}.}
\item
  Scalar Line Integral
\end{enumerate}

\[f(x,y) = x+y, \quad C: \mathbf{r}(t) = \langle t, t^2 \rangle, \; 0 \leq t \leq 1.\]

Then

\[\int_C f(x,y)\, ds = \int_0^1 (t+t^2)\sqrt{(1)^2+(2t)^2}\, dt.\]

\begin{enumerate}
\def\labelenumi{\arabic{enumi}.}
\item
  Work Done by a Force
\end{enumerate}

\[\mathbf{F}(x,y) = \langle y, x \rangle, \quad C: \mathbf{r}(t) = \langle t, t^2 \rangle, \; 0 \leq t \leq 1.\]

\[\int_C \mathbf{F} \cdot d\mathbf{r} = \int_0^1 \langle t^2, t \rangle \cdot \langle 1, 2t \rangle\, dt = \int_0^1 (t^2 + 2t^2)\, dt = \int_0^1 3t^2\, dt = 1.\]

\subsubsection{Physical Interpretation}\label{physical-interpretation}

\begin{itemize}
\item
  Scalar line integral: accumulation of density along a wire.
\item
  Vector line integral: work done by a force moving an object along a
  path.
\end{itemize}

\subsubsection{Why This Matters}\label{why-this-matters-24}

\begin{itemize}
\item
  Line integrals connect vector fields with physical quantities like
  work and circulation.
\item
  They are building blocks for Green's Theorem and Stokes' Theorem.
\item
  Appear in physics (electric potential, fluid flow, mechanics).
\end{itemize}

\subsubsection{Exercises}\label{exercises-43}

\begin{enumerate}
\def\labelenumi{\arabic{enumi}.}
\item
  Compute where is the line segment from (0,0) to (1,1).
\item
  Evaluate for along the unit circle .
\item
  Interpret the meaning of .
\item
  For , compute the line integral along .
\item
  Explain the difference between scalar and vector line integrals.
\end{enumerate}

\subsection{10.3 Surface Integrals}\label{103-surface-integrals}

A surface integral generalizes line integrals to two-dimensional
surfaces in three-dimensional space. They allow us to compute flux
through surfaces and accumulation of scalar fields over curved surfaces.

\subsubsection{Scalar Surface Integral}\label{scalar-surface-integral}

If a surface is parameterized by

\[\mathbf{r}(u,v) = \langle x(u,v), y(u,v), z(u,v) \rangle,\]

then the surface integral of a scalar function is

\[\iint_S f(x,y,z)\, dS = \iint_D f(\mathbf{r}(u,v)) \, |\mathbf{r}_u \times \mathbf{r}_v| \, du\,dv,\]

where and are partial derivatives of , and is the parameter domain.

\subsubsection{Vector Surface Integral
(Flux)}\label{vector-surface-integral-flux}

For a vector field , the flux through a surface is

\[\iint_S \mathbf{F}\cdot d\mathbf{S} = \iint_S \mathbf{F}\cdot \mathbf{n}\, dS,\]

where is the unit normal vector. Using parameterization,

\[\iint_S \mathbf{F}\cdot d\mathbf{S} = \iint_D \mathbf{F}(\mathbf{r}(u,v)) \cdot (\mathbf{r}_u \times \mathbf{r}_v)\,du\,dv.\]

\subsubsection{Examples}\label{examples-27}

\begin{enumerate}
\def\labelenumi{\arabic{enumi}.}
\item
  Scalar Surface Integral\\
  Surface: plane over unit disk .
\end{enumerate}

\[\iint_S 1\, dS = \text{area of the disk} = \pi.\]

\begin{enumerate}
\def\labelenumi{\arabic{enumi}.}
\item
  Flux Through a Sphere\\
  Let , and = sphere of radius .\\
  Normal vector is .
\end{enumerate}

\[\mathbf{F}\cdot \mathbf{n} = \frac{x^2+y^2+z^2}{R} = R.\]

So

\[\iint_S \mathbf{F}\cdot d\mathbf{S} = \iint_S R\, dS = R \cdot 4\pi R^2 = 4\pi R^3.\]

\subsubsection{Why This Matters}\label{why-this-matters-25}

\begin{itemize}
\item
  Scalar surface integrals measure area and surface distributions.
\item
  Vector surface integrals measure flux: the amount of a field passing
  through a surface.
\item
  Applications: electromagnetism, fluid flow, heat transfer, and more.
\end{itemize}

\subsubsection{Exercises}\label{exercises-44}

\begin{enumerate}
\def\labelenumi{\arabic{enumi}.}
\item
  Compute for the surface of a cube of side length 2.
\item
  Find the flux of through the unit sphere.
\item
  Evaluate for the paraboloid .
\item
  For , compute flux through the plane , .
\item
  Explain physically what it means if the flux of a vector field through
  a closed surface is zero.
\end{enumerate}

\subsection{10.4 Green's Theorem}\label{104-greens-theorem}

Green's Theorem is a fundamental result in vector calculus that connects
a line integral around a closed curve to a double integral over the
region it encloses. It is a two-dimensional version of Stokes' Theorem.

\subsubsection{Statement of Green's
Theorem}\label{statement-of-greens-theorem}

Let be a positively oriented, simple, closed curve in the plane, and let
be the region it encloses. If has continuous partial derivatives on an
open region containing , then

\[\oint_C \mathbf{F} \cdot d\mathbf{r} = \oint_C P\,dx + Q\,dy = \iint_R \left( \frac{\partial Q}{\partial x} - \frac{\partial P}{\partial y} \right)\, dA.\]

\subsubsection{Interpretation}\label{interpretation-3}

\begin{itemize}
\item
  The line integral around measures the circulation of the vector field
  along the boundary.
\item
  The double integral over measures the total curl (rotation) of the
  field inside the region.
\end{itemize}

\subsubsection{Example 1: Area Formula}\label{example-1-area-formula}

If , then

\[\frac{\partial Q}{\partial x} - \frac{\partial P}{\partial y} = 1.\]

Thus, Green's Theorem gives

\[\text{Area}(R) = \iint_R 1\,dA = \oint_C \left(-\tfrac{y}{2}\,dx + \tfrac{x}{2}\,dy\right).\]

This provides a way to compute area using a line integral.

\subsubsection{Example 2: Circulation}\label{example-2-circulation}

Let , and be the unit circle.

\begin{itemize}
\item
  .
\item
  .
\item
  Double integral over the unit disk:
\end{itemize}

\[\iint_R 2\,dA = 2\pi (1^2) = 2\pi.\]

So the circulation around the circle is .

\subsubsection{Why This Matters}\label{why-this-matters-26}

\begin{itemize}
\item
  Converts difficult line integrals into double integrals, or vice
  versa.
\item
  Provides a bridge between local properties (curl) and global
  properties (circulation).
\item
  Widely used in physics for fluid flow, electromagnetism, and planar
  vector fields.
\end{itemize}

\subsubsection{Exercises}\label{exercises-45}

\begin{enumerate}
\def\labelenumi{\arabic{enumi}.}
\item
  Use Green's Theorem to compute the area inside the ellipse .
\item
  Verify Green's Theorem for along the square with vertices (0,0),
  (1,0), (1,1), (0,1).
\item
  Compute the circulation of around the unit circle.
\item
  Show that if , then the line integral of around any closed curve is
  zero.
\item
  Use Green's Theorem to show that
\end{enumerate}

\[\oint_C x\,dy = -\oint_C y\,dx\]

for any closed curve .

\subsection{10.5 Stokes' Theorem}\label{105-stokes-theorem}

Stokes' Theorem generalizes Green's Theorem to three dimensions. It
relates a surface integral of the curl of a vector field over a surface
to a line integral of the vector field around the boundary of that
surface.

\subsubsection{Statement of Stokes'
Theorem}\label{statement-of-stokes-theorem}

Let be an oriented, smooth surface with boundary curve (positively
oriented). If is a vector field with continuous partial derivatives,
then

\[\iint_S (\nabla \times \mathbf{F}) \cdot d\mathbf{S} = \oint_C \mathbf{F} \cdot d\mathbf{r}.\]

\begin{itemize}
\item
  Left side: flux of the curl of through the surface.
\item
  Right side: circulation of along the boundary curve.
\end{itemize}

\subsubsection{Interpretation}\label{interpretation-4}

\begin{itemize}
\item
  The line integral around the boundary equals the total "rotation"
  inside the surface.
\item
  Extends Green's Theorem (a special case when the surface lies in the
  plane).
\end{itemize}

\subsubsection{Example 1: Green's Theorem as a Special
Case}\label{example-1-greens-theorem-as-a-special-case}

If is a flat region in the -plane, Stokes' Theorem reduces to Green's
Theorem.

\subsubsection{Example 2: Circulation on a
Hemisphere}\label{example-2-circulation-on-a-hemisphere}

Let , and be the upper hemisphere of radius 1.

\begin{itemize}
\item
  Boundary : unit circle in the -plane.
\item
  By Stokes' Theorem:
\end{itemize}

\[\oint_C \mathbf{F}\cdot d\mathbf{r} = \iint_S (\nabla \times \mathbf{F})\cdot d\mathbf{S}.\]

\begin{itemize}
\item
  Curl: .
\item
  Normal to hemisphere points outward: .
\item
  So integrand = 2.
\item
  Area of hemisphere = .
\end{itemize}

\[\iint_S 2\, dS = 2 \cdot 2\pi = 4\pi.\]

Thus, circulation around the equator is .

\subsubsection{Why This Matters}\label{why-this-matters-27}

\begin{itemize}
\item
  Provides a deep connection between surface integrals and line
  integrals.
\item
  Simplifies calculations by allowing choice of convenient surfaces.
\item
  Widely used in electromagnetism (Faraday's Law) and fluid dynamics.
\end{itemize}

\subsubsection{Exercises}\label{exercises-46}

\begin{enumerate}
\def\labelenumi{\arabic{enumi}.}
\item
  Verify Stokes' Theorem for over the unit disk in the -plane.
\item
  Compute where , and is the boundary of the triangle with vertices
  (0,0,0), (1,0,0), (0,1,0).
\item
  Show that if , then the circulation around any closed curve is zero.
\item
  Apply Stokes' Theorem to compute circulation of around the boundary of
  the unit square in the plane .
\item
  Explain how Stokes' Theorem generalizes Green's Theorem.
\end{enumerate}

\subsection{10.6 Divergence Theorem}\label{106-divergence-theorem}

The Divergence Theorem (also called Gauss's Theorem) relates the flux of
a vector field through a closed surface to the triple integral of the
divergence of the field inside the surface.

\subsubsection{Statement of the Divergence
Theorem}\label{statement-of-the-divergence-theorem}

Let be a solid region in with boundary surface (oriented outward). If is
a vector field with continuous partial derivatives on , then

\[\iint_S \mathbf{F} \cdot d\mathbf{S} = \iiint_E (\nabla \cdot \mathbf{F}) \, dV.\]

\begin{itemize}
\item
  Left side: flux of across the closed surface .
\item
  Right side: triple integral of the divergence inside the region.
\end{itemize}

\subsubsection{Divergence}\label{divergence}

The divergence of a vector field is

\[\nabla \cdot \mathbf{F} = \frac{\partial P}{\partial x} + \frac{\partial Q}{\partial y} + \frac{\partial R}{\partial z}.\]

It measures the ``net outflow'' per unit volume at each point.

\subsubsection{Example 1: Flux of a Radial
Field}\label{example-1-flux-of-a-radial-field}

Let , and let be the unit ball .

\begin{itemize}
\item
  Divergence: .
\item
  Volume of unit ball: .\\
  So
\end{itemize}

\[\iiint_E (\nabla \cdot \mathbf{F})\, dV = 3 \cdot \tfrac{4}{3}\pi = 4\pi.\]

Thus, the flux across the unit sphere is .

\subsubsection{Example 2: Constant
Field}\label{example-2-constant-field}

Let .

\begin{itemize}
\item
  Divergence: .
\item
  So flux through any closed surface is zero, consistent with intuition
  (no net outflow).
\end{itemize}

\subsubsection{Why This Matters}\label{why-this-matters-28}

\begin{itemize}
\item
  Converts surface integrals into simpler volume integrals.
\item
  Used in physics: Gauss's Law in electromagnetism, fluid flow, and heat
  transfer.
\item
  Completes the unifying framework:

  \begin{itemize}
  \item
    Green's Theorem (2D curl ↔ circulation)
  \item
    Stokes' Theorem (3D curl ↔ circulation on surfaces)
  \item
    Divergence Theorem (3D divergence ↔ flux on closed surfaces)
  \end{itemize}
\end{itemize}

\subsubsection{Exercises}\label{exercises-47}

\begin{enumerate}
\def\labelenumi{\arabic{enumi}.}
\item
  Use the Divergence Theorem to compute the flux of across the surface
  of a sphere of radius .
\item
  Verify the Divergence Theorem for on the unit cube .
\item
  Show that if , then the total flux through any closed surface is zero.
\item
  Compute the flux of through the unit sphere.
\item
  Explain how the Divergence Theorem generalizes the one-dimensional
  Fundamental Theorem of Calculus.
\end{enumerate}

\section{Part IV. Infinite Processes}\label{part-iv-infinite-processes}

\section{\texorpdfstring{Chapter 11. Sequences and convergence
}{Chapter 11. Sequences and convergence }}\label{chapter-11-sequences-and-convergence}

\subsection{11.1 Definitions and
Examples}\label{111-definitions-and-examples}

A sequence is an ordered list of numbers, usually written as

\[a_1, a_2, a_3, \dots\]

or more generally . Each is called the nth term of the sequence.

\subsubsection{Defining a Sequence}\label{defining-a-sequence}

A sequence can be defined in two ways:

\begin{enumerate}
\def\labelenumi{\arabic{enumi}.}
\item
  Explicit formula -- gives a direct rule for the nth term.

  \begin{itemize}
  \item
    Example: defines the sequence

    \[1, \tfrac{1}{2}, \tfrac{1}{3}, \tfrac{1}{4}, \dots\]
  \end{itemize}
\item
  Recursive definition -- defines terms using earlier terms.

  \begin{itemize}
  \item
    Example: Fibonacci sequence:

    \[a_1 = 1, \quad a_2 = 1, \quad a_{n} = a_{n-1} + a_{n-2} \quad (n \geq 3).\]
  \end{itemize}
\end{enumerate}

\subsubsection{Examples of Sequences}\label{examples-of-sequences}

\begin{enumerate}
\def\labelenumi{\arabic{enumi}.}
\item
  Arithmetic sequence:

  \[a_n = a_1 + (n-1)d.\]

  Example: → sequence of odd numbers.
\item
  Geometric sequence:

  \[a_n = a_1 r^{n-1}.\]

  Example: → powers of 2.
\item
  Harmonic sequence:

  \[a_n = \frac{1}{n}.\]
\item
  Alternating sequence:

  \[a_n = (-1)^n.\]
\end{enumerate}

\subsubsection{Sequences in Calculus}\label{sequences-in-calculus}

Sequences are the foundation for infinite processes:

\begin{itemize}
\item
  Limits of sequences → define convergence.
\item
  Series → infinite sums built from sequences.
\item
  Functions approximated by sequences and series.
\end{itemize}

\subsubsection{Why This Matters}\label{why-this-matters-29}

\begin{itemize}
\item
  Sequences provide the building blocks for infinite series and
  approximations.
\item
  They allow us to rigorously define "approaching infinity" and
  convergence.
\item
  Many important functions (exponential, trigonometric) can be expressed
  through sequences and series.
\end{itemize}

\subsubsection{Exercises}\label{exercises-48}

\begin{enumerate}
\def\labelenumi{\arabic{enumi}.}
\item
  Write the first five terms of the sequence .
\item
  Determine if is bounded.
\item
  Give a recursive definition for the sequence .
\item
  Find the 10th term of the arithmetic sequence with and .
\item
  Write an explicit formula for the sequence defined by , .
\end{enumerate}

\subsection{11.2 Monotone and Bounded
Sequences}\label{112-monotone-and-bounded-sequences}

To understand whether a sequence converges, we need to study its
behavior: does it increase, decrease, stay within bounds, or grow
without limit? Two important concepts are monotonicity and boundedness.

\subsubsection{Monotone Sequences}\label{monotone-sequences}

A sequence is called monotone if it is always increasing or always
decreasing.

\begin{itemize}
\item
  Monotone increasing:

  \[a_{n+1} \geq a_n \quad \text{for all } n.\]
\item
  Monotone decreasing:

  \[a_{n+1} \leq a_n \quad \text{for all } n.\]
\end{itemize}

Examples:

\begin{enumerate}
\def\labelenumi{\arabic{enumi}.}
\item
  is monotone increasing.
\item
  is monotone decreasing.
\end{enumerate}

\subsubsection{Bounded Sequences}\label{bounded-sequences}

A sequence is bounded above if there exists a number such that for all
.\\
It is bounded below if there exists such that for all .

If both conditions hold, the sequence is bounded.

Examples:

\begin{enumerate}
\def\labelenumi{\arabic{enumi}.}
\item
  is bounded between 0 and 1.
\item
  is bounded between -1 and 1.
\item
  is not bounded.
\end{enumerate}

\subsubsection{Monotone Convergence
Theorem}\label{monotone-convergence-theorem}

A fundamental result in analysis:

\begin{itemize}
\item
  Every monotone increasing sequence that is bounded above converges.
\item
  Every monotone decreasing sequence that is bounded below converges.
\end{itemize}

This theorem guarantees convergence without finding the limit
explicitly.

\subsubsection{Example}\label{example-2}

\begin{enumerate}
\def\labelenumi{\arabic{enumi}.}
\item
  Sequence: .

  \begin{itemize}
  \item
    Increasing: since .
  \item
    Bounded above by 1.
  \item
    Therefore, it converges.
  \item
    Limit: .
  \end{itemize}
\end{enumerate}

\subsubsection{Why This Matters}\label{why-this-matters-30}

\begin{itemize}
\item
  Monotonicity and boundedness give quick tests for convergence.
\item
  They are essential in proofs and in constructing limits rigorously.
\item
  These ideas extend naturally to functions and series.
\end{itemize}

\subsubsection{Exercises}\label{exercises-49}

\begin{enumerate}
\def\labelenumi{\arabic{enumi}.}
\item
  Determine whether is monotone and bounded.
\item
  Show that is monotone increasing but not bounded.
\item
  Prove that converges, and find its limit.
\item
  Give an example of a bounded sequence that is not monotone.
\item
  Apply the monotone convergence theorem to .
\end{enumerate}

\subsection{11.3 Limits of Sequences}\label{113-limits-of-sequences}

The central question about a sequence is whether its terms approach a
single value as grows. This leads to the concept of the limit of a
sequence.

\subsubsection{Definition}\label{definition-12}

A sequence has a limit if, for every , there exists an integer such that

\[|a_n - L| < \varepsilon \quad \text{whenever } n > N.\]

We then write

\[\lim_{n\to\infty} a_n = L.\]

If no such exists, the sequence diverges.

\subsubsection{Intuition}\label{intuition}

\begin{itemize}
\item
  The terms of the sequence get arbitrarily close to as becomes large.
\item
  Beyond some index , all terms stay within a tiny band around .
\end{itemize}

\subsubsection{Examples}\label{examples-28}

\begin{enumerate}
\def\labelenumi{\arabic{enumi}.}
\item
  .\\
  As grows, terms shrink toward 0.

  \[\lim_{n\to\infty} \frac{1}{n} = 0.\]
\item
  .\\
  Terms alternate between -1 and 1, so no single limit exists. The
  sequence diverges.
\item
  .\\
  As , numerator and denominator are nearly equal, so

  \[\lim_{n\to\infty} \frac{n}{n+1} = 1.\]
\end{enumerate}

\subsubsection{Properties of Limits}\label{properties-of-limits}

If and :

\begin{itemize}
\item
  .
\item
  .
\item
  for constant .
\item
  If and , then

  \[\lim \frac{a_n}{b_n} = \frac{A}{B}.\]
\end{itemize}

\subsubsection{Theorem: Squeeze
Principle}\label{theorem-squeeze-principle}

If for all large , and

\[\lim_{n\to\infty} a_n = \lim_{n\to\infty} c_n = L,\]

then

\[\lim_{n\to\infty} b_n = L.\]

Example:

\[a_n = -\tfrac{1}{n}, \quad b_n = \tfrac{\sin n}{n}, \quad c_n = \tfrac{1}{n}.\]

Since and both bounding sequences go to 0,

\[\lim_{n\to\infty} \frac{\sin n}{n} = 0.\]

\subsubsection{Why This Matters}\label{why-this-matters-31}

\begin{itemize}
\item
  Limits make rigorous the idea of sequences ``approaching'' a value.
\item
  Convergence of sequences underpins infinite series and continuity.
\item
  These concepts are essential in defining real numbers via limits.
\end{itemize}

\subsubsection{Exercises}\label{exercises-50}

\begin{enumerate}
\def\labelenumi{\arabic{enumi}.}
\item
  Find .
\item
  Determine if converges.
\item
  Does converge? Why or why not?
\item
  Use the Squeeze Principle to show .
\item
  Prove that if , then .
\end{enumerate}

\section{Chapter 12. Infinite series}\label{chapter-12-infinite-series}

\subsection{12.1 Series and
Convergence}\label{121-series-and-convergence}

A series is the sum of the terms of a sequence. Instead of just listing
numbers, we add them together and study whether the infinite sum
approaches a finite value.

\subsubsection{Definition}\label{definition-13}

Given a sequence , the corresponding series is

\[\sum_{n=1}^\infty a_n = a_1 + a_2 + a_3 + \dots\]

We define the nth partial sum as

\[S_n = \sum_{k=1}^n a_k.\]

If the sequence converges to a finite limit , then the series converges
and

\[\sum_{n=1}^\infty a_n = S.\]

If diverges, then the series diverges.

\subsubsection{Examples}\label{examples-29}

\begin{enumerate}
\def\labelenumi{\arabic{enumi}.}
\item
  Geometric series
\end{enumerate}

\[\sum_{n=0}^\infty ar^n = \frac{a}{1-r}, \quad |r| < 1.\]

Example:

\[1 + \tfrac{1}{2} + \tfrac{1}{4} + \tfrac{1}{8} + \dots = 2.\]

\begin{enumerate}
\def\labelenumi{\arabic{enumi}.}
\item
  Harmonic series
\end{enumerate}

\[\sum_{n=1}^\infty \frac{1}{n}.\]

This series diverges, even though the terms go to 0.

\begin{enumerate}
\def\labelenumi{\arabic{enumi}.}
\item
  p-series
\end{enumerate}

\[\sum_{n=1}^\infty \frac{1}{n^p}.\]

\begin{itemize}
\item
  Converges if .
\item
  Diverges if .
\end{itemize}

\subsubsection{Necessary Condition for
Convergence}\label{necessary-condition-for-convergence}

If converges, then necessarily

\[\lim_{n\to\infty} a_n = 0.\]

If , the series diverges.\\
But the converse is not true: does not guarantee convergence (e.g.,
harmonic series).

\subsubsection{Why This Matters}\label{why-this-matters-32}

\begin{itemize}
\item
  Series extend finite addition to infinite processes.
\item
  Convergent series are used to approximate functions, compute areas,
  and model physical processes.
\item
  The study of series leads to powerful convergence tests.
\end{itemize}

\subsubsection{Exercises}\label{exercises-51}

\begin{enumerate}
\def\labelenumi{\arabic{enumi}.}
\item
  Determine whether converges, and find its sum.
\item
  Show that converges.
\item
  Does converge?
\item
  Write the first four partial sums of the series .
\item
  Explain why is necessary but not sufficient for convergence.
\end{enumerate}

\subsection{12.2 Convergence Tests}\label{122-convergence-tests}

Since many series cannot be summed directly, mathematicians developed
tests to decide whether a series converges or diverges. These tests are
tools for analyzing infinite sums.

\subsubsection{1. The nth-Term Test for
Divergence}\label{1-the-nth-term-test-for-divergence}

If

\[\lim_{n\to\infty} a_n \neq 0 \quad \text{or does not exist},\]

then

\[\sum a_n\]

diverges.

If , the test is inconclusive.

\subsubsection{2. Comparison Test}\label{2-comparison-test}

Suppose for all .

\begin{itemize}
\item
  If converges, then also converges.
\item
  If diverges, then also diverges.
\end{itemize}

\subsubsection{3. Limit Comparison Test}\label{3-limit-comparison-test}

If and

\[\lim_{n\to\infty} \frac{a_n}{b_n} = c,\]

where , then and either both converge or both diverge.

\subsubsection{4. Ratio Test}\label{4-ratio-test}

For , compute

\[L = \lim_{n\to\infty} \left| \frac{a_{n+1}}{a_n} \right|.\]

\begin{itemize}
\item
  If , the series converges absolutely.
\item
  If or , the series diverges.
\item
  If , the test is inconclusive.
\end{itemize}

\subsubsection{5. Root Test}\label{5-root-test}

For , compute

\[L = \lim_{n\to\infty} \sqrt[n]{|a_n|}.\]

\begin{itemize}
\item
  If , the series converges absolutely.
\item
  If , the series diverges.
\item
  If , the test is inconclusive.
\end{itemize}

\subsubsection{6. Alternating Series Test (Leibniz's
Test)}\label{6-alternating-series-test-leibnizs-test}

For series of the form

\[\sum (-1)^n b_n \quad \text{or} \quad \sum (-1)^{n+1} b_n,\]

if

\begin{enumerate}
\def\labelenumi{\arabic{enumi}.}
\item
  (decreasing), and
\item
  ,
\end{enumerate}

then the series converges.

\subsubsection{Examples}\label{examples-30}

\begin{enumerate}
\def\labelenumi{\arabic{enumi}.}
\item
  : Comparison Test → converges.
\item
  : Harmonic series → diverges.
\item
  : Alternating series test → converges.
\item
  : Ratio Test → converges.
\item
  : Root Test → diverges.
\end{enumerate}

\subsubsection{Why This Matters}\label{why-this-matters-33}

\begin{itemize}
\item
  Convergence tests let us classify series without needing explicit
  sums.
\item
  They provide systematic ways to handle infinite processes in calculus.
\item
  They are critical for later topics like power series and Fourier
  series.
\end{itemize}

\subsubsection{Exercises}\label{exercises-52}

\begin{enumerate}
\def\labelenumi{\arabic{enumi}.}
\item
  Test convergence of .
\item
  Use the ratio test for .
\item
  Apply the root test to .
\item
  Determine convergence of .
\item
  Use the limit comparison test with to test .
\end{enumerate}

\subsection{12.3 Absolute vs Conditional
Convergence}\label{123-absolute-vs-conditional-convergence}

Not all series behave the same way when signs alternate. To handle this,
we distinguish between absolute convergence and conditional convergence.

\subsubsection{Absolute Convergence}\label{absolute-convergence}

A series is absolutely convergent if

\[\sum |a_n|\]

converges.

Theorem: If a series converges absolutely, then it also converges.

Example:

\[\sum \frac{(-1)^n}{n^2}.\]

Here converges (p-series, ).\\
So the series is absolutely convergent.

\subsubsection{Conditional Convergence}\label{conditional-convergence}

A series is conditionally convergent if it converges, but not
absolutely.

Example:

\[\sum \frac{(-1)^n}{n}.\]

\begin{itemize}
\item
  Alternating series test → converges.
\item
  But diverges (harmonic series).\\
  So the series is conditionally convergent.
\end{itemize}

\subsubsection{Rearrangement Theorem}\label{rearrangement-theorem}

For conditionally convergent series, rearranging the terms can change
the sum - even make it diverge or converge to a different value.

This surprising result shows the delicate nature of conditional
convergence.

\subsubsection{Why This Matters}\label{why-this-matters-34}

\begin{itemize}
\item
  Absolute convergence is stronger and guarantees stability.
\item
  Conditional convergence highlights the importance of order in infinite
  sums.
\item
  Many alternating series encountered in practice are only conditionally
  convergent.
\end{itemize}

\subsubsection{Exercises}\label{exercises-53}

\begin{enumerate}
\def\labelenumi{\arabic{enumi}.}
\item
  Show that converges absolutely.
\item
  Show that is conditionally convergent.
\item
  Test for absolute and conditional convergence.
\item
  Explain why absolute convergence implies convergence, but the converse
  is not true.
\item
  Research and summarize the Riemann rearrangement theorem in your own
  words.
\end{enumerate}

\section{\texorpdfstring{Chapter 13. Power Series and Expansions
}{Chapter 13. Power Series and Expansions }}\label{chapter-13-power-series-and-expansions}

\subsection{13.1 Power Series}\label{131-power-series}

A power series is an infinite series in which each term involves a power
of the variable. Power series are central in calculus because they let
us represent functions as infinite polynomials.

\subsubsection{General Form}\label{general-form}

A power series centered at has the form

\[\sum_{n=0}^\infty c_n (x-a)^n,\]

where are constants called the coefficients.

\begin{itemize}
\item
  If , the series is centered at the origin:

  \[\sum_{n=0}^\infty c_n x^n.\]
\end{itemize}

\subsubsection{Examples}\label{examples-31}

\begin{enumerate}
\def\labelenumi{\arabic{enumi}.}
\item
  Geometric series
\end{enumerate}

\[\sum_{n=0}^\infty x^n = \frac{1}{1-x}, \quad |x|<1.\]

\begin{enumerate}
\def\labelenumi{\arabic{enumi}.}
\item
  Exponential function
\end{enumerate}

\[e^x = \sum_{n=0}^\infty \frac{x^n}{n!}.\]

\begin{enumerate}
\def\labelenumi{\arabic{enumi}.}
\item
  Sine and cosine
\end{enumerate}

\[\sin x = \sum_{n=0}^\infty (-1)^n \frac{x^{2n+1}}{(2n+1)!}, \quad  
\cos x = \sum_{n=0}^\infty (-1)^n \frac{x^{2n}}{(2n)!}.\]

\subsubsection{Interval of Convergence}\label{interval-of-convergence}

For each power series, there exists a radius of convergence such that:

\begin{itemize}
\item
  The series converges if .
\item
  The series diverges if .
\item
  At , convergence must be checked separately.
\end{itemize}

\subsubsection{Why This Matters}\label{why-this-matters-35}

\begin{itemize}
\item
  Power series allow us to approximate functions by polynomials.
\item
  They connect calculus with analysis and differential equations.
\item
  Many special functions in mathematics and physics are defined by their
  power series.
\end{itemize}

\subsubsection{Exercises}\label{exercises-54}

\begin{enumerate}
\def\labelenumi{\arabic{enumi}.}
\item
  Write the power series for .
\item
  Find the first four terms of the power series for .
\item
  Express as a power series centered at 0.
\item
  Determine whether the series converges at .
\item
  Explain why power series are sometimes called ``infinite
  polynomials.''
\end{enumerate}

\subsection{13.2 Radius of Convergence}\label{132-radius-of-convergence}

Every power series converges for some values of and diverges for others.
The boundary between these two behaviors is described by the radius of
convergence.

\subsubsection{Definition}\label{definition-14}

For a power series

\[\sum_{n=0}^\infty c_n (x-a)^n,\]

there exists a number (possibly infinite) such that:

\begin{itemize}
\item
  The series converges absolutely if .
\item
  The series diverges if .
\item
  At , convergence must be checked separately.
\end{itemize}

This number is called the radius of convergence.

\subsubsection{Finding the Radius of
Convergence}\label{finding-the-radius-of-convergence}

Two common methods:

\begin{enumerate}
\def\labelenumi{\arabic{enumi}.}
\item
  Ratio Test
\end{enumerate}

\[R = \lim_{n\to\infty} \left| \frac{c_n}{c_{n+1}} \right|.\]

\begin{enumerate}
\def\labelenumi{\arabic{enumi}.}
\item
  Root Test
\end{enumerate}

\[R = \frac{1}{\limsup_{n\to\infty} \sqrt[n]{|c_n|}}.\]

\subsubsection{Examples}\label{examples-32}

\begin{enumerate}
\def\labelenumi{\arabic{enumi}.}
\item
  Series:
\end{enumerate}

\[\sum_{n=0}^\infty \frac{x^n}{n!}.\]

Using ratio test:

\[\lim_{n\to\infty} \frac{1/(n!)}{1/((n+1)!)} = \infty.\]

So (converges for all real ).

\begin{enumerate}
\def\labelenumi{\arabic{enumi}.}
\item
  Series:
\end{enumerate}

\[\sum_{n=0}^\infty x^n.\]

Here .

\[R = 1.\]

Converges for .

\begin{enumerate}
\def\labelenumi{\arabic{enumi}.}
\item
  Series:
\end{enumerate}

\[\sum_{n=1}^\infty \frac{x^n}{n}.\]

Ratio test:

\[\lim_{n\to\infty} \left|\frac{(x^{n+1}/(n+1))}{(x^n/n)}\right| = |x|.\]

So . Converges for , diverges for . At , test separately.

\subsubsection{Interval of Convergence}\label{interval-of-convergence-2}

The set of -values where the series converges is called the interval of
convergence.

\begin{itemize}
\item
  Always centered at .
\item
  Extends units in both directions.
\item
  Endpoints must be checked individually.
\end{itemize}

\subsubsection{Why This Matters}\label{why-this-matters-36}

\begin{itemize}
\item
  Radius of convergence tells us where power series behave like
  functions.
\item
  Essential for using Taylor series expansions in practice.
\item
  Determines the domain of validity of series solutions in physics and
  engineering.
\end{itemize}

\subsubsection{Exercises}\label{exercises-55}

\begin{enumerate}
\def\labelenumi{\arabic{enumi}.}
\item
  Find the radius of convergence of .
\item
  Compute the radius of convergence of .
\item
  Use the ratio test to find for .
\item
  Determine the interval of convergence for .
\item
  Explain why the exponential series converges for all , while the
  geometric series only converges for .
\end{enumerate}

\subsection{13.3 Taylor and Maclaurin
Series}\label{133-taylor-and-maclaurin-series}

Power series become especially powerful when they are used to represent
familiar functions. This is done through Taylor series, and the special
case centered at 0 is called a Maclaurin series.

\subsubsection{Taylor Series}\label{taylor-series}

If a function is infinitely differentiable at , its Taylor series about
is

\[f(x) = \sum_{n=0}^\infty \frac{f^{(n)}(a)}{n!}(x-a)^n.\]

Here denotes the -th derivative of at .

\subsubsection{Maclaurin Series}\label{maclaurin-series}

A Taylor series centered at :

\[f(x) = \sum_{n=0}^\infty \frac{f^{(n)}(0)}{n!} x^n.\]

\subsubsection{Examples}\label{examples-33}

\begin{enumerate}
\def\labelenumi{\arabic{enumi}.}
\item
  Exponential function
\end{enumerate}

\[e^x = 1 + x + \frac{x^2}{2!} + \frac{x^3}{3!} + \cdots\]

\begin{enumerate}
\def\labelenumi{\arabic{enumi}.}
\item
  Sine and cosine
\end{enumerate}

\[\sin x = x - \frac{x^3}{3!} + \frac{x^5}{5!} - \cdots\]

\[\cos x = 1 - \frac{x^2}{2!} + \frac{x^4}{4!} - \cdots\]

\begin{enumerate}
\def\labelenumi{\arabic{enumi}.}
\item
  Natural logarithm (for )
\end{enumerate}

\[\ln(1+x) = x - \frac{x^2}{2} + \frac{x^3}{3} - \frac{x^4}{4} + \cdots\]

\subsubsection{Taylor Polynomial
Approximation}\label{taylor-polynomial-approximation}

The finite sum of the first terms is the Taylor polynomial of degree :

\[P_n(x) = \sum_{k=0}^n \frac{f^{(k)}(a)}{k!}(x-a)^k.\]

This polynomial approximates near .

\subsubsection{Remainder (Error Term)}\label{remainder-error-term}

The difference between the function and its Taylor polynomial is the
remainder:

\[R_n(x) = f(x) - P_n(x).\]

One form (Lagrange's form) is

\[R_n(x) = \frac{f^{(n+1)}(c)}{(n+1)!}(x-a)^{n+1},\]

for some between and .

\subsubsection{Why This Matters}\label{why-this-matters-37}

\begin{itemize}
\item
  Taylor series provide polynomial approximations to complicated
  functions.
\item
  They are essential in numerical analysis, physics, and engineering.
\item
  Maclaurin series expansions give simple formulas for exponential,
  trigonometric, and logarithmic functions.
\end{itemize}

\subsubsection{Exercises}\label{exercises-56}

\begin{enumerate}
\def\labelenumi{\arabic{enumi}.}
\item
  Find the Maclaurin series for .
\item
  Write the Taylor series for centered at .
\item
  Compute the degree-3 Taylor polynomial for at .
\item
  Use the Maclaurin series for to approximate .
\item
  Explain why Taylor series often provide good local approximations but
  may diverge for large .
\end{enumerate}

\subsection{13.4 Applications of Taylor
Series}\label{134-applications-of-taylor-series}

Taylor series are not only theoretical tools - they are used to
approximate functions, solve equations, and analyze physical systems.
Their applications span mathematics, science, and engineering.

\subsubsection{Function Approximation}\label{function-approximation}

Complicated functions can be approximated by polynomials near a point.

Example: Approximate near using the degree-3 Maclaurin polynomial:

\[P_3(x) = 1 + x + \tfrac{x^2}{2} + \tfrac{x^3}{6}.\]

For small , this gives accurate estimates of .

\subsubsection{Numerical Methods}\label{numerical-methods}

Taylor series provide the basis for numerical algorithms:

\begin{itemize}
\item
  Approximating square roots, logarithms, and trigonometric values.
\item
  Error estimation through the remainder term.
\item
  Used in iterative methods like Newton's method (where local
  linearization comes from the Taylor series).
\end{itemize}

\subsubsection{Solving Differential
Equations}\label{solving-differential-equations}

Many differential equations have solutions expressed as Taylor (or
power) series.

Example: The solution to with is , which arises naturally from its
Maclaurin series.

\subsubsection{Physics and Engineering}\label{physics-and-engineering}

\begin{itemize}
\item
  Small-angle approximation:

  \[\sin x \approx x, \quad \cos x \approx 1 - \tfrac{x^2}{2}, \quad |x| \ll 1.\]

  Used in pendulum motion, optics, and wave mechanics.
\item
  Relativity and quantum mechanics: Taylor expansions simplify nonlinear
  expressions for practical use.
\item
  Approximating energy functions: In mechanics, potential energy
  functions are expanded near equilibrium points.
\end{itemize}

\subsubsection{Probability and
Statistics}\label{probability-and-statistics}

\begin{itemize}
\item
  Moment generating functions and characteristic functions use power
  series.
\item
  Approximations of probability distributions (e.g., normal
  approximation to binomial) use Taylor expansions.
\end{itemize}

\subsubsection{Why This Matters}\label{why-this-matters-38}

\begin{itemize}
\item
  Taylor series provide a bridge between exact formulas and practical
  computation.
\item
  They allow us to reduce complex problems to manageable polynomial
  approximations.
\item
  Applications make them one of the most important tools in applied
  mathematics.
\end{itemize}

\subsubsection{Exercises}\label{exercises-57}

\begin{enumerate}
\def\labelenumi{\arabic{enumi}.}
\item
  Use the Maclaurin series for to approximate up to four decimal places.
\item
  Apply the small-angle approximation to estimate .
\item
  Solve the differential equation using a power series approach.
\item
  Expand up to the 4th degree and use it to approximate .
\item
  Explain why polynomial approximations are especially useful for
  computers and calculators.
\end{enumerate}

\section{\texorpdfstring{Appendices }{Appendices }}\label{appendices}

\subsection{\texorpdfstring{Appendix A. Pre-Calculus Essentials
}{Appendix A. Pre-Calculus Essentials }}\label{appendix-a-pre-calculus-essentials}

\subsubsection{A.1 Algebra Refresher}\label{a1-algebra-refresher}

Before diving into calculus, it helps to review some algebra skills that
will appear again and again. These are the ``tools'' you'll need for
manipulating expressions, solving equations, and simplifying results.

\paragraph{Exponents and Powers}\label{exponents-and-powers}

\begin{itemize}
\item
  Basic rules:

  \[a^m \cdot a^n = a^{m+n}, \quad \frac{a^m}{a^n} = a^{m-n}, \quad (a^m)^n = a^{mn}.\]
\item
  Negative exponents:

  \[a^{-n} = \frac{1}{a^n}, \quad a \neq 0.\]
\item
  Fractional exponents:

  \[a^{1/n} = \sqrt[n]{a}, \quad a^{m/n} = \sqrt[n]{a^m}.\]
\end{itemize}

\paragraph{Factoring}\label{factoring}

Factoring simplifies expressions and helps in solving equations.

\begin{enumerate}
\def\labelenumi{\arabic{enumi}.}
\item
  Common factor:

  \[6x^2+9x = 3x(2x+3).\]
\item
  Difference of squares:

  \[a^2-b^2 = (a-b)(a+b).\]
\item
  Quadratic trinomials:

  \[x^2+5x+6 = (x+2)(x+3).\]
\end{enumerate}

\paragraph{Polynomials}\label{polynomials}

\begin{itemize}
\item
  Standard form: .
\item
  Degree: the largest power of .
\item
  Long division and synthetic division are useful for simplifying
  rational functions.
\end{itemize}

\paragraph{Rational Expressions}\label{rational-expressions}

Simplify by factoring numerator and denominator:

\[\frac{x^2-1}{x^2-2x+1} = \frac{(x-1)(x+1)}{(x-1)^2} = \frac{x+1}{x-1}, \quad x \neq 1.\]

\paragraph{Logarithms}\label{logarithms}

\begin{itemize}
\item
  Definition: means .
\item
  Common bases: natural log () and base 10 ().
\item
  Rules:

  \[\log(ab) = \log a + \log b, \quad \log\left(\frac{a}{b}\right) = \log a - \log b, \quad \log(a^n) = n\log a.\]
\end{itemize}

\paragraph{Equations}\label{equations}

\begin{itemize}
\item
  Linear: solve → .
\item
  Quadratic: has solutions

  \[x=\frac{-b\pm \sqrt{b^2-4ac}}{2a}.\]
\item
  Exponential: → .
\end{itemize}

\subsubsection{A.2 Trigonometry Basics}\label{a2-trigonometry-basics}

Trigonometry provides the language of angles and periodic phenomena.
Since calculus often deals with oscillations, motion, and waves, a solid
grasp of trigonometric functions and their properties is essential.

\paragraph{The Unit Circle}\label{the-unit-circle}

\begin{itemize}
\item
  Defined as the circle of radius 1 centered at the origin in the
  coordinate plane.
\item
  For an angle measured from the positive -axis:

  \[(\cos \theta, \sin \theta)\]

  gives the coordinates of the point on the circle.
\end{itemize}

Special values:

\begin{longtable}[]{@{}llll@{}}
\toprule\noalign{}
& & & \\
\midrule\noalign{}
\endhead
\bottomrule\noalign{}
\endlastfoot
& 0 & 1 & 0 \\
& 1/2 & & \\
& & & 1 \\
& & 1/2 & \\
& 1 & 0 & undefined \\
\end{longtable}

\paragraph{Fundamental Identities}\label{fundamental-identities}

\begin{enumerate}
\def\labelenumi{\arabic{enumi}.}
\item
  Pythagorean identity
\end{enumerate}

\[\sin^2\theta + \cos^2\theta = 1.\]

\begin{enumerate}
\def\labelenumi{\arabic{enumi}.}
\item
  Quotient identities
\end{enumerate}

\[\tan\theta = \frac{\sin\theta}{\cos\theta}, \quad \cot\theta = \frac{\cos\theta}{\sin\theta}.\]

\begin{enumerate}
\def\labelenumi{\arabic{enumi}.}
\item
  Reciprocal identities
\end{enumerate}

\[\sec\theta = \frac{1}{\cos\theta}, \quad \csc\theta = \frac{1}{\sin\theta}.\]

\paragraph{Angle Addition Formulas}\label{angle-addition-formulas}

\[\sin(\alpha+\beta) = \sin\alpha\cos\beta + \cos\alpha\sin\beta,\]

\[\cos(\alpha+\beta) = \cos\alpha\cos\beta - \sin\alpha\sin\beta.\]

Special cases:

\begin{itemize}
\item
  Double-angle:

  \[\sin(2\theta) = 2\sin\theta\cos\theta, \quad 
  \cos(2\theta) = \cos^2\theta - \sin^2\theta.\]
\end{itemize}

\paragraph{Graphs}\label{graphs}

\begin{itemize}
\item
  : wave starting at 0, amplitude 1, period .
\item
  : wave starting at 1, amplitude 1, period .
\item
  : repeats every , undefined at odd multiples of .
\end{itemize}

\subsubsection{A.3 Coordinate Geometry}\label{a3-coordinate-geometry}

Coordinate geometry links algebra and geometry by describing geometric
objects (lines, circles, curves) using equations. Calculus relies
heavily on this framework for graphing functions, finding slopes, and
analyzing curves.

\paragraph{The Cartesian Plane}\label{the-cartesian-plane}

\begin{itemize}
\item
  A point is represented by coordinates .
\item
  Distance between two points and :

  \[d = \sqrt{(x_2-x_1)^2 + (y_2-y_1)^2}.\]
\item
  Midpoint of a line segment:

  \[M = \left(\frac{x_1+x_2}{2}, \frac{y_1+y_2}{2}\right).\]
\end{itemize}

\paragraph{Lines}\label{lines}

\begin{enumerate}
\def\labelenumi{\arabic{enumi}.}
\item
  Slope formula

  \[m = \frac{y_2-y_1}{x_2-x_1}.\]
\item
  Equation of a line

  \begin{itemize}
  \item
    Point-slope form:

    \[y-y_1 = m(x-x_1).\]
  \item
    Slope-intercept form:

    \[y = mx+b.\]
  \end{itemize}
\item
  Parallel and perpendicular lines

  \begin{itemize}
  \item
    Parallel lines: same slope.
  \item
    Perpendicular lines: slopes satisfy .
  \end{itemize}
\end{enumerate}

\paragraph{Circles}\label{circles}

Equation of a circle with center and radius :

\[(x-h)^2+(y-k)^2 = r^2.\]

Special case: unit circle centered at origin:

\[x^2+y^2=1.\]

\paragraph{Conic Sections}\label{conic-sections}

\begin{enumerate}
\def\labelenumi{\arabic{enumi}.}
\item
  Parabola:

  \begin{itemize}
  \item
    Standard form (opening up/down):

    \[y = ax^2+bx+c.\]
  \end{itemize}
\item
  Ellipse (centered at origin):

  \[\frac{x^2}{a^2}+\frac{y^2}{b^2}=1.\]
\item
  Hyperbola (centered at origin):

  \[\frac{x^2}{a^2}-\frac{y^2}{b^2}=1.\]
\end{enumerate}

\subsection{\texorpdfstring{Appendix B. Key Formulas and Tables
}{Appendix B. Key Formulas and Tables }}\label{appendix-b-key-formulas-and-tables}

\subsubsection{B.1 Derivative Table}\label{b1-derivative-table}

Derivatives measure rates of change and slopes of functions. Having a
quick-reference table helps learners avoid re-deriving formulas each
time.

\paragraph{Basic Rules}\label{basic-rules-2}

\begin{enumerate}
\def\labelenumi{\arabic{enumi}.}
\item
  Constant rule
\end{enumerate}

\[\frac{d}{dx}[c] = 0\]

\begin{enumerate}
\def\labelenumi{\arabic{enumi}.}
\item
  Power rule
\end{enumerate}

\[\frac{d}{dx}[x^n] = nx^{n-1}, \quad (n \in \mathbb{R})\]

\begin{enumerate}
\def\labelenumi{\arabic{enumi}.}
\item
  Constant multiple rule
\end{enumerate}

\[\frac{d}{dx}[c f(x)] = c f'(x)\]

\begin{enumerate}
\def\labelenumi{\arabic{enumi}.}
\item
  Sum and difference rule
\end{enumerate}

\[\frac{d}{dx}[f(x)\pm g(x)] = f'(x)\pm g'(x)\]

\paragraph{Trigonometric Functions}\label{trigonometric-functions}

\[\frac{d}{dx}[\sin x] = \cos x\]

\[\frac{d}{dx}[\cos x] = -\sin x\]

\[\frac{d}{dx}[\tan x] = \sec^2 x, \quad x \neq \tfrac{\pi}{2}+k\pi\]

\[\frac{d}{dx}[\cot x] = -\csc^2 x\]

\[\frac{d}{dx}[\sec x] = \sec x \tan x\]

\[\frac{d}{dx}[\csc x] = -\csc x \cot x\]

\paragraph{Exponential and Logarithmic
Functions}\label{exponential-and-logarithmic-functions}

\[\frac{d}{dx}[e^x] = e^x\]

\[\frac{d}{dx}[a^x] = a^x \ln a, \quad a>0, a\neq 1\]

\[\frac{d}{dx}[\ln x] = \frac{1}{x}, \quad x>0\]

\[\frac{d}{dx}[\log_a x] = \frac{1}{x\ln a}, \quad a>0, a\neq 1\]

\paragraph{Inverse Trigonometric
Functions}\label{inverse-trigonometric-functions}

\[\frac{d}{dx}[\arcsin x] = \frac{1}{\sqrt{1-x^2}}, \quad |x|<1\]

\[\frac{d}{dx}[\arccos x] = -\frac{1}{\sqrt{1-x^2}}, \quad |x|<1\]

\[\frac{d}{dx}[\arctan x] = \frac{1}{1+x^2}, \quad x \in \mathbb{R}\]

\[\frac{d}{dx}[\arccot x] = -\frac{1}{1+x^2}, \quad x \in \mathbb{R}\]

\[\frac{d}{dx}[\arcsec x] = \frac{1}{|x|\sqrt{x^2-1}}, \quad |x|>1\]

\[\frac{d}{dx}[\arccsc x] = -\frac{1}{|x|\sqrt{x^2-1}}, \quad |x|>1\]

\paragraph{Product, Quotient, and Chain
Rules}\label{product-quotient-and-chain-rules}

\begin{enumerate}
\def\labelenumi{\arabic{enumi}.}
\item
  Product Rule
\end{enumerate}

\[\frac{d}{dx}[f(x)g(x)] = f'(x)g(x)+f(x)g'(x)\]

\begin{enumerate}
\def\labelenumi{\arabic{enumi}.}
\item
  Quotient Rule
\end{enumerate}

\[\frac{d}{dx}\left[\frac{f(x)}{g(x)}\right] = \frac{f'(x)g(x)-f(x)g'(x)}{[g(x)]^2}, \quad g(x)\neq 0\]

\begin{enumerate}
\def\labelenumi{\arabic{enumi}.}
\item
  Chain Rule
\end{enumerate}

\[\frac{d}{dx}[f(g(x))] = f'(g(x))\cdot g'(x)\]

\subsubsection{B.3 Common Series
Expansions}\label{b3-common-series-expansions}

Power series let us express functions as infinite polynomials. These
expansions are essential for approximations, solving differential
equations, and building intuition about functions in calculus.

\paragraph{Geometric Series}\label{geometric-series}

\[\frac{1}{1-x} = \sum_{n=0}^\infty x^n, \quad |x| < 1\]

\paragraph{Exponential Function}\label{exponential-function}

\[e^x = \sum_{n=0}^\infty \frac{x^n}{n!} 
= 1 + x + \frac{x^2}{2!} + \frac{x^3}{3!} + \cdots\]

Valid for all .

\paragraph{Trigonometric Functions}\label{trigonometric-functions-2}

\[\sin x = \sum_{n=0}^\infty (-1)^n \frac{x^{2n+1}}{(2n+1)!} 
= x - \frac{x^3}{3!} + \frac{x^5}{5!} - \cdots\]

\[\cos x = \sum_{n=0}^\infty (-1)^n \frac{x^{2n}}{(2n)!} 
= 1 - \frac{x^2}{2!} + \frac{x^4}{4!} - \cdots\]

\[\tan^{-1} x = \sum_{n=0}^\infty (-1)^n \frac{x^{2n+1}}{2n+1}, \quad |x|\leq 1\]

\paragraph{Logarithm}\label{logarithm}

\[\ln(1+x) = \sum_{n=1}^\infty (-1)^{n+1} \frac{x^n}{n}, \quad -1 < x \leq 1\]

\paragraph{Binomial Expansion
(Generalized)}\label{binomial-expansion-generalized}

\[(1+x)^r = \sum_{n=0}^\infty \binom{r}{n} x^n, \quad |x|<1\]

where

\[\binom{r}{n} = \frac{r(r-1)(r-2)\cdots(r-n+1)}{n!}.\]

\subsection{Appendix C. Proof Sketches}\label{appendix-c-proof-sketches}

\subsubsection{C.1 Limit Laws and the --
Definition}\label{c1-limit-laws-and-the--ux3b5----ux3b4--definition}

Calculus rests on the precise meaning of a limit. While intuition
(``values get closer and closer'') is helpful, a formal definition
ensures rigor and avoids paradoxes.

\paragraph{Intuitive Idea}\label{intuitive-idea}

We write

\[\lim_{x \to a} f(x) = L\]

to mean that as gets arbitrarily close to , the values of get
arbitrarily close to .

\paragraph{Formal (--)
Definition}\label{formal--ux3b5----ux3b4--definition}

We say that

\[\lim_{x \to a} f(x) = L\]

if for every , there exists a such that whenever

\[0 < |x-a| < \delta,\]

we have

\[|f(x) - L| < \varepsilon.\]

\begin{itemize}
\item
  : how close we want to be to .
\item
  : how close must be to to achieve that.
\end{itemize}

\paragraph{Example}\label{example-3}

Show that

\[\lim_{x \to 2} (3x+1) = 7.\]

\begin{itemize}
\item
  Let .
\item
  We want .
\item
  Simplify: .
\item
  This holds if we choose .
\end{itemize}

Thus, by the definition, the limit is 7.

\paragraph{Limit Laws}\label{limit-laws}

If and , then:

\begin{enumerate}
\def\labelenumi{\arabic{enumi}.}
\item
  Sum/Difference
\end{enumerate}

\[\lim_{x \to a} [f(x) \pm g(x)] = L \pm M\]

\begin{enumerate}
\def\labelenumi{\arabic{enumi}.}
\item
  Constant Multiple
\end{enumerate}

\[\lim_{x \to a} [c f(x)] = cL\]

\begin{enumerate}
\def\labelenumi{\arabic{enumi}.}
\item
  Product
\end{enumerate}

\[\lim_{x \to a} [f(x)g(x)] = LM\]

\begin{enumerate}
\def\labelenumi{\arabic{enumi}.}
\item
  Quotient (if )
\end{enumerate}

\[\lim_{x \to a} \frac{f(x)}{g(x)} = \frac{L}{M}\]

\begin{enumerate}
\def\labelenumi{\arabic{enumi}.}
\item
  Powers and Roots
\end{enumerate}

\[\lim_{x \to a} [f(x)]^n = L^n, \quad \lim_{x \to a} \sqrt[n]{f(x)} = \sqrt[n]{L} \ (\text{if defined}).\]

\subsubsection{C.2 Proof Sketch: The Fundamental Theorem of
Calculus}\label{c2-proof-sketch-the-fundamental-theorem-of-calculus}

The Fundamental Theorem of Calculus (FTC) links the two central
operations of calculus: differentiation and integration. It shows that
they are, in fact, inverse processes.

\paragraph{Statement of the Theorem}\label{statement-of-the-theorem}

Part I (Differentiation of an Integral):\\
If is continuous on and we define

\[F(x) = \int_a^x f(t)\,dt,\]

then is differentiable on and

\[F'(x) = f(x).\]

Part II (Evaluation of a Definite Integral):\\
If is any antiderivative of on , then

\[\int_a^b f(x)\,dx = F(b)-F(a).\]

\paragraph{Proof Sketch of Part I}\label{proof-sketch-of-part-i}

\begin{enumerate}
\def\labelenumi{\arabic{enumi}.}
\item
  Start with the definition of the derivative:

  \[F'(x) = \lim_{h\to 0} \frac{F(x+h)-F(x)}{h}.\]
\item
  Substituting :

  \[F(x+h)-F(x) = \int_a^{x+h} f(t)\,dt - \int_a^x f(t)\,dt.\]
\item
  By the additivity of integrals:

  \[F(x+h)-F(x) = \int_x^{x+h} f(t)\,dt.\]
\item
  Therefore:

  \[\frac{F(x+h)-F(x)}{h} = \frac{1}{h}\int_x^{x+h} f(t)\,dt.\]
\item
  By the Mean Value Theorem for integrals, there exists such that

  \[\frac{1}{h}\int_x^{x+h} f(t)\,dt = f(c).\]
\item
  As , , and since is continuous:

  \[\lim_{h\to 0} f(c) = f(x).\]
\end{enumerate}

Thus, .

\paragraph{Proof Sketch of Part II}\label{proof-sketch-of-part-ii}

\begin{enumerate}
\def\labelenumi{\arabic{enumi}.}
\item
  Let be an antiderivative of , so .
\item
  By Part I, the function

  \[G(x) = \int_a^x f(t)\,dt\]

  is also an antiderivative of .
\item
  Since and differ only by a constant,

  \[F(x) = G(x) + C.\]
\item
  Evaluating at the endpoints:

  \[\int_a^b f(x)\,dx = G(b)-G(a) = (F(b)+C)-(F(a)+C) = F(b)-F(a).\]
\end{enumerate}

\subsubsection{C.3 Proof Sketch: Convergence of the Geometric
Series}\label{c3-proof-sketch-convergence-of-the-geometric-series}

The geometric series is one of the simplest and most important infinite
series. It serves as a model for understanding convergence and is the
foundation for many later results in calculus.

\paragraph{The Series}\label{the-series}

\[\sum_{n=0}^\infty ar^n = a + ar + ar^2 + ar^3 + \cdots\]

where is the first term and is the common ratio.

\paragraph{Partial Sum Formula}\label{partial-sum-formula}

The -th partial sum is

\[S_n = a + ar + ar^2 + \cdots + ar^n.\]

Multiply both sides by :

\[rS_n = ar + ar^2 + \cdots + ar^{n+1}.\]

Subtract the two equations:

\[S_n - rS_n = a - ar^{n+1}.\]

\[S_n(1-r) = a(1-r^{n+1}).\]

So

\[S_n = \frac{a(1-r^{n+1})}{1-r}, \quad r \neq 1.\]

\paragraph{Convergence}\label{convergence}

Take the limit as :

\begin{itemize}
\item
  If , then .

  \[\lim_{n\to\infty} S_n = \frac{a}{1-r}.\]
\item
  If , then does not go to 0. The series diverges.
\end{itemize}

\paragraph{Result}\label{result}

\[\sum_{n=0}^\infty ar^n = 
\begin{cases}
\dfrac{a}{1-r}, & |r|<1, \\[6pt]
\text{diverges}, & |r|\geq 1.
\end{cases}\]

\subsection{Appendix D. Applications and
Connections}\label{appendix-d-applications-and-connections}

\subsubsection{D.1 Physics Connections: Velocity, Acceleration, and
Work}\label{d1-physics-connections-velocity-acceleration-and-work}

Calculus was originally developed to solve problems in physics -
especially motion and change. Here are some of the most important
connections.

\paragraph{Position, Velocity, and
Acceleration}\label{position-velocity-and-acceleration-2}

\begin{itemize}
\item
  Position function: gives the location of an object at time .
\item
  Velocity: the derivative of position.

  \[v(t) = s'(t) = \frac{ds}{dt}\]
\item
  Acceleration: the derivative of velocity (or second derivative of
  position).

  \[a(t) = v'(t) = s''(t) = \frac{d^2s}{dt^2}\]
\end{itemize}

Example:\\
If meters, then:

\[v(t) = 8t, \quad a(t) = 8.\]

So the object moves faster linearly with time, under constant
acceleration.

\paragraph{Work and Force}\label{work-and-force}

In physics, work is the product of force and distance. If force varies
with position, calculus gives:

\[W = \int_a^b F(x)\, dx\]

where is the force at position , and the object moves from to .

Example:\\
A spring with Hooke's law force requires work

\[W = \int_0^d kx\, dx = \frac{1}{2}kd^2\]

to stretch the spring a distance .

\paragraph{Energy and Areas Under
Curves}\label{energy-and-areas-under-curves}

\begin{itemize}
\item
  Kinetic energy: .
\item
  Potential energy often involves integrals (e.g., gravitational
  potential energy from force of gravity).
\item
  In general, integrating a force function gives energy stored or work
  done.
\end{itemize}

\paragraph{Quick Practice}\label{quick-practice}

\begin{enumerate}
\def\labelenumi{\arabic{enumi}.}
\item
  If , find and .
\item
  Compute the work done by a constant force of 10 N moving an object 5
  m.
\item
  A spring has constant . How much work is needed to stretch it 0.1 m?
\item
  Show that acceleration is the second derivative of position.
\item
  Explain how the integral relates to displacement.
\end{enumerate}

\subsubsection{D.2 Probability and Statistics
Connections}\label{d2-probability-and-statistics-connections}

Calculus is deeply connected with probability and statistics, especially
when dealing with continuous random variables. Integrals become
essential for defining probabilities, averages, and expectations.

\paragraph{Probability Density Functions
(PDFs)}\label{probability-density-functions-pdfs}

For a continuous random variable , probabilities are described by a
probability density function :

\begin{enumerate}
\def\labelenumi{\arabic{enumi}.}
\item
  for all .
\item
  Total probability equals 1:

  \[\int_{-\infty}^{\infty} f(x)\, dx = 1.\]
\end{enumerate}

The probability that lies in an interval is

\[P(a \leq X \leq b) = \int_a^b f(x)\, dx.\]

\paragraph{Expected Value (Mean)}\label{expected-value-mean-2}

The expected value (average outcome) is

\[E[X] = \int_{-\infty}^{\infty} x f(x)\, dx.\]

This is the calculus version of a weighted average.

\paragraph{Variance}\label{variance}

Variance measures spread:

\[\text{Var}(X) = E[(X-\mu)^2] = \int_{-\infty}^{\infty} (x-\mu)^2 f(x)\, dx,\]

where .

\paragraph{Common Distributions}\label{common-distributions}

\begin{enumerate}
\def\labelenumi{\arabic{enumi}.}
\item
  Uniform distribution on :

  \[f(x) = \frac{1}{b-a}, \quad a \leq x \leq b.\]

  Mean: .
\item
  Exponential distribution with parameter :

  \[f(x) = \lambda e^{-\lambda x}, \quad x \geq 0.\]

  Mean: .
\item
  Normal (Gaussian) distribution:

  \[f(x) = \frac{1}{\sqrt{2\pi\sigma^2}} e^{-(x-\mu)^2/(2\sigma^2)}.\]

  Integrals of this distribution connect to the error function.
\end{enumerate}

\paragraph{Why This Matters}\label{why-this-matters-39}

\begin{itemize}
\item
  Integrals turn probabilities into areas under curves.
\item
  Expectation and variance link calculus to averages and variability.
\item
  Most real-world data models (finance, physics, biology, AI) use these
  continuous probability distributions.
\end{itemize}

\paragraph{Quick Practice}\label{quick-practice-2}

\begin{enumerate}
\def\labelenumi{\arabic{enumi}.}
\item
  For on , compute .
\item
  For exponential distribution with , compute .
\item
  Show that the total area under the standard normal curve equals 1.
\item
  Find the mean of a uniform distribution on .
\item
  Explain why probabilities are computed with integrals, not sums, for
  continuous variables.
\end{enumerate}

\subsubsection{D.3 Computer Science Connections: Taylor Approximations
in
Algorithms}\label{d3-computer-science-connections-taylor-approximations-in-algorithms}

Calculus is not only for physics - it also underpins many tools and
techniques in computer science. One of the clearest bridges is through
Taylor series, which provide efficient ways to approximate functions in
numerical computing and algorithms.

\paragraph{Function Approximation for
Computing}\label{function-approximation-for-computing}

Computers cannot directly store or calculate most functions exactly
(like , , or ). Instead, they use polynomial approximations derived from
Taylor expansions.

Example:\\
To approximate , truncate the Maclaurin series:

\[e^x \approx 1 + x + \frac{x^2}{2!} + \frac{x^3}{3!}.\]

For small , this polynomial gives accurate results with only a few
terms.

\paragraph{Efficiency in Algorithms}\label{efficiency-in-algorithms}

\begin{itemize}
\item
  Trigonometric functions: Algorithms for calculators and CPUs often use
  series expansions (or variations like Chebyshev polynomials).
\item
  Exponential/logarithm: Taylor expansions are the foundation of fast
  approximations in numerical libraries.
\item
  Root finding: Newton's method is based on linear approximation, a
  direct application of the Taylor series (first derivative).
\end{itemize}

\paragraph{Numerical Analysis}\label{numerical-analysis}

Taylor expansions are central in error analysis:

\begin{itemize}
\item
  Approximating the error term using the remainder formula:

  \[R_n(x) = \frac{f^{(n+1)}(c)}{(n+1)!}(x-a)^{n+1}.\]
\item
  This tells us how many terms are needed for a given accuracy.
\end{itemize}

\paragraph{Machine Learning
Connections}\label{machine-learning-connections}

\begin{itemize}
\item
  Gradient-based optimization (like gradient descent) uses derivatives
  to update parameters efficiently.
\item
  Activation functions (like or ) are often approximated by polynomials
  or piecewise functions for speed.
\item
  Series approximations can speed up training and inference in
  constrained environments.
\end{itemize}

\paragraph{Why This Matters}\label{why-this-matters-40}

\begin{itemize}
\item
  Taylor approximations bridge continuous mathematics with discrete
  computing.
\item
  They show how calculus concepts are used in algorithms, numerical
  methods, and machine learning.
\item
  Understanding the approximations helps avoid pitfalls when relying on
  computers for calculations.
\end{itemize}

\paragraph{Quick Practice}\label{quick-practice-3}

\begin{enumerate}
\def\labelenumi{\arabic{enumi}.}
\item
  Approximate using the first three terms of its Maclaurin series.
\item
  Use the remainder term to estimate the error in approximating with a
  degree-3 polynomial.
\item
  Explain how Newton's method uses Taylor's theorem.
\item
  Why might computers prefer polynomial approximations to exact formulas
  for functions?
\item
  In machine learning, why is the derivative (gradient) so critical for
  optimization?
\end{enumerate}

\end{document}
